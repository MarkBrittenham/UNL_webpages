%\baselineskip=18pt plus 2pt

\magnification=1200

\parindent=50pt

\def\ni{\noindent}
\def\ctln{\centerline}
\def\msk{\medskip}
\def\ssk{\smallskip}
\def\bsk{\bigskip}

\def\iit{\itemitem}
\def\htp{\hskip10pt}
\def\vtp{\vskip.02in}
\def\hsk{\hskip.2in}
%\input shorthand

\nopagenumbers

\ctln{\bf Math 314 (814) Matrix Theory}

%\vtp

\ctln{\bf Section 002}

\smallskip

\ni{\bf Lecture:} TuTh 11:00-12:15 \htp Oldfather Hall (OldH) 304

\msk

\ni{\bf Instructor:} Mark Brittenham

%\smallskip

\ni{\bf Office:} Oldfather Hall (OldH) 819
%\smallskip

\ni{\bf Telephone:} (47)2-7222

%\ssk

\ni{\bf E-mail:} mbritten@math.unl.edu

\ni{\bf WWW:} http://www.math.unl.edu/\~{ }mbritten/

\ni{\bf WWW pages for this class:} http://www.math.unl.edu/\~{ }mbritten/classwk/314s2k/

\ssk

\ni(There you will find copies of nearly every handout from class, lists of homework 
problems assigned, dates for exams, etc.)

\smallskip

\ni{\bf Office Hours:} (tentatively) Mo 1:30-2:30, Tu 2:00 - 3:00, We 10:00-11:00, 
and Th 9:30-10:30, and whenever you can find me in my office and I'm not 
horrendously busy. You are also quite welcome to make an appointment
for any other time; this is easiest to arrange just before or 
after class.

\ssk

\ni{\bf Text:} {\it Applied Linear Algebra and Matrix analysis}, 
by Thomas Shores (revised 2nd edition, McGraw-Hill).

\ssk

\ni As it happens, our textbook is available online, at 

http://www.math.unl.edu/\~{ }tshores/linalgtext.html 

\ni The pages are delivered as individual gif files, though, which
can take a looong time to load, so I do not recommend these
pages as a substitute for purchasing the book!

\msk

\ni This course, as the name is meant to imply, is intended to illustrate the 
theory, techniques, and applications of linear algebra (i.e., solutions to linear 
equations) through
the use of matrices (whatever they are).
Our basic goal will be to work through the first five chapters of the book:

\ssk

\ni\hsk Ch. 1, Linear Systems of Equations

\ni\hsk Ch. 2, Matrix Algebra

\ni\hsk Ch. 3, Vector Spaces

\ni\hsk Ch. 4, The Eigenvalue Problem

\ni\hsk Ch, 5, Geometrical Aspects of Vectors

\msk

\ni{\bf Homework} will be assigned from each section, as we finish it. 
It is an essential ingredient to the course - as with almost all of 
mathematics, we learn best by doing (again and again and ...). Cooperation 
with other students on these assignments is acceptable, and even 
encouraged. However, you must write up solutions on your own - after 
all, you get to bring only one brain to exams (and it can't be someone 
else's). For the same reason, I also recommend that you try working 
each problem on your own, first. Some fraction of the problems will be 
designated to be turned in, and graded; these will be due the Thursday 
after the day in which they were assigned. The homework grades will 
count 20\% toward your grade.
Late homework may be marked as turned in but not graded.

\ssk

\ni None of the homework problems will require the use of a calculator or computer
algebra system (CAS) for their solution. However, in order to give you some 
experience with `large scale' scientific computation, a few times during the semester
(current best guess: twice) you will be assigned a larger problem whose solution
will involve using a CAS like MapleV. For this purpose, you will be given an account
in the Math Lab to carry out this work. Those of you unfamiliar with the computing 
resources in the lab (e.g., MapleV!) are strongly encouraged to sign up for one 
of the orientations
which will be given in the first two weeks of the semester. You can sign up for a session
on the door of the Math Lab. These larger assignments will count 10\% toward your grade.

\ssk

\ni{\bf Midterm exams} will be given three times during the 
semester, approximately every four weeks - the specific dates will 
be announced in class well in advance (likely candidates: early February, 
end of March, mid-April). Each exam will count 15\% toward your grade. 
You can take a 
make-up exam only if there are compelling reasons (a doctor SAYS 
you were sick, jury duty, etc.) for you to miss an exam. Make-up 
exams tend to be harder than the originals (because make-up exams 
are harder to write!). 

\ssk

\ni Finally, there will be a regularly scheduled {\bf final exam} on 
Thursday, May 4, from 10:00am to 12:00 noon.
It will cover the entire course, with a slight emphasis 
on material covered after the last midterm exam. It will count the 
remaining 25\% toward your grade.


\msk

\ni {\bf Your course grade} will be calculated numerically using the above scales,
and will be converted to a letter grade based partly on the overall average of the
class. However, a score of 90\% or better will guarantee some kind of {\bf A}, 80\%
or better at least some sort of {\bf B}, 70\% or better at least a flavor of 
{\bf C}, and 60\% or 
better at least a {\bf D}.

\bsk 

\ni\hskip.2in In mathematics, new concepts continually rely upon the mastery
of old ones; it is therefore essential that you thoroughly understand each 
new topic before moving on. Our classes are an important opportunity for you to ask
questions; to make \underbar{sure} that you are understanding concepts correctly.
Speak up! It's \underbar{your} education at stake. Make every effort to resist
the temptation to put off work, and to fall behind. Every topic has to be gotten 
through, not around. And it's alot easier to read 50 pages in a week than it is
in a day. Try to do some mathematics every single day. (I do.)
{\bf Class attendance} is probably your best way to insure that you will keep 
up with the material, and make sure that you understand all of the
concepts. I will not be taking attendance; I expect that you will simply 
see the wisdom of attending class, for yourselves.

\msk

\ni{\bf Departmental Grading Appeals Policy:} Students who believe their
academic evaluation has been perjudiced or capricious have recourse for appeals 
to (in order) the instructor, the departmental chair, the departmental appeals 
committee, and 
the college appeals committee.

\msk

\ctln{\bf Some important academic dates}

\ssk

{\bf Jan. 10} First day of classes.

{\bf Jan. 17} Martin Luther King Day - no classes.

{\bf Jan. 21} Last day to withdraw from a course without receiving a {\bf `W'}.

{\bf Mar. 3} Last day to change to or from P/NP.

{\bf Mar. 12-19} Spring break - no classes.

{\bf Apr. 7} Last day to withdraw from a course.

{\bf Apr. 29} Last day of classes.

\vfill

\end

\vfill\eject
