

\magnification=1200
\overfullrule=0pt
\parindent=0pt
\voffset=-.5in
\vsize=10in
\nopagenumbers

\input amstex

\input epsf
\loadmsbm

\def\ctln{\centerline}
\def\u{\underbar}
\def\ssk{\smallskip}
\def\msk{\medskip}
\def\bsk{\bigskip}
\def\hsk{\hskip.1in}
\def\hhsk{\hskip.2in}

\def\dsl{\displaystyle}
\def\hskp{\hskip1.5in}
\def\ra{\rightarrow}
\def\lra{$\Leftrightarrow$}
\def\pu{\pi_1}
\def\mpu{$\pi_1$}
\def\bra{$\Rightarrow$}
\def\bbr{{\Bbb R}}
\def\bbz{{\Bbb Z}}
\def\del{\partial}
\def\indt{\item{}}



\ctln{\bf Math 990 Topics in Topology}

\ssk

\ctln{Problem Set \#\ 1}

\ssk

\ctln{Two problems (of your choice) are due Thursday, February 7}

\bsk

\item{\bf 1.} Show that two elements $u,v$ of the free group
$G=F(x_1,\ldots,x_n)$ are conjugate ($v=wuw^{-1}$ for some
$w\in G$ \lra\ $v$, as a (reduced) word
in the $x_i$, is a cyclic permutation of $u$; 
$u=x_{i_1}^{\epsilon_1}\cdots x_{i_k}^{\epsilon_k}$ and
$v=x_{i_r}^{\epsilon_r}\cdots x_{i_k}^{\epsilon_k}
x_{i_1}^{\epsilon_1}\cdots x_{i_{r-1}}^{\epsilon_{r-1}}$ .

\ssk

(N.B.: We can therefore test elements in the free group to decide if 
they are conjugate, i.e., we can solve the ``conjugacy problem'' in free groups.)

\msk

\item{\bf 2.} Show that using our more formal definition of a
presentation that for a finite group $G$, the group with
finite presentation 
$<\{g : g\in G\}\ |\ \{ghk^{-1} : g,h,k\in G \ \text{and}\ gh=_Gk\}>$
is isomorphic to $G$. (Build homomorphisms each way, show they are
inverses.)

\msk

\item{\bf 3.} Show that the group $G=<a,b\ |\ a^3,b^2,(ab)^2>$
is a non-trivial group. (Hint: find a non-trivial homomorphism
to some small group.) Use this (or argue directly,) to show that the
group $H=<a,b\ |\ a^3,b^4,ab^2a^2b^2>$ is also non-trivial.

\msk

\item{\bf 4.} Show that the group with presentation $<x,y\ |\ x^2=y^3,xyx=yxy>$
is the trivial group, i.e., show that the relators imply that $x=1$ and $y=1$.
(One way: find may different ways to express the word $x^2yxy$ in $G$ ...)

\msk

\item{\bf 5.} Show that a group $G$ is generated by finitely
many $x_i$, all conjugate to one another \lra\ $G$ is finitely 
generated, and there
is an $x$ with $\{gxg^{-1} : g\in G\}$ a (not necessarily
finite) generating set for $G$.


\msk

\item{\bf 6.} Show that in Johnson's construction of knots 
with a given 
quotient, in constructing the ``tongue'' we may allow it 
to twist around itself (see figure below) without 
interfering with the conclusion of the construction. 
(This allows us to build a wider class of knots with a given quotient.)

\msk

\epsfxsize=2in
\ctln{{\epsfbox{e1f1.eps}}}

\msk

\item{\bf 7.} Show that the connected sum operation (figure below)
results in a knot $K_1\#K_2$ whose group 
$G=\pi(\pi_1(K_1\#K_2)=\pi_1(\bbr^3\setminus(K_1\#K_2))$
has quotients $\pi(K_1)$ and $\pi(K_2)$. (Hint: think in
terms of ``coloring'' the connected sum so that its group maps 
onto each group (separately).)

\msk

\epsfxsize=2in
\ctln{{\epsfbox{e1f3.eps}}}

\msk

\item{\bf 8.} Find several knots $K$ whose knot groups
$G=\pi_1(\bbr^3\setminus K)$
have quotient the group of the trefoil knot (figure below). 
(Don't use the connected sum operation...)

\msk

\epsfxsize=.8in
\ctln{{\epsfbox{e1f2.eps}}}


\vfill
\end