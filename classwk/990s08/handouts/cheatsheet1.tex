\magnification=1500


\input amstex

\input epsf
\loadmsbm

\def\ctln{\centerline}
\def\u{\underbar}
\def\ssk{\smallskip}
\def\msk{\medskip}
\def\bsk{\bigskip}
\def\hsk{\hskip.1in}
\def\hhsk{\hskip.2in}

\def\dsl{\displaystyle}
\def\hskp{\hskip1.5in}
\def\ra{\rightarrow}
\def\lra{$\Leftrightarrow$}
\def\pu{\pi_1}
\def\mpu{$\pi_1$}
\def\bra{$\Rightarrow$}
\def\bbr{{\Bbb R}}
\def\bbc{{\Bbb C}}
\def\bbz{{\Bbb Z}}
\def\del{\partial}
\def\indt{\item{}}

\nopagenumbers
\parindent=0pt

\centerline{Representations into $SL_2(\bbc)$ ; the players so far}

\bsk

$\Gamma$ = finitely-generated group, generators $g_1,\ldots,g_n$

\msk

$R(\Gamma)=\{\rho:\Gamma\ra SL_2(\bbc)$ homomorphisms $\}$ linear representations

\msk

$R(\Gamma)\hookrightarrow\bbc^{4n}$ (coords of generators - \underbar{not} a homom!) 

\ssk

is defined by (finitely many) polynomimal equations; 

it is an {\bf affine algebraic set}.

\msk

$V\subseteq\bbc^N$ alg set, the coordinate ring $\bbc[V] = $

$\{f:V\ra\bbc\ :\ f=g|_V, g:\bbc^N\ra \bbc$ polynomial fcn$\}$

\ssk

If $V$ is \underbar{irreducible} (no proper alg subset), then $\bbc[V]$ = integral domain.

$\bbc(V)$ = field of fractions = \underbar{function} \underbar{field} of $V$, think of as rational functions
defined on open dense subsets $U\subseteq V$.

\msk

$R_0\subseteq R(\Gamma)$, $\gamma\in\Gamma$, define $E_\gamma:R_0\ra SL_2(\bbc)$ by $E_\gamma(\rho)=\rho(\gamma)$.

\ssk

Entries of $\rho(\gamma)$ are polynomials in the entries of the $\rho(g_i)$, so are in $\bbc[R_0]$, so
$E_\gamma\in SL_2(\bbc[R_0])$.

\msk

Characters: 

$\chi_\rho:\Gamma\ra\bbc$ defined by $\chi_\rho(\gamma)=\ $tr$(\rho(\gamma))=\ $tr$(E_\gamma(\rho)\equiv I_\gamma(\rho)$

\ssk

Set of all characters = $X(\Gamma)$.

\ssk

$I_\gamma:R(\Gamma)\ra \bbc$, so $I_\gamma\in \bbc[R(\Gamma)]$

\msk

$T(\Gamma)$ = trace ring = subring of $\bbc[R(\Gamma)]$ generated by the $I\gamma$.

\ssk

$T(\Gamma)$ is generated by $I_\gamma$, 

$\gamma\in S = \{ g_{i_1}\cdots g_{i_k}\ :\ i_1<\cdots i_k\}=\{w_1,\ldots,w_{2^n-1}\}$, so

\ssk

$T(\Gamma)\hookrightarrow \bbc^{|S|}=\bbc^{2^n-1}$

\msk

$t:R(\Gamma)\ra \bbc^{2^n-1}$, $t(\rho)=(I_{w_1}(\rho),\ldots,I_{w_{2^n-1}}(\rho))$

\ssk

$t(\rho)=t(\rho^\prime)$ $\Leftrightarrow$ $I_\gamma(\rho)=I_\gamma(\rho^\prime)$ $\forall\ \gamma$ $\Leftrightarrow$
$\chi_\rho(\gamma)=\chi_{\rho^\prime}(\gamma)$ $\forall\ \gamma$ $\Leftrightarrow$ $\chi_\rho=\chi_{\rho^\prime}$, 

\ssk

so $X(\Gamma) \leftrightarrow t(R(\Gamma))\subseteq \bbc^{2^n-1}$. 

\ssk

$t(R(\Gamma))$ is an algebraic set; so $X(\Gamma)$
may be thought of as one, too.

\msk

The surjection $t:R(\Gamma)\rightarrow X(\Gamma)$ induces an injection 

$J:\bbc[X(\Gamma)]\hookrightarrow \bbc[R(\Gamma)]$,
with image generated by the $I_{w_j}$.




\vfill
\end


