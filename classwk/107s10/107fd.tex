%\baselineskip=18pt plus 2pt
\magnification=1200
\overfullrule=0pt
%\hsize=6.9 true in
%\hoffset=-.3 true in
\parindent=50pt
\def\ni{\noindent}
\def\ctln{\centerline}
\def\msk{\medskip}
\def\ssk{\smallskip}
\def\bsk{\bigskip}
\def\iit{\itemitem}
\def\htp{\hskip10pt}
\def\vtp{\vskip.02in}
\def\hsk{\hskip.2in}
%\input shorthand
\nopagenumbers
\ctln{\bf Math 107 Analytic Geometry and Calculus II}
\ctln{\bf Section 150}
\smallskip
\ni{\bf Lecture:} TR 9:30-10:45 \htp Avery Hall (AVH) 115
\ni{\bf Recitations:} WF 9:30-10:20
\ni\htp\htp [151] Avery Hall (AVH) 115, Doug Heltibridle
\ni\htp\htp [152] Henzlik Hall [HENZ] 36, Nora Youngs
\ni\htp\htp [153] Avery Hall (AVH) 118, Brittney Hinds
\ni\htp\htp [154] Military and Naval [M\&N] B6, Andrew Ray
\msk
\ni{\bf Instructor:} Mark Brittenham
%\smallskip
\ni{\bf Office:} Avery Hall (AVH) 317
%\smallskip
\ni{\bf Telephone:} (47)2-7222
%\ssk
\ni{\bf E-mail:} mbrittenham2@math.unl.edu
\ni{\bf WWW:} http://www.math.unl.edu/$\sim$mbrittenham2/
\ni{\bf WWW pages for this class:} http://www.math.unl.edu/$\sim$mbrittenham2/classwk/107s10/
\ssk
\ni(There you will find copies of every handout from class, dates for exams, review materials, etc.)
\smallskip
\ni{\bf Office Hours:} (tentatively) Mo 12:00 - 1:00, Tu 1:00-1:50, 
and We 10:30 - 11:30, and whenever you can find me in my office and I'm not 
horrendously busy. You are also quite welcome to make an appointment
for any other time; this is easiest to arrange by email or just before or 
after class. Any alteration of these hours will be announced in class.
\ssk
\ni{\bf Text:} {\it University Calculus}, 
by Hass, Weir, and Thomas (Addison-Wesley, 2007).
\msk
\ni{\bf ACE outcome 3:} This course satisfies ACE Outcome 3. You will apply 
mathematical reasoning and computations to draw conclusions, solve 
problems, and learn to check to see if your answer is reasonable. Your 
instructor will provide examples, you will discuss them in class, and 
you will practice with numerous homework problems. The exams will test 
how well you�ve mastered the material.
\ssk
\ni This course, as the name is intended to imply, is the second of several where you
learn the basics of what we call calculus. Our goal for the semester is to cover approximately
the second third of the text. In particular, we will cover sections from
the following chapters of the book (although not necessarily in this order):
\ssk
Ch. 5, Integration: sections 5.4 thru 5.5
Ch. 6, Applications of Definite Integrals: sections 6.1 thru 6.3, 6.5 thru 6.6
Ch. 7, Techniques of Integration: sections 7.1 thru 7.7
Ch. 8, Infinite Sequences and Series: sections 8.1 thru 8.9
Ch. 9, Polar Coordinates and Conics: sections 9.1 thru 9.3
Ch. 10, Vectors and the Geometry of Space: sections 10.1 thru 10.3, 10.5
Ch. 11, Vector-Valued Functions and Motion in Space: sections 11.1 thru 11.3
\ssk
\ni{\bf Homework} will be assigned from each section, as we finish it. 
It is an essential ingredient to the course - as with almost all of 
mathematics, we learn best by doing (again and again and ...). Cooperation 
with other students on these assignments is acceptable, and even 
encouraged. However, you should make sure you are understanding the
process of finding the solution, on your own - after 
all, you get to bring only one brain to exams (and it can't be someone 
else's). For the same reason, I also recommend that you try working 
each problem on your own, first. A small selection of the homework 
problems will be collected, graded, and returned; your homework grades will
contribute up to 50 points toward your total grade.
You should treat the list of assigned problems as an \underbar{absolute} 
\underbar{minimum} collection of problems to work to help you to review 
the material. For any problem that gives you difficulty you should work problems
in its vicinity, since they will focus on similar skills.
\ssk
\ni{\bf Quizzes} will be given in recitation section every Friday that we do not
have a scheduled exam. These will typically consist of one problem modelled on the 
homework assignments from the sections covered up until the Tuesday prior to the quiz.
The quizzes will contribute up to 100 of the points toward your total. No make-up quizzes
will be allowed, but your lowest two quiz grades will be dropped before computing
your score; a missed quiz will be counted as a 0 for this purpose.
\ssk
\ni{\bf Midterm exams} will be given three times during the 
semester, in recitation class, approximately every five weeks - the specific dates are
(currently) February 19, April 2, and April 23. Any deviation from this 
schedule will be announced well in advance.
Each exam will contribute up to 100 points toward your grade. 
You can take a make-up exam only if there are compelling reasons (a doctor SAYS 
you were sick, jury duty, etc.) for you to miss an exam. Make-up 
exams tend to be harder than the originals (because make-up exams 
are harder to write!). 
\ssk
\ni A {\bf Gateway Exam} will test your mastery of integration techniques.
It consists of 7 questions, and you must pass it with a score of 6 out of 7 
or better to receive 
full credit for the exam. No partial credit is awarded; the answer must be
completely correct. You will take the exam on paper in recitation class
shortly after we have covered the neccessary material. 
If you do not pass it the first time you take it, 
you may retake the exam, up to once a (week)day, at the College Testing Center (Burnett 127), 
until March 26. A picture ID will be required when taking the exam
at the CTC. The gateway exam will contribute up to 50 points toward your total.
No calculators, books, or notes are allowed while taking the gateway exam.
\ssk
\ni A {\bf Project} will be assigned for you to work in small groups of no fewer 
than 3 and no more than 5 people. Students from different recitation sections
in our course are welcome to form a group.
The project will explore a longer and more open-ended question than a typical 
homework or exam problem. The goal of the project is two-fold: you and your fellow 
group-members will solve a more challenging problem, and you will write a 
report on your work, describing background, methods, and conclusions. Your 
group will submit a written report on the project and you will be graded on both 
the quality of both the mathematical solution and of the exposition.
The project will contribute up to 50 points towards your total.
\msk
\ni Finally, there will be a regularly scheduled {\bf Final Exam} on 
Thursday, May 6, from 6:00 to 8:00pm. [Note: that this time is \underbar{not}
based on the time that the course meets; it is common to all sections of Math 107.]
The final will cover the entire course. It will contribute up to 200 points 
toward your final grade. In accordance with department policy, you will be allowed 
to bring one 3$\times$5 card with your own hand-written notes to the final exam.
\msk
\ni {\bf Your course grade} will be calculated numerically using the above amounts,
to give a total out of 750 points ($=50+100+3\times 100+50+50+200$),
and will be converted to a letter grade based partly on the overall average of the
class. However, a score of 90\% or better will guarantee some kind of {\bf A}, 80\%
or better some sort of {\bf B}, 70\% or better a flavor of {\bf C}, and 60\% or 
better a {\bf D}.
\bsk 
\ni{\bf Calculators:} A graphing calculator may be useful for this course,
but it is not required. 
The TI-83, 84 and 86 are all reasonable options. No calculators of any kind
may be used during the gateway exam. Calculators may be used during the other 
quizzes and exams (although it is not necessarily
recommended); however, calculators other than those above require the 
approval of the instructor prior to use. 
A calculator with a built-in computer algebra system (CAS), such as the 
TI-89, TI-92, TI-Nspire, HP-40, HP-41, Casio ALGEBRA FX 2.0, and
Casio Classpad 300 and 330, may not be used. 
A cell phone calculator cannot be used during a quiz or exam.
\ssk
\ni{\bf Cell phones} should be silenced for the duration of all classes, and \underbar{extreme}
restraint should be exercised in answering a call during class. If you feel that you must 
answer a call, please excuse yourself from the room before beginning to take the call.
\ssk
\ni The {\bf Math Resource Center} is located in Avery 013B, and students in Math 107 are 
encouraged to use this additional resource if they have questions related to this course, 
or as a place to meet and discuss group projects.  
Hours for the MRC are MTWR 12:30 - 8:30 pm, F 12:30 - 2:30 pm, and Su 1:00 - 5:00pm.
\msk
\ni{\bf Stay current!} In mathematics, new concepts continually rely upon the mastery
of old ones; it is therefore essential that you thoroughly understand each 
new topic before moving on. Our classes are an important opportunity for you to ask
questions; to make \underbar{sure} that you are understanding concepts correctly.
Speak up! It's \underbar{your} education at stake. Make every effort to resist
the temptation to put off work, and to fall behind. Every topic has to be gotten 
through, not around. And it's a lot easier to read 50 pages in a week than it is
in a day. Try to do some mathematics every single day.
{\bf Class attendance} is probably your best way to insure that you will keep 
up with the material, and make sure that you understand all of the
concepts. [And on a more pragmatic note, the instructor writes the exams, so it pays
to know what the instructor said!] Even more, {\bf stay ahead!} You are strongly
encouraged to read the section to be covered in class prior to its presentation in 
lecture; this will both improve your ability to follow the lecture and help to focus your 
attention on any areas where extra effort on your part will be required.
\msk
\ctln{\bf Some important academic dates}
\msk
{\bf Jan. 11} First day of classes.
{\bf Jan. 18} Martin Luther King Day - no classes.
{\bf Jan. 22} Last day to withdraw from a course without a {\bf `W'}.
{\bf Mar. 5} Last day to change to or from P/NP.
{\bf Mar. 14-18} Spring break - no classes.
{\bf Apr. 9} Last day to withdraw from a course.
{\bf May 1} Last day of classes.
\bsk
\ni{\bf Departmental Grading Appeals Policy:} The Department of Mathematics does not 
tolerate discrimination or harassment on the basis of race, gender, 
religion or sexual orientation. If you believe you have been subject to such 
discrimination or harassment, in this or any math course, please contact the 
Department. If, for this or any other reason, you believe your grade was 
assigned incorrectly or capriciously, appeals may be made (in order) to the 
instructor, the Department Chair, the Departmental Grading Appeals Committee, 
the College Grading Appeals Committee, and the University Grading Appeals Committee. 
\bsk
\centerline{\bf Homework Problems, by section}
\bigskip
\item{Section 5.4,} p.351: 5, 8, 9, 14, 20, 25, 29, 30, 34, 35, 41, 44, 57, 74
\item{Section 5.5,} p.358: 1, 2, 6, 8, 14, 19, 22, 23, 32, 37, 40, 51, 61, 64
\item{Section 7.1,} p.453: 1, 3, 6, 7, 8, 10, 13, 20, 21, 25, 28
\item{Section 7.2,} p.460: 1, 4, 5, 7, 11, 16, 24, 25, 29, 34, 37
\item{Section 7.3,} p.463: 1, 4, 5, 7, 11, 15, 24, 25, 32
\item{Section 7.4,} p.469: 1, 3, 5, 10, 12, 16, 20, 21, 25, 31
\item{Section 7.5,} p.476: 15, 21, 37, 40
\item{Section 7.6,} p.484: 15, 19, 20
\item{Section 7.7,} p.495: 1, 2, 4, 7, 10, 13, 17, 24, 25, 35, 42, 51, 52, 55, 58, 66
\item{Section 6.1,} p.399: 1, 5, 8, 15, 17, 20, 23
\item{Section 6.2,} p.406: 2, 3, 9, 10, 15, 16, 17
\item{Section 6.3,} p.413: 1, 2, 3, 8, 9, 11, 17
\item{Section 6.5,} p.428: 6, 8, 9, 12, 16, 23, 26, 35
\item{Section 6.6,} p.433: 2, 5, 7, 8, 13, 17, 18, 23
\item{Section 8.1,} p.511: 4, 7, 11, 16, 19, 21, 23, 26, 27, 32, 36, 41, 43, 45, 50, 69, 75
\item{Section 8.2,} p.522: 1, 3, 5, 7, 8, 13, 16, 21, 23, 24, 25, 26, 29, 36, 45, 48, 49, 51, 56
\item{Section 8.3,} p.527: 2, 4, 6, 9, 11, 12, 16, 20, 25
\item{Section 8.4,} p.532: 2, 3, 4, 6, 10, 11, 20, 21, 25, 34, 35
\item{Section 8.5,} p.536: 1, 3, 4, 6, 7, 9, 12, 14, 15, 18, 21, 23, 27, 30, 41
\item{Section 8.6,} p.542: 2, 3, 6, 9, 12, 13, 15, 20, 25, 26, 32, 36, 37, 45, 47
\item{Section 8.7,} p.552: 2, 3, 6, 7, 9, 11, 13, 22, 23, 25, 27
\item{Section 8.8,} p.558: 1, 3, 6, 8, 11, 13, 15, 18, 22, 23, 25, 26, 27
\item{Section 8.9,} p.567: 2, 5, 8, 15, 17, 19, 21, 23, 25, 27, 29, 33
\item{Section 9.1,} p.581: 1, 4, 6(a, d, h), 8, 9, 11, 13, 17, 24, 26, 27, 30,45, 53, 55
\item{Section 9.2,} p.585: 1, 4, 5, 7, 17-19, 21(a), 24(a)
\item{Section 9.3,} p.589: 2, 3, 7, 9,  13, 14, 17, 19, 23, 24
\item{Section 10.1,} p.617: 1, 3, 6, 9, 13, 15, 17, 20, 22, 29, 35, 38, 41, 45, 49, 53
\item{Section 10.2,} p.626: 3, 6, 9, 10, 13, 15, 17, 21, 23, 25, 28, 33, 40, 41
\item{Section 10.3,} p.634: 1, 3, 8, 13, 15, 27, 29, 31
\item{Section 10.5,} p.650: 1, 2, 6, 16, 19
\item{Section 11.1,} p.670: 1, 4, 5, 8, 9, 11, 15, 16, 19, 21, 23(a, c)
\item{Section 11.2,} p.676: 2, 3, 4, 6, 7, 10, 11 13, 17
\item{Section 11.3,} p.681: 1, 3, 5, 6, 9, 11, 12
\vfill
\end
\vfill\eject
