

\input amstex
\magnification=1200

\loadmsbm

\define\dl{\displaystyle}
\define\ctln{\centerline}
\define\ssk{\smallskip}
\define\msk{\medskip}
\define\bsk{\bigskip}
\define\bbr{{\Bbb R}}
\define\bbc{{\Bbb C}}
\define\ii{{\italic i}}

\overfullrule=0pt
\nopagenumbers

\input epsf

%\documentstyle{amsppt}

\ctln{\bf Math 423/823 Exercise Set 3 Solutions}

\bsk

\bsk

\item{9.} [BC\#2.12.3] If $z=x+y\ii$ and $f(z)=(x^2-y^2-2y)+(2x-2xy)\ii$, use the
formulas 

\ssk 

\ctln{$\dl x={{z+\overline{z}}\over{2}}$ \hskip.1in and \hskip.1in $\dl y={{z-\overline{z}}\over{2\ii}}$}

\ssk

\item{} to write $f(z)$ in terms of $z$ (and $\overline{z}$) and simplify the result.

\msk

\item{} $\dl f(z)=
(\big({{z+\overline{z}}\over{2}}\big)^2-\big({{z-\overline{z}}\over{2\ii}}\big)^2-
2\big({{z-\overline{z}}\over{2\ii}}\big))+(2\big({{z+\overline{z}}\over{2}}\big)-
2\big({{z+\overline{z}}\over{2}}\big)\big({{z-\overline{z}}\over{2\ii}}\big))\ii$

$\dl =(\big({{z+\overline{z}}\over{2}}\big)^2-(-\ii)^2\big({{z-\overline{z}}\over{2}}\big)^2+
2\ii\big({{z-\overline{z}}\over{2}}\big))+(2\big({{z+\overline{z}}\over{2}}\big)+
2\ii\big({{z+\overline{z}}\over{2}}\big)\big({{z-\overline{z}}\over{2}}\big))\ii$

$\dl =\big({{z+\overline{z}}\over{2}}\big)^2+\big({{z-\overline{z}}\over{2}}\big)^2+
\ii\big({z-\overline{z}}\big))+\big({{z+\overline{z}}}\big)\ii-
\big({{z+\overline{z}}}\big)\big({{z-\overline{z}}\over{2}}\big)$

$\dl ={{1}\over{4}}(z^2+2z\overline{z}+\overline{z}^2+z^2-2z\overline{z}+\overline{z}^2-
2z^2+2\overline{z}^2)+2z\ii$

\hskip.2in $\dl ={{1}\over{4}}(4\overline{z}^2)+2z\ii$ = $\overline{z}^2+2z\ii$

\bsk

\item{10.} [BC\#2.14.3]  Sketch the regions onto which the sector

\ssk

\ctln{$A=\{z=re^{\ii\theta} : 0\leq r\leq 1, 0\leq \theta\leq \pi/4\}$}

\ssk

\item{} is mapped by the functions

\ssk

\item{} (a) $w=z^2$ \hskip.2in (b) $w=z^3$ \hskip.2in (c) $w=z^4$

\msk

\item{} Writing the maps in exponential notation, we have 

\item{} (a) $w=(re^{\ii\theta})^2=r^2e^{\ii 2\theta}$ \hskip.1in 
(b) $w=(re^{\ii\theta})^3=r^3e^{\ii 3\theta}$ \hskip.1in 
(c) $w=(re^{\ii\theta})^4=r^4e^{\ii 4\theta}$

\msk

\item{} So the complex numbers with $0\leq r\leq 1$ will, in all cases, be carried to complex numbers
with modulus between 0 and 1 (inclusive), and the complex numbers with arguments between 0 and $\pi/4$
(inclusive) will be carried to the complex numbers with arguments between 0 and 
(a) $\pi/2$, \hskip.1in (b) $3\pi/4$, \hskip.1in and (c) $\pi$ (inclusive), respectively.
So our region $A$ will be mapped to the points in the unit disk whose arguments lie in the
ranges given above. These regions are sketched below.

\bsk
\leavevmode

\ctln{\epsfxsize=4in
{\epsfbox{hw3sol.eps}}}



\vfill\eject

\item{11.} Show that the reciprocal function, $f(z)=1/z$, maps the disk
$D=\{z : |z-1|<2\ \text{and}\ z\neq 0\}$ onto the region that lies \underbar{outside} of the 
circle $\{w: |w+1/3|=2/3\}$.

\bsk

$|1/z+1/3|>2/3$ $\Leftrightarrow$ $\dl |{{1}\over{x+y\ii}}+{{1}\over{3}}|^2>{{4}\over{9}}$ 

\ssk

$\Leftrightarrow$ $\dl \big|{{x+y\ii}\over{x^2+y^2}}+{{1}\over{3}}\big|^2>{{4}\over{9}}$

\ssk

$\Leftrightarrow$ $\dl \big|\big({{x}\over{x^2+y^2}}+{{1}\over{3}}\big)+\ii{{y}\over{x^2+y^2}}\big|^2>{{4}\over{9}}$
\hskip.2in
$\Leftrightarrow$ $\dl \big({{x}\over{x^2+y^2}}+{{1}\over{3}}\big)^2+\big({{y}\over{x^2+y^2}}\big)^2>{{4}\over{9}}$

\ssk

$\Leftrightarrow$ $\dl {{x^2}\over{(x^2+y^2)^2}}+{{2}\over{3}}{{x}\over{x^2+y^2}}+{{1}\over{9}}+{{y^2}\over{(x^2+y^2)^2}}>{{4}\over{9}}$

\ssk

$\Leftrightarrow$ $\dl {{x^2+y^2}\over{(x^2+y^2)^2}}+{{2}\over{3}}{{x}\over{x^2+y^2}}+{{1}\over{9}}>{{4}\over{9}}$
\hskip.2in
$\Leftrightarrow$ $\dl {{1}\over{x^2+y^2}}+{{2}\over{3}}{{x}\over{x^2+y^2}}+{{1}\over{9}}>{{4}\over{9}}$

\ssk

$\Leftrightarrow$ $\dl 9+6x+(x^2+y^2)>4(x^2+y^2)$ [muliplying both sides by $9(x^2+y^2)$]

\ssk

$\Leftrightarrow$ $\dl 3(x^2+y^2)-6x<9$
\hskip.2in
$\Leftrightarrow$ $\dl x^2+y^2-2x<3$
\hskip.2in
$\Leftrightarrow$ $\dl x^2-2x+1+y^2<4$

\ssk

$\Leftrightarrow$ $\dl (x-1)^2+y^2<4$
\hskip.2in
$\Leftrightarrow$ $\dl |(x+y\ii)-1|^2<4$
\hskip.2in
$\Leftrightarrow$ $\dl |(x+y\ii)-1|<2$

\ssk

$\Leftrightarrow$ $\dl |z-1|<2$

\msk

\item{} Then, reading bottom to top, every point in 
$D=\{z : |z-1|<2\ \text{and}\ z\neq 0\}$ lands in $\{w: |w+1/3|>2/3\}$ (notes that $z\neq 0$ is used, because at one point
we divide by $x^2+y^2$), and reading top to bottom every point in $\{w: |w+1/3|>2/3\}$ \underbar{is} $1/z$ for some point
in $D$. So $D$ is mapped (into and) onto $\{w: |w+1/3|>2/3\}$ under $f$.

\bsk

Alternatively, $|1/z+1/3|>2/3$ $\Leftrightarrow$  $\dl\big|{{3+z}\over{3z}}\big|>{{2}\over{3}}$
$\Leftrightarrow$  $\dl|3+z|>2|z|$ \hskip.2in (this uses $|z|\neq 0$)

\ssk

$\Leftrightarrow$  $\dl|3+z|^2>4|z|^2$ $\Leftrightarrow$  $\dl(x+3)^2+y^2>4(x^2+y^2)$

\ssk

and the argument can be finished as above.

\bsk

\item{12.} Find $\dl\lim_{z\rightarrow 1+\ii} {{z^2+z-1-3\ii}\over{z^2-2z+2}}$.

\msk

\item{} Plugging in $z=1+\ii$ yields $0/0$, which implies that $z-(1+\ii)$ evenly divides both
top and bottom. Since
$z^2+z-1-3\ii=(z-r)(z-(1+\ii))$, we must have $-1-3\ii=r(1+\ii)$, so 
$r(1+\ii)(1-\ii)=2r=(-1-3\ii)(1-\ii)=-1-3\ii+\ii-3=-4-2\ii$, and so $r=-2-\ii$. And since $1+\ii$ is a root of 
$z^2-2z+2$, the other root is its conjugate, $1-\ii$. So:

\ssk

\item{} $\dl\lim_{z\rightarrow 1+\ii} {{z^2+z-1+-3\ii}\over{z^2-2z+2}}$
 = $\dl\lim_{z\rightarrow 1+\ii} {{(z-(1+\ii))(z-(-2-\ii))}\over{(z-(1+\ii))(z-(1-\ii))}}$
 = $\dl\lim_{z\rightarrow 1+\ii} {{z-(-2-\ii)}\over{z-(1-\ii)}}$
= $\dl {{(1+\ii)-(-2-\ii)}\over{((1+\ii)-(1-\ii)}}$
= $\dl {{3+2\ii}\over{2\ii}}$ = $\dl {{-3\ii+2}\over{2}}$ = $\dl {{2-3\ii}\over{2}}$  = $\dl 1-{{3}\over{2}}\ii$




\vfill\end










