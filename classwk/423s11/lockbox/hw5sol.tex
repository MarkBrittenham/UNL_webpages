



\input amstex
\magnification=1200


\loadmsbm


\define\dl{\displaystyle}
\define\ctln{\centerline}
\define\ssk{\smallskip}
\define\msk{\medskip}
\define\bsk{\bigskip}
\define\bbr{{\Bbb R}}
\define\bbc{{\Bbb C}}
\define\ii{{\italic i}}
\define\del{\partial}


\overfullrule=0pt
\nopagenumbers


%\documentstyle{amsppt}


\ctln{\bf Math 423/823 Exercise Set 5 Solutions}


\bsk


\bsk


\item{17.} [BC\#2.26.7] Let the function $f(z)=u(x,y)+\ii v(x,y)$ be analytic in a domain $D$ and 
consider the families of level curves \hfill


\ssk


\ctln{${\Cal U}=\{\{(x,y) : u(x,y)=c_1\} : c_1\in\bbc\}$ and 
${\Cal V}=\{\{(x,y) : v(x,y)=c_2\} : c_2\in\bbc\}$.}


\ssk


\item{} Show that wherever they meet, the curves
in ${\Cal U}$ are \underbar{orthogonal} to the curves in ${\Cal V}$. That is, the slopes
of the two curves, at a point of intersection, are negative reciprocals. 


\msk


\item{} [Hint: for each curve
treat it as implicitly defining $y$ as a function of $x$ and use the multivariate chain rule
to, e.g., differentiate both sides of $u(x,y(x))=c_1$ w.r.t. $x$.]

\msk

\item{} With the notation above, suppose the level curve $u(x,y)=c_1$ is given (in a little
neighborhood of a point) by the function $y=g(x)$. Then using the multivariable 
chain rule, since $u(x,g(x))=c_1$ is the constant function, we have

\ssk

\ctln{$\dl 0={{du}\over{dx}}={{\del u}\over{\del x}}+{{\del u}\over{\del y}}{{dy}\over{dx}}
={{\del u}\over{\del x}}+{{\del u}\over{\del y}}g^\prime(x)$, so
$\dl g^\prime(x)=-{{\del u}\over{\del x}}\big/{{\del u}\over{\del y}}$.}

\ssk

\item{} An analogous argument, supposing that $v(x,y)=c_2$ defines $y=h(x)$ as a function of $x$,
gives

\ssk

\ctln{$\dl h^\prime(x)=-{{\del v}\over{\del x}}\big/{{\del v}\over{\del y}}$.}

\ssk

\item{} But since $f$ is analytic, we have $\dl {{\del u}\over{\del x}}={{\del v}\over{\del y}}$
and $\dl {{\del u}\over{\del y}}=-{{\del v}\over{\del x}}$. 
So 

\ssk

\ctln{$\dl g^\prime(x)\cdot h^\prime(x)=
(-{{\del u}\over{\del x}}\big/{{\del u}\over{\del y}})\cdot
(-{{\del v}\over{\del x}}\big/{{\del v}\over{\del y}})
=(-{{\del u}\over{\del x}}\big/{{\del u}\over{\del y}})\cdot
({{\del u}\over{\del y}}\big/{{\del u}\over{\del x}})=-1$,}

\ssk

\item{} so the two level curves have negative reciprocal slopes, so have orthogonal
tangent lines, as desired.


\bsk


\item{18.} [BC\#3.29.12] For $z=x+y\ii$, write $\text{Re}(e^{1/z})$ in terms of $x$ and $y$. Explain why
this function is harmonic in every domain $D$ that does not contain $0$.
 

\msk

\item{} $\dl 1/z=\overline{z}/|z|^2= {{x}\over{x^2+y^2}}-\ii{{y}\over{x^2+y^2}}$, so 


\item{} $\dl \text{Re}(e^{1/z})=\text{Re}(e^{u+iv})=e^u\cos(v)
=e^{{{x}\over{x^2+y^2}}}\cos\Big(-{{y}\over{x^2+y^2}}\Big)
=e^{{{x}\over{x^2+y^2}}}\cos\Big({{y}\over{x^2+y^2}}\Big)$

\msk

\item{} But we know that $h(z)=1/z$ is analytic except at $z=0$ (since it is the quotient of $1$ and $z$, both
analytic everywhere, and the denominator $z$ is non-zero except at $z=0$), and $g(z)=e^z$ is entire,
so their composition $f(z)=g(h(z))=e^{1/z}$ is analytic except where $g$ is not, i.e, except 
at $z=0$. But then the real part of $f(z)$ is consequently harmonic everywhere that $f(z)$ is
analytic. So $\dl e^{{{x}\over{x^2+y^2}}}\cos\Big({{y}\over{x^2+y^2}}\Big)$, is harmonic
on any domain that does not contain $z=0$.

\vfill
\eject


\item{19.} [BC\#3.31.5] Show that:


\ssk


\item{(a):} the set of values of $\log(\ii^{1/2})$ is $\dl \big(n+{{1}\over{4}}\big)\pi\ii$ for $n$ any integer,
and that the same is true for $\dl{{1}\over{2}}\log(\ii)$


\msk

\item{} Since $\ii=e^{\pi i/2}$, the two values of $\ii^{1/2}$ are $e^{\pi i/4}$ and $e^{5\pi i/4}$.
So the values of $\log(\ii^{1/2})$ are

\ssk

\ctln{$\dl \log(e^{\pi i/4})=\ln|e^{\pi i/4}|+\ii\arg(e^{\pi i/4})
=\ln(1)+({{\pi}\over{4}}+2k\pi)\ii = ({{\pi}\over{4}}+2k\pi)\ii$}

\item{} and

\ssk

\ctln{$\dl \log(e^{5\pi i/4})=\ln|e^{5\pi i/4}|+\ii\arg(e^{5\pi i/4})
=\ln(1)+({{5\pi}\over{4}}+2k\pi)\ii = ({{\pi}\over{4}}+(2k+1)\pi)\ii$}

\ssk

\item{} So the one set of values is $\dl {{\pi}\over{4}}$ plus even multiples of $\pi$, and the other is
$\dl {{\pi}\over{4}}$ plus odd multiples of $\pi$, so the complete set of values is $\dl {{\pi}\over{4}}$
plus \underbar{all} multiples of $\pi$.

\ssk

\item{} [Actually, most of your solutions were better than this one....]

\msk

\item{} On the other hand, $\log(\ii)=\ln(e^{\pi\ii/2})+\ii\arg(e^{\pi\ii/2}) = (\pi/2+2k\pi)\ii$, so
$\dl{{1}\over{2}}\log(\ii) = {{1}\over{2}}(\pi/2+2k\pi)\ii = (\pi/4+k\pi)\ii$, as well.

\bsk


\item{(b):} the set of values of $\log(\ii^2)$ is \underbar{not} the same as the set of values for $2\log(\ii)$.

\msk

\item{} In a similar vein, $\log(\ii^2)= \log(e^{\pi\ii})=\arg(e^{\pi\ii}) = 
(\pi+2k\pi)\ii$ = all odd multiples of $\pi\ii$.

\ssk

\item{} But $2\log(\ii) = 2(\pi/2+2k\pi)\ii = (\pi+4\pi)\ii$ consists of \underbar{every}
\underbar{other} odd multiple of $\pi\ii$, so the two sets of values are not the same.


\bsk


\item{20.} For $z=x+y\ii$, does $1^z$ always equal $1$ ?

\msk

\item{} By definition, $a^z=e^{z\log(a)}$. So $1^z=e^{z\log(1)}$. But 
$\log(1)=\ln|1|+\ii\arg(1)=0+\ii(0+2k\pi)=2k\pi\ii$, depending on which branch
of $\arg(z)$ we choose. So $1^z=e^{2k\pi iz}$ which, depending on our choice of
$k$, need not equal $1$. For $k=0$, $1^z=e^{0z}=e^0=1$, but for, e.g., $k=1$,
$1^{\ii}=e^{2\pi\ii\ii}=e^{-2\pi}\neq 1$. So depending upon which value of 
$\log(1)$ that we choose, $1^z$ need not equal $1$.


\msk

\item{} Shorter, pithier solution: $1^{1/2}$ should be \underbar{allowed} to be $-1$,
under any reasonable definition of exponentials, so, no.


\vfill\end




















