



\input amstex
\magnification=1200


\loadmsbm


\define\dl{\displaystyle}
\define\ctln{\centerline}
\define\ssk{\smallskip}
\define\msk{\medskip}
\define\bsk{\bigskip}
\define\bbr{{\Bbb R}}
\define\bbc{{\Bbb C}}
\define\ii{{\italic i}}
\define\ubr{\underbar}


\overfullrule=0pt
\nopagenumbers


%\documentstyle{amsppt}


\ctln{\bf Math 423/823 Exercise Set 6 Solutions}


\bsk


\bsk


\item{21.} [BC\#4.38.4] The integral $\dl \int_0^\pi e^{(1+i)x}\ dx$ is, technically, equal to


\msk


\ctln{$\dl \int_0^\pi e^{x}\cos x\ dx+\ii \int_0^\pi e^{x}\sin x\ dx$}


\msk


\item{} Evaluate these two integrals 
$\dl \int_0^\pi e^{x}\cos x\ dx$ and $\dl\int_0^\pi e^{x}\sin x\ dx$ by 
applying the Fundamental Theorem of Calculus (p.119, bottom) directly to the top
integral and equating the real and imaginary parts.

\msk

\item{} We can find an antiderivative of $f(x)=e^{(1+i)x}$ via the chain rule:
$\dl{{d}\over{dx}}(e^{cx})=ce^{cx}$, so $\dl F(x)={{1}\over{1+i}}e^{(1+i)x}
={{1-i}\over{2}}e^{(1+i)x}$ has derivative $f(x)$.

\ssk

\item{} So $\dl \int_0^\pi e^{(1+i)x}\ dx = {{1-i}\over{2}}e^{(1+i)x}\Big|_0^\pi$

\hfill $\dl = 
{{1-i}\over{2}}(e^{(1+i)\pi}-e^{(1+i)0}) = 
{{1-i}\over{2}}(e^{\pi}e^{i\pi}-1) = -(e^{\pi}+1)/2+\ii(e^{\pi}+1)/2$.

\ssk

\item{} Equating the real and imaginary parts, we then have

\ssk

\item{} $\int_0^\pi e^{x}\cos x\ dx = -(e^{\pi}+1)/2$ and 
$\dl \int_0^\pi e^{x}\sin x\ dx = (e^{\pi}+1)/2$.


\bsk


\item{22.} Find a parametriation of the curve which follows the circle of radius
$2$ counterclockwise from $z=2$ to $z=2\ii$, \ubr{followed} \ubr{by} the line segment
that runs from $z=2\ii$ to $z=-1$.


\msk


\item{} [Note: there are literally an infinite number of ways to answer this 
question (correctly!);
take pity on your poor instructor when choosing your parametrization....]
 

\msk

\item{} We can parametrize the (one-fourth of the) circle as $\alpha(t)=2e^{it}$ for
$0\leq t\leq \pi/2$, and we can parametrize the line segment as 
$\beta(t)=(1-t)(2\ii)+t(-1)$ for $0\leq t\leq 1$. If we shift the time
interval for $\alpha$ to $-\pi/2\leq t\leq 0$ we can stitch the two parametrizations
together:

\msk

$\gamma(t)=\cases
2e^{i(t+\pi/2)} & \text{if } -\pi/2\leq t\leq 0\cr
(1-t)(2\ii)-t & \text{if } 0\leq t\leq 1\crcr
\endcases$

\msk

\item{} is one possible parametrization.

\bsk


\item{23.} [BC\#3.42.1(part)] Find the integrals $\dl \int_C{{z+2}\over{z}}\ dz$, 
where


\msk


\item{} (a): $C$ is the semicircle $z=2e^{\ii\theta}$, $0\leq \theta\leq \pi$

\msk

\item{} We now have much better tools to compute these contour integrals, but in the spirit
in which the problems were intended:

\msk

\item{} $C(\theta)=2e^{\ii\theta}$, so $C^\prime(\theta)=2\ii e^{\ii\theta}$, and

\ssk

\item{} $\dl \int_C{{z+2}\over{z}}\ dz = 
\int_0^\pi{{2e^{\ii\theta}+2}\over{2e^{\ii\theta}}}\ 2\ii e^{\ii\theta}\ d\theta
 = \int_0^\pi 2\ii e^{\ii\theta}+2\ii\ d\theta = [2e^{\ii\theta}+2\ii\theta]\Big|_0^\pi$

\item{} $\dl = [2e^{\ii\pi}+2\ii\pi]-[2e^{\ii 0}+2\ii(0)] = -2+2\pi\ii-2 = 2\pi\ii-4$.

\msk


\item{} (c): $C$ is the circle $z=2e^{\ii\theta}$, $0\leq \theta\leq 2\pi$

\msk

\item{}Stealing much of the work from the first part, 

\ssk

\item{}$\dl \int_C{{z+2}\over{z}}\ dz = 
\int_0^{2\pi}{{2e^{\ii\theta}+2}\over{2e^{\ii\theta}}}\ 2\ii e^{\ii\theta}\ d\theta
 = \int_0^{2\pi} 2\ii e^{\ii\theta}+2\ii\ d\theta = [2e^{\ii\theta}+2\ii\theta]\Big|_0^{2\pi}$

\item{} $\dl = [2e^{2\ii\pi}+4\ii\pi]-[2e^{\ii 0}+2\ii(0)] = 2+4\pi\ii-2 = 4\pi\ii$.


\bsk


\item{24.} [BC\#4.42.8] Find the integral $\dl \int_C z^n(\overline{z})^m\ dz$, where
$C$ is the unit circle $|z|=1$ traversed in a counterclockwise direction.


\msk


\item{} [Note: you will fnd it helpful to know that $\dl \int_{0}^{2\pi} e^{\ii k\theta}\ d\theta$
is $0$ if $k\neq 0$, and $2\pi$ if $k=0$. You need not prove this.]


\msk


\item{} Extra credit: why does it not matter \ubr{where} we choose to start our parametrization
of the circle (i.e., at what point along the circle)?


\msk

\item{} Parametrizing the circle as $\gamma(t)=e^{it}$ for $0\leq t\leq 2\pi$, we have
$\gamma^\prime(t) = \ii e^{it}$ and

\ssk

\item{} $\dl \int_C z^n(\overline{z})^m\ dz = \int_0^{2\pi} (e^{it})^n (\overline{e^{it}})^m\ii e^{it}\ dt
= \ii\int_0^{2\pi} (e^{it})^{n+1}(e^{-it})^m\ dt$

\item{} $\dl
= \ii\int_0^{2\pi} (e^{it})^{n-m+1}\ dt = \ii\int_0^{2\pi} e^{i(n-m+1)t}\ dt$,

\ssk

\item{} which, according to our note, is $0$ if $n-m+1\neq 0$, and is $2\pi\ii$ if $n-m+1=0$. More precisely,

\msk

\item{} $\dl \int_C z^n(\overline{z})^m\ dz$ equals $0$, if $m\neq n+1$, and equals $2\pi\ii$, if $m=n+1$.

\bsk

\item{} It doesn't matter where we start the parametrization, so long as we go counterclockwise, since if
we chose a different starting point, then using \underbar{both}
starting points, we could cut $C$ into two pieces, and treat it as \underbar{two} paths taken one after the other.
Then we would write $C$ as $C_1$ followed by $C_2$. Using the other starting point, it would
be $C_2$ followed by $C_1$, so we would be comparing 

\ssk

\ctln{$\dl\int_{C_1}f(z)\ dz + \int_{C_2}f(z)\ dz$
\hskip.3in to \hskip.3in $\dl\int_{C_2}f(z)\ dz + \int_{C_1}f(z)\ dz$,}

\ssk

\item{} which are the same!



\vfill\end




















