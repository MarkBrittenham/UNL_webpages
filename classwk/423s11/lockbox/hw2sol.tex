

\input amstex
\magnification=1200

\loadmsbm

\define\dl{\displaystyle}
\define\ctln{\centerline}
\define\ssk{\smallskip}
\define\msk{\medskip}
\define\bsk{\bigskip}
\define\bbr{{\Bbb R}}
\define\bbc{{\Bbb C}}
\define\ii{{\italic i}}

\overfullrule=0pt
\nopagenumbers

%\documentstyle{amsppt}

\ctln{\bf Math 423/823 Exercise Set 2 Solutions}

\bsk

\bsk

\item{5.} [BC\#1.8.1 sort of] Find the exponential form ($re^{\ii\theta}$) of the following
numbers:

\msk

\item{} (a) $\dl z={{\ii}\over{-2-2\ii}}$ \hskip1in (b) $z=(\sqrt{3}-\ii)^6$


\msk

\item{(a)} $\dl z={{\ii}\over{-2-2\ii}} = 
{{\ii(-2+2\ii)}\over{(-2-2\ii)(-2+2\ii)}} = 
{{-2-2\ii}\over{4+4}} = {{-1-\ii}\over{4}}$

\ssk

\item{} So $\dl |z|=(1/4)\sqrt{(-1)^2+(-1)^2}=\sqrt{2}/4$ and $\text{Arg}(z)=\text{arctan}((-1)/(-1))=\pi/4-\pi=-3\pi/4$,
since $(-1,-1)$ is in the third quadrant. So $z=(\sqrt{2}/4)e^{-3\pi\ii/4}$.

\msk

\item{(b)} $\dl z=(\sqrt{3}-\ii)^6$ can be snuck up upon. $w=\sqrt{3}-\ii$ has $|w|=\sqrt{3+1}=2$, and
$\text{arg}(w)=\arctan(-1/\sqrt{3})=-\pi/6$. 
So $|z|=|w|^6=2^6=64$, and $\text{arg}(z)=6\text{arg}(w)=6(-\pi/6)=-\pi$. So $\text{Arg}(z)=-\pi+2\pi=\pi$.
So $z=64e^{\ii\pi}$.

\bsk

\item{6.} [BC\#1.8.8] Show that for complex numbers $z_1$ and $z_2$ we have $|z_1|=|z_2|$
if and only if there are complex numbers $c_1$ and $c_2$ so that $z_1=c_1c_2$ and $z_2=c_1\overline{c_2}$.

\msk

\item{} For $(\Leftarrow)$, since $|\overline{c_2}|=|c_2|$, we have

\item{} $|z_1|=|c_1c_2|=|c_1|\cdot|c_2|=|c_1|\cdot|\overline{c_2}|=|c_1\overline{c_2}|=|z_2|$.

\msk

\item{} For $(\Rightarrow)$, if we write $z_1=r_1e^{\ii\theta_1}$ and $z_2=r_2e^{\ii\theta_2}$, then 
$r_1=|z_1|=|z_2|=r_2$. So $z_1/z_2=e^{\ii\theta_1}/e^{\ii\theta_2}=e^{\ii(\theta_1-\theta_2)}
=e^{\ii\theta}$. 

\msk

\item{} In essence, the key to what we want is the fact that $\dl x= {{x+y}\over{2}}+{{x-y}\over{2}}$ and
$\dl y= {{x+y}\over{2}}-{{x-y}\over{2}}$. So if we set $\dl\theta={{\theta_1+\theta_2}\over{2}}$
and $\dl\psi={{\theta_1-\theta_2}\over{2}}$, then $\theta_1=\theta+\psi$ and $\theta_2=\theta-\psi$.

\msk

\item{} So if we set $c_1=r_1e^{\ii\theta}$ and $c_2=e^{\ii\psi}$, then $\overline{c_2}=e^{-\ii\psi}$
and $c_1c_2=r_1e^{\ii(\theta+\psi)}=r_1e^{\ii\theta_1}=z_1$ and 
$c_1\overline{c_2}=r_1e^{\ii(\theta-\psi)}=r_1e^{\ii\theta_2}=r_2e^{\ii\theta_2}=z_2$, as desired.

\msk

\item{} (Note: technically, this argument fails if $z_1$ or $z_2$ is 0 (i.e., has modulus 0), but then both are
0, and we can set $c_1=0$ and $c_2=$anything!)

\bsk

\item{7.} [BC\#1.10.3] Find all of the roots indicated:

\msk

\item{} (a) $(-1)^{1/3}$ \hskip1in (b) $8^{1/6}$

\msk

\item{(a)} $(-1)^3=-1$, and so one cube root of $-1$ is $-1=e^{\ii\pi}$. So the set of all
cube roots are

\ssk

\item{} $e^{\ii\pi}$, $e^{\ii(\pi-2\pi/3)}=e^{\ii(\pi-2\pi/3)}=e^{\ii\pi/3}$, and $e^{\ii(\pi-4\pi/3)}=e^{-\ii\pi/3}$

\ssk

\item{} [Or, if you prefer, $-1=e^{\ii\pi}$, so one cube root is $e^{\ii\pi/3}$ and work from there.]

\vfill\eject

\item{(b)} $8^{1/6}=(2^3)^{1/6}=\sqrt{2}=\sqrt{2}e^{0\ii}$ is one root, and so the set of six are

\ssk

\item{} $\sqrt{2}e^{0\ii}$, $\sqrt{2}e^{\ii\pi/3}$, $\sqrt{2}e^{\ii2\pi/3}$, 
$\sqrt{2}e^{\ii\pi}=-\sqrt{2}$, $\sqrt{2}e^{\ii4\pi/3}=\sqrt{2}e^{-\ii2\pi/3}$, and
$\sqrt{2}e^{\ii5\pi/3}=\sqrt{2}e^{-\ii\pi/3}$

\bsk

\item{8.} Show that $(z-z_0)(z-\overline{z_0})=z^2-2\text{Re}(z_0)z+|z_0|^2$, which has real
coefficients. Use this and the results of Problem \#7 to express the polynomial
$p(x)=x^6-8$ as a product of linear and quadratic polynomials with real coefficients.

\msk

\item{} $(z-z_0)(z-\overline{z_0})=z^2-zz_0-z\overline{z_0}+z_0\overline{z_0}=
z^2-z(z_0+\overline{z_0})+|z_0|^2$, But if
$z_0=z_0+\ii y_0$, then $z_0+\overline{z_0}=2x_0=2\text{Re}(z_0)$, so

\ssk

\item{} $(z-z_0)(z-\overline{z_0})=z^2-(2\text{Re}(z_0))z+|z_0|^2$. And since $1$, $2\text{Re}(z_0)$ and
$|z_0|^2$ are real, this quadratic polynomial has real coefficients.

\msk

\item{} The polynomial $p(x)=x^6-8$ has roots the sixth roots $r_i$ of 8, which were found in problem \#7(b).
Writing $x^6-8=(x-r_1)(x-r_2)(x-r_3)(x-r_4)(x-r_5)(x-r_6)$, and collecting conjugate pairs together, we
have

\msk

\item{} $(x-\sqrt{2}e^{\ii\pi/3})(x-\sqrt{2}e^{-\ii\pi/3})=x^2-2\text{Re}(\sqrt{2}e^{\ii\pi/3})x+2=
x^2-2\sqrt{2}\cos(\pi/3)x+1=x^2-2\sqrt{2}(1/2)x+1=x^2-\sqrt{2}x+2$, and 

\ssk

\item{} $(x-\sqrt{2}e^{\ii 2\pi/3})(x-\sqrt{2}e^{-\ii 2\pi/3})=x^2-2\text{Re}(\sqrt{2}e^{\ii 2\pi/3})x+2=
x^2-2\sqrt{2}\cos(2\pi/3)x+1=x^2-2\sqrt{2}(-1/2)x+1=x^2+\sqrt{2}x+2$.

\msk

\item{} Putting it all together, we get

\item{} $x^6-8=(x-\sqrt{2})(x+\sqrt{2})(x^2-\sqrt{2}x+2)(x^2+\sqrt{2}x+2)$

\bsk

\item{} As a check, we can multiply things together! As a shortcut, 

\item{} $(x+\sqrt{2})(x^2-\sqrt{2}x+2)=x^3+(\sqrt{2})^3$, and 

\item{} $(x-\sqrt{2})(x^2+\sqrt{2}x+2)=x^3-(\sqrt{2})^3$,
whose product is $x^6-(\sqrt{2})^6=x^6-8$.


\vfill\end










