

\input amstex
\magnification=1200

\loadmsbm

\define\dl{\displaystyle}
\define\ctln{\centerline}
\define\ssk{\smallskip}
\define\msk{\medskip}
\define\bsk{\bigskip}
\define\bbr{{\Bbb R}}
\define\bbc{{\Bbb C}}
\define\ii{{\italic i}}
\define\hhh{\hskip.72in}
\define\hhhh{\hskip.3in}

\overfullrule=0pt
\nopagenumbers

%\documentstyle{amsppt}

\ctln{\bf Math 423/823 Exercise Set 1 Solutions}

\msk

\ctln{Due Thursday, Jan. 27}

\bsk

\bsk

\item{1.} [BC\#1.2.6(b)] For complex numbers $z_1=a_1+b_1\ii$, etc., verify the
distributive law:

\msk

\ctln{$z_1(z_2+z_3)=z_1z_2+z_1z_3$}

\msk

\item{} We can write out both sides, using our rules for addition and muliplication:

\ssk

\item{} $z_1(z_2+z_3) = (a_1+b_1\ii)((a_2+b_2\ii)+(a_3+b_3\ii))$

\hhh $= (a_1+b_1\ii)((a_2+a_3)+(b_2+b_3)\ii)$

\hhh $= (a_1(a_2+a_3)-b_1(b_2+b_3))+(a_1(b_2+b_3)+(a_2+a_3)b_1)\ii$

\hhh $= (a_1a_2+a_1a_3-b_1b_2-b_1b_3)+(a_1b_2+a_1b_3+a_2b_1+a_3b_1)\ii$

\ssk

\item{} $z_1z_2+z_1z_3 = (a_1+b_1\ii)(a_2+b_2\ii)+(a_1+b_1\ii)(a_3+b_3\ii)$

\hhh $ =
((a_1a_2-b_1b_2)+(a_1b_2+a_2b_1)\ii)+((a_1a_3-b_1b_3)+(a_1b_3+a_3b_1)\ii)$

\hhh $ =
((a_1a_2-b_1b_2)+(a_1a_3-b_1b_3)) +(a_1b_2+a_2b_1)+(a_1b_3+a_3b_1))\ii$

\hhh $ = 
(a_1a_2+a_1a_3-b_1b_2-b_1b_3) +(a_1b_2+a_1b_3+a_2b_1+a_3b_1)\ii$

\ssk

\item{} But these last two expressions are identical! So  $z_1(z_2+z_3)=z_1z_2+z_1z_3$, as desired.

\ssk

\item{} [N.B.: apparently I didn't follow my own advice....]

\bsk

\item{2.} [BC\#1.3.1] Reduce each of the quantities to a real number:

\msk

\ctln{(a) $\dl {{1+2\ii}\over{3-4\ii}}+{{2-\ii}\over{5\ii}}$ \hskip1in 
(c) $(1-\ii)^4$}

\msk

\item{} $\dl {{1+2\ii}\over{3-4\ii}}+{{2-\ii}\over{5\ii}} = 
{{(1+2\ii)(3+4\ii)}\over{3^2+4^2}}+{{(2-\ii)(-5\ii)}\over{0^2+5^2}}$

$\dl = 
{{(3-8)+(4+6)\ii}\over{25}}+{{(0-5)+(-10+0)\ii}\over{25}}$

$\dl = 
{{(-5-5)+(10-10)\ii}\over{25}} = {{-10}\over{25}} = {{-2}\over{5}}$

\msk

\item{} $(1-\ii)^4 = [(1-\ii)^2]^2 = [(1-\ii)(1-\ii)]^2$

$ = 
[((1)(1)-(-1)(-1))+((1)(-1)+(1)(-1))\ii]^2$

$ = [-2\ii]^2=(-2)^2\ii^2=(4)(-1)=-4$

\bsk

\item{3.} [BC\#1.5.11] Use mathematical induction to show that for all
natural numbers $n$, and complex numbers $z_1,\ldots,z_n$, 

\msk

\ctln{$\overline{z_1+\cdots +z_n}=\overline{z_1}+\cdots +\overline{z_n}$ \hskip.3in and
\hskip.3in $\overline{z_1\cdots z_n}=\overline{z_1}\cdots \overline{z_n}$}

\msk

\item{} From class, or direct computation, we know that $\overline{z_1+z_2} = \overline{z_1}+\overline{z_2}$
and $\overline{z_1z_2} = \overline{z_1}\cdot\overline{z_2}$. This is the base case of our
induction. [Technically, $n=1$ really is, and $\overline{z_1}=\overline{z_1}$ is even 
more immediately true.] For our inductive hypothesis, we suppose that

\ssk

\ctln{$\overline{z_1+\cdots +z_n}=\overline{z_1}+\cdots +\overline{z_n}$ \hskip.3in and
\hskip.2in $\overline{z_1\cdots z_n}=\overline{z_1}\cdots \overline{z_n}$}

\ssk

\item{} and show that the result is also true with $n$ replaced by $n+1$:

\vfill\eject

\item{} $\overline{z_1+\cdots +z_{n+1}}$

$ = \overline{(z_1+\cdots +z_n)+z_{n+1}}
= \overline{(z_1+\cdots +z_n)}+\overline{z_{n+1}} =
(\overline{z_1}+\cdots +\overline{z_n})+\overline{z_{n+1}}$

$ = \overline{z_1}+\cdots +\overline{z_{n+1}}$

\ssk

\item{} where the third equality is the base case of our induction, and the fourth equality 
is our inductive hypothesis. So by induction, the result holds for all $n$.
Similarly, 

\msk

\item{} $\overline{z_1\cdots z_{n+1}}$

$ = \overline{(z_1\cdots z_n)\cdot z_{n+1}} =
\overline{(z_1\cdots z_n)}\cdot\overline{z_{n+1}} =
(\overline{z_1}\cdots \overline{z_n})\cdot\overline{z_{n+1}}$

$ =
\overline{z_1}\cdots \overline{z_{n+1}}$

\ssk

\item{} where, again, the third equality is the base case of our induction, and the fourth equality 
is our inductive hypothesis. So by induction, the result also holds for all $n$.

\bsk

\item{4.} Show that if $p(x)=a_nx^n+\cdots +a_0$ is a polynomial with
real coefficients, and $z=a+b\ii$ is a complex root of $p$ [i.e., 
$p(z)=a_nz^n+\cdots +a_0=0$], then $\overline{z}$ is \underbar{also} a root of $p$.

\msk

\item{} The main point is that the coefficients, being real, are equal to their own complex conjujates;
$a_i=a_i+0\ii$, so $\overline{a_i}=a_i-0\ii=a_i$. Then if $z$ is a root of $f$, since, by problem 
\#3, $\overline{a_iz^i}=\overline{a_i}(\overline{z})^i=a_i(\overline{z})^i$, we have

\ssk

\item{} $\overline{p(z)} = 
\overline{a_nz^n+\cdots +a_1z+a_0}$

\hhhh $ = \overline{a_nz^n}+\cdots +\overline{a_1z}+\overline{a_0}$

\hhhh $ =
\overline{a_n}(\overline{z})^n+\cdots +\overline{a_1}\overline{z}+\overline{a_0}$

\hhhh $ = 
a_n(\overline{z})^n+\cdots +a_1\overline{z}+a_0$

\hhhh $ = p(\overline{z})$

\ssk

\item{} But since $z$ is a root of $p$, $p(z)=0$, and so $p(\overline{z})=\overline{p(z)}=\overline{0}=0$,
so $\overline{z}$ is a root of $p$, as well.

\ssk

\item{} [N.B.: Note that this line of work only works for polynomials with real coeeficients! For example,
the roots of $p(z)=z^2-\ii$ (find them!) are not complex conjugates of one another....]


\vfill\end










