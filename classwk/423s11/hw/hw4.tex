

\input amstex
\magnification=1200

\loadmsbm

\define\dl{\displaystyle}
\define\ctln{\centerline}
\define\ssk{\smallskip}
\define\msk{\medskip}
\define\bsk{\bigskip}
\define\bbr{{\Bbb R}}
\define\bbc{{\Bbb C}}
\define\ii{{\italic i}}

\overfullrule=0pt
\nopagenumbers

%\documentstyle{amsppt}

\ctln{\bf Math 423/823 Exercise Set 4}

\msk

\ctln{Due Thursday, Feb. 17}

\bsk

\bsk

\item{13.} [BC\#2.18.11] Show that if $\dl T(z)={{az+b}\over{cz+d}}$ 
(where $a,b,c,d\in{\Bbb C}$ and $ad-bc\neq 0$ then 

\msk

\item{(a)} if $c=0$ then $\dl\lim_{z\rightarrow\infty} T(z) = \infty$.

\msk

\item{(b)} if $c\neq 0$ then $\dl\lim_{z\rightarrow\infty} T(z) = {{a}\over{c}}$ and
$\dl\lim_{z\rightarrow -d/c} T(z) = \infty$.

\bsk

\item{14.} [BC\#2.20.9]  Let $f$ be the function $\dl f(z)=\cases
(\overline{z})^2/z & \text{if}\ z\neq 0\\
0 & \text{if}\ z=0
\endcases$ . 

\ssk

\item{} Show that $f$ is not differentiable at $0$, even though the
limit of the difference quotient exists (and both agree) when you let $\Delta z\rightarrow 0$
along the vertical and horizontal axes; show that if you approach $0$ along the line $h=k$ (where
$\Delta z = h+\ii k$) you find a different limit.

\bsk

\item{15.} [BC\#2.23.6] Revisit problem \#14 from the viewpoint of the Cauchy-Riemann equations. 
That is, write $f(z)=f(x+\ii y)=u(x,y)+\ii v(x,y)$ (noting that we define $u(0,0)=v(0,0)=0$). 
Show that $u_x,u_y,v_x$, and $v_y$ all exist at $(0,0)$ {\it and} that they satisfy the
Cauchy-Riemann equations at $(0,0)$. 

\ssk

\item{} [N.B. This shows that the CR equations alone are not enough to guarantee differentiability
at a point.]
 
\bsk

\item{16.} Let $f(z)=z^3+1$ and $\dl a={{1+\sqrt{3}\ii}\over{2}}$ , $\dl b={{1-\sqrt{3}\ii}\over{2}}$.
Show that there is {\it no} value of $w$ on the  line segment 
$\dl\{ {{1+t\sqrt{3}\ii}\over{2}}\ :\ -1\leq t\leq 1\}$
where $\dl f^\prime(w)={{f(b)-f(a)}\over{b-a}}$.

\ssk

\item{} (Note: It is probably most efficient to determine \underbar{all} of the $w$ for 
which $f^\prime(w)$ = that specific value, and show that none of those points lie on the
line segment!)

\ssk

\item{} [N.B. Consequently, the direct analogue of the Mean Value Theorem (central to most
theoretical results in calculus!) does not hold in general for analytic functions.]


\vfill\end










