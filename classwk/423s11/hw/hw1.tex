

\input amstex
\magnification=1200

\loadmsbm

\define\dl{\displaystyle}
\define\ctln{\centerline}
\define\ssk{\smallskip}
\define\msk{\medskip}
\define\bsk{\bigskip}
\define\bbr{{\Bbb R}}
\define\bbc{{\Bbb C}}
\define\ii{{\italic i}}

\overfullrule=0pt
\nopagenumbers

\documentstyle{amsppt}

\ctln{\bf Math 423/823 Homework 1}

\msk

\ctln{Due Thursday, Jan. 27}

\bsk

\bsk

\item{1.} [BC\#1.2.6(b)] For complex numbers $z_1=a_1+b_1\ii$, etc., verify the
distributive law:

\msk

\ctln{$z_1(z_2+z_3)=z_1z_2+z_1z_3$}

\msk

\item{} [N.B.: probably better to write things as $z_1=(a_1,b_1)$, etc. and expand out
both sides, to avoid inadvertently `assuming' something?]

\bsk

\item{2.} [BC\#1.3.1] Reduce each of the quantities to a real number:

\msk

\ctln{(a) $\dl {{1+2\ii}\over{3-4\ii}}+{{2-\ii}\over{5\ii}}$ \hskip1in 
(c) $(1-\ii)^4$}

\bsk

\item{3.} [BC\#1.5.11] Use mathematical induction to show that for all
natural numbers $n$, and complex numbers $z_1,\ldots,z_n$, 

\msk

\ctln{$\overline{z_1+\cdots +z_n}=\overline{z_1}+\cdots +\overline{z_n}$ \hskip.3in and
\hskip.3in $\overline{z_1\cdots z_n}=\overline{z_1}\cdots \overline{z_n}$}

\bsk

\item{4.} Show that if $p(x)=a_nx^n+\cdots +a_0$ is a polynomial with
real coefficients, and $z=a+b\ii$ is a complex root of $p$ [i.e., 
$p(z)=a_nz^n+\cdots +a_0=0$], then $\overline{z}$ is \underbar{also} a root of $p$.


\vfill\end










