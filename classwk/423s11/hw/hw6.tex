

\input amstex
\magnification=1200

\loadmsbm

\define\dl{\displaystyle}
\define\ctln{\centerline}
\define\ssk{\smallskip}
\define\msk{\medskip}
\define\bsk{\bigskip}
\define\bbr{{\Bbb R}}
\define\bbc{{\Bbb C}}
\define\ii{{\italic i}}
\define\ubr{\underbar}

\overfullrule=0pt
\nopagenumbers

%\documentstyle{amsppt}

\ctln{\bf Math 423/823 Exercise Set 6}

\msk

\ctln{Due Thursday, Mar. 17}

\bsk

\bsk

\item{21.} [BC\#4.38.4] The integral $\dl \int_0^\pi e^{(1+i)x}\ dx$ is, technically, equal to

\msk

\ctln{$\dl \int_0^\pi e^{x}\cos x\ dx+\ii \int_0^\pi e^{x}\sin x\ dx$}

\msk

\item{} Evaluate these two integrals 
$\dl \int_0^\pi e^{x}\cos x\ dx$ and $\dl\int_0^\pi e^{x}\sin x\ dx$ by 
applying the Fundamental Theorem of Calculus (p.119, bottom) directly to the top
integral and equating the real and imaginary parts.

\bsk

\item{22.} Find a parametriation of the curve which follows the circle of radius
$2$ counterclockwise from $z=2$ to $z=2\ii$, \ubr{followed} \ubr{by} the line segment
that runs from $z=2\ii$ to $z=-1$.

\msk

\item{} [Note: there are literally an infinite number of ways to answer this question (correctly!);
take pity on your poor instructor when choosing your parametrization....]
 
\bsk

\item{23.} [BC\#3.42.1(part)] Find the integrals $\dl \int_C{{z+2}\over{z}}\ dz$, 
where

\msk

\item{} (a): $C$ is the semicircle $z=2e^{\ii\theta}$, $0\leq \theta\leq \pi$

\item{} (c): $C$ is the circle $z=2e^{\ii\theta}$, $0\leq \theta\leq 2\pi$

\bsk

\item{24.} [BC\#4.42.8] Find the integral $\dl \int_C z^n(\overline{z})^m\ dz$, where
$C$ is the unit circle $|z|=1$ traversed in a counterclockwise direction.

\msk

\item{} [Note: you will fnd it helpful to know that $\dl \int_{0}^{2\pi} e^{\ii k\theta}\ d\theta$
is $0$ if $k\neq 0$, and $2\pi$ if $k=0$. You need not prove this.]

\msk

\item{} Extra credit: why does it not matter \ubr{where} we choose to start our parametrization
of the circle (i.e., at what point along the circle)?



\vfill\end










