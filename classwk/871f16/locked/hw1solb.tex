\input amstex

\magnification=1200

\def\ctln{\centerline}
\def\msk{\medskip}
\def\bsk{\bigskip}
\def\ssk{\smallskip}
\def\ra{\rightarrow}
\def\ubr{\underbar}
\def\nidt{\noindent}

\UseAMSsymbols

\ctln{\bf Math 871 Problem Set 1 Solutions}

\msk

\item{1.} [Munkres, p.14, \#2 (part)] For each statement below,
determine whether or not it is true. If true, show why; if 
not, give an example demonstrating this.

\ssk

\item{(*)} (a) For any sets $A,B$, $A \setminus (A\setminus B) = B$ .

\ssk

Since $A\setminus\text{(anything)}$ is contained in $A$, this statement is false
whenever $B$ is not contained in $A$. For example, if $A=\{1,2\}$ and $B=\{3\}$, then 
$A\setminus B = A$, and so $A\setminus(A\setminus B)=A\setminus A = \emptyset$, 
not $B$.

\ssk

\item{(*)} (f) For any sets $A,B,C,D$, $(A\times B)\cap(C\times D) = (A\cap C)\times (B\cap D)$ .

\ssk

This is true; if $(a,b)\in (A\times B)\cap(C\times D)$, then $(a,b)\in A\times B$ and $(a,b)\in C\times D$.
So $a\in A$, $b\in B$, $a\in C$, and $b\in D$, and so $a\in A\cap C$ and $b\in B\cap D$.
Therefore $(a,b)\in (A\cap C)\times (B\cap D)$, and so $(A\times B)\cap(C\times D) \subseteq (A\cap C)\times (B\cap D)$.
For the opposite inclusion, if 
$(a,b)\in (A\cap C)\times (B\cap D)$, then $a\in A\cap C$ and $b\in B\cap D$. So
$a\in A$, $a\in C$, $b\in B$, $b\in D$, and so $(a,b)\in A\times B$ and $(a,b)\in B\times D$,
and so $(a,b)\in (A\times B)\cap(C\times D)$, as desired. With both inclusions established, we
know that the two sets are equal.

\msk

\item{4.} [Munkres, p.20, \#2 (part)] If $f:A\ra B$ is a function, $A_0,A_1\subseteq A$,
and $B_0,B_1\subseteq B$, then

\ssk

\item{(*)} (a) $f^{-1}(B_0\cap B_1)$ = $f^{-1}(B_0) \cap f^{-1}(B_1)$

\ssk

$f^{-1}(B_0\cap B_1)$ = $\{x\in A\ :\ f(x)\in B_0\cap B_1\}$

\hskip1in = $\{x\in A\ :\ f(x)\in B_0\ \text{and}\ f(x)\in B_1\}$

\hskip1in = $\{x\in A\ :\ f(x)\in B_0\}\cap\{x\in A\ :\ f(x)\in B_1\}$

\hskip1in = $f^{-1}(B_0) \cap f^{-1}(B_1)$ .

\ssk

Or: If $x\in f^{-1}(B_0\cap B_1)$, then $f(x)\in B_0\cap B_1$, so 
$f(x)\in B_0$ and $f(x)\in B_1$, so $x\in f^{-1}(B_0)$ and $x\in f^{-1}(B_1)$,
and so $x\in f^{-1}(B_0) \cap f^{-1}(B_1)$. This shows that 
$f^{-1}(B_0\cap B_1) \subseteq f^{-1}(B_0) \cap f^{-1}(B_1)$.

For the reverse inclusion, if $x\in f^{-1}(B_0) \cap f^{-1}(B_1)$, then 
$x\in f^{-1}(B_0)$ and $x\in f^{-1}(B_1)$, so $f(x)\in B_0$ and $f(x)\in B_1$.
This implies that $f(x)\in B_0\cap B_1$ and so $x\in f^{-1}(B_0\cap B_1)$. 
This gives $f^{-1}(B_0) \cap f^{-1}(B_1)\subseteq f^{-1}(B_0\cap B_1)$. 
Taken together the two inclusions give 
$f^{-1}(B_0\cap B_1)$ = $f^{-1}(B_0) \cap f^{-1}(B_1)$ .

\ssk

\item{(*)} (c) $f(A_0\cap A_1) \subseteq f(A_0)\cap f(A_1)$, but equality does not always hold.

\ssk

If $x\in f(A_0\cap A_1)$, then $x=f(a)$ for some $a\in A_0\cap A_1$, and so $a\in A_0$ and
$a\in A_1$. Then $f(a)\in f(A_0)$ and $f(a)\in f(A_1)$, and so 
$x=f(a)\in f(A_0)\cap f(A_1)$. So everything that is in $f(A_0\cap A_1)$ is also in $f(A_0)\cap f(A_1)$,
giving containment.

\ssk

In general, these sets are not equal; in the most extreme case, we may have 
$A_0\cap A_1=\emptyset$, and then $f(A_0\cap A_1)=f(\emptyset)=\emptyset$, 
even though $f(A_0)$ and $f(A_1)$ may intersect. For example, if $f:{\Bbb R}\ra {\Bbb R}$
is given by $f(x)=12$, then for $A_0=[0,1]$ and $A_1=[2,3]$, we have
$f(A_0)=\{12\}=f(A_1)$, but $A_0\cap A_1=\emptyset$.

\ssk

\item{6.} [Munkres, p.51, \#5 (part)] For each of the following sets, determine
whether or not it is countable:

\ssk

\item{(*)} (c) $F$ = $\{f:{\Bbb Z}_+\ra {\Bbb Z}_+\ :\ \text{there}\ \text{is}\ N\in{\Bbb Z}_+\ \text{with}
f(n)=1\ \text{for}\ \text{all}\ n\geq N\}$, all eventually-1 functions.

\ssk

$F$ is the union of the sets $F_N=\{f:{\Bbb Z}_+\ra {\Bbb Z}_+\ :\ f(n)=1\ \text{for}\ \text{all}\ n\geq N\}$,
for $N\in {\Bbb Z}_+$; that is, $F$ is the union of countably many sets $\{F_N\}$. If we show that each of the sets
$F_N$ is countable, then as a countable union, $F$ will be countable.

\ssk

But each of the sets $F_N$ \ubr{is} countable: this can be established either by building a surjective
function from ${\Bbb N}$ to $F_N$ or by building an injective function from $F_N$ to ${\Bbb N}$.
In the first case, it is quicker to build a function from an $N$-fold cartesian product of ${\Bbb N}$'s
to $F_N$, as $g(n_1,\ldots,n_N)=f$, where $f(k)=n_k$ if $k\leq N$ and $f(k)=1$ for $k>N$. But then we can use
the fact that such a cartesian product is countable to build a surjective function
$h:{\Bbb N}\ra {\Bbb N}\times\cdots\times {\Bbb N}$; the composition $g\circ h:{\Bbb N}\ra F_N$ is
then surjective.

\ssk

For an injective function $g:F_N\ra {\Bbb N}$, we can steal from number theory (again): letting
$p_1,\ldots,p_N$ be a set of distinct prime numbers greater than $1$, we can define
$g(f)=p_1^{f(1)}\cdots p_N^{f(N)}$. Since prime factorizations of integers are unique,
if $f_1,f_2\in F_N$ have $f_1\neq f_2$, then $f_1(n)\neq f_2(n)$ for some $n$, and therefore
$n\leq N$ (since $f_1(n)=1=f_2(n)$ for $n>N$). Therefore $g(f_1)$ and $g(f_2)$ are integers
with different prime factorizations, and so $g(f_1)\neq g(f_2)$.

\msk

\vfill\eject

{\bf A selection of further solutions}

\msk

\item{1.} [Munkres, p.14, \#2 (part)] For each statement below,
determine whether or not it is true. If true, show why; if 
not, give an example demonstrating this.

\ssk

\item{} (d) For any sets $A,B,C,D$, \{$A\subseteq C$ and $B\subseteq D$\} implies that 
$A\times C \subseteq B\times D$ .

\msk

We know that $A\subseteq C$ and $B\subseteq D$. Suppose that $(x,y)\in A\times B$, 
then $x\in A$ and $y\in B$. But since $A\subseteq C$ this means that $x\in C$, and since
$B\subseteq D$ we have $y\in D$. So $(x,y)\in C\times D$. Consequently, 
$(x,y)\in A\times B$ implies that $(x,y)\in C\times D$, so $A\times B \subseteq C\times D$ .

\ssk

\item{} (e) For any sets $A,B,C,D$, $A\times C \subseteq B\times D$ implies that 
\{$A\subseteq C$ and $B\subseteq D$\} .

\msk

This is not true! Although it is almost true. 

\ssk

Suppose that $A\times B \subseteq C\times D$, and suppose we pick
$x\in A$ and $y\in B$. Then $(x,y)\in A\times B$, and so since
$A\times B \subseteq C\times D$, we have $(x,y)\in C\times D$, 
so $x\in C$ and $y\in D$. This \underbar{appears} to show that
$A\subseteq C$ and $B\subseteq D$, but it doesn't! 

\ssk

There is a subtle difference between ``\{$x\in A$ and $y\in B$\}
implies \{$x\in C$ and $y\in D$\}'' and 
``\{$x\in A$ implies $x\in C$\} and \{$y\in B$ implies $y\in D$\}''.
The first is what we've shown, the second is what we want. The difference
is that if, for example, $A$ is empty ($A = \emptyset$), then the first
statement is \underbar{always} true; you can't pick points in $A$ \underbar{and}
$B$, or to put it differently, $\emptyset\times B = \emptyset\subseteq C\times D$,
no matter what $B$ is. So, for example, if $A=\emptyset$, $C=B=\{0,1\}$ and $D=\{0\}$,
then $A\times B = \emptyset \subseteq C\times D$, but $B = \{0,1\}\not\subseteq \{0\} = D$.

\ssk

Put still differently, \underbar{if} $A\neq\emptyset$, we can pick an
$x\in A$. Then for any $y\in B$ we have $(x,y)\in A\times B$, so 
$(x,y)\in C\times D$, so $y\in D$, and so $B\subseteq D$. Similarly,
if $B\neq\emptyset$, then $A\times B \subseteq C\times D$ implies that
$A\subseteq C$. But without knowing that $A$ and $B$ are nonempty, we
cannot establish our desired conclusion.

\ssk

\item{3.} [Munkres, p.20, \#1] Show that if $f:A\ra B$ is a function, then

\ssk

\item{} (a) If $A_0\subseteq A$, then $A_0\subseteq f^{-1}(f(A_0))$; the sets are equal,
if $f$ is injective.

\ssk

If $a\in A_0$, then $f(a)\in f(A_0)$< and so $a$ is in the set $\{x\in A\ :\ f(x)\in f(A_0)\}=f^{-1}(f(A_0)$.
So $A_0\subseteq f^{-1}(f(A_0)$.

\msk


\item{5.} [Munkres, p.44, \#7] If $A$ and $B$ are finite sets, show that the
set $B^A = \{f:A\ra B\}$ of all functions from $A$ to $B$ is also finite.

\msk


The notation $B^A$ almost gives away the idea; the set has $|B|^{|A|}$ elements. 
One way to see this is to build a bijective correspondence with an $|A|$-fold 
Cartesian product of copies of $B$; induction on $|A|$ together with the
finitenes of $B$ shows that this Cartesian product is finite, so $B^A$ is finite.

We can build an injective/surjective map to/from $\{1,\ldots,|B|^{|A|}\}$ by,
essentially, writing integers in base $b=|B|$. For example, given an
injective map $\varphi:B\hookrightarrow\{1,\ldots,b\}\hookrightarrow\{0,\ldots,b-1\}$
(the second map is ``subtract one'') and a bijection $\theta:\{1,\ldots,|A|\}\ra A$,, then the map

\ssk

\ctln{$\Phi:B^A\rightarrow \{1,\ldots |B|^{|A|}\}$}

\ssk

\nidt given by $\displaystyle \Phi(f) = \Sigma_{i=1}^{|A|} \varphi(f(\theta(i)))b^{i-1}$ is an injection; 
any two such sums (corresponding to functions $f$ and $g$)
representing the same number, since $0\leq\varphi(f(\theta(i)))\leq b-1$ for each $i$,
must have $\varphi(f(\theta(i)))=\varphi(g(\theta(i)))$, so $f(\theta(i))=g(\theta(i))$ for each $i$.
Since $\theta$ is surjective, this means that $f=g$. A surjective map
$\{1,\ldots,|B|^{|A|}\}\twoheadrightarrow B^A$ can similarly be built using the $i$-th `digit' of the
representation of a number in base $b$ to determine the image of the $i$-th element of $A$.

\ssk

\item{6.} [Munkres, p.51, \#5 (part)] For each of the following sets, determine
whether or not it is countable:

\ssk

\item{} (a) $A$ = $\{f:\{0,1\}\ra {\Bbb Z}\}$, all functions from ${0,1}$ to $\Bbb Z$

\ssk

Since a function is determined its values - its graph is all pairs $(a,f(a)$ - 
the function is completely determine by the set of pairs $\{(0,f(0)),(1,f(1))\}$, which in
turn can be recovered fromt he ordered pair $(f(0),f(1))$. This means, really, that we can
build a surjective map ${\Bbb Z}\times{\Bbb Z}\twoheadrightarrow A$ by send the 
pair $(a,b)$ to the function $f$ with $f(0)=1$ and $f(1)=b$. But ${\Bbb Z}\times{\Bbb Z}$
is countable; the map ${\Bbb Z}_+\times\{-1,0,1\}\ra {\Bbb Z}$ give by $(n,s)\mapsto sn$
is onto. But ${\Bbb Z}_+$ and $\{-1,0,1\}$ are countable, so ${\Bbb Z}_+\times\{-1,0,1\}$
is countable, so there is a surjection ${\Bbb Z}_+\twoheadrightarrow{\Bbb Z}_+\times\{-1,0,1\}$
which via composition gives a surjection ${\Bbb Z}_+\twoheadrightarrow{\Bbb Z}$, so 
${\Bbb Z}$ is countable. Then ${\Bbb Z}\times{\Bbb Z}$ is countable, meaning there is a
surjection from ${\Bbb Z}_+$ to ${\Bbb Z}\times{\Bbb Z}$; composing with the surjection
above gives a surjection from ${\Bbb Z}_+$ to $A$, so $A$ is countable.

\msk

\item{} (e) $P$ = $\{f:{\Bbb Z}_+\ra {\Bbb Z}_+\ :\ n>m\ \text{implies}
f(n)>f(m)\}$, all increasing functions.

\ssk

This set is \underbar{not} countable; we can show this, since we know, for example, 
that the set $2^{{\Bbb Z}_+} = \{f:{\Bbb Z}_+\ra\{0,1\}$ is not countable, by building
an injective map $2^{{\Bbb Z}_+}\hookrightarrow P$. [If $P$ were countable,
composing this
injection with an injection from $P$ to ${\Bbb Z}_+$ would show that $2^{{\Bbb Z}_+}$
is countable, a contradiction.]

\ssk

We can build the desired injection in many ways; perhaps the shortest is to define
$\Phi(f)=g$ where $g(n)=10^n+f(n)$. This is an injective map; 
if $\Phi(f_1)=\Phi(f_2)$, then $10^n+f_1(n)=10^n+f_2(n)$ for all $n$, so 
$f_1(n)=f_2(n)$ for all $n$ and $f_1=f_2$. $\Phi(f)=g$ is an increasing function, 
since $m>n$ implies that $g(m)=10^m+f(m)=10^{m-n}10^n+f(m) > 10\cdot 10^n+f(m) > 2\cdot 10^n+f(m)
=10^n+10^n+f(m) \geq 10+10^n+f(m) > 10^n+f(n) = g(n)$, where the inequality
towards the end follows from $10+f(m) >f(n)$, since $f(n)-f(m)\leq 1$.

[From the proof, it would seem that $g(n)=2^n+f(n)$ would actually suffice...]

\msk

\item{} (f) The set $F$ of all finite subsets of ${\Bbb N}$.

\msk

This \underbar{is} countable; it is the union, over all integers $n\geq 0$, of the
set $F_n$ of all subsets of ${\Bbb N}$ of size \ubr{at} \ubr{most} $n$. Treat size 0 differently,
for $n\geq 1$ the function $f_n:{\Bbb N}^n\ra F_n$ which sends $(k_1,\ldots k_n)$ to $\{k_1,\ldots,k_n\}$
is surjective, so each $F_n$ is countable. So their (countable) union is countable, as well.

\vfill
\end





\item{} (b) For any sets $A,B,C$, \{$A\subseteq B$ and $A\subseteq C$\} 
if and only if $A\subseteq B\cap C$ .

\ssk

\item{} (c) For any sets $A,B,C$, $A\cap(B\setminus C) = (A\cap B)\setminus(A\cap C)$ .

\ssk


\item{2.} If $A,C\subseteq X$ and $B,D\subseteq Y$, show how to express
$(A\times B)\setminus (C\times D)$ as a union of Cartesian products 
(i.e., sets of the form $U\times V$).

\msk


\item{} (b) If $B_0\subseteq B$, then $f(f^{-1}(B_0))\subseteq B_0$; the sets are equal,
if $f$ is surjective.

\msk

\item{4.} [Munkres, p.20, \#2 (part)] If $f:A\ra B$ is a function, $A_0,A_1\subseteq A$,
and $B_0,B_1\subseteq B$, then

\ssk

\item{} (b) $f^{-1}(B_0\cup B_1)$ = $f^{-1}(B_0) \cup f^{-1}(B_1)$

\ssk

\item{} (d) $f(A_0\cup A_1)$ = $f(A_0)\cup f(A_1)$ .

\msk


\item{} (b) $B$ = $\{f:{\Bbb Z}_+\ra {\Bbb Z}_+\ :\ f(n)=0\ \text{for}\ \text{all}\ n\geq 3\}$

\ssk

\item{} (d) $H$ = $\{f:{\Bbb Z}_+\ra {\Bbb Z}_+\ :\ \text{there}\ \text{is}\ N\in{\Bbb Z}_+\ \text{with}
f(n)=f(N)\ \text{for}\ \text{all}\ n\geq N\}$, all eventually constant functions.

\ssk


\item{7.} Let $P(A)=\{B\ :\ B\subseteq A\}$ (the {\it power set} of $A$), and 
suppose $f:A\ra P(A)$ is a function. Show that the set $B = \{ a\in A\ :\ a\not\in f(a)\}$
is not in the image of $f$; conclude that $f$ can never be surjective. What does this say
about functions $g:P(A)\ra A$ ?
