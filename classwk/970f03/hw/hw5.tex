\input amstex

\magnification=1200

\def\ctln{\centerline}
\def\msk{\medskip}
\def\bsk{\bigskip}
\def\ssk{\smallskip}
\def\u{\underbar}

\UseAMSsymbols

\ctln{\bf Math 970 Homework \# 5}

\msk

\ctln{{\bf Due: } Oct. 9}

\bsk


\item{22.} Show that if $X$ is a space with topology generated by a basis ${\Cal B}$, 
then $X$ is Hausdorff if and only if for every $x,y\in X$ with $x\neq y$, there are
$B,B^\prime\in {\Cal B}$ with $x\in B$, $y\in B^\prime$ and $B\cap B^\prime = \emptyset$.

\bsk

\item{23.} Show that if ${\Cal T}$ is the usual topology on ${\Bbb R}$, the space $X$ = 
${\Bbb R}\cap \{*\}$ , with topology generated by the basis ${\Bbb B} = 
{\Cal T} \cup \{(U\setminus 0)\cup\{*\} : U\in {\Cal T}$ and $0\in U\}$ is not Hausdorff, 
but every one-point subset of $X$ is closed. [FYI: $X$ is called the {\it line with two origins}.]

\bsk

\item{24.} Show that the line with two origins is the quotient of two disjoint
copies of ${\Bbb R}$ (think: ${\Bbb R}\times \{0,1\}$). Conclude that the quotient of a 
Hausdorff space need not be Hausdorff.

\bsk

\item{25.} Show that the quotient space obtained by the equivalence relation $\sim$
on $[0,1]\times [0,1]$ generated by (i.e., add $a\sim a$, and $a\sim b$ whenever $b\sim a$, and
any relation that transitivity would {\it force} on you)

\ssk

\ctln{$(0,y)\sim (1,y)$ for all $y\in [0,1]$ and $(x,0)\sim (x,1)$ for all $x\in [0,1]$}

\ssk

\item{} admits a continuous bijection to $S^1\times S^1$ .

\bsk

\item{26.} Find an example of subspaces $A,B\subseteq {\Bbb R}$ (giving ${\Bbb R}$ the usual 
topology) for which there is a continuous bijection

\ssk

\ctln{$f : A \rightarrow B$}

\ssk

\item{} whose inverse is {\bf not} continuous.


\vfill
\end