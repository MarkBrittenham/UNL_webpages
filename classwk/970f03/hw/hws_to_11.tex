\input amstex

\magnification=1200

\def\u{\underbar}
\def\ctln{\centerline}
\def\ni{\noindent}
\def\ssk{\smallskip}
\def\msk{\medskip}
\def\bsk{\bigskip}
\def\ra{\rightarrow}
\def\itt{\item}
\def\pr{\hskip-10pt}
\def\hsk{\hskip10pt}
\def\hhsk{\hskip20pt}
\def\lra{$\Leftrightarrow\ $ }
\def\exs{$\exists\ $}
\def\foa{$\forall\ $}
\def\cau{{\Cal U} }
\def\cav{{\Cal V}}
\def\sset{\subseteq}
\def\finv{f^{-1}}
\def\catm{${\Cal T}\ $}
\def\cat{{\Cal T}}
\def\catpm{$\cat^\prime\ $}
\def\catp{\cat^\prime}
\def\top{topology}
\def\tops{topologies}
\def\cts{continuous}
\def\smin{\setminus}
\def\cab{{\Cal B}}
\def\cabm{${\Cal B}\ $}
\def\cas{{\Cal S}}
\def\casm{${\Cal S}\ $}
\def\bbbr{{\Bbb R}}
\def\bbbn{{\Bbb N}}
\def\bbbq{{\Bbb Q}}

\UseAMSsymbols

\ctln{\bf Math 970 Homework and Midterm problems}

\msk

\item{1.} Show that if $f$:$X\rightarrow Y$ is a function, then the inverse
image of subsets of $Y$ satisfies:

\ssk

\item{(a)} $\displaystyle f^{-1}(\bigcup_{i\in I} {\Cal U}_i) =\bigcup_{i\in I} f^{-1}({\Cal U}_i)$

\ssk

\item{(b)} $\displaystyle f^{-1}(\bigcap_{j\in J} {\Cal V}_j) =\bigcap_{j\in J} f^{-1}({\Cal V}_j)$

\ssk

\item{(c)} $\displaystyle f^{-1}(Y\setminus B) = X\setminus f^{-1}(B)$

\ssk

\item{2.} With notation as in problem \# 1, show, by contrast, that some 
of the corresponding results for the {\it image} of subsets of $X$ do {\it not}
hold in general. Under what conditions of the function $f$ would each 
property that fails actually hold true?

\ssk

\item{3.} Show that if $f$:$(X,d)\rightarrow (Y,d^\prime)$ is a function between
metric spaces which satisfies, for some $K\in{\Bbb R}$, 
$d^\prime(f(x),f(y))\leq K\cdot d(x,y)$ for all $x,y\in X$, then $f$ is continuous. In particular, if
$f$ decreases distances, then $f$ is continuous.

\ssk

\item{4.} Show that the metrics $d_1$ and $d_2$ on ${\Bbb R}^n$ satisfy

\ssk

\ctln{$d_2(\vec{x},\vec{y})\leq d_1(\vec{x},\vec{y})\leq 
n\cdot $max$\{|x_1-y_1|,\ldots |x_n-y_n|\} \leq
n\cdot d_2(\vec{x},\vec{y})$}

\ssk

\item{} Conclude that $d_1$ and $d_2$ give the same open sets for ${\Bbb R}^n$ .

\ssk


\item{5.} Show that if $(X,d)$ is a metric space, then $(X,\bar{d})$, where

\ssk

\ctln{$\bar{d}(x,y) = \min\{d(x,y),1\}$}

\ssk

\item{} is also a metric space, with the {\it same} open sets as $(X,d)$ .

\ssk

\item{6.} If $(X,{\Cal T})$ is a topological space, $Y$ is a set, and $f:X\rightarrow Y$ is a function, show that


\ssk

\ctln{${\Cal T}^\prime$ = $\bigl\{{\Cal U}^\prime\subseteq Y : f^{-1}({\Cal U}^\prime)\in {\Cal T}\bigr\}$}

\ssk

is the finest topology on $Y$ for which $f:(X,{\Cal T})\rightarrow (Y,{\Cal T}^\prime)$ is continuous.

\ssk

(Note that this problem is actually asking you to show \u{three} things...)

\ssk

\item{7.} If $(X,{\Cal T})$ is a topological space, and $A\subseteq X$,  then $A\in {\Cal T}$ if and only if

\ssk

\ctln{for all $x\in A$, there is a $U\in {\Cal T}$ so that $x\in U\subseteq A$}

\ssk

\item{8.} Show that ${\Cal B} = \{(a,\infty)\times(b,\infty) : a,b\in{\Bbb R}\}$ is a basis for a topology ${\Cal T}$
on ${\Bbb R}^2={\Bbb R}\times {\Bbb R}$, which is coarser than the usual Euclidean topology on ${\Bbb R}^2$.
Show that ${\Cal B}^\prime = \{[a,\infty)\times[b,\infty) : a,b\in{\Bbb R}\}$ is a basis for a topology ${\Cal T}^\prime$ which is 
strictly finer than ${\Cal T}$, and not comparable to the usual Euclidean topology.

\ssk

\item{9.} Show that, in general, if ${\Cal B}$ and ${\Cal B}^\prime$ are both bases for topologies 
on $X$, that
${\Cal B}\cap{\Cal B}^\prime$ and ${\Cal B}\cup{\Cal B}^\prime$ \u{need} \u{not} \u{be}. Show,
however, that ${\Cal B}^{\prime\prime} = \{B\cap B^\prime : B\in {\Cal B}, B^\prime \in{\Cal B}^\prime \}$
\u{is} a basis for a topology, and ${\Cal T}({\Cal B}^{\prime\prime})$ is the coarest topology containing both 
${\Cal B}$ and ${\Cal B}^\prime$.

\ssk

\item{10.} Show that the topology generated by a basis $\Cal B$ is the coarsest topology \underbar{containing}
$\Cal B$ (i.e., it is the intersection of all such topologies).

\ssk

\item{11.} Let $(X,{\Cal T})$ be a topological space, $B\subseteq X$ a subset, and 
${\Cal T}_B$ the subspace topology on $B$. If $A\subseteq B$, show that the subspace
topology that it inherits from $B$ is the \u{same} as the subspace topology that it
inherits from $X$.

\ssk

\item{12.} Show that if $A\subseteq X$ and $(X,{\Cal T})$ is Hausdorff, then the
subspace topology on $A$ is Hausdorff.

\ssk

\item{13.} Show that if $(X,d)$ and $(Y,d^\prime)$ are metric spaces, then the product
topology on $X\times Y$ is metrizable. [There are lots of (correct) choices of metric on 
$X\times Y$; you can take your cue from ${\Bbb R}^2$ .]

\ssk

\item{14.} Show that if $(X,{\Cal T}), (Y,{\Cal T}^\prime)$ are topological spaces and $x_0\in X$, then
the function 

\ssk

\ctln{${\iota}_{x_0} : Y\rightarrow X\times Y$ , ${\iota}_{x_0}(y) = (x_0,y)$}

\ssk

\item{} is continuous.

\ssk

\item{15.} Show that if $(X,d)$ is a metric space, then the metric $d : X\times X\rightarrow 
{\Bbb R}$ is continuous (where $X\times X$ has the product topology). Show, further, that
the metric topology ${\Cal T}$ is the \u{coarsest} topology on $X$ for which $d$ is continuous.

\ssk

\item{} (Hint: show that if ${\Cal T}^\prime\subsetneq {\Cal T}$, then $N_d(x_0,\epsilon)\notin{\Cal T}^\prime$
for some $x_0$ and $\epsilon$ ; now look at problem \# 14.)

\ssk

\item{16.} Show that, \u{if} $X$ is an infinite set, then the finite complement topology ${\Cal T}_f$ on 
$X\times X$ is \u{not} a product topology, i.e., there do not exist topologies 
${\Cal T},{\Cal T}^\prime$ on $X$ whose product topology is ${\Cal T}_f$. On the other hand, 
if $X$ is finite, show that ${\Cal T}_f$ on $X\times X$ \u{is} a product topology.

\ssk

\item{} (Hint: the \u{basis} for the product topology would have to be $\subseteq {\Cal T}_f$ ...)

\ssk
\item{17.} For $A,B\subseteq X$ with $(X,{\Cal T})$ a topological space, 
if $A$ is open in $X$ and $B$ is closed in $X$, 
then $A\setminus B$ is open and $B\setminus A$ is closed.

\ssk

\item{18.} Show that if $A,B\subseteq X$, then

\msk

(a) $\overline{A\cup B} = \overline{A}\cup\overline{B}$

\ssk

(b) $\overline{A\cap B} \subseteq \overline{A}\cap\overline{B}$, but that equality does not hold in general,

\ssk

(c) $\overline{A\setminus B} \supseteq \overline{A}\setminus\overline{B}$, but that equality does not hold in general.

\ssk

\item{19.} Show that if $A\subseteq X$ and $X$ has two topologies ${\Cal T}\subseteq {\Cal T}^\prime$, then 
if $x\in X$ is a limit point of $A$ w.r.t. ${\Cal T}^\prime$, then it is a limit point of $A$ w.r.t. ${\Cal T}$.

\ssk

\item{20.}Show that if $A_i\subseteq X_i$ for all $i\in I$, then

\msk

\ctln{$\displaystyle \overline{\prod_i A_i}$= $\displaystyle \prod_i \overline{A_i}$ $\subseteq$ $\displaystyle \prod_i X_i$}

for \underbar{both} the product and box topologies.



\ssk

\item{21.}Find the closure of the set $(0,1)\subseteq{\Bbb R}$, when ${\Bbb R}$
has the 

\msk

(a) finite complement topology

\ssk

(b) infinite (open) ray to the right topology

\ssk

(c) discrete topology

\ssk

(d) {\it lower limit topology}, generated by the basis ${\Cal B} = \{[a,b) : a,b\in {\Bbb R}\}$

\ssk
\item{22.} Show that if $X$ is a space with topology generated by a basis ${\Cal B}$, 
then $X$ is Hausdorff if and only if for every $x,y\in X$ with $x\neq y$, there are
$B,B^\prime\in {\Cal B}$ with $x\in B$, $y\in B^\prime$ and $B\cap B^\prime = \emptyset$.

\ssk

\item{23.} Show that if ${\Cal T}$ is the usual topology on ${\Bbb R}$, the space $X$ = 
${\Bbb R}\cap \{*\}$ , with topology generated by the basis ${\Bbb B} = 
{\Cal T} \cup \{(U\setminus 0)\cup\{*\} : U\in {\Cal T}$ and $0\in U\}$ is not Hausdorff, 
but every one-point subset of $X$ is closed. [FYI: $X$ is called the {\it line with two origins}.]

\ssk

\item{24.} Show that the line with two origins is the quotient of two disjoint
copies of ${\Bbb R}$ (think: ${\Bbb R}\times \{0,1\}$). Conclude that the quotient of a 
Hausdorff space need not be Hausdorff.

\ssk

\item{25.} Show that the quotient space obtained by the equivalence relation $\sim$
on $[0,1]\times [0,1]$ generated by (i.e., add $a\sim a$, and $a\sim b$ whenever $b\sim a$, and
any relation that transitivity would {\it force} on you)

\ssk

\ctln{$(0,y)\sim (1,y)$ for all $y\in [0,1]$ and $(x,0)\sim (x,1)$ for all $x\in [0,1]$}

\ssk

\item{} admits a continuous bijection to $S^1\times S^1$ .

\ssk

\item{26.} Find an example of subspaces $A,B\subseteq {\Bbb R}$ (giving ${\Bbb R}$ the usual 
topology) for which there is a continuous bijection

\ssk

\ctln{$f : A \rightarrow B$}

\ssk

\item{} whose inverse is {\bf not} continuous.

\ssk
\item{27.} Show that if ${\Cal T}\subseteq {\Cal T}^\prime$ are topologies on
$X$ and $(X,{\Cal T}^\prime)$ is connected, then so is $(X,{\Cal T})$.

\ssk

\item{28.} Find an example of a space $X$ and subset $A\subseteq X$ where 
int$(A)$ and cl$(A)$ are both connected, but $A$ is not.

\ssk

\item{29.} Show by example that for $f : X \rightarrow Y$ continuous and $A\subseteq Y$, 
having one of  $f^{-1}(A)$ and $A$  connected does \underbar{not} necessarily imply that the 
other is connected.

\ssk
\item{30.} Show that if $X_\alpha, \alpha\in I$ are all path-connected, then so is
$\displaystyle \prod_{\alpha\in I} X_\alpha$, if we use the product topology.

\ssk

\item{31.} Show that if $A_\alpha\subseteq X, \alpha\in I$ are all path-connected, and
$\displaystyle\bigcap_{\alpha\in I}A_\alpha\neq\emptyset$, then 
$\displaystyle\bigcup_{\alpha\in I}A_\alpha$ is path-connected.

\ssk

\item{32.} Show that if $C\subseteq{\Bbb R}^3$ is countable, then ${\Bbb R}^3\setminus C$
is path-connected. (Hint: a plane in ${\Bbb R}^3$ will hit $C$ in how many points?)

\ssk

\item{33.} Show that if $(X,{\Cal T})$ is compact, and ${\Cal T}^\prime\sset{\Cal T}$, then
$(X,{\Cal T}^\prime)$ is compact.

\ssk

\item{34.} Show that if $(X,{\Cal T})$ is a topological space and $A,B$ are compact subsets
of $X$, then $A\cup B$ is compact.

\ssk

\item{35.} Give an example of a space $(X,{\Cal T})$ and subsets $A,B\sset X$ so that
$A$ and $B$ are compact but $A\cap B$ is not.

\ssk

\item{} (Note: your space $X$ cannot be Hausdorff....)

\ssk

\item{36.} Let $X$ = $\Bbb R$ with the infinite ray topology 

\ssk

\ctln{$\Cal T$ = $\{(a,\infty) : a\in {\Bbb R}\}\cup \{\emptyset,{\Bbb R}\}$}

\ssk

Show that $A$ = $\{0\}$ is a compact subset of $X$, but its closure $\bar{A}$ \u{isn't}.

\ssk

\item{37.} Show that if $(X,\cat)$ is a Hausdorff space and $A,B\sset X$ are disjoint compact
subsets of $X$, then there are subsets $\cau,\cav\in \cat$ so that $A\sset \cau$, $B\sset \cav$,
and $\cau\cap\cav =\emptyset$ .

\ssk

\item{38.} Show that if $X$ is limit point compact, and $A$ is a closed subset of $X$, then
$A$ is limit point compact.

\ssk

\item{39.} Give an example of a limit point compact space $X$ and a continuous function
$f: X\rightarrow Y$ for which $f(X)\subseteq Y$ is not limit point compact.

\ssk

\ni Note: for the purposes of the following problems, `` regular'' means 
points and (disjoint) closed sets can be separated with open sets, ``normal'' means 
disjoint closed sets can be separated, $T_3$ means $T_1$ and regular, and
$T_4$ means $T_1$ and normal.

\ssk

\item{40.} Show by examples that the continuous image of $T_3$ need not
be $T_3$, and that the continuous image of a non-$T_3$ space can be $T_3$!

\ssk

\item{41.} Show that every closed subset of a normal space is normal, and that every 
closed subset of a $T_4$ space is $T_4$.

\ssk

\item{42.} Show that for any collection $X_\alpha\neq \emptyset$ of topological spaces, if 
$\displaystyle\prod_\alpha X_\alpha$ is $T_4$ in the product topology, then $X_\alpha$ is 
$T_4$ for all $\alpha$.

\ssk

\item{} (Hint: embed $X_\alpha$ in $\displaystyle\prod_\alpha X_\alpha$ as a closed subset!)

\ssk

\item{43.} Show that a compact metric space $(X,d)$ is second countable.

\ssk

\item{} (Hint: look at $\{ N_d(x,1/n) : x\in X\}$ for each $n$ .)

\ssk

\item{44.} Show that a closed subset of a Lindel\"of space is Lindel\"of.

\ssk

\item{45.} Show by example that a closed subset of a separable space 
need not be separable.

\ssk

\item{46.} Show  that the continuous image of a separable space 
is separable, and the continuous image of a Lindel\"of space is Lindel\"of..

\ssk

\item{47.} Show that if $(X,{\Cal T}({\Cal C}))$ is second countable 
(with ${\Cal C} = \{ C_n\}_{n=1}^\infty$ 
countable), then \underbar{every}
basis ${\Cal B}$ for ${\Cal T} = {\Cal T}({\Cal C})$
contains a countable basis ${\Cal B}^\prime\subseteq{\Cal B}$.

\ssk

\item{} (Hint: look at all $B\in{\Cal B}$ with $C_m\subseteq B\subseteq C_n$ for some $m,n$ ;  then pick (at most) one for each pair...)

\ssk

\item{48.} Show that if $X$ is Hausdorff and 
$f$:$X\rightarrow X$ is continuous, then the fixed 
point set 

\ssk

\ctln{Fix$(f)$ = $\{ x\in X : f(x) = x\}$}

\ssk

\item{} of $f$ is a closed subset of $X$.

\ssk

\item{} A subset $A\subseteq X$ is a \underbar{retract} of $X$ if there is a continuous
map $r$:$X\rightarrow A$ with $r\circ i = Id$, i.e., $r(a)=a$ for all $a\in A$. The map
$r$ is called a \underbar{retraction}.

\ssk

\item{49.} Show that if $X$ is Hausdorff and $A\sset X$ is a retract of $X$, then $A$ is
closed. 

\ssk

\item{} (Hint: show that $A$ is the fixed point set of some map!)

\ssk

\item{50.} Show that if $r: X\rightarrow A$ is a retraction and $a\in A$, then

\ssk

\ctln{$r_*$:$\pi_1(X,a)\rightarrow \pi_1(A,a)$}

\ssk

\item{} is a surjective homomorphism.

\ssk

\item{51.} Show that if $a\in A\subseteq X$, $\pi_1(X,a) = \{1\}$, and 
$f$:\hsk $(A,a)\rightarrow (Y,b)$ is continuous, then if $f$ extends to a continuous 
map $g$:$X\rightarrow Y$ (i.e., $g|_A = f$), then 
$f_*$:$\pi_1(A,a)\rightarrow \pi_1(Y,b)$ is the trivial homomorphism.

\ssk

\item{} (The contrapositive of the last part of this statement sounds stronger....)

\ssk

\item{52.} A space $X$ is \underbar{contractible} if the identity map $I$:$X\rightarrow X$
is homotopic to the constant map $c(x)=x_0$. Show that if $X$ is contractible then
any two maps $f,g$\hsk :\hsk $Y\rightarrow X$ are homotopic. Show that this implies that 
$\pi_1(X,x_0) = \{1\}$.

\ssk

\item{M1.} For $X$ any set, and $a,b\in X$, show that the collection

\ssk

\ctln{\catm = $\{A\sset X : a\in A$ or $b\notin A\}$}

\ssk

\item{} forms a topology on $X$.

\ssk

\item{M2.} Show that if $(X,\cat)$ and $(Y,\cat^\prime)$ are both Hausdorff, 
then $X\times Y$, with the product topology, is Hausdorff.

\ssk

\item{M3.}  Show that if $\cat\sset\cat^\prime$ are topologies on $X$, and $A\sset X$, then
cl$_{\cat^\prime}(A)\sset$cl$_\cat(A)$ . Show, further, that if $\cat\neq\cat^\prime$ then there
is an $A\sset X$ with cl$_{\cat^\prime}(A)\neq$cl$_\cat(A)$ .

\ssk

\item{M4.} For $X=\bbbr$, let $\cat_1$ be the excluded point topology, excluding $0$, and let
$\cat_2$ be the included point topology, including $1$. Show that if a continuous function 

\ctln{$f:(X,\cat_1)\ra (X,\cat_2)$}

\item{} has $f(0)=1$, then $f$ is constant. Show, more generally, that any continuous function $f:(X,\cat_1)\ra (X,\cat_2)$ has image consisting of at most 2 points.

\ssk
\item{M5.} Show that if $\cat\sset\cat^\prime$ are topologies on the set $X$ and
$(X,\cat^\prime)$ is path-connected, then $(X,\cat)$ is path-connected.


\vfill
\end