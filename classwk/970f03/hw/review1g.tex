\input amstex
\UseAMSsymbols

%\hoffset=10pt

\def\ctln{\centerline}
\def\ni{\noindent}
\def\ssk{\smallskip}
\def\msk{\medskip}
\def\bsk{\bigskip}
\def\ra{\rightarrow}
\def\itt{\item}
\def\pr{\hskip-10pt}
\def\hsk{\hskip10pt}
\def\hhsk{\hskip20pt}
\def\lra{$\Leftrightarrow\ $ }
\def\exs{$\exists\ $}
\def\foa{$\forall\ $}
\def\cau{{\Cal U} }
\def\cav{{\Cal V}}
\def\sset{\subseteq}
\def\finv{f^{-1}}
\def\catm{${\Cal T}\ $}
\def\cat{{\Cal T}}
\def\catpm{$\cat^\prime\ $}
\def\catp{\cat^\prime}
\def\top{topology}
\def\tops{topologies}
\def\cts{continuous}
\def\smin{\setminus}
\def\cab{{\Cal B}}
\def\cabm{${\Cal B}\ $}
\def\cas{{\Cal S}}
\def\casm{${\Cal S}\ $}
\def\bbbr{{\Bbb R}}
\def\bbbn{{\Bbb N}}
\def\bbbq{{\Bbb Q}}

\magnification=1200

\ctln{\bf Math 970 mid-semester review}

\bsk

\pr {\bf Set-theoretic beginnings:}

Functions: $f:X\ra Y$ .  injection, surjection, bijection; image, 

inverse image $f^{-1}(A) = \{x\in X : f(x)\in A\}$

\pr Image: $f(\bigcup A_\alpha) = \bigcup f(A_\alpha)$ , 
$f(\bigcap A_\alpha) \subseteq \bigcap f(A_\alpha)$

\pr Inverse image: $f^{-1}(\cup A_\alpha) = \cup f(A_\alpha)$ , $f^{-1}(\cap A_\alpha) = \cap f^{-1}(A_\alpha)$, 
$f^{-1}(Y\setminus A) = X\setminus f^{-1}(A)$

Finite sets, infinite sets, countable sets

\hsk $A$ is finite \lra \exs a surjection $\{1,\ldots , n\}\rightarrow A$ \lra \exs an injection $A \rightarrow \{1,\ldots , n\}$ .

\hsk $A$ is countable \lra \exs a surjection ${\Bbb N}\rightarrow A$ \lra \exs an injection $A \rightarrow {\Bbb N}$ .

\hsk countable union of countables is countable, product of two countables is countable.

Cardinality: $|A|=|B|$ if $\exists$ a bijection $f:A\ra B$

Shroeder-Bernstein Thm: if \exs injection $A\ra B$ and \exs injection $B\ra A$, then $|A|=|B|$

\msk

\pr {\bf Topologies}

Idea: extend continuity to more general settings. 

 Metric spaces: $(X,d)$ , $d: X\times X \ra {\Bbb R}$ satisfies 

\hsk $d(x,y)\geq 0$, $d(x,y)=0$ \lra $x=y$, $d(x,y)=d(y,x)$, and $d(x,z)\leq d(x,y)+d(y,z)$ .

$f:(X,d)\ra (Y,d^\prime)$ continuous ( = cts) if 

\hsk \foa $a\in X$ and \foa $\epsilon > 0$ \exs $\delta = \delta(a,\epsilon) > 0$
so that $d(a,x) < \delta \Rightarrow d^\prime(f(a),f(x)) > \epsilon$.

(Open) neighborhood: $N_d(x,\epsilon) = \{y\in X : d(x,y) < \epsilon\}$

Open set: $\cau\sset X$ is open if \foa $x\in\cau$ \exs $\epsilon > 0$ so that $N_d(x,\epsilon)\sset \cau$

\hsk $\cau\sset X$ is open \lra $\cau$ = a union of neighborhoods.

\hsk $f : X\ra Y$ is cts \lra $\finv\cau$ is open in $X$ \foa $\cau$ open in $Y$

The collection \catm of open sets in $(X,d)$ satisfies

\hsk $\emptyset, X\in \cat$ 

\hsk if $\cau,\cav\in \cat$, then $\cau\cap\cav\in \cat$

\hsk if $\cau_\alpha\in \cat$ \foa $\alpha\in I$, then $\bigcap\cau_\alpha\in \cat$

\msk

\ni For $X$ any set, a {\bf topology} on $X$ is any collection \catm of subsets of $X$ satisfying the above three conditions.

$f : (X,\cat)\ra (Y,\cat^\prime)$ is cts \lra $\finv(\cau)\in\cat$  for all  $\cau\in\cat^\prime$

comparing topologies: $\cat\sset\cat^\prime$, then \catm is {\it coarser} than \catpm ; \catpm is {\it finer} than \catm .

$\cat=\cat^\prime$ \lra $\cat\sset\cat^\prime$ and $\cat^\prime\sset\cat$

\ni Examples: 

$\cat_i$ = $\{\emptyset, X\}$ = trivial \top\ ( = indiscrete \top).

$\cat_d$ = ${\Cal P}(X)$ = all subsets of $X$ = discrete \top.

\catm = open sets for a metric $d$ on $X$ = metric topology on $X$.

\hsk $(X,\cat)$ is {\it metrizable} if $\cat$ is the metric topology for some metric on $X$.

\catm = $\{\cau\in X : X\smin\cau$ is finite$\}\cup\{\emptyset\}$ = finite complement \top.

\catm = $\{\cau\in X : X\smin\cau$ is countable$\}\cup\{\emptyset\}$ = countable complement \top.

 For $a\in X$, \catm = $\{\cau\sset X : a\in\cau\}\cup\{\emptyset\}$ = included point \top.

 For $a\in X$, \catm = $\{\cau\sset X : a\notin\cau\}\cup\{X \}$ = excluded point \top.

On ${\Bbb R}$, \catm = $\{(a,\infty) : a\in{\Bbb R}\}\cup\{\emptyset, {\Bbb R}\}$ = infinite open ray topology.

On ${\Bbb R}$, \catm = $\{(a,\infty) : a\in{\Bbb R}\}\cup\{[a,\infty) : a\in{\Bbb R}\}\cup\{\emptyset, {\Bbb R}\}$ = infinite ray topology.

$f:X\ra (Y,\cat^\prime)$ , then \catm = $\{\finv(\cau) : \cau\in\cat^\prime\}$ = coarsest top. on $X$ making $f$ cts.

$f:(X,\cat)\ra Y$ , then $\cat^\prime$ = $\{\cau : \finv(\cau)\in\cat\}$ = finest topology on $Y$ making $f$ cts.

\msk

\ni Metric \tops\ also satisfy: if $x,y\in X$, $x\neq y$, then \exs $\cau,\cav$ open with $x\in\cau, y\in\cav$ and $\cau\cap\cav=\emptyset$
\hsk A topological space satisfying this property is called {\it Hausdorff}.

A {\it topological property} is a property which can be described in terms of open sets and 
relations between them. (For example, Hausdorffness.) Topology is, essentially, the study 
of topological properties and the relationships between them.

\msk

\pr{\bf Bases and subbases}

Open sets for metric spaces were defined as unions of neighborhoods ( = nbhds); 

this gives a topology \underbar{because}:

\hsk $X$ = union of nbhds, and the intersection of two nbhds is a union of nbhds.

A collection \cabm of subsets of $X$ is a {\it basis} if it satisfies those two properties, i.e.:

\hsk $X=\bigcup\{B : B\in\cab\}$ , and 

\hsk if $B,B^\prime\in \cab$ and $x\in B\cap B^\prime$, then \exs $B^{\prime\prime}\in\cab$ with
$x\in B^{\prime\prime}\sset B\cap B^\prime$ .

The topology $\cat(\cab)$ that it {\it generates} is the unions of elements of \cabm .

\ssk

A {\it subbasis} is a collection \casm of subsets whose union is $X$. 

The basis $\cab(\cas)$ that it generates is the set of all finite intersections of elements of \casm .

\ssk

\hsk $f : (X,\cat)\ra (Y,\cat(\cab))$ is cts \lra $\finv(B)\in\cat$  for all  $B\in\cab$

\hsk $f : (X,\cat)\ra (Y,\cat(\cab(\cas)))$ is cts \lra $\finv(S)\in\cat$  for all  $S\in\cas$

\ssk

$\cau\in\cat(\cab)$ \lra \foa $x\in\cau$ \exs $B\in\cab$ so that $x\in B\sset \cau$ \hsk ; \hsk $\cat(\cab)\sset\cat$ \lra $\cab\sset\cat$

\ni On $\bbbr$, \cabm = $\{(a,b) : a,b\in\bbbr\}$ is a basis for the usual (metric) \top .

\ni \cabm = $\{[a,b) : a,b\in\bbbr\}$ is also a basis; $\bbbr$ with this \top\ is called the {\it Sorgenfrey line}.

\msk

\ni{\bf New spaces from old}

\ni Basic idea: \tops\ on new sets should be defined to make reasonable functions cts.

$A\sset X$ , $(X,\cat)$ , then would like $i: A\ra X$ \cts , so define

$\cat_A = \{i^{-1}(\cau) : \cau\in\cat\}$ = $\{\cau\cap A : \cau\in\cat\}$ = subspace topology

\hsk if \cabm is a basis for \catm, then $\{B\cap A : B\in\cab\}$ = $\cab_A$ is a basis for $\cat_A$

\hsk If $f: X\ra Y$ is \cts, then $f|_A : A\ra Y$ is \cts ; $f|_A = f\circ i$

If $B\sset A\sset X$ then the subspace topology $B$ gets from $(A,\cat_A)$ is the same as it gets
from $(X,\cat)$.

\ssk

$(X,\cat), (Y,\cat^\prime)$ top spaces, would like $p_X:X\times Y \ra X$ and $p_Y:X\times Y \ra Y$ 

(coord projections) to be cts. 

So need $p_X^{\-1}(\cau) = \cau\times Y$ and $p_Y^{\-1}(\cav) = X\times\cav$ open. These form a subbasis, with basis

$\cab = \{\cau\times\cav : \cau\in\cat, \cav\in\cat^\prime\}$ \hsk ;\hsk $\cat(\cab)$ = the {\it product topology} on $X\times Y$ = $\cat\times\cat^\prime$.

\hsk $f : Z\ra X\times Y$ is cts \lra $p_X\circ f$ and $p_Y\circ f$ are both continuous

\hsk If $\cat=\cat(\cab), \cat^\prime = \cat(\cab^\prime)$, then $\{B\times B^\prime : B\in\cab, B^\prime\in\cab^\prime\}$ is a 
basis for $\cat\times\cat^\prime$.

\hsk If $A\sset X, B\sset Y$, then the subspace topology on $A\times B\sset X\times Y$ is the same as $\cat_A\times \cat_B$

\ssk

\ni Products in general:

$(X_\alpha,\cat_\alpha)$ top. spaces, $\alpha\in I$, then there are two reasonable topologies on $\prod X_\alpha$: 

\hsk box \top: basis is $\{\prod\cau_\alpha : \cau_\alpha\in\cat_\alpha\}$

\hsk product \top: subbasis is $\bigcup\{p_{\alpha}^{-1}(\cau_\alpha) : \cau_\alpha\in\cat_\alpha\}$ ; $p_\alpha$ = proj to $X_\alpha$

In the product \top, $f: Z\ra \prod X_\alpha$ is cts \lra $p_{\alpha}\circ f$ is cts for all $\alpha$

\msk

\ni{\bf Closed sets}

$C\sset X$ is closed if $X\setminus C = \cau\in \cat$ \hsk ; \hsk i.e., $C$ is closed if $C=X\smin \cau$ for some $\cau\in\cat$

\hsk $\emptyset, X$ are closed ; $C,D$ closed $\Rightarrow C\cup D$ closed ; $C_\alpha$ closed $\Rightarrow \bigcap C_\alpha$ closed.

\hsk $f : X\ra Y$ is cts \lra $\finv\cau$ is closed in $X$ \foa $\cau$ closed in $Y$

\hsk $D\sset A\sset X$ is closed in $(A,\cat_A)$ \lra $C=D\cap A$ for some $D$ closed in $(X,\cat)$

\ssk

Closure:  cl$(A)$ = $\overline{A} = \bigcap\{C\sset X$ closed : $A\sset C\}$ = smallest closed set containing $A$

Interior: int$(A) = \bigcup\{\cat\in\cat : \cau\sset A\}$ = largest open set contained in $A$

\hsk cl$(X\smin A) = X\smin$int$(A)$ \hsk ;\hsk int$(X\smin A) = X\smin$cl$(A)$ .

\hsk $C$ closed and $A\sset C \Rightarrow \overline{A}\sset C$

\hsk $A\sset B \Rightarrow \overline{A}\sset\overline{B}$ \hsk ; \hsk $\overline{A\cup B} = \overline{A}\cup\overline{B}$

\hsk $A$ is closed \lra $\overline{A} = A$ \hsk ; \hsk $A$ is open \lra int$(A) = A$

\hsk The closure of $B\sset A$ as a subset of $A$ = $A\cap$cl$_X(B)$

\hsk The interior of $B\sset A$ as a subset of $A$ = $A\cap$int$_X(B)$

\hsk $f: X\ra Y$ is cts \lra for all $A\sset X$ , $f(\overline{A})\sset\overline{f(A)}$

\hsk If $A\sset X$ and $B\sset Y$, then $\overline{A\times B} = \overline{A}\times\overline{B}$

\hsk $x \in \overline{A}$ \lra every open $\cau\in\cat$ that contains $x$ intersects $A$.

\ssk

$x\in X$ is a limit point of $A\sset X$ if $x\in\overline{A\smin\{x\}}$ , i.e, every open set in $X$ 

that contains $x$ hits $A$ in a point other than $x$. 

The set of limit points of $A$ = $A^\prime$ = the derived set of $A$ 

\hsk $\overline{A}=A \cup A'$.

\ni {\bf More on continuity}

$f:X\ra Y$ and $g: Y\ra Z$ both cts $\Rightarrow$ $g\circ f : X\ra Z$ is cts

If $X=\bigcup \cau_alpha$, $\cau_alpha\in\cat$ for all $\alpha$, and $f: X\ra Y$ has $f|_{\cau_alpha}: \cau_\alpha\ra Y$ is cts
for all $\alpha$, then $f$ is cts.

If $X=C\cup D$, $C,D$ both closed, and $f: X\ra Y$ has $f|_C: C\ra Y$ and $f|_D: D\ra Y$ cts, then $f$ is cts.

In reverse: if $X=C\cup D$, $C,D$ both closed, $f: C\ra Y$ and $g: D\ra Y$ are both cts, and
$f(x)=g(x)$ for all $x\in C\cap D$, then $h: X\ra Y$, defined by $h(x)=f(x)$ if 
$x\in C$, $h(x)=g(x)$ if $x\in D$, is cts. A similar statement is true for $X$ = union of open sets.

\ssk

A cts bijection $f:X\ra Y$ is a {\it homeomorphism} if the inverse function $\finv : Y\ra X$ is also 
cts. $X$ and $Y$ are called {\it homeomorphic}. A homeo gives not only a bijection between points
of the spaces, but also between the open sets in the two topologies. Homeomorphic spaces have the 
same topological properties.

\msk

\ni {\bf Quotient spaces}

Given an equivalence relation $\sim$ on a topological space $(X,\cat)$, its quotient $X/\sim$
is the set of equivalence classes under $\sim$. The quotient map $p: X\ra X/\sim$ can be used
to induce a topology on $X/\sim$; $\cau\sset X/\sim$ is open \lra $p^{-1}(\cau)\in \cat$. This 
is the {\it quotient topology} on $X/\sim$.

\ssk

Given a quotient map $p:X \ra X/\sim$ and a cts function $f:X \ra Z$ with $g(a)=g(b)$ 
whenever $p(a)=p(b)$, then $f$ induces a continuous map $\overline{f}:X/\sim \ra Z$ 
with $f=\overline{f}\circ p$.

\ssk

\hsk If $A\sset X$, we can define an equiv reln generated by $x\sim y$ if $x,y\in A$; the quotient is $X/A$.

\hsk If $A\sset X$, $B\sset Y$ and $h:A\ra B$ is a homeo, then we have an equiv reln generated by $x\sim y$ if $h(x)=y$; quotient is $X\cup_{A=B} Y$

\hsk If $f: X\ra Y$ is continuous, then we have the equiv reln on $(X\times I)\cup Y$ generated by $(x,1)\sim f(x)$;
the quotient is the mapping cylinder $M_f$.

\msk

\ni{\bf Connectedness}

Motivation: understand the topological property underlying the Intermediate Value Theorem:

\hsk If $f:[a,b]\ra\bbbr$ is cts and $c$ is between $f(a)$ and $f(b)$, then $f(d)=c$ for some $d\in [a,b]$.

Idea: focus on when IVT \underbar{fails}: If $f: X\ra \bbbr$ fails IVT, then

\hsk $\finv((-\infty,c))=\cau\in\cat, \finv((c,\infty))=\cav\in\cat$ satisfy $\cau\cup\cav =X$, $\cau\cap\cav = \emptyset$, $a\in\cau,b\in\cav$.

Conversely, a pair of such sets allows us to build a cts $f: X\ra \{0,1\}\sset \bbbr$ failing IVT.

\ssk

A {\it separation} (or {\it disconnection}) of $(X,\cat)$ is a pair $\cau\cav\in\cat$ with $\cau\cup\cav =X$, $\cau\cap\cav = \emptyset$, and $\cau,\cav\neq\emptyset$. $X$ is {\it connected} if it admits no separation.

\hsk A subset $A\sset X$ is connected if $(A,\cat_A)$ is a connected space.

\hsk $A\sset X$ is connected \lra whenever $\cau,\cav\in\cat$ with $A\sset\cau\cup\cav$ and 
$A\cap\cau\cap\cav=\emptyset$, either $A\sset\cau$ or $A\sset\cav$.

\hsk If $A\sset X$ is connected and $\cau,\cav$ separate $X$, then either $A\sset\cau$ or $A\sset\cav$.

\hsk If $(X,\cat)$ is connected and $\cat^\prime\sset \cat$, then $(X,\cat^\prime)$ is connected.

\hsk If $A\sset X$ is connected and $f: X\ra Y$ is cts, then $f(A)\sset Y$ is connected.

\hsk If $A_\alpha\sset X$ are connected \foa $\alpha$ and $\bigcap A_\alpha\neq\emptyset$, then 
$\bigcup A_\alpha$ is connected.

\hsk If $A\sset X$ is connected and $A\sset\ B\sset \overline{A}$, then $B$ is connected.

\hsk If $X_\alpha$ are all connected, then $\prod X_\alpha$ is connected, when given the product topology.

\hsk\hsk This is false in general, when using the box topology.

\ssk

The connected subsets of $\bbbr$ are precisely the intervals: 

\hsk $(a,b),[a,b),(a,b],[a,b],(-\infty,b),(-\infty,b],(a,\infty),[a,\infty),\emptyset,\bbbr$ .

\ssk


\ni{\bf Path-connectedness}

A {\it path} in $X$ is a cts function $\gamma: [0,1]\ra X$. $X$ is {\it path-connected} if 
foa $x,y\in X$ \exs a path $\gamma: [0,1]\ra X$ with $\gamma(0)=x,\gamma(1)=y$ .

\hsk $(X,cat)$ path-connected $\Rightarrow$ $(X,cat)$ connected.

\hsk\hsk The converse is not true; there are connected spaces which are not path-connected.

\hsk If $A\sset X$ is path-connected and $f: X\ra Y$ is cts, then $f(A)\sset Y$ is path-connected.

\ni The relation $x\sim y$ if \exs connected $A\sset X$ with $x,y\in A$ is an equivalence relation;
the equivalence classes are the {\it connected components} $[x]$ of $X$.

\hsk $[x] = \bigcup\{A\sset X$ connected : $x\in A\}$ = largest connected subset containing $x$.

\hsk Connected components are closed subsets of $X$.

\ni The relation $x\sim y$ if \exs path in $X$ joining $x$ and $y$ is an equivalence relation;
the equivalence classes are the {\it path components} $[x]_p$ of $X$.

\hsk $[x]_p = \bigcup\{A\sset X$ path connected : $x\in A\}$ = largest path connected subset containing $x$.

\hsk $[x]_p\sset [x]$ ; each $[x]$ is a disjoint union of $[y]_p$ 's.

























\vfill
\end

