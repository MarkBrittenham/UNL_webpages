

\documentclass[12pt]{article}
\usepackage{amsmath}
\usepackage{amsfonts}
\usepackage{amssymb}
\usepackage{dutchcal}
\usepackage{graphicx}

\textheight=10.3in
\textwidth=7.6in
\voffset=-1.5in
\hoffset=-1.3in

\begin{document}

\def\ctln{\centerline}
\def\msk{\medskip}
\def\bsk{\bigskip}
\def\ssk{\smallskip}
\def\hsk{\hskip.3in}
\def\ra{\rightarrow}
\def\ubr{\underbar}

\def\mt{{\mathcal T}}
\def\mb{{\mathcal B}}
\def\ms{{\mathcal S}}
\def\mu{{\mathcal U}}
\def\mv{{\mathcal V}}

\def\bbc{{\mathbb C}}
\def\bbd{{\mathbb D}}
\def\bbh{{\mathbb H}}
\def\bbr{{\mathbb R}}
\def\bbz{{\mathbb Z}}
\def\bbq{{\mathbb Q}}
\def\bbn{{\mathbb N}}
\def\spc{$~$\hskip.15in$~$}

\def\sset{\subseteq}
\def\del{\partial}
\def\lra{$\Leftrightarrow$}
\def\bra{$\Rightarrow$}
\def\dsp{\displaystyle}

\def\ddr{d_{\bbr^2_+}}
\def\ddd{d_{\bbd^2}}

%%\UseAMSsymbols

\ctln{\bf Math 990 Hyperbolic Geometry and Topology}

\ctln{\bf Problem Set 3}

%%{\it General notation: $A=\begin{pmatrix} a & b \\ c & d \end{pmatrix}\in SL(2,\bbr)$ and
%%$\dsp f_A(z)=\frac{az+b}{cz+d}$ is the corresponding $\bbr^2_+$-isometry. 
%%$\ddr(z_1,z_2)$ is the hyperbolic distance function in $\bbr^2_+$, and 
%%$\ddd(w_1,w_2)$ is the hyperbolic distance function in $\bbd^2$.}

\begin{description}


\item{17.} (a) ["note this simplifying fact..."] If $X$ is a (metric) space, $f,g,h:X\ra X$ 
are isometries, and $f\circ g=g\circ f$, then setting $F=h\circ f\circ h^{-1}$ and 
$G=h\circ g\circ h^{-1}$ we have $F\circ G=G\circ F$.

\item{\spc} (b) Show that if $f,g\in Isom(\bbh^3)$ and $f\circ g=g\circ f$, then either
(1) $f$ and $g$ are both hyperbolic and share the same axis, (2) $f$ and $g$ are both parabolic
and share the same fixed point at infinity, or $f$ and $g$ are both elliptic, and \ubr{either} 
share the same axis \ubr{or} are both of order 2 and have axes that meet orthogonally.
[Hint: (a)! Make one `standard' and use their representations as matrices.]

\item{18.} Dylan Thurston (son of Bill!) showed that from a diagram of a knot $K$ with $k$ crossings 
you can tile the complement of $K$ using $k$ octahedra (with common north and south poles).
[See a concise description of the construction in the first two pages of the paper 
"Bipyramids and bounds on volumes of hyperbolic 
links" by Colin Adams on the public webpage.] By cutting an octagon into tetrahedra, 
and arguing that a geodesic hyperbolic tetrahedron with some finite vertices must have 
volume no more than the regular ideal tetrahedron, give an upper bound on the volume of 
the complement of a hyperbolic knot with $k$ crossings, in terms of $k$ and $v_3$ = the
volume of the regular hyperbolic ideal tetrahedron.

\item{19.} Given a knot diagram $D$ for $K$, a {\it twist region} for the diagram is a 
collection of crossings (possibly one!) which form a single pair of twisted strands, as in the diagram.
It is a fact that if a link (complement) $L=K\cup C$ has a component $C$ bounded by a disk pierced twice by the 
remainder of the link $K$, then the result of Dehn filling along $C$ with surgery coefficient 
$1/n$ results in the complement of the link $K$ \ubr{with} $n$ \ubr{full} twists added to the 
two piercing strands. The {\it twist number} $tw(K)$ of a knot is the minimum number of disjoint
twist regions needed to encompass all of its crossings. Use Problem \#18 and the fact that Dehn filling
lowers volume to find an upper bound on the volume of a hyperbolic knot in terms of $t(K)$ and $v_3$.


\begin{figure}[h]
\begin{center}
\includegraphics[width=3.5in]{twisting.eps}
%%\caption{6_3}\label{fig:flype1}
\end{center}
\end{figure} 

\item{20.} Show that $1/n$ Dehn filling on the complement of the unknot (the knot with no crossings),
i.e., filling so that $1\cdot\mu+n\cdot\lambda$ bounds a disk, always yields the 3-sphere $S^3$.
[This is why in Problem \#19 above, `all' that happens when filling $C$ is that the strands get twisted...]

\item{20.}  The {\it Seifert-Weber dodecahedral space} is a 3-manifold obtained by gluing the
opposite faces of a dodecahedron by a $3\pi/10$ twist. (See the figure for a depiction of one of these
gluings.) Show that the resulting space is a (closed) 3-manifold, by verifying that the link of 
every vertex is a 2-sphere. If we treat the dodecahedron as a \ubr{regular} hyperbolic dodecahedron
(i.e., the convex hull of symmmetrically placed points around the center), what must the dihedral 
angles be for the gluing consistency equations to provide a hyperbolic structure around the edges?
If you want to dig more, show that there is a regular hyperbolic dodecahedron with those angles!

\begin{figure}[h]
\begin{center}
\includegraphics[width=3.5in]{dodecahedron.eps}
%%\caption{6_3}\label{fig:flype1}
\end{center}
\end{figure} 

\item{20.} Play with the Seifert-Weber dodecahedral space!
It is found in the SnapPy census as 
{\it M = DodecahedralOrientableClosedCensus(solids = 1)[-1]}. Try random retriangulations
to make its Diriclet domain look `dodecahedra-like'.
Use SnapPy to (try to!) find a representation of the Seifert-Weber dodecahedral space as 
Dehn filling on a link in $S^3$; you can try to move towards
a knot complement by Dehn drilling (and then looking at the 'also known as' table),
and filling and drilling...
[Note: SnapPy reports that $M$ has first homology $\bbz_5+\bbz_5+\bbz_5$, which means that a link to 
Dehn fill to get it will need at least 3 components.] [N.B. I have not succeeded at this, myself, yet.]

\item{21.} Two manifolds $M_1,M_2$ are {\it commensurable} if there is a third manifold that
is a finite-sheeted covering space of both. (The number of sheets need not be the same.) Show that
``is commensurable to'' is an equivalence relation. [N.B. This is `really' a group theory 
question...]

\item{22.} (a) Show that if $\alpha,\beta$ are isometries of $\bbh^3$ that share no fixed point
at infinity, then there is a third isometry $\gamma$ so that $\gamma\alpha\gamma^{-1}=\alpha^{\-1}$
and $\gamma\beta\gamma^{-1}=\beta^{-1}$. [Hint: think about axes, and a rotation through $pi$
around a well-chosen other axis! Or look in Thurston's notes...]

\item{\spc} (b) (Show that!) the hypothesis of (a) is true for any pair of 
elements of the image $\varphi(\pi_1(M))\subseteq PSL(2,\bbc)$ of the fundamental group of a closed
hyperbolic 3-manifold under the natural identification with a group of isometries, 
at least if $\alpha,\beta$ don't both lie in an abelian subgroup. Conclude that for 
any word w=w(x,y) in the letters $x,y,$, if 
$w(\alpha,\beta)=1$ in $\pi_1(M)$, then $w(\alpha^{-1},\beta^{-1})=1$, as well.
[N.B. Note that the same is actually true if they \ubr{do} lie in an abelian subgroup, but 
for a different reason!]


\item{24.} It is \ubr{unknown} if starting from a hyperbiolic alternating knot $K$ and an alternating diagram $D$,
changing some of its crossings can result in a hyperbolic knot $K^\prime$ with volume \ubr{larger} than the volume of 
$K$. It has, however, been verified that this cannot happen for a \ubr{single} crossing change, 
for alternating knots through 16 crossings
(as of 2015). Devise a SnapPy experiment to randomly build alternating knots, randomly change some crossings, 
and test to see if the volume has gone up. Don't forget to write any successes to a file, so that you can 
claim the prize of finding a counterexample!

\end{description}
\vfill

\end{document}


\item{23.} A list of volumes of knots through 14 crossings, ordered by volume, can be found on the class
webpage. Pick some pairs that appear to have the same volume, query SnapPy to see if that is 
likely right, and try to decide if any of our tools to prove equality (cutting and pasting along 3-punctured
or 4-punctured spheres, commensurability) could \ubr{explain} the equality. 


\vspace{-.1in}

\begin{figure}[h]
\begin{center}
\includegraphics[width=1.5in]{dodecahedron.eps}
%%\caption{6_3}\label{fig:flype1}
\end{center}
\end{figure} 

\vspace{-.3in}

