

\documentclass[12pt]{article}
\usepackage{amsmath}
\usepackage{amsfonts}
\usepackage{amssymb}
\usepackage{dutchcal}

\textheight=10in
\textwidth=6.5in
\voffset=-1in
\hoffset=-1in

\begin{document}

\def\ctln{\centerline}
\def\msk{\medskip}
\def\bsk{\bigskip}
\def\ssk{\smallskip}
\def\hsk{\hskip.3in}
\def\ra{\rightarrow}
\def\ubr{\underbar}

\def\mt{{\mathcal T}}
\def\mb{{\mathcal B}}
\def\ms{{\mathcal S}}
\def\mu{{\mathcal U}}
\def\mv{{\mathcal V}}

\def\bbc{{\mathbb C}}
\def\bbd{{\mathbb D}}
\def\bbr{{\mathbb R}}
\def\bbz{{\mathbb Z}}
\def\bbq{{\mathbb Q}}
\def\bbn{{\mathbb N}}
\def\spc{$~$\hskip.15in$~$}

\def\sset{\subseteq}
\def\del{\partial}
\def\lra{$\Leftrightarrow$}
\def\bra{$\Rightarrow$}
\def\dsp{\displaystyle}

\def\ddr{d_{\bbr^2_+}}
\def\ddd{d_{\bbd^2}}

%%\UseAMSsymbols

\ctln{\bf Math 990 Hyperbolic Geometry and Topology}

\ctln{\bf Problem Set 1}

\msk

{\it General notation: $A=\begin{pmatrix} a & b \\ c & d \end{pmatrix}\in SL(2,\bbr)$ and
$\dsp f_A(z)=\frac{az+b}{cz+d}$ is the corresponding $\bbr^2_+$-isometry. 
$\ddr(z_1,z_2)$ is the hyperbolic distance function in $\bbr^2_+$, and 
$\ddd(w_1,w_2)$ is the hyperbolic distance function in $\bbd^2$.}

\begin{description}

\item{1.} Show that if $z_1=0$ is the origin in $\bbd^2$ and $z_2\in\bbd^2$, then 

\hfill
$\dsp\ddd(z_1,z_2)=\int_0^{|z_2|}\frac{1}{1-t^2}\ dt=\frac{1}{2}\ln\Big|\frac{1+|z_2|}{1-|z_2|}\Big|$
( = arctanh$(|z_2|)$ ).

\msk

\item{2.} Show that for $w_1\in\bbd^2$, the map $\dsp \varphi:z\mapsto \frac {z-w_1}{-\overline{w_1}z+1}$
is an isometry of $\bbd^2$ sending $w_1$ to the origin. [Hint: use our particular isometry
$\psi:\bbd^2\ra\bbr^2_+$ and show that $\psi\circ\varphi\circ\psi^{-1}=f_A$ for some $A\in SL(2,\bbr)$.]
Use this, together with the previous problem,
to give an explicit formula for $\ddd(z_1,z_2)$ for all $z_1,z_2\in\bbd^2$. 

\msk

\item{3.} Use Prbolem \#2 and the map $\psi$ to give a fairly ugly explicit formula for 
$\ddr(z_1,z_2)$ for all $z_1,z_2\in\bbr^2_+$.

\msk


\item{4.} Show that if $z_1,z_2$ are distinct points in $\bbr^2_+$, as are
$w_1,w_2$, and $\ddr(z_1,z_2)=\ddr(w_1,w_2)$, then there is an $A\in SL(2,\bbr)$
with $f_A(z_i)=w_i$ for all $i$. [Hint: you could brute-force this, and solve 
equations, or show that transitivity on point-direction pairs is enough to 
guarantee it?]

\msk

\item{5.} Show that if $\alpha_1<\alpha_2<\alpha_3$ and 
$\alpha_1^\prime<\alpha_2^\prime<\alpha_3^\prime$
are all in $\bbr\cup\{\infty\}$ (where we interpret $x<\infty$ for all $x\in\bbr$),
then there is an $A\in SL(2,\bbr)$ with $f_A(\alpha_i)=\alpha_i^\prime$ for all $i$.
[Hint: do this (first) for $\alpha_1=0$, $\alpha_2=1$ and $\alpha_3=\infty$. Or: `just'
treat it as a system of linear equations!]

\msk

\item{6.} The isometry of $\bbr^2_+$ given by $\varphi(z)= -\overline{z}$ is a `reflection': it fixes the 
positive imaginary axis (a geodesic) pointwise and swaps the two sides of the axis. Call anything
conjugate to this a reflection ($\psi=f_A\circ\varphi\circ f_A^{-1}$ fixes pointwise (check!) the image
of the imaginary axis under $f_A$). Show that every elliptic or loxodromic isometry of $\bbr^2_+$ is a 
composition of two reflections. Is this also true for a parabolic isometry?

\msk

\item{7.} Show that the `AAA Congruence Theorem' for triangles holds when some of the vertices of the
triangle are ideal vertices. [Problem \#5 is basically the all-ideal-vertices case. When one of them isn't
ideal, move it to the origin!]

\msk

\item{8.} Generalize the `area equals angle defect' result to all polygons. [What is the correct
notion of `angle defect'?]. Use this to compute the hyperbolic area of the all-right-angle hexagon,
and then compute the hyperbolic area of a closed orientable surface of genus $g\geq 2$. 

\msk



\end{description}
\vfill

\end{document}


