

\documentclass[12pt]{article}
\usepackage{amsmath}
\usepackage{amsfonts}
\usepackage{amssymb}
\usepackage{dutchcal}
\usepackage{graphicx}

\textheight=10.3in
\textwidth=7.6in
\voffset=-1.5in
\hoffset=-1.3in

\begin{document}

\def\ctln{\centerline}
\def\msk{\medskip}
\def\bsk{\bigskip}
\def\ssk{\smallskip}
\def\hsk{\hskip.3in}
\def\ra{\rightarrow}
\def\ubr{\underbar}

\def\mt{{\mathcal T}}
\def\mb{{\mathcal B}}
\def\ms{{\mathcal S}}
\def\mu{{\mathcal U}}
\def\mv{{\mathcal V}}

\def\bbc{{\mathbb C}}
\def\bbd{{\mathbb D}}
\def\bbh{{\mathbb H}}
\def\bbr{{\mathbb R}}
\def\bbz{{\mathbb Z}}
\def\bbq{{\mathbb Q}}
\def\bbn{{\mathbb N}}
\def\spc{$~$\hskip.15in$~$}

\def\sset{\subseteq}
\def\del{\partial}
\def\lra{$\Leftrightarrow$}
\def\bra{$\Rightarrow$}
\def\dsp{\displaystyle}

\def\ddr{d_{\bbr^2_+}}
\def\ddd{d_{\bbd^2}}

%%\UseAMSsymbols

\ctln{\bf Math 990 Hyperbolic Geometry and Topology}

\ctln{\bf Problem Set 4}

%%{\it General notation: $A=\begin{pmatrix} a & b \\ c & d \end{pmatrix}\in SL(2,\bbr)$ and
%%$\dsp f_A(z)=\frac{az+b}{cz+d}$ is the corresponding $\bbr^2_+$-isometry. 
%%$\ddr(z_1,z_2)$ is the hyperbolic distance function in $\bbr^2_+$, and 
%%$\ddd(w_1,w_2)$ is the hyperbolic distance function in $\bbd^2$.}

\begin{description}


\item{25.} Show that if an ideal tetrahedron in $\bbh^3$ has (pairs of) dihedral angles
$\alpha, \beta, \gamma$, in clockwise order around a vertex, 
then the shape invariant associated to the edge with angle $\alpha$ is 
$\displaystyle\frac{\sin\gamma}{\sin\beta}e^{i\alpha}$ .

\item{26} One way to build doubly primitive knots on the standard genus-2 surface $\sigma$ is the following.
Pick \underbar{disjoint} non-separating simple closed curves $\alpha,\beta$ on $\Sigma$, one bounding a disk on 
one side of $\Sigma$ and
one bounding a disk on the other side (see figure for an example). Then draw any simple loop $K$ that hits $\alpha$ and 
$\beta$ each exactly once. The loops $\alpha.\beta$ cut $\Sigma$ into a 4-punctured sphere $P$ (for `pillowcase'). 
$K$ meets $P$ in a pair of arcs; these therefore \underbar{must} look like a pair of arcs of the same
rational slope when viewing $P$ as a pillowcase (see figure!). Use this perspective to build families of 
doubly-primitive knots. Main question: does this approach (starting with disjoint loops to build $K$)
\underbar{ever} build anything other than the unknot? In particular, what happens with the specific example of
loops in the figure? [N.B. I don't actually know the answer, my first attempt to build an `interesting' knot
was the unknot...]

\vspace{-.1in}

\begin{figure}[h]
\begin{center}
\includegraphics[width=5in]{doublyprim.eps}
%%\caption{6_3}\label{fig:flype1}
\end{center}
\end{figure} 

\vspace{-.3in}

\item{27.} Starting from the `standard' genus-1 Seifert/spanning surface $\Sigma$ for the trefoil knot pictured below, 
construct some knots on $\Sigma$, use SnapPy to determine which knots they are (or if you hit `[]', which means 
`I don't know', at least find their volume!), and determine which `surface slope' Dehn surgery fraction $a/b$ 
yields the Lens space.
(This, in theory, requires you to build a Seifert surface $S$ for your new knot? You can then count the net number of 
times the surface slope hits $S$. Or you could try to find it
by experiment with SnapPy, the coefficient will (always) be $a/1$, and SnapPy will report a 
volume of $0$ when it encounters a Lens space.)

\vspace{-.1in}

\begin{figure}[h]
\begin{center}
\includegraphics[width=1.5in]{trefoil.eps}
%%\caption{6_3}\label{fig:flype1}
\end{center}
\end{figure} 

\vspace{-.3in}

\item{28.} Repeat Problem \#27, for the `standard' genus-1 Seifert surface 
(although this is a `non-standard', checkerboard, view of it) for the figure-8 knot pictured below.


\vspace{-.1in}

\begin{figure}[h]
\begin{center}
\includegraphics[width=2in]{figure8.eps}
%%\caption{6_3}\label{fig:flype1}
\end{center}
\end{figure} 

\vspace{-.3in}

\vfill\eject

\item{29.} Show that if a link $L$ has a `belt' component (as in the pictures in Problem \#30), 
that is, a component $K$ that bounds a disk $D$ that meets the
other components of $L$ in exactly two points, but does \underbar{not} bound a disk that is disjoint from 
the other components, then the twice-punctured disk $P\subseteq D$ obtained by removing
disk neighborhoods of the two points of $D\cap L$ is an {\it essential} 3-punctured sphere in 
$X(K)=S^3\setminus N(L)$. That is, any (simple) loop
$\gamma\subseteq P$ that bounds a disk in $X(K)$ must also bound a disk in $P$. [This geometric notion of essential
agrees with the $pi_1$-version, by the Loop Theorem. Hint: there aren't very many simple
loops in $P$ that \underbar{don't} bound disks in $P$...]


\item{30.} Colin Adams' result on cutting-twistg-regluing along 3-punctured spheres applies
also to \underbar{multiple} such surfaces. As a result, show that there is an `addition'
theorem: if two hyperbolic links $L_i$ in $S^3$ each have a component which locally looks like the figure
(each with a tangle $T_i$ on the `outside'), then the {\it belted sum} of the two links,
shown in the figure, is hyperbolic and has volume the sum of the volumes of the $L_i$. 
Use this and induction to express the volume of the `belted' $n$-link chain (shown), in terms of the 
volume of the Whitehead link (shown!).

\vspace{-.05in}

\begin{figure}[h]
\begin{center}
\includegraphics[width=6.5in]{belted.eps}
%%\caption{6_3}\label{fig:flype1}
\end{center}
\end{figure} 

\vspace{-.3in}

\item{31.}  A list of volumes of knots through 14 crossings, ordered by volume, can be found on the class
webpage. Pick some pairs that appear to have the same volume, query SnapPy to see if that is 
likely right, and use SnapPy and its computation of hyperbolic
structures/volumes of cyclic branched covers of $S^3$ branched over $K$, and Dehn fillings,
to decide if the pairs are not, or maybe are, mutants of one another.

\end{description}
\vfill

\end{document}


\item{32.} 

\item{23.} 


\vspace{-.1in}

\begin{figure}[h]
\begin{center}
\includegraphics[width=1.5in]{dodecahedron.eps}
%%\caption{6_3}\label{fig:flype1}
\end{center}
\end{figure} 

\vspace{-.3in}

\begin{figure}[h]
\begin{center}
\includegraphics[width=3.5in]{twisting.eps}
%%\caption{6_3}\label{fig:flype1}
\end{center}
\end{figure} 