

\documentclass[12pt]{article}
\usepackage{amsmath}
\usepackage{amsfonts}
\usepackage{amssymb}
\usepackage{dutchcal}
\usepackage{graphicx}

\textheight=11in
\textwidth=7.6in
\voffset=-1.5in
\hoffset=-1.3in

\begin{document}

\def\ctln{\centerline}
\def\msk{\medskip}
\def\bsk{\bigskip}
\def\ssk{\smallskip}
\def\hsk{\hskip.3in}
\def\ra{\rightarrow}
\def\ubr{\underbar}

\def\mt{{\mathcal T}}
\def\mb{{\mathcal B}}
\def\ms{{\mathcal S}}
\def\mu{{\mathcal U}}
\def\mv{{\mathcal V}}

\def\bbc{{\mathbb C}}
\def\bbd{{\mathbb D}}
\def\bbh{{\mathbb H}}
\def\bbr{{\mathbb R}}
\def\bbz{{\mathbb Z}}
\def\bbq{{\mathbb Q}}
\def\bbn{{\mathbb N}}
\def\spc{$~$\hskip.15in$~$}

\def\sset{\subseteq}
\def\del{\partial}
\def\lra{$\Leftrightarrow$}
\def\bra{$\Rightarrow$}
\def\dsp{\displaystyle}

\def\ddr{d_{\bbr^2_+}}
\def\ddd{d_{\bbd^2}}

%%\UseAMSsymbols

\ctln{\bf Math 990 Hyperbolic Geometry and Topology}

\ctln{\bf Problem Set 2}

%%{\it General notation: $A=\begin{pmatrix} a & b \\ c & d \end{pmatrix}\in SL(2,\bbr)$ and
%%$\dsp f_A(z)=\frac{az+b}{cz+d}$ is the corresponding $\bbr^2_+$-isometry. 
%%$\ddr(z_1,z_2)$ is the hyperbolic distance function in $\bbr^2_+$, and 
%%$\ddd(w_1,w_2)$ is the hyperbolic distance function in $\bbd^2$.}

\begin{description}

\item{9.} A sphere with 4 punctures has a decomposition into four ideal triangles. 
Use this to describe all of the complete hyperbolic structures on a 4�punctured sphere,
in terms of "translation coordinates" along the edges of the triangles. How many
independent parameters do we have?

\item{10.}  Show that in hyperbolic 3-space $\bbh^3$ for any two geodesics 
$\gamma_1,\gamma_2$ that do not share a common
endpoint on the sphere at infinity there is a third geodesic $\gamma_3$ that 
meets both $\gamma_1$ and $\gamma_2$ and is perpendicular to both. [As usual,
pick an isometry to make them look as nice as possible, first...]

\item{11.} Show that every geodesic in $\bbh^3$ is the intersection of two totally
geodesic planes in $\bbh^3$, and, conversely, that the intersection of two
totally goedesic planes (if non-empty!) is a geodesic in $\bbh^3$.

\item{12.} Given $z_1,z_2,z_3,w_1,w_2,w_3\in\bbc$, under what conditions is there a fractional linear
transformation $\dsp f(z)=\frac{az+b}{cz+d}$ with $f(z_i)=w_i$ for $i=1,2,3$ ? 
Under what conditions is the map unique? [Try to directly write one down and see when you fail...]

\item{13.} (We found all of the isometries.) Use the fact that the geodesics of $\bbh^3$ (in the upper
halfspace model) are the semicircles perpendicular to $\bbr^2\times\{0\}$ and vertical lines to show that
the isometries of $\bbh^3$ are precisely the fractional linear transformations of Problem \#12 (composed
with $(x,y,z)\mapsto (-x,y,z)$ to get the orientation-reversing maps). That is, show that "behavior
at a single point" determines an isometry uniquely, and all such behaviors are captured by these maps.

\item{14.} Prove that if $M$ is any 3-manifold obtained by gluing the faces of ideal tetrahedra, and the link of every 
vertex of the associated complex (by including the vertices) is a torus, then 
the number of 1-cells in the (ideal) triangulation of $M$ equals the number of 3-cells
(i.e., tetrahedra) in the triangulation. [Hint: Think about counting (fractions of) things 
in the links of the vertices, knowing what the Euler characteristics should be.]

\item{15.} Find a decomposition of the complement $X$ of the knot $6_3$ in $S^3$ into ideal tetrahedra, 
and write down the shape parameters and shape gluing equations for a hyperbolic structure on $X$.
(See Purcell's notes in the "public" directory, section 5.1, for a refresher on decomposing $X$, 
if you need it.)

\vspace{-.1in}

\begin{figure}[h]
\begin{center}
\includegraphics[width=1.5in]{6_3.eps}
%%\caption{6_3}\label{fig:flype1}
\end{center}
\end{figure} 

\vspace{-.3in}

\item{16.} In Purcell's notes, page 45, she gives the following ideal tetrahedral decomposition
of the complement $X$ of the knot $6_1$:

\begin{figure}[h]
\begin{center}
\includegraphics[width=5in]{ps2prob16.eps}
%%\caption{Flype}\label{fig:flype1}
\end{center}
\end{figure} 

\vspace{-.3in}

\item{\spc} Use this to give the shape gluing equations for the hyperbolic structure(s) on $X$.
Draw the induced triangulation on the link of the ideal vertex, and give the corresponding
completeness equations for a hyperbolic structure on $X$.

\end{description}
\vfill

\end{document}


