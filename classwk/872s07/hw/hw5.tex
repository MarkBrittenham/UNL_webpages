

\magnification=1200
\overfullrule=0pt
\parindent=0pt

\input amstex

\input epsf
\loadmsbm

\def\ctln{\centerline}
\def\u{\underbar}
\def\ssk{\smallskip}
\def\msk{\medskip}
\def\bsk{\bigskip}
\def\hsk{\hskip.1in}
\def\hhsk{\hskip.2in}

\def\dsl{\displaystyle}
\def\hskp{\hskip1.5in}
\def\ra{\rightarrow}
\def\lra{$\Leftrightarrow$}
\def\pu{\pi_1}
\def\mpu{$\pi_1$}
\def\bra{$\Rightarrow$}
\def\bbr{{\Bbb R}}
\def\bbz{{\Bbb Z}}
\def\bbq{{\Bbb Q}}
\def\del{\partial}
\def\indt{\item{}}
\def\wtl{\widetilde}



\ctln{\bf Math 872 Algebraic Topology}

\ssk

\ctln{Problem Set \#\ 5}

\ssk

\ctln{Starred (*) problems due Thursday, March 8}

\bsk

\item{\bf 22.} Find a $\Delta$-complex structure for, and compute the (simplicial)
homology groups of, the space obtained from an annulus $A=S^1\times I$ 
by gluing $S^1\times\{1\}$ to $S^1\times\{0\}$ by a map representing 2 times the
generator of $\pi_1(S^1)$. (See figure below.)

\msk

\ctln{\vbox{\hsize=4in
\leavevmode
\epsfxsize=4in
\epsfbox{hw5f1.ai}}}

\msk

\item{\bf (*) 23.} Find a $\Delta$-complex structure for, and compute the (simplicial)
homology groups of, the space obtained from an annulus $A=S^1\times I$ 
by gluing $s^1\times\{1\}$ to $s^1\times\{0\}$ by a map representing -2 times the
generator of $\pi_1(S^1)$. (See figure above.)


\msk

\item{\bf (*) 24.} Compute the simplicial 
homology groups of the 3-simplex $\Delta^3$ (that is, the $\Delta$-complex
obtained from 4 vertices by gluing on 6 1-simplices, 4 2-simplices and a 
single 3-simplex in the ``obvious'' way). 

\msk

\item{\bf 25.} Regarding the $n$-simplex $X=\Delta^n$ as a $\Delta$-complex 
in the natural way, show that if $A\subseteq X$ is a subcomplex with $H_{n-1}(A)\neq 0$,
then $A=\del \Delta^n$. (Hint: show that an $(n-1)$-cycle for $A$ (and hence for $X$)
must either be 0 or have non-zero coefficient for \underbar{every}
$(n-1)$-dimensional face of $X$.)

\msk

\item{\bf 26.} Show that if $A\subseteq X$ is a retract of $X$, then the inclusion map
$\iota:A\ra X$ induces an injection on all singular homology groups. 


\vfill
\end