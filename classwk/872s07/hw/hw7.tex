

\magnification=1200
\overfullrule=0pt
\parindent=0pt
\nopagenumbers

\input amstex

\input epsf
\loadmsbm

\def\ctln{\centerline}
\def\u{\underbar}
\def\ssk{\smallskip}
\def\msk{\medskip}
\def\bsk{\bigskip}
\def\hsk{\hskip.1in}
\def\hhsk{\hskip.2in}

\def\dsl{\displaystyle}
\def\hskp{\hskip1.5in}
\def\ra{\rightarrow}
\def\lra{$\Leftrightarrow$}
\def\pu{\pi_1}
\def\mpu{$\pi_1$}
\def\bra{$\Rightarrow$}
\def\bbr{{\Bbb R}}
\def\bbz{{\Bbb Z}}
\def\bbq{{\Bbb Q}}
\def\del{\partial}
\def\indt{\item{}}
\def\wtl{\widetilde}
\def\coker{\text{coker}}
\def\im{\text{im}}
\def\wtih{\widetilde{H}}

\ctln{\bf Math 872 Algebraic Topology}

\ssk

\ctln{Problem Set \#\ 7}

\ssk

\ctln{Starred (*) problems due Thursday, April 5}

\bsk

\item{\bf (*) 33.} Find examples of spaces and subspaces
$A_0\subseteq X_0$ and $A_1\subseteq X_1$ so that
$H_*(X_0)\cong H_*(X_1)$ and $H_*(A_0)\cong H_*(A_1)$,
but $H_*(X_0,A_0)\not\cong H_*(X_1,A_1)$ . (If you want to make
it more challenging, find examples with all of the spaces path-connected?
Note that Problem \#35 gives a hint on how \u{not} to solve this problem...)

\msk

\item{\bf 34.} Show that if $A\subseteq X$ and the identity map 
$I:X\ra X$ is homotopic 
to a map $f:X\ra X$ with $f(X)\subseteq A$, then
for every $n$, $H_n(A)\cong H_n(X)\oplus H_{n+1}(X,A)$.

\item{} (So $A$ has more ``holes'' than $X$ does...)

\msk

\item{\bf 35.} (a): Let $f:(X,A)\ra (Y,B)$ be a map of pairs such that
both $f:X\ra Y$ and $f:A\ra B$ are homotopy equivalences. Show that
the induced map $f_*:H_n(X,A)\ra H_n(Y,B)$ is an isomorphism for all $n$.

\ssk

\item{} (b): Show that the inclusion map $\iota:(D^n,\del D^n)\ra (D^n,D^n\setminus \{0\})$
satisfies the hypotheses of (a), but is \u{not} a {\it homotopy of pairs}, that is, there
is \u{not} a map 

\item{} $f:(D^n,D^n\setminus \{0\})\ra (D^n,\del D^n)$ so that $f\circ \iota$ and $\iota\circ f$
are both homotopic, as maps of pairs, to the identity maps.


\msk

\item{\bf 36.} Compute the singular homology groups of the pseudo-projective 
planes $P_n$, $n\geq 2$, shown below, where the boundary has been subdivided into $n$ equal arcs.

\msk



\ctln{\vbox{\hsize=1.4in
\leavevmode
\epsfxsize=1.4in
\epsfbox{hw7.eps}}}



\msk

\item{\bf (*) 37.} For a space $X$ the {\it cone} on $X$ is the quotient space 

\item{} $cX=X\times I/\{(x,0)\sim (y,0) : x,y\in X\}$ = $X\times I/X\times \{0\}$,
and the {\it suspension} of $X$ is the quotient space 
$SX=X\times I/\{(x,0)\sim (y,0) , (x,1)\sim (y,1) : x,y\in X\}$,
which can be thought of as two cones on $X$ glued along their common copy of $X$.
Show that for any path connected space $X$, $\wtih_i(cX)=0$ and $\wtih_i(SX)\cong \wtih_{i-1}(X)$
for all $i$.

\msk

\item{\bf 38.} Show that, for any collection of finitely generated abelian groups $G_1,\ldots ,G_n$,
there is a path-connected space $X$ with $\wtih_i(X)\cong G_i$ for all $i=1,\ldots,n$ and
$\wtih_i(X)=0$ for all other $i$.

\vfill
\end









\ctln{\vbox{\hsize=4in
\leavevmode
\epsfxsize=4in
\epsfbox{hw6.ai}}}


