

\magnification=1200
\overfullrule=0pt
\parindent=0pt
\nopagenumbers

\input amstex

\input epsf
\loadmsbm

\def\ctln{\centerline}
\def\u{\underbar}
\def\ssk{\smallskip}
\def\msk{\medskip}
\def\bsk{\bigskip}
\def\hsk{\hskip.1in}
\def\hhsk{\hskip.2in}

\def\dsl{\displaystyle}
\def\hskp{\hskip1.5in}
\def\ra{\rightarrow}
\def\lra{$\Leftrightarrow$}
\def\pu{\pi_1}
\def\mpu{$\pi_1$}
\def\bra{$\Rightarrow$}
\def\bbr{{\Bbb R}}
\def\bbz{{\Bbb Z}}
\def\bbq{{\Bbb Q}}
\def\del{\partial}
\def\indt{\item{}}
\def\wtl{\widetilde}
\def\coker{\text{coker}}
\def\im{\text{im}}

\ctln{\bf Math 872 Algebraic Topology}

\ssk

\ctln{Problem Set \#\ 6}

\ssk

\ctln{Starred (*) problems due Thursday, March 29}

\bsk

\item{\bf (*) 27.} Show that for chain maps $f,g$ between chain
complexes ${\Cal A} = \{A_n\},{\Cal B}=\{B_n\}$, the relation 
``$f$ and $g$ are chain homotopic'' is an equivalence relation.

\msk

\item{\bf 28.} Show that if $A\subseteq X$, then the inclusion map
$i:A\ra X$ induces an isomorphism on homology groups \lra\ 
$H_n(X,A)=0$ for all $n\geq 0$.


\msk

\item{\bf 29.} Show that if a short exact sequence
\hskip.2in
$0\ra A{\buildrel{\alpha}\over\ra} B{\buildrel{\beta}\over\ra} C\ra 0$
\hskip.2in
{\it splits}, that is, there is a map $\gamma:B \ra A$ with
$\gamma\circ\alpha=I$, then the map $\varphi:B\ra A\oplus C$
given by $b\mapsto (\gamma(b),\beta(b))$, is an isomorphism.

\ssk

[This is part of the Splitting Lemma, proved in Hatcher, p.147. Splitting is equivalent 
to the existence of $\delta:C\ra B$ satisfying
$\beta\circ\delta = I$, but this is irrelevant to the question above.]

\msk

\item{\bf (*) 30.} Show that if $A\subseteq X$ and $r:X\ra A$ is a retraction, then
for every $n$,

\item{} $H_n(X)\cong H_n(A)\oplus H_n(X,A)$. 

\ssk

\item{} [Hint: show that the (piece of) the long exact homology sequence

\item{} $H_n(A)\ra H_n(X)\ra H_n(X,A)$ is ``really'' 

\item{} $0\ra H_n(A)\ra H_n(X)\ra H_n(X,A)\ra 0$, and splits.]

\msk

\item{\bf 31.} Prove the Snake Lemma: given a diagram of abelian groups

\ssk

$$\CD
0 @>>> A @>>> B @>>> C @>>> 0 \\
 @. @V\alpha VV @V\beta VV @V\gamma VV \\
0 @>>> D @>>> E @>>> F @>>> 0 \\
\endCD
$$

\ssk

\item{} with the horizontal rows exact and where each rectangle commutes,
then there are induced maps and a connecting homomorphism making the
sequence 

\ssk

\ctln{$0\ra \ker\alpha \ra \ker\beta \ra \ker\gamma \ra \coker\alpha \ra \coker\beta \ra \coker\gamma \ra 0$}
 
\ssk

\item{} exact. (For $f:R\ra S$, $\coker f = S/\im(f)$.)

\msk

\item{\bf 32.} Compute the singular homology groups of the 
topologist's sine curve

\ssk

\ctln{$X=\{(x,\sin(1/x) : 0<x\leq 1 \}\cup (\{0\}\times[-1,1])$}


\vfill
\end

Find examples of spaces and subspaces
$A_0\subseteq X_0$ and $A_1\subseteq X_1$ so that
$H_*(X_0)\cong H_*(X_1)$ and $H_*(A_0)\cong H_*(A_1)$,
but $H_*(X_0,A_0)\not\cong H_*(X_1,A_1)$ .



If $A\subseteq X$ and $I:X\ra X$ is homotopic to a map $f:X\ra X$ with $f(X)\subseteq A$, then
for every $n$, $H_n(A)\cong H_n(X)\oplus H_{n+1}(X,A)$.



\ctln{\vbox{\hsize=4in
\leavevmode
\epsfxsize=4in
\epsfbox{hw6.ai}}}


