

\magnification=1200
\overfullrule=0pt
\parindent=0pt
\nopagenumbers

\input amstex

\input epsf
\loadmsbm

\def\ctln{\centerline}
\def\u{\underbar}
\def\ssk{\smallskip}
\def\msk{\medskip}
\def\bsk{\bigskip}
\def\hsk{\hskip.1in}
\def\hhsk{\hskip.2in}

\def\dsl{\displaystyle}
\def\hskp{\hskip1.5in}
\def\ra{\rightarrow}
\def\lra{$\Leftrightarrow$}
\def\pu{\pi_1}
\def\mpu{$\pi_1$}
\def\bra{$\Rightarrow$}
\def\bbr{{\Bbb R}}
\def\bbz{{\Bbb Z}}
\def\bbq{{\Bbb Q}}
\def\del{\partial}
\def\indt{\item{}}
\def\wtl{\widetilde}
\def\coker{\text{coker}}
\def\im{\text{im}}
\def\wtih{\widetilde{H}}

\ctln{\bf Math 872 Algebraic Topology}

\ssk

\ctln{Problem Set \#\ 8}

\ssk

\ctln{Starred (*) problems due Thursday, April 13}

\bsk

\item{\bf 39.} Giving the $n$-simplex $X=\Delta^n$ its standard $\Delta$-complex
structure, show that the $k$-skeleton of $X$ has homology

\ssk

\ctln{
$\wtih_i(X^{(k)})=
\cases 
\bbz^{r(k,n)} & \text{if}\ i=k\cr
0 & \text{otherwise} \cr
\endcases$}

\ssk

\item{} where $r(k,n)={{n}\choose{k+1}}$. 

\ssk

\item{} (Hint: our computations preliminary to 
the proof of ``cellular = singular'' will help.)

\msk

\item{\bf (*) 40.} Show that if the sequence

\ssk

\ctln{$0\ra C_n\ra C_{n-1}\ra \cdots \ra C_1\ra C_0\ra 0$}

\ssk

\item{} is exact, then $\sum (-1)^i{\text rank}\ C_i=0$.
(Hint: pretend it's a chain complex...)

\msk

\item{\bf (*) 41.} Show that if $\{{\Cal U},{\Cal V}\}$ is an open cover of $X$ and 
${\Cal U}\cap{\Cal V},{\Cal U},{\Cal V}$ and $X$ all have $\oplus_i H_i(\text{blah})$ of finite 
rank, then $\chi(X)=\chi({\Cal U})+\chi({\Cal V})-\chi({\Cal U}\cap{\Cal V})$.

\msk

\item{\bf 42.} Show that if $\widetilde{X}$ is an $n$-sheeted covering space of the finite
CW-complex $X$, then 

\item{} $\chi(\widetilde{X})=n\chi(X)$. Conclude that the only non-trivial finite
group that can act on an even-dimensional sphere $S^{2k}$ without fixed points is
$\bbz_2$. (Skip the hard part: the quotient by the group action is a CW-complex...)

\msk

\item{\bf 43.} Show that $H_i(X\times S^n)\cong H_i(X)\oplus H_{i-n}(X)$ for every $i$;
here $H_i(X)=0$ if $i<0$.

\ssk

\item{} [One approach: show that $H_i(X\times S^n)\cong H_i(X)\oplus H_i(X\times S^n,X\times D^n_+)$ 
(Problem \#30 will help), and that $H_i(X\times S^n,X\times D^n_+)\cong H_{i-1}(X\times S^{n-1},X\times D^{n-1}_+)$
by excision and the long exact sequence of the triple $(X\times D^{n-1}_+,X\times S^{n-1},X\times D^{n}_-)$.

\ssk

\item{} Hatcher, p.158, \# 36 gives a different approach.]

\vfill
\end









\ctln{\vbox{\hsize=4in
\leavevmode
\epsfxsize=4in
\epsfbox{hw6.ai}}}


