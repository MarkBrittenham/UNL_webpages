\magnification=1200
\overfullrule=0pt
\parindent=0pt

%\nopagenumbers

\input amstex

%\voffset=-.6in
%\hoffset=-.5in
%\hsize = 7.5 true in
%\vsize=10.4 true in

%\voffset=1.4in
%\hoffset=-.5in
%\hsize = 10.2 true in
%\vsize=8 true in

\input colordvi

\def\cltr{\Red}		  % Red  VERY-Approx PANTONE RED
\def\cltb{\Blue}		  % Blue  Approximate PANTONE BLUE-072
\def\cltg{\PineGreen}	  % ForestGreen  Approximate PANTONE 349
\def\cltp{\DarkOrchid}	  % DarkOrchid  No PANTONE match
\def\clto{\Orange}	  % Orange  Approximate PANTONE ORANGE-021
\def\cltpk{\CarnationPink}	  % CarnationPink  Approximate PANTONE 218
\def\clts{\Salmon}	  % Salmon  Approximate PANTONE 183
\def\cltbb{\TealBlue}	  % TealBlue  Approximate PANTONE 3145
\def\cltrp{\RoyalPurple}	  % RoyalPurple  Approximate PANTONE 267
\def\cltp{\Purple}	  % Purple  Approximate PANTONE PURPLE

\def\cgy{\GreenYellow}     % GreenYellow  Approximate PANTONE 388
\def\cyy{\Yellow}	  % Yellow  Approximate PANTONE YELLOW
\def\cgo{\Goldenrod}	  % Goldenrod  Approximate PANTONE 109
\def\cda{\Dandelion}	  % Dandelion  Approximate PANTONE 123
\def\capr{\Apricot}	  % Apricot  Approximate PANTONE 1565
\def\cpe{\Peach}		  % Peach  Approximate PANTONE 164
\def\cme{\Melon}		  % Melon  Approximate PANTONE 177
\def\cyo{\YellowOrange}	  % YellowOrange  Approximate PANTONE 130
\def\coo{\Orange}	  % Orange  Approximate PANTONE ORANGE-021
\def\cbo{\BurntOrange}	  % BurntOrange  Approximate PANTONE 388
\def\cbs{\Bittersweet}	  % Bittersweet  Approximate PANTONE 167
%\def\creo{\RedOrange}	  % RedOrange  Approximate PANTONE 179
\def\cma{\Mahogany}	  % Mahogany  Approximate PANTONE 484
\def\cmr{\Maroon}	  % Maroon  Approximate PANTONE 201
\def\cbr{\BrickRed}	  % BrickRed  Approximate PANTONE 1805
\def\crr{\Red}		  % Red  VERY-Approx PANTONE RED
\def\cor{\OrangeRed}	  % OrangeRed  No PANTONE match
\def\paru{\RubineRed}	  % RubineRed  Approximate PANTONE RUBINE-RED
\def\cwi{\WildStrawberry}  % WildStrawberry  Approximate PANTONE 206
\def\csa{\Salmon}	  % Salmon  Approximate PANTONE 183
\def\ccp{\CarnationPink}	  % CarnationPink  Approximate PANTONE 218
\def\cmag{\Magenta}	  % Magenta  Approximate PANTONE PROCESS-MAGENTA
\def\cvr{\VioletRed}	  % VioletRed  Approximate PANTONE 219
\def\parh{\Rhodamine}	  % Rhodamine  Approximate PANTONE RHODAMINE-RED
\def\cmu{\Mulberry}	  % Mulberry  Approximate PANTONE 241
\def\parv{\RedViolet}	  % RedViolet  Approximate PANTONE 234
\def\cfu{\Fuchsia}	  % Fuchsia  Approximate PANTONE 248
\def\cla{\Lavender}	  % Lavender  Approximate PANTONE 223
\def\cth{\Thistle}	  % Thistle  Approximate PANTONE 245
\def\corc{\Orchid}	  % Orchid  Approximate PANTONE 252
\def\cdo{\DarkOrchid}	  % DarkOrchid  No PANTONE match
\def\cpu{\Purple}	  % Purple  Approximate PANTONE PURPLE
\def\cpl{\Plum}		  % Plum  VERY-Approx PANTONE 518
\def\cvi{\Violet}	  % Violet  Approximate PANTONE VIOLET
\def\clrp{\RoyalPurple}	  % RoyalPurple  Approximate PANTONE 267
\def\cbv{\BlueViolet}	  % BlueViolet  Approximate PANTONE 2755
\def\cpe{\Periwinkle}	  % Periwinkle  Approximate PANTONE 2715
\def\ccb{\CadetBlue}	  % CadetBlue  Approximate PANTONE (534+535)/2
\def\cco{\CornflowerBlue}  % CornflowerBlue  Approximate PANTONE 292
\def\cmb{\MidnightBlue}	  % MidnightBlue  Approximate PANTONE 302
\def\cnb{\NavyBlue}	  % NavyBlue  Approximate PANTONE 293
\def\crb{\RoyalBlue}	  % RoyalBlue  No PANTONE match
%\def\cbb{\Blue}		  % Blue  Approximate PANTONE BLUE-072
\def\cce{\Cerulean}	  % Cerulean  Approximate PANTONE 3005
\def\ccy{\Cyan}		  % Cyan  Approximate PANTONE PROCESS-CYAN
\def\cpb{\ProcessBlue}	  % ProcessBlue  Approximate PANTONE PROCESS-BLUE
\def\csb{\SkyBlue}	  % SkyBlue  Approximate PANTONE 2985
\def\ctu{\Turquoise}	  % Turquoise  Approximate PANTONE (312+313)/2
\def\ctb{\TealBlue}	  % TealBlue  Approximate PANTONE 3145
\def\caq{\Aquamarine}	  % Aquamarine  Approximate PANTONE 3135
\def\cbg{\BlueGreen}	  % BlueGreen  Approximate PANTONE 320
\def\cem{\Emerald}	  % Emerald  No PANTONE match
%\def\cjg{\JungleGreen}	  % JungleGreen  Approximate PANTONE 328
\def\csg{\SeaGreen}	  % SeaGreen  Approximate PANTONE 3268
\def\cgg{\Green}	  % Green  VERY-Approx PANTONE GREEN
\def\cfg{\ForestGreen}	  % ForestGreen  Approximate PANTONE 349
\def\cpg{\PineGreen}	  % PineGreen  Approximate PANTONE 323
\def\clg{\LimeGreen}	  % LimeGreen  No PANTONE match
\def\cyg{\YellowGreen}	  % YellowGreen  Approximate PANTONE 375
\def\cspg{\SpringGreen}	  % SpringGreen  Approximate PANTONE 381
\def\cog{\OliveGreen}	  % OliveGreen  Approximate PANTONE 582
\def\pars{\RawSienna}	  % RawSienna  Approximate PANTONE 154
\def\cse{\Sepia}		  % Sepia  Approximate PANTONE 161
\def\cbr{\Brown}		  % Brown  Approximate PANTONE 1615
\def\cta{\Tan}		  % Tan  No PANTONE match
\def\cgr{\Gray}		  % Gray  Approximate PANTONE COOL-GRAY-8
\def\cbl{\Black}		  % Black  Approximate PANTONE PROCESS-BLACK
\def\cwh{\White}		  % White  No PANTONE match


\loadmsbm

\input epsf

\def\ctln{\centerline}
\def\u{\underbar}
\def\ssk{\smallskip}
\def\msk{\medskip}
\def\bsk{\bigskip}
\def\hsk{\hskip.1in}
\def\hhsk{\hskip.2in}
\def\dsl{\displaystyle}
\def\hskp{\hskip1.5in}

\def\lra{$\Leftrightarrow$ }
\def\ra{\rightarrow}
\def\mpto{\logmapsto}
\def\pu{\pi_1}
\def\mpu{$\pi_1$}
\def\sig{\Sigma}
\def\msig{$\Sigma$}
\def\ep{\epsilon}
\def\sset{\subseteq}
\def\del{\partial}
\def\inv{^{-1}}
\def\wtl{\widetilde}
%\def\lra{\Leftrightarrow}
\def\del{\partial}
\def\delp{\partial^\prime}
\def\delpp{\partial^{\prime\prime}}
\def\sgn{{\roman{sgn}}}
\def\wtih{\widetilde{H}}
\def\bbz{{\Bbb Z}}
\def\bbr{{\Bbb R}}



\ctln{\bf Math 872 \hsk Algebraic Topology}

\ssk

\ctln{Running lecture notes}

\bsk

{\bf Homology theory:} Fundamental groups are a remarkably powerful
tool for studying spaces; they capture a great deal of the global
structure of a space, and so they are very good a distinguishing
between homotopy-inequivalent spaces. In theory! But in practice,
they suffer from the fact that deciding whether two groups are 
isomorphic or not is, in general, undecideable! Homology theory
is designed to get around this deficiency; the theory, by design,
builds (a sequence of) {\it abelian} groups $H_i(X)$ from a topological
space. And deciding whether or not two \u{abelian} groups, at least
if you're given a presentation for them, is, in the end, a matter of
fairly routine linear algebra. Mostly because of the Fundamental Theorem
of Finitely-generated Abelian groups; each such has a unique representation
as ${\Bbb Z}^m\oplus{\Bbb Z}_{m_1}\oplus\cdots\oplus{\Bbb Z}_{m_n}$
with $m_{i+1}|m_i$ for every $i$ .

\msk

There are also ``higher'' homotopy groups beyond the fundamental group \mpu ,
(hence the name pi-{\it one}); elements are homtopy classes, rel boundary, 
of based maps $(I^n,\del I^n)\ra(X,x_0)$. Multiplication is again by
concatenation. But unlike \mpu , where we have a chance to compute it
via Seifert-van Kampen, nobody, for example knows what all of the 
homotopy groups $\pi_n(S^2$ are (except that nearly all of them are
non-trivial!). Like \mpu, it describes, essentially, maps of $S^n$ into
$X$ which don't extend to maps of $D^{n+1}$, i.e., it turns the ``$n$-dimensional
holes'' of $X$ into a group.

\msk

Homology theory does the exact same thing, counting $n$-dimensional holes.
In the end we will find it to be extremely computable; but it will require
building a fair bit of machinery before it will become so transparent to
calculate. But the short version is that the homology groups compute
``cycles mod boundaries'', that is, $n$-dimesional objects/subsets that
have no boundary (in the appropriate sense) modulo objects that are the
boundary of $(n+1)$-dimensional ones. There are, in fact, probably as many
ways to {\it define} homology groups as there are people actively working
in the field; we will focus on two, simplicial homology and singular homology.
The first is quick to define and compute, but hard to show is an invariant!
The second is quick to see is an invariant, but, on the face of it, hard
to compute! Luckily, for spaces where they are both defined, they are
isomorphic. So, in the end, we get an invariant that is quick to compute.
Of course, so is the invariant ``4''; but this one will be a bit more
informative....

\msk

First, simplicial homology. This is a sequence of groups defined for spaces
for which they are easiest to define, which Hatcher calls $\Delta$-complexes.
Basically, they are spaces defined by gluing simplices together using
nice enough maps. More precisely, the {\it standard $n$-simplex} $\Delta^n$ is
the set of points 
$\{(x_1,\ldots x_{n+1})\in{\Bbb R}^{n+1}$ : $\sum x_i=1 , x_i\geq 0$
 for all $i\}$. This can also be expressed as convex linear combinations
(literally, that's the conditions on the $x_i$'s) of the points
$e_i=(0,\ldots ,0,1,0,\ldots ,0)$, the {\it vertices} of the standard
simplex. More generally, an $n$-simplex is the set $[v_0,\ldots v_n]$ of
convex linear combinations of points $v_0,\ldots ,v_n\in{\Bbb R}^{k}$
for which $v_1-v_0,\ldots ,v_n-v_0$ are linearly independent.
Any bijection from the vertices of the standard simplex to the points
$v_0,\ldots ,v_n$ extends (linearly) to a homeomorphism of
the simplices. The $n+1$ {\it faces} of a simplex, each sitting opposite
a vertex $v_i$, are obtained by setting the corresponding coefficient $x_i$ to $0$. 
Each forms an $(n-1)$-simplex, which we denote 
$[v_0,\ldots,v_{i-1},v_{i+1},\ldots ,v_n]$ or
$[v_0,\ldots,\widehat{v_{i}},\ldots ,v_n]$ . A {\it $\Delta$-complex} $X$ is a cell
complex obtained by gluing simplices together, but we insist on an extra
condition:
the restriction of the attaching map to any face is equal to a (lower-dimensional)
cell. As before, we use the weak topology on the space; a set is open iff
it's inverse image under the induced map of a cell into the complex is open.
Each $n$-cell comes equipped with a (continuous) map
$\sigma:\Delta^n\ra X$, which is one-to-one on its interior, whose restriction
to the boundary is the attaching map, and whose restriction to each face is the
associated map for that $(n-1)$-simplex. We will typically blur the 
distinction between the map $\sigma$ (called the {\it characteristic map}
of the simplex) and its image, and denote the
image by $\sigma$ (or $\sigma^n$), when this will cause no confusion,
and call $\sigma$ an $n$-simplex {\it in} $X$. When we feel the need for the 
distinction, we will use $e^n$ for the image and $\sigma^n$ for the map.

\ssk

For example, taking our standard,
identifications of the sides of a rectangle, cell structure for the 2-torus,
and cutting the rectangle into two triangles (= 2-simplices) along a diagonal,
we obtain a $\Delta$-structure with 2 2-simplices, 3 1-simplices, and 1 0-simplex.
A genus $g$ surface can be built, by cutting the $2g$-gon into triangles, with
$g+1$ 2-simplices, $3g$ 1-simplices, and 1 0-simplex.

\bsk


We typically think of building a $\Delta$-complex $X$ inductively. 
The {\it 0-simplices}
(i.e., points), or {\it vertices}, form the 0-skeleton 
$X^{(0)}$. $n$-simplices $\sigma^n = [v_0,\ldots v_n]$ attach 
to the $(n-1)$-skeleton
to form the $n$-skeleton $X^{(n)}$; the restriction
of the attaching map to each face of $\sigma^n$ is, by definition,
an $(n-1)$-simplex in $X$. The attaching map is (by induction)
really determined by a map $\{v_0,\ldots ,v_n\}\ra X^{(0)}$, since this 
determines the attaching maps for the 1-simplices in the boundary of the
$n$-simplex, which 
gives 1-simplices in $X$, which then give the attaching maps for
the 2-simplices in the boundary, etc. Note that the reverse is not true;
the vertices of two different $n$-simplices in $X$ can be the same.
For example, think of the 2-sphere as a pair of 2-simplices whose 
boundaries are glued by the identity. $\Delta$-complexes generalize
{\it simplicial complexes} where the simplices are required to attach
by homeomorphisms to the skeleton, and the intersection of two 
simplices are a (single) sub-simplex of each. This has the advantage
over $\Delta$-complexes that an $n$-simplex is determined uniquely
by the set of vertices in $X^{(0)}$ that it attaches to. This means that,
in principle, a simplicial complex (and everything associated with it, e.g., 
its homology groups!) can be treated purely combinatorially;
the complex is ``really'' a certain collection of subsets of the vertices
(since these determine the simplices), with the property that any subset $B$
of a subset $A$ that has been declared to be a simplex is also a simplex.
But they have the disadvantage that it typically takes far more
simplices to build a simplicial structure on a space $X$ that it does to build a 
$\Delta$-structure. This makes the computations we are about to do take 
far longer.

\msk

The final detail that we need before defining (simplicial) homology
groups is the notion of an {\it orientation} on a simplex of $X$.
Each simplex $\sigma^n$ is determined by a map 
$f:\{v_0,\ldots ,v_n\}\ra X^{(0)}$; an orientation on $\sigma^n$ is an
(equivalence class of) the ordered $(n+1)$-tuple $(f(v_0),\ldots f(v_n)) = (V_0,\ldots ,V_n)$.
Another ordering of the
same vertices represents the same orientation if there is an {\it even} permutation
taking the entries of the first $(n+1)$-tuple to the second. This should be thought 
of as a generalization of the right-hand rule for ${\Bbb R}^3$, interpreted as
orienting the vertices of a 3-simplex. Note that there are precisely two
orientations on a simplex.

\msk

Now to define homology! We start by defining $n$-{\it chains};
these are (finite) formal linear combinations of the (oriented!) $n$-simplices
of $X$, where $-\sigma$ is interpreted as $\sigma$ with the opposite
(i.e., other) orientation. Adding formal linear combinations formally,
we get the $n$-th {\it chain group} 
$C_n(X) = \{\sum n_\alpha \sigma_\alpha$ : $\sigma_\alpha$ an oriented $n$-simplex in $X\}$ .
We next define a {\it boundary operator} $\del:C_n(X)\ra C_{n-1}(X)$, whose image will be 
the $(n-1)$-chains that are the ``boundaries'' of $n$-chains. The idea is that the boundary of a 2-simplex,
for example, should be a ``sum'' of its three faces (since they do make up the boundary of the 
simplex), but we need to take into account their orientations, in order to be getting the correct
sum. Thinking of the orientation on a 1-simplex $[v,w]$ as an arrow pointing from $v$ to $w$, we are lead to 
believe that the boundary of a 2-simplex $[u,v,w]$ should be $[u,v]+[v,w]+[w,u]$. Similarly, the boundary 
of $[u,v]$, on reflection, should be $[v]-[u]$, to distinguish the head of the arrow (the $+$ side) from the
tail (the $-$ side). On the basis of these examples, trying to find a consistent formula, one might eventually
be led to the following formulation.
We define the boundary on the basis
elements $\sigma_\alpha = \sigma$ of $C_n(X)$ as
$\del\sigma = \sum (-1)^i\sigma|_{[v_0,\ldots,\widehat{v_{i}},\ldots ,v_n]}$ , 
where $\sigma:[v_0,\ldots ,v_n]\ra X$ is the characteristic map of $\sigma_\alpha$ .
$\del\sigma$ is therefore an alternating sum of the faces of $\sigma$. 
We then extend the definition by linearity to all of $C_n(X)$. When a notation indicating
dimension is needed, we write $\del=\del_n$ . We define $\del_0=0$. 

\msk

This definition, it utrns out, is cooked up to make the maxim ``boundaries have no boundary'' true;
that is, $\delta_{n-1}\circ \delta_n = 0$, the $0$ map. This is because, for any simplex
$\sigma = [v_0,\ldots v_n]$, 

\msk

$\displaystyle \delta\circ\delta(\sigma) = 
\delta(\sum_{i=0}^n  (-1)^i\sigma|_{[v_0,\ldots,\widehat{v_{i}},\ldots ,v_n]})$

= $\displaystyle (\sum_{j<i}(-1)^j(-1)^i\sigma|_{[v_0,\ldots,\widehat{v_{j}},\ldots,\widehat{v_{i}},\ldots ,v_n]})
+(\sum_{j>i}(-1)^{j-1}(-1)^i\sigma|_{[v_0,\ldots,\widehat{v_{i}},\ldots,\widehat{v_{j}},\ldots ,v_n]})$

\msk

The distinction between the two pieces is that in the second part, $v_j$ is actually the $(j-1)$-st vertex
of the face. Switching the roles of $i$ and $j$ in the second sum, we find that the two are
negatives of one another, so they sum to $0$, as desired.

\msk

And this little calculation is all that it takes to define homology groups! What this tells 
us is that im$(\delta_{n+1})\subseteq\ker(\delta_{n}$ for every $n$. 
$\ker(\delta_{n}=Z_n(X)$ are called the {\it $n$-cycles} of $X$; they are the $n$-chains with
$0$ (i.e., empty) boundary. They form a (free) abelian subgroup of $C_n(X)$. 
im$(\delta_{n+1} = B_n(X)$ are the {\it $n$-boundaries} of $X$; they are, of course,
the boundaries of $(n+1)$-chains in $X$. The $n$-th homology group of $X$,
$H_n(X)$ is the quotient $Z_n(X)/B_n(X)$ ; it is an abelian group.

\msk

A key observation is that the boundary maps $\delta_n$ are linear, that is,
they are homomorphisms between the free abelian groups 
$\delta_n:C_n(X)\ra C_{n-1}(X)$. Consequently, they can be expressed as
(integer-valued) matrices $\Delta_n$. Row reducing $\Delta_n$ 
(over the integers!) allows us to find a 
basis $v_1,\ldots ,v_k$ for $Z_n(X)$ (clearing denomenators
to get vectors over ${\Bbb Z}$). Then since $\Delta_n\Delta_{n+1}=0$, the
columns of $\Delta_{n+1}$ are in the kernel of $\Delta_n$, so can be
expressed as linear combinations of the $v_i$ . These combinations can be
determined by row reducing the augmented matrix
$( v_1\cdots v_k | \Delta_{n+1} )$ . This will row reduce to 
$\pmatrix I&|&C\\ 0&|&0\\ \endpmatrix$, and $C$ basically describes the boundaries
$B_n(X)$ in terms of the basis $v_1,\ldots ,v_k$ . The homology group
$H_n(X)$ is then the {\it cokernel} of $C$, i.e., ${\Bbb Z}^k/$im$C$ .
Note that $C$ will have integer entries, since we know that 
the columns of $\Delta_{n+1}$ can be expressed as integer linear
combinations of the $v_i$, and, being a basis, there is only
one such expression. 

\bsk

{\bf Some examples:} the Klein bottle $K$ has a $\Delta$-complex structure with 2 2-simplices,
3 1-simplices, and 1 0-simplex; we will call them 
$f_1=[0,1,2],f_2=[1,2,3],
e_1=[0,2]=[1,3],e_2=[1,0]=[2,3],e_3=[1,2]$, 
and $v_1=[0]=[1]=[2]=[3]$.
Computing, we find 
$\del_2 f_1 = \del[0,1,2]=[1,2]-[0,2]+[0,1]=e_3-e_1-e_2$ , $\del_2 f_2 = e_2-e_1+e_3$ , 
$\del_1 e_1 = \del_1 e_2 = \del_1 e_3 = 0$ and $\del_i = 0$ for all other $i$
(as well). So we have the chain complex

$\cdots \ra 0 \ra {\Bbb Z}^2 \ra {\Bbb Z}^3 \ra {\Bbb Z} \ra 0$

and all of the boundary maps are 0, except for $\del_2$, which has the matrix
$\pmatrix
 -1&-1\\ -1&1\\ 1&1\\
\endpmatrix$ . This matrix is injective, so $\ker \del_2 = 0$,
so $H_2(K)=0$, on the other hand, $H_1(K)$ = coker$(\del_2)$, and applying column
operations we can transform the matrix for $\del_2$ to $\pmatrix 1&0\\ 1&2\\ -1&0\\ \endpmatrix$,
which implies that the cokernel is ${\Bbb Z}\oplus{\Bbb Z}_2$, since 
$\pmatrix 1\\ 1\\ -1\\ \endpmatrix ,\pmatrix 0\\ 1\cr 0\\ \endpmatrix , \pmatrix 0\cr 0\\ 1\\ \endpmatrix$
is a basis for ${\Bbb Z}^3$. Finally, $H_0(K)={\Bbb Z}$, since $\del_1,\del_0=0$,
and all higher homology groups are also $0$, for the same reason.

\msk

As another example, the topologist's dunce hat has a $\Delta$-structure with
1 2-simplex, 1 1-simplex, and 1 0-simplex. The boundary maps, we can work out
(starting from $C_2(X)$ ), are $(1),(0)$, and $(0)$, so $H_2(X)=H_1(X)=0$,
and $H_0(X)={\Bbb Z}$. all higher groups are also $0$.

\msk

These homology groups are, in the end, fairly routine to calculate from a 
$\Delta$-complex structure. But there is one very large problem; the calculations
\u{depend} on the $\Delta$ structure! This is not a group defined from the space
$X$; it is defined from the space and a $\Delta$ structure on it. A priori, we don't
know that if we chose a different structure on the same space, that we would get
isomorphic groups! We should really denote our groups by $H_i^\Delta(X)$, to 
acknowledge this dependence on the structure.

\bsk

But we don't {\it want} a group that depends on this structure. We want groups that
just depend on the topological space $X$, i.e., which are topological invariants.
In really turns out that these groups $H_i^\Delta(X)$ \u{are} topological invariants,
but we will need to take a very roundabout route to show this. What we will do
now is to define another sequence $H_i(X)$ of groups, the {\it singular homology
groups}, which their definition makes apparent from the outset 
that they are topological invariants.
But this definition will also make it very unclear how to really compute them! 
Then we will show that for $\Delta$-complexes these two sequences of groups
are really the same. In so doing, we will have built a sequence of topological
invariants that for a large class of spaces are fairly routine to compute. Then
all we will need to show is that they also capture useful information about
a space (i.e., we can prove useful theorems with them!).

\msk

And the basic idea behind defining them is that, with simplicial homology,
we have already done all of the hard work. What we do is, as before, build a 
sequence of (free) abelian groups, the chain groups $C_n(X)$, 
and boundary maps between them,
with consecutive maps composing to 0. Then, as before, the homology groups are
kernels mod images, i.e., cycles mod boundaries. And, as before, the basis
elements for each of our chain groups $C_n(X)$ will be the $n$-simplices
in $X$. But now $X$ is \u{any} topological space. So how do we get $n$-simplices
in such a space? We do the only thing we can; we {\it map} them in. 

\msk

More precisely, we work with {\it singular $n$-chains}, that is,
formal (finite) linear combinations $\sum a_i\sigma_i$, where $a_i\in{Bbb Z}$
and the $\sigma_i$ are {\it singular simplices}, that is, (continuous) 
maps $\sigma_i:\Delta^n\ra X$ from the (standard) $n$-simplex into $X$.
The boundary maps are really exactly as before; they are the alternating sum of
the restrictions of $\sigma_i$ to the $n+1$ faces of $\Delta^n$ . (Formally,
we must precompose these face maps with the (orientation-preserving) linear
isomorphism from the standard $(n-1)$-simplex to each of the faces, preserving
the ordering of their vertices.) The same proof as before (except that we interpret
the faces as restrictions of the map $\sigma_i$, instead of as physical faces)
shows that the composition of two successive boundaries are 0,
and so all of the machinery is in place to define the {\it singular homology
groups} $H_i(X)$ as the kernel of $\del_i$ modulo the image of $\del_{i+1}$ = $Z_i(X)/B_i(X)$ .
They are, by their definition, groups defined using the topological space $X$ as input,
and so are topological invariants of $X$. The elements are equivalence classes of $i$-cycles,
where $z_1\simeq z_2$ if $z_1-z_2=\del w$ for some $(i+1)$-chain $w$ . We say that $z_1$ and $z_2$ are
{\it homologous}.

\bsk


Singular homology groups are very quick to define, but what do they measure?
The basic idea is that we are trying to mimic simplicial homology, but because a
general topological space $X$ cannot be thought of as being built out of simplices,
we do the next best thing; we study the space by \u{mapping} simplices \u{in}. Formally,
this is what we did with simplicial homology anyway, except that we restricted 
ourselves to a very few special singular simplices (the characteristic maps of the
building blocks for $X$). In the end an $n$-cycle $\sum a_i\sigma_i^n$, since the 
faces of the $\sigma_i$ must match up precisely, in order to cancel in the sum, 
can be thought of as a map of an $n$-complex into $X$, made by gluing the
$n$-simplices $\sigma_i$ together \u{before} mapping in. The fact that faces
cancel really means that these simplices restrict to the same maps on their faces.
The integer coefficients can really be interpreted as taking multiple copies of 
$\Delta^n$ and gluing them together along their boundaries (the signs tell us
the underlying orientations). The idea is that this $n$-complex is being
mapped ``around a hole'', unless it \u{extends} to a map of an $(n+1)$-complex
into $X$ (having our $n$-complex as boundary). So singular homology really
is trying to detect holes, it is just doing it with maps.....

\msk

The ``fun'' with singular homology groups, though, comes when you try to compute them. 
$C_n(X) = \{\sum a_i\sigma_i$ : $a_i\in {\Bbb Z}$ and $\sigma_i:\Delta^n\ra X$ 
is continuous$\}$ is typically a \u{huge} group, since there will be immense
numbers of maps $\Delta^n\ra X$ . About the only space for which this is not true is
the one-point space $*$; then there is, for each $n$, exactly one (distinct)
map $\sigma_n :\Delta^n\ra *$ ; the constant map. Therefore each face of $\Delta^n$
gives the same restriction map $\sigma^{n-1}$, and so the boundary maps can 
be dirctly computed (the depend on the parity of the number $n+1$ of faces 
an $n$-simplex has). We find that $\del_{2n}=Id$ and $\del_{2n-1}=0$ . so in 
computing homology groups, we either have kernel everything ($\del_i=0$) and
image everything ($\del_{i+1}=Id$) or kernel nothing ($\del_i=Id$) and
image nothing ($\del_{i+1}=0$), so in both cases $H_i(*)=0$ . Except for $i=0$;
then $\del_0=0$ (by definition) and $\del_1=0$, so $H_0(*)={\Bbb Z}$ .
But other than this example (and, well, OK, spaces with the discrete topology;
it's the same calculation as above for every point!), computing singular 
homology from the definition is quite a chore! so we need to build
some labor-saving devices, namely, some theorems to help us break the problem
of computing these groups into smaller, more managable pieces.

\msk

First set of managable pieces: if we decompose $X$ into its path components, $X=\bigcup X_\alpha$,
then $H_i(X) \cong \bigoplus H_i(X_\alpha)$ for every $i$. This is because every singular simplex,
since $\Delta^i$ is path-connected, maps into some $X_\alpha$ . So $C_i(X) \cong \bigoplus C_i(X_\alpha)$.
Since the boundary of a simplex mapping into $X_\alpha$ consists of simplices in $X_\alpha$, the 
boundary maps respect the decomposistions of the chain groups, so 
$B_i(X) \cong \bigoplus B_i(X_\alpha)$ and $Z_i(X) \cong \bigoplus Z_i(X_\alpha)$, and so 
the quotients are $H_i(X) \cong \bigoplus H_i(X_\alpha)$ . 

\msk

So, if we wish to, we can focus on computing homology groups for path-connected spaces $X$. For such a space, 
$H_0(X)\cong {\Bbb Z}$, generated by any map of a 0-simplex (= a point) into $X$. This is because any pair
of 0-simplices are homologous; given any two points $x,y\in X$, there is a path $\gamma: I\ra X$ from $x$ to $y$,
This path can be interpreted as a singular 1-simplex, and $\del\gamma = y-x$ . So $H_0(X)$ is generated
by a single point $[x]$ . No multiple of this point is null-homologous, because for any 1-chain $\sum n_i \sigma_i$,
the sum of the coefficients of its boundary is 0 (since this is true for each singular 1-simplex), and any 0-chain
$\sum n_i [x_i]$ is homologous to $(\sum n_i)[x]$ by the above argument.

\msk

A small techincal aside: the fact that $H_0(*)={\Bbb Z}$ is annoying to some,
and often requires treating 0-dimensional homology as a special case. 
But since the boundary of a singular 1-simplex is always of the form $v-w$, we find that the 
image of $\del_1$ is always contained in the subgroup of $C_0(X)$ consisting
of chains whose coefficients sum to 0. This means that we can, for free, 
{\it augment} the singular chain complex by a map
$\cdots \ra C_1(X) {\del_1\atop\ra}C_0)X) {\alpha\atop \ra} {\Bbb Z} \ra 0$
where $\alpha$ is the map $\alpha(\sum a_i\sigma_i^0) = \sum a_i$ . This 
is still a chain complex (compositions of consecutive maps are 0); the resulting
homology groups are called {\it reduced} homology $\widetilde{H}_i(X)$ . 
The only affect this really has is to remove one copy of ${\Bbb Z}$ from 
$H_0$; $\widetilde{H}_0(X)\oplus {\Bbb Z} \cong H_0(X)$ . All other
homology groups are unchanged. There is a reduced relative homology 
as well, since we can augment with the same map (1-simplices always have 2 ends!),
but in this case it has (essentially) no effect; $\widetilde{H}_i(X,A)\cong H_i(X,A)$
for all $i$ \u{unless} $A=\emptyset$, in which case we lose the ${\Bbb Z}$ in
dimension 0 that we expect to. 

\bsk


Perhaps the most important property of the fundamental group is that a continuouos map 
between spaces induces a homomorphism between groups. (This implied, for instance,
that homeomorphic spaces have isomorphic \mpu ). The same is true for homology groups, 
for essentially the same reason. Given a map $f:X\ra Y$, there is an induced map $f_\#:C_n(X)\ra C_n(Y)$
defined by postcomposition; for a singular simplex $\sigma$, $f_\#(\sigma) = f\circ\sigma$, and we extend
the map linearly. Since $f\circ(g|_A) = (f\circ g)|_A$ (postcomposition commutes with restriction of the domain),
$f_\#$ commutes with $\del$ : $f_\#(\del \sigma) = \del(f_\#(\sigma))$. A homomorphism between
chain complexes (i.e., a sequence of such maps, one for each chain group) which commutes with the 
boundaries maps in this way, is called a {\it chain map}.
A chain map, such as $f_\#$, therefore, takes cycles to cycles,
and boundaries to boundaries, and so $f_\#:Z_i(X)\ra Z_i(Y)$ (which is linear, hence a homomorphism)
induces a homomorphism $f_*:H_i(X)\ra H_i(Y)$ by $f_*[z] = [f_\#(z)]$ . 
Since it is defined by composition with singular simplices, it is 
immediate that, for a map $g:Y\ra Z$, $(g\circ f)_*=g_*\circ f_*$ . And since the identity map $I:X\ra X$
satisfies $I_\#=Id$, so $I_*=Id$, homeomorphic spaces have isomorphic homology groups.

\msk

Another important property of \mpu\ is that homotopic maps give the same
induced map (after correcting for basepoints). This is also true for homology;
if $f\simeq g:X\ra Y$, then $f_*=g_*$ . The proof, however, is not quite as straightforward
as for homotopy. And it requires some new technology; the chain homotopy.
A chain homotopy $H$ between the chain complexes $f_\#,g_\#:C_*(X)\ra C_*(Y)$ 
is a sequence of homomorphisms $H_i:C_i(X)\ra C_{i+1}(Y)$ satisfying
$H_{i-1}\del_i+\del_{i+1}H_i = f_\#-g_\#:C_i(X)\ra C_i(Y)$ . The existence of $H$
implies that $f_*=g_*$, since for an $i$-cycle $z$ (with $\del_i(z)=0$) we have

$f_*[z]-g_*[z] = [f_\#(z)-g_\#(z)] = [H_{i-1}\del_i(z)+\del_{i+1}H_i(z)] = [H_{i-1}(0)+\del_{i+1}(w)]
=[\del_{i+1}(w)]=0$.

And the existence of a homotopy between $f$ and $g$ implies the existence of a 
chain homotopy between $f_\#$ and $g_\#$ . This is because the homotopy 
gives a map $H:X\times I\ra Y$, which induces a map $H_\#:C_{i+1}(X\times I)\ra C_{i+1}(Y)$ .
Then we pull, from our back pocket, a {\it prism map}
$P:C_i(X)\ra C_{i+1}(X\times I)$; the composition $H_\#\circ P$ will be our chain homotopy.
The prism map  takes a (singular) $i$-simplex $\sigma$ and sends it to a sum of singular $(i+1)$-simplices
in $X\times I$. and the way we define it is to take the $i$-simplex $\Delta^i$, and taking it 
to $\Delta^i\times I$ (i.e., a {\it prism}), and thinking of this as a sum of $(i+1)$-simplices. Using the
map $\sigma^\prime = \sigma\times Id : \Delta^i\times I\ra X\times I$ 
restricted to each of these $(i+1)$-simplices
yields the prism map. Now, there are many ways of decomposing a prism into simplices,
but we need to be careful to choose one which restricts well to each of 
the \u{faces} of $\Delta^i$,
in order to get the chain homotopy property we require. In the end, what 
this requires is that the
decomposition, when restricted to any face of $\Delta^i$ (which we think of as a copy
of $\Delta^{i-1}$), is the same as the decomposition we would have applied to a prism over
an $(i-1)$-simplex. After some exploration, we are led to the following formulation.

\msk

If we write $\Delta^n\times\{0\}=[v_0,\ldots ,v_n]$ and 
$\Delta^n\times\{1\}=[w_0,\ldots ,w_n]$, then we can decompose
$\Delta^n\times I$ as the (n+1)-simplices $[v_0,\ldots ,v_i,w_i,\ldots ,w_n]$. 
We then define $P(\sigma) = \sum (-1)^i \sigma^\prime|_{[v_0,\ldots ,v_i,w_i,\ldots ,w_n]}$.
A routine calculation verifies that 
$(\del P+P\del)(\sigma) = \sigma^\prime|_{[w_0,\ldots ,w_n]}-\sigma^\prime|_{[v_0,\ldots v_n]}$ ;
Composing with $H_{\#}$ yields our result.

\msk

Consequently, for example, homotopy equivalent spaces have isomporphic
(reduced) homology groups; homotopy equivalences induce isomorphisms.
So all contractible spaces have trivial reduced homology
in all dimensions, since they are all homotopy to a point. If we think of a cell complex as
a collection of disks glued together, this lends some hope that we can compute
their homology groups, since we can compute the homology of the building blocks. 
Our next goal is to make turn this idea into action; but we need another tool, to
frame our answer in the best way possible.

\msk

{\bf Exact sequences:} Most of the fundamental properties of homology groups
are described in terms of exact sequences. A sequence of homomorphisms
\hhsk $\cdots {f_{n+1}\atop\ra} A_n {f_n\atop\ra} A_{n-1} {f_{n-1}\atop\ra} a_{n-2} 
\ra \cdots$ \hhsk
of abelian groups is called {\it exact} if im$(f_n)=\ker(f_{n-1})$ 
for every $n$. In most cases,
we get the most mileage out of an exact sequence when some of the groups
are trivial; $0\ra A {f\atop\ra}B$ is exact \lra\ $f$ is injective, and 
$A {f\atop\ra}B\ra 0$ is exact \lra\ $f$ is surjective. 
An exact sequence $0\ra A {\ra}B\ra C\ra 0$ is called a {\it short exact sequence}.

\bsk

The main tool we will use turns a family of short exact sequences of chain maps
between three chain complexes into a single {\it long exact homology sequence}.
Given chain complexes ${\Cal A}=(A_n,\del)$ , 
${\Cal B}=(B_n,\del^\prime)$ , and ${\Cal C}=(C_n,\del^{\prime\prime})$
and short exact sequences of chain maps (i.e., 
$\del^\prime i_n  = i_n\del $ , $\del^{\prime\prime}j_n = j_n\del^\prime $)
\hhsk

$0\ra A_n{i_n\atop \ra}B_n{j_n\atop \ra}C_n\ra 0$
\hhsk
there is a general result which provides us with a long exact sequence

\ctln{$\cdots {\del \atop \ra} H_n({\Cal A}) {i_{*}\atop \ra} H_n({\Cal B})
{j_{*}\atop \ra} H_n({\Cal C}) {\del\atop\ra} H_{n-1}({\Cal A}) {i_{*}\atop\ra} \cdots$}

Most of the work is in defining the ``boundary'' map $\del$. Given an 
element $[z]\in H_n({\Cal C})$, a representative $z\in C_n$ satisfies 
$\del^{\prime\prime}(z)=0$. But $j_n$ is onto, so there is a $b\in B_n$ with
$j_n(b)=z$, Then $ i_{n-1}\del^\prime(b) = \del^{\prime\prime}j_n(b)
=0$, so $\del^\prime(b)\in\ker(j_{n-1}=$im$(a_{n-1})$. So there is an $a\in A_{n-1}$
with $i_{n-1}(a)=\del^\prime(b)$ . But then 
$i_{n-2}\del (a) = \del^\prime i_{n-1}(a)=\del^\prime\del^\prime(b)=0$,
so, since $i_{n-2}$ is injective, $\del a=0$, so $a\in Z_{n-1}({\Cal A})$, and
so represents a homology class $[a]\in H_n({\Cal A})$. We define
$\del([z])=[a]$ . 


To show that this is well-defined, we need to show that the
class $[a]$ we end up with is independent of the choices made along the 
way. The choice of $a$ was not really a choice; $i_{n-1}$ is, by assumption, 
injective. For $b$, if $j_n(b)=z=j_n(b^\prime)$, then
$j_n(b-b^\prime)=0$, so $b-b^\prime=i_n(w)$ for some $w\in A_n$. Then 
$\del^\prime b^\prime = \del^\prime b - \del^\prime i_n(w) = 
\del^\prime b - i_{n-1} \del(w)$, so choosing $a^\prime = a-\del(w)$ we have
$i_{n-1}(a^\prime)=\del^\prime(b^\prime)$. But then
$[a^\prime]=[a-\del w] =[a] -[del w] =[a]$. Finally, there is actually a choice
of $z$ ; if $[z]=[z^\prime]$, then $z^\prime = z+\del^{\prime\prime}w$
for some $w\in C_{n+1}$; but then choosing $b^\prime,w^\prime$ with 
$j_n(b^\prime)=z^\prime$ , $j_{n+1}(w^\prime)=w$ , we have 

$\del^{\prime\prime}w=\del^{\prime\prime}j_{n+1}(w^\prime) = j_n\del^\prime(w^\prime)$ ,
so 

$z^\prime= z+\del^{\prime\prime}w= j_n(b+\del^\prime w^\prime)$, so we may choose
$b^\prime = b+\del^\prime w^\prime$ (since the result is independent of this choice!),
then since $\del^\prime b^\prime = \del^\prime b$ everything continues the same.

\msk

Now to exactness! We need to show three (types of) equalities, which means six
containments. Three (image contained in kernel) 
are shown basically by showing that compositions of
two consecutive homomorphisms are trivial. $j_ni_n=0$ 
immediately implies $j_*i_*=0$ . From the definition of $\del$,
$i_*\del[z] = [i_n(a)] = [\del^\prime(b)] = 0$, and 
$\del j_*[z] = \del[j_n(z)] = [a]$, where $i_{n-1}(a)=\del^\prime(z) = 0$,
so $a=0$ (since $i_{n-1}$ is injective), so $[a]=0$. 

\ssk

For the opposite containments,
if $j_*[z]=[j_n(z)]=0$, then $j_n(z)=\del^{\prime\prime}w$ for some $w$. 
Since $j_{n+1}$ is onto, $w=j_{n+1}(b)$ for some $b$. Then 
$j_n(z-\delp b) = \delpp w-\delpp j_{n+1} b = 0$, so 
$z=\delp b = i_n(a)$ for some $a$, so $i_*[a] = [z-\delp b] = [z]$ . 
So $\ker j_*\subseteq$im$i_*$ . If $i_*[z]=0$, then $i_n(z)=\delp w$ for some $w\in B_{n+1}$.
Setting $c=j_{n+1}(w)$, then $\delpp c = j_n \delp w - i_n i_n(Z) = 0$, so 
$[c]\in h_{n+1}({\Cal C})$, and computing $\del [c]$ we find that we can choose $w$ for the 
first step and $z$ for the second step, so $\del [c] = [z]$ . So $\ker j_n\subseteq$im$\del$ .
Finally, if $\del [z] = 0$, then $z=j_n(b)$ for some $b$, and $\delp b = i_{n-1}(a)$ with
$[a]=0$, i.e., $a=\del w$ for some $w$. So $\delp b = i_{n-1}\del w = \delp i_n w$ But
then $\delp (b-i_n w) = 0$, and 
$j_n(b-i_n w) = z-0 = z$, so $z\in$im$(j_n)$, so $[z]\in$im$(j_*)$ . So 
$\ker\del\subseteq$im$(j_n)$ . Which finishes the proof!

\msk

Now all we need are some new chain complexes! 

\bsk




\vfill
\end


















