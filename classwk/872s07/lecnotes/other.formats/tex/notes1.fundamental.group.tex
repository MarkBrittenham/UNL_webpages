\magnification=1200
\overfullrule=0pt
\parindent=0pt

%\nopagenumbers

\input amstex

%\voffset=-.6in
%\hoffset=-.5in
%\hsize = 7.5 true in
%\vsize=10.4 true in

%\voffset=1.4in
%\hoffset=-.5in
%\hsize = 10.2 true in
%\vsize=8 true in

\input colordvi

\def\cltr{\Red}		  % Red  VERY-Approx PANTONE RED
\def\cltb{\Blue}		  % Blue  Approximate PANTONE BLUE-072
\def\cltg{\PineGreen}	  % ForestGreen  Approximate PANTONE 349
\def\cltp{\DarkOrchid}	  % DarkOrchid  No PANTONE match
\def\clto{\Orange}	  % Orange  Approximate PANTONE ORANGE-021
\def\cltpk{\CarnationPink}	  % CarnationPink  Approximate PANTONE 218
\def\clts{\Salmon}	  % Salmon  Approximate PANTONE 183
\def\cltbb{\TealBlue}	  % TealBlue  Approximate PANTONE 3145
\def\cltrp{\RoyalPurple}	  % RoyalPurple  Approximate PANTONE 267
\def\cltp{\Purple}	  % Purple  Approximate PANTONE PURPLE

\def\cgy{\GreenYellow}     % GreenYellow  Approximate PANTONE 388
\def\cyy{\Yellow}	  % Yellow  Approximate PANTONE YELLOW
\def\cgo{\Goldenrod}	  % Goldenrod  Approximate PANTONE 109
\def\cda{\Dandelion}	  % Dandelion  Approximate PANTONE 123
\def\capr{\Apricot}	  % Apricot  Approximate PANTONE 1565
\def\cpe{\Peach}		  % Peach  Approximate PANTONE 164
\def\cme{\Melon}		  % Melon  Approximate PANTONE 177
\def\cyo{\YellowOrange}	  % YellowOrange  Approximate PANTONE 130
\def\coo{\Orange}	  % Orange  Approximate PANTONE ORANGE-021
\def\cbo{\BurntOrange}	  % BurntOrange  Approximate PANTONE 388
\def\cbs{\Bittersweet}	  % Bittersweet  Approximate PANTONE 167
%\def\creo{\RedOrange}	  % RedOrange  Approximate PANTONE 179
\def\cma{\Mahogany}	  % Mahogany  Approximate PANTONE 484
\def\cmr{\Maroon}	  % Maroon  Approximate PANTONE 201
\def\cbr{\BrickRed}	  % BrickRed  Approximate PANTONE 1805
\def\crr{\Red}		  % Red  VERY-Approx PANTONE RED
\def\cor{\OrangeRed}	  % OrangeRed  No PANTONE match
\def\paru{\RubineRed}	  % RubineRed  Approximate PANTONE RUBINE-RED
\def\cwi{\WildStrawberry}  % WildStrawberry  Approximate PANTONE 206
\def\csa{\Salmon}	  % Salmon  Approximate PANTONE 183
\def\ccp{\CarnationPink}	  % CarnationPink  Approximate PANTONE 218
\def\cmag{\Magenta}	  % Magenta  Approximate PANTONE PROCESS-MAGENTA
\def\cvr{\VioletRed}	  % VioletRed  Approximate PANTONE 219
\def\parh{\Rhodamine}	  % Rhodamine  Approximate PANTONE RHODAMINE-RED
\def\cmu{\Mulberry}	  % Mulberry  Approximate PANTONE 241
\def\parv{\RedViolet}	  % RedViolet  Approximate PANTONE 234
\def\cfu{\Fuchsia}	  % Fuchsia  Approximate PANTONE 248
\def\cla{\Lavender}	  % Lavender  Approximate PANTONE 223
\def\cth{\Thistle}	  % Thistle  Approximate PANTONE 245
\def\corc{\Orchid}	  % Orchid  Approximate PANTONE 252
\def\cdo{\DarkOrchid}	  % DarkOrchid  No PANTONE match
\def\cpu{\Purple}	  % Purple  Approximate PANTONE PURPLE
\def\cpl{\Plum}		  % Plum  VERY-Approx PANTONE 518
\def\cvi{\Violet}	  % Violet  Approximate PANTONE VIOLET
\def\clrp{\RoyalPurple}	  % RoyalPurple  Approximate PANTONE 267
\def\cbv{\BlueViolet}	  % BlueViolet  Approximate PANTONE 2755
\def\cpe{\Periwinkle}	  % Periwinkle  Approximate PANTONE 2715
\def\ccb{\CadetBlue}	  % CadetBlue  Approximate PANTONE (534+535)/2
\def\cco{\CornflowerBlue}  % CornflowerBlue  Approximate PANTONE 292
\def\cmb{\MidnightBlue}	  % MidnightBlue  Approximate PANTONE 302
\def\cnb{\NavyBlue}	  % NavyBlue  Approximate PANTONE 293
\def\crb{\RoyalBlue}	  % RoyalBlue  No PANTONE match
%\def\cbb{\Blue}		  % Blue  Approximate PANTONE BLUE-072
\def\cce{\Cerulean}	  % Cerulean  Approximate PANTONE 3005
\def\ccy{\Cyan}		  % Cyan  Approximate PANTONE PROCESS-CYAN
\def\cpb{\ProcessBlue}	  % ProcessBlue  Approximate PANTONE PROCESS-BLUE
\def\csb{\SkyBlue}	  % SkyBlue  Approximate PANTONE 2985
\def\ctu{\Turquoise}	  % Turquoise  Approximate PANTONE (312+313)/2
\def\ctb{\TealBlue}	  % TealBlue  Approximate PANTONE 3145
\def\caq{\Aquamarine}	  % Aquamarine  Approximate PANTONE 3135
\def\cbg{\BlueGreen}	  % BlueGreen  Approximate PANTONE 320
\def\cem{\Emerald}	  % Emerald  No PANTONE match
%\def\cjg{\JungleGreen}	  % JungleGreen  Approximate PANTONE 328
\def\csg{\SeaGreen}	  % SeaGreen  Approximate PANTONE 3268
\def\cgg{\Green}	  % Green  VERY-Approx PANTONE GREEN
\def\cfg{\ForestGreen}	  % ForestGreen  Approximate PANTONE 349
\def\cpg{\PineGreen}	  % PineGreen  Approximate PANTONE 323
\def\clg{\LimeGreen}	  % LimeGreen  No PANTONE match
\def\cyg{\YellowGreen}	  % YellowGreen  Approximate PANTONE 375
\def\cspg{\SpringGreen}	  % SpringGreen  Approximate PANTONE 381
\def\cog{\OliveGreen}	  % OliveGreen  Approximate PANTONE 582
\def\pars{\RawSienna}	  % RawSienna  Approximate PANTONE 154
\def\cse{\Sepia}		  % Sepia  Approximate PANTONE 161
\def\cbr{\Brown}		  % Brown  Approximate PANTONE 1615
\def\cta{\Tan}		  % Tan  No PANTONE match
\def\cgr{\Gray}		  % Gray  Approximate PANTONE COOL-GRAY-8
\def\cbl{\Black}		  % Black  Approximate PANTONE PROCESS-BLACK
\def\cwh{\White}		  % White  No PANTONE match


\loadmsbm

\input epsf

\def\ctln{\centerline}
\def\u{\underbar}
\def\ssk{\smallskip}
\def\msk{\medskip}
\def\bsk{\bigskip}
\def\hsk{\hskip.1in}
\def\hhsk{\hskip.2in}
\def\dsl{\displaystyle}
\def\hskp{\hskip1.5in}

\def\lra{$\Leftrightarrow$ }
\def\ra{\rightarrow}
\def\mpto{\logmapsto}
\def\pu{\pi_1}
\def\mpu{$\pi_1$}
\def\sig{\Sigma}
\def\msig{$\Sigma$}
\def\ep{\epsilon}
\def\sset{\subseteq}
\def\del{\partial}
\def\inv{^{-1}}
\def\wtl{\widetilde}
%\def\lra{\Leftrightarrow}
\def\del{\partial}
\def\delp{\partial^\prime}
\def\delpp{\partial^{\prime\prime}}
\def\sgn{{\roman{sgn}}}
\def\wtih{\widetilde{H}}
\def\bbz{{\Bbb Z}}
\def\bbr{{\Bbb R}}



\ctln{\bf Math 872 \hsk Algebraic Topology}

\ssk

\ctln{Running lecture notes}

\bsk

{\bf First a word from our sponsor...}

\ssk

Algebraic topology is an umbrella term for that part or topology which uses
algebraic tools to study and answer topological problems. The most basic
problem in topology is, given two topological spaces $X$ and $Y$, to determine
whether or not they are homeomorphic. Other basic questions are to understand
the different ways (if any) that one space can be embedded in another
(i.e., the ways in which $X$ can be homeomorphic to a subspace of $Y$), and to 
understand the continuous maps from $X$ to $Y$. In all of these tools from 
alebraic topology have a role to play.

\ssk

At its heart the idea to assign to each topological space (in some reasonable
collection (read: category) of spaces that you are interested in) an algebraic
object (group, ring, field, module...) in some intelligent way. Typically, to tie 
the construction to the topology on $X$, the object is built using continuous
functions into or out of $X$. That is, after all, what a topology is really
good for; it tells you what maps are continuous, i.e., aren't ripping your
space apart. A construction which ``really'' only uses the topology on $X$ (and 
not something more) will have the property that homeomorphic spaces have
isomorphic objects assigned to them (you can almost take this as a definition
of ``really using the topology''). Then the algebraic objects can be used to
distinguish spaces; if the objects aren't isomorphic, then the spaces they came
from can't be homeomorphic. The basic idea is that distinguishing groups 
from one another is ``easier'' than distinguishing spaces; whether or not this 
is really true we will discuss at a later date! But, for example, finitely generated
abelian groups are easy to distinguish (when given to you as a direct sum of
cyclic groups, for example), which can be turned into a method for distinguishing
spaces, when our method assigns such groups to spaces.

\ssk

This kind of process can tell us that two spaces are different, if the {\it algebraic
invariants} that we assign to the spaces are different, that is enough. But it 
doesn't run the other way; if the algebraic invariants are the same, we cannot
conclude that the spaces are the same. (Dumb example: we assign to every topological
space the field with two elements. Homeomorphic spaces have the same associated 
field, but...) But this doesn't stop us from continuing to try to find invariants
that continue to do better at distinguishing spaces....

\msk

In this course we will explore two basic approaches to building algebraic invariants:
homotopy theory and homology theory. Each builds a sequence of groups (all but one
of them, the first homotopy group, or {\it fundamental group}, are abelian) which
serve as algebraic invariants of the space. History has shown us that the homotopy
groups, typically, are more powerful; they are better at distinguishing spaces. They
pay for this, however by being more difficult to compute in practice. For example,
there is no general formula for the homotopy groups of the 2-sphere $S^2$; all
that is generally known is that all but two are finite, and all but one (?) are non-trivial.
The first few hundred, probably, have actually been computed. We will focus mostly on
the fundamental group $\pi_1(X)$; its computation, properties, and applications.
The fundamental group has found its way into a wide variety of mathematical fields
(essentially, anywhere that continuity has?).

Homology groups, on the other hand, are typically ``easier'' to compute, at least
once you have gotten some theory out of the way! They pay for this by being less
adept at distinguishing spaces. At least straight out of the box; a lot of effort
has been invested in finding more subtle ways to use the techniques of homology
theory to winkle ever more detailed information out of a topological space.
Homology groups are \u{designed} to be abelian (for the higher homotopy groups
this is more a matter of some delightful accident); literally, they are each
the quotient of a free abelian group by a subgroup. So the main computational 
tool is some fairly straighforward linear algebra. 

\bsk

Ultimately our goal is to construct these two theories, explore their properties, and 
then use them to prove some topological results that, in the end, one might never 
have guessed that algebraic techniques would have played a role in proving. Some
sample results:

\msk

There is a (continuous) map from the surface of genus 3, $\Sigma_3$ to the surface
of genus 2, $\Sigma_2$, such that every point inverse is finite. There is
no such map $\Sigma_2\rightarrow \Sigma_3$ (or $\Sigma_4\rightarrow \Sigma_3$ or...).
One can in fact give a precise statement of when such a map $\Sigma_n\rightarrow \Sigma_m$
exists; it is that $m-1|n-1$ .

\ssk

The real projective plane $\bbr P^2$ cannot embed in $\bbr^3$.

\ssk

Invariance of Domain: If $U\subseteq \bbr^n$ is open and $f:U\ra\bbr^n$ is continuous
and injective, then $f(U)\subseteq \bbr^n$ is open. (I.e., being a {\it domain}, an open 
subset of $\bbr^n$, is invariant under continuous injections.)

\bsk

There is one topological fact which we will use constantly which you might not have
seen in your point-set topology class (although it may have come up in an analysis
class?): the Lebesgue Number Theorem. If $(X,d)$ is a compact metric space
(in our applications, it is always a compact subset of Euclidean space), and 
$\{{\Cal U}_i\}$ is an open cover of $X$, then there is an $\epsilon>0$ so that
for every $x\in X$, its $\epsilon$-neighborhood $N_d(x,\epsilon)$ is contained
in ${\Cal U}_i$ for some $i$. 
For if not, then for every $n\in{\Bbb N}$ there is an $x_n\in X$ 
whose $1/n$-neighborhood is contained in no ${\Cal U}_i$; that is, for 
every $i\in I$, there is an $x_{n,i}$ with
$d(x_n,x_{n,i})<1/n$ and $x_{n,i}\notin{\Cal U}_i$, so $x_{n,i}\in C_i=X\setminus {\Cal U}_i$, 
a closed set. 
But since $X$ is compact, there is a convergent subsequence of the $x_n$; $x_{n_k}\ra y\in X$.
[Proof: if not, then no point is the limit of a subsequence, so for every $x\in X$ there is
an $\epsilon(x)>0$ and an $N=N(x)$ so that $n\geq N$ implies $x_n\notin N_d(x,\epsilon(x)$.
But these neighborhoods cover $X$, so a finite number of them do; for any $n$ gretaer than
the maximum of the associated $N(x)$'s $x_n$ lies in none of the neighborhoods, a contradiction,
since $x_n\in X=$ the union of theose neighborhoods.]
But then since $d(x_{n_k},y)\ra 0$ and $d(x_{n_k},x_{n_k,i})\ra 0$, for every $i$
the $x_{n_k,i}$ also converge to $y$; since the $x_{n_k,i}$ all lie in the closed
set $C_i$, so does $y$. So $y\in C_i$ for all $i$, so $y\notin {\Cal U}_i$ for all $i$,
a contradication, since the ${\Cal U}_i$ cover $X$. So some $\epsilon>0$, a 
{\it Lebesgue number} for the covering, must exist.




\vfill
\eject

{\bf The fundamental group:}

\msk

One of the main themes of topology is that a space $X$ can be studied
by looking at maps (= continuous functions) of ``useful''
spaces into $X$. Maps provide the means to explore a space,
and literally map out the topology of the space. The most basic 
useful space to use is an interval $I=[0,1]$; understanding how intervals
map into a space provides information on the many ways to get
from ``here'' to ``there''. This is the basis for the fundamental
group.

The key to understanding the fundamental group is to understand
how you could conceivably turn maps of intervals (i.e., {\it paths}) into a group,
i.e., how to multiply them. That is, how do you take two maps
$f,g:I\ra X$ and produce a single map $f\cdot g:I\ra X$ ?
After fussing a bit, you would probably hit upon what Poincar\'e did; 
two intervals glued end to end form an interval; so two maps
glued end to end build a map. {\it If} you can do the gluing;
that requires $f(1)=g(0)$ in order to be well-defined.
In order to multiply any two paths in any order (so as to form a group) we need to 
impose a compatibility condition, that is we need to assume
that all paths have compatible endpoints, so we focus on loops,
that is paths $f:[0,1]\ra X$ so that $f(0)=f(1)=x_0\in X$ for some
fixed basepoint $x_0\in X$. These can be concatenated in any order;

\ssk

$f*g(t)=\cases
f(2t), & \text{if}\ t\leq 1/2\cr
g(2t-1)& \text{if}\ t\geq 1/2\cr 
\endcases$

\ssk

\noindent runs across $f$ first, and then $g$. This gives us a 
mutliplication on the loops at $x_0$. But this really can't be turned
into a group; we could never find an appropriate identity element, since
the product of $f$ with anything contains a copy of $f$ in it, so our
identity would have to have a copy of every loop in it? 

\msk

The solution is to define our group elements to be equivalence classes of loops,
so that our identity element can ``have'' a copy of every map $f$ in it!
The other point is that by making elements of the group ``big'' (having
lots of representatives), this will make the group `smaller'', and more
manageable. Finally, by letting many loops really be the ``same'', we
can focus on more important, global, features of a space rather than
inessential local information. The equivalence relation we use is
homotopy, that us, continuous deformation. In general, two maps
$f,g:Y\ra X$ are homotopic if there is a map $H:Y\times I\ra X$
such that $H(y,0)=f(y)$ and $H(y,1)=g(y)$ for every $y\in Y$. That is,
the map $f\coprod g:Y\times \{0,1\}\ra X$ extends to a map on $Y\times I$.
There is also a notion of homotopy for a map of pairs $f:(Y,B)\ra (x,A)$ 
(that is, $f(B)\subseteq A$), requiring that $H$ is also
a map of pairs, $H(B\times I)\subseteq A$. A loop at $x_0$ is really a map of 
pairs $f:(I,\del I)\ra (X,\{x_0\})$, and the elements of the
fundamental group $\pi_1(X,x_0)$ will be equivalence classes of 
loops at $x_0$, under the equivalence relation of homotopy as maps of pairs.
We of course need to show that homotopy is an equivalvence relation,
that is, $f\simeq f$, if $f\simeq g$ then $g\simeq f$, and if $f\simeq g$ 
and $g\simeq h$ then $f\simeq h$. For each of these it is fairly straightforward
to build the required homotopy ($H(t,s)=f(t)$ , $K(t,s)=H((t,1-s)$, and 
$L(t,s)=$ the concatenation of two homotopies $H$ and $K$, on the second
variable; the Pasting Lemma assures its continuity). 

\msk

With this dealt with, elements are equivalence classes $[f]$ of loops at $x_0$,
we define our multiplication to be $[f]\cdot[g]=[f*g]$. For this to be 
well-defined, we need to check that if $f\simeq f^\prime$ and $g\simeq g^\prime$, then
$f*g\simeq f^\prime*g^\prime$, but the required homotopy can be built by 
concatenating the hypothesized homotopies, along the {\it first} variable.
But now, since the elements are so big, we can construct a meaningful
identity element, and verify that we under this multiplication we have a group.
The identity element is the loop which does nothing, that is, the equivalence 
class containing the constant map $c_0$ at $x_0$; $e=[c_0]$. The inverse $[f]^{-1}$
is $[\overline{f}]$, where $\overline{f}(t)=f(1-t)$ is $f$ run in the
reverse direction. The verifications $f*c_0\simeq f\simeq c_0*f$ and 
$f*\overline{f}\simeq c_0\simeq \overline{f}*f$ 
can be verified by building the appropriate homotopies:

\ssk

$H(t,s)=\cases f(2t/(s+1)), & \text{if}\ t\leq (s+1)/2 \cr x_0, & \text{if}\ t\geq (s+1)/2\cr \endcases$
\hskip.1in 
$H(t,s)=\cases  x_0, & \text{if}\ t\leq s/2\cr f((t-s/2)/(1-s/2)), & \text{if}\ t\geq s/2 \cr \endcases$

\msk

\hskip-.2in
$H(t,s)=\cases x_0, & \text{if}\ t\leq s/2 \cr f(2t-s), & \text{if}\ s/2\leq t\leq 1/2 \cr 
               f(2-s-2t), & \text{if}\ 1/2\leq t\leq 1-s/2 \cr x_0, & \text{if}\ 1-s/2\leq t\leq 1\cr \endcases$
\hskip.1in 
$H(t,s)=\cases x_0, & \text{if}\ t\leq (1-s)/2 \cr f(2-s-2t), & \text{if}\ (1-s)/2\leq t\leq 1/2 \cr 
               f(2t-s), & \text{if}\ 1/2\leq t\leq (1+s)/2 \cr x_0, & \text{if}\ (1+s)/2\leq t\leq 1\cr \endcases$

\ssk

These give us that $g\cdot e=g=e\cdot g$ and $g\cdot g^{-1}=e=g^{_1}\cdot g$ for every 
$g\in\pi_1(X,x_0)$. Associativity, $g\cdot (h\cdot k)=(g\cdot h)\cdot k$, by
another homotopy: if $g=[\alpha], h=[\beta]$ and $k=[\gamma]$, then 
$\alpha *(\beta *\gamma)\simeq (\alpha *\beta)*\gamma$ via the homotopy

\ssk

$H(t,s)=\cases \alpha(4t/(2-s)), & \text{if}\ t\leq (2-s)/4 \cr 
               \beta(4t-2+s), & \text{if}\ (2-s)/4 \leq t\geq (3-s)/4 \cr 
               \gamma((4t-3+s)/(s+1)), & \text{if}\ t\geq (3-s)/4 \cr \endcases$

\ssk

The Pasting Lemma assures that all of these maps are continuous.
All of these homotopies are probably best understood pictorially, looking at what 
is happening in each individual region of definition.

\msk

So with this (well-defined) multiplication, we have an associative
product on $\pi_1(X,x_0)$ with an identity and two-sided inverse, making
$\pi_1(X,x_0)$ a group, the {\it fundamental group of $X$ based at $x_0$}.
This group literally enumerates the number of different ways to walk around the
space $X$ and return to our starting point, where two ways are different
if one cannot be continuously deformed to the other. The fact that
this collection of different ways together form a group under concatenation
provides extra structure, giving us a better chance to be able to compute
this object when we need to.

\ssk

But just jumping in and computing it turns out to be rather difficult, 
at least straight from the definition. How do you really decide when
two loops are homotopic? What we need to do is to erect a theory around
our basic definitions, to give us a way to work with them and to 
illuminate their importance and utility.

\ssk

Several basic properties are important to the utility of the fundamental
group. The first is that if $f:(X,x_0)\ra (Y,y_0)$ is a map of pairs,
then there is an induced homomorphism $f_*:\pi_1(X,x_0)\ra \pi_1(Y,y_0)$
given by $f_*[\gamma]=[f\circ\gamma]$. Since $\gamma\simeq\beta$ implies
$f\circ\gamma\simeq f\circ\beta$ via $K(t,s)=f(H(t,s))$, the map 
is well-defined, and since
$f\circ(\gamma*\beta)=(f\circ\gamma)*(f\circ\beta)$, it is a homomorphism.
Further, $(f\circ g)_*=f_*\circ g_*$ follows directly, as does $(I_X)_*=I_{\pi_1(X)}$.
From which it follows that if $f:X\ra Y$ is a homeomorphism, then
$f_*:\pi_1(X,x_0)\ra \pi_1(Y,f(x_0))$ is an isomorphism. So homeomorphic
spaces have isomorphic fundamental groups; the fundamental group is a
homeomorphism invariant.

\ssk

In many instances, we supress the basepoint $x_0$ in our work with the
fundamental group. The basis for doing this is that if 
$x_0,x_1\in X$ and $\gamma:I\ra X$ is a path in $X$ with 
$\gamma(0)=x_0$ and $\gamma(1)=x_1$, then given a loop $f$ at $x_0$,
$\overline{\gamma}*f*\gamma$ is a loop at $x_1$. The map
$F=F_\gamma:\pi_1(X,x_0)\ra \pi_1(X,x_1)$ given by $F([f])=[\overline{\gamma}*f*\gamma]$
is well-defined and provides an isomorphism between the two groups, 
with inverse $F^{-1}([g])=[\gamma*g*\overline{\gamma}]$.
(The relevant homotopies are readily constructed.) So for a
path-connected space, the fundamental groups based at any two points
are isomorphic. So we can sensibly talk about \underbar{the}
fundamental group of a path-connected space, without mentioning 
basepoints. (There needn't be a {\it canonical} isomorphism between
the groups; a different path between two basepoints might yield a 
different isomorphism. This is occasionally a {\it very} important
consideration!)

\msk

An important combination of these two facts allows us to understand the
behavior of induced maps under homotopy. If $f,g:X\ra Y$ are homotopic
maps, via a homotopy $H$, and $x_0$ is a baspoint in $X$, let
$\gamma(t)=H(x_0,t)$ be the path in $Y$ traced out by $x_0$
under the homotopy, and set $y_0=\gamma(0),y_1=\gamma(1)$.
Then $f:(X,x_0)\ra (Y,y_0)$ and $g:(X,x_0)\ra (Y,y_1)$ as maps of pairs,
and we have an isomorphism $F_\gamma:\pi_1(Y,y_0)\ra \pi_1(Y,y_1)$.
Then we can see that $g_*=F_\gamma\circ f_*:\pi_1(X,x_0)\ra \pi_1(Y,y_1)$,
by properly ``reparametrizing'' the homotopy $H$:
given a loop $\alpha:(I,\del I)\ra (X,x_0)$, 
$g\circ\alpha\simeq \overline{\gamma}*(f\circ\alpha)*\gamma$ via the
homotopy $K(s,t)=H(\alpha(t),s)$ together with a map of the square $I\times I$
to itself which takes the top edge to itself and stretches the bottom edge
around the remaining three edges (thereby taking the two vertical edges
each to the endpoints of the top edge). Such a map is readily constructed,
although a formula for it might be a bit ugly...

\ssk

This has special meaning when one of $f,g$ is a homeomorphism (think: the identity); 
then since
the homeo induces an iso in $\pi_1$ and $F_\gamma$ is an iso, the other
map induces an iso on $\pi_1$, as well. Another special case is when the homotopy
between $f$ and $g$ is basepoint-preserving; that is, $\gamma$ is a constant map. 
Then $F_\gamma=I_{\pi_1}$, since we can smooth out the constant maps we 
pre- and post-append to a given loop to return us to the given loop, as we did
above in verifying that the constant loop is the identity element in $pi_1$.
So basepoint-preserving homotopic maps induce the same map on $\pi_1$. 

\ssk

This line of thought reaches its logical conclusion with the introduction of
homotopy equivalences. Two spaces $X,Y$ are homotopy equivalent if there are maps
$f:X\ra Y$ and $g:Y\ra X$ so that $f\circ g:Y\ra Y$ and $g\circ f:X\ra X$ are
both homotopic to the identity. It is straighforward (as in the discussion of homotopy above)
that ``are homotopy equivalent'' is an equivalence relation (hence the name).
Each of $f$ and $g$ are called homotopy equivalences. The discussions above
combine to give the result: if $f:X\ra Y$ is a homotopy equivalence, then
$f_*:\pi_1(X,x_0)\ra \pi_1(Y,f(x_0))$ is an isomorphism. This is because the
compositions $f_*\circ g_*=(f\circ g)_*$ and $g_*\circ f_*=(g\circ f)_*$ are 
isomorphisms, so the first is surjective (hence $f_*$ is surjective) and 
the second is injective (so $f_*$ is injective). So homotopy equivalent
spaces have isomorphic fundamental groups.

\ssk

A subset $a\subseteq X$ is a {\it retract} of $X$ if there is a map $r:X\ra A$ so 
that $r(a)=a$ for all $a\in A$. That is, for $\iota:A\ra X$ the inclusion map,
$r\circ \iota = I_A$. $A$ is a {\it deformation retract} of $X$ if in addition
$\iota\circ r:X\ra X$ is homotopic to the identity. $r$ is a {\it strong deformation
retraction} if this homotopy leaves every element of $A$ fixed; $H(z,t)=a$
for all $a\in A$. If $r$ is a deformation retraction, then it is a homotopy
equivalence (we take the identity homotopy for $r\circ \iota$). So the inclusion
$\iota:A\ra X$ induces an isomorphism on $\pi_1$. This idea simplifies
many computations, by allowing us to compute $\pi_1(A)$ instead of 
$\pi_1(X)$. FOr example, $\bbr^n$ deformation retracts to the origin $x_0=0$;
the homotopy $H(x,t)=tx$ interpolates between the identity and $\iota\circ r=c_{x_0}$.
Since $\pi_1(x_0)=\{1\}$ (there \underbar{is} only one map $\gamma:I\ra \{x_0\}$), 
we deduce that $\pi_1(\bbr^n)=1$. The same is true for disks $D^n$. More generally,
a space is called {\it contractible} if it is homotopy equivalent to a point;
every contractible space has trivial fundamental group.

\msk

A loop is map $\gamma:I\ra X$ with $\gamma(0)=\gamma(1)=x_0$; it therefore 
decends to a well-defined map of the quotient space $I/{0,1}\cong S^1$ to $X$,
and so a loop can be thought of as a map from the unit circle $(S^1,1)\ra(X,x_0)$. Elements of
$\pi_1(X,x_0)$ could have been defined as equivalence classes of such maps
under homotopy of pairs (it is not difficult to see how to lift such a 
homotopy to a homotopy $(I\times I,\del I\times I)\ra (X,x_0)$). The 
multiplication is a little more convoluted to work out; the reader is 
invited to do so. From this perspective, though, understanding 
what the identity element looks like has a more geometric feel:
$\gamma:S^1\ra X$ represents the identity in $\pi_1(X)$ \lra\
$\gamma$ extends to a map $\Gamma:D^2\ra X$ ( i.e., $\Gamma|_{\del D^2}=\gamma$),
where $D^2$ is the unit disk in $\bbr^2$. The basic idea is that if $\gamma$
is trivial, there is a homotopy $S^1\times I\ra X$ which on $S^1\times \{0\}$ is
$\gamma$ and which sends $Y=\{1\}\times I\cup S^1\times \{1\}$ to $x_0$.
The homotopy descends to a map from $S^1\times I$, with $Y$ crushed to a point, to $X$.
But $S^1\times I/Y$ is homeomorphic to $D^2$, with $S^1\times\{0\}$ being sent to $\del D^2$;
the composition $D^2\ra S^1\times I/Y\ra X$ is the desired extension.

In a similar vein, two paths $\alpha,\beta:I\ra X$ joining the same pair of points
$x_0,x_1\in X$ are homotopic rel endpoints (i.e., the maps $(I\del I)\ra(X,\{x_0,x_1\})$
are homotopic as maps of pairs) \lra\ the loop $\alpha *\overline{\beta}$ is 
trivial in $\pi_1(X,x_0)$. (The extension to $D^2$ is built from the
homotopy by crushing each of the vertical boundary segments to points.) So, for
example, in a contractible space, any two paths between the same two points are
homotopic rel endpoints.

\msk

{\bf The fundamental group of the circle:}

\ssk

Our first really non-trivial computation is to determine the fundamental group of the
circle $S^1$. Since $S^1$ is path-connected, the answer is independent of basepoint,
so we will choose $x_0=(1,0)\in S^!\subseteq\bbr^2$. First the answer: $\pi_1(S^1)\cong\bbz$ .
The proof is a little involved, but it will introduce several basic ideas that will
become central to the development of our more general theory.

The basic idea is, given an element $[\gamma]\in\pi_1(S^1)$, to find a (more or
less) canonical representative $\alpha\in[\gamma]$, and show that these canonical
representatives can be put into one-to-one correspondence with $\bbz$. The idea
is to cover $S^1$ by a pair of contractible open sets, which we can take to be
${\Cal U}_- = \{(x,y)\in S^1 : y < \epsilon\}$ and ${\Cal U}_+ = \{(x,y)\in S^1 : y > -\epsilon\}$
for some small $\epsilon > 0$. (These are slightly larger than the lower and upper
semicircles in $S^1$.) Given a loop $\gamma:I\ra S^1$, the sets 
${\Cal V}_\pm=\gamma^{-1}({\Cal U}_\pm)$
form an open cover of the compact metric space $I$, so, by the Lebesgue Number Theorem,
there is a $\delta>0$ so that every interval of length $\delta$ in $I$ lies in either
${\Cal V}_-$ or ${\Cal V}_+$ Choose an $n>1/\delta$ and cut $I$ into $n$ subintervals
of equal length; then each (closed) subinterval (has length less than $\delta$ so) 
is mapped by $\gamma$ into either 
${\Cal U}_-$ or ${\Cal U}_+$ (or both); choose one, assigning a $+$ or $-$ to each
subinterval. If two adjacent subintervals have the same sign, amalgamate them into a
single larger interval. Continuing in this fashion, we arrive at a collection of subintervals
whose signs alternate. The endpoints of the intervals (since they really have both signs) 
map into ${\Cal U}_-\cap{\Cal U}_+$= two short intervals. 

Now we start ``straightening'' $\gamma$. We have $I$ cut into subintervals $I_1,\ldots,I_k$,
each mapping into ${\Cal U}_\pm$. These subsets are contractible, so any pair
of paths between the same two points are homotopic. Put differently,
we can replace $\gamma|_{I_j}$ with any other path between the same two
points, to obtain a new map homotopic, rel endpoints, to $\gamma$, yielding a 
new representative of $[\gamma]$. This is our basic simplification process; the 
new path we continually choose is the arc (in ${\Cal U}_\pm$) between the points.
So for each subinterval make this switch, yielding a new loop (because the endpoints
are unaffected) homotopic to $\gamma$, which we still call $\gamma$. If any of
the subintervals maps into ${\Cal U}_-\cap{\Cal U}_+$, then we can switch its sign
and amalgamate it with its neighboring subintervals, producing fewer subintervals, 
and repeat the process. So eventually (read: by induction) we reach a stage where
each subinterval ``crosses'' ${\Cal U}_\pm$; each has its endpoints in different
components of ${\Cal U}_-\cap{\Cal U}_+$. (There is a degenerate case where
the entire interval $I$ lies in ${\Cal U}_-\cap{\Cal U}_+$; this implies 
that our altered $\gamma$ is the constant map, which is our canonical form 
for the identity element...) By inserting short paths from the endpoints of our subintervals
to $(1,0)$ or $(-1,0)$ (whichever one is in the component of
${\Cal U}_-\cap{\Cal U}_+$ containing our endpoint) and back, and straightening,
we may assume the our endpoints map to $(\pm 1,0)$. Finally, a reparametrization of the 
interval makes all of the subintervals we now have of the same length $I_j=[j/m,(j+1)/m]$,
making $\gamma|_{I_j}$ one of precisely four maps: reparametrizing $I_j$ as $I$, for convenience,
they are
$\alpha_1:t\mapsto (\cos(\pi t),\sin(\pi t))$ , $\overline{\alpha_1}:t\mapsto (-\cos(\pi t),\sin(\pi t))$ , 
$\alpha_2:t\mapsto (-\cos(\pi t),-\sin(\pi t))$ , $\overline{\alpha_2}:t\mapsto (\cos(\pi t),-\sin(\pi t))$.
($\alpha_1$ and $\alpha_2$ map counterclockwise around the top and bottom of the circle,
respectively; their reverses go clockwise.)
Any occurances of $\alpha_1*\overline{\alpha_1}$,$\overline{\alpha_1}*\alpha_1$,
$\alpha_2*\overline{\alpha_2}$,$\overline{\alpha_2}*\alpha_2$ may be replaced by
the constant map (since these loops are null-homotopic) and amalgamated away, and
the combintations $\alpha_1*\overline{\alpha_2}$,$\overline{\alpha_2}*\alpha_1$,
$\alpha_2*\overline{\alpha_1}$,$\overline{\alpha_1}*\alpha_2$ cannot occur because in each 
the two paths do not share endpoints properly. So after this further amalgamation the only
posibilities are
$c_{x_0}$ (the degenerate case), $(\alpha_1*\alpha_2)^n$, or 
$(\overline{\alpha_1*\alpha_2})^n=(\overline{\alpha_2}*\overline{\alpha_1})^n$
(where we must have an even number of factors by basepoint considerations; we have
loops). These are our canonical forms. {\bf If} these canonical forms are unique (no
two of them are homotopic), then we can construct our isomorphism 
$\pi_1(S^1)\ra \bbr$ by sending $[c_{x_0}]\mapsto 0$, $(\alpha_1*\alpha_2)^n\mapsto n$, and 
$(\overline{\alpha_1*\alpha_2})^n\mapsto -n$. That this is an isomorphism follows by inspection.

\ssk

Provided we show uniqueness! To do this, we use another technique which we will exploit 
much further later on. Essentially, we wish to show that for $n\neq m$, the maps
$t\mapsto (\cos(2\pi mt),\sin(2\pi mt))$ and $t\mapsto (\cos(2\pi nt),\sin(2\pi nt))$,
which is what our canonical forms really turn out to be, represent distinct elements of 
$\pi_1(S^1)$, i.e., are not homotopic to one another. To do this, we introduce the {\it 
winding number}; Given a loop $\gamma:I\ra S^1$, we look at the covering 
${\Cal U}_{x+}=\{(x,y)\in S^1 : x>0\},{\Cal U}_{x-}=\{(x,y)\in S^1 : x<0\},
{\Cal U}_{y+}=\{(x,y)\in S^1 : y>0\},{\Cal U}_{y-}=\{(x,y)\in S^1 : y<0\}$
and choose a partition $x_0=0,<x_1<\cdots <x_k=1$ of $I$ so that $\gamma|_{[x_i,x_{i+1}]}$
maps into one of the ${\Cal U}$'s (by Lebesgue number). Then for each $i$ let
$\theta_i=$ the angle (strictly between $-\pi$ and $\pi$) between the rays from the
origin through $\gamma(x_i)$ and $\gamma(x_{i+1})$ (measured from the first to the 
second). Finally, let $w(\gamma)=\sum\theta_i/2\pi$. There are three things to
prove:

\ssk

(1) $w(\gamma)$ is independent of the partition used to compute it. This is a standard
trick: given two partitions, show that both compute the same number as the union of the two
partitions. This follows by showing the the number doesn't change by adding a single
extra point to a partition (which is immediate: don't change the ${\Cal U}$'s you 
assign (noting that it makes no difference to the angle computation if you switch 
between two such that can be assigned), and note that angles add).

\ssk

(2) If $\gamma\simeq \beta$, then $w(\gamma)=w(\beta)$. This is also a standard sort of argument;
the basic idea is that a homotopy can be thought of as a long sequence of ``small'' homotopies.
Given the homotopy $H:I\times I\ra S^1$, a Lebesgue number argument, applied to the same 
cover above, implies that there is an $\epsilon>0$ so that every square with side 
$<\epsilon$ maps into one of the four sets. Then choose an $n>1/\epsilon$ and partition $I\times I$
into $n$ rows of $n$ squares, each of which map into one of the four sets. 
Let the corners of these squares be denoted $(x_i,x_j)$. If we let
$\theta_{i,j}$ denote the angle between $(x_i,x_j)$ and $(x_{i+1},x_j)$ and
$\varphi_{i,j}$ the angle between $(x_i,x_j)$ and $(x_{i},x_{j+1})$, then since the 
four corners of a square are all in the same set and angles add, we find that 
$\theta_{i,j}+\varphi_{i+1,j}=\varphi_{i,j}+\theta_{i,j+1}$, so
$\theta_{i,j}+\varphi_{i+1,j}-\varphi_{i,j}=\theta_{i,j+1}$. Summing both sides over $i$, most
of the left terms telescope, and since $\varphi_{0,j}=\varphi_{n,j}=0$ (since these lie
on the vertical sides, where the homotopy is constant), we find that
$w(H|_{I\times\{x_j\}})=\sum \theta_{i,j} = \sum \theta_{i,j+1} = w(H|_{I\times\{x_{j+1}\}})$
So, by induction, $w(\gamma)=w(H|_{I\times\{x_0\}})=w(H|_{I\times\{x_{n+1}\}})=w(\beta)$.

\ssk

(3) Each of our canonical forms have different winding numbers. This is immediate;
they can be computed to be $0$, $n$, and $-n$, respectively.

\msk

Together these facts imply that $\pi_1(S^1)\cong \bbz$. Note that, in fact, The map
$[\gamma]\mapsto w(\gamma)$ \underbar{is} our isomorphism, but we couldn't know that
without both parts of the argument. The second part could be extended to show that
this map is a homomrphism, and onto, but the first part is needed to show that it 
is injective, i.e.., loops with the same winding number are both homotopic to the
same canonical form.

\msk

This is a very fundamental (pardon the pun) computation in homotopy theory, and 
a great deal can be proved just this one fact. It is also the basis for nearly 
every other fundamental group calculation that we will do. Spheres and disk make up the
basic building blocks for the topological spaces which we will study in this course,
and as we shall see the circle is the only one of these whose fundamental group 
is non-trivial, and so the fundamental group of every space is founded upon the
circles that are built into its initial construction. To formalize this, we 
need to understand how that fundamental group of a space can be assembled out of
the fundamental groups of the pieces used to build it. The basic idea, formalized
in the Seifert-van Kampen Theorem, is that if $X=A\cup B$, and we understand 
the fundamental groups of $A, B$, and $A\cap B$, then we can compute $\pi_1(X)$
from this. But before embarking on this line of thought, let us first put our
computation $\pi_1(S^1)\cong \bbz$ to work.

\msk

One of the standard results of calculus is that the intermediate value theorem 
implies that every map $f:I\ra I$ has a fixed point: $f(x_0)=x_0$ for some
$x_0\in I$. This has a higher-dimensional analogue:

\ssk

{\bf Brouwer Fixed Point Theorem:} Every map $f:D^2\ra D^2$ has a fixed point.
Proof: If not, then we can construct a retraction $r:D^2\ra \del D^2$ by
sending $x\in D^2$ to the point on $\del D^2$ lying on the ray from $f(x)$ to $x$
(this uses the hypothesis that $f(x)\neq x$);
such a ray intersects the boundary in exactly one point. A little analytic
geometry will allow you to write down a formula for this map, which uses
only elementary operations and the function $f$, so $r$ is continuous.
But a retraction induces a surjective homomorphism on $\pi_1$, so $r_*$
is a surjection from $\pi_1(D^2)=1$ to $\pi_1(\del D^2)=\pi_1(S^1)=\bbz$,
a contradiction. So $f$ must have a fixed point.

\msk

Another quick result using $\pi_1(S^1)\cong \bbz$ is the {\bf Fundamental Theorem
of Algebra:} Every non-constant polynomial (with complex coefficients)
has a complex root: for every $f(z)=a_nz^n+\cdots +a_0$ with $n\geq 1$ and $a_n\neq 0$,
$a_i\in{\Bbb C}$, there is a $z_0\in{\Bbb C}$ with $f(z_0)=0$.For, thinking
of ${\Bbb C}=\bbr^2$, if not, then $f$ is a map $\bbr^2\ra \bbr^2\setminus\{0\}$.
We can divide through by $a_n$ without affeting this, and assume that $f$ is monic.
Setting $\gamma_m(t)=f(m\cos(2\pi t),m\sin(2\pi t))$, then thought of as a map 
of the circle into $\bbr^2\setminus\{0\}$, it manifestly extends to a map of the 
disk $D^2$, as $\Gamma_n(x)=f(mx)$, so $\gamma_m$ is null-homotopic for all $m$. But
$\bbr^2\setminus\{0\}$ deformation retracts to the unit circle (the retraction
is $r(z)=z/|z|$), so 
$\pi_1(\bbr^2\setminus\{0\})\cong \bbz$, and by the above all of the $[\gamma_m]$
represent $0$ in $\bbz$, and so $r_*[\gamma_m]=[r\circ\gamma_m]=0$, as well. 
But for large $m$, it turns out, we can compute
$w(r\circ\gamma_m)=n$, which, since $n\geq 1$, is a contradiction. To see this,
we write $\gamma_m(t)=f(me^{2\pi it})=
m^n(e^{2\pi nit}+(a_{n-1}/m)e^{2\pi (n-1)it}+\cdots +(a_0/m^n)) = m^n(e^{2\pi nit}+R(m,t))$,
so $r\circ\gamma_m(t)=(e^{2\pi nit}+R(m,t))/|e^{2\pi nit}+R(m,t)|$.
But as $m\ra \infty$, $R(m,t)\ra 0$ uniformly in $t$, since every term in 
$R(m,t)$ is a number with constant norm divided by a positive power of $m$. So for 
large enough $m$ $|R(m,t)|<1/2$ for all $t$, and then 
for every $s\in I$, $|e^{2\pi nit}+sR(m,t)|\neq 0$, since this could be $0$
only if $e^{2\pi nit}=-sR(m,t)$, which is impossible since the left-hand side has
norm $1$ and the right has norm at most $1/2$. Then the homotopy
$H(t,s)=(e^{2\pi nit}+sR(m,t))/|e^{2\pi nit}+sR(m,t)|$ is well-defined and
continuous, $H:I|times I\ra S^1$, and defines a homotopy from 
$\alpha:t\mapsto e^{2\pi nit}$ (at $s=0$) to $r\circ\gamma_m$ (at $s=1$).
Since $w(\alpha)=n$, we have shown that, for large enough $m$,
$w(r\circ\gamma_m)=n$. This contradiction implies that $f$ must have a root, as
desired.

\msk


{\bf Group theory ``done right'': presentations}

\msk

Our next task is to build up machinery for computing the fundamental group of 
still more spaces. The basic idea is that if we understand how a space it built
up out of simpler pieces, then its fundamental group is similarly built up out
of ``simpler'' groups. This building up of groups is best understood in the
language of combinatorial group theory, using presentations of groups by
generators and relations. 

\msk

{\it Free groups:} $\Sigma$ = a set; a {\it reduced word} on \msig\ is a (formal)
product $a_1^{\ep_1}\cdots a_n^{\ep_n}$ with $a_i\in\sig$ and $\ep_i=\pm 1$,
and either $a_i\neq a_{i+1}$ or $\ep_i\neq -\ep_{i+1}$ for every $i$. (I.e., no
$aa^{-1},a^{-1}a$ in the product.)

\ssk

The free group $F(\sig)$ = the set of reduced words, with multiplication = concatenation 
followed by reduction; remove all possible $aa^{-1},a^{-1}a$ from the site of concatenation.

\ssk

identity element = the empty word, 
$(a_1^{\ep_1}\cdots a_n^{\ep_n})^{-1} = a_n^{-\ep_n}\cdots a_1^{-\ep_1}$. 
$F(\sig)$ is generated by \msig, with no relations among the generators
other than the ``obvious'' ones.

\msk

Important property of free groups: any function $f:\sig\ra G$ , $G$ a group, extends
uniquely to a homomorphism $\phi: F(\sig)\ra G$.

\msk

If $R\sset F(\sig)$, then $<R>^N$ = normal subgroup generated by $R$ 

= $\displaystyle \{\prod_{i=1}^n g_i r_i g_i^{-1} : n\in{\Bbb N}_0 , g_i\in F(\sig) , r_i\in R\}$

=smallest normal subgroup containing $R$.

\ssk

$F(\sig)/<R>^N$ = the group with {\it presentation} $<\sig | R>$ ; it is the largest quotient
of $F(\sig)$ in which the elements of $R$ are the identity. Every group has a presentation:

\ctln{$G$ = $F(G)/<gh(gh)^{-1} : g,h\in G>^N$}

where $(gh)$ is interpreted as a single letter in $G$.

\msk

If $G_1=<\sig_1 | R_1>$ and $G_2=<\sig_2 | R_2>$, then their {\it free product}
$G_1*G_2 = <\sig_1\coprod\sig_2 | R_1\cup R_2>$ ($\sig_1,\sig_2$ must be 
treated as (formally) disjoint). Each element has a unique reduced form as
$g_1\cdots g_n$ where the $g_i$ alternate from $G_1,G_2$.
$G_1,G_2$ can be thought of as subgroups for $G_1*G_2$, in the obivous way.
Important property of free products: any pair of
homoms $\phi_i:G_i\ra G$ extends uniquely to a homom $\phi:G_1*G_2\ra G$
(exactly the way you think it does).

\bsk



Gluing groups: given groups $G_1,G_2$, with subgroups $H_1,H_2$ that are
isomorphic $H_1\cong H_2$, how can we ``glue'' $G_1$ and $G_2$ together along their
``common'' subgroup? More generally (and with our eye on van Kampen's Theorem)
given a group $H$ and homomorphisms $\phi_1 : H\ra G_i$, we wish to build the largest group ``generated'' by
$G_1$ and $G_2$, in which $\phi_1(h)=\phi_2(h)$ for all $h\in H$. 

\msk

We can do this by starting
with $G_1*G_2$ (to get the first part), and then take a quotient to insure that 
$\phi_1(h)(\phi_2(h))^{-1} =1$ for every $h$. Using presentations 
$G_1=<\sig_1 | R_1>$ , $G_2=<\sig_2 | R_2>$ , if we insist on quotienting out by 
as little as possible to get our desired result, we can do this very succinctly as

\msk

\ctln{$G = (G_1*G_2)/<\phi_1(h)(\phi_2(h))^{-1} : h\in H>^N = 
<\sig_1\coprod\sig_2 | R_1\cup R_2\cup\{\phi_1(h)(\phi_2(h))^{-1} : h\in H\}>$}
 
\msk

This group $G= =G_1*_HG_2$ is the {\it largest} group generated by $G_1$ and $G_2$ in which 
$\phi_1(h)=\phi_2(h)$ for all $h\in H$, and is called the {\it amalgamated 
free product} or {\it free product with amalgamation (over $H$)} . [{\bf Warning!} 
Group theorists will generally use this term only if both homoms $\phi_1,\phi_2$
are injective. (This insures that the natural maps of $G_1,G_2$ into $G_1*_HG_2$
are injective.) But we will use this term for all $\phi_1,\phi_2$. (Some people use the term {\it pushout}
in this more general case.)]

\msk

Important special cases : $G*_H\{1\} = G/<\phi(H)>^N = <\sig | R\cup \phi(H)>$ , and
$G_1*_\{1\}G_2 \cong G_1*G_2$

\bsk

The relevance to \mpu : the Seifert-van Kampen Theorem.

\msk

If we express a topological space as the union $X=X_1\cup X_2$, then we have 
inclusion-induced homomorphisms 

\ctln{$j_{1*}: \pu(X_1)\ra \pu(X)$ , $j_{2*}: \pu(X_2)\ra \pu(X)$}

 - to be precise, we should choose a common basepoint in $A=X_1\cap X_2$. This 
in turn gives a homomorphism $\phi:\pi(X_1)*\pu(X_2)\ra \pu(X)$ . Under the
hypotheses

\ctln{$X_1,X_2$ are open, and $X_1,X_2,X_1\cap X_2$ are path-connected}

we can see that this homom is onto:

\msk

Given $x_0\in X_1\cap X_2$ and a loop $\gamma:(I,\del I)\ra (X,x_0)$, 
we wish to show that it is homotopic rel endpoints to a product of
loops which lie alternately in $X_1$ and $X_2$. But 
$\{\gamma^{-1}(X_1),\gamma^{-1}(X_2)\}$ is an open cover of the compact
metric space $I$, and so there is an $\ep > 0$  (a {\it Lebesgue number})
so that every interval of
length $\ep$ in $I$ lies in one of these two sets, i.e., maps, under $\gamma$,
into either $X_1$ or $X_2$. If we set $N=\lceil 1/\ep\rceil$, then 
setting $a_i=i/N$, then we get a sequence of intervals $J_i=[a_i,a_{i+1}], i=0,\ldots N-1$, 
each mapping
into $X_1$ or $X_2$. If $J_i$ and $J_{i+1}$ both map into the same subpace,
replace them in the sequence with their union. Continuing in this fashion, reducing the number of 
subintervals by one each time, we will eventually
find a collection $I_k$, $k=1,\ldots m$, of intervals covering $I$, 
overlapping only on their
endpoints, which alternately map into $X_1$ and $X_2$. Their endpoints, 
therefore, all map into $X_1\cap X_2$. Setting $y_k=\gamma(I_k\cap I_{k+1})$,
we can, since $X_1\cap X_2$ is path-connected, find a path $\delta_k:I\ra X_1\cap X_2$ 
with $\delta_k(0)=y_k$ and $\delta_k(1)=x_0$. Choosing our favorite homeomorphisms
$h_k:I\ra I_k$ and defining $\eta_k=\gamma|_{I_k}\circ h_k$, we have that, in $\pu(X,x_0)$,

\ctln{$[\gamma]=[\eta_1 * \cdots * \eta_m] 
= [\eta_1*(\delta_1*\overline{\delta_1})*\eta_2* \cdots *\eta_{m-1}* (\delta_{m-1}*\overline{\delta_{m-1}})*\eta_m]$}

\ctln{= $[\eta_1*\delta_1][\overline{\delta_1}*\eta_2*\delta_2] \cdots 
[\overline{\delta_{m-2}}*\eta_{m-1}* \delta_{m-1}][\overline{\delta_{m-1}}*\eta_m]$}


We can insert the $\delta_k*\overline{\delta_k}$ into these products because each is
homotopic to the constant map, and $\eta_k*$(constant) is homotopic to $\eta_k$ by the same sort of homotopy
that established that the constant map represents the identity in the fundamantal group.

\msk

This results in a product of loops (based at $x_0$) which alternately lie in $X_1$ and $X_2$. This product can
therefore be interpreted as lying in $\pi(X_1)*\pu(X_2)$, and maps, under $\phi$, to $[\gamma]$ .
$\phi$ is therefore onto, and
$\pu(X)$ is isomorphic to the free product modulo the kernel of this map $\phi$. 

\msk

Loops $\gamma:(I,\del I)\ra (A,x_0)$, can, using the inclusion-induced maps  
$i_{1*}:\pu(A)\ra \pu(X_1)$ , $i_{2*}:\pu(A)\ra \pu(X_2)$, be thought as either in 
$\pu(X_1)$ or $\pu(X_2)$ . So the word $i_{1*}(\gamma)(i_{2*}(\gamma))^{-1}$, in 
$\pi(X_1)*\pu(X_2)$, is set to the identity in $\pu(X)$ under $\phi$. So these 
elements lie in the kernel of $\phi$.

\msk

{\bf Seifert - van Kampen Theorem:} $\ker(\phi) = <i_{1*}(\gamma)(i_{2*}(\gamma))^{-1} : \gamma\in\pu(A) >^N$,
so $\pu(X)\cong \pu(X_1)*_{\pu(A)}\pu(X_2)$ . 

\bsk

Before we explore the proof of this, let's see what we can do with it!

\msk

{\bf Fundamental groups of graphs:} Every finite connected graph $\Gamma$ has a {\it maximal tree} $T$,
a connected subgraph with no simple circuits. Since any tree is the 
union of smaller trees joined at a vertex, we can, by induction, show that 
$\pu(T) = \{ 1\}$ . In fact, if $e$ is an outermost edge of $T$, then 
$T$ deformation retracts to $T\setminus e$, so, by induction, $T$ is 
contractible. Consequently ({\it Hatcher, Proposition 0.17}), $\Gamma$ and the quotient space $\Gamma/T$
are homotopy equivalent, and so have the same \mpu . But $\Gamma/T=\Gamma_n$
is a bouquet of $n$ circles for some $n$. If we let ${\Cal U}$ = a neighborhood of 
the vertex in $\Gamma_n$, which is contractible, then, by singling out one petal of the bouquet,
we have

\ssk

\ctln{$\Gamma_n = (\Gamma_{n-1}\cup{\Cal U})\cup (\Gamma_1\cup{\Cal U}) = X_1\cup X_2$}

\ssk

with $\Gamma_{k}\cup{\Cal U}\sim (\Gamma_{k}\cup{\Cal U})/{\Cal U}\cong \Gamma_{k}$. 
And since $X_1\cap X_2={\Cal U}\sim *$, we have that 

\ctln{$\pu(\Gamma_{n}) \cong \pu(\Gamma_{n-1})*_{1}\pu(\Gamma_1) = \pu(\Gamma_{n-1})*{\Bbb Z}$}

So, by induction, $\pu(\Gamma) \cong \pu(\Gamma_{n})\cong {\Bbb Z}*\cdots *{\Bbb Z} = F(n)$, the free group on $n$ letters, where $n$ = the number of edges not in a maximal tree for $\Gamma$. The generators for the group consist of the
edges not in the tree, prepended and appended by paths to the basepoint.

\msk

{\it Gluing on a 2-disk:} If $X$ is a topological space and $f:\del {\Bbb D}^2\ra X$ is continuous, then we
can construct the quotient space $Z=(X\coprod {\Bbb D}^2)/\{x\simeq f(x) : x\in\del{\Bbb D}^2\}$,
the result of gluing ${\Bbb D}^2$ to $X$ along $f$. 
We can use Seifert - van Kampen to compute \mpu\ of the resulting space, although if we
wish to be careful with basepoints $x_0$ 
(e.g., the image of $f$ might not contain $x_0$, and/or we
may wish to glue several disks on, in remote parts of $X$),
we should also include a rectangle $R$, the mapping cylinder of a path $\gamma$ running from 
$f(1,0)$ to $x_0$, glued to 
${\Bbb D}^2$ along the arc from $(1/2,0)$ to $(1,0)$ (see figure). 
This space $Z_+$ deformation retracts to $Z$, but it
is technically simpler to do our calculations with the basepoint $y_0$ lying above $x_0$.
If we write $D_1 = \{x\in {\Bbb D}^2 : ||x||<1\}\cup(R\setminus X)$ 
and $D_2 = \{x\in {\Bbb D}^2 : ||x||>1/3\}\cup R$ , then we can write $Z_+=D_+\cup(X\cup D_2) = X_1\cup X_2$.
But since $X_1\sim *$ , $X_2\sim X$ 
(it is essentially the mapping cylinder of the maps $f$ and $\gamma$ )
and $X_1\cap X_2 = \{x\in {\Bbb D}^2 : 1/3<||x||<1\}\cap(R\setminus X)\simeq S^1$, we find that 

\ctln{$\pu(Z,y_0)\cong \pu(X_2,y_0)*_{\Bbb Z}\{1\} = \pu(X_2)/<{\Bbb Z}>^N \cong \pu(X_2)/<[\overline{\delta}*\overline{\gamma}*f*\gamma*\delta]>^N$}

If we then use $\delta$ as a path for a change of basepoint isomorphism, and then a homotopy
equivalence from $X_2$ to $X$ (fixing $x_0$), we have, in terms of group presentations, 
if $\pu(X,x_0)=<\sig | R>$ , then $\pu(Z) = <\sig | R\cup\{[\overline{\gamma}*f*\gamma]\}>$ . 
So the effect of gluing on a 2-disk on the fundamental group is to add a new relator, 
namely the word represented by the attaching map (adjusting for basepoint).

\msk


\vbox{\hsize=5.8in

\leavevmode

\epsfxsize=5.8in
\epsfbox{0201f1.ai}}



\msk

This in turn opens up huge possibilities for the computation of $\pu(X)$. For example, for cell complexes,
we can inductively compute \mpu\ by starting with the 1-skeleton, with free fundamental group, and 
attaching the 2-cells one by one, which each add a relator to the presentation of $\pu(X)$ . [{\bf Exercise:}
(Hatcher, p.53, \#\ 6) Attaching $n$-cells, for $n\geq 3$, has no effect on \mpu .] As a specific example, 
we can compute the fundamental group of any compact surface.

\bsk

{\bf CW complexes:} The ``right'' spaces to do algebraic topology on.

\msk

The basic idea: CW complexes are built inductively, by gluing 
disks onto lower-dimensional strata. $X=\bigcup X^{(n)}$, where

\ssk

$X^{(0)}$ = a disjoint union of points, and, inductively,

\ssk

$X^{(n)}$ is built from $X^{(n-1)}$ by gluing $n$-disks $D^n_i$
along their boundaries. That is we have $f_i:\del D^n_i\ra X^{(n-1)}$
and $X^{(n)}=X^{(n-1)}\cup(\coprod D^n_i)/\sim$ where
$x\sim f_i(x)$ for all $x\in\del D^n_i$. We have (natural)
inclusions $X^{(n-1)}\subseteq X^{(n)}$, and $X=\bigcup X^{(n)}$
is given the {\it weak topology}; that is,  $C\subseteq X$ is closed \lra\
$C\cap X^{(n)}$ is closed for all $n$. 

(Note: this is reasonable;
$X^{(n-1)}$ is closed in $X^{(n)}$ for all $n$.)

\ssk

Each disk $D^n_i$ has a {\it characteristic map} $\phi_i:D^n_i\ra X$
given by 

$D^n_i\ra X^{(n-1)}\cup(\coprod D^n_i)\ra X^{(n)}\subseteq X$.

$f:X\ra Y$ is cts \lra\ $f\circ \phi_i:D^n_i\ra X\ra Y$ is cts for 
all $D^n_i$. 

(This is a consequence of using the weak topology.) 

\msk

A {\it CW pair} $(X,A)$ is a CW complex $X$ and a {\it subcomplex}
$A$, which is a subset which is a union of images of cells, so it is 
a CW complex in its own right. We can induce CW structures under
many standard constructions; e.g., if $(X,A)$ is a CW pair, then
$X/A$ admits a CW structure whose cells are $[A]$ and the cells of 
$X$ not in $A$. We can glue two CW complexes $X,Y$ along isomorphic
subcomplexes $A\subseteq X,Y$, yielding $X\cup_AY$.

\msk

``CW''=closure finiteness, weak topology

\msk

Perhaps the most important property of CW complexes (for algebraic topology,
anyway) is the {\it homotopy extension property}; given a CW pair
$(X,A)$, a map $f:X\ra Y$, and a homotopy $H:A\times I\ra Y$ such that
$H|_{A\times 0}=f|_A$, there is a homotopy (extension)
$K:X\times I\ra Y$ with $K|_{A\times I}=H$. This is because 
$B=X\times\{0\}\cup A\times I$ is a retract of $X\times I$; $K$ is the 
composition of this retraction and the ``obvious'' map from $B$ to $Y$.

\ssk

To build the retraction, we do it one cell of $X$ at a time. The idea is that
the retraction is defined on the cells of $A$ (it's the identity), so look
at cells of $X$ not in $A$. Working our way up in dimension, we can assume
the the retraction $r_{n-1}$
is defined on (the image of) $\del D^n\times I$, i.e., on $X^{(n-1)}\times I$.
But $D^n\times I$ (strong deformation) retracts onto 
$D^n\times 0\cup \del D^n\times I$; composition of $r_{n-1}$ with 
this retraction extends the retraction over $\phi(D^n)\times I$, and 
so over $X^{(n)}\times I$..

\msk

This, for example, lets us show that if $(X,A)$ is a CW pair and 
$A$ is contractible, then $X/A\simeq X$. This is because the
composition $A\ra *\ra A$ is homotopic to the identity $I_A$, via some map
$H:A\times I\ra A$, with $H|_{A\times 0}=I_A$. Thinking of $H$ as
mapping into $X$, then together with the map $I_X:X\ra X$ the HEP
provides a map $K:X\times I\ra X$ with $K_0=I_X$ and $K_1(A)=*$.
Setting $K_1=g:X\ra X$, it induces a map $h:X/A\ra X$. This is a homotopy
inverse of the projection $p:X\ra X/A$:

\msk

$h\circ p=g\simeq I_X$ via $K$, and $p\circ h:X/A\ra X/A$
is homotopic to $I_{X/A}$ since $K_t(A)\subseteq A$ for every $t$,
so induces a map $\overline{K}_t:X/A\ra X/A$, giving a homotopy
between 

$\overline{K}_0=\overline{I_X}=I_{X/A}$ and 
$\overline{K}_1=\overline{g}=p\circ h$.

\bsk

{\bf Proving Seifert - van Kampen:}

\msk

We now turn our attention to proving Seifert - van Kampen; understanding
the kernel of the map $\phi : \pu(X_1)*\pu(X_2)\ra\pu(X)$ , 
under the hypotheses
that $X_1,X_2$ are open, $A=X_1\cap X_2$ is path-connected, and the
basepoint $x_0\in A$ . So we start with a product $g = g_1\cdots g_n$ 
of loops alternately in $X_1$ and $X_2$, which when thought of in $X$
is null-homotopic. We wish to show that $g$ can be expressed as a 
product of conjugates of elements of the form $i_{1*}(a)(i_{2*}(a))^{-1}$
(and their inverses). The basic idea is that a ``big'' homotopy can be
viewed as a large number of ``little'' homotopies, which
we essentially deal with one at a time, and we find out how
little ``little'' is by using the same Lebesgue number agument that we
used before.

\msk

Specifically, if $H$ is the homotopy, rel basepoint, from 
$\gamma_1*\cdots *\gamma_n$,
where $\gamma_i$ is a based loop representing $g_i$, and the constant
loop, then, as before, $\{H^{-1}(X_1),H^{-1}(X_2)\}$ is an open cover 
of $I\times I$, and so has a Lebesgue number $\ep$. If we cut
$I\times I$ into subsquares, with length $1/N$ on a side, where $1/N<\ep$,
then each subsquare maps into either $X_1$ or $X_2$. The idea is to 
think of this as a collection of horizontal strips, each cut into squares.
Arguing by induction, starting from the bottom (where our conclusion
will be obvious), we will argue that if the bottom of the strip
can be expressed as an element of the group 

$N = <i_{1*}(\gamma)(i_{2*}(\gamma))^{-1} : 
\gamma\in\pu(A) >^N\sset \pu(X_1)*\pu(X_2)$ 

(i.e., as a product
of conjugates of such loops), then so can the 
top of the strip. 

\msk

\leavevmode

\epsfxsize=4in
\ctln{{\epsfbox{0203f1.ai}}}

\msk

And to do \underbar{this}, we work as before. We have a strip of squares,
each mapping into either $X_1$ or $X_2$. If adjacent squares map into the
same subpace, amalgamate them into a single larger rectangle. Continuing
in this way, we can break the strip into subrectangles which
alternately map into $X_1$ or $X_2$. This means that the vertical arcs
in between map into $X_1\cap X_2 = A$, and represent paths $\eta_i$ in 
$A$. Their endpoints also map into $A$,
and so can be joined by paths ($\delta_i$  on the top, $\epsilon_i$ 
on the bottom) in $A$ to the basepoint. The top of the strip is
homotopic, rel basepoint, to 

$(\alpha_1*\delta_1)*(\overline{\delta_1}*\alpha_2*\delta_2)*\cdots *
(\overline{\delta_{k-1}}*\alpha_k)$

each grouping mapping into either $X_1$ or $X_2$.
The rectangles demonstrate that each grouping is homotopic, rel basepoint,
to the product of loops

$(\overline{\delta_i}*\eta_i*\epsilon_i)*
(\overline{\epsilon_i}*\beta_i*\epsilon_{i+1})*
(\overline{\epsilon_{i+1}}*\overline{\eta_{i+1}}*\delta_{i+1})
=a_i b_i a_{i+1}^{-1}$

where this is thought of as a product in either $\pu(X_1)$ or
$\pu(X_2)$. The point is that when strung together, this appears
to give $(b_1a_2^{-1})(a_2b_2a_3^{-1})\cdots (a_kb_k)$ , with lots
of cancellation, but in reality, the terms $a_i^{-1}a_i$ represent
elements of $N$, since the two ``cancelling'' 
factors are thought of as living in the
different groups $\pu(X_1),\pu(X_2)$. The remaining terms, if we
delete these ``cancelling'' pairs, is 
$b_1\cdots b_k = 
\beta_1*\epsilon_1*\cdots *\overline{\epsilon_i}*\beta_i*\epsilon_{i+1}
*\cdots * \overline{\epsilon_k}*\beta_k$, which is homotopic
rel endpoints to $\beta_1*\cdots *\beta_k$, which, by induction, can 
be represented as a product which lies in $N$. 

\msk




\leavevmode


\epsfxsize=3.5in
\ctln{{\epsfbox{0203f2.ai}}}

\bsk

So, we can obtain the element represented by the top of the strip by 
inserting elements of $N$ into the bottom, which 
is a word having a representation as an 
element of $N$. The final problem to overcome is that the
insertions represented by the vertical arcs might not be occuring
where we want them to be! But this doesn't matter; inserting a word $w$
in the middle of another $uv$ (to get $uwv$) 
is the same as multiplying $uv$ by a conjugate of $w$;
$uwv = (uv)(v^{-1}wv)$, so since the bottom of the strip is 
in $N$, and we obtain the top of the strip by inserting elements
of $N$ into the bottom, the top is represented by a product of 
conjugates of elements of $N$, so (since $N$ is normal) is in $N$.
And a \underbar{final} final point; the subrectangles may not have 
cut the bottom of the strip up into the same pieces that the inductive
hypothesis used to express the bottom as an element of $N$. It didn't even 
cut it into loops; we added paths at the break points to make that happen.
The inductive hypothesis would have, in fact, added its own extra paths,
at possibly different points!
But if we add \underbar{both} sets of paths, 
and cut the loop up into even more pieces, 
then we end up with a loop, which we have expressed as a product in
$\pu(X_1)*\pu(X_2)$ in two (possibly different) ways, since the 
two points of view will have interpreted pieces as living in 
different subspaces. But when this happens, it must be because
the subloop really lives in $X_1\cap X_2=A$. Moving from one
to the other amounts to repeatedly changing ownership between the 
two sets, which in $\pu(X_1)*\pu(X_2)$ means \underbar{inserting}
an element of $N$ into the product (that is literally what elements
of $N$ do). But as before, these insertions can be collected at one
end as products of conjugates. So if one of the elements is in $N$, 
the other one is, too.

\msk

Which completes the proof! 

\bsk


The inherent complications above derived from needing open
sets can be legislated away, by introducing additional
hypotheses:

\msk

{\bf Theorem:} If $X=X_1\cup X_2$ is a union of closed
sets $X_1,X_2$, with $A=X_1\cap X_2$ path-connected, 
and if $X_1,X_2$ have open neighborhood ${\Cal U}_1,{\Cal U}_2$
so that ${\Cal U}_1,{\Cal U}_2,{\Cal U}_1\cap{\Cal U}_2$
deformation retract onto $X_1,X_2,A$ respectively, then 
$\pu(X)\cong \pu(X_1)*_{\pu(A)}\pu(X_2)$ as before.

\msk

The hypotheses are satisfied, for example, if $X_1.X_2$ are subcomplexes of the
cell complex $X$.

\msk

{\bf Some more computations:}

\msk

The {\it real projective plane} ${\Bbb R}P^2$ is the quotient of 
the 2-sphere $S^2$ by the antipodal map $x\mapsto -x$; 
it can also be thought of as the upper hemisphere, with 
identification only along the boundary. This in turn can be 
interpreted as a 2-disk glued to a circle, whose boundary
wraps around the circle twice. So $\pu({\Bbb R}P^2)\cong 
<a | a^2>\cong {\Bbb Z}_2 = {\Bbb Z}/2{\Bbb Z}$ .
A surface $F$ of genus 2 can be given a cell structure with 1 0-cell,
4 1-cells, and 1 2-cell, as in the figure, as in the first of the 
figures below. The fundamental group
of the 1-skeleton is therefore free of rank 4, and $\pu(F)$ has
a presentation with 4 generators and 1 relator. Reading the attaching
map from the figure, the presentation is $<a,b,c,d\ |\ [a,b][c,d] >$ .

\msk



\leavevmode

\epsfxsize=5.6in
\ctln{{\epsfbox{0201f2.ai}}}

\msk

Giving it a different cell structure, as in the second figure, with 2 0-cells,
6 1-cells, and 2 2-cells, after choosing a maximal tree, we can read off the
two relators from the 2-cells to arrive at a different presentation
$\pu(F) = <a,b,c,d,e\ |\ aba^{-1}eb^{-1},cde^{-1}c^{-1}d^{-1}>$ . A posteriori,
these two presentations describe isomorphic groups.

\bsk

Using the same technology, we can also see that, in general, 
any group is the fundamental group of some 2-complex $X$;
starting with a presentation $G = <\sig | R>$, build $X$ by starting
with a bouquet of $|\sig|$ circles, and attach $|R|$ 2-disks along
loops which represent each of the generators of $R$. (This works just
as well for infinite sets $\sig$ and/or $R$; essentially the same proofs
as above apply.)

\bsk

{\bf Wirtinger presentations for knot complements:}

\msk

A {\it knot} $K$ is (the image of) an embedding $h:S^1\hookrightarrow\bbr^3$. Wirtinger 
gave a prescription for taking a planar projection of $K$ and producing a presentation
of $\pi_1(\bbr^3\setminus K)=\pi_1(X)$. The idea: think of $K$ as lying on the projection
plane, except near the crossings, where it arches under itself. 
We build a CW-complex $Y\subseteq X$ that $X$ deformation retracts to. A presentation for $\pi_1(Y)$
gives us $\pi_1(X)$.

\msk

\ctln{\vbox{\hsize=2in
\leavevmode
\epsfxsize=2in
\epsfbox{wirtinger1a.ai}}
\hskip.5in
\vbox{\hsize=2in
\leavevmode
\epsfxsize=2in
\epsfbox{wirtinger2.ai}}}
\ssk


To build $Y$, glue rectangles arching under the strands of $K$ to a horzontal plane lying just above the projection plane of $K$. 
At the crossing, the rectangle is glued to the rectngle arching under the over-strand. $X$ deformation retracts to $Y$; the top half 
of $\bbr^3$ deformation retracts to the top plane, the parts of $X$ inside the tubes formed by the rectangles
radially retract to the boundaries of the tubes, and the bottom part of $X$ vertically retracts onto $Y$.
Formally, we should really keep a ``slab'' above the plane, to give us a place to run arcs to a fixed
basepoint in the interior of the slab.

\vfill
\eject

We think of $Y$ as being built up from the slab $C$, by gluing on annuli $A_i\cong S^1\times I$, one for each rectangle $R_i$ glued on;
the rectangle $S_i$ lying above $R_i$ in the bottom of the slab $C$ is the other half of the annulus. Then we glue on the 2-disks $D_j$,
one for each crossing of the knot projection. A little thought shows that there are as many annuli as disks;
the annuli correspond to the unbroken strands of the knot projetion, which each have two ends, and each crossing is where
two ends terminate (so there are two ends for every $A_i$ and two ends for every $D_j$, so there are half as
many of each as there are total number of ends). To make sure that all of our interections are path connected,
and to formally use a single basepoint in all of our computations, we join every one of the annuli and disks to a basepoint lying
in the slab by a collection of (disjoint) paths.

\msk

Now starting with the slab (which is simply-connected), we begin to add the $A_i$ one at a time; each has fundamental group $\bbz$,
generated by a loop which travels once around the $S^1$-direction, and its intersection with $C\cup$ the
previously glued on annuli is the rectangle $S_i$, which is simply connected. So, inductively,
$\pi_1(C\cup A_1\cup\cdots \cup A_i)\cong \pi_1(C\cup A_1\cup\cdots \cup A_{i-1})*\pi_1(A_i)\cong F(i-1)*\bbz\cong F(i)$ is the 
free group on $i$ letters, so, adding all $n$ (say) of the annuli yields $F(n)$. To finish, we glue on the $n$ 2-disks $D_j$;
these amount to adding $n$ relators to the presentation $\langle x_1,\ldots,x_n | \rangle$. To determine these relators, we need
to \underbar{choose} specific generators for our $\pi_1(A_i)$; a standard choice is made by {\it orienting} the knot
(choosing a direction to travel around it) and choosing the loop which goes counter-clockwise around the annulus (when you
stand it vertically using the orientation of the strand it is going around. Then going around the boundary of the 2-disk
$D_j$ spells out the word $x_rx_sx_r^{-1}x_t^{-1}$, if the overstrand at the crossing corresponds to $A_r$ and the
understrand runs from $A_s$ on the left to $A_t$ on the right. [There is the possibility of the mirror image, when the
orientation of the strands goes from right to left under the overstrand; then the proper relator is $x_rx_s^{-1}x_r^{-1}x_t$ .]
Carrying this out for every 2-disk completes the presentation of $\pi_1(Y)\cong \pi_1(X)$.


\ctln{\vbox{\hsize=2in
\leavevmode
\epsfxsize=2in
\epsfbox{wirtinger3.ai}}}

\ssk

With practice, it becomes completely routine to read off a presentation for the 
fundamental group of $\bbr^3\setminus K$ from a projection of $K$. For example, from the projection above, we have

\ssk

$\pi_1(\bbr^3\setminus K)\cong \langle x_1,\ldots ,x_8 | x_8x_1=x_2x_8, x_2x_7=x_8x_2, x_5x_8=x_1x_5,
x_1x_5=x_6x_1, x_3x_6=x_7x_3 ,x_7x_2=x_3x_7 ,x_3x_2=x_2x_4, x_7x_4=x_5x_7\rangle$




\bsk

{\bf Postscript:  why should we care?} The role of the fundamental group
in distinguishing spaces has already been touched upon; if two 
(path-connected) spaces have non-isomorphic fundamental groups, then
the spaces are not homeomorphic, and even not homotopy equivalent.
It is one of the most basic, and in many cases the best such invariant
we will have in our arsenal
(hence the name ``fundamental''). As we have seen with the circle, it
captures the notion of how many times a loop ``winds around'' in a space.
And the idea of using paths to understand a space is very basic; we 
explore a space by mapping familiar objects into it. (This is 
a theme we keep returning to in this course.) The concepts we 
have introduced play a role in analysis, for instance with the notion
of a path integral; the invariance of the integral under homotopies
rel endpoints is an important property, related to Green's Theorem
and (locally) conservative vector fields. And the \underbar{space}
of all paths in $X$ plays an important (theoretical, although
pprobably not practical) role in what we will do next.


\vfill
\end



