

\magnification=1200
\overfullrule=0pt
\parindent=0pt

%\nopagenumbers

\input amstex

%\voffset=-.6in
%\hoffset=-.5in
%\hsize = 7.5 true in
%\vsize=10.4 true in

%\voffset=1.4in
%\hoffset=-.5in
%\hsize = 10.2 true in
%\vsize=8 true in

\input colordvi

\def\cltr{\Red}		  % Red  VERY-Approx PANTONE RED
\def\cltb{\Blue}		  % Blue  Approximate PANTONE BLUE-072
\def\cltg{\PineGreen}	  % ForestGreen  Approximate PANTONE 349
\def\cltp{\DarkOrchid}	  % DarkOrchid  No PANTONE match
\def\clto{\Orange}	  % Orange  Approximate PANTONE ORANGE-021
\def\cltpk{\CarnationPink}	  % CarnationPink  Approximate PANTONE 218
\def\clts{\Salmon}	  % Salmon  Approximate PANTONE 183
\def\cltbb{\TealBlue}	  % TealBlue  Approximate PANTONE 3145
\def\cltrp{\RoyalPurple}	  % RoyalPurple  Approximate PANTONE 267
\def\cltp{\Purple}	  % Purple  Approximate PANTONE PURPLE

\def\cgy{\GreenYellow}     % GreenYellow  Approximate PANTONE 388
\def\cyy{\Yellow}	  % Yellow  Approximate PANTONE YELLOW
\def\cgo{\Goldenrod}	  % Goldenrod  Approximate PANTONE 109
\def\cda{\Dandelion}	  % Dandelion  Approximate PANTONE 123
\def\capr{\Apricot}	  % Apricot  Approximate PANTONE 1565
\def\cpe{\Peach}		  % Peach  Approximate PANTONE 164
\def\cme{\Melon}		  % Melon  Approximate PANTONE 177
\def\cyo{\YellowOrange}	  % YellowOrange  Approximate PANTONE 130
\def\coo{\Orange}	  % Orange  Approximate PANTONE ORANGE-021
\def\cbo{\BurntOrange}	  % BurntOrange  Approximate PANTONE 388
\def\cbs{\Bittersweet}	  % Bittersweet  Approximate PANTONE 167
%\def\creo{\RedOrange}	  % RedOrange  Approximate PANTONE 179
\def\cma{\Mahogany}	  % Mahogany  Approximate PANTONE 484
\def\cmr{\Maroon}	  % Maroon  Approximate PANTONE 201
\def\cbr{\BrickRed}	  % BrickRed  Approximate PANTONE 1805
\def\crr{\Red}		  % Red  VERY-Approx PANTONE RED
\def\cor{\OrangeRed}	  % OrangeRed  No PANTONE match
\def\paru{\RubineRed}	  % RubineRed  Approximate PANTONE RUBINE-RED
\def\cwi{\WildStrawberry}  % WildStrawberry  Approximate PANTONE 206
\def\csa{\Salmon}	  % Salmon  Approximate PANTONE 183
\def\ccp{\CarnationPink}	  % CarnationPink  Approximate PANTONE 218
\def\cmag{\Magenta}	  % Magenta  Approximate PANTONE PROCESS-MAGENTA
\def\cvr{\VioletRed}	  % VioletRed  Approximate PANTONE 219
\def\parh{\Rhodamine}	  % Rhodamine  Approximate PANTONE RHODAMINE-RED
\def\cmu{\Mulberry}	  % Mulberry  Approximate PANTONE 241
\def\parv{\RedViolet}	  % RedViolet  Approximate PANTONE 234
\def\cfu{\Fuchsia}	  % Fuchsia  Approximate PANTONE 248
\def\cla{\Lavender}	  % Lavender  Approximate PANTONE 223
\def\cth{\Thistle}	  % Thistle  Approximate PANTONE 245
\def\corc{\Orchid}	  % Orchid  Approximate PANTONE 252
\def\cdo{\DarkOrchid}	  % DarkOrchid  No PANTONE match
\def\cpu{\Purple}	  % Purple  Approximate PANTONE PURPLE
\def\cpl{\Plum}		  % Plum  VERY-Approx PANTONE 518
\def\cvi{\Violet}	  % Violet  Approximate PANTONE VIOLET
\def\clrp{\RoyalPurple}	  % RoyalPurple  Approximate PANTONE 267
\def\cbv{\BlueViolet}	  % BlueViolet  Approximate PANTONE 2755
\def\cpe{\Periwinkle}	  % Periwinkle  Approximate PANTONE 2715
\def\ccb{\CadetBlue}	  % CadetBlue  Approximate PANTONE (534+535)/2
\def\cco{\CornflowerBlue}  % CornflowerBlue  Approximate PANTONE 292
\def\cmb{\MidnightBlue}	  % MidnightBlue  Approximate PANTONE 302
\def\cnb{\NavyBlue}	  % NavyBlue  Approximate PANTONE 293
\def\crb{\RoyalBlue}	  % RoyalBlue  No PANTONE match
%\def\cbb{\Blue}		  % Blue  Approximate PANTONE BLUE-072
\def\cce{\Cerulean}	  % Cerulean  Approximate PANTONE 3005
\def\ccy{\Cyan}		  % Cyan  Approximate PANTONE PROCESS-CYAN
\def\cpb{\ProcessBlue}	  % ProcessBlue  Approximate PANTONE PROCESS-BLUE
\def\csb{\SkyBlue}	  % SkyBlue  Approximate PANTONE 2985
\def\ctu{\Turquoise}	  % Turquoise  Approximate PANTONE (312+313)/2
\def\ctb{\TealBlue}	  % TealBlue  Approximate PANTONE 3145
\def\caq{\Aquamarine}	  % Aquamarine  Approximate PANTONE 3135
\def\cbg{\BlueGreen}	  % BlueGreen  Approximate PANTONE 320
\def\cem{\Emerald}	  % Emerald  No PANTONE match
%\def\cjg{\JungleGreen}	  % JungleGreen  Approximate PANTONE 328
\def\csg{\SeaGreen}	  % SeaGreen  Approximate PANTONE 3268
\def\cgg{\Green}	  % Green  VERY-Approx PANTONE GREEN
\def\cfg{\ForestGreen}	  % ForestGreen  Approximate PANTONE 349
\def\cpg{\PineGreen}	  % PineGreen  Approximate PANTONE 323
\def\clg{\LimeGreen}	  % LimeGreen  No PANTONE match
\def\cyg{\YellowGreen}	  % YellowGreen  Approximate PANTONE 375
\def\cspg{\SpringGreen}	  % SpringGreen  Approximate PANTONE 381
\def\cog{\OliveGreen}	  % OliveGreen  Approximate PANTONE 582
\def\pars{\RawSienna}	  % RawSienna  Approximate PANTONE 154
\def\cse{\Sepia}		  % Sepia  Approximate PANTONE 161
\def\cbr{\Brown}		  % Brown  Approximate PANTONE 1615
\def\cta{\Tan}		  % Tan  No PANTONE match
\def\cgr{\Gray}		  % Gray  Approximate PANTONE COOL-GRAY-8
\def\cbl{\Black}		  % Black  Approximate PANTONE PROCESS-BLACK
\def\cwh{\White}		  % White  No PANTONE match


\loadmsbm

\input epsf

\def\ctln{\centerline}
\def\u{\underbar}
\def\ssk{\smallskip}
\def\msk{\medskip}
\def\bsk{\bigskip}
\def\hsk{\hskip.1in}
\def\hhsk{\hskip.2in}
\def\dsl{\displaystyle}
\def\hskp{\hskip1.5in}

\def\lra{$\Leftrightarrow$ }
\def\ra{\rightarrow}
\def\mpto{\logmapsto}
\def\pu{\pi_1}
\def\mpu{$\pi_1$}
\def\sig{\Sigma}
\def\msig{$\Sigma$}
\def\ep{\epsilon}
\def\sset{\subseteq}
\def\del{\partial}
\def\inv{^{-1}}
\def\wtl{\widetilde}
%\def\lra{\Leftrightarrow}
\def\del{\partial}
\def\delp{\partial^\prime}
\def\delpp{\partial^{\prime\prime}}
\def\sgn{{\roman{sgn}}}
\def\wtih{\widetilde{H}}
\def\bbz{{\Bbb Z}}
\def\bbr{{\Bbb R}}





{\bf Simplicial homology = singular homology:}
We have so far introduced two homologies; simplicial, $H_*^\Delta$, whose computation 
``only'' required some linear algebra,
and singular, $H_*$, which is formally less difficult to work with, and which, you may suspect by now, is also becoming
less difficult to compute... For $\Delta$-complexes, these homology groups are the same, $H_n^\Delta(X)\cong H_n(X)$
for every $X$. In fact, the isomorphism is induced by the inclusion $C_n^\Delta(X)\sset C_n(X)$. And we have
now assembled all of the tools necessary to prove this. Or almost; we need to note that most of the edifice we
have built for singular homology \u{could} have been built for simplicial homology, including relative 
homology (for a sub-$\Delta$-complex $A$ of $X$), and a SES of chain groups, giving a LES sequence for the pair,

\ssk

$\cdots \ra H_n^\Delta(A) \ra H_n^\Delta(X) \ra H_n^\Delta(X,A) \ra H_{n-1}^\Delta(A) \ra \cdots$

\ssk

The proof of the isomorphism between the two homologies proceeds by first showing that the
inclusion induces an isomorphism on $k$-skeleta, $H_n^\Delta(X^{(k)})\cong H_n(X^{(k)})$,
and this goes by induction on $k$ using the Five Lemma applied to the diagram

\ssk

\ctln{$\displaystyle 
\matrix 
H_{n+1}^\Delta(X^{(k)},X^{(k-1)})&\ra&H_n^\Delta(X^{(k-1)})&\ra&H_n^\Delta(X^{(k)}) & \ra & H_{n}^\Delta(X^{(k)},X^{(k-1)}) & \ra & H_{n-1}^\Delta(X^{(k-1)})\cr
\downarrow & & \downarrow & & \downarrow & & \downarrow & & \downarrow & \cr
H_{n+1}(X^{(k)},X^{(k-1)})&\ra&H_n(X^{(k-1)})&\ra&H_n(X^{(k)}) & \ra & H_{n}(X^{(k)},X^{(k-1)}) & \ra & H_{n-1}(X^{(k-1)}) \cr
\endmatrix$}

\ssk

The second and fifth vertical arrows are, by an inductive hypothesis, isomorphisms. The first and fourth vertical arrows are
isomorphisms because, essentially, we can, in each case, identify these groups. 
$H_{n}(X^{(k)},X^{(k-1)})\cong H_{n}(X^{(k)}/X^{(k-1)})\cong \widetilde{H}_n(\vee S^k)$
are either 0 (for $n\neq k$) or $\oplus \bbz$ (for $n=k$), one summand for each $n$-simplex in $X$. 
But the same is true for $H_{n}^\Delta(X^{(k)},X^{(k-1)})$; and for $n=k$ the generators are precisely
the $n$-simplices of $X$. The inclusion-induced map takes generators to generators, so is an isomorphism.
\hhsk So by the Five Lemma, the middle rows are also isomorphisms, completing our inductive proof.

\ssk

Returning to $H_n^\Delta(X) {\buildrel {I_*}\over \ra} H_n(X)$, we wish now to show that this map is an isomorphism.
Any $[z]\in H_n(X)$ is represented by a cycle $z=\sum a_i\sigma_i$ for $\sigma_i:\Delta^n\ra X$ . But each
$\sigma_i(\Delta^n)$ is a compact subset of $X$, and so meets only finitely-many cells of $X$. This is true for every
singular simplex, and so there is a $k$ for which all of the simplices map into $X^{(k)}$, and so we may
treat $z\in C_n(X^{(k)}$. Thought of in this way, it is still a cycle, and so $[z]\in H_n(X^{(k)})\cong H_n^\Delta(X^{(k)})$
so there is a $z^\prime in C_n^\Delta(X^{(k)})$ and a $w\in C_{n+1}(X^{(k)})$ with $i_\#z^\prime -z=\del w$. 
But thinking of  $z^\prime in C_n^\Delta(X)$ and $w\in C_{n+1}(X)$, we have the same equality, so 
$[z^\prime] \in H_n^\Delta(X)$ and $i_*[z^\prime] = [z]$ . So $i_*$ is surjective.
If $i_*([z]) = 0$, then the cycle $z=\sum a_i\sigma_i$ is a sum of characteristic maps of $n$-simplices of $X$, and
so can be thought of as an element of $C_n^\Delta(X^{n)})$ . Being $0$ in $H_n(X)$, $z=\del w$ for some
$w\in C_{n+1}(X)$ . But as before, $w\in C_n(X^{r)})$ for some $r$, and so thought of as an element of 
the image of the isomorphism $i_*: H_n^\Delta(X^{(r)})\ra H_n(X^{(r)})$, $i_*([z])=0$, so $[z]=0$ . So 
$z=\del u$ for some $u\in C_{n+1}^\Delta(X^{r)})\sset C_{n+1}^\Delta(X)$ . So $[z]=0$ in $H_n^\Delta(X)$.
Consequently, simplicial and singular homology groups are isomorphic.

\bsk






The isomorphism between simplicial and singular homology provides very quick proofs
of several results about singular homology, which would other would require some effort:

\ssk

{\it If the $\Delta$-complex $X$ has no simplices in dimension greater than $n$, then 
$H_i(X)=0$ for all $i>n$.}

\ssk

This is because the simplicial chain groups $C_i^\Delta(X)$ are $0$, so $H_i^\Delta(X)=0$ .

\ssk

{\it If for each $n$, the $\Delta$-complex $X$ has finitely many $n$-simplices, then 
$H_n(X)$ is finitely generated for every $n$.}

\ssk

This is because the simplicial chain groups $C_n^\Delta(X)$ are all finitely generated,
so $H_n^\Delta(X)$, being a quotient of a subgroup, is also finitely generated. [We
are using here that the number of generators of a subgroup $H$ of an {\it abelian} 
group $G$ is no larger than that for $G$; this is not true for groups in general!]


\bigskip

Some more topological results with homological proofs: The Klein bottle and real projective plane cannot 
embed in $\bbr^3$. This is because a surface $\Sigma$ embedded in $\bbr^3$ has a (the proper word is {\it normal})
neighborhood $N(\Sigma)$, which deformation retracts to $\Sigma$; literally, it is all points within a (uniformly) short distance
in the normal direction from the point on the surface $\Sigma$. Our non-embeddedness result follows (by contradiction)
from applying Mayer-Vietoris to the pair $(A,B) = (\overline{N(\Sigma)},\overline{\bbr^3\setminus N(\Sigma)})$, whose intersection
is the boundary $F=\del N(\Sigma)$ of the normal neighborhood. The point, though, is that
$F$ is an orientable surface; the outward normal (pointing away from $N(\Sigma)$) at every point, taken as
the first vector of a right-handed orientation of $\bbr^3$ allows us to use the other two vectors as an 
orientation of the surface. So $F$ is one of the surface $F_g$ above whose homologies we just computed.
This gives the LES
\hhsk
$\wtih_2(\bbr^3) \ra \wtih_1(F) \ra \wtih_1(A)\oplus \wtih_1(B)\ra \wtih_1(\bbr ^3)$
\hhsk 
which renders as 
\hhsk
$0\ra\bbz^{2g}\ra \wtih(\Sigma)\oplus G\ra 0$
\hhsk , i.e., \hhsk
$\bbz^{2g}\cong \wtih(\Sigma)\oplus G$ 
\hhsk . But for the Klein bottle and projective plane (or any closed, non-orientable
surface for that matter), $\wtih_1(\Sigma)$ has torsion, so it cannot be the direct
summand of a torsion-free group! So no such embedding exists. This result holds
more generally for any 2-complex $K$ whose (it turns out it would have to be first)
homology has torsion; any embedding into $\bbr^3$ would have a neighborhood 
deformation retracting to $K$, with boundary a (for the exact same reasons as above)
closed orientable surface.

\msk

{\bf Invariance of Domain:} If ${\Cal U}\subseteq {\Bbb R}^n$ and $f:{\Cal U}\ra {\Bbb R}^n$
is continuous and injective, then $f({\Cal U})\subseteq {\Bbb R}^n$ is open.

\msk

We will approach this through the {\bf Brouwer-Jordan Separation Theorem:} an embedded $(n-1)$-sphere
in $\bbr^n$ separates $\bbr^n$ into two path components. And for this we need to do a slightly
unusual homology calculation:

\ssk

For $k<n$ and $h:I^k\ra S^n$ an embedding of a $k$-cube in to the $n$-sphere,
$\wtih_i(S^n\setminus h(I^k))=0$ for all $i$.

\ssk 

Here $I=[-1,1]$ . The proof proceeds by induction on $k$. For $k=0$, $S^n\setminus h(I^k)\cong \bbr^n$,
and the result follows. Now suppose the result os true for all embeddings of $C=I^{k-1}$,
but is false for some embedding $h:I^k\ra S^n$ and some $i$. Then if we divide the cube along its last coordinate, say,
as $I^{k-1}\times [-1,0] = C\times [-1,0]$ and $C\times [0,1]$, we can set
$A=S^n\setminus h(C\times [-1,0])$, $B = S^n\setminus h(C\times [0,1])$, 
$A\cup B = S^n\setminus h(C\times \{0\})$, and $A\cap B = S^n\setminus h(I^k)$ .
These sets are all open, since the image under $h$ of the various sets is compact, hence closed.
By hypothesis, $A\cup B = S^n\setminus h(C\times \{0\})$ has trivial reduced homology, while
$A\cap B = S^n\setminus h(I^k)$ has non-trivial reduced homology in some dimension $i$. Then 
the Mayer-Vietoris sequence

\ssk

\ctln{$\cdots \ra \wtih_{i+1}(A\cup B) \ra \wtih_i(A\cap B) \ra \wtih_i(A)\oplus \wtih_i(B) \ra \wtih_i(A\cup B) \ra \cdots$}

\ssk  reads $0\ra \wtih_i(A\cap B) \ra \wtih_i(A)\oplus \wtih_i(B) \ra 0$ so 
$\wtih_i(A\cap B) \cong \wtih_i(A)\oplus \wtih_i(B)$ , so at least one of the groups
on the right must be non-trivial, as well. WOLOG $\wtih_i(B)=\wtih(S^n\setminus h(C\times [0,1]))\neq 0$.
Even more, choosing (once and for all) a non-zero element $[z]\in \wtih_I(A\cap B)$, snce its image in the
direct sum is non-zero, it's coordinate in (say) $\wtih_i(B)$ is non-zero.

\bsk

Continuing with: \hhsk
For $k<n$ and $h:I^k\ra S^n$ an embedding of a $k$-cube in to the $n$-sphere,
$\wtih_i(S^n\setminus h(I^k))=0$ for all $i$.

\msk

We've shown how we can throw away half of the cube without losing a (chosen)
non-zero homology element.
Now we continue inductively, cutting $C\times [0,1]$ in two along the last coordinate as
$C\times [0,1/2],C\times [1/2,1]$ and repeat the same argument. We fnd that
$\wtih_i(S^n\setminus h(C\times [a,b]))\neq 0$, and $[z]$ maps to a non-zero 
element under the inclusion-induced homomorphism.. Continuing inductively, we find a
sequence of nested intervals $I_n=[a_n,b_n]\supseteq [a_{n+1},b_{n+1}]$ 
whose lengths tend to zero (so $a_n,b_n\ra x_0\in I$ as $n\ra\infty$), and injective inclusion-induced maps

\ctln{$0\neq \wtih_i(S^n\setminus h(I^n)\ra \cdots \ra \wtih_i(S^n\setminus h(C\times I_n)
\ra \wtih_i(S^n\setminus h(C\times I_{n+1})$}

all of which send a certain non-zero element $[z]\in\wtih_i(S^n\setminus h(I^n)$ to 
a non-zero element, and all of which have an inclusion-induced map to $\wtih_i(S^n\setminus h(C\times \{x_0\}) = 0$.
So there is a non-trivial element $[z]\in \wtih_i(S^n\setminus h(I^n)$ which \u{remains}
non-zero in all $\wtih_i(S^n\setminus h(C\times I_n))$, but is zero in $\wtih_i(S^n\setminus h(C\times \{x_0\})$.
Consequently, $z\del w$ for some chain $w=\sum a_j\sigma_j^{i+1}\in C_{i+1}(S^n\setminus h(C\times \{x_0\}))$.
Each singular simplex, however, is a map $\sigma_j^{i+1}:\Delta^{i+1}\ra S^n\setminus h(C\times \{x_0\})$,
and so has compact image. But the sets $S^n\setminus h(C\times I_n)$ form a nested open cover of
$S^n\setminus h(C\times \{x_0\})$, and so of $\sigma_j^{i+1}(\Delta^{i+1})$, and so there is an
$n_j$ with $\sigma_j^{i+1}(\Delta^{i+1})\subseteq S^n\setminus h(C\times I_{n_j})$ .
Then setting $N=$max$\{n_j\}$, we have $\sigma_j^{i+1}:\Delta^{i+1}\ra S^n\setminus h(C\times I_N)$
for every $j$,
so $w\in C_{i+1}(S^n\setminus h(C\times I_N)$, so $0=[z]\in \wtih_i(S^n\setminus h(C\times I_N)$,
a contradiction. So $\wtih_i(S^n\setminus h(I^k))=0$, and our inductive step is proved.

\msk

One immediate consequence of this is that if $h:S^k\ra S^n$ is an embedding of the $k$-sphere into the $n$-sphere,
then thinking of $S^k$ as the union of its upper and lower hemispheres, $D^k_+,D^k_-$, each of which is homeomorphic
to $I^k$, we have  $D^k_+\cap D^k_-=S^{k-1}$, the equatorial $(k-1)$-sphere, and so by Mayer-Vietoris we have

\ssk

$\cdots \ra 
\wtih_{i+1}(S^n\setminus h(D^k_-))\oplus \wtih_{i+1}(S^n\setminus h(D^k_+)\ra
\wtih_{i+1}(S^n\setminus h(S^{k-1}))\ra \wtih_{i}(S^n\setminus h(S^k))\ra$

\hfill $\wtih_{i}(S^n\setminus h(D^k_-))\oplus \wtih_{i}(S^n\setminus h(D^k_+)\ra \cdots $

i.e., $\wtih_{i}(S^n\setminus h(S^k)) \cong \wtih_{i+1}(S^n\setminus h(S^{k-1})) 
\cong \cdots \cong \wtih_{i+k}(S^n\setminus h(S^0))\cong \wtih_{i+k}(S^{n-1})$ ,
since $S^0$ = 2 points, and  so $S^n\setminus h(S^0)\cong S^{n-1}\times \bbr \sim S^{n-1}$.
So $\wtih_{i}(S^n\setminus h(S^k)) = 0$ unless $i+k=n-1$ (i.e., $i=n-k-1$), when it is $\bbz$.

\ssk

In particular, $\wtih_0(S^n\setminus h(S^{n-1}))=\bbz$, so 
we have the \crr{Jordan-Brouwer Separation Theorem: every embedded $S^{n-1}$ in $S^n$
has two complementary path-components $A,B$} . With a little work, one can show that
$\overline{A}\cap \overline{B} = h(S^{n-1}$ , so the $(n-1)$-sphere is the frontier of each
complementary component. [Removing a point from $S^n$ to get $\bbr^n$ does not change the
conclusion (for $n>1$); a point does not disconnect an open subset of $S^n$.]

\msk

When $n=2$, the Jordan Curve Theorem (as it is then called) has the additional
consequence that the closure of each complementary region is a compact 2-disk,
each having the embedded circle $h(S^1)$ as its boundary. This stronger result
does not extend to higher dimensions, without putting extra restrictions 
on the embedding. This was shown by Alexander (shortly after publishing an
incorrect proof without restrictions) for $n=3$; these examples are known as
the Alexander horned spheres.

\msk

To prove Invariance of Domain, let ${\Cal U}\subseteq \bbr^n\subseteq S^n$ be an open 
set, and $f:{\Cal U}\ra \bbr^n \hookrightarrow S^n$ be injective and continuous. It suffices
to show, for every $x\in {\Cal U}$, that there is an open neighborhood ${\Cal V}$ with
$f(x)\subseteq {\Cal V} \subseteq f({\Cal U})$ . Since ${\Cal U}$ is open,
there is an open ball $B^n$ centered at $x$ whose closure $D^n$ is contained in ${\Cal U}$. 
$f$ is then an embedding of $\del D^n = S^{n-1}$ into $S^n$, and of $D^n\cong I^n$ into $S^n$.
By our calculations above, $S^n\setminus f(S^{n-1})$ has two path components $A,B$; being an open set 
and contained in a locally path-connected space, these are also the connected components
of the complement. But our calculations above also show that $S^n\setminus f(D^n)$ is
path-connected, hence connected, and $f(B^n)$, being the image of a connected set, is connected.
Since $f(B^n)\cup (S^n\setminus f(D^n)) = S^n\setminus f(S^{n-1} = A\cup B)$, it follows that
$f(B^n)=A$ and $S^n\setminus f(D^n) = B$ (or vice versa). In particular,
$f(B^n)$ is open, forming an open subset of $f({\Cal U})$ containing $f(x)$, as desired.

\bsk

Invariance of Domain in turn implies the ``other'' invariance of domain; if 
$f: {\Bbb R}^n\ra {\Bbb R}^m$ is continuous and injective, then
$n\leq m$, since if not, then composition of $f$ with the inclusion 
$i:{\Bbb R}^m\ra {\Bbb R}^n$, $i(x_1,\ldots ,x_m) = 
(x_1,\ldots ,x_m,0,\ldots ,0)$ is injective and continuous with non-open 
image (it lies in a hyperplane in ${\Bbb R}^n$),
a contradiction.

\msk

This also gives the more elementary: if ${\Bbb R}^n\cong {\Bbb R}^m$, via $h$, then $n=m$ .
Another proof: by composing with a translation, that $h(0)=0$, and then we have 
$({\Bbb R}^n,{\Bbb R}^n\setminus 0)\cong {\Bbb R}^m,({\Bbb R}^m\setminus 0)$, which gives


\ssk

\ctln{$\widetilde{H}_i(S^{n-1})\cong H_{i+1}({\Bbb D}^n,\del {\Bbb D}^n) \cong H_{i+1}({\Bbb D}^n,{\Bbb D}^n\setminus 0)
\cong H_{i+1}({\Bbb R}^n,{\Bbb R}^n\setminus 0) \cong H_{i+1}({\Bbb R}^m,{\Bbb R}^m\setminus 0)$}

\ctln{$\cong H_{i+1}({\Bbb D}^m,{\Bbb D}^m\setminus 0) \cong H_{i+1}({\Bbb D}^m,\del {\Bbb D}^m)
\cong \widetilde{H}_i(S^{m-1})$}

\ssk

Setting $i=n-1$ gives the result, since $\widetilde{H}_{n-1}(S^{m-1})\cong {\Bbb Z}$ implies $n-1=m-1$ .

\msk

{\bf Homology and homotopy groups:} There are connections between homology groups and
the fundamental (and higher) homotopy groups, provided by what is known as the
{\it Hurewicz map} $H:\pi_n(X,x_0)\ra H_n(X)$ . or $n=1$ (higher $n$ are similar)
the idea is that elements of $\pu(X)$ are loops, which can be thought of as maps
$\gamma:S^1\ra X$ (or more precisely, mapping into the path component containing 
$x_0$), inducing a map $\gamma_*:\bbz = H_1(S^1)\ra H_1(X)$ . 
We define $H([\gamma])=\gamma_*(1)$ . Because homotopic maps give the same induced
map on homology, this really is well-defined map on homotopy
classes, i.e. from $\pu(X)$ to $H_1(X)$. [A different view: 
a loop $\gamma:(I,\del I)\ra (X,x_0)$ defines a singular 1-chain which, being a loop,
has zero boundary, so is a 1-cycle. Since based homotopic maps give homologous
chains (essentially by the same homotopy invariance property above), we get a 
well-defined map $\pu(X,x_0)\ra H_1(X)$.

Since as 1-chains, the concatenation $\gamma*\delta$ of two loops is homologous
to the sum $\gamma+\delta$ - the map $K:I\times I\ra X$ given by $K(s,t)=(\gamma*\delta)(s)$,
after crushing the left and right vertical boundaries to points, can be thought of as
a singular 2-simplex with boundary $\gamma + \delta - (\gamma*\delta)$ - the map 
$H$ is a homomorphism.

\msk

When $X$ is path-connected, this map $H:\pu(X)\ra H_1(X)$ is onto . [When it isn't it maps onto the 
summand of $H_1(X)$ corresponding to the path component containing our chosen basepoint.]
To see this, note that any cycle $z\in Z_1(X)$ can be represented as a sum of singular
1-simplices $\sum \sigma_i^1$ , i.e. we can (by reversing the orientations on simplices
to make coefficient positive, and then writing a multiple of a simplex as a sum of
simplices) assume all coefficients in our sum are 1. Then 
$0 = \del z = \sum (\sigma_i^1(0,1) - \sigma_i^1(1,0))$ means that, starting with any positive term, 
we can match it with a 
negative term to cancel that term, which is paired with a postive term, having a matching negative term, etc.,
until the initial positive term is cancelled. This sub-chain represents a collection of paths which 
concatenate to a loop, so $z=$ (this loop) + (the remainiung terms) . Induction implies that
$z$ can be written as a sum of (sums of paths forming loops), which is (as above) homologous  to the
sum of loops. Choosing paths from the start of these loops to our chosen basepoint (which is the only
place where we use path connectedness, we can concatenate the based loops $\overline{\gamma}*\sigma*\gamma$
to a single based loop $\eta$, which under $H$ is sent to a chain homologous to $z$.
So $H[\eta] = [z]$ .

\msk 

Since $H_1(X)$ is abelian (and $\pu(X)$ need not be), the kernel of $H$ contains the
commutator subgroup $[\pu(X),\pu(X)]$ . 
We now show that, if $X$ is path connected,
$H$ induces an isomorphism $H_1(X) \cong \pu(X)/[\pu(X),\pu(X)]$ . 
To show this, it remains to show that  ker$(H)\subseteq [\pu(X),\pu(X)]$ . 
Or put differently, the ineduced map from $\pu(X)_{ab} = \pu(X)/[\pu(X),\pu(X)]$ 
(i.e., $\pu(X)$, written using additive notation) to $H_1(X)$  is injective.
So suppose $[\gamma]\in \pu(X)$ and, thought of as a singular 1-simplex, 
$\gamma = \del w$ for some 2-simplex $w=\sum a_i\sigma_i^2$ . 
As before, we may assume that all $a_i=1$, by reversing orientation and writing multiples as 
sums. By adding ``tails'' from each image of a vertex of each $\sigma_i^2$ to our chosen 
basepoint $x_0$, we may assume that the image of every face of $\Delta^2$, under
the $\sigma_i$ , is a loop at $x_0$ (by essentially replacing each $\sigma_i$ with a $\tau_i$
which first collapses little triangle at each vertex to arcs, maps the resulting central triangle
via $\sigma_i$, and the arcs via the paths). 

\ssk

Once we have made this slight alteration, the equation
$\displaystyle \gamma = \del w = \sum_{i=1}^n \sum_{j=0}^2 \del_j \sigma_i = 0$
makes sense (and is true) in both ($C_1(X)$ hence $Z_1(X)$ hence) $H_1(X)$ and
$\pu(X)_{ab}$, the first essentially by definition and the second because 
all of the $\del_j \sigma_i$ are loops at $x_0$ and, in 
$\pu(X)$, $(\del_0\sigma_i)\overline{\del_1\sigma_i}(\del_2\sigma_i)$ is null-homotopic,
so is trivial in $\pu(X)$. Written additively, this means that in
$\pu(X)_{ab}$ , $\del_0\sigma_i - \del_1\sigma_i + \del_2\sigma_i = 0$. 
So $\gamma = 0$ in $\pu(X)_{ab}$ , as desired.


\bsk

he Hurewicz map $H:\pu(X)\ra H_1(X)$ induces, when $X$ is path-connected,
an isomorphism from $\pu(X)/[\pu(X),\pu(X)]$ to $H_1(X)$ . 
This result can be used in two ways; knowing a (presentation for) $\pu(X)$
allows us to compute $H_1(X)$, by writing the relators additively, giving
$H_1(X)$ as the free abelian group on the generators, modulo the kernel 
of the ``presentation matrix'' given by the resulting linear equations. Conversely,
knowing $H_1(X)$ provides information about $\pu(X)$. For example,
a calculation on the way to invariance of domain implied that for every
knot $K$ in $S^3$ (i.e., the image of an embedding $h:S^1\hookrightarrow S^3$),
$H_1(S^3\setminus K) \cong bbz$ . This implies that the abelianization of 
$G_K = \pu(S^3\setminus K)$ (i.e., the largest abelian quotient of $G_K$  is $\bbz$.
But this in turn implies that for every integer $n\geq 2$, there is a \u{unique}
surjective homomorphism $G_K\ra \bbz_n$, since such a homomorphism must factor 
through the abelianization, and there is exactly one surjective homomorphism
$\bbz \ra \bbz_n$ ! Consequently, there is a unique (normal) subroup (the kernel
of this homomorphism) $K_n\sset G_K$ with quotient $\bbz_n$ . Using the Galois
correspondence, there is a (unique) covering space $X_n$ of $X=S^3\setminus K$
corresponding to $K_n$, called the $n$-fold cyclic covering of $K$ . This space is determined
by $K$ and $n$, and so its homology groups are determined by the same data.
And even though homology cannot distinguish between two knot complements,
$K$, $K^\prime$, it might be the case that homology \u{can} distinguish between 
their cyclic coverings. Consequently, 
if $H_1(X_n)\not\cong H_1(X_n^\prime)$, then $K$ and $K^\prime$ have non-homeomorphic
complement, and so represent ``different'' embeddings, hence different knots.
In practice, one can compute presentations for $\pu(X_n)$ (in several different ways),
and so one can compute $H_1(X_n)$, providing an effective way to use homology
to distinguish knots! This approach was ultimately formalized (by Alexander) into a polynomial
invariant of knots, known as the Alexander polynomial.

\msk

Computing the homology of the cyclic coverings can be done in several ways. The
Reidemeister-Schreier method will allow one to compute a presentation for the
kernel of a homomorphism $\varphi:G\ra H$, given a presentation of $G$ and a
{\it transversal} of the map, which is a representative of each coset of $G$ modulo
the kernel. Abelianizing this will give homology computation. Another approach
uses {\it Seifert surfaces}, orientable surfaces with $\del \Sigma = K$, to cut
$S^3\setminus K$ open along. Writing $S^3\setminus K = (S^3\setminus N(\Sigma))\cup N(\Sigma)$
allows us to use Mayer-Vietoris to compute homology. But the cyclic covering spaces
can be built by ``unwinding'' this view of $S^3\setminus K$; instead of gluing the two
ends of $N(K)$ to the same $S^3\setminus N(\Sigma)$, we can take $n$ copies
of $S^3\setminus N(\Sigma)$ and glue them together in a circle. Mayer-Vietoris
again tells us how to compute the homology of the resulting space. Details may be found on the accompanying
pages taken from Rolfsen's ``Knots and Links''.


\vfill
\end





\magnification=2000
\overfullrule=0pt
\parindent=0pt

\nopagenumbers

\input amstex

%\voffset=-.6in
%\hoffset=-.5in
%\hsize = 7.5 true in
%\vsize=10.4 true in

\voffset=1.8 true in
\hoffset=-.6 true in
\hsize = 10.2 true in
\vsize=8 true in

\input colordvi


