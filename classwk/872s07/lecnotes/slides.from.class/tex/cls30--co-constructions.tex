



\magnification=2000
\overfullrule=0pt
\parindent=0pt

\nopagenumbers

\input amstex

%\voffset=-.6in
%\hoffset=-.5in
%\hsize = 7.5 true in
%\vsize=10.4 true in

\voffset=1.8 true in
\hoffset=-.6 true in
\hsize = 10.2 true in
\vsize=8 true in

\input colordvi



\def\cltr{\Red}		  % Red  VERY-Approx PANTONE RED
\def\cltb{\Blue}		  % Blue  Approximate PANTONE BLUE-072
\def\cltg{\PineGreen}	  % ForestGreen  Approximate PANTONE 349
\def\cltp{\DarkOrchid}	  % DarkOrchid  No PANTONE match
\def\clto{\Orange}	  % Orange  Approximate PANTONE ORANGE-021
\def\cltpk{\CarnationPink}	  % CarnationPink  Approximate PANTONE 218
\def\clts{\Salmon}	  % Salmon  Approximate PANTONE 183
\def\cltbb{\TealBlue}	  % TealBlue  Approximate PANTONE 3145
\def\cltrp{\RoyalPurple}	  % RoyalPurple  Approximate PANTONE 267
\def\cltp{\Purple}	  % Purple  Approximate PANTONE PURPLE

\def\cgy{\GreenYellow}     % GreenYellow  Approximate PANTONE 388
\def\cyy{\Yellow}	  % Yellow  Approximate PANTONE YELLOW
\def\cgo{\Goldenrod}	  % Goldenrod  Approximate PANTONE 109
\def\cda{\Dandelion}	  % Dandelion  Approximate PANTONE 123
\def\capr{\Apricot}	  % Apricot  Approximate PANTONE 1565
\def\cpe{\Peach}		  % Peach  Approximate PANTONE 164
\def\cme{\Melon}		  % Melon  Approximate PANTONE 177
\def\cyo{\YellowOrange}	  % YellowOrange  Approximate PANTONE 130
\def\coo{\Orange}	  % Orange  Approximate PANTONE ORANGE-021
\def\cbo{\BurntOrange}	  % BurntOrange  Approximate PANTONE 388
\def\cbs{\Bittersweet}	  % Bittersweet  Approximate PANTONE 167
%\def\creo{\RedOrange}	  % RedOrange  Approximate PANTONE 179
\def\cma{\Mahogany}	  % Mahogany  Approximate PANTONE 484
\def\cmr{\Maroon}	  % Maroon  Approximate PANTONE 201
\def\cbr{\BrickRed}	  % BrickRed  Approximate PANTONE 1805
\def\crr{\Red}		  % Red  VERY-Approx PANTONE RED
\def\cor{\OrangeRed}	  % OrangeRed  No PANTONE match
\def\paru{\RubineRed}	  % RubineRed  Approximate PANTONE RUBINE-RED
\def\cwi{\WildStrawberry}  % WildStrawberry  Approximate PANTONE 206
\def\csa{\Salmon}	  % Salmon  Approximate PANTONE 183
\def\ccp{\CarnationPink}	  % CarnationPink  Approximate PANTONE 218
\def\cmag{\Magenta}	  % Magenta  Approximate PANTONE PROCESS-MAGENTA
\def\cvr{\VioletRed}	  % VioletRed  Approximate PANTONE 219
\def\parh{\Rhodamine}	  % Rhodamine  Approximate PANTONE RHODAMINE-RED
\def\cmu{\Mulberry}	  % Mulberry  Approximate PANTONE 241
\def\parv{\RedViolet}	  % RedViolet  Approximate PANTONE 234
\def\cfu{\Fuchsia}	  % Fuchsia  Approximate PANTONE 248
\def\cla{\Lavender}	  % Lavender  Approximate PANTONE 223
\def\cth{\Thistle}	  % Thistle  Approximate PANTONE 245
\def\corc{\Orchid}	  % Orchid  Approximate PANTONE 252
\def\cdo{\DarkOrchid}	  % DarkOrchid  No PANTONE match
\def\cpu{\Purple}	  % Purple  Approximate PANTONE PURPLE
\def\cpl{\Plum}		  % Plum  VERY-Approx PANTONE 518
\def\cvi{\Violet}	  % Violet  Approximate PANTONE VIOLET
\def\clrp{\RoyalPurple}	  % RoyalPurple  Approximate PANTONE 267
\def\cbv{\BlueViolet}	  % BlueViolet  Approximate PANTONE 2755
\def\cpe{\Periwinkle}	  % Periwinkle  Approximate PANTONE 2715
\def\ccb{\CadetBlue}	  % CadetBlue  Approximate PANTONE (534+535)/2
\def\cco{\CornflowerBlue}  % CornflowerBlue  Approximate PANTONE 292
\def\cmb{\MidnightBlue}	  % MidnightBlue  Approximate PANTONE 302
\def\cnb{\NavyBlue}	  % NavyBlue  Approximate PANTONE 293
\def\crb{\RoyalBlue}	  % RoyalBlue  No PANTONE match
%\def\cbb{\Blue}		  % Blue  Approximate PANTONE BLUE-072
\def\cce{\Cerulean}	  % Cerulean  Approximate PANTONE 3005
\def\ccy{\Cyan}		  % Cyan  Approximate PANTONE PROCESS-CYAN
\def\cpb{\ProcessBlue}	  % ProcessBlue  Approximate PANTONE PROCESS-BLUE
\def\csb{\SkyBlue}	  % SkyBlue  Approximate PANTONE 2985
\def\ctu{\Turquoise}	  % Turquoise  Approximate PANTONE (312+313)/2
\def\ctb{\TealBlue}	  % TealBlue  Approximate PANTONE 3145
\def\caq{\Aquamarine}	  % Aquamarine  Approximate PANTONE 3135
\def\cbg{\BlueGreen}	  % BlueGreen  Approximate PANTONE 320
\def\cem{\Emerald}	  % Emerald  No PANTONE match
%\def\cjg{\JungleGreen}	  % JungleGreen  Approximate PANTONE 328
\def\csg{\SeaGreen}	  % SeaGreen  Approximate PANTONE 3268
\def\cgg{\Green}	  % Green  VERY-Approx PANTONE GREEN
\def\cfg{\ForestGreen}	  % ForestGreen  Approximate PANTONE 349
\def\cpg{\PineGreen}	  % PineGreen  Approximate PANTONE 323
\def\clg{\LimeGreen}	  % LimeGreen  No PANTONE match
\def\cyg{\YellowGreen}	  % YellowGreen  Approximate PANTONE 375
\def\cspg{\SpringGreen}	  % SpringGreen  Approximate PANTONE 381
\def\cog{\OliveGreen}	  % OliveGreen  Approximate PANTONE 582
\def\pars{\RawSienna}	  % RawSienna  Approximate PANTONE 154
\def\cse{\Sepia}		  % Sepia  Approximate PANTONE 161
\def\cbr{\Brown}		  % Brown  Approximate PANTONE 1615
\def\cta{\Tan}		  % Tan  No PANTONE match
\def\cgr{\Gray}		  % Gray  Approximate PANTONE COOL-GRAY-8
\def\cbl{\Black}		  % Black  Approximate PANTONE PROCESS-BLACK
\def\cwh{\White}		  % White  No PANTONE match


\loadmsbm

\input epsf

\def\ctln{\centerline}
\def\u{\underbar}
\def\ssk{\smallskip}
\def\msk{\medskip}
\def\bsk{\bigskip}
\def\hsk{\hskip.1in}
\def\hhsk{\hskip.2in}
\def\dsl{\displaystyle}
\def\hskp{\hskip1.5in}

\def\lra{$\Leftrightarrow$ }
\def\ra{\rightarrow}
\def\mpto{\logmapsto}
\def\pu{\pi_1}
\def\mpu{$\pi_1$}
\def\sig{\Sigma}
\def\msig{$\Sigma$}
\def\ep{\epsilon}
\def\sset{\subseteq}
\def\del{\partial}
\def\inv{^{-1}}
\def\wtl{\widetilde}
%\def\lra{\Leftrightarrow}
\def\del{\partial}
\def\delp{\partial^\prime}
\def\delpp{\partial^{\prime\prime}}
\def\sgn{{\roman{sgn}}}
\def\wtih{\widetilde{H}}
\def\bbz{{\Bbb Z}}
\def\bbr{{\Bbb R}}
\def\bbq{{\Bbb Q}}
\def\bbc{{\Bbb C}}
\def\hdsk{\hskip.7in}
\def\hdskb{\hskip.9in}
\def\hdskc{\hskip1.1in}
\def\hdskd{\hskip1.3in}
\def\Hom{\text{Hom}}
\def\Ext{\text{Ext}}
\def\larr{\leftarrow}



By universal coefficients, $H^n(X;G) \cong \Hom(H_n(X),G)\oplus \Ext(H_{n-1}(X),G)$, 
so cohomology is not really anything ``new''; the groups themselves provide no new
information to distinguish spaces. So why should we care? And what does it measure, 
anyway? One way to answer that question is to study what a cochain \u{is}, and what 
it means for a cocycle to \u{not} be a coboundary. 

\ssk

Let us think in terms of simplicial cochains for a $\Delta$-cplx $X$.
An $n$-cochain assigns elements of $G$ to the $n$-simplices of $X$. Think for 
example of a $0$-cochain $\varphi$, which assigns values to the vertices. The coboundary
$\delta\varphi$ assigns the difference $\varphi(v_1)-\varphi(v_0)$ to each oriented $1$-simplex
$[v_0,v_1]$. A $1$-cochain, assigning values to each $1$-simplex, is a coboundary
if those values represent the {\it differences} of a fcn defined on the vertices. 
A $1$-cocycle, on the other hand, is a $1$-cochain $\psi$ for which 

\ctln{$\delta\psi([v_0,v_1,v_2]) = \psi([v_0,v_1])-\psi([v_0,v_2])+\psi([v_0,v_1])=0$ (*)}

for every $2$-simplex (since the map is $0$ \lra\ it is $0$ on each basis element). 
The point is that \u{if} a $1$-cochain represents the differences of the values of
some globally defined function on the vertices, across each $1$-simplex, then (*)
must certainly be true; $(a-b)+(b-c)+(c-a)=0$. This is what
$1$-coboundary $\Rightarrow$ $1$-cocycle says. The fact that the opposite need not be true is a
reflection of the topology of $X$; the condition (*) essentially says
that the values on edges represent differences of values on the vertices ``locally''. 
This can be related to the idea of vector fields versus
conservative vector fields, in analysis; a vector field will integrate around the boundary of
a 2-simplex to give $0$, but a vector field is 
conservative (equal to the gradient of a function) \lra\ it integrates to $0$ around
every closed loop. In higher degrees we can view things analogously, and these can be 
related, analytically, to integrals over correspondingly higher dimensional regions.

\msk

As mentioned previously, essentially all of the machinery we built to study homology 
can be adapted to study cohomology. A map $f:X\ra Y$ induces a (chain) map

\ctln{$f_\#:C_n(X)\ra C_n(Y)$;} 

dualizing, we get a (chain) map 

\ctln{$f^\#=f_\#^*:C_n^*(Y;G)\ra C_n^*(X;G)$,}

which gives a homomorphism 

\ctln{$f^*:H^n(Y;G)\ra H^n(X;G)$.}

 This satisfies $(f\circ g)^*=g^*\circ f^*$
(since this is true for the chain complexes and carries over to the duals) and $I^*=I$,
so we immediately recover the analogous result that a homeo induces isos on cohomology.
More, homotopic maps induce equal maps on cohomology, since if $f,g:X\ra Y$ are homotopic, then
$f_\#,g_\#$ are chain homotopic, via a chain homotopy 

\ctln{$H$; $H\del+\del H = f_\#-g_\#$.}]

Dualizing, we obtain a ``chain cohomotopy'' (?) $H^*$, \u{decreasing} degree by one, with

\ctln{$\delta H^*+H^*\delta = f^\#-g^\#$} 

and therefore, by the same proof, $f^\#$ and $g^\#$ induce
the same map $f^*=g^*$ on cohomology. Consequently, we recover the result that a homotopy
equivalence between spaces induces isomorphisms between their cohomology groups.

\vfill
\eject

We can define relative cohomology groups $H^n(X,A;G)$ by defining

\ctln{$C^n(X,A;G)=\Hom(C_n(X,A),G) = \Hom(C_n(X)/C_n(A),G)$.}

Maps of pairs induce, as in the homology case, homomorphisms between relative
cohomology groups.
Dualizing  the SES

\ssk

\ctln{$0\ra C_n(A)\buildrel{\iota}\over\ra C_n(X)\buildrel{p}\over\ra C_n(X)/C_n(A)\ra 0$}

\ssk

we get a sequence

\ctln{(*) $0\ra\Hom(C_n(X)/C_n(A),G)\buildrel{p^*}\over\ra
\Hom(C_n(X),G)\buildrel{\iota^*}\over\ra\Hom(C_n(A),G)\ra 0$}

\ssk

which turns out to be exact; this is basically because $C_n(X)/C_n(A)$ has basis
chains in $X$ that do not map completely into $A$, so really

\ssk

\ctln{$C_n(X)=C_n(A)\oplus C_n(X)/C_n(A)$,}

\ssk

which consequently means that

\ssk

\ctln{$\Hom(C_n(X),G)\cong \Hom(C_n(A),G)\oplus \Hom(C^n(X)/C_n(A),G)$,}

\ssk

and under
the isomorphism the sequence (*) becomes the ``obvious'' one, which is exact.
The coboundary maps 

\ssk

\ctln{$\delta:\Hom(C_n(X)/C_n(A),G)\ra \Hom(C_{n+1}(X)/C_{n+1}(A),G)$}

\ssk

are given by, for $\varphi: C_n(X)/C_n(A)\ra G$, composing with $p$ to get
a map 

$\varphi_1=\varphi\circ p:C_n(X)\ra G$, taking its coboundary (in $X$), and noting that
the resulting map $\delta\varphi_1=\psi_1$ is $0$ on $C_{n+1}(A)$ (since
$\psi_1(a)=\varphi_1(\del a)=\varphi(p(\del a)) = \varphi(0) = 0$, since 
$\del a\in C_n(A)$, so $p(\del a) = 0$.

\vfill
\eject

%% \ctln{(*) $0\ra\Hom(C_n(X)/C_n(A),G)\buildrel{p^*}\over\ra
%% \Hom(C_n(X),G)\buildrel{\iota^*}\over\ra\Hom(C_n(A),G)\ra 0$}

\ssk

These exact sequences (*) consequently, as before, give rise to a LEHS (LECS?)

\ssk

\ctln{$\cdots \ra H^{n+1}(A;G)\ra H^n(X,A;G)\ra H^n(X;G)\ra H^n(A;G)\ra H^{n-1}(X,A;G)\ra \cdots$}

\ssk

Similarly there is a LEHS for a triple $B\subseteq A\subseteq X$. The universal 
coefficients theorem applies to relative homology, since the relevant chain groups are
free abelian, so 

\ssk

\ctln{$H^n(X,A;G)\cong \Hom(H_n(X,A),G)\oplus \Ext(H_{n-1}(X,A),G)$}

\ssk

Cohomology on small chains can be defined analogously;  $H^n_{\Cal U}(X;G)$ is the homology
of the cochain complex $\Hom(C_n^{\Cal U}(X),G)$. The inclusion-induced map

\ssk

\ctln{$\iota^\#:\Hom(C_n(X),G)\ra\Hom(C_n^{\Cal U}(X),G)$}

\ssk

induces isomorphisms of cohomology
groups, by dualizing the proof we \u{didn't} do for homology, namely that there is a 
chain map 
$b:C_n(X)\ra C_n^{\Cal U}(X)$ such that $\iota\circ b$ and $b\circ\iota$ are chain 
homotopic to the identity. As above, the duals of the chain homotopies form the
necessary chain (co)homotopies. We can then recover excision:

\ctln{\cltr{If $A,B\subseteq$ satisfy
the usual requirements for excsion, then}}

\ctln{\cltr{$\iota^*:H^n(X,A;G)\ra H^n(B,A\cap B;G)$ is an isomorphism.}}

the proof of building the iso

\ssk

\ctln{$H^n(C_n(X)/C_n(A);G) \ra H_n(\Hom(C_n^{\{A,B\}}(X)/C_n(A),G))$}

\ssk

via SESs 
(arguing as above that the duals of the relevant SESs are exact) and the Five Lemma
applied to the resulting LEHSs, together with 

\ssk

\ctln{$\Hom(C_n^{\{A,B\}}(X)/C_n(A),G)\cong \Hom(C_n(B)/C_n(A\cap B),G)$}

\ssk

because the domains are isomorphic, inducing the corresponding
iso in cohomology, goes through without change. 

\vfill
\eject

The same reinterpretation from homology also gives
the excision isomorphism 

\ssk

\ctln{$H^n(X,A;G)\ra H^n(X\setminus B,A\setminus B;G)$.}

\ssk

Dualizing the SES 

\ssk

\ctln{$0\ra C_n(A\cap B) \ra C_n(A)\oplus C_n(B)\ra C_n^{\{A,B\}}(X)\ra 0$}

\ssk

gives the (short exact, by the argument above) sequence

\ssk

\ctln{$0\ra \Hom(C_n^{\{A,B\}}(X),G) \ra \Hom(C_n(A),G)\oplus \Hom(C_n(B),G)\ra \Hom(C_n(A\cap B),G)\ra 0$}

\ssk

yielding the Mayer-Vietoris sequence for cohomology:

\ssk

\ctln{$\cdots \ra H^{n-1}(A\cap B)\ra H^n(X)\ra H^n(A)\oplus 
H^n(B)\ra H^n(A\cap B)\ra H^{n+1}(X)\ra \cdots$}

\ssk

(supressing the coefficient group $G$ to make this fit on a line...). 

\ssk

Reduced cohomology can be defined by taking the dual of the augmented chain complex defining homology.
As with homology, $\wtih^n(X;G)\cong H^n(X;G)$ for $n\geq 1$ and, from the universal coefficients 
theorem, $\wtih^0(X;G)\cong \Hom(\wtih_0(X),G)$.
We can think of $\Hom(\wtih_0(X),G)$ as a direct product of $G$s,
one for each path component of $X$; alternatively, this is the set of all functions from $X$ to $G$ that are
constant on path components. $\Hom(\wtih_0(X),G)$ is slightly smaller; since the augmentation map 
$C_0(X)\ra \bbz$ sends every point to $1$, the dual map 

\ssk

\ctln{$G\cong \Hom(\bbz,G)\ra \Hom(C_0(X),G)\ra \cdots$}

\ssk

sends $g\in G$ to the map which sends each basis element of $C_0(X)$ (i.e., points) to $g$. So $\wtih^0(X;G)$
is built by modding outm in addition, by this map; so $\wtih^0(X;G)$ can be identified with the set
of {\it non-constant} functions from the path-components of $X$ to $G$.

\msk

As for homology, a combination of excision,
the LES of a pair, and the Five Lemma implies that $H^n(X,A;G)\cong \wtih^n(X/A;G)$ when $A$ has a neighborhood 
that deformation retracts to it.

\msk

With these facts in hand, we can carry out calculations of cohomology groups in much the same spirit as we did for homology.
For example, Mayer-Vietoris and induction implies that $H^k(S^n;G)\cong G$ for $k=0,n$ and $0$ otherwise. In most
cases, though, if all that we are after are the groups themselves, the universal coefficients theorem provides a 
faster computational route. In fact, at least for $\bbz$-coefficients, if 
we define the torsion subgroup $T$ of an abelian group $G$ to be the set of 
elements of finite order in $G$, then if the homology groups of $X$ are
finitely generated, then $\Hom(H_n(X),\bbz)$ is isomorphic to the free abelian part of $H_n(X)$ 
(which is isomorphic to $H_n/T_n$), and $\Ext(H_{n-1}(X),\bbz)$ is the torsion part $T_{n-1}$ of $H_{n-1}(X)$, so 

\ssk

\ctln{$H^n(X;\bbz)\cong (H_n/T_n)\oplus T_{n-1}$}

\ssk

is the direct sum of the free part of $H_n(X)$ and the torsion part of $H_{n-1}(X)$, so the difference between homology
and cohomology is that cohomology carries its torsion one degree higher than homology does! With more general 
coefficients, the situation becomes more involved...



\vfill
\end

