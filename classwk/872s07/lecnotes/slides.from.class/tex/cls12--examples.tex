
\magnification=2000
\overfullrule=0pt
\parindent=0pt

\nopagenumbers

\input amstex

%\voffset=-.6in
%\hoffset=-.5in
%\hsize = 7.5 true in
%\vsize=10.4 true in

\voffset=1.8 true in
\hoffset=-.6 true in
\hsize = 10.2 true in
\vsize=8 true in

\input colordvi

\def\cltr{\Red}		  % Red  VERY-Approx PANTONE RED

\loadmsbm

\input epsf

\def\ctln{\centerline}
\def\u{\underbar}
\def\ssk{\smallskip}
\def\msk{\medskip}
\def\bsk{\bigskip}
\def\hsk{\hskip.1in}
\def\hhsk{\hskip.2in}
\def\dsl{\displaystyle}
\def\hskp{\hskip1.5in}

\def\lra{$\Leftrightarrow$ }
\def\ra{\rightarrow}
\def\mpto{\logmapsto}
\def\pu{\pi_1}
\def\mpu{$\pi_1$}
\def\sig{\Sigma}
\def\msig{$\Sigma$}
\def\ep{\epsilon}
\def\sset{\subseteq}
\def\del{\partial}
\def\inv{^{-1}}
\def\wtl{\widetilde}
%\def\lra{\Leftrightarrow}
\def\del{\partial}
\def\delp{\partial^\prime}
\def\delpp{\partial^{\prime\prime}}
\def\sgn{{\roman{sgn}}}
\def\wtih{\widetilde{H}}
\def\bbz{{\Bbb Z}}
\def\bbr{{\Bbb R}}
\def\rtar{$\Rightarrow$}

\def\cltr{\Red}		  % Red  VERY-Approx PANTONE RED
\def\cltb{\Blue}		  % Blue  Approximate PANTONE BLUE-072
\def\cltg{\PineGreen}	  % ForestGreen  Approximate PANTONE 349


{\bf Some examples:}

\ssk

{\bf Fundamental groups of graphs:} Every finite connected graph $\Gamma$ has a {\it maximal tree} $T$,
a connected subgraph with no simple circuits. Since any tree is the 
union of smaller trees joined at a vertex, we can, by induction, show that 
$\pu(T) = \{ 1\}$ . In fact, if $e$ is an outermost edge of $T$, then 
$T$ deformation retracts to $T\setminus e$, so, by induction, $T$ is 
contractible. Consequently ({\it Hatcher, Proposition 0.17}), $\Gamma$ and the quotient space $\Gamma/T$
are homotopy equivalent, and so have the same \mpu . But $\Gamma/T=\Gamma_n$
is a bouquet of $n$ circles for some $n$. If we let ${\Cal U}$ = a neighborhood of 
the vertex in $\Gamma_n$, which is contractible, then, by singling out one petal of the bouquet,
we have

\ssk

\ctln{$\Gamma_n = (\Gamma_{n-1}\cup{\Cal U})\cup (\Gamma_1\cup{\Cal U}) = X_1\cup X_2$}

\ssk

with $\Gamma_{k}\cup{\Cal U}\sim (\Gamma_{k}\cup{\Cal U})/{\Cal U}\cong \Gamma_{k}$. 
Since $X_1\cap X_2={\Cal U}\sim *$, we have 

\ctln{$\pu(\Gamma_{n}) \cong \pu(\Gamma_{n-1})*_{1}\pu(\Gamma_1) = \pu(\Gamma_{n-1})*{\Bbb Z}$}

So, by induction, $\pu(\Gamma) \cong \pu(\Gamma_{n})\cong {\Bbb Z}*\cdots *{\Bbb Z} = F(n)$, 
the free group on $n$ letters, where $n$ = the number of edges not in a maximal tree for $\Gamma$. 
The generators for the group consist of the
edges not in the tree, prepended and appended by paths to the basepoint.

\vfill
\eject

{\bf Gluing on a 2-disk:} $f:\del {\Bbb D}^2\ra X$ continuous, then we
construct the quotient space $Z=(X\coprod {\Bbb D}^2)/\{x\simeq f(x) : x\in\del{\Bbb D}^2\}$,
the result of gluing ${\Bbb D}^2$ to $X$ along $f$. 

\ssk

We can use Seifert - van Kampen to compute \mpu\ of the resulting space; if we
wish to be careful with basepoints $x_0$,
we include a rectangle $R$, the mapping cylinder of a path $\gamma$ running from 
$f(1,0)$ to $x_0$, glued to 
${\Bbb D}^2$ along the arc from $(1/2,0)$ to $(1,0)$ (see figure). 
This space $Z_+$ deformation retracts to $Z$; it
is simpler work with the basepoint $y_0$ lying above $x_0$.

\ssk

Write $D_1 = \{x\in {\Bbb D}^2 : ||x||<1\}\cup(R\setminus X)$ 
and $D_2 = \{x\in {\Bbb D}^2 : ||x||>1/3\}\cup R$ , 
then we can write $Z_+=D_+\cup(X\cup D_2) = X_1\cup X_2$.
But since $X_1\simeq *$ , $X_2\simeq X$ 
(it is essentially the mapping cylinder of the maps $f$ and $\gamma$ )
and $X_1\cap X_2 = \{x\in {\Bbb D}^2 : 1/3<||x||<1\}\cap(R\setminus X)\simeq S^1$, 
we find that 

\ctln{$\pu(Z,y_0)\cong \pu(X_2,y_0)*_{\Bbb Z}\{1\} = \pu(X_2)/<{\Bbb Z}>^N \cong \pu(X_2)/<[\overline{\delta}*\overline{\gamma}*f*\gamma*\delta]>^N$}

Use $\delta$ for a change of basepoint isomorphism, and then a homotopy
equivalence from $X_2$ to $X$ (fixing $x_0$), we have, 
if $\pu(X,x_0)=<\sig | R>$ , then $\pu(Z) = <\sig | R\cup\{[\overline{\gamma}*f*\gamma]\}>$ . 
So the effect of gluing on a 2-disk on the $\pi_1$ is to add a new relator, 
namely the word represented by the attaching map (adjusting for basepoint).

\msk


\vbox{\hsize=5in

\leavevmode

\epsfxsize=5in
\epsfbox{gluedisk.ai}}

\vfill
\eject

{\bf Wirtinger presentations for knot complements:}

\msk

A {\it knot} $K$ is (the image of) an embedding $h:S^1\hookrightarrow\bbr^3$. Wirtinger 
gave a prescription for taking a planar projection of $K$ and producing a presentation
of $\pi_1(\bbr^3\setminus K)$. The idea: think of $K$ as lying on the projection
plane, except near the crossings, where it arches over and dives under. 
Express $\bbr^3\setminus N(K)\simeq X$ as the union of
$C_-=X\cap\{(x,y,z\in \bbr^3 : z\leq 0\}$ and $C_+=X\cap\{(x,y,z\in \bbr^3 : z\geq 0\}$,
with intersection ($\bbr^2\times\{0\})\setminus N(K)$.

But each of $C_-,C_+$ is $\simeq$ a bouquet of $n$ circles, where 
$n=$ the number of crossings in the projection of $K$.

\ssk



\vfill
\end

