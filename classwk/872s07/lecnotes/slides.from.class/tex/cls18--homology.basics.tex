

\magnification=2000
\overfullrule=0pt
\parindent=0pt

\nopagenumbers

\input amstex

%\voffset=-.6in
%\hoffset=-.5in
%\hsize = 7.5 true in
%\vsize=10.4 true in

\voffset=1.8 true in
\hoffset=-.6 true in
\hsize = 10.2 true in
\vsize=8 true in

\input colordvi

\def\cltr{\Red}		  % Red  VERY-Approx PANTONE RED

\loadmsbm

\input epsf

\def\ctln{\centerline}
\def\u{\underbar}
\def\ssk{\smallskip}
\def\msk{\medskip}
\def\bsk{\bigskip}
\def\hsk{\hskip.1in}
\def\hhsk{\hskip.2in}
\def\dsl{\displaystyle}
\def\hskp{\hskip1.5in}

\def\lra{$\Leftrightarrow$ }
\def\ra{\rightarrow}
\def\mpto{\logmapsto}
\def\pu{\pi_1}
\def\mpu{$\pi_1$}
\def\sig{\Sigma}
\def\msig{$\Sigma$}
\def\ep{\epsilon}
\def\sset{\subseteq}
\def\del{\partial}
\def\inv{^{-1}}
\def\wtl{\widetilde}
%\def\lra{\Leftrightarrow}
\def\del{\partial}
\def\delp{\partial^\prime}
\def\delpp{\partial^{\prime\prime}}
\def\sgn{{\roman{sgn}}}
\def\wtih{\widetilde{H}}
\def\bbz{{\Bbb Z}}
\def\bbr{{\Bbb R}}
\def\rtar{$\Rightarrow$}

\def\cltr{\Red}		  % Red  VERY-Approx PANTONE RED
\def\cltb{\Blue}		  % Blue  Approximate PANTONE BLUE-072
\def\cltg{\PineGreen}	  % ForestGreen  Approximate PANTONE 349





{\bf Homology theory:} Fundamental groups are a remarkably powerful
tool for studying spaces; they capture a great deal of the global
structure of a space, and so they are very good a detecting
between homotopy-inequivalent spaces. In theory! \cltr{{\bf But}} in practice,
they suffer from the fact that deciding whether two groups are 
isomorphic or not is, in general, undecideable.... 

\msk

Homology theory
is designed to get around this deficiency; the theory, by design,
builds (a sequence of) {\it abelian} groups $H_i(X)$ from a topological
space. And deciding whether or not two \u{abelian} groups are isomorphic, at least
if you're given a presentation for them, is, in the end, a matter of
fairly routine linear algebra. Mostly because of the Fundamental Theorem
of Finitely-generated Abelian groups; each such has a unique representation
as ${\Bbb Z}^m\oplus{\Bbb Z}_{m_1}\oplus\cdots\oplus{\Bbb Z}_{m_n}$
with $m_{i+1}|m_i$ for every $i$ .

\msk

There are also ``higher'' homotopy groups beyond the fundamental group \mpu ,
(hence the name pi-{\it one}); elements are homotopy classes, rel boundary, 
of based maps 

$(I^n,\del I^n)\ra(X,x_0)$. Multiplication is again by
concatenation. But unlike \mpu , where we have a chance to compute it
via Seifert-van Kampen, nobody, for example knows what all of the 
homotopy groups $\pi_n(S^2)$ are (except that nearly all of them are
non-trivial!). Like \mpu, it describes, essentially, maps of $S^n$ into
$X$ which don't extend to maps of $D^{n+1}$, i.e., it turns the ``$n$-dimensional
holes'' of $X$ into a group.

\vfill
\eject

Homology theory does the same thing, it counts $n$-dimensional holes.
In the end, it is extremely computable; but we will need
a fair bit of machinery before it will become transparent to
calculate. The short version is that homology groups compute
``cycles mod boundaries'', that is, $n$-dimesional objects/subsets that
have no boundary (in the appropriate sense) modulo objects that are the
boundary of $(n+1)$-dimensional ones. 

\ssk

We will focus on two approached to homology: simplicial and singular.
The first is quick to define and compute, but hard to show is an invariant.
The second is quick to see is an invariant, but, at the start, hard
to compute! But for spaces where they are both defined, they are
isomorphic. So between the two we get an invariant that is quick to compute.

\msk

{\bf Simplicial homology:} This is a sequence of groups defined for spaces
called $\Delta$-complexes. They are a particular kind of CW-complex,
defined by gluing simplices together using ``nice enough'' maps. 

\ssk

More precisely, the {\it standard $n$-simplex} $\Delta^n$ is
the set of points 

\ctln{$\{(x_1,\ldots x_{n+1})\in{\Bbb R}^{n+1}$ : $\sum x_i=1 , x_i\geq 0$
for all $i\}$.}

 This is the set of {\it convex} linear combinations
of the points
$e_i=(0,\ldots ,0,1,0,\ldots ,0)$, the {\it vertices} of the standard
simplex. An $n$-simplex is the set $[v_0,\ldots v_n]$ of
convex linear combinations of points $v_0,\ldots ,v_n\in{\Bbb R}^{k}$
for which $v_1-v_0,\ldots ,v_n-v_0$ are linearly independent.
Any bijection $\{$vertices of $\Delta^n\}$ $\ra$
$\{v_0,\ldots ,v_N\}$ extends (linearly) to a homeo b/w
$\Delta^n$ and $[v_0,\ldots v_n]$. The {\it faces} of a simplex, each opposite
a vertex $v_i$, are obtained by setting the corresponding coefficient $x_i$ to $0$. 
Each forms an $(n-1)$-simplex, which we denote 
$[v_0,\ldots,v_{i-1},v_{i+1},\ldots ,v_n]$ =
$[v_0,\ldots,\widehat{v_{i}},\ldots ,v_n]$ . 

\vfill
\eject

A {\it $\Delta$-complex} $X$ is a cell
complex obtained by gluing simplices together, but we insist on an extra
condition:
the restriction of the attaching map to any face is equal to a (lower-dimensional)
simplex. As before, we use the weak topology on the space; a set is open iff
it's inverse image under the induced map of each simplex into the complex is open.
Each $n$-cell comes equipped with a characteristic map
$\sigma:\Delta^n\ra X$, which is one-to-one on its interior, whose restriction
to the boundary is the attaching map, and whose restriction to each face is the
characteristic map for that $(n-1)$-simplex. We will typically blur the 
distinction between the characteristic map $\sigma$ and its image, and denote the
image by $\sigma$ (or $\sigma^n$), when this will cause no confusion,
and call $\sigma$ an $n$-simplex {\it in} $X$. When we feel the need for the 
distinction, we will use $e^n$ for the image and $\sigma^n$ for the map.

\ssk

For example, taking our standard,
identifications of the sides of a rectangle as a cell structure for the 2-torus $T^2$,
and cutting the rectangle into two triangles (= 2-simplices) along a diagonal,
we obtain a $\Delta$-structure  for $T^2$ with 2 2-simplices, 3 1-simplices, and 1 0-simplex.
A genus $g$ surface can be built, by cutting the $2g$-gon into triangles, with
$g+1$ 2-simplices, $3g$ 1-simplices, and 1 0-simplex.

\bsk


As with CW-cplxes, we typically think of building a $\Delta$-complex $X$ inductively. 
The {\it 0-simplices} or {\it vertices} form the 0-skeleton 
$X^{(0)}$. $n$-simplices $\sigma^n = [v_0,\ldots v_n]$ attach 
to the $(n-1)$-skeleton
to form the $n$-skeleton $X^{(n)}$; the restriction
of the attaching map to each face of $\sigma^n$ is
an $(n-1)$-simplex in $X$. This attaching map is
really determined by a map $\{v_0,\ldots ,v_n\}\ra X^{(0)}$, since this 
determines the attaching maps for the 1-simplices in the boundary of the
$n$-simplex, which 
gives 1-simplices in $X$, which then give the attaching maps for
the 2-simplices in the boundary, etc. 

\vfill
\eject

The reverse is not true;
the vertices of two different $n$-simplices in $X$ can be the same.
For example, build 2-sphere as a pair of 2-simplices whose 
boundaries are glued by the identity. $\Delta$-complexes generalize
{\it simplicial complexes} where the simplices are required to attach
by homeomorphisms to the skeleton, and the intersection of two 
simplices are a (single) sub-simplex of each. This has the advantage
over $\Delta$-complexes that an $n$-simplex is determined uniquely
by the set of vertices in $X^{(0)}$ that it attaches to. This means that,
in principle, a simplicial complex (and everything associated with it, e.g., 
its homology groups!) can be treated purely combinatorially;
the complex is ``really'' a certain collection of subsets of the vertices
(since these determine the simplices), with the property that any subset $B$
of a subset $A$ that has been declared to be a simplex is also a simplex.
But they have the disadvantage that it typically takes far more
simplices to build a simplicial structure on a space $X$ that it does to build a 
$\Delta$-structure. This makes the computations we are about to do take 
far longer.

\msk

The final detail that we need before defining (simplicial) homology
groups is the notion of an {\it orientation} on a simplex of $X$.
Each simplex $\sigma^n$ is determined by a map 
$f:\{v_0,\ldots ,v_n\}\ra X^{(0)}$; an orientation on $\sigma^n$ is an
(equivalence class of) the ordered $(n+1)$-tuple $(f(v_0),\ldots f(v_n)) = (V_0,\ldots ,V_n)$.
Another ordering of the
same vertices represents the same orientation if there is an {\it even} permutation
taking the entries of the first $(n+1)$-tuple to the second. This should be thought 
of as a generalization of the right-hand rule for ${\Bbb R}^3$, interpreted as
orienting the vertices of a 3-simplex. Note that there are precisely two
orientations on a simplex.

\vfill
\end


