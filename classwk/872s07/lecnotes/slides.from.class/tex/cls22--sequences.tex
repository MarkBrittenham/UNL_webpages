\magnification=2000
\overfullrule=0pt
\parindent=0pt

\nopagenumbers

\input amstex

%\voffset=-.6in
%\hoffset=-.5in
%\hsize = 7.5 true in
%\vsize=10.4 true in

\voffset=1.8 true in
\hoffset=-.6 true in
\hsize = 10.2 true in
\vsize=8 true in

\input colordvi

\def\cltr{\Red}		  % Red  VERY-Approx PANTONE RED
\def\cltb{\Blue}		  % Blue  Approximate PANTONE BLUE-072
\def\cltg{\PineGreen}	  % ForestGreen  Approximate PANTONE 349
\def\cltp{\DarkOrchid}	  % DarkOrchid  No PANTONE match
\def\clto{\Orange}	  % Orange  Approximate PANTONE ORANGE-021
\def\cltpk{\CarnationPink}	  % CarnationPink  Approximate PANTONE 218
\def\clts{\Salmon}	  % Salmon  Approximate PANTONE 183
\def\cltbb{\TealBlue}	  % TealBlue  Approximate PANTONE 3145
\def\cltrp{\RoyalPurple}	  % RoyalPurple  Approximate PANTONE 267
\def\cltp{\Purple}	  % Purple  Approximate PANTONE PURPLE

\def\cgy{\GreenYellow}     % GreenYellow  Approximate PANTONE 388
\def\cyy{\Yellow}	  % Yellow  Approximate PANTONE YELLOW
\def\cgo{\Goldenrod}	  % Goldenrod  Approximate PANTONE 109
\def\cda{\Dandelion}	  % Dandelion  Approximate PANTONE 123
\def\capr{\Apricot}	  % Apricot  Approximate PANTONE 1565
\def\cpe{\Peach}		  % Peach  Approximate PANTONE 164
\def\cme{\Melon}		  % Melon  Approximate PANTONE 177
\def\cyo{\YellowOrange}	  % YellowOrange  Approximate PANTONE 130
\def\coo{\Orange}	  % Orange  Approximate PANTONE ORANGE-021
\def\cbo{\BurntOrange}	  % BurntOrange  Approximate PANTONE 388
\def\cbs{\Bittersweet}	  % Bittersweet  Approximate PANTONE 167
%\def\creo{\RedOrange}	  % RedOrange  Approximate PANTONE 179
\def\cma{\Mahogany}	  % Mahogany  Approximate PANTONE 484
\def\cmr{\Maroon}	  % Maroon  Approximate PANTONE 201
\def\cbr{\BrickRed}	  % BrickRed  Approximate PANTONE 1805
\def\crr{\Red}		  % Red  VERY-Approx PANTONE RED
\def\cor{\OrangeRed}	  % OrangeRed  No PANTONE match
\def\paru{\RubineRed}	  % RubineRed  Approximate PANTONE RUBINE-RED
\def\cwi{\WildStrawberry}  % WildStrawberry  Approximate PANTONE 206
\def\csa{\Salmon}	  % Salmon  Approximate PANTONE 183
\def\ccp{\CarnationPink}	  % CarnationPink  Approximate PANTONE 218
\def\cmag{\Magenta}	  % Magenta  Approximate PANTONE PROCESS-MAGENTA
\def\cvr{\VioletRed}	  % VioletRed  Approximate PANTONE 219
\def\parh{\Rhodamine}	  % Rhodamine  Approximate PANTONE RHODAMINE-RED
\def\cmu{\Mulberry}	  % Mulberry  Approximate PANTONE 241
\def\parv{\RedViolet}	  % RedViolet  Approximate PANTONE 234
\def\cfu{\Fuchsia}	  % Fuchsia  Approximate PANTONE 248
\def\cla{\Lavender}	  % Lavender  Approximate PANTONE 223
\def\cth{\Thistle}	  % Thistle  Approximate PANTONE 245
\def\corc{\Orchid}	  % Orchid  Approximate PANTONE 252
\def\cdo{\DarkOrchid}	  % DarkOrchid  No PANTONE match
\def\cpu{\Purple}	  % Purple  Approximate PANTONE PURPLE
\def\cpl{\Plum}		  % Plum  VERY-Approx PANTONE 518
\def\cvi{\Violet}	  % Violet  Approximate PANTONE VIOLET
\def\clrp{\RoyalPurple}	  % RoyalPurple  Approximate PANTONE 267
\def\cbv{\BlueViolet}	  % BlueViolet  Approximate PANTONE 2755
\def\cpe{\Periwinkle}	  % Periwinkle  Approximate PANTONE 2715
\def\ccb{\CadetBlue}	  % CadetBlue  Approximate PANTONE (534+535)/2
\def\cco{\CornflowerBlue}  % CornflowerBlue  Approximate PANTONE 292
\def\cmb{\MidnightBlue}	  % MidnightBlue  Approximate PANTONE 302
\def\cnb{\NavyBlue}	  % NavyBlue  Approximate PANTONE 293
\def\crb{\RoyalBlue}	  % RoyalBlue  No PANTONE match
%\def\cbb{\Blue}		  % Blue  Approximate PANTONE BLUE-072
\def\cce{\Cerulean}	  % Cerulean  Approximate PANTONE 3005
\def\ccy{\Cyan}		  % Cyan  Approximate PANTONE PROCESS-CYAN
\def\cpb{\ProcessBlue}	  % ProcessBlue  Approximate PANTONE PROCESS-BLUE
\def\csb{\SkyBlue}	  % SkyBlue  Approximate PANTONE 2985
\def\ctu{\Turquoise}	  % Turquoise  Approximate PANTONE (312+313)/2
\def\ctb{\TealBlue}	  % TealBlue  Approximate PANTONE 3145
\def\caq{\Aquamarine}	  % Aquamarine  Approximate PANTONE 3135
\def\cbg{\BlueGreen}	  % BlueGreen  Approximate PANTONE 320
\def\cem{\Emerald}	  % Emerald  No PANTONE match
%\def\cjg{\JungleGreen}	  % JungleGreen  Approximate PANTONE 328
\def\csg{\SeaGreen}	  % SeaGreen  Approximate PANTONE 3268
\def\cgg{\Green}	  % Green  VERY-Approx PANTONE GREEN
\def\cfg{\ForestGreen}	  % ForestGreen  Approximate PANTONE 349
\def\cpg{\PineGreen}	  % PineGreen  Approximate PANTONE 323
\def\clg{\LimeGreen}	  % LimeGreen  No PANTONE match
\def\cyg{\YellowGreen}	  % YellowGreen  Approximate PANTONE 375
\def\cspg{\SpringGreen}	  % SpringGreen  Approximate PANTONE 381
\def\cog{\OliveGreen}	  % OliveGreen  Approximate PANTONE 582
\def\pars{\RawSienna}	  % RawSienna  Approximate PANTONE 154
\def\cse{\Sepia}		  % Sepia  Approximate PANTONE 161
\def\cbr{\Brown}		  % Brown  Approximate PANTONE 1615
\def\cta{\Tan}		  % Tan  No PANTONE match
\def\cgr{\Gray}		  % Gray  Approximate PANTONE COOL-GRAY-8
\def\cbl{\Black}		  % Black  Approximate PANTONE PROCESS-BLACK
\def\cwh{\White}		  % White  No PANTONE match


\loadmsbm

\input epsf

\def\ctln{\centerline}
\def\u{\underbar}
\def\ssk{\smallskip}
\def\msk{\medskip}
\def\bsk{\bigskip}
\def\hsk{\hskip.1in}
\def\hhsk{\hskip.2in}
\def\dsl{\displaystyle}
\def\hskp{\hskip1.5in}

\def\lra{$\Leftrightarrow$ }
\def\ra{\rightarrow}
\def\mpto{\logmapsto}
\def\pu{\pi_1}
\def\mpu{$\pi_1$}
\def\sig{\Sigma}
\def\msig{$\Sigma$}
\def\ep{\epsilon}
\def\sset{\subseteq}
\def\del{\partial}
\def\inv{^{-1}}
\def\wtl{\widetilde}
%\def\lra{\Leftrightarrow}
\def\del{\partial}
\def\delp{\partial^\prime}
\def\delpp{\partial^{\prime\prime}}
\def\sgn{{\roman{sgn}}}
\def\wtih{\widetilde{H}}
\def\bbz{{\Bbb Z}}
\def\bbr{{\Bbb R}}




Now all we need are some new chain complexes! 

\msk



{\bf Relative homology:} Start with
a pair $(X,A)$ , i.e., of a space $X$ and a subspace $A\subseteq X$ .
As abelian groups we can think of 
$C_n(A)$ as a subgroup of $C_n(X)$ (under the injective homom induced by the 
inclusion $i:A\ra X$), and we can set $C_n(X,A)= C_n(X)/C_n(A)$ . The
boundary map $\del_n:C_n(X)\ra C_{n-1}(X)$ satisfies
$\del_n(C_n(A)\subseteq C_{n-1}(A)$ (the boundary of a map into $A$ maps into $A$),
so we get an induced boundary map $\del_n:C_n(X,A)\ra C_{n-1}(X,A)$ . The
$(C_n(X,A),\del_n)$ are a chain complex; its homology groups 
are the {\it singular relative homology groups of the pair} $(X,A)$,
denoted $H_n(X,A)$. A cycle is $[z]$ with $\del z\in C_{n-1}(A)$,
i.e., a chain with boundary in $A$. A boundary satisfies
$z=\del w +a$ for some $w\in C_{n+1}(X)$ and $a\in C_n(A)$ , i.e., it {\it cobounds}
a chain in $A$ ($\del w = z-a$). Note that the relative homology of the pair 
$(X,\emptyset)$ is just the ordinary homology of $X$; we quotient out by nothing.

\ssk

There is a reduced relative homology 
as well, since we can augment with the same map (1-simplices always have 2 ends!),
but in this case it has (essentially) no effect; $\widetilde{H}_i(X,A)\cong H_i(X,A)$
for all $i$ \u{unless} $A=\emptyset$, in which case we lose the ${\Bbb Z}$ in
dimension 0 that we expect to. 

\ssk

The inclusion $i_n$ and projection  $p_n$ maps give us short exact sequences 

\ssk

\ctln{$0\ra C_n(A)\ra C_n(X) \ra C_n(X,A)\ra 0$} 

and since the boundary on chains
in $X$ restricts to the boundary on $A$ and induces the boundary on $(X,A)$,
$i_n$ and $p_n$ are chain maps. So we get a long exact homology sequence

\ssk

\cltr{\ctln{$\cdots \ra H_n(A) \ra H_n(X) \ra H_n(X,A) \ra H_{n-1}(A) \ra H_{n-1}(X) \ra \cdots$}}

\ssk

There is also a long exact sequence of a triple $(X,A,B)$ , where by triple we
mean $B\sset A\sset X$ . From the short exact sequences 

\ssk

\ctln{$0\ra C_n(A,B) \ra C_n(X,B) \ra C_n(X,A)\ra 0$ \hhsk , i.e.,}

\ctln{$0\ra C_n(A)/C_n(B) \ra C_n(X)/C_n(B) \ra C_n(X)/C_n(A)\ra 0$} 

\ssk

we get the long exact sequence 

\ssk

\cltr{\ctln{$\cdots \ra H_n(A,B) \ra H_n(X,B) \ra H_n(X,A) \ra H_{n-1}(A,B) 
\ra H_{n-1}(X,B) \ra \cdots$}}

\msk

A map of pairs $f: (X,A) \ra (Y,B)$
induces (by postcomposition) a homom of relative homology $f_*:H_i(X,A)\ra H_i(Y,B)$ , just as with 
absolute homology.
We also get a homotopy-invariance result: if $f,g: (X,A) \ra (Y,B)$ are maps of pairs
which are {\it homotopic as maps of pairs},
i.e., there is a map $(X\times I,A\times I)\ra (Y,B)$ which is $f$ on one end and $g$ on the other, 
then $f_*=g_*$ . The proof is identical to the one given before; the prism map 
$P$ sends chains in $A$ to chains in $A$, so induces a map $C_i(X\times I,A\times I)\ra C_{i+1}(X,A)$
which does precisely what we want.

\vfill
\eject

There is one other main piece of homological algebra
that we will find useful ; the {\bf Five Lemma}. Now that we have a way of building long exact sequences, we 
will soon have ways of building maps between them. So the next result becomes useful.

\msk

\cltr{If we have abelian groups and homoms, giving two exact sequences}

\ssk

\cltr{\ctln{$\displaystyle 
\matrix 
A_n&{\buildrel f_n\over\ra}&B_n&{\buildrel g_n\over\ra}&C_n & {\buildrel h_n\over\ra} & D_n & {\buildrel i_n\over\ra} & E_n\cr
\alpha\downarrow & & \beta\downarrow & & \gamma\downarrow & & \delta\downarrow & & \epsilon\downarrow & \cr
A_{n-1} & {\buildrel f_{n-1}\over\ra} & B_{n-1} & {\buildrel g_{n-1}\over\ra} & C_{n-1} & {\buildrel h_{n-1}\over\ra} & D_{n-1} & {\buildrel i_{n-1}\over\ra} & E_{n-1}\cr
\endmatrix$}}

\ssk

\cltr{and the homoms $\alpha,\beta,\delta,\epsilon$ are all isomorphisms, then $\gamma$ is an isomorphism.}

\msk

The proof is literally a matter of doing the only thing you can. To show injectivity, suppose
$x\in C_n$ and $\gamma x = 0$, 
then $h_{n-1}\gamma x = \delta h_n x = 0$, so, 
since $\delta$ is injective,
$h_n x = 0$. So 
by the exactness at $C_n$, $x=g_n y$ for some $y\in B_n$. 
Then $g_{n-1} \beta y = \gamma g_n y = \gamma x = 0$, so 
by exactness at $B_{n-1}$, $\beta y = f_{n-1} z$ for some $z\in A_{n-1}$. Then 
since $\alpha$ is surjective,
$f_{n-1}z = \alpha w$ for some $w$. Then 
$0=g_n f_n w$ . But 
$\beta f_n w = f_{n-1} \alpha w \ f_{n-1} z = \beta y$, so since
$\beta$ is injective, 
$y= f_n w$ . So $0=g_n f_n w = g_n y = x$. So $x=0$.

\ssk

For surjectivity, suppose $x\in C_{n-1}$.
Then $h_{n-1} x \in D_{n-1}$, so
since $\delta$ is surjective, 
$h_{n-1} x = \delta y$ for some $y\in D_n$.
Then $\epsilon i_n y = i_{n-1}\delta y = i_{n-1} h_{n-1} x = 0$, so 
since $\epsilon$ is injective, $i_n y= 0$.
So by exactness at $D_n$, 
$y=h_n z$ for some $z\in C_n$. Then 
$h_{n-1}\gamma z = \delta h_n z = \delta y = h_{n-1} x$,
so $h_{n-1} (\gamma z-x) = 0$, so 
by exactness at $C_{n-1}$, 
$\gamma z-x = g_{n-1}w$ for some $w\in B_{n-1}$. Then
since $\beta$ is surjective, 
$w= \beta u$ for some $u\in B_n$. Then 
$\gamma g_n u = g_{n-1} \beta u = g_{n-1}w = \gamma z-x$,
so $x=\gamma z - \gamma g_n u = \gamma (z-g_n u)$ . 
So $\gamma$ is onto.


\vfill
\end


In the end, the big result that allows us to get our homology machine really running is what is
known as {\bf excision}. In a sense, it is the analogue of
Seifert - van Kampen. We start with $X=A\cup B$, and we want to
express the homology of $X$ in terms of that of $A$, $B$, and $A\cap B$. 
With long exact homology sequences in mind, we try to first build
a short exact sequence out of the chain complexes $C_*(A\cap B), C_*(A), C_*(B)$, and $C_*(X)$.
Taking our cue from the \u{proof} of S-vK, we might think of chains in $X$ as sums of 
chains in $A$ and $B$, except that we mod out by chains in $A\cap B$. So
we might try the sequence

\ssk

\ctln{$0\ra C_n(A\cap B) \ra C_n(A)\oplus C_n(B) \ra C_n(X) \ra 0$}

\ssk

where $j_n:C_n(A)\oplus C_n(B) \ra C_n(X)$ is defined as $j_n(a,b)=a+b$ . In order to get exactness
at the middle term (i.e., image = the kernel of this map, which is $\{ (x,-x) : x\in C_n(A)\cap C_n(B)\}$),
we set $i_n:C_n(A\cap B) \ra C_n(A)\oplus C_n(B)$ to be $i_n(x) = (x,-x)$ , since
$C_n(A\cap B) = C_n(A)\cap C_n(B)$ ! $i_n$ is then injective, and we certainly have that
this sequence is exact at the middle term. But, in general, $j_n$ is far from surjective! The image of $j_n$
is the set of $n$-chains that can be expressed as sums of chains in $A$ and $B$. Which of course
not every chain in $X$ can be; singular simplices in $X$ need not map entirely
into either $A$ or $B$. 

\msk

We can solve this by \u{replacing} $C_n(X)$ with the image of $j_n$, calling it, say,
$C_n^{\{A,B\}}(X)$. [Note: these groups \u{would} form a chain complex!]
Then we have a short exact sequence, and hence a long exact homology sequence.
But it involves a ``new'' homology group $H_n^{\{A,B\}}(X)$ . The \u{point} is that, like S-vK,
under the right conditions, this new homology is the same as $H_n(X)$ !

\vfill
\eject

Starting from scratch, the idea is that, starting with an {\it open cover} $\{{\Cal U}_\alpha\}$
of $X$ (or, more generally, with a collections of subspaces $A_\alpha$ whose interiors 
${\Cal U}_\alpha$ cover $X$), we build the {\it chain groups subordinate to the cover},

\ssk

\ctln{$C_n^{\Cal U}(X) = \{\sum a_i \sigma_i^n : \sigma_i : n:\Delta^n\ra X , \sigma_i^n(\Delta^n)\subseteq {\Cal U}_\alpha$
for some $\alpha\}$ $\subseteq$ $C_n(X)$.}
 
\ssk

Since the face of any simplex mapping into ${\Cal U}_\alpha$ also maps into ${\Cal U}_\alpha$,
our ordinary boundary maps induce boundary maps on these groups, turning
$(C_n^{\Cal U}(X),\partial_n)$ into a chain complex. Our main result is that the inclusion
$i$ of these groups into $C_n(X)$ induces an isomophism on homology. And to show this, we (could) once
again use the notion of a chain homotopy.

\ssk

{\bf Theorem:} There is a chain map $b:C_n(X)\ra C_n^{\Cal U}(X)$ so that $i\circ b$ and $b\circ i$ are both chain 
homotopic to the identity. $i$ consequently induces isomorphisms on homology.

\ssk

But we won't prove it quite that way! Another approach is to use the short exact sequence of chain 
complexes

\ssk

\ctln{$0 \ra C_n^{\Cal U}(X) {\buildrel i\over \ra} C_n(X) \ra C_n(X)/C_n^{\Cal U}(X) \ra 0$}

\ssk

to build a long exact homology sequence. Every third group is $H_n(C_*(X)/C_*^{\Cal U}(X))$~;
if we show that these groups are $0$, then $i_*$ will be an isomorphism. And to show \u{this},
working back through the definition of homology classes in $H_n(C_*(X)/C_*^{\Cal U}(X))$,
we need to show that if $z\in C_n(X)$ with $\del z\in C_{n-1}^{\Cal U}(X)$ (i.e., $z$ is a relative cycle), 
then there is a $w\in C_{n+1}(X)$ with $z-\del w\in C_n^{\Cal U}(X)$ (i.e., $z$ is a relative boundary).
In words, if $z$ has boundary a sum of small simplices, then there is a chain $z^\prime$
made of small simplices so that $z-z^\prime$ is a boundary.

\vfill
\eject




And the key to building $z^\prime$ and $w$ is a process known as {\it barycentric subdivision}.
The idea is really the same as for S-vK; we cut our singular simplices up into tiny enough
pieces so that (via the Lebesgue number theorem) 
each piece maps into \u{some} ${\Cal U}_\alpha$ . Unlike S-vK, though, we want to do 
this in a more structured way, so that the cutting up process is ``compatible'' with our
boundary maps. And the best way to describe this cutting up is through {\it barycentric
coordinates}. Recall that an $n$-simplex is the set of convex linar combinations
$\sum x_i v_i$ with $x_i\geq 0$ and $\sum x_i=1$ . The map which sends \u{an}
$n$-simplex to \u{the} $n$-simplex $\Delta^n$ is literally the map 
$\sum x_i v_i \mapsto (x_0,\ldots ,x_n)$ . These are the barycentric coordinates of an $n$-simplex.
Since, formally, all singular simplices are considered to have $\Delta^n$ for their
domain, we can describe barycentric subdivision by describing how to cut up $\Delta^n$.
The idea is to build a process that is compatible with the boundary map, so that the
subdivision, when restricted to a sub-simplex, is the subdivision of that sub-simplex.
A 1-simplex $[v_0,v_1]$ is subdivided by adding the barycenter $w=(v+0+v_1)/2$ as a vertex,
cutting $[v_0,v_1]$ into two 1-simplices ,$[v_0,w]$,$[w,v_1]$ . A 2-simplex 
$[v_0,v_1,v_2]$ will, to be compatible with the boundary map,
have its boundary cut into 6 1-simplices; using the barycenter $(v_0+v_1+v_2)/3$
we can cone off each of these 1-simplices to subdivide $[v_0,v_1,v_2]$
into 6 2-simplices. Taking the cue that $2=(1+1)!$ , $6=(2+1)!$ is probably no accident, we
might expect that an $n$-simplex will be cut into $(n+1)!$ $n$-simplices. 
Note that this is the number of ways of ordering the vertices of our simplex. 
And following the ``pattern'' of our two test cases, where each new simplex was the convex
span of vertices chosen as (vertex) , (barycenter of a 1-simplex having (vertex) as a vertex),
(barycenter of a 2-simplex containing the previous 2 vertices), etc., we are led to the idea that
the barycentric subdivision of an $n$-simplex $[v_0,\ldots , v_n]$ is the 
$(n+1)!$ $n$-simplices, 

\ctln{$[v_{\alpha(0)},(v_{\alpha(0)}+v_{\alpha(1)})/2,(v_{\alpha(0)}+v_{\alpha(1)}+v_{\alpha(2)})/3,\ldots,
(v_{\alpha(0)}+\cdots v_{\alpha(n)})/(n+1)]$}

one for every permutation $\alpha$ of $\{0,\ldots ,n\}$ . And since we want to take into account
\u{orientations} as well, the natural thing to do is to define the barycentric subdivision of a singular
$n$-simplex $\sigma:[v_0,\ldots ,v_n]\ra X$ to be

\ctln{$\displaystyle S(\sigma) = \sum_\alpha (-1)^{\sgn (\alpha)} \sigma
|_{[v_{\alpha(0)},(v_{\alpha(0)}+v_{\alpha(1)})/2,(v_{\alpha(0)}+v_{\alpha(1)}+v_{\alpha(2)})/3,\ldots,
(v_{\alpha(0)}+\cdots v_{\alpha(n)})/(n+1)]}$}

where the sum is taken over all permutations of $\{0,\ldots ,n\}$ .
This (extending linearly over the chain group) is the subdivision operator, $S:C_n(X)\ra C_n(X)$ . 
A ``routine'' calculation establishes that $\del S = S\del$ , i.e., it is a chain map
(i.e., it behaves well on the boundary of our simplices). The point to this operator is that all of the 
subsimplices in the sum are a definite \u{factor} smaller than the original simplex. In fact,
if the diameter of $[v_0,\ldots ,v_n]$ is $d$ (the largest distance between points, which will,
because it is the convex span of the vertices, be the largest distance between vertices), then
every individual simplex in $S(\sigma)$ will have diameter at most $nd/(n+1)$ (the result of a little
Euclidean geometry and induction). So by \u{repeatedly} applying the subdivision operator
$S$ to a singular simplex, we will obtain a singular chain $S^k(\sigma)$,
which is ``really'' $\sigma$ written as a sum of tiny simplices, whose singular simplices 
have image as small as we want. Or put more succinctly, if $\{{\Cal U}_\alpha\}$ is an open cover of $X$
and $\sigma:\Delta^n\ra X$ is a singular $n$-simplex, then choosing a Lebesgue number $\epsilon$ for
the open cover $\sigma^{-1}({\Cal U}_\alpha)$ of the compact metric space $\Delta^n$, and choosing 
a $k$ with $d(n/(n+1))^k<\epsilon$, we find that $S^k(\sigma)$ is a sum of singular simplices
each of which maps into one of the ${\Cal U}_\alpha$, i.e., $S^k(\sigma)\in C_n^{\Cal U}(X)$.

\vfill
\eject

In the end, we will choose our needed ``small'' cycle to be $z^\prime = S^k z$. and to show that their difference
is a boundary, we will build a chain homotopy between $Id$ and $S^k$.
And to do \u{that}, we define a map $R:C_n(X)\ra C_{n+1}(X\times I)$; when followed by the projection-induced
map $p_\# : C_{n+1}(X\times I)\ra C_{n+1}(X)$, we get a map $T:C_n(X)\ra C_{n+1}(X)$,
and show that $\del T + T\del = I-S$ .
Then we set $H$ = $\sum TS^j$, where the sum is taken over $j=0,\ldots k-1$. Once we define $T$ (!) , we will have
$\del H_k+H_k\del = \sum \del TS^j + TS^j \del = \sum (\del T+T\del) S^j =\sum (S^j-S^{j+1}) = I-S^k$
(since the last sum telescopes). And defining $R$, is, formally, just another particular sum.
Setting up some notation,
thinking of $\Delta^n\times I$ , as before, as having vertices $\{v_0,\ldots v_n\}$ on the 0-end and 
$\{w_0,\ldots ,w_n\}$ on the 1-end,  $N=\{0,\ldots ,n\}$, $\Pi(Q)$ = the group of permutations of $Q$,
and $\sigma^\prime = \sigma\times I:\Delta^n\times I\ra X\times I$), we have

\ssk

$\displaystyle R(\sigma) = 
\sum_{A\subseteq N}\sum_{\pi\in\Pi(N\setminus A)}\big\{ (-1)^{|A|}(-1)^{\sgn(\pi)}\prod_{j\in N\setminus A}(-1)^j \big\}$

\hfill $\displaystyle  \sigma^\prime
|_{[v_{i_0},\ldots ,v_{i_j},(w_{i_0}+\cdots w_{i_j})/(j+1),
(w_{i_0}+\cdots w_{i_j}+w_{\pi(i_{j+1}})/(j+2),
\ldots ,
(w_{i_0}+\cdots w_{i_j}+w_{\pi(i_{j+1})}+\cdots w_{\pi(i_n)})/(n+1)]}$

\ssk

where we sum over all \u{non-empty} subsets of $\{0,\ldots n\}$ (with the induced ordering on vertices
from the ordering on $\{0,\ldots ,n\}$).
Intuitively, this map ``interpolates'' between the simplex $[v_0,\ldots v_n]$ and the 
barycentric subdivision on $w_0,\ldots ,w_n$, by taking the (signed sums of the) convex spans of
simplices on the bottom (0) and simplices on the top (1). Again, a ``routine'' calculation will 
establish that $\del T + T\del = I-S$ , as desired. [At any rate, I verified it for n=1,2; the formula
for the sign of each simplex was determined by working backwards from these examples.]

\vfill
\eject

{\bf Homology on ``small'' chains = singular homology:} \hsk
The point to all of these calculations was that if $\{{\Cal U}_\alpha\}$ is an open cover of $X$, then the 
inclusions $i_n:C_n^{\Cal U}(X)\ra C_n(X)$ induce isomorphisms on homology. This gives us two
big theorems. First:

\ssk

{\bf Mayer-Vietoris Sequence}: If $X={\Cal U}\cup{\Cal V}$ is the union of two open sets, then
the short exact sequences \hhsk 
$0\ra C_n({\Cal U}\cap {\Cal V}) \ra C_n({\Cal U})\oplus C_n({\Cal V}) \ra C_n^{\{ {\Cal U},{\Cal V}\}}(X)\ra 0$
\hhsk , together with the isomorphism above, give the long exact sequence

\ctln{$\cdots \ra H_n({\Cal U}\cap {\Cal V}) {\buildrel{(i_{{\Cal U}*},-i_{{\Cal V}*})}\over\ra}
H_n({\Cal U})\oplus H_n({\Cal V}) {\buildrel{j_{{\Cal U}*}+j_{{\Cal V}*}}\over\ra}H_n(X)
{\buildrel \del\over\ra} H_{n-1}({\Cal U}\cap {\Cal V}) \ra \cdots$}

\ssk

As with Seifert - van Kampen, we can replace open sets by sets $A,B$ having nbhds ${\Cal U},{\Cal V}$ which def.
retract to them, so that ${\Cal U}\cap{\Cal V}$ def. retracts to $A\cap B$. E.g.,
subcomplexes $A,B\sset X$ of a CW-complex, with $A\cup B = X$ have homology satisfying a l.e.s.

\ssk

\ctln{$\cdots \ra H_n(A\cap B) {\buildrel{(i_{A*},-i_{B*})}\over\ra}
H_n(A)\oplus H_n(B) {\buildrel{j_{A*}+j_{B*}}\over\ra}H_n(X)
{\buildrel \del\over\ra} H_{n-1}(A\cap B) \ra \cdots$}

\ssk

For reduced homology, we augment the chain complexes used above with the 
s.e.s. \hhsk $0\ra {\Bbb Z}\ra {\Bbb Z}\oplus {\Bbb Z} \ra {\Bbb Z} \ra 0$ , 
where the maps are $a\mapsto (a,-a)$ and $(a,b)\mapsto a+b$ .

\msk

E.g., an $n$-sphere $S^n$
is the union $S^n_+\cup S^n_-$ of its upper and lower hemispheres, each of which 
is contractible, and have intersection $S^n_+\cap S^n_-=S^{n-1}_0$ the equatorial
$(n-1)$-sphere. So Mayer-Vietoris gives us the exact sequence

\hhsk $\cdots \ra \widetilde{H}_k(S^n_+)\oplus \widetilde{H}_k(S^n_-) \ra \widetilde{H}_k(S^n)
\ra \widetilde{H}_{k-1}(S^{n-1}_0) \ra \widetilde{H}_{k-1}(S^n_+)\oplus \widetilde{H}_{k-1}(S^n_-)\ra \cdots$ 
\hhsk , i.e, \hhsk

$0 \ra \widetilde{H}_k(S^n)\ra \widetilde{H}_{k-1}(S^{n-1}_0) \ra  0$ \hhsk 
i.e., $\widetilde{H}_k(S^n)\cong \widetilde{H}_{k-1}(S^{n-1})$ for every $k$ and $n$. 
So by induction, 

\ssk

\ctln{$\widetilde{H}_k(S^n)\cong\widetilde{H}_{k-n}(S^0)\cong    
\cases
{\Bbb Z}, & \text{if}\ $k=n$\cr
0, & \text{otherwise} $ $\cr 
\endcases$}

\vfill
\eject

The second result that this machinery gives us is what is properly known as {\it excision}:

\msk

If $B\sset A\sset X$ and cl$_X(B)\sset$ int$_X(A)$, then for every $k$ the inclusion-induced map 
$H_k(X\setminus B,A\setminus B)\ra H_k(X,A)$ is an isomorphism. 

\msk

An equivalent formulation of this is that if $A,B\sset X$ and int$_X(A)\cup$ int$_X(B) = X$, then the
inclusion-induced map $H_k(B,A\cap B)\ra H_k(X,A)$ is an isomorphism. [From first to second
statement, set $B^\prime = X\setminus B$ .] 

\ssk

To prove the second statement, we know that
the inclusions $C_n^{\{A,B\}}(X) \ra C_n(X)$ induce isomorphisms on homology, as does 
$C_n(A) \ra C_n(A)$, so, by the five lemma, the induced map

\ctln{$C_n^{\{A,B\}}(X)/C_n(A) \ra C_n(X)/C_n(A) = C_n(X,A)$}

induces isomorphisms on homology. 
But the inclusion 

\ctln{$C_n(B) \ra C_n^{\{A,B\}}(X)$}

induces a map 

\ctln{$C_n(B,A\cap B) = C_n(B)/C_n(A\cap B) \ra C_n^{\{A,B\}}(X)/C_n(A)$}

which is an isomorphism of chain groups;
a basis for $C_n^{\{A,B\}}(X)/C_n(A)$ consists of singular simplices which map into $A$ or $B$, but don't map into $A$,
i.e., of simplices mapping into $B$ but not $A$, i.e., of simplices mapping into $B$ but not $A\cap B$.
But this is the \u{same} as the basis for $C_n(B,A\cap B)$ !

\vfill
\eject

With these tools, we can start making some \u{real} homology computations. First, we show that 
if $\emptyset\neq A\sset X$ is ``nice enough'', then $H_n(X,A)\cong \widetilde{H}_n(X/A)$ .
The definition of nice enough, like Seifert - van Kampen, is that $A$ is closed and has an open neighborhood
${\Cal U}$ that deformation retracts to $A$ (think: $A$ is the subcomplex of a CW-complex $X$).
Then using ${\Cal U},X\setminus A$ as a cover of $X$, and ${\Cal U}/A,(X\setminus A)/A$ as a cover of $X/A$,
 we have

\ssk

$\widetilde{H}_n(X/A) {\buildrel {(1)}\over \cong} H_n(X/A,A/A){\buildrel {(2)}\over \cong} 
H_n(X/A,{\Cal U}/A) {\buildrel {(3)}\over \cong} H_n(X/A\setminus A/A,{\Cal U}/A\setminus A/A) {\buildrel {(4)}\over \cong}
H_n(X\setminus A,{\Cal U}\setminus A){\buildrel {(5)}\over \cong} H_n(X,A)$

\ssk

Where (1),(2) follow from the LES for a pair, (3),(5) by excision, and (4) because the restriction of the quotient
map $X\ra X/A$ gives a homeomorphism of pairs.

\msk

Second, if $X,Y$ are $T_1$, $x\in X$ and $y\in Y$ each have neighborhoods 
${\Cal U},{\Cal V}$ which deformation retract to each point, then the 
one-point union $Z=X\vee Y = (X\coprod Y)/(x=y)$ has $\widetilde{H}_n(Z) \cong \widetilde{H}_n(X)\oplus \widetilde{H}_n(Y)$;
this follows from a similar sequence of isomorphisms. Setting $z$ = the image of $\{x,y\}$ in $Z$, we have

\ssk

$\widetilde{H}_n(Z) \cong H_n(Z,z) \cong H_n(Z,{\Cal U}\vee{\Cal V}) \cong H_n(Z\setminus z,{\Cal U}\vee{\Cal V}\setminus z)
\cong H_n([X\setminus x]\coprod[Y\setminus y],[{\Cal U}\setminus x]\coprod [{\Cal V}\setminus y])
\cong H_n(X\setminus x,{\Cal U}\setminus x)\oplus H_n(Y\setminus y,{\Cal V}\setminus y) 
\cong H_n(X,x)\oplus H_n(Y,y)\cong \widetilde{H}_n(X)\oplus \widetilde{H}_n(Y)$

\ssk

By induction, we then have $\displaystyle \widetilde{H}_n(\vee_{i=1}^k X_i) \cong \oplus_{i=1}^k \widetilde{H}_n(X_i)$

\vfill
\end







