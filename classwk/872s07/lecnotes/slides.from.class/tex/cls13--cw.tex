
\magnification=2000
\overfullrule=0pt
\parindent=0pt

\nopagenumbers

\input amstex

%\voffset=-.6in
%\hoffset=-.5in
%\hsize = 7.5 true in
%\vsize=10.4 true in

\voffset=1.8 true in
\hoffset=-.6 true in
\hsize = 10.2 true in
\vsize=8 true in

\input colordvi

\def\cltr{\Red}		  % Red  VERY-Approx PANTONE RED

\loadmsbm

\input epsf

\def\ctln{\centerline}
\def\u{\underbar}
\def\ssk{\smallskip}
\def\msk{\medskip}
\def\bsk{\bigskip}
\def\hsk{\hskip.1in}
\def\hhsk{\hskip.2in}
\def\dsl{\displaystyle}
\def\hskp{\hskip1.5in}

\def\lra{$\Leftrightarrow$ }
\def\ra{\rightarrow}
\def\mpto{\logmapsto}
\def\pu{\pi_1}
\def\mpu{$\pi_1$}
\def\sig{\Sigma}
\def\msig{$\Sigma$}
\def\ep{\epsilon}
\def\sset{\subseteq}
\def\del{\partial}
\def\inv{^{-1}}
\def\wtl{\widetilde}
%\def\lra{\Leftrightarrow}
\def\del{\partial}
\def\delp{\partial^\prime}
\def\delpp{\partial^{\prime\prime}}
\def\sgn{{\roman{sgn}}}
\def\wtih{\widetilde{H}}
\def\bbz{{\Bbb Z}}
\def\bbr{{\Bbb R}}
\def\rtar{$\Rightarrow$}

\def\cltr{\Red}		  % Red  VERY-Approx PANTONE RED
\def\cltb{\Blue}		  % Blue  Approximate PANTONE BLUE-072
\def\cltg{\PineGreen}	  % ForestGreen  Approximate PANTONE 349



{\bf CW complexes:} The ``right'' spaces to do algebraic topology on.

\msk

The basic idea: CW complexes are built inductively, by gluing 
disks onto lower-dimensional strata. $X=\bigcup X^{(n)}$, where

\ssk

$X^{(0)}$ = a disjoint union of points, and, inductively,

\ssk

$X^{(n)}$ is built from $X^{(n-1)}$ by gluing $n$-disks $D^n_i$
along their boundaries. That is we have $f_i:\del D^n_i\ra X^{(n-1)}$
and $X^{(n)}=X^{(n-1)}\cup(\coprod D^n_i)/\sim$ where
$x\sim f_i(x)$ for all $x\in\del D^n_i$. We have (natural)
inclusions $X^{(n-1)}\subseteq X^{(n)}$, and $X=\bigcup X^{(n)}$
is given the {\it weak topology}; that is,  $C\subseteq X$ is closed \lra\
$C\cap X^{(n)}$ is closed for all $n$. 

(Note: this is reasonable;
$X^{(n-1)}$ is closed in $X^{(n)}$ for all $n$.)

\ssk

Each disk $D^n_i$ has a {\it characteristic map} $\phi_i:D^n_i\ra X$
given by 

$D^n_i\ra X^{(n-1)}\cup(\coprod D^n_i)\ra X^{(n)}\subseteq X$.

$f:X\ra Y$ is cts \lra\ $f\circ \phi_i:D^n_i\ra X\ra Y$ is cts for 
all $D^n_i$. 

(This is a consequence of using the weak topology.) 

\msk

A {\it CW pair} $(X,A)$ is a CW complex $X$ and a {\it subcomplex}
$A$, which is a subset which is a union of images of cells, so it is 
a CW complex in its own right. We can induce CW structures under
many standard constructions; e.g., if $(X,A)$ is a CW pair, then
$X/A$ admits a CW structure whose cells are $[A]$ and the cells of 
$X$ not in $A$. We can glue two CW complexes $X,Y$ along isomorphic
subcomplexes $A\subseteq X,Y$, yielding $X\cup_AY$.

\msk

``CW''=closure finiteness, weak topology

\vfill
\eject

Perhaps the most important property of CW complexes (for algebraic topology,
anyway) is the {\it homotopy extension property}; given a CW pair
$(X,A)$, a map $f:X\ra Y$, and a homotopy $H:A\times I\ra Y$ such that
$H|_{A\times 0}=f|_A$, there is a homotopy (extension)
$K:X\times I\ra Y$ with $K|_{A\times I}=H$. This is because 
$B=X\times\{0\}\cup A\times I$ is a retract of $X\times I$; $K$ is the 
composition of this retraction and the ``obvious'' map from $B$ to $Y$.

\ssk

To build the retraction, we do it one cell of $X$ at a time. The idea is that
the retraction is defined on the cells of $A$ (it's the identity), so look
at cells of $X$ not in $A$. Working our way up in dimension, we can assume
the the retraction $r_{n-1}$
is defined on (the image of) $\del D^n\times I$, i.e., on $X^{(n-1)}\times I$.
But $D^n\times I$ (strong deformation) retracts onto 
$D^n\times 0\cup \del D^n\times I$; composition of $r_{n-1}$ with 
this retraction extends the retraction over $\phi(D^n)\times I$, and 
so over $X^{(n)}\times I$..

\msk

This, for example, lets us show that if $(X,A)$ is a CW pair and 
$A$ is contractible, then $X/A\simeq X$. This is because the
composition $A\ra *\ra A$ is homotopic to the identity $I_A$, via some map
$H:A\times I\ra A$, with $H|_{A\times 0}=I_A$. Thinking of $H$ as
mapping into $X$, then together with the map $I_X:X\ra X$ the HEP
provides a map $K:X\times I\ra X$ with $K_0=I_X$ and $K_1(A)=*$.
Setting $K_1=g:X\ra X$, it induces a map $h:X/A\ra X$. This is a homotopy
inverse of the projection $p:X\ra X/A$:

\msk

$h\circ p=g\simeq I_X$ via $K$, and $p\circ h:X/A\ra X/A$
is homotopic to $I_{X/A}$ since $K_t(A)\subseteq A$ for every $t$,
so induces a map $\overline{K}_t:X/A\ra X/A$, giving a homotopy
between 

$\overline{K}_0=\overline{I_X}=I_{X/A}$ and 
$\overline{K}_1=\overline{g}=p\circ h$.


\vfill
\end







