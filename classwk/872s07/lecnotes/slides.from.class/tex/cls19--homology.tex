
\magnification=2000
\overfullrule=0pt
\parindent=0pt

\nopagenumbers

\input amstex

%\voffset=-.6in
%\hoffset=-.5in
%\hsize = 7.5 true in
%\vsize=10.4 true in

\voffset=1.8 true in
\hoffset=-.6 true in
\hsize = 10.2 true in
\vsize=8 true in

\input colordvi

\def\cltr{\Red}		  % Red  VERY-Approx PANTONE RED

\loadmsbm

\input epsf

\def\ctln{\centerline}
\def\u{\underbar}
\def\ssk{\smallskip}
\def\msk{\medskip}
\def\bsk{\bigskip}
\def\hsk{\hskip.1in}
\def\hhsk{\hskip.2in}
\def\dsl{\displaystyle}
\def\hskp{\hskip1.5in}

\def\lra{$\Leftrightarrow$ }
\def\ra{\rightarrow}
\def\mpto{\logmapsto}
\def\pu{\pi_1}
\def\mpu{$\pi_1$}
\def\sig{\Sigma}
\def\msig{$\Sigma$}
\def\ep{\epsilon}
\def\sset{\subseteq}
\def\del{\partial}
\def\inv{^{-1}}
\def\wtl{\widetilde}
%\def\lra{\Leftrightarrow}
\def\del{\partial}
\def\delp{\partial^\prime}
\def\delpp{\partial^{\prime\prime}}
\def\sgn{{\roman{sgn}}}
\def\wtih{\widetilde{H}}
\def\bbz{{\Bbb Z}}
\def\bbr{{\Bbb R}}
\def\rtar{$\Rightarrow$}

\def\cltr{\Red}		  % Red  VERY-Approx PANTONE RED
\def\cltb{\Blue}		  % Blue  Approximate PANTONE BLUE-072
\def\cltg{\PineGreen}	  % ForestGreen  Approximate PANTONE 349





{\bf Homology groups:} We start by defining $n$-{\it chains};
these are (finite) formal linear combinations of the (oriented!) $n$-simplices
of $X$, where $-\sigma$ is interpreted as $\sigma$ with the opposite
(i.e., other) orientation. Adding formal linear combinations formally,
we get the $n$-th {\it chain group} 
$C_n(X) = \{\sum n_\alpha \sigma_\alpha$ : $\sigma_\alpha$ an oriented $n$-simplex in $X\}$ .
We next define a {\it boundary operator} $\del:C_n(X)\ra C_{n-1}(X)$, whose image will be 
the $(n-1)$-chains that are the ``boundaries'' of $n$-chains. The idea is that the boundary of a 2-simplex,
for example, should be a ``sum'' of its three faces (since they do make up the boundary of the 
simplex), ``accounting for'' orientations. Thinking of the orientation on a 1-simplex $[v,w]$ 
as an arrow pointing from $v$ to $w$, we are led to 
believe that the boundary of a 2-simplex $[u,v,w]$ should be $[u,v]+[v,w]+[w,u]$. Similarly, the boundary 
of $[u,v]$, on reflection, should be $[v]-[u]$, to distinguish the head of the arrow (the $+$ side) from the
tail (the $-$ side). On the basis of these examples, trying to find a consistent formula, one might eventually
be led to the following definition:
\cltr{we define the boundary map on the basis
elements $\sigma_\alpha = \sigma$ of $C_n(X)$ as
$\del\sigma = \sum (-1)^i\sigma|_{[v_0,\ldots,\widehat{v_{i}},\ldots ,v_n]}$ , 
where $\sigma:[v_0,\ldots ,v_n]\ra X$ is the characteristic map of $\sigma_\alpha$} .
$\del\sigma$ is therefore an alternating sum of the faces of $\sigma$. 
We then extend the definition by linearity to all of $C_n(X)$. When a notation indicating
dimension is needed, we write $\del=\del_n$ . We define $\del_0=0$. 

\vfill
\eject

This definition, it turns out, is cooked up to make the maxim ``boundaries have no boundary'' true;
that is, $\del_{n-1}\circ \del_n = 0$, the $0$ map. This is because, for any simplex
$\sigma = [v_0,\ldots v_n]$, 

\msk

$\displaystyle \del\circ\del(\sigma) = 
\del(\sum_{i=0}^n  (-1)^i\sigma|_{[v_0,\ldots,\widehat{v_{i}},\ldots ,v_n]})$

= $\displaystyle (\sum_{j<i}(-1)^j(-1)^i\sigma|_{[v_0,\ldots,\widehat{v_{j}},\ldots,\widehat{v_{i}},\ldots ,v_n]})
+(\sum_{j>i}(-1)^{j-1}(-1)^i\sigma|_{[v_0,\ldots,\widehat{v_{i}},\ldots,\widehat{v_{j}},\ldots ,v_n]})$

\msk

The distinction between the two pieces is that in the second part, $v_j$ is actually the $(j-1)$-st vertex
of the face. Switching the roles of $i$ and $j$ in the second sum, we find that the two are
negatives of one another, so they sum to $0$, as desired.

\msk

And this calculation is all that it takes to define homology groups. What it tells 
us is that im$(\del_{n+1})\subseteq\ker(\del_{n})$ for every $n$. 
$\ker(\del_{n}=Z_n(X)$ is called the {\it $n$-cycles} of $X$; they are the $n$-chains with
$0$ (i.e., empty) boundary. They form a (free) abelian subgroup of $C_n(X)$. 
im$(\del_{n+1} = B_n(X)$ is the {\it $n$-boundaries} of $X$; they are, of course,
the boundaries of $(n+1)$-chains in $X$. The $n$-th homology group of $X$,
$H_n(X)$ is the quotient $Z_n(X)/B_n(X)$ ; it is an abelian group.

\vfill
\eject

A key observation is that the boundary maps $\del_n$ are linear, that is,
they are homomorphisms between the free abelian groups 
$\del_n:C_n(X)\ra C_{n-1}(X)$. Consequently, they can be expressed as
(integer-valued) matrices $\Delta_n$. Row reducing $\Delta_n$ 
(over the integers!) allows us to find a 
basis $v_1,\ldots ,v_k$ for $Z_n(X)$ (clearing denomenators
to get vectors over ${\Bbb Z}$). Then since $\Delta_n\Delta_{n+1}=0$, the
columns of $\Delta_{n+1}$ are in the kernel of $\Delta_n$, so can be
expressed as linear combinations of the $v_i$ . These combinations can be
determined by row reducing the augmented matrix
$( v_1\cdots v_k | \Delta_{n+1} )$ . This will row reduce to 
$\pmatrix I&|&C\\ 0&|&0\\ \endpmatrix$, and $C$ basically describes the boundaries
$B_n(X)$ in terms of the basis $v_1,\ldots ,v_k$ . The homology group
$H_n(X)$ is then the {\it cokernel} of $C$, i.e., ${\Bbb Z}^k/$im$C$ .
Note that $C$ will have integer entries, since we know that 
the columns of $\Delta_{n+1}$ can be expressed as integer linear
combinations of the $v_i$, and, being a basis, there is only
one such expression. 

\vfill
\eject

{\bf Some examples:} the Klein bottle $K$ has a $\Delta$-complex structure with 2 2-simplices,
3 1-simplices, and 1 0-simplex; we will call them 
$f_1=[0,1,2],f_2=[1,2,3],
e_1=[0,2]=[1,3],e_2=[1,0]=[2,3],e_3=[1,2]$, 
and $v_1=[0]=[1]=[2]=[3]$.
Computing, we find 
$\del_2 f_1 = \del[0,1,2]=[1,2]-[0,2]+[0,1]=e_3-e_1-e_2$ , $\del_2 f_2 = e_2-e_1+e_3$ , 
$\del_1 e_1 = \del_1 e_2 = \del_1 e_3 = 0$ and $\del_i = 0$ for all other $i$
(as well). So we have the chain complex

$\cdots \ra 0 \ra {\Bbb Z}^2 \ra {\Bbb Z}^3 \ra {\Bbb Z} \ra 0$

and all of the boundary maps are 0, except for $\del_2$, which has the matrix
$\pmatrix
 -1&-1\\ -1&1\\ 1&1\\
\endpmatrix$ . This matrix is injective, so $\ker \del_2 = 0$,
so $H_2(K)=0$, on the other hand, $H_1(K)$ = coker$(\del_2)$, and applying column
operations we can transform the matrix for $\del_2$ to $\pmatrix 1&0\\ 1&2\\ -1&0\\ \endpmatrix$,
which implies that the cokernel is ${\Bbb Z}\oplus{\Bbb Z}_2$, since 
$\pmatrix 1\\ 1\\ -1\\ \endpmatrix ,\pmatrix 0\\ 1\cr 0\\ \endpmatrix , \pmatrix 0\cr 0\\ 1\\ \endpmatrix$
is a basis for ${\Bbb Z}^3$. Finally, $H_0(K)={\Bbb Z}$, since $\del_1,\del_0=0$,
and all higher homology groups are also $0$, for the same reason.

\msk

As another example, the topologist's dunce hat has a $\Delta$-structure with
1 2-simplex, 1 1-simplex, and 1 0-simplex. The boundary maps, we can work out
(starting from $C_2(X)$ ), are $(1),(0)$, and $(0)$, so $H_2(X)=H_1(X)=0$,
and $H_0(X)={\Bbb Z}$. all higher groups are also $0$.

\vfill
\end

