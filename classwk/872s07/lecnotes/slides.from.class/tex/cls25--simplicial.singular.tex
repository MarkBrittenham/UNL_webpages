


\magnification=2000
\overfullrule=0pt
\parindent=0pt

\nopagenumbers

\input amstex

%\voffset=-.6in
%\hoffset=-.5in
%\hsize = 7.5 true in
%\vsize=10.4 true in

\voffset=1.8 true in
\hoffset=-.6 true in
\hsize = 10.2 true in
\vsize=8 true in

\input colordvi



\def\cltr{\Red}		  % Red  VERY-Approx PANTONE RED
\def\cltb{\Blue}		  % Blue  Approximate PANTONE BLUE-072
\def\cltg{\PineGreen}	  % ForestGreen  Approximate PANTONE 349
\def\cltp{\DarkOrchid}	  % DarkOrchid  No PANTONE match
\def\clto{\Orange}	  % Orange  Approximate PANTONE ORANGE-021
\def\cltpk{\CarnationPink}	  % CarnationPink  Approximate PANTONE 218
\def\clts{\Salmon}	  % Salmon  Approximate PANTONE 183
\def\cltbb{\TealBlue}	  % TealBlue  Approximate PANTONE 3145
\def\cltrp{\RoyalPurple}	  % RoyalPurple  Approximate PANTONE 267
\def\cltp{\Purple}	  % Purple  Approximate PANTONE PURPLE

\def\cgy{\GreenYellow}     % GreenYellow  Approximate PANTONE 388
\def\cyy{\Yellow}	  % Yellow  Approximate PANTONE YELLOW
\def\cgo{\Goldenrod}	  % Goldenrod  Approximate PANTONE 109
\def\cda{\Dandelion}	  % Dandelion  Approximate PANTONE 123
\def\capr{\Apricot}	  % Apricot  Approximate PANTONE 1565
\def\cpe{\Peach}		  % Peach  Approximate PANTONE 164
\def\cme{\Melon}		  % Melon  Approximate PANTONE 177
\def\cyo{\YellowOrange}	  % YellowOrange  Approximate PANTONE 130
\def\coo{\Orange}	  % Orange  Approximate PANTONE ORANGE-021
\def\cbo{\BurntOrange}	  % BurntOrange  Approximate PANTONE 388
\def\cbs{\Bittersweet}	  % Bittersweet  Approximate PANTONE 167
%\def\creo{\RedOrange}	  % RedOrange  Approximate PANTONE 179
\def\cma{\Mahogany}	  % Mahogany  Approximate PANTONE 484
\def\cmr{\Maroon}	  % Maroon  Approximate PANTONE 201
\def\cbr{\BrickRed}	  % BrickRed  Approximate PANTONE 1805
\def\crr{\Red}		  % Red  VERY-Approx PANTONE RED
\def\cor{\OrangeRed}	  % OrangeRed  No PANTONE match
\def\paru{\RubineRed}	  % RubineRed  Approximate PANTONE RUBINE-RED
\def\cwi{\WildStrawberry}  % WildStrawberry  Approximate PANTONE 206
\def\csa{\Salmon}	  % Salmon  Approximate PANTONE 183
\def\ccp{\CarnationPink}	  % CarnationPink  Approximate PANTONE 218
\def\cmag{\Magenta}	  % Magenta  Approximate PANTONE PROCESS-MAGENTA
\def\cvr{\VioletRed}	  % VioletRed  Approximate PANTONE 219
\def\parh{\Rhodamine}	  % Rhodamine  Approximate PANTONE RHODAMINE-RED
\def\cmu{\Mulberry}	  % Mulberry  Approximate PANTONE 241
\def\parv{\RedViolet}	  % RedViolet  Approximate PANTONE 234
\def\cfu{\Fuchsia}	  % Fuchsia  Approximate PANTONE 248
\def\cla{\Lavender}	  % Lavender  Approximate PANTONE 223
\def\cth{\Thistle}	  % Thistle  Approximate PANTONE 245
\def\corc{\Orchid}	  % Orchid  Approximate PANTONE 252
\def\cdo{\DarkOrchid}	  % DarkOrchid  No PANTONE match
\def\cpu{\Purple}	  % Purple  Approximate PANTONE PURPLE
\def\cpl{\Plum}		  % Plum  VERY-Approx PANTONE 518
\def\cvi{\Violet}	  % Violet  Approximate PANTONE VIOLET
\def\clrp{\RoyalPurple}	  % RoyalPurple  Approximate PANTONE 267
\def\cbv{\BlueViolet}	  % BlueViolet  Approximate PANTONE 2755
\def\cpe{\Periwinkle}	  % Periwinkle  Approximate PANTONE 2715
\def\ccb{\CadetBlue}	  % CadetBlue  Approximate PANTONE (534+535)/2
\def\cco{\CornflowerBlue}  % CornflowerBlue  Approximate PANTONE 292
\def\cmb{\MidnightBlue}	  % MidnightBlue  Approximate PANTONE 302
\def\cnb{\NavyBlue}	  % NavyBlue  Approximate PANTONE 293
\def\crb{\RoyalBlue}	  % RoyalBlue  No PANTONE match
%\def\cbb{\Blue}		  % Blue  Approximate PANTONE BLUE-072
\def\cce{\Cerulean}	  % Cerulean  Approximate PANTONE 3005
\def\ccy{\Cyan}		  % Cyan  Approximate PANTONE PROCESS-CYAN
\def\cpb{\ProcessBlue}	  % ProcessBlue  Approximate PANTONE PROCESS-BLUE
\def\csb{\SkyBlue}	  % SkyBlue  Approximate PANTONE 2985
\def\ctu{\Turquoise}	  % Turquoise  Approximate PANTONE (312+313)/2
\def\ctb{\TealBlue}	  % TealBlue  Approximate PANTONE 3145
\def\caq{\Aquamarine}	  % Aquamarine  Approximate PANTONE 3135
\def\cbg{\BlueGreen}	  % BlueGreen  Approximate PANTONE 320
\def\cem{\Emerald}	  % Emerald  No PANTONE match
%\def\cjg{\JungleGreen}	  % JungleGreen  Approximate PANTONE 328
\def\csg{\SeaGreen}	  % SeaGreen  Approximate PANTONE 3268
\def\cgg{\Green}	  % Green  VERY-Approx PANTONE GREEN
\def\cfg{\ForestGreen}	  % ForestGreen  Approximate PANTONE 349
\def\cpg{\PineGreen}	  % PineGreen  Approximate PANTONE 323
\def\clg{\LimeGreen}	  % LimeGreen  No PANTONE match
\def\cyg{\YellowGreen}	  % YellowGreen  Approximate PANTONE 375
\def\cspg{\SpringGreen}	  % SpringGreen  Approximate PANTONE 381
\def\cog{\OliveGreen}	  % OliveGreen  Approximate PANTONE 582
\def\pars{\RawSienna}	  % RawSienna  Approximate PANTONE 154
\def\cse{\Sepia}		  % Sepia  Approximate PANTONE 161
\def\cbr{\Brown}		  % Brown  Approximate PANTONE 1615
\def\cta{\Tan}		  % Tan  No PANTONE match
\def\cgr{\Gray}		  % Gray  Approximate PANTONE COOL-GRAY-8
\def\cbl{\Black}		  % Black  Approximate PANTONE PROCESS-BLACK
\def\cwh{\White}		  % White  No PANTONE match


\loadmsbm

\input epsf

\def\ctln{\centerline}
\def\u{\underbar}
\def\ssk{\smallskip}
\def\msk{\medskip}
\def\bsk{\bigskip}
\def\hsk{\hskip.1in}
\def\hhsk{\hskip.2in}
\def\dsl{\displaystyle}
\def\hskp{\hskip1.5in}

\def\lra{$\Leftrightarrow$ }
\def\ra{\rightarrow}
\def\mpto{\logmapsto}
\def\pu{\pi_1}
\def\mpu{$\pi_1$}
\def\sig{\Sigma}
\def\msig{$\Sigma$}
\def\ep{\epsilon}
\def\sset{\subseteq}
\def\del{\partial}
\def\inv{^{-1}}
\def\wtl{\widetilde}
%\def\lra{\Leftrightarrow}
\def\del{\partial}
\def\delp{\partial^\prime}
\def\delpp{\partial^{\prime\prime}}
\def\sgn{{\roman{sgn}}}
\def\wtih{\widetilde{H}}
\def\bbz{{\Bbb Z}}
\def\bbr{{\Bbb R}}
\def\hdsk{\hskip.7in}
\def\hdskb{\hskip.9in}
\def\hdskc{\hskip1.1in}
\def\hdskd{\hskip1.3in}



{\bf Simplicial homology = singular homology:}
We have so far introduced two homologies; simplicial, $H_*^\Delta$, whose computation 
``only'' required some linear algebra,
and singular, $H_*$, which is formally less difficult to work with, and which, you may suspect by now, is also becoming
less difficult to compute. For $\Delta$-complexes, these homology groups are the same, $H_n^\Delta(X)\cong H_n(X)$
for every $X$. In fact, the isomorphism is induced by the inclusion $C_n^\Delta(X)\sset C_n(X)$. We almost have
the tools necessary to prove this; we need to note that most of the edifice we
have built for singular homology \u{could} have been built for simplicial homology, including relative 
homology (for a sub-$\Delta$-complex $A$ of $X$), and a SES of chain groups, giving a LES sequence for the pair,

\ssk

$\cdots \ra H_n^\Delta(A) \ra H_n^\Delta(X) \ra H_n^\Delta(X,A) \ra H_{n-1}^\Delta(A) \ra \cdots$

\ssk

The proof of the isomorphism between the two homologies proceeds by first showing that the
inclusion induces an isomorphism on $k$-skeleta, $H_n^\Delta(X^{(k)})\cong H_n(X^{(k)})$,
by induction on $k$ using the Five Lemma applied to the diagram

\ssk

\ctln{$H_{n+1}^\Delta(X^{(k)},X^{(k-1)}) \ra H_n^\Delta(X^{(k-1)}) \ra 
H_n^\Delta(X^{(k)}) \ra H_{n}^\Delta(X^{(k)},X^{(k-1)}) \ra H_{n-1}^\Delta(X^{(k-1)})$}

\ctln{
$\downarrow$\hdskd $\downarrow$\hdsk $\downarrow$\hdskc $\downarrow$\hdskc $\downarrow$}

\ctln{
$H_{n+1}(X^{(k)},X^{(k-1)})\ \ra H_n(X^{(k-1)}) \ra H_n(X^{(k)}) \ra 
H_{n}(X^{(k)},X^{(k-1)}) \ra H_{n-1}(X^{(k-1)})$}

\ssk

The second and fifth vertical arrows are, by an inductive hypothesis, isomorphisms. 


\vfill
\eject


\ctln{$H_{n+1}^\Delta(X^{(k)},X^{(k-1)}) \ra H_n^\Delta(X^{(k-1)}) \ra 
H_n^\Delta(X^{(k)}) \ra H_{n}^\Delta(X^{(k)},X^{(k-1)}) \ra H_{n-1}^\Delta(X^{(k-1)})$}

\ctln{
$\downarrow$\hdskd $\downarrow$\hdsk $\downarrow$\hdskc $\downarrow$\hdskc $\downarrow$}

\ctln{
$H_{n+1}(X^{(k)},X^{(k-1)})\ \ra H_n(X^{(k-1)}) \ra H_n(X^{(k)}) \ra 
H_{n}(X^{(k)},X^{(k-1)}) \ra H_{n-1}(X^{(k-1)})$}

\ssk

The first and 
fourth vertical arrows are
isomorphisms because, essentially, we can, in each case, identify these groups. 
$H_{n}(X^{(k)},X^{(k-1)})\cong H_{n}(X^{(k)}/X^{(k-1)})\cong \widetilde{H}_n(\vee S^k)$
are either 0 (for $n\neq k$) or $\oplus \bbz$ (for $n=k$), one summand for each $n$-simplex in $X$. 
But the same is true for $H_{n}^\Delta(X^{(k)},X^{(k-1)})$ (the chain groupsare $0$
for $n\neq k$); and for $n=k$ the generators are precisely
the $n$-simplices of $X$. The inclusion-induced map takes generators to generators, so is an isomorphism.
\hhsk So by the Five Lemma, the middle rows are also isomorphisms, completing our inductive proof.

\ssk

Returning to $H_n^\Delta(X) {\buildrel {I_*}\over \ra} H_n(X)$, we show that this map is an isomorphism.
Any $[z]\in H_n(X)$ is rep'd by a cycle $z=\sum a_i\sigma_i$ for $\sigma_i:\Delta^n\ra X$ . But each image
$\sigma_i(\Delta^n)$ is compact, and so meets only finitely-many cells of $X$. 
So there is a $k$ for which all of the simplices map into $X^{(k)}$, and so we may
treat $z\in C_n(X^{(k)}$. Viewed this way, it is still a cycle, and so $[z]\in H_n(X^{(k)})\cong H_n^\Delta(X^{(k)})$
so there is a $z^\prime \in C_n^\Delta(X^{(k)})$ and a $w\in C_{n+1}(X^{(k)})$ with $i_\#z^\prime -z=\del w$. 
But thinking of  $z^\prime \in C_n^\Delta(X)$ and $w\in C_{n+1}(X)$, we have the same equality, so 
$[z^\prime] \in H_n^\Delta(X)$ and $i_*[z^\prime] = [z]$ . So $i_*$ is surjective.
If $i_*([z]) = 0$, then the cycle $z=\sum a_i\sigma_i$ is a sum of characteristic maps of $n$-simplices of $X$, and
so can be thought of as an element of $C_n^\Delta(X^{(n)})$ . Being $0$ in $H_n(X)$, $z=\del w$ for some
$w\in C_{n+1}(X)$ . But as before, $w\in C_n(X^{(r)})$ for some $r$, and so thought of as an element of 
the image of the isomorphism $i_*: H_n^\Delta(X^{(r)})\ra H_n(X^{(r)})$, $i_*([z])=0$, so $[z]=0$ . So 
$z=\del u$ for some $u\in C_{n+1}^\Delta(X^{(r)})\sset C_{n+1}^\Delta(X)$ . So $[z]=0$ in $H_n^\Delta(X)$.
Consequently, simplicial and singular homology groups are isomorphic.

\bsk






The isomorphism between simplicial and singular homology provides very quick proofs
of several results about singular homology, which would other would require some effort:

\ssk

{\it If the $\Delta$-complex $X$ has no simplices in dimension greater than $n$, then 
$H_i(X)=0$ for all $i>n$.}

\ssk

This is because the simplicial chain groups $C_i^\Delta(X)$ are $0$, so $H_i^\Delta(X)=0$ .

\ssk

{\it If for each $n$, the $\Delta$-complex $X$ has finitely many $n$-simplices, then 
$H_n(X)$ is finitely generated for every $n$.}

\ssk

This is because the simplicial chain groups $C_n^\Delta(X)$ are all finitely generated,
so $H_n^\Delta(X)$, being a quotient of a subgroup, is also finitely generated. [We
are using here that the number of generators of a subgroup $H$ of an {\it abelian} 
group $G$ is no larger than that for $G$; this is not true for groups in general!]

\msk

A quick Mayer-Vietoris computation allows us to compute the homology groups of surfaces:
$\Sigma_g$ = a 2-disk $D$ glued to a bouquet $X$ of $2g$ circles, with ``intersection'' a circle, so
we have

\ssk

$\wtih_2(X)\oplus \wtih_2(D)\ra \wtih_2(\Sigma_g)\ra
\wtih_1(S^1)\ra \wtih_1(X)\oplus \wtih_1(D)\ra \wtih_1(\Sigma_g)\ra \wtih_0(S^1)$

i.e., 
$0\oplus 0\ra \wtih_2(\Sigma_g) \buildrel{\del}\over\ra \bbz \ra \bbz^{2g}\oplus 0 \ra \wtih_1(\Sigma_g)
\ra 0$

\ssk

But the map $\bbz \ra \bbz^{2g}$ is $0$; the generator of $H_1(S^1)$ is taken to the sum of the edges
in the identifcation map, which cancel in pairs. So we really have the SES's

$0\ra \wtih_2(\Sigma_g) \buildrel{\del}\over\ra \bbz\ra 0$ and 
$0\ra \bbz^{2g} \ra \wtih_1(\Sigma_g) \ra 0$, so $\wtih_2(\Sigma_g)=\bbz$ and 

$\wtih_1(\Sigma_g)=\bbz^{2g}$; all others are $0$ by dimension and connectedness considerations.


\vfill
\eject

Some more topological results with homological proofs: The Klein bottle and real projective plane cannot 
embed in $\bbr^3$. This is because a surface $\Sigma$ embedded in $\bbr^3$ has a (the proper word is {\it normal})
neighborhood $N(\Sigma)$, which deformation retracts to $\Sigma$; literally, it is all points within a (uniformly) short distance
in the normal direction from the point on the surface $\Sigma$. Our non-embeddedness result follows (by contradiction)
from applying Mayer-Vietoris to the pair $(A,B) = (\overline{N(\Sigma)},\overline{\bbr^3\setminus N(\Sigma)})$, whose intersection
is the boundary $F=\del N(\Sigma)$ of the normal neighborhood. The point, though, is that
$F$ is an orientable surface; the outward normal (pointing away from $N(\Sigma)$) at every point, taken as
the first vector of a right-handed orientation of $\bbr^3$ allows us to use the other two vectors as an 
orientation of the surface. So $F$ is one of the surface $F_g$ above whose homologies we just computed.
This gives the LES
\hhsk
$\wtih_2(\bbr^3) \ra \wtih_1(F) \ra \wtih_1(A)\oplus \wtih_1(B)\ra \wtih_1(\bbr ^3)$
\hhsk 
which renders as 
\hhsk
$0\ra\bbz^{2g}\ra \wtih_1(\Sigma)\oplus G\ra 0$
\hhsk , i.e., \hhsk
$\bbz^{2g}\cong \wtih_1(\Sigma)\oplus G$ 
\hhsk . But for the Klein bottle and projective plane (or any closed, non-orientable
surface for that matter), $\wtih_1(\Sigma)$ has torsion, so it cannot be the direct
summand of a torsion-free group! So no such embedding exists. This result holds
more generally for any 2-complex $K$ whose (it turns out it would have to be first)
homology has torsion; any embedding into $\bbr^3$ would have a neighborhood 
deformation retracting to $K$, with boundary a (for the exact same reasons as above)
closed orientable surface.

\vfill
\end



