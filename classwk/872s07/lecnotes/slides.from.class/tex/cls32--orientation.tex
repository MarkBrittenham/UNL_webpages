







\magnification=2000
\overfullrule=0pt
\parindent=0pt

\nopagenumbers

\input amstex

%\voffset=-.6in
%\hoffset=-.5in
%\hsize = 7.5 true in
%\vsize=10.4 true in

\voffset=1.8 true in
\hoffset=-.6 true in
\hsize = 10.2 true in
\vsize=8 true in

\input colordvi



\def\cltr{\Red}		  % Red  VERY-Approx PANTONE RED
\def\cltb{\Blue}		  % Blue  Approximate PANTONE BLUE-072
\def\cltg{\PineGreen}	  % ForestGreen  Approximate PANTONE 349
\def\cltp{\DarkOrchid}	  % DarkOrchid  No PANTONE match
\def\clto{\Orange}	  % Orange  Approximate PANTONE ORANGE-021
\def\cltpk{\CarnationPink}	  % CarnationPink  Approximate PANTONE 218
\def\clts{\Salmon}	  % Salmon  Approximate PANTONE 183
\def\cltbb{\TealBlue}	  % TealBlue  Approximate PANTONE 3145
\def\cltrp{\RoyalPurple}	  % RoyalPurple  Approximate PANTONE 267
\def\cltp{\Purple}	  % Purple  Approximate PANTONE PURPLE

\def\cgy{\GreenYellow}     % GreenYellow  Approximate PANTONE 388
\def\cyy{\Yellow}	  % Yellow  Approximate PANTONE YELLOW
\def\cgo{\Goldenrod}	  % Goldenrod  Approximate PANTONE 109
\def\cda{\Dandelion}	  % Dandelion  Approximate PANTONE 123
\def\capr{\Apricot}	  % Apricot  Approximate PANTONE 1565
\def\cpe{\Peach}		  % Peach  Approximate PANTONE 164
\def\cme{\Melon}		  % Melon  Approximate PANTONE 177
\def\cyo{\YellowOrange}	  % YellowOrange  Approximate PANTONE 130
\def\coo{\Orange}	  % Orange  Approximate PANTONE ORANGE-021
\def\cbo{\BurntOrange}	  % BurntOrange  Approximate PANTONE 388
\def\cbs{\Bittersweet}	  % Bittersweet  Approximate PANTONE 167
%\def\creo{\RedOrange}	  % RedOrange  Approximate PANTONE 179
\def\cma{\Mahogany}	  % Mahogany  Approximate PANTONE 484
\def\cmr{\Maroon}	  % Maroon  Approximate PANTONE 201
\def\cbr{\BrickRed}	  % BrickRed  Approximate PANTONE 1805
\def\crr{\Red}		  % Red  VERY-Approx PANTONE RED
\def\cor{\OrangeRed}	  % OrangeRed  No PANTONE match
\def\paru{\RubineRed}	  % RubineRed  Approximate PANTONE RUBINE-RED
\def\cwi{\WildStrawberry}  % WildStrawberry  Approximate PANTONE 206
\def\csa{\Salmon}	  % Salmon  Approximate PANTONE 183
\def\ccp{\CarnationPink}	  % CarnationPink  Approximate PANTONE 218
\def\cmag{\Magenta}	  % Magenta  Approximate PANTONE PROCESS-MAGENTA
\def\cvr{\VioletRed}	  % VioletRed  Approximate PANTONE 219
\def\parh{\Rhodamine}	  % Rhodamine  Approximate PANTONE RHODAMINE-RED
\def\cmu{\Mulberry}	  % Mulberry  Approximate PANTONE 241
\def\parv{\RedViolet}	  % RedViolet  Approximate PANTONE 234
\def\cfu{\Fuchsia}	  % Fuchsia  Approximate PANTONE 248
\def\cla{\Lavender}	  % Lavender  Approximate PANTONE 223
\def\cth{\Thistle}	  % Thistle  Approximate PANTONE 245
\def\corc{\Orchid}	  % Orchid  Approximate PANTONE 252
\def\cdo{\DarkOrchid}	  % DarkOrchid  No PANTONE match
\def\cpu{\Purple}	  % Purple  Approximate PANTONE PURPLE
\def\cpl{\Plum}		  % Plum  VERY-Approx PANTONE 518
\def\cvi{\Violet}	  % Violet  Approximate PANTONE VIOLET
\def\clrp{\RoyalPurple}	  % RoyalPurple  Approximate PANTONE 267
\def\cbv{\BlueViolet}	  % BlueViolet  Approximate PANTONE 2755
\def\cpe{\Periwinkle}	  % Periwinkle  Approximate PANTONE 2715
\def\ccb{\CadetBlue}	  % CadetBlue  Approximate PANTONE (534+535)/2
\def\cco{\CornflowerBlue}  % CornflowerBlue  Approximate PANTONE 292
\def\cmb{\MidnightBlue}	  % MidnightBlue  Approximate PANTONE 302
\def\cnb{\NavyBlue}	  % NavyBlue  Approximate PANTONE 293
\def\crb{\RoyalBlue}	  % RoyalBlue  No PANTONE match
%\def\cbb{\Blue}		  % Blue  Approximate PANTONE BLUE-072
\def\cce{\Cerulean}	  % Cerulean  Approximate PANTONE 3005
\def\ccy{\Cyan}		  % Cyan  Approximate PANTONE PROCESS-CYAN
\def\cpb{\ProcessBlue}	  % ProcessBlue  Approximate PANTONE PROCESS-BLUE
\def\csb{\SkyBlue}	  % SkyBlue  Approximate PANTONE 2985
\def\ctu{\Turquoise}	  % Turquoise  Approximate PANTONE (312+313)/2
\def\ctb{\TealBlue}	  % TealBlue  Approximate PANTONE 3145
\def\caq{\Aquamarine}	  % Aquamarine  Approximate PANTONE 3135
\def\cbg{\BlueGreen}	  % BlueGreen  Approximate PANTONE 320
\def\cem{\Emerald}	  % Emerald  No PANTONE match
%\def\cjg{\JungleGreen}	  % JungleGreen  Approximate PANTONE 328
\def\csg{\SeaGreen}	  % SeaGreen  Approximate PANTONE 3268
\def\cgg{\Green}	  % Green  VERY-Approx PANTONE GREEN
\def\cfg{\ForestGreen}	  % ForestGreen  Approximate PANTONE 349
\def\cpg{\PineGreen}	  % PineGreen  Approximate PANTONE 323
\def\clg{\LimeGreen}	  % LimeGreen  No PANTONE match
\def\cyg{\YellowGreen}	  % YellowGreen  Approximate PANTONE 375
\def\cspg{\SpringGreen}	  % SpringGreen  Approximate PANTONE 381
\def\cog{\OliveGreen}	  % OliveGreen  Approximate PANTONE 582
\def\pars{\RawSienna}	  % RawSienna  Approximate PANTONE 154
\def\cse{\Sepia}		  % Sepia  Approximate PANTONE 161
\def\cbr{\Brown}		  % Brown  Approximate PANTONE 1615
\def\cta{\Tan}		  % Tan  No PANTONE match
\def\cgr{\Gray}		  % Gray  Approximate PANTONE COOL-GRAY-8
\def\cbl{\Black}		  % Black  Approximate PANTONE PROCESS-BLACK
\def\cwh{\White}		  % White  No PANTONE match


\loadmsbm

\input epsf

\def\ctln{\centerline}
\def\u{\underbar}
\def\ssk{\smallskip}
\def\msk{\medskip}
\def\bsk{\bigskip}
\def\hsk{\hskip.1in}
\def\hhsk{\hskip.2in}
\def\dsl{\displaystyle}
\def\hskp{\hskip1.5in}

\def\lra{$\Leftrightarrow$ }
\def\ra{\rightarrow}
\def\mpto{\logmapsto}
\def\pu{\pi_1}
\def\mpu{$\pi_1$}
\def\sig{\Sigma}
\def\msig{$\Sigma$}
\def\ep{\epsilon}
\def\sset{\subseteq}
\def\del{\partial}
\def\inv{^{-1}}
\def\wtl{\widetilde}
%\def\lra{\Leftrightarrow}
\def\del{\partial}
\def\delp{\partial^\prime}
\def\delpp{\partial^{\prime\prime}}
\def\sgn{{\roman{sgn}}}
\def\wtih{\widetilde{H}}
\def\bbz{{\Bbb Z}}
\def\bbr{{\Bbb R}}
\def\bbq{{\Bbb Q}}
\def\bbc{{\Bbb C}}
\def\hdsk{\hskip.7in}
\def\hdskb{\hskip.9in}
\def\hdskc{\hskip1.1in}
\def\hdskd{\hskip1.3in}
\def\Hom{\text{Hom}}
\def\Ext{\text{Ext}}
\def\larr{\leftarrow}


{\bf Orientations:} The cap product plays an especially important role in connecting the homology and cohomology groups of 
{\it orientable manifolds}.
An {\it $n$-manifold} $M$ is a $2^{\text{nd}}$ countable, Hausdorff space with the additional property that every point $x\in M$ has an
open neighborhood $\Cal U$ homeomorphic to $\bbr^n$. [Note: by this definition, $D^n$ is not a manifold! It is instead a
``manifold with boundary'', which have their own parallel and very similar theory.] 
%Invariance of Domain $\Rightarrow$
%the $n$ that works for a point is unique; open-and-closed argument 
%$\Rightarrow$ in a connected space each pt uses the same $n$.
Using excision
we find that for any $n$-manifold $M$ and $x\in M$, 

\ssk

$H_k(M,M\setminus x;R)\cong H_k({\Cal U},{\Cal U}\setminus x;R)\cong H_k(\bbr^n,\bbr^n\setminus 0)$

\hfill $\cong H_k(D^n,D^n\setminus 0;R)\cong H_k(D^n,\del D^n;R)\cong \wtih_k(D^n/\del D^n;R)\cong \wtih_k(S^n;R)$

\ssk

and so equals $0$ for $k\neq n$ and $R$ for $k=n$. An $R$-{\it orientation} on an $n$-manifold $M$ is a choice of generator
(as an $R$-module) $r_x\in H_n(M,M\setminus x;R)\cong R$ (called a {\it local} $R$-{\it orientation} at $x$) for every 
$x\in M$, which are locally compatible: 
\cltr{for every $x\in M$ there is a nbhd ${\Cal U}$ of $x$ and $r_{\Cal U}\in H_n(M,M\setminus {\Cal U};R)$ such that,
for every $y\in {\Cal U}$, the inclusion-induced map
$\iota_*:H_n(M,M\setminus {\Cal U};R)\ra H_n(M,M\setminus y;R)$ sends $r_{\Cal U}$ to $r_y$.}

\msk

For example, every manifold is $\bbz_2$-orientable; a local $\bbz_2$-orientation is a choice of the (one and only)
non-zero element of $H_n(M,M\setminus x;\bbz_2)$, and for any pair of locally Euclidean nbhds 
${\Cal U}\subseteq \overline{\Cal U}\subseteq {\Cal V}$ of $x$, and $y\in {\Cal U}$,

\ssk

\ctln{$H_n(M,M\setminus{\Cal U};\bbz_2)\cong H_n({\Cal V},{\Cal V}\setminus{\Cal U};\bbz_2)\cong 
H_n({\Cal V},{\Cal V}\setminus y;\bbz_2)\cong H_n(M,M\setminus y;\bbz_2)\cong\bbz_2$}

\ssk

where all isomorphisms are inclusion-induced, and the second assumes that ${\Cal U}$ is a ball in ${\Cal V}$, so that
$\overline{\Cal U}\setminus y$ deformation retracts to $\del\overline{\Cal U}\subseteq {\Cal V}$, so 
${\Cal V}\setminus y$ deformation retracts to ${\Cal V}\setminus{\Cal U}$. Consequently, the inclusion-induced 
isomorphism $H_n(M,M\setminus{\Cal U};\bbz_2)\cong H_n(M,M\setminus y;\bbz_2)$ sends the non-zero element of
$H_n(M,M\setminus{\Cal U};\bbz_2)$, which we \u{define} to be $r_{\Cal U}$, to the non-zero element of 
$H_n(M,M\setminus y;\bbz_2)$, which is $r_y$.

\vfill
\eject

As an example, the closed orientable surfaces $F_g$ of genus $g$ are $R$-orientable for every $R$ (hence the name...);
from the LES for a pair we have

\ssk

\ctln{$\cdots\ra H_2(F_g\setminus x;R)\ra H_2(F_g;R)\ra H_2(F_g,F_g\setminus x;R)
\ra H_1(F_g\setminus x;R)\ra H_1(F_g;R)\ra\cdots$}

\ssk

which, via universal coefficients and because $F_g\setminus x$ deformation retracts to the 1-skeleton, is

\ctln{$0\ra R\buildrel{i_*}\over\ra R\ra R^{2g}\buildrel{j_*}\over\ra R^{2g}$ .}

\ssk

But $j_*$ is an isomorphism (the 2-cell has boundary 0, so there ``are'' no 1-boundaries),
therefore by exactness so is $i_*$, so the image of a generator of $H_2(F_g)$ defines a (compatible: 
the open cover \u{is} $F_g$) set of local orientations at each point.

\msk

The first basic fact about orientations is that what just happened is not an accident; 
if $M^n$ is $R$-orientable and {\it compact}, then there is a 
(unique!) homology class
$[M]\in H_n(M;R)=H_n(M,\emptyset;R)$, the {\it orientation class} of $M$, such that the image of $[M]$ in 
$H_n(M,M\setminus x;R)$ defines the same orientation on $M$ (i.e., it equals $r_x$ for every $x$). 

\ssk

To prove this, start with ball neighborhoods ${\Cal U}_x$ of each point as an 
open cover, and take smaller ball neighborhoods ${\Cal V}_x\subseteq \overline{{\Cal V}_x}\subseteq {\Cal U}_x$,
with compact closures (e.g., the inverse image of the unit ball under a homeo $h:{\Cal U}_x\ra \bbr^n$)
as another open cover. By compactness, finitely many $\{{\Cal V}_i\}_{i=1}^m$ of the ${\Cal V}_x$ cover $M$. 
For notational sanity, let us write $H_n(M|A)$ for $H_n(M,M\setminus A;R)$.
The isomorphisms $H_n(M|{\Cal U_i})\cong H_n(M|\overline{\Cal V_i})\cong H_n(M,|y)\cong R$
(since each subspace deformation retracts to the next smaller one) implies that there is a unique class
$r_i\in H_n(M|\overline{\Cal V_i})$ (the image of the class $r_{{\Cal U}_i}\in H_n(M|{\Cal U_i})$)
which maps to each $r_y$ under inclusion (unique because we have isos).

\ssk

For notational convenience, we set $K_i=\overline{{\Cal V}_i}$. 
We wish to show that there is a unique class $[M]$ in 
$H_n(M;R) = H_n(M|M)$ which restricts to each of the $r_i$; further restriction then implies that 
it maps to each $r_x$, as desired. First we prove uniqueness. Suppose there were two 
classes $u,v$ restricting to each of the $r_i$. Then their difference, $u-v=w$, restricts to $0$ in 
every $H_n(M|K_i)$. We show by induction that then $w$ restricts to $0$ in the groups
$G_j=H_n(M|K_1\cup\cdots\cup K_j)$. But since $G_m=H_n(M|\bigcup_i K_i)=H_n(M|M)=H_n(M;R)$,
$w=0$ as desired. For the inductive step, we use the relative Mayer-Vietoris sequence

\ssk

$\cdots\ra H_{n+1}(M|(K_1\cup\cdots\cup K_i)\cap K_{i+1})\ra
H_{n}(M|(K_1\cup\cdots\cup K_i)\cup K_{i+1}))\ra$

\hskip.3in $H_{n}(M|K_1\cup\cdots\cup K_i)\oplus H_n(M|K_{i+1})\ra
H_{n}(M|(K_1\cup\cdots\cup K_i)\cap K_{i+1})\ra\cdots$

\ssk

A separate induction (which we skip) establishes that 
$H_{n+1}(M|K_1\cup\cdots\cup K_i)\cap K_{i+1})=H_{n+1}(M|(K_1\cap K_{i+1})\cup\cdots\cup (K_i\cap K_{i+1})) = 0$.
So $H_{n}(M|(K_1\cup\cdots\cup K_i)\cup K_{i+1}))$ injecting into $H_{n}(M|K_1\cup\cdots\cup K_i)\oplus H_n(M|K_{i+1})$.
But the image of $w$ in $H_{n}(M|(K_1\cup\cdots\cup K_i)\cup K_{i+1}))$ is then carried to
$0$ in both $H_{n}(M|K_1\cup\cdots\cup K_i)$ and $H_n(M|K_{i+1})$ by the inductive hypothesis
(and initial step), so by injectivity is itself $0$, establishing the inductive step.

\vfill
\eject

From uniqueness, we can go on to establish existence, again by induction. Note that the uniqueness
argument above applies more generally; for any compact set $K\subseteq M$ there is \u{at} \u{most} \u{one} class
$r_K\in H_n(M|K)$ which restricts to $r_x\in H_n(M|x)$ for every $x\in K$. This essentially allows us
to stitch together the classes which compatibility guarantees exist for small $K$ to ever larger $K$.
Formally, we just parallel the argument above; given $M=K_1\cup\cdots\cup K_m$, we have classes $r_i\in H_n(M|K_i)$
which restrict to the local orientations. The point is that in the relative Mayer-Vietoris sequence

\ssk

\ctln{$H_n(M|K_{m-1}\cup K_m)\ra H_n(M|K_{m-1})\oplus H_n(M|K_m)\ra H_n(M|K_{m-1}\cap K_m)$}

\ssk

the classes $r_{m-1},r_m$ each map to a class in $H_n(M|K_{m-1}\cap K_m)$ (under the inclusion-induced maps)
which restricts to $r_x$ for every $x\in K_{m-1}\cap K_m$, and so by uniqueness map to the \u{same} class.
So in the Mayer-Vietoris sequence, $(r_{m-1},r_m)$ maps to $0$, and so is in the image of $H_n(M|K_{m-1}\cup K_m)$,
so there is a class $r_{m-1}^\prime\in H_n(M|K_{m-1}\cup K_m)$ which restricts to $r-x$ for every 
$x\in K_{m-1}\cup K_m$. Now replace $K_{m-1},K_m$ by $K_{m-1}\cup K_m$ in our cover of $M$ by compact sets,
and continue (or declare victory!) by induction, since we have a cover by fewer sets having the hypothesized
classes $r_i$; once we reach one such set, we have $K_1=M$.

\vfill
\eject

Given a compact, connected $R$-orientable $n$-manifold, we now have an orientation class $[M]\in H_n(M;R)$ which defines
an orientation on $M$. This class plays a central role in 

\ssk

\cltr{{\it Poincar\'e Duality}: If $M$ is a compact, connected, $R$-orientable $n$-manifold, then for every $k$, the map
$P[\varphi]=[M]\frown [\varphi]$ , $P:H^k(M;R)\ra H_{n-k}(M;R)$ is an isomorphism.}

\ssk

We will not prove this; the proof is in many respects parallel to the one given above, inducting on a
number of ``small'' compact subsets whose union is $M$, but it formally requires introducing a new 
cohomology theory, {\it cohomology with compact supports}, which we will not take the time to explore.
Instead, we will outline some of the consequences of this result.

\msk

For a connected $R$-orientable compact $n$-manifold $M$, 
$H_n(M;R)\cong H^0(M;R)\cong\Hom(H_0(M),R)\oplus\Ext(H_{-1}(M),R)\cong
\Hom(\bbz,R)\oplus\Ext(0,R)\cong R$.
[Note that this
immediately implies that $\bbr P^{2k}$ is not orientable, since $H_{2k}(\bbr P^{2k} = 0$.]
This is really a kind of cheap consequence, though, because in proving Poincar\'e duality, 
you basically have to prove this first...

\ssk

For a connected $\bbz$-orientable compact manifold $M$,
$H_{n-1}(M;\bbz)\cong H^1(M;\bbz)\cong \Hom(H_1(M),\bbz)\oplus \Ext(H_0(M),\bbz)
\cong \Hom(H_1(M),\bbz)\oplus \Ext(\bbz,\bbz)=\Hom(H_1(M),\bbz)$ = the torsion-free
part of $H_1(M)$, so $H_{n-1}M)$ is, in particular, torsion-free. [Note that this also
immediately implies that $\bbr P^{2k}$ is not orientable, since $H_{2k-1}(\bbr P^{2k}\cong \bbz_2$.]

\vfill
\eject

We've seen that the Euler characteristic can be computed using a singular chain complex with coefficients
in any (non-trivial!) field, for finite CW-complexes. Using the (somewhat hard) fact that a compact manifold
admits the structure of a finite CW-complex, we can then use this to learn a few things about the Euler 
characteristics of compact manifolds. Since Poincare duality holds for any compact manifold $M$ when using
$\bbz_2$ coefficients, we have

\ssk

\ctln{$H_{n-k}(M;\bbz_2) \cong H^k(M;\bbz_2)\cong \Hom(H_k(M),\bbz_2)\oplus \Ext(H_{k-1}(M),\bbz_2)$}

\ssk

But we have seen that $H_k(M;\bbz_2)\cong H_k(M)/\{2[z]: [z]\in H_k(M)\}\oplus\{[z]\in H_{k-1}(M) : 2[z]=0\}$,
and a quick bit of algebra will show that, at least for finitely generated abelian groups $G$,
$G/\{2z: z\in G\}\cong\Hom(G,\bbz_2)$ and $\{z\in G : 2z=0\}\cong\Ext(G,\bbz_2)$ (since everybody behaves
well under direct sums and these isomorphisms hold for $\bbz$ and $\bbz_n$). So 
$H_{n-k}(M;\bbz_2) \cong H_k(M;\bbz_2)$. Consequently, when computing Euler characteristic, for an 
odd-dimensional manifold, we find that (using dimension as a $\bbz_2$-vector space)

\ssk

$\chi(M)=\sum(-1)^i\dim(H_i(M;\bbz_2))$

\hskip.4in $=\sum_{i<n/2}(-1)^i\dim(H_i(M;\bbz_2))+\sum_{i>n/2}(-1)^i\dim(H_i(M;\bbz_2))$

\hskip.4in $=\sum_{i<n/2}(-1)^i\dim(H_i(M;\bbz_2))+\sum_{i>n/2}(-1)^{-i}\dim(H_{n-i}(M;\bbz_2))$

\hskip.4in $=\sum_{i<n/2}(-1)^i\dim(H_i(M;\bbz_2))-\sum_{i>n/2}(-1)^{n-i}\dim(H_{n-i}(M;\bbz_2))$

\hskip.4in $=\sum_{i<n/2}(-1)^i\dim(H_i(M;\bbz_2))-\sum_{j<n/2}(-1)^{j}\dim(H_{j}(M;\bbz_2))$

\hskip.4in $=0$ .

\vfill
\eject

So odd-dimensional compact manifolds (without boundary) $M$ all have Euler characteristic $0$.
$\chi(M)=0$ then holds no matter \u{how} we (correctly) compute it, so this tells us something about the ranks
of the $\bbz$-homology groups, as well as about the CW-structure itself on $M$. 




\vfill
\end

