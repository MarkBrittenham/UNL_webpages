
\magnification=2000
\overfullrule=0pt
\parindent=0pt

\nopagenumbers

\input amstex

%\voffset=-.6in
%\hoffset=-.5in
%\hsize = 7.5 true in
%\vsize=10.4 true in

\voffset=1.8 true in
\hoffset=-.6 true in
\hsize = 10.2 true in
\vsize=8 true in

\input colordvi

\def\cltr{\Red}		  % Red  VERY-Approx PANTONE RED

\loadmsbm

\input epsf

\def\ctln{\centerline}
\def\u{\underbar}
\def\ssk{\smallskip}
\def\msk{\medskip}
\def\bsk{\bigskip}
\def\hsk{\hskip.1in}
\def\hhsk{\hskip.2in}
\def\dsl{\displaystyle}
\def\hskp{\hskip1.5in}

\def\lra{$\Leftrightarrow$ }
\def\ra{\rightarrow}
\def\mpto{\logmapsto}
\def\pu{\pi_1}
\def\mpu{$\pi_1$}
\def\sig{\Sigma}
\def\msig{$\Sigma$}
\def\ep{\epsilon}
\def\sset{\subseteq}
\def\del{\partial}
\def\inv{^{-1}}
\def\wtl{\widetilde}
%\def\lra{\Leftrightarrow}
\def\del{\partial}
\def\delp{\partial^\prime}
\def\delpp{\partial^{\prime\prime}}
\def\sgn{{\roman{sgn}}}
\def\wtih{\widetilde{H}}
\def\bbz{{\Bbb Z}}
\def\bbr{{\Bbb R}}
\def\rtar{$\Rightarrow$}

\def\cltr{\Red}		  % Red  VERY-Approx PANTONE RED
\def\cltb{\Blue}		  % Blue  Approximate PANTONE BLUE-072
\def\cltg{\PineGreen}	  % ForestGreen  Approximate PANTONE 349

{\bf Group presentations:}

\msk

$\pi_1(S^1)\cong \bbz$ and $\pi_1(D^2)\cong \{1\}$. Spheres and disks form the basic building 
blocks for many important spaces. The Seifert-van Kampen Theorem describes how to \cltr{construct
$\pi_1(X)$ for $X=A\cup B$ from $\pi_1(A)$, $\pi_1(B)$, and $\pi_1(A\cap B)$}. This allows
us to compute the fundamental groups of increasingly sophisticated spaces. To do so, we
need the language of presentations.

\msk

The basic idea: a group $G$ is \cltr{{\it generated by a set $\Sigma\subseteq G$}} if every element of $G$
can be expressed as a finite product of elements of $\Sigma$ (and their inverses).
A product of elements of $\Sigma^{\pm 1}$ which equals $1$ in the group is a {\it relator};
a set of relators $R$ form a {\it presentation} $\langle \Sigma | R\rangle$ for $G$ if
all of the relators in $G$ are ``consequences'' of those in $R$. 

\vfill
\eject

{\bf Free groups:} $\Sigma$ = a set; a {\it reduced word} on \msig\ is a (formal)
product $a_1^{\ep_1}\cdots a_n^{\ep_n}$ with $a_i\in\sig$ and $\ep_i=\pm 1$,
and either $a_i\neq a_{i+1}$ or $\ep_i\neq -\ep_{i+1}$ for every $i$. (I.e., no
$aa^{-1},a^{-1}a$ in the product.)

\ssk

\cltr{The free group $F(\sig)$ = the set of reduced words, with multiplication = concatenation 
followed by {\it reduction}: remove all occurances of $aa^{-1},a^{-1}a$ .}

\ssk

identity element = the empty word, 
$(a_1^{\ep_1}\cdots a_n^{\ep_n})^{-1} = a_n^{-\ep_n}\cdots a_1^{-\ep_1}$. 
$F(\sig)$ is generated by \msig, with no relations among the generators
other than the ``obvious'' ones.

\ssk

Important property of free groups: any function $f:\sig\ra G$ , $G$ a group, extends
uniquely to a homomorphism $\phi: F(\sig)\ra G$.

\msk

If $R\sset F(\sig)$, then $<R>^N$ = normal subgroup generated by $R$ =

\cltr{$\displaystyle \{\prod_{i=1}^n g_i r_i g_i^{-1} : n\in{\Bbb N}_0 , g_i\in F(\sig) , r_i\in R\}$}
= smallest normal subgroup containing $R$.

\ssk

$F(\sig)/<R>^N$ = the group with {\it presentation} $<\sig | R>$ ; it is the largest quotient
of $F(\sig)$ in which the elements of $R$ are the identity. 

\msk

Every group has a presentation: \hskip.1in
\cltr{$G \cong F(G)/<gh(gh)^{-1} : g,h\in G>^N$}

where $(gh)$ is interpreted as a single letter in $G$.

\vfill
\eject

If $G_1=<\sig_1 | R_1>$ and $G_2=<\sig_2 | R_2>$, then their {\it free product}
$G_1*G_2 = <\sig_1\coprod\sig_2 | R_1\cup R_2>$ ($\sig_1,\sig_2$ must be 
treated as (formally) disjoint). Each $g\in G_1*G_2$ is
$g=g_1\cdots g_n$ where the $g_i$ alternate from $G_1,G_2$ (uniquely).
$G_1,G_2$ are subgroups of $G_1*G_2$ in the obvious way.
Important property of free products: any pair of
homoms $\phi_i:G_i\ra G$ extends uniquely to a homom $\phi:G_1*G_2\ra G$ .

\ssk

{\bf Gluing groups:} given groups $G_1,G_2$, with subgroups $H_1,H_2$ that are
isomorphic $H_1\cong H_2$, we can ``glue'' $G_1$ and $G_2$ together along their
``common'' subgroup. More generally,
given a group $H$ and homomorphisms $\phi_1 : H\ra G_i$, we can build the largest group ``generated'' by
$G_1$ and $G_2$, in which $\phi_1(h)=\phi_2(h)$ for all $h\in H$. 

\msk

Starting
with $G_1*G_2$ (to get the first part), we then take a quotient to insure that 
$\phi_1(h)(\phi_2(h))^{-1} =1$ for every $h$. Using presentations 
$G_1=<\sig_1 | R_1>$ , $G_2=<\sig_2 | R_2>$ , quotienting out by 
as little as possible, we have

\ssk

\cltr{$G = (G_1*G_2)/<\phi_1(h)(\phi_2(h))^{-1} : h\in H>^N =$}

\hfill \cltr{$<\sig_1\coprod\sig_2 | R_1\cup R_2\cup\{\phi_1(h)(\phi_2(h))^{-1} : h\in H\}>$}
 
\msk

$G= =G_1*_HG_2$ is the {\it largest} group generated by $G_1$ and $G_2$ in which 
$\phi_1(h)=\phi_2(h)$ for all $h\in H$, and is called the {\it amalgamated 
free product} or {\it free product with amalgamation (over $H$)} or {\it pushout over $H$}. 

\msk

Important special cases : $G*_H\{1\} = G/<\phi(H)>^N = <\sig | R\cup \phi(H)>$ , and
$G_1*_{\{1\}}G_2 \cong G_1*G_2$

\bsk


\vfill
\end

[{\bf Warning!} 
Group theorists will generally use this term only if both homoms $\phi_1,\phi_2$
are injective. (This insures that the natural maps of $G_1,G_2$ into $G_1*_HG_2$
are injective.) But we will use this term for all $\phi_1,\phi_2$. (Some people use the term {\it pushout}
in this more general case.)]

