

\magnification=2000
\overfullrule=0pt
\parindent=0pt

\nopagenumbers

\input amstex

\def\ai{a^{-1}}
\def\bi{b^{-1}}\def\ci{c^{-1}}

\input colordvi

\def\cltr{\Red}		  % Red  VERY-Approx PANTONE RED

\loadmsbm

\input epsf

\def\ctln{\centerline}
\def\u{\underbar}
\def\ssk{\smallskip}
\def\msk{\medskip}
\def\bsk{\bigskip}
\def\hsk{\hskip.1in}
\def\hhsk{\hskip.2in}
\def\dsl{\displaystyle}
\def\hskp{\hskip1.5in}

\def\lra{$\Leftrightarrow$ }
\def\ra{\rightarrow}
\def\mpto{\logmapsto}
\def\pu{\pi_1}
\def\mpu{$\pi_1$}
\def\sig{\Sigma}
\def\msig{$\Sigma$}
\def\ep{\epsilon}
\def\sset{\subseteq}
\def\del{\partial}
\def\inv{^{-1}}
\def\wtl{\widetilde}
%\def\lra{\Leftrightarrow}
\def\del{\partial}
\def\delp{\partial^\prime}
\def\delpp{\partial^{\prime\prime}}
\def\sgn{{\roman{sgn}}}
\def\wtih{\widetilde{H}}
\def\bbz{{\Bbb Z}}
\def\bbr{{\Bbb R}}
\def\rtar{$\Rightarrow$}

\def\cltr{\Red}		  % Red  VERY-Approx PANTONE RED
\def\cltb{\Blue}		  % Blue  Approximate PANTONE BLUE-072
\def\cltg{\PineGreen}	  % ForestGreen  Approximate PANTONE 349

{\bf A non-trivial presentation of the trivial group}

\msk

$G=\langle a,b\ |\ bba\bi\ai,aab\ai\bi\rangle $

$= \langle a,b\ |\ ab\ai=b^2,ba\bi=a^2\rangle$


\msk

Then:

\msk

$b=$
\hhsk
$\cltr{(aab\ai\bi)}b=$
\hhsk
$aab\ai\cltr{(\bi b)}=$

$aab\ai=$
\hhsk
$aab\cltr{b\bi}\ai=$

$aabb\cltg{(ab\ai)^{-1}}\cltr{(bba\bi\ai)^{-1}}\cltg{(ab\ai)}\bi\ai=$

$aabb\cltg{a\bi\ai}\cltr{ab\ai\bi\bi}\cltg{ab\ai}\bi\ai=$

$aabba\bi\cltr{\ai a}b\ai\bi\bi ab\ai\bi\ai=$

$aabba\cltr{\bi b}\ai\bi\bi ab\ai\bi\ai=$

$aabb\cltr{a\ai}\bi\bi ab\ai\bi\ai=$

$aab\cltr{b\bi}\bi ab\ai\bi\ai=$
\hhsk
$aa\cltr{b\bi}ab\ai\bi\ai=$


$a\cltr{aab\ai\bi}\ai=$
\hhsk
$a\ai=$
\hhsk
$1$

\msk

Similarly, $a=1$ in $G$. So every word in $a$ and $b$ is $1$, so $G=\{1\}$.

\msk

$\langle a,b,c\ |\ ab\ai=b^2,bc\bi=c^2,ca\ci=a^2\rangle$ is also a
presentation for the trivial group, 
but the proof is much more involved.

\msk

On the other hand, $H= \langle a,b\ |\ aba=b^2,bab=a^2\rangle$ isn't trivial, since

$\varphi:F(a,b)\ra\bbz_3$ defined by $\varphi(a)=1,\varphi(b)=2$ is onto and
satisfies $\varphi(aba)=4=1=4=\varphi(b^2)$ and $\varphi(bab)=5=2=\varphi(a^2)$,
so descends to a surjective homomorphism $\vartheta:H\ra \bbz_3$. So $H\neq\{1\}$.

\vfill
\end





$aaab\ai\bi\ai=$
\hhsk
