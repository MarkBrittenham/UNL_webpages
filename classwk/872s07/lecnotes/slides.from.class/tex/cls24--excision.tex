\magnification=2000
\overfullrule=0pt
\parindent=0pt

\nopagenumbers

\input amstex

%\voffset=-.6in
%\hoffset=-.5in
%\hsize = 7.5 true in
%\vsize=10.4 true in

\voffset=1.8 true in
\hoffset=-.6 true in
\hsize = 10.2 true in
\vsize=8 true in

\input colordvi

\def\cltr{\Red}		  % Red  VERY-Approx PANTONE RED
\def\cltb{\Blue}		  % Blue  Approximate PANTONE BLUE-072
\def\cltg{\PineGreen}	  % ForestGreen  Approximate PANTONE 349
\def\cltp{\DarkOrchid}	  % DarkOrchid  No PANTONE match
\def\clto{\Orange}	  % Orange  Approximate PANTONE ORANGE-021
\def\cltpk{\CarnationPink}	  % CarnationPink  Approximate PANTONE 218
\def\clts{\Salmon}	  % Salmon  Approximate PANTONE 183
\def\cltbb{\TealBlue}	  % TealBlue  Approximate PANTONE 3145
\def\cltrp{\RoyalPurple}	  % RoyalPurple  Approximate PANTONE 267
\def\cltp{\Purple}	  % Purple  Approximate PANTONE PURPLE

\def\cgy{\GreenYellow}     % GreenYellow  Approximate PANTONE 388
\def\cyy{\Yellow}	  % Yellow  Approximate PANTONE YELLOW
\def\cgo{\Goldenrod}	  % Goldenrod  Approximate PANTONE 109
\def\cda{\Dandelion}	  % Dandelion  Approximate PANTONE 123
\def\capr{\Apricot}	  % Apricot  Approximate PANTONE 1565
\def\cpe{\Peach}		  % Peach  Approximate PANTONE 164
\def\cme{\Melon}		  % Melon  Approximate PANTONE 177
\def\cyo{\YellowOrange}	  % YellowOrange  Approximate PANTONE 130
\def\coo{\Orange}	  % Orange  Approximate PANTONE ORANGE-021
\def\cbo{\BurntOrange}	  % BurntOrange  Approximate PANTONE 388
\def\cbs{\Bittersweet}	  % Bittersweet  Approximate PANTONE 167
%\def\creo{\RedOrange}	  % RedOrange  Approximate PANTONE 179
\def\cma{\Mahogany}	  % Mahogany  Approximate PANTONE 484
\def\cmr{\Maroon}	  % Maroon  Approximate PANTONE 201
\def\cbr{\BrickRed}	  % BrickRed  Approximate PANTONE 1805
\def\crr{\Red}		  % Red  VERY-Approx PANTONE RED
\def\cor{\OrangeRed}	  % OrangeRed  No PANTONE match
\def\paru{\RubineRed}	  % RubineRed  Approximate PANTONE RUBINE-RED
\def\cwi{\WildStrawberry}  % WildStrawberry  Approximate PANTONE 206
\def\csa{\Salmon}	  % Salmon  Approximate PANTONE 183
\def\ccp{\CarnationPink}	  % CarnationPink  Approximate PANTONE 218
\def\cmag{\Magenta}	  % Magenta  Approximate PANTONE PROCESS-MAGENTA
\def\cvr{\VioletRed}	  % VioletRed  Approximate PANTONE 219
\def\parh{\Rhodamine}	  % Rhodamine  Approximate PANTONE RHODAMINE-RED
\def\cmu{\Mulberry}	  % Mulberry  Approximate PANTONE 241
\def\parv{\RedViolet}	  % RedViolet  Approximate PANTONE 234
\def\cfu{\Fuchsia}	  % Fuchsia  Approximate PANTONE 248
\def\cla{\Lavender}	  % Lavender  Approximate PANTONE 223
\def\cth{\Thistle}	  % Thistle  Approximate PANTONE 245
\def\corc{\Orchid}	  % Orchid  Approximate PANTONE 252
\def\cdo{\DarkOrchid}	  % DarkOrchid  No PANTONE match
\def\cpu{\Purple}	  % Purple  Approximate PANTONE PURPLE
\def\cpl{\Plum}		  % Plum  VERY-Approx PANTONE 518
\def\cvi{\Violet}	  % Violet  Approximate PANTONE VIOLET
\def\clrp{\RoyalPurple}	  % RoyalPurple  Approximate PANTONE 267
\def\cbv{\BlueViolet}	  % BlueViolet  Approximate PANTONE 2755
\def\cpe{\Periwinkle}	  % Periwinkle  Approximate PANTONE 2715
\def\ccb{\CadetBlue}	  % CadetBlue  Approximate PANTONE (534+535)/2
\def\cco{\CornflowerBlue}  % CornflowerBlue  Approximate PANTONE 292
\def\cmb{\MidnightBlue}	  % MidnightBlue  Approximate PANTONE 302
\def\cnb{\NavyBlue}	  % NavyBlue  Approximate PANTONE 293
\def\crb{\RoyalBlue}	  % RoyalBlue  No PANTONE match
%\def\cbb{\Blue}		  % Blue  Approximate PANTONE BLUE-072
\def\cce{\Cerulean}	  % Cerulean  Approximate PANTONE 3005
\def\ccy{\Cyan}		  % Cyan  Approximate PANTONE PROCESS-CYAN
\def\cpb{\ProcessBlue}	  % ProcessBlue  Approximate PANTONE PROCESS-BLUE
\def\csb{\SkyBlue}	  % SkyBlue  Approximate PANTONE 2985
\def\ctu{\Turquoise}	  % Turquoise  Approximate PANTONE (312+313)/2
\def\ctb{\TealBlue}	  % TealBlue  Approximate PANTONE 3145
\def\caq{\Aquamarine}	  % Aquamarine  Approximate PANTONE 3135
\def\cbg{\BlueGreen}	  % BlueGreen  Approximate PANTONE 320
\def\cem{\Emerald}	  % Emerald  No PANTONE match
%\def\cjg{\JungleGreen}	  % JungleGreen  Approximate PANTONE 328
\def\csg{\SeaGreen}	  % SeaGreen  Approximate PANTONE 3268
\def\cgg{\Green}	  % Green  VERY-Approx PANTONE GREEN
\def\cfg{\ForestGreen}	  % ForestGreen  Approximate PANTONE 349
\def\cpg{\PineGreen}	  % PineGreen  Approximate PANTONE 323
\def\clg{\LimeGreen}	  % LimeGreen  No PANTONE match
\def\cyg{\YellowGreen}	  % YellowGreen  Approximate PANTONE 375
\def\cspg{\SpringGreen}	  % SpringGreen  Approximate PANTONE 381
\def\cog{\OliveGreen}	  % OliveGreen  Approximate PANTONE 582
\def\pars{\RawSienna}	  % RawSienna  Approximate PANTONE 154
\def\cse{\Sepia}		  % Sepia  Approximate PANTONE 161
\def\cbr{\Brown}		  % Brown  Approximate PANTONE 1615
\def\cta{\Tan}		  % Tan  No PANTONE match
\def\cgr{\Gray}		  % Gray  Approximate PANTONE COOL-GRAY-8
\def\cbl{\Black}		  % Black  Approximate PANTONE PROCESS-BLACK
\def\cwh{\White}		  % White  No PANTONE match


\loadmsbm

\input epsf

\def\ctln{\centerline}
\def\u{\underbar}
\def\ssk{\smallskip}
\def\msk{\medskip}
\def\bsk{\bigskip}
\def\hsk{\hskip.1in}
\def\hhsk{\hskip.2in}
\def\dsl{\displaystyle}
\def\hskp{\hskip1.5in}

\def\lra{$\Leftrightarrow$ }
\def\ra{\rightarrow}
\def\mpto{\logmapsto}
\def\pu{\pi_1}
\def\mpu{$\pi_1$}
\def\sig{\Sigma}
\def\msig{$\Sigma$}
\def\ep{\epsilon}
\def\sset{\subseteq}
\def\del{\partial}
\def\inv{^{-1}}
\def\wtl{\widetilde}
%\def\lra{\Leftrightarrow}
\def\del{\partial}
\def\delp{\partial^\prime}
\def\delpp{\partial^{\prime\prime}}
\def\sgn{{\roman{sgn}}}
\def\wtih{\widetilde{H}}
\def\bbz{{\Bbb Z}}
\def\bbr{{\Bbb R}}


{\bf Homology on ``small'' chains = singular homology:} \hsk
The point to all of these calculations was that if $\{{\Cal U}_\alpha\}$ is an open cover of $X$, then the 
inclusions $i_n:C_n^{\Cal U}(X)\ra C_n(X)$ induce isomorphisms on homology. This gives us two
big theorems. First:

\ssk

{\bf Mayer-Vietoris Sequence}: If $X={\Cal U}\cup{\Cal V}$ is the union of two open sets, then
the short exact sequences \hhsk 
$0\ra C_n({\Cal U}\cap {\Cal V}) \ra C_n({\Cal U})\oplus C_n({\Cal V}) \ra C_n^{\{ {\Cal U},{\Cal V}\}}(X)\ra 0$
\hhsk , together with the isomorphism above, give the long exact sequence

\ctln{$\cdots \ra H_n({\Cal U}\cap {\Cal V}) {\buildrel{(i_{{\Cal U}*},-i_{{\Cal V}*})}\over\ra}
H_n({\Cal U})\oplus H_n({\Cal V}) {\buildrel{j_{{\Cal U}*}+j_{{\Cal V}*}}\over\ra}H_n(X)
{\buildrel \del\over\ra} H_{n-1}({\Cal U}\cap {\Cal V}) \ra \cdots$}

\ssk

As with Seifert - van Kampen, we can replace open sets by sets $A,B$ having nbhds ${\Cal U},{\Cal V}$ which def.
retract to them, so that ${\Cal U}\cap{\Cal V}$ def. retracts to $A\cap B$. E.g.,
subcomplexes $A,B\sset X$ of a CW-complex, with $A\cup B = X$ have homology satisfying a l.e.s.

\ssk

\ctln{$\cdots \ra H_n(A\cap B) {\buildrel{(i_{A*},-i_{B*})}\over\ra}
H_n(A)\oplus H_n(B) {\buildrel{j_{A*}+j_{B*}}\over\ra}H_n(X)
{\buildrel \del\over\ra} H_{n-1}(A\cap B) \ra \cdots$}

\ssk

For reduced homology, we augment the chain complexes used above with the 
s.e.s. \hhsk $0\ra {\Bbb Z}\ra {\Bbb Z}\oplus {\Bbb Z} \ra {\Bbb Z} \ra 0$ , 
where the maps are $a\mapsto (a,-a)$ and $(a,b)\mapsto a+b$ .

\msk

E.g., an $n$-sphere $S^n$
is the union $S^n_+\cup S^n_-$ of its upper and lower hemispheres, each of which 
is contractible, and have intersection $S^n_+\cap S^n_-=S^{n-1}_0$ the equatorial
$(n-1)$-sphere. So Mayer-Vietoris gives us the exact sequence

\hhsk $\cdots \ra \widetilde{H}_k(S^n_+)\oplus \widetilde{H}_k(S^n_-) \ra \widetilde{H}_k(S^n)
\ra \widetilde{H}_{k-1}(S^{n-1}_0) \ra \widetilde{H}_{k-1}(S^n_+)\oplus \widetilde{H}_{k-1}(S^n_-)\ra \cdots$ 
\hhsk , i.e, \hhsk

$0 \ra \widetilde{H}_k(S^n)\ra \widetilde{H}_{k-1}(S^{n-1}_0) \ra  0$ \hhsk 
i.e., $\widetilde{H}_k(S^n)\cong \widetilde{H}_{k-1}(S^{n-1})$ for every $k$ and $n$. 
So by induction, 

\ssk

\ctln{$\widetilde{H}_k(S^n)\cong\widetilde{H}_{k-n}(S^0)\cong    
\cases
{\Bbb Z}, & \text{if}\ $k=n$\cr
0, & \text{otherwise} $ $\cr 
\endcases$}

\vfill
\eject

The second result that this machinery gives us is what is properly known as {\it excision}:

\msk

If $B\sset A\sset X$ and cl$_X(B)\sset$ int$_X(A)$, then for every $k$ the inclusion-induced map 
$H_k(X\setminus B,A\setminus B)\ra H_k(X,A)$ is an isomorphism. 

\msk

An equivalent formulation of this is that if $A,B\sset X$ and int$_X(A)\cup$ int$_X(B) = X$, then the
inclusion-induced map $H_k(B,A\cap B)\ra H_k(X,A)$ is an isomorphism. [From first to second
statement, set $B^\prime = X\setminus B$ .] 

\ssk

To prove the second statement, we know that
the inclusions $C_n^{\{A,B\}}(X) \ra C_n(X)$ induce isomorphisms on homology, as does 
$C_n(A) \ra C_n(A)$, so, by the five lemma, the induced map

\ctln{$C_n^{\{A,B\}}(X)/C_n(A) \ra C_n(X)/C_n(A) = C_n(X,A)$}

induces isomorphisms on homology. 
But the inclusion 

\ctln{$C_n(B) \ra C_n^{\{A,B\}}(X)$}

induces a map 

\ctln{$C_n(B,A\cap B) = C_n(B)/C_n(A\cap B) \ra C_n^{\{A,B\}}(X)/C_n(A)$}

which is an isomorphism of chain groups;
a basis for $C_n^{\{A,B\}}(X)/C_n(A)$ consists of singular simplices which map into $A$ or $B$, but don't map into $A$,
i.e., of simplices mapping into $B$ but not $A$, i.e., of simplices mapping into $B$ but not $A\cap B$.
But this is the \u{same} as the basis for $C_n(B,A\cap B)$ !

\vfill
\eject

With these tools, we can start making some \u{real} homology computations. First, we show that 
if $\emptyset\neq A\sset X$ is ``nice enough'', then $H_n(X,A)\cong \widetilde{H}_n(X/A)$ .
The definition of nice enough, like Seifert - van Kampen, is that $A$ is closed and has an open neighborhood
${\Cal U}$ that deformation retracts to $A$ (think: $A$ is the subcomplex of a CW-complex $X$).
Then using ${\Cal U},X\setminus A$ as a cover of $X$, and ${\Cal U}/A,(X\setminus A)/A$ as a cover of $X/A$,
 we have

\ssk

$\widetilde{H}_n(X/A) {\buildrel {(1)}\over \cong} H_n(X/A,A/A){\buildrel {(2)}\over \cong} 
H_n(X/A,{\Cal U}/A) {\buildrel {(3)}\over \cong} H_n(X/A\setminus A/A,{\Cal U}/A\setminus A/A) {\buildrel {(4)}\over \cong}
H_n(X\setminus A,{\Cal U}\setminus A){\buildrel {(5)}\over \cong} H_n(X,A)$

\ssk

Where (1),(2) follow from the LES for a pair, (3),(5) by excision, and (4) because the restriction of the quotient
map $X\ra X/A$ gives a homeomorphism of pairs.

\msk

Second, if $X,Y$ are $T_1$, $x\in X$ and $y\in Y$ each have neighborhoods 
${\Cal U},{\Cal V}$ which deformation retract to each point, then the 
one-point union $Z=X\vee Y = (X\coprod Y)/(x=y)$ has $\widetilde{H}_n(Z) \cong \widetilde{H}_n(X)\oplus \widetilde{H}_n(Y)$;
this follows from a similar sequence of isomorphisms. Setting $z$ = the image of $\{x,y\}$ in $Z$, we have

\ssk

$\widetilde{H}_n(Z) \cong H_n(Z,z) \cong H_n(Z,{\Cal U}\vee{\Cal V}) \cong H_n(Z\setminus z,{\Cal U}\vee{\Cal V}\setminus z)
\cong H_n([X\setminus x]\coprod[Y\setminus y],[{\Cal U}\setminus x]\coprod [{\Cal V}\setminus y])
\cong H_n(X\setminus x,{\Cal U}\setminus x)\oplus H_n(Y\setminus y,{\Cal V}\setminus y) 
\cong H_n(X,x)\oplus H_n(Y,y)\cong \widetilde{H}_n(X)\oplus \widetilde{H}_n(Y)$

\ssk

By induction, we then have $\displaystyle \widetilde{H}_n(\vee_{i=1}^k X_i) \cong \oplus_{i=1}^k \widetilde{H}_n(X_i)$

\vfill
\end







