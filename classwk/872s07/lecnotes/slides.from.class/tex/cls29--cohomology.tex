



\magnification=2000
\overfullrule=0pt
\parindent=0pt

\nopagenumbers

\input amstex

%\voffset=-.6in
%\hoffset=-.5in
%\hsize = 7.5 true in
%\vsize=10.4 true in

\voffset=1.8 true in
\hoffset=-.6 true in
\hsize = 10.2 true in
\vsize=8 true in

\input colordvi



\def\cltr{\Red}		  % Red  VERY-Approx PANTONE RED
\def\cltb{\Blue}		  % Blue  Approximate PANTONE BLUE-072
\def\cltg{\PineGreen}	  % ForestGreen  Approximate PANTONE 349
\def\cltp{\DarkOrchid}	  % DarkOrchid  No PANTONE match
\def\clto{\Orange}	  % Orange  Approximate PANTONE ORANGE-021
\def\cltpk{\CarnationPink}	  % CarnationPink  Approximate PANTONE 218
\def\clts{\Salmon}	  % Salmon  Approximate PANTONE 183
\def\cltbb{\TealBlue}	  % TealBlue  Approximate PANTONE 3145
\def\cltrp{\RoyalPurple}	  % RoyalPurple  Approximate PANTONE 267
\def\cltp{\Purple}	  % Purple  Approximate PANTONE PURPLE

\def\cgy{\GreenYellow}     % GreenYellow  Approximate PANTONE 388
\def\cyy{\Yellow}	  % Yellow  Approximate PANTONE YELLOW
\def\cgo{\Goldenrod}	  % Goldenrod  Approximate PANTONE 109
\def\cda{\Dandelion}	  % Dandelion  Approximate PANTONE 123
\def\capr{\Apricot}	  % Apricot  Approximate PANTONE 1565
\def\cpe{\Peach}		  % Peach  Approximate PANTONE 164
\def\cme{\Melon}		  % Melon  Approximate PANTONE 177
\def\cyo{\YellowOrange}	  % YellowOrange  Approximate PANTONE 130
\def\coo{\Orange}	  % Orange  Approximate PANTONE ORANGE-021
\def\cbo{\BurntOrange}	  % BurntOrange  Approximate PANTONE 388
\def\cbs{\Bittersweet}	  % Bittersweet  Approximate PANTONE 167
%\def\creo{\RedOrange}	  % RedOrange  Approximate PANTONE 179
\def\cma{\Mahogany}	  % Mahogany  Approximate PANTONE 484
\def\cmr{\Maroon}	  % Maroon  Approximate PANTONE 201
\def\cbr{\BrickRed}	  % BrickRed  Approximate PANTONE 1805
\def\crr{\Red}		  % Red  VERY-Approx PANTONE RED
\def\cor{\OrangeRed}	  % OrangeRed  No PANTONE match
\def\paru{\RubineRed}	  % RubineRed  Approximate PANTONE RUBINE-RED
\def\cwi{\WildStrawberry}  % WildStrawberry  Approximate PANTONE 206
\def\csa{\Salmon}	  % Salmon  Approximate PANTONE 183
\def\ccp{\CarnationPink}	  % CarnationPink  Approximate PANTONE 218
\def\cmag{\Magenta}	  % Magenta  Approximate PANTONE PROCESS-MAGENTA
\def\cvr{\VioletRed}	  % VioletRed  Approximate PANTONE 219
\def\parh{\Rhodamine}	  % Rhodamine  Approximate PANTONE RHODAMINE-RED
\def\cmu{\Mulberry}	  % Mulberry  Approximate PANTONE 241
\def\parv{\RedViolet}	  % RedViolet  Approximate PANTONE 234
\def\cfu{\Fuchsia}	  % Fuchsia  Approximate PANTONE 248
\def\cla{\Lavender}	  % Lavender  Approximate PANTONE 223
\def\cth{\Thistle}	  % Thistle  Approximate PANTONE 245
\def\corc{\Orchid}	  % Orchid  Approximate PANTONE 252
\def\cdo{\DarkOrchid}	  % DarkOrchid  No PANTONE match
\def\cpu{\Purple}	  % Purple  Approximate PANTONE PURPLE
\def\cpl{\Plum}		  % Plum  VERY-Approx PANTONE 518
\def\cvi{\Violet}	  % Violet  Approximate PANTONE VIOLET
\def\clrp{\RoyalPurple}	  % RoyalPurple  Approximate PANTONE 267
\def\cbv{\BlueViolet}	  % BlueViolet  Approximate PANTONE 2755
\def\cpe{\Periwinkle}	  % Periwinkle  Approximate PANTONE 2715
\def\ccb{\CadetBlue}	  % CadetBlue  Approximate PANTONE (534+535)/2
\def\cco{\CornflowerBlue}  % CornflowerBlue  Approximate PANTONE 292
\def\cmb{\MidnightBlue}	  % MidnightBlue  Approximate PANTONE 302
\def\cnb{\NavyBlue}	  % NavyBlue  Approximate PANTONE 293
\def\crb{\RoyalBlue}	  % RoyalBlue  No PANTONE match
%\def\cbb{\Blue}		  % Blue  Approximate PANTONE BLUE-072
\def\cce{\Cerulean}	  % Cerulean  Approximate PANTONE 3005
\def\ccy{\Cyan}		  % Cyan  Approximate PANTONE PROCESS-CYAN
\def\cpb{\ProcessBlue}	  % ProcessBlue  Approximate PANTONE PROCESS-BLUE
\def\csb{\SkyBlue}	  % SkyBlue  Approximate PANTONE 2985
\def\ctu{\Turquoise}	  % Turquoise  Approximate PANTONE (312+313)/2
\def\ctb{\TealBlue}	  % TealBlue  Approximate PANTONE 3145
\def\caq{\Aquamarine}	  % Aquamarine  Approximate PANTONE 3135
\def\cbg{\BlueGreen}	  % BlueGreen  Approximate PANTONE 320
\def\cem{\Emerald}	  % Emerald  No PANTONE match
%\def\cjg{\JungleGreen}	  % JungleGreen  Approximate PANTONE 328
\def\csg{\SeaGreen}	  % SeaGreen  Approximate PANTONE 3268
\def\cgg{\Green}	  % Green  VERY-Approx PANTONE GREEN
\def\cfg{\ForestGreen}	  % ForestGreen  Approximate PANTONE 349
\def\cpg{\PineGreen}	  % PineGreen  Approximate PANTONE 323
\def\clg{\LimeGreen}	  % LimeGreen  No PANTONE match
\def\cyg{\YellowGreen}	  % YellowGreen  Approximate PANTONE 375
\def\cspg{\SpringGreen}	  % SpringGreen  Approximate PANTONE 381
\def\cog{\OliveGreen}	  % OliveGreen  Approximate PANTONE 582
\def\pars{\RawSienna}	  % RawSienna  Approximate PANTONE 154
\def\cse{\Sepia}		  % Sepia  Approximate PANTONE 161
\def\cbr{\Brown}		  % Brown  Approximate PANTONE 1615
\def\cta{\Tan}		  % Tan  No PANTONE match
\def\cgr{\Gray}		  % Gray  Approximate PANTONE COOL-GRAY-8
\def\cbl{\Black}		  % Black  Approximate PANTONE PROCESS-BLACK
\def\cwh{\White}		  % White  No PANTONE match


\loadmsbm

\input epsf

\def\ctln{\centerline}
\def\u{\underbar}
\def\ssk{\smallskip}
\def\msk{\medskip}
\def\bsk{\bigskip}
\def\hsk{\hskip.1in}
\def\hhsk{\hskip.2in}
\def\dsl{\displaystyle}
\def\hskp{\hskip1.5in}

\def\lra{$\Leftrightarrow$ }
\def\ra{\rightarrow}
\def\mpto{\logmapsto}
\def\pu{\pi_1}
\def\mpu{$\pi_1$}
\def\sig{\Sigma}
\def\msig{$\Sigma$}
\def\ep{\epsilon}
\def\sset{\subseteq}
\def\del{\partial}
\def\inv{^{-1}}
\def\wtl{\widetilde}
%\def\lra{\Leftrightarrow}
\def\del{\partial}
\def\delp{\partial^\prime}
\def\delpp{\partial^{\prime\prime}}
\def\sgn{{\roman{sgn}}}
\def\wtih{\widetilde{H}}
\def\bbz{{\Bbb Z}}
\def\bbr{{\Bbb R}}
\def\bbq{{\Bbb Q}}
\def\bbc{{\Bbb C}}
\def\hdsk{\hskip.7in}
\def\hdskb{\hskip.9in}
\def\hdskc{\hskip1.1in}
\def\hdskd{\hskip1.3in}
\def\Hom{\text{Hom}}
\def\Ext{\text{Ext}}
\def\larr{\leftarrow}



{\bf Cohomology:} There is ``dual'' theory to the homology theory, called
``cohomology theory'', which is based on the following observation: if we
have a chain complex

\ssk

\ctln{$\cdots \ra C_n\buildrel{\del_n}\over\ra C_{n-1} \ra \cdots \ra C_0\ra 0$}

\ssk

with, we assume for the moment, finitely-generated free abelian chain groups, then,
thinking in terms of $\bbz$-vector spaces,
the boundary maps are represented by integer matrices $d_n:C_n\ra C_{n-1}$.
The {\it transposes} of these matrices give rise to linear maps $d_n^{\text{T}}:C_{n-1}\ra C_n$,
which form a chain complex (since $   d_{n}^{\text{T}}d_{n-1}^{\text{T}} = (d_{n-1}d_n)^{\text{T}} = 0^{\text{T}} = 0$)
{\it running the opposite direction}

\ssk

\ctln{$0 \ra C_0\ra C_{1} \ra \cdots \ra C_{n-1}\buildrel{\delta_n}\over\ra C_n\ra \cdots$}

\ssk

whose homology groups we can then compute in the usual way. This is the basic idea behind
cohomology. The transpose is not ``really'' a map from $C_{n-1}$ to $C_n$, though, it
is more properly thought of as a map between the dual vector spaces $C_{n-1}^*,C_n^*$.
The formal way to construct the theory is to introduce the ``Hom'' functor: for a
(fixed) abelian group $G$ and an abelian group $A$, 

\ssk

\ctln{$\Hom(A,G) = \{f: A\ra G : f\text{ is a homomomorphism}\}$ .}

\ssk

For $\bbr$-vector spaces $V$, for example, the usual dual vector space is $V^*=\Hom(V,\bbr)$.
The basic point is that Hom ``turns arrows around''; given a homomorphism 

$\varphi:A\ra B$, there
is an induced homomorphism $\varphi^*:\Hom(B,G)\ra\Hom(A,G)$ given by
$\varphi^*(f)=f\circ\varphi$ . This satisfies $(\varphi\circ\psi)^*=\psi^*\circ\varphi^*$ and
$I^*=I$, as a quick computation establishes. Just as important for us is that $0^*=0$.

\vfill
\eject

Given a chain complex 
\hskip.2in 
$\cdots \ra C_n\buildrel{\del_n}\over\ra C_{n-1} \ra \cdots \ra C_0\ra 0$
\hskip.2in 
(with no assumptions on the chain groups) and a ``coefficient'' group $G$, we can dualize
the complex to obtain

\ssk

\ctln{(*) \hskip.2in $0 \ra \Hom(C_0,G)\ra \Hom(C_{1},G) \ra \cdots 
\ra \Hom(C_{n-1},G)\buildrel{\del_n^*}\over\ra \Hom(C_n,G)\ra \cdots$}

\ssk

We will usually write $\del_n^* = \delta_n$, calling them the {\it coboundary} maps.
These coboundary maps are defined by $\del_n^*(f)(x) = f(\del_n x)$ for 
$f\in \Hom(C_{n-1},G)$ and $x\in C_n$ (since $\del_n^*(f)\in \Hom(C_n,G)$).
Just as in the finitely generated case, $\delta_{n+1}\circ\delta_n=0$, since
for any $f\in\Hom(C_{n-1},G)$, and $x\in C_{n+1}$, 

\ssk

\ctln{$\delta_{n+1}\circ\delta_n(f)(x)=\delta_n(f)(\del_{n+1}x) = f(\del_n\del_{n+1}x) \ f(0)=0$,}

\ssk

so $\delta_{n+1}\circ\delta_n(f) = 0$. Consequently, (*) is a chain complex (although 
technically, with its indices rising it is called a {\it cochain complex}), with the
attendant homology groups (which we call {\it cohomology groups}!). We could treat 
this, really, as homology, by renumbering so that $C_{-n}^\prime = \Hom(C_n,G)$;
then the coboundary map decreases index, although our homology groups then live
in negative dimensions! But cohomology is inherently arrow-reversing, so it seems
better to just live with index-increasing maps?

\msk

In particular, starting with a space (or $\Delta$-complex or CW-complex)
$X$, dualizing the standard singular (or simplicial or cellular) chain complex
$C_n(X)$ we obtain the singular (or...) cochain complex $(C^n(X;G),\delta_{n+1})$,
whose elements are {\it cochains}, and whose homology groups are
the singular (or...) cohomology groups of $X$, with coefficients in $G$. As with
homology, it is not immediately clear that the simplicial and cellular cohomology 
groups are topological invariants, but the singular homology groups are; the
only input is $X$, from which we build $C_n(X)$ and $C^n(X)=\Hom(C_n(X),G)$.

\vfill
\eject

%%% So what do the singular cohomology groups \u{measure}?



We shall see that much of the edifice that we have built around singular homology
goes through, with small changes made necessary by the reversal of arrows. One way
to see a large part of this is by the fact that the \cltr{homology groups of a chain complex 
\u{determine}
the associated cohomology groups.} To see how this might be formulated, we first note that there is a 
homomorphism $h:H^n({\Cal C};G)\ra \Hom(H_n({\Cal C}),G)$, for any chain complex ${\Cal C}$, 
defined as follows: given $[f]\in H^n({\Cal C};G)$ and $[z]\in H_n({\Cal C})$,
we have $f\in \Hom(C_n,G)$ (with $\delta f = 0$; it is a {\it cocycle}),
and $z\in Z_n\subseteq C_n$, so the element $f(z)\in G$
makes sense. So why not just try $h([f])([z]) = f(z)$ ? We can show that this is
well-defined; if $[z]=[z^\prime]$, then $[z]-[z^\prime] = [z-z^\prime]=0$, so 
$z-z^\prime=\del w$ for some $w\in C_{n+1}$, and then 
$f(z)-f(z^\prime) = f(z-z^\prime)=f(\del w) = (\delta f)(w)=0$, since
$\delta f = 0$. OTOH, if $[f]=[f^\prime]$, then $f-f^\prime = \delta g$ 
for some $g\in \Hom(C_{n-1},G)$, and then 
$f(z)-f^\prime(z) = \delta g(z) = g(\del z) = g(0)=0$, since $z$ is an $n$-cycle.
So $h$ is well-defined. And since 
$h([f]-[f^\prime])([z]) = (f-f^\prime)(z) = f(z)-f^\prime(z) = 
h([f])([z]) - h([f^\prime])([z])$, we have $h([f]-[f^\prime]) = h([f])-h([f^\prime])$,
so $h$ is a homomorphism. 

\ssk

Even more, though, \cltg{if the chain groups $C_n$ are free abelian,} then $h$ is onto. 
To see this, note that any 
$\varphi:H_n{\Cal C}=Z_n/B_n\ra G$ gives rise to a homom $\varphi_1:Z_n\ra G$,
by $\varphi_1(z)=\varphi([z])$. But since $C_n$ is free abelian, $Z_n=\ker \del_n$ 
is a direct summand of $C_n$; $B_{n-1}=\text{im}\ \del_n\subseteq C_{n-1}$ is a 
subgroup of a free 
abelian group, so is free abelian, and a basis for $B_{n-1}$, pulled back 
to a collection of elements $\{v_i\}$ of $C_n$, together with
a basis for $Z_n$, gives a basis for $C_n$. [Showing that the two bases span complementary
subspaces, and together span $C_n$, is straightforward.] The point to this is that
our homom $\varphi_1$ can be extended to a homom $\varphi_2:C_n\ra G$ by declaring
that $\varphi_2(v_i)=0$ for all $i$ and that $\varphi_2=\varphi_1$ on $Z_n$. \u{Then}
$\delta(\varphi_2)=0$, since 
$\delta(\varphi_2)(x)=\varphi_2(\del x) = \varphi_1(\del x) = \varphi([\del x]) = \varphi(0)=0$
for all $x$. [$\varphi_2(\del x) = \varphi_1(\del x)$ since $\del x\in B_n\subseteq Z_n$ .]

\vfill
\eject

So $\varphi_2$ is a cocycle, and $h([\varphi_2])([z]) = \varphi_2(z) = \varphi_1(z)
=\varphi([z])$, so $h([\varphi_2]) = \varphi$, as desired.

So we have the beginnings of a short exact sequence; 

\ssk

\ctln{$0\ra \ker h\ra H^n({\Cal C};G)\buildrel{h}\over\ra \Hom(H_n({\Cal C}),G)\ra 0$}

\ssk

This sequence splits; our construction above actually describes a homom
$k:\varphi\mapsto \varphi_2$, since $\varphi\mapsto \varphi_1$ is a homom,
and $\varphi_2$ is essentially $\varphi_1$ extended by $0$ to a subspace
complementary to $Z_n$ (in matrix terms, we pad the matrix for $\varphi_1$
with columns of $0$'s). Since $h(\varphi_2) = \varphi$, $k$ is a right
inverse to $h$. This then implies, by the Splitting Lemma (?), that

\ssk

\ctln{$H^n({\Cal C};G) \cong \Hom(H_n({\Cal C}),G)\oplus \ker h$}

\ssk

and so to show that cohomology depends only on the homology groups of ${\Cal C}$,
it remains to show that $\ker h$ can be computed from the homology groups.

\ssk

\ctln{$\ker h = \{ [f] \text{ } : \text{ } f:C_n\ra G \text{ and } 
f(z)=0 \text{ for all } z\in C_n \text{ with } \del z=0\}$}

%%% \ctln{$ = \{ f:C_n\ra G \text{ } \text{ } : \text{ } f(z) = 0 \text{ for all } z\in Z_n\}/\{\delta g \text{ } : \text{ } g:C_{n-1}\ra G\}$}

\msk

There is another map $j:\Hom(B_{n-1},G)\ra H^n({\Cal C};G)$ given by
$j(\varphi) = [\psi]$, where $\psi:C_n\ra G$ is defined by $\psi(x)=\varphi(\del x)$
[note that $\delta\psi(x) = \psi(\del x) = \varphi(\del^2 x) = 0$ for all $x$, so 
$\psi$ is a cocycle],
and again, this map is a homomorphism. Further, $\text{im}\ j = \ker h$, since
$h(j(\varphi))([z]) = h([\psi])([z]) = \psi(z) = \varphi(\del z) = \varphi(0)=0$ for all $[z]$, giving one 
containment, and given $\psi$ with $h(\psi)=0$, we \u{define} $\varphi: B_{n-1}\ra G$
by $\varphi(\del x) = \psi(x)$; if $\del x = \del y$, then $\psi(x-y)=0$ since
$\del(x-y)=0$, so $\psi(x)=\psi(y)$, and so $\varphi$ is well-defined. 
Yet again, $\varphi$ is a homom.
And certainly $j(\varphi) = \psi$ (by pretending that the equation
$\varphi(\del x) = \psi(x)$ defines $\psi$ !).

\vfill
\eject

\ctln{$j(\varphi) = [\psi:x\mapsto \varphi(\del x)]\in H^n({\Cal C},G)\}$}

\ssk

Therefore, by one of the isomorphism theorems, $\ker h \cong \Hom(B_{n-1},G)/\ker j$.
But $\ker j$ consists of those maps $\varphi:B_{n-1}\ra G$ for which $x\mapsto \varphi(\del x)$ is a
coboundary $(\delta\psi)(x) = \psi(\del x)$ for
some $\psi:C_{n-1}\ra G$. On the face of it, it looks like $\varphi$
itself could stand in for $\psi$, but the point is that 
\cltr{$\varphi$ and $\psi$ have different domains}.
$\psi$ has domain $C_{n-1}$, while $\varphi$ has domain $B_{n-1}$. 
But this means that $\ker j$ is the image of the map 
$\Hom(C_{n-1},G)\ra \Hom(B_{n-1},G)$ dual to the inclusion map 
$B_{n-1}\hookrightarrow C_{n-1}$ . Note that we can put a third term in the middle of these
two; 

\ctln{$\Hom(C_{n-1},G)\ra \Hom(Z_{n-1},G)\ra \Hom(B_{n-1},G)$}

since $B_{n-1}\hookrightarrow Z_{n-1}\hookrightarrow C_{n-1}$. But the map
$\Hom(C_{n-1},G)\ra \Hom(Z_{n-1},G)$ is surjective, since $Z_{n-1}$ is a direct summand 
of $C_{n-1}$ (as before, we extend a map $\varphi$ from $z_{n-1}$ by zero of a complementary subspace
to build a map from $C_{n-1}$ whose image is $\varphi$). So $\ker j$
is also the image of the dual to the inclusion $i:B_{n-1}\hookrightarrow Z_{n-1}$.

\msk

The reason for tinkering with things in this way is that $B_{n-1}$ and $Z_{n-1}$ fit into
a short exact sequence 

\ssk

\ctln{(**) $0\ra B_{n-1} \ra Z_{n-1} \ra H_{n-1}({\Cal C}) \ra 0$}

\ssk

with dual sequence

\ssk

\ctln{(***) $0\leftarrow \Hom(B_{n-1},G) \leftarrow \Hom(Z_{n-1},G) \leftarrow \Hom(H_{n-1}({\Cal C}),G) \leftarrow 0$}

\ssk

This sequence is not exact, but it is a cochain complex, and so has its own
homology groups. Note that the group that we are after is the homology of this cochain complex
at the spot $\Hom(B_{n-1},G)$. Since by hypothesis $C_{n-1}$ is free abelian, so are
$B_{n-1}$ and $Z_{n-1}$; (**) is then an example of a {\it free resolution} of the abelian
group $H_{n-1}({\Cal C})$. 

\vfill
\eject

Since I am getting tired of doing homological algebra and not
topology, we will finish our proof that $\ker h$ (which we now know to be
the (co)homology group mentioned above) depends only on the homology of ${\Cal C}$ by
appealing to:

\msk

\cltr{The homology groups of the cochain complex dual to a free resolution of a group $H$ depend
only on the group $H$, and not on the particular free resolution chosen.}

\msk

The particular homology group that we are interested in is known in the literature
a $\Ext(H,G)$. We will not be interested in knowing why it is called that, but only 
in the fact that, as a consequence we have the

\ssk

{\bf Universal coefficients Theorem:} For any chain complex ${\Cal C}$,

$H^n({\Cal C};G) \cong \Hom(H_n({\Cal C}),G)\oplus \Ext(H_{n-1}({\Cal C}),G)$ .

\ssk

together with some observations on how to calculate Ext, based on its indifference to
the resolution used to compute it. From the exact sequence

\ssk

$0\ra 0 \ra \bbz \ra \bbz \ra 0$, the dual $0\larr 0 \larr G \larr G \larr 0$
gives $\Ext(\bbz,G) = 0$ ; from 

\ssk

$0\ra \bbz \buildrel{\times n}\over\ra \bbz \ra \bbz_n\ra 0$, the dual 
 
\ssk

$0\larr G \buildrel{\times n}\over\larr G \ra \Hom(\bbz_n,G)\ra 0$ gives
$\Ext(\bbz_n,G) = G/nG$ ; and the fact that

\ssk

$\Ext(H_1\oplus H_2,G) \cong \Ext(H_1,G)\oplus \Ext(H_2,G)$ (by taking the direct sum
of two resolutions, and using the fact that $\Hom(-,G)$ respects direct products, \u{and}
that homology respects direct products)

\ssk

suffice to compute $\Ext(H,G)$ for any finitely-generated abelian group $H$, which will
usually suffice for our purposes.

\msk

Applying all of this homological algebra (there is no topology underlying any of the
above work, except as motivation) to our chain complexes from a space $X$, we find
that 

\ssk

$H^n(X;G) \cong \Hom(H_n(X),G)\oplus \Ext(H_{n-1}(X),G)$ .

\ssk

Since the groups on the right are the same whether we use singular, simplicial, or 
cellular chain complexes to build them, the same is true for the left. So the
singular, simplicial, and cellular cohomology groups are all isomorphic (when any two
of them are defined)! If we were to chase through the computations above, we could
recover the fact that the isomorphisms can be induced by the inclusion maps of the
various chain groups.
















\vfill
\end

