
\magnification=1900
\overfullrule=0pt
\parindent=0pt

\nopagenumbers

\input amstex

%\voffset=-.6in
%\hoffset=-.5in
%\hsize = 7.5 true in
%\vsize=10.4 true in

\voffset=1.8 true in
\hoffset=-.6 true in
\hsize = 10.2 true in
\vsize=8 true in

\input colordvi

\def\cltr{\Red}		  % Red  VERY-Approx PANTONE RED

\loadmsbm

\input epsf

\def\ctln{\centerline}
\def\u{\underbar}
\def\ssk{\smallskip}
\def\msk{\medskip}
\def\bsk{\bigskip}
\def\hsk{\hskip.1in}
\def\hhsk{\hskip.2in}
\def\dsl{\displaystyle}
\def\hskp{\hskip1.5in}

\def\lra{$\Leftrightarrow$ }
\def\ra{\rightarrow}
\def\mpto{\logmapsto}
\def\pu{\pi_1}
\def\mpu{$\pi_1$}
\def\sig{\Sigma}
\def\msig{$\Sigma$}
\def\ep{\epsilon}
\def\sset{\subseteq}
\def\del{\partial}
\def\inv{^{-1}}
\def\wtl{\widetilde}
%\def\lra{\Leftrightarrow}
\def\del{\partial}
\def\delp{\partial^\prime}
\def\delpp{\partial^{\prime\prime}}
\def\sgn{{\roman{sgn}}}
\def\wtih{\widetilde{H}}
\def\bbz{{\Bbb Z}}
\def\bbr{{\Bbb R}}
\def\rtar{$\Rightarrow$}

{\bf Applications: the Brouwer Fixed Point Theorem}

\ssk

Calculus: $f:I\ra I$ has a fixed point; $f(x_0)=x_0$ [Pf: apply IVT to $g(x)=f(x)-x$.]

\ssk

\cltr{{\bf Brouwer:} Every map $f:D^2\ra D^2$ has a fixed point.}

\msk

Proof: If not, then $f(x)\neq x$ for all $x\in D^2$. 
Construct a retraction $r:D^2\ra \del D^2$ by
sending $x\in D^2$ to the point on $\del D^2$ lying on the ray from $f(x)$ to $x$.
Formally:

\msk

Send $x$ to $y=x+t(f(x)-x)$, $t\leq 0$ such that $||x+t(f(x)-x)||=1$.

\ssk
 
$||x+t(f(x)-x)||^2-1=\langle x+t(f(x)-x),x+t(f(x)-x)\rangle -1 =$ \hfill

$||f(x)-x||^2t^2+2\langle x,f(x)-x\rangle t +(||x||^2-1)$ $=$ 
$at^2+bt+c=0$ \hskip.1in
and $t\leq 0$ . 

I.e. (note that $c\leq 0$ and $a>0$), 
$t=(-b-\sqrt{b^2-4ac})/2a=$ \hfill

\ssk

$(-\langle x,f(x)-x\rangle-\sqrt{\langle x,f(x)-x\rangle^2-||f(x)-x||^2(||x||^2-1)})/||f(x)-x||^2$,
so 

\msk

\cltr{$\displaystyle r(x)=x+{{-\langle x,f(x)-x\rangle-\sqrt{\langle x,f(x)-x\rangle^2-||f(x)-x||^2(||x||^2-1)}
}\over{||f(x)-x||^2}}(f(x)-x)$}

\msk

which, since $||x-f(x)||$ is bounded away from $0$ (it has a positive minimum on $D^2$), is continuous.
Check: if $||x||=1$, then $r(x)=x$ (because $\langle x,f(x)-x\rangle < 0$).
But a retraction induces a surjective homomorphism on $\pi_1$, so $r_*$
is a surjection from $\pi_1(D^2)=1$ to $\pi_1(\del D^2)=\pi_1(S^1)=\bbz$,
a contradiction. So $f$ must have a fixed point.

\msk

Basic idea: if no fixed point, then a \underbar{new} map that we build has a property (retraction)
which translates into the algebra (surjection) to something which we know can't be true.

\msk

The exact same proof will apply in higher dimensions (every $f:D^n\ra D^n$ has a fixed point), once
we build an algebraic gadget $H$ (``the $(n-1)^{\text{st}}$ homology group'') for which
$H(D^n)=\{1\}$ and $H(\del D^n)=H(S^{n-1})\neq\{1\}$.


\vfill
\end