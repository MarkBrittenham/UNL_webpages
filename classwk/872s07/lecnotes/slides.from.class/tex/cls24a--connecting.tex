




\magnification=2000
\overfullrule=0pt
\parindent=0pt

\nopagenumbers

\input amstex

%\voffset=-.6in
%\hoffset=-.5in
%\hsize = 7.5 true in
%\vsize=10.4 true in

\voffset=1.8 true in
\hoffset=-.6 true in
\hsize = 10.2 true in
\vsize=8 true in

\input colordvi



\def\cltr{\Red}		  % Red  VERY-Approx PANTONE RED
\def\cltb{\Blue}		  % Blue  Approximate PANTONE BLUE-072
\def\cltg{\PineGreen}	  % ForestGreen  Approximate PANTONE 349
\def\cltp{\DarkOrchid}	  % DarkOrchid  No PANTONE match
\def\clto{\Orange}	  % Orange  Approximate PANTONE ORANGE-021
\def\cltpk{\CarnationPink}	  % CarnationPink  Approximate PANTONE 218
\def\clts{\Salmon}	  % Salmon  Approximate PANTONE 183
\def\cltbb{\TealBlue}	  % TealBlue  Approximate PANTONE 3145
\def\cltrp{\RoyalPurple}	  % RoyalPurple  Approximate PANTONE 267
\def\cltp{\Purple}	  % Purple  Approximate PANTONE PURPLE

\def\cgy{\GreenYellow}     % GreenYellow  Approximate PANTONE 388
\def\cyy{\Yellow}	  % Yellow  Approximate PANTONE YELLOW
\def\cgo{\Goldenrod}	  % Goldenrod  Approximate PANTONE 109
\def\cda{\Dandelion}	  % Dandelion  Approximate PANTONE 123
\def\capr{\Apricot}	  % Apricot  Approximate PANTONE 1565
\def\cpe{\Peach}		  % Peach  Approximate PANTONE 164
\def\cme{\Melon}		  % Melon  Approximate PANTONE 177
\def\cyo{\YellowOrange}	  % YellowOrange  Approximate PANTONE 130
\def\coo{\Orange}	  % Orange  Approximate PANTONE ORANGE-021
\def\cbo{\BurntOrange}	  % BurntOrange  Approximate PANTONE 388
\def\cbs{\Bittersweet}	  % Bittersweet  Approximate PANTONE 167
%\def\creo{\RedOrange}	  % RedOrange  Approximate PANTONE 179
\def\cma{\Mahogany}	  % Mahogany  Approximate PANTONE 484
\def\cmr{\Maroon}	  % Maroon  Approximate PANTONE 201
\def\cbr{\BrickRed}	  % BrickRed  Approximate PANTONE 1805
\def\crr{\Red}		  % Red  VERY-Approx PANTONE RED
\def\cor{\OrangeRed}	  % OrangeRed  No PANTONE match
\def\paru{\RubineRed}	  % RubineRed  Approximate PANTONE RUBINE-RED
\def\cwi{\WildStrawberry}  % WildStrawberry  Approximate PANTONE 206
\def\csa{\Salmon}	  % Salmon  Approximate PANTONE 183
\def\ccp{\CarnationPink}	  % CarnationPink  Approximate PANTONE 218
\def\cmag{\Magenta}	  % Magenta  Approximate PANTONE PROCESS-MAGENTA
\def\cvr{\VioletRed}	  % VioletRed  Approximate PANTONE 219
\def\parh{\Rhodamine}	  % Rhodamine  Approximate PANTONE RHODAMINE-RED
\def\cmu{\Mulberry}	  % Mulberry  Approximate PANTONE 241
\def\parv{\RedViolet}	  % RedViolet  Approximate PANTONE 234
\def\cfu{\Fuchsia}	  % Fuchsia  Approximate PANTONE 248
\def\cla{\Lavender}	  % Lavender  Approximate PANTONE 223
\def\cth{\Thistle}	  % Thistle  Approximate PANTONE 245
\def\corc{\Orchid}	  % Orchid  Approximate PANTONE 252
\def\cdo{\DarkOrchid}	  % DarkOrchid  No PANTONE match
\def\cpu{\Purple}	  % Purple  Approximate PANTONE PURPLE
\def\cpl{\Plum}		  % Plum  VERY-Approx PANTONE 518
\def\cvi{\Violet}	  % Violet  Approximate PANTONE VIOLET
\def\clrp{\RoyalPurple}	  % RoyalPurple  Approximate PANTONE 267
\def\cbv{\BlueViolet}	  % BlueViolet  Approximate PANTONE 2755
\def\cpe{\Periwinkle}	  % Periwinkle  Approximate PANTONE 2715
\def\ccb{\CadetBlue}	  % CadetBlue  Approximate PANTONE (534+535)/2
\def\cco{\CornflowerBlue}  % CornflowerBlue  Approximate PANTONE 292
\def\cmb{\MidnightBlue}	  % MidnightBlue  Approximate PANTONE 302
\def\cnb{\NavyBlue}	  % NavyBlue  Approximate PANTONE 293
\def\crb{\RoyalBlue}	  % RoyalBlue  No PANTONE match
%\def\cbb{\Blue}		  % Blue  Approximate PANTONE BLUE-072
\def\cce{\Cerulean}	  % Cerulean  Approximate PANTONE 3005
\def\ccy{\Cyan}		  % Cyan  Approximate PANTONE PROCESS-CYAN
\def\cpb{\ProcessBlue}	  % ProcessBlue  Approximate PANTONE PROCESS-BLUE
\def\csb{\SkyBlue}	  % SkyBlue  Approximate PANTONE 2985
\def\ctu{\Turquoise}	  % Turquoise  Approximate PANTONE (312+313)/2
\def\ctb{\TealBlue}	  % TealBlue  Approximate PANTONE 3145
\def\caq{\Aquamarine}	  % Aquamarine  Approximate PANTONE 3135
\def\cbg{\BlueGreen}	  % BlueGreen  Approximate PANTONE 320
\def\cem{\Emerald}	  % Emerald  No PANTONE match
%\def\cjg{\JungleGreen}	  % JungleGreen  Approximate PANTONE 328
\def\csg{\SeaGreen}	  % SeaGreen  Approximate PANTONE 3268
\def\cgg{\Green}	  % Green  VERY-Approx PANTONE GREEN
\def\cfg{\ForestGreen}	  % ForestGreen  Approximate PANTONE 349
\def\cpg{\PineGreen}	  % PineGreen  Approximate PANTONE 323
\def\clg{\LimeGreen}	  % LimeGreen  No PANTONE match
\def\cyg{\YellowGreen}	  % YellowGreen  Approximate PANTONE 375
\def\cspg{\SpringGreen}	  % SpringGreen  Approximate PANTONE 381
\def\cog{\OliveGreen}	  % OliveGreen  Approximate PANTONE 582
\def\pars{\RawSienna}	  % RawSienna  Approximate PANTONE 154
\def\cse{\Sepia}		  % Sepia  Approximate PANTONE 161
\def\cbr{\Brown}		  % Brown  Approximate PANTONE 1615
\def\cta{\Tan}		  % Tan  No PANTONE match
\def\cgr{\Gray}		  % Gray  Approximate PANTONE COOL-GRAY-8
\def\cbl{\Black}		  % Black  Approximate PANTONE PROCESS-BLACK
\def\cwh{\White}		  % White  No PANTONE match


\loadmsbm

\input epsf

\def\ctln{\centerline}
\def\u{\underbar}
\def\ssk{\smallskip}
\def\msk{\medskip}
\def\bsk{\bigskip}
\def\hsk{\hskip.1in}
\def\hhsk{\hskip.2in}
\def\dsl{\displaystyle}
\def\hskp{\hskip1.5in}

\def\lra{$\Leftrightarrow$ }
\def\ra{\rightarrow}
\def\mpto{\logmapsto}
\def\pu{\pi_1}
\def\mpu{$\pi_1$}
\def\sig{\Sigma}
\def\msig{$\Sigma$}
\def\ep{\epsilon}
\def\sset{\subseteq}
\def\del{\partial}
\def\inv{^{-1}}
\def\wtl{\widetilde}
%\def\lra{\Leftrightarrow}
\def\del{\partial}
\def\delp{\partial^\prime}
\def\delpp{\partial^{\prime\prime}}
\def\sgn{{\roman{sgn}}}
\def\wtih{\widetilde{H}}
\def\bbz{{\Bbb Z}}
\def\bbr{{\Bbb R}}
\def\hdsk{\hskip.7in}
\def\hdskb{\hskip.9in}
\def\hdskc{\hskip1.1in}
\def\hdskd{\hskip1.3in}


{\bf Connecting homomorphisms:} There is a reason the connecting homomorphism in the LEHS
is denoted by ``$\del$'': usually that's what it is. For example, in the LES of a pair,

\ssk

$$\CD
0 @>>> C_{n}(A) @>>> C_n(X) @>>> C_n(X,A) @>>> 0 \\
 @. @VVV @VVV @VVV \\
0 @>>> C_{n-1}(A) @>>> C_{n-1}(X) @>>> C_{n-1}(X,A) @>>> 0 \\
\endCD
$$

\ssk

starting with a relative cycle $[z]\in C_n(X)/C_n(A)$, $\del[z]=0$, this means that 

$z\in C_n(X)$ and
$\del z\in C_{n-1}(A)$, and the connecting homomorphism simply chooses $a=\del z$,
since $\del a=\del\del z=0$, so $a\in Z_{n-1}(A)$. Similarly, in the Mayer-Vietoris
sequence, 

\ssk

$$\CD
0 @>>> C_{n}(A\cap B) @>>> C_n(A)\oplus C_n(B) @>>> C_n^{\{A,B\}}(X) @>>> 0 \\
 @. @VVV @VVV @VVV \\
0 @>>> C_{n-1}(A\cap B) @>>> C_{n-1}(A)\oplus C_{n-1}(B) @>>> C_{n-1}^{\{A,B\}}(X) @>>> 0 \\
\endCD
$$

\ssk

a cycle $z\in C_n^{\{A,B\}}(X)$, $\del z=0$, can be expressed as $z=a+b$ with 

$a\in C_n(A),b\in C_n(B)$,
so $\del b=-\del a$ is in both $C_{n-1}(B)$ and $C_{n-1}(A)$, so 

$w=\del a\in C_{n-1}(A\cap B)$.
Which is what the connecting homomorphism chooses, since $\del w=\del\del a=0$,
so $w\in Z_{n-1}(A\cap B)$.

\vfill\eject

So, for example, the isomorphism $\del: H_n(S^n)\ra H_{n-1}(S^{n-1})$ amounts to \u{choosing}
a generator $w$ for $H_{n-1}(S^{n-1})$ (inductively, essentially, a homeomorhism

$h:\del \Delta^n\ra S^{n-1}$ treated as a sum of singular simplices) and choosing
chains $w_+,w_-$ in $S^n_+$ and $S^n_-$ with boundaries equal to $h$ (essentially,
the obvious extensions of $h$ mapping $\Delta^n$ onto $S^n_\pm$); then 
$w_+-w_-$ is a cycle in in $C_n(S^n)$ and maps onto $w$ under $\del$. We leave
it to the interested observer to verify our inductive claim that $w_+-w_-$
is ``really'' a homeomorphism $\del\Delta^{n+1}\ra S^n$ represented as a sum of
singular simplices.



\vfill
\end