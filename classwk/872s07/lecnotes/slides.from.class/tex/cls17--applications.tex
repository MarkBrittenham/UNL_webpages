
\magnification=2000
\overfullrule=0pt
\parindent=0pt

\nopagenumbers

\input amstex

%\voffset=-.6in
%\hoffset=-.5in
%\hsize = 7.5 true in
%\vsize=10.4 true in

\voffset=1.8 true in
\hoffset=-.6 true in
\hsize = 10.2 true in
\vsize=8 true in

\input colordvi

\def\cltr{\Red}		  % Red  VERY-Approx PANTONE RED

\loadmsbm

\input epsf

\def\ctln{\centerline}
\def\u{\underbar}
\def\ssk{\smallskip}
\def\msk{\medskip}
\def\bsk{\bigskip}
\def\hsk{\hskip.1in}
\def\hhsk{\hskip.2in}
\def\dsl{\displaystyle}
\def\hskp{\hskip1.5in}

\def\lra{$\Leftrightarrow$ }
\def\ra{\rightarrow}
\def\mpto{\logmapsto}
\def\pu{\pi_1}
\def\mpu{$\pi_1$}
\def\sig{\Sigma}
\def\msig{$\Sigma$}
\def\ep{\epsilon}
\def\sset{\subseteq}
\def\del{\partial}
\def\inv{^{-1}}
\def\wtl{\widetilde}
%\def\lra{\Leftrightarrow}
\def\del{\partial}
\def\delp{\partial^\prime}
\def\delpp{\partial^{\prime\prime}}
\def\sgn{{\roman{sgn}}}
\def\wtih{\widetilde{H}}
\def\bbz{{\Bbb Z}}
\def\bbr{{\Bbb R}}
\def\rtar{$\Rightarrow$}

\def\cltr{\Red}		  % Red  VERY-Approx PANTONE RED
\def\cltb{\Blue}		  % Blue  Approximate PANTONE BLUE-072
\def\cltg{\PineGreen}	  % ForestGreen  Approximate PANTONE 349




{\bf Some applications:}

\ssk

Given a free group
$G=F(a_1,\ldots a_n)$ and a collection of words $w_1,\ldots w_m\in G$,
we can determine the rank and index of the subgroup it $H$ they
generate by building the corresponding cover. The idea is
to start with a bouquet of $m$ circles, each subdivided 
and labelled to spell
out the words $w_i$. Then we repeatedly identify edges sharing
on common vertex if they are labeled precisely the same (same
letter {\it and} same orientation). This process is known
as {\it folding}. 

\ssk

One can inductively show that the (obvious)
maps from these graphs to the bouquet of $n$ circles $X_n$ both
have image $H$ under the induced maps on \mpu ; since the map 
for the unfolded graph
factors through the one for the folded graph, the image from the
folded graph can only get smaller, but we can still spell out
the same words as loops in the folded graph, so the image can
also only have gotten bigger! We continue this folding process until there
is no more folding to be done; the resulting graph $X$ is what is 
known (in combinatorics) as a {\it graph covering}; the map to $X_n$
is locally injective. If this map is a covering map, then our subgroup
$H$ has finite index (equal to the degree of the
covering) and we can compute the rank of $H$ (and a basis!) from the 
folded graph. If it is not a covering map, then the map is not locally surjective at
some vertices; if we graft trees onto these vertices, we can extend the map
to an (infinite-sheeted) covering map without changing the homotopy
type of the graph. $H$ therefore has infinite index in $G$, and its
rank can be computed from $H\cong \pu(X)$. 

\vfill
\eject

Note that for a graph $\Gamma$ to be a covering of another graph, with $k$ sheets, say,
the number of vertices and edges of $\Gamma$ must both be a mulitple of $k$. This
little observation can be very useful when trying to decide what graphs $\Gamma$ might
cover!

\msk

{\it Kurosh Subgroup Theorem}: If $H < G_1*G_2$ is a subgroup of
a free product, then $H$ is (isomorphic to) a free product of a
collection of conjugates of subgroups of $G_1$ and $G_2$ and a 
free froup. The proof is to build a space by taking 2-complexes
$X_1$ and $X_2$ with $\pu$'s isomorphic to $G_1,G_2$ and join
their basepoints by an arc. The covering space of this space $X$
corresponding to $H$ consists of spaces that cover $X_1,X_2$
(giving, after basepoint considerations, the conjugates)
connected by a collection of arcs (which, suitably interpreted,
gives the free group).

\msk

{\it Residually finite groups}: $G$ is said to be residually finite if for every $g\neq 1$ there is a 
finite group $F$ and a homomorphism $\varphi: G\ra F$ with $\varphi(g)\neq 1$ in $F$. This 
amounts to saying that $g\notin$ the (normal) subgroup $\ker(\varphi)$, which amounts to
saying that a loop corresponding to $g$ does \underbar{not} lift to a loop in the finite-sheeted
covering space corresponding to $\ker(\varphi)$. So residual finiteness of a group can be
verified by building coverings of a space $X$ with $\pu(X)=G$. For example, free groups can be
shown to be residually finite in this way. 

\msk

{\it Ranks of free (sub)groups:} A free group on $n$
generators is isomorphic to a free group on $m$ generators
\lra\ $n=m$; this is because the abelianizations of the two 
groups are ${\Bbb Z}^n,{\Bbb Z}^m$. The (minimal) number of 
generators for a free group is called its {\it rank}.


\vfill
\eject


{\bf Postscript: why care about covering spaces?} The preceding discussion
probably makes it clear that covering places play a central role in
(combinatorial) group theory. It also plays a role in embedding 
problems; a common scenario is to have a map $f:Y\ra X$ which is 
injective on \mpu , and we wish to know if we can lift $f$ to a 
finite-sheeted covering so that the lifted map $\widetilde{f}$ is 
homotopic to an embedding. Information that is easier to obtain 
in the case of an embedding can then be passed down to gain information
abut the original map $f$. And covering spaces underlie the 
theory of analytic continuation in complex analysis; starting
with a domain $D\subseteq {\Bbb C}$, what analytic continuation really
builds is an (analytic) function from a covering space of $D$ to ${\Bbb C}$.
For example, the logarithm is really defined as a map from 
the universal cover of ${\Bbb C}\setminus\{0\}$ to ${\Bbb C}$. 
The various ``branches'' of the logarithm refer to which sheet
in this cover you are in.




\vfill
\end



Given words $w_1,\ldots,w_n\in F(x_1,\ldots x_m)$, we can build the covering space correpsonding
to the subgroup $H=\langle w_1,\ldots,w_n\rangle$ by a process of {\it folding}, in so doing 
determining the index of $H$ and a basis for $H$ as a free group. 

\msk

The idea is to build a 
covering $\widetilde{X}$ of the bouquet $X_m$ of $m$ circles, 
the image of whose fundamental group 
is $H$. Start with a bouquet $Y$ of $n$ circles, each subdivided and (orientedly) 
labeled to spell out the words 
$w_i$. This is a 1-complex; the labeling tells us how to \underbar{map}
$Y$ to $X_m$. Then inductively, we fold together any two edges at a vertex with the same oriented edge,
since they are supposed to be mapping together in $X_m$, and that mapping will 
\underbar{not} give a local homeo!
Note two things: folding is (almost) a homotopy equivalence, 
and the original words still always spell out loops in the intermediate folded spaces.

\ssk

Stop when you run out of folds. The ``obvious'' map from the resulting space to $X_m$ is locally
injective, otherwise we have another fold to do.
One of two things will occur at the end; either the map is everywhere a local homeo, and so is a covering
map, or there are points where it is not locally surjective. In the first case, we have succeeded
in building a finite covering $\widetilde{X}$ with 
(since the $w_i$ still generate the fundamental group) fundamental group having image $H$, and we can 
read off the index of and a basis for $H$ from the covering. In the second case, we can
\underbar{extend} our space $\widetilde{X}$ to a covering by grafting on (infinite) trees, so $H$ has
infinite index; since the grafted space deformation retracts to $\widetilde{X}$,
we can still read off a basis for $H$ by the same process.
