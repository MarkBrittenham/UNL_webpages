
\magnification=2000
\overfullrule=0pt
\parindent=0pt

\nopagenumbers

\input amstex

%\voffset=-.6in
%\hoffset=-.5in
%\hsize = 7.5 true in
%\vsize=10.4 true in

\voffset=1.8 true in
\hoffset=-.6 true in
\hsize = 10.2 true in
\vsize=8 true in

\input colordvi

\def\cltr{\Red}		  % Red  VERY-Approx PANTONE RED

\loadmsbm

\input epsf

\def\ctln{\centerline}
\def\u{\underbar}
\def\ssk{\smallskip}
\def\msk{\medskip}
\def\bsk{\bigskip}
\def\hsk{\hskip.1in}
\def\hhsk{\hskip.2in}
\def\dsl{\displaystyle}
\def\hskp{\hskip1.5in}

\def\lra{$\Leftrightarrow$ }
\def\ra{\rightarrow}
\def\mpto{\logmapsto}
\def\pu{\pi_1}
\def\mpu{$\pi_1$}
\def\sig{\Sigma}
\def\msig{$\Sigma$}
\def\ep{\epsilon}
\def\sset{\subseteq}
\def\del{\partial}
\def\inv{^{-1}}
\def\wtl{\widetilde}
%\def\lra{\Leftrightarrow}
\def\del{\partial}
\def\delp{\partial^\prime}
\def\delpp{\partial^{\prime\prime}}
\def\sgn{{\roman{sgn}}}
\def\wtih{\widetilde{H}}
\def\bbz{{\Bbb Z}}
\def\bbr{{\Bbb R}}
\def\rtar{$\Rightarrow$}

\def\cltr{\Red}		  % Red  VERY-Approx PANTONE RED
\def\cltb{\Blue}		  % Blue  Approximate PANTONE BLUE-072
\def\cltg{\PineGreen}	  % ForestGreen  Approximate PANTONE 349


{\bf Universal covering spaces}: A particularly
important covering space  of a space $X$ to identify 
is one which is simply connected. Such a covering 
is essentially unique:

\msk

If $X$ is {\it locally path-connected} and has two connected, simply connected
covering spaces $p_1:X_1\ra X$ and $p_2:X_2\ra X$, then since 
$p_{1*}(\pu(X_1,x_1)) = p_{2*}(\pu(X_2,x_2))=\{1\}\sset \pu(X,x_0)$,
the lifting criterion applied twice
gives maps $\wtl{p}_1:(X_1,x_1)\ra (X_2,x_2)$ and 
$\wtl{p}_2:(X_2,x_2)\ra (X_1,x_1)$ with $p_2\circ\wtl{p}_1=p_1$
and $p_1\circ\wtl{p}_2=p_2$. Consequently, 
$p_2\circ\wtl{p}_1\circ \wtl{p}_2 = p_1\circ\wtl{p}_2=p_2$
and similarly, 
$p_1\circ\wtl{p}_2\circ \wtl{p}_1 =p_2\circ\wtl{p}_1=p_1$.
So $\wtl{p}_1\circ \wtl{p}_2:(X_2,x_2)\ra (X_2,x_2)$, for example,
is a lift of $p_2$ to the covering map $p_2$. But so is the identity map! By
uniqueness, therefore, $\wtl{p}_1\circ \wtl{p}_2=Id$ . Similarly,
$\wtl{p}_2\circ \wtl{p}_1=Id$. So $(X_1,x_1)$ and $(X_2,x_2)$ 
are homeomorphic. (More: there is a homeo interpolating between the
covering maps.) So up to homeomorphism, a space has
only one connected, simply-connected covering space. It is known
as the {\it universal covering} of the space $X$. 

\msk

Not every (locall path-connected) space $X$ has a universal covering; a 
(further) necessary condition is that $X$ be \cltr{{\it semi-locally simply connected}}.
The idea is that If $p:\wtl{X}\ra X$ is the universal cover, then for every 
point $x\in X$, we have an evenly-covered neighborhood ${\Cal U}$ of $x$.
The inclusion $i:{\Cal U}\ra X$, by definition, lifts to $\wtl{X}$, so
$i_*(\pu({\Cal U},x))\sset p_*(\pu(\wtl{X},\wtl{x}) = \{1\}$, so
$i_*$ is the trivial map. Consequently, every loop in ${\Cal U}$ is 
null-homotopic in $X$. This is \cltr{semi-local simple connectivity;
every point has a neighborhood whose inclusion-induced homomorphism
is trivial.} Not all spaces have this property; the most famous is the 
Hawaiian earrings 
$\displaystyle X=\bigcup_{n}\{x\in {\Bbb R}^2 :  ||x-(1/n,0)||=1/n\}$ .
The point $(0,0)$ has no such neighborhood. 

\vfill
\eject


{\bf Building universal coverings:} \cltr{If a space $X$ is path connected, locally path connected, and
semi-locally simply connected (S-LSC), then it has a universal covering.}

\ssk

The idea is that a covering
space has the path lifting and homotopy
lifting properties, and the universal 
cover is the only covering space for 
which {\it only} null-homotopic loops lift to loops. \underbar{So} we build a 
space and a map which \underbar{must} have these properties.
We do this by making a space $\widetilde{X}$ whose
points are (equivalence classes $[\gamma]$ of)
based paths $\gamma:(I,0)\ra (X,x_0)$, where two paths are equivalent
if they are homotopic rel endpoints! The projection map is
$p([\gamma])=\gamma(1)$. 

\ssk

The S-LSCness of $X$ guarantees that this is a 
covering map; choosing $x\in X$, a path $\gamma_0$ from $x_0$ to $x$,
and a nbhd ${\Cal U}$ of $x$ guaranteed by S-LSC, a path $\theta$ from 
$x_0$ to points in ${\Cal U}$ is homotopic to $\gamma*\gamma_0*\eta$
where $\gamma$ is a loop at $x_0$ and $\eta$ is a path in ${\Cal U}$.
[$\gamma=\theta*\overline{\eta}*\overline{\gamma_0}$.]
But by S-LSC, a path in ${\Cal U}$ is determined up to homotopy
by its endpoints, and so, with $\gamma$ fixed, these paths are in one-to-one
correspondence with ${\Cal U}$. So $p\inv({\Cal U})$ is a disjoint union,
indexed by $\pu(X,x_0)$, of sets that are in 1-to-1 corresp with ${\Cal U}$.

\ssk


The appropriate topology on $\widetilde{X}$ is essentially given as a basis
by triples $(\gamma,\gamma_0,{\Cal U})$ as above. This topology makes $p$ a covering map.
Note that the inverse image of 
the basepoint $x_0$ is the equivalence classes of \underbar{loops} at $x_0$,
i.e., $\pu(X,x_0)$. A path $\gamma$ lifts to the path of paths
$\gamma_t$, where $\gamma_t(s)=\gamma(ts)$, and so the only 
loops in $X$ which lift to a loop in $\widetilde{X}$ have
$[\gamma_1]=[\gamma]=[c_{x_0}]$, i.e., $[\gamma]=1$ in $\pu(X,x_0)$. This
implies that $p_*(\pu(\widetilde{X},[c_{x_0}]))=\{1\}$, so 
$\pu(\widetilde{X},[c_{x_0}])=\{1\}$ . 

\msk

However, nobody in their
right minds would go about building $\widetilde{X}$ in this way!


\vfill
\eject

{\bf Why Care?} The universal cover gives
a unified approach to building \underbar{all} connected covering
spaces of $X$. The key to this is the {\it deck transformation group
(Deckbewegungsgruppe)}
of a covering space $p:\wtl{X}\ra X$; this is \cltr{the set of all
homeomorphisms $h:\wtl{X}\ra\wtl{X}$ such that $p\circ h = p$.}

\ssk

By def'n, these $h$ permute each of the pt inverses
of $p$. Since $h$ is a lift of the projection map
$p$, by the lifting criterion $h$ is det'd by which point in $p^{-1}(x_0)$ 
it takes the basepoint 
$\wtl{x}_0$ of $\wtl{X}$ to. A deck transformation sending
$\wtl{x}_0$ to $\wtl{x}_1$ exists $\Leftrightarrow$
$p_*(\pu(\wtl{X},\wtl{x}_0)=p_*(\pu(\wtl{X},\wtl{x}_1)$
[we need one inclusion to give $h$, and the opposite inclusion
to ensure it is a bijection]. 

\msk

These two groups
are {\it conjugate}, by the projection of a path from 
$\wtl{x}_0$ to $\wtl{x}_1$ (follow the change
of basept iso down into $G=\pu(X,x_0)$). 
Paths in $\wtl{X}$ from $\wtl{x}_0$ to $\wtl{x}_1$ are in 1-to-1
corresp with the cosets of $H=p_*(\pu(\wtl{X},\wtl{x}_0)$ in 
$p_*(\pu({X},{x}_0)$; so deck transformations are in 1-to-1 
corresp with cosets whose representatives conjugate 
$H$ to itself. The set of such elements in $G$ is called the 
{\it normalizer of $H$ in $G$}, and denoted $N_G(H)$ or simply
$N(H)$. The deck transformation group is therefore
in 1-to-1 correspondence with the group $N(H)/H$ under
$h\mapsto$ the coset with representative the projection of the path from 
$\wtl{x}_0$ to $h(\wtl{x}_0)$. And since the lift $h$ is essentially built
by lifting paths, it follows quickly that this map is a
homomorphism, hence an isomorphism.

\vfill
\eject

Applying this to the universal covering space
$p:\wtl{X}\ra X$, in this case $H=\{1\}$, so $N(H)=\pu(X,x_0)$.
So the deck transformation group is isomorphic to $\pu(X,x_0)$. 
For example, this gives the quickest possible proof 
that $\pu(S^1)\cong {\Bbb Z}$, since ${\Bbb R}$ is a 
contractible covering space, whose deck transformations
are the translations by integer distances. 

\ssk

Thus $\pu(X)$ acts on its universal cover as a group of
homeomorphisms. And since this action is {\it simply transitive}
on point inverses [there is exactly one (that's the simple
part) deck transformation carrying any one point in a point 
inverse to any other one (that's the transitive part)], the 
quotient map from $\wtl{X}$ to the orbits of this action \underbar{is}
the projection map $p$. The evenly covered property of $p$ implies
that $X$ does have the quotient topology under this action.

\msk

So \cltr{every space it $X$ the quotient of its universal cover} (if it has
one!) \cltr{by its fundamental group $G=\pu(X,x_0)$, acting as the group
of deck transformations.} And the quotient map is the covering 
projection. 
So $X\cong \wtl{X}/G$ . 

\ssk

In general, a quotient of a 
space $Z$ by a group action $G$ 
need not be 
a covering map. The action must be {\it properly discontinuous}: 
for every point 
$z\in Z$, there is a neighborhood ${\Cal U}$ of $x$ so that $g\neq 1$ $\Rightarrow$
${\Cal U}\cap g{\Cal U}=\emptyset$. The evenly covered neighborhoods
provide these for the universal cover. And conversely, the quotient of a space by a 
p.d. group action is a covering space. 

\vfill
\eject

But! Given $G=\pu(X,x_0)$ and its 
action on a univ cover $\wtl{X}$, we can, instead of modding out by $G$,
mod out by any \underbar{subgroup} $H$ of $G$, to build $X_H=\wtl{X}/H$. 
This is a space with $\pi_1(X_H)\cong H$, having $\wtl{X}$ as univ covering.
And since the quotient (covering) map $p_G:\wtl{X}\ra X=\wtl{X}/G$ factors through $\wtl{X}/H$,
we have an induced map $p_H:\wtl{X}/H\ra X$, which is a covering map; open sets with
trivial inclusion-induced homomorphism lift homeomorphically to $\wtl{X}$,
hence homeomorphically to $\wtl{X}/H$; choosing lifts to each point inverse of $x\in X$
builds the evenly covering nbhds for $p_H$ . So every subgroup of $G$ is the
fundamental group of a covering of $X$. 

\msk

{\bf The Galois correspondence:} Two coverings
$p_1:X_1\ra X$ , $p_2:X_2\ra X$ are {\it isomorphic} if there is a homeo
$h:X_1\ra X_2$ with $p_1=p_2\circ h$. Choosing basepts $x_1,x_2$ mapping to $x_0\in X$,
then if $h(x_1)=x_2$, then $p_{1*}(\pu(X_1,x_1)) = p_{2*}(h_*(\pu(X_1,x_1))) = 
p_{2*}(\pu(X_2,x_2))$ . If instead $h(x_1)=x_3$, then $p_{1*}(\pu(X_1,x_1)) = p_{2*}(\pu(X_2,x_3))$.
But  $\pu(X_2,x_2)$ and $\pu(X_2,x_3)$ are isomorphic, via a change 
of basept isomorphism $\widehat{\eta}$ , where $\eta$ is a path in $X_2$ from $x_2$ to $x_3$.
Such a path projects to $X$ as a loop at $x_0$, and since the change of basept isom
is by ``conjugating'' by the path $\eta$, the resulting groups 
$p_{2*}(\pu(X_2,x_2))$ and $p_{2*}(\pu(X_2,x_3))$
are conjugate, by $[p_2\circ \eta]$ . 

\ssk

So choosing \underbar{any} basepts over $x_0$, isomorphic coverings give,
under projection, conjugate subgroups of $\pu(X,x_0)$ . But conversely, given covering spaces
$X_1,X_2$ whose subgroups $p_{1*}(\pu(X_1,x_1))$ and $p_{2*}(\pu(X_2,x_2))$ are conjugate,
lifting a loop $\gamma$ representing the conjugating element to a loop $\wtl{\gamma}$ in
$X_2$ starting at $x_2$ gives, as its terminal endpoint, a point $x_3$ with 
$p_{1*}(\pu(X_1,x_1)) = p_{2*}(\pu(X_2,x_3))$ (since it conjugates back!), and so, by the lifting criterion,
there is an isomorphism $h:(X_1,x_1) \ra (X_2,x_3)$. So conjugate subgroups give isomorphic coverings.
Thus: 

\vfill
\eject

\cltr{{\bf The Galois correspondence:} For a path-connected, locally path-connected, semi-locally simply-connected space $X$,  
the image of the induced homomorphism on \mpu\ 
gives a one-to-one correspondence between 
[isomorphism classes of (connected) coverings of $X$] and 
[conjugacy classes of subgroups of $\pu(X)$].}

\msk

So, for example, if you have a group $G$ that you are interested in, you know of a (nice enough) 
space $X$ with $\pu(X)\cong G$, and you know enough about the coverings of $X$, then you can
gain information about the subgroup structure of $G$. 

\msk

For example, a free group $F(\Sigma)$ is \mpu\ of a bouquet of circles $X$. 
Any covering space $\wtl{X}$ of $X$ is a union of vertices and edges, so is a graph.  Collapsing
a maximal tree to a point, $\wtl{X}$ is $\sim$ a bouquet of circles, so has free \mpu . So every
subgroup of a free group is free. 
A subgroup $H$ of index $n$ in $F(\Sigma)$ corresponds to a $n$-sheeted covering $\wtl{X}$ of $X$. If
$|\Sigma| = m$, then $\wtl{X}$ will have $n$ vertices and $nm$ edges. Collapsing a maximal
tree, having $n-1$ edges, to a point, leaves a bouquet of $nm-n+1$ circles, so $H\cong F(nm-n+1)$.
For example, for $m=3$, index $n$ subgroups are free on $2n+1$ generators, so every free subgroup
on 4 generators has infinite index in $F(3)$. [Try proving that directly!]

\vfill
\eject


Given a free group
$G=F(a_1,\ldots a_n)$ and a collection of words $w_1,\ldots w_m\in G$,
we can determine the rank and ndex of the subgroup it $H$ they
generate by building the corresponding cover. The idea is
to start with a bouquet of $m$ circles, each subdivided 
and labelled to spell
out the words $w_i$. Then we repeatedly identify edges sharing
on common vertex if they are labelled precisely the same (same
letter {\it and} same orientation). This process is known
as {\it folding}. One can inductively show that the (obvious)
maps from these graphs to the bouquet of $n$ circles $X_n$ both
have image $H$ under the induced maps on \mpu ; since the map 
for the unfolded graph
factors through the one for the folded graph, the image from the
folded graph can only get smaller, but we can still spell out
the same words as loops in the folded graph, so the image can
also only have gotten bigger! We continue this folding process until there
is no more folding to be done; the resulting graph $X$ is what is 
known (in combinatorics) as a {\it graph covering}; the map to $X_n$
is locally injective. If this map is a covering map, then our subgroup
$H$ has finite index (equal to the degree of the
covering) and we can compute the rank of $H$ (and a basis!) from the 
folded graph. If it is not a covering map, then the map is not locally surjective at
some vertices; if we graft trees onto these vertices, we can extend the map
to an (infinite-sheeted) covering map without changing the homotopy
type of the graph. $H$ therefore has infinite index in $G$, and its
rank can be computed from $H\cong \pu(X)$. 

\vfill
\eject

Given words $w_1,\ldots,w_n\in F(x_1,\ldots x_m)$, we can build the covering space correpsonding
to the subgroup $H=\langle w_1,\ldots,w_n\rangle$ by a process of {\it folding}, in so doing 
determining the index of $H$ and a basis for $H$ as a free group. 

\msk

The idea is to build a 
covering $\widetilde{X}$ of the bouquet $X_m$ of $m$ circles, 
the image of whose fundamental group 
is $H$. Start with a bouquet $Y$ of $n$ circles, each subdivided and (orientedly) 
labeled to spell out the words 
$w_i$. This is a 1-complex; the labeling tells us how to \underbar{map}
$Y$ to $X_m$. Then inductively, we fold together any two edges at a vertex with the same oriented edge,
since they are supposed to be mapping together in $X_m$, and that mapping will 
\underbar{not} give a local homeo!
Note two things: folding is (almost) a homotopy equivalence, 
and the original words still always spell out loops in the intermediate folded spaces.

\ssk

Stop when you run out of folds. The ``obvious'' map from the resulting space to $X_m$ is locally
injective, otherwise we have another fold to do.
One of two things will occur at the end; either the map is everywhere a local homeo, and so is a covering
map, or there are points where it is not locally surjective. In the first case, we have succeeded
in building a finite covering $\widetilde{X}$ with 
(since the $w_i$ still generate the fundamental group) fundamental group having image $H$, and we can 
read off the index of and a basis for $H$ from the covering. In the second case, we can
\underbar{extend} our space $\widetilde{X}$ to a covering by grafting on (infinite) trees, so $H$ has
infinite index; since the grafted space deformation retracts to $\widetilde{X}$,
we can still read off a basis for $H$ by the same process.

\vfill
\end

Note that for a graph $\Gamma$ to be a covering of another graph, with $k$ sheets, say,
the number of vertices and edges of $\Gamma$ must both be a mulitple of $k$. This
little observation can be very useful when trying to decide what graphs $\Gamma$ might
cover!

\msk

{\it Kurosh Subgroup Theorem}: If $H < G_1*G_2$ is a subgroup of
a free product, then $H$ is (isomorphic to) a free product of a
collection of conjugates of subgroups of $G_1$ and $G_2$ and a 
free froup. The proof is to build a space by taking 2-complexes
$X_1$ and $X_2$ with $\pu$'s isomorphic to $G_1,G_2$ and join
their basepoints by an arc. The covering space of this space $X$
corresponding to $H$ consists of spaces that cover $X_1,X_2$
(giving, after basepoint considerations, the conjugates)
connected by a collection of arcs (which, suitably interpreted,
gives the free group).

\msk

{\it Residually finite groups}: $G$ is said to be residually finite if for every $g\neq 1$ there is a 
finite group $F$ and a homomorphism $\varphi: G\ra F$ with $\varphi(g)\neq 1$ in $F$. This 
amounts to saying that $g\notin$ the (normal) subgroup $\ker(\varphi)$, which amounts to
saying that a loop corresponding to $g$ does \underbar{not} lift to a loop in the finite-sheeted
covering space corresponding to $\ker(\varphi)$. So residual finiteness of a group can be
verified by building coverings of a space $X$ with $\pu(X)=G$. For example, free groups can be
shown to be residually finite in this way. 

\msk

{\it Ranks of free (sub)groups:} A free group on $n$
generators is isomorphic to a free group on $m$ generators
\lra\ $n=m$; this is because the abelianizations of the two 
groups are ${\Bbb Z}^n,{\Bbb Z}^m$. The (minimal) number of 
generators for a free group is called its {\it rank}.


\bsk


{\bf Postscript: why care about covering spaces?} The preceding discussion
probably makes it clear that covering places play a central role in
(combinatorial) group theory. It also plays a role in embedding 
problems; a common scenario is to have a map $f:Y\ra X$ which is 
injective on \mpu , and we wish to know if we can lift $f$ to a 
finite-sheeted covering so that the lifted map $\widetilde{f}$ is 
homotopic to an embedding. Information that is easier to obtain 
in the case of an embedding can then be passed down to gain information
abut the original map $f$. And covering spaces underlie the 
theory of analytic continuation in complex analysis; starting
with a domain $D\subseteq {\Bbb C}$, what analytic continuation really
builds is an (analytic) function from a covering space of $D$ to ${\Bbb C}$.
For example, the logarithm is really defined as a map from 
the universal cover of ${\Bbb C}\setminus\{0\}$ to ${\Bbb C}$. 
The various ``branches'' of the logarithm refer to which sheet
in this cover you are in.



\vfill
\end