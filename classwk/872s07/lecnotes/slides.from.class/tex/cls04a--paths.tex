
\magnification=1900
\overfullrule=0pt
\parindent=0pt

\nopagenumbers

\input amstex

%\voffset=-.6in
%\hoffset=-.5in
%\hsize = 7.5 true in
%\vsize=10.4 true in

\voffset=1.8 true in
\hoffset=-.6 true in
\hsize = 10.2 true in
\vsize=8 true in

\input colordvi

\def\cltr{\Red}		  % Red  VERY-Approx PANTONE RED

\loadmsbm

\input epsf

\def\ctln{\centerline}
\def\u{\underbar}
\def\ssk{\smallskip}
\def\msk{\medskip}
\def\bsk{\bigskip}
\def\hsk{\hskip.1in}
\def\hhsk{\hskip.2in}
\def\dsl{\displaystyle}
\def\hskp{\hskip1.5in}

\def\lra{$\Leftrightarrow$ }
\def\ra{\rightarrow}
\def\mpto{\logmapsto}
\def\pu{\pi_1}
\def\mpu{$\pi_1$}
\def\sig{\Sigma}
\def\msig{$\Sigma$}
\def\ep{\epsilon}
\def\sset{\subseteq}
\def\del{\partial}
\def\inv{^{-1}}
\def\wtl{\widetilde}
%\def\lra{\Leftrightarrow}
\def\del{\partial}
\def\delp{\partial^\prime}
\def\delpp{\partial^{\prime\prime}}
\def\sgn{{\roman{sgn}}}
\def\wtih{\widetilde{H}}
\def\bbz{{\Bbb Z}}
\def\bbr{{\Bbb R}}
\def\rtar{$\Rightarrow$}

{\bf Straight-line homotopies:}

\msk

Any two maps $f,g:X\ra \bbr^n$ are homotopic:

\msk

$H(x,t)=(1-t)f(x)+tg(x)$

\msk

is continuous, since ``it is constructed out of continuous functions'' in a manner which we
know preserves continuity.

\ssk

Note that if $f(x_0)=g(x_0)=y_0$, then $H(x_0,t)=(1-t)y_0+ty_0=y_0$.
So the homotopy is relative to the set $A=\{x\in X: f(x)=g(x)\}$.

\msk

So, for example, any two paths $\alpha,\beta:I\ra \bbr^n$ in $\bbr^n$ (or any convex subset of $\bbr^n$)
between the same endpoints are homotopic rel endpoints.

\ssk

The same (by composing with a homeomorphism) is true of any space homeomorphic to (a convex subset
of) $\bbr^n$.

\bsk

Grafting a ``tail'' onto a path also does not change it's homotopy class:

\ssk

Since $\gamma*\overline{\gamma}$ is homotopic, rel endpoints, to a constant map $c_0$,

\ssk

\hskip.2in $\alpha*\beta\simeq \alpha*c_0*\beta\simeq\alpha*\gamma*\overline{\gamma}*\beta
\simeq(\alpha*\gamma)*(\overline{\gamma}*\beta)$ .

\vfill
\end