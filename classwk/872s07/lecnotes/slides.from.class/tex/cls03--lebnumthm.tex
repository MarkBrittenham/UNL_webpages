
\magnification=2000
\overfullrule=0pt
\parindent=0pt

\nopagenumbers

\input amstex

%\voffset=-.6in
%\hoffset=-.5in
%\hsize = 7.5 true in
%\vsize=10.4 true in

\voffset=2 true in
\hoffset=-.6 true in
\hsize = 10.2 true in
\vsize=8 true in

\input colordvi

\loadmsbm

\input epsf

\def\ctln{\centerline}
\def\u{\underbar}
\def\ssk{\smallskip}
\def\msk{\medskip}
\def\bsk{\bigskip}
\def\hsk{\hskip.1in}
\def\hhsk{\hskip.2in}
\def\dsl{\displaystyle}
\def\hskp{\hskip1.5in}

\def\lra{$\Leftrightarrow$ }
\def\ra{\rightarrow}
\def\mpto{\logmapsto}
\def\pu{\pi_1}
\def\mpu{$\pi_1$}
\def\sig{\Sigma}
\def\msig{$\Sigma$}
\def\ep{\epsilon}
\def\sset{\subseteq}
\def\del{\partial}
\def\inv{^{-1}}
\def\wtl{\widetilde}
%\def\lra{\Leftrightarrow}
\def\del{\partial}
\def\delp{\partial^\prime}
\def\delpp{\partial^{\prime\prime}}
\def\sgn{{\roman{sgn}}}
\def\wtih{\widetilde{H}}
\def\bbz{{\Bbb Z}}
\def\bbr{{\Bbb R}}

{\bf The Lebesgue Number Theorem:}

\msk

If $(X,d)$ is a compact metric space
(in our applications, it is always a compact subset of Euclidean space), and 
$\{{\Cal U}_i\}$ is an open cover of $X$, then there is an $\epsilon>0$ so that
for every $x\in X$, its $\epsilon$-neighborhood $N_d(x,\epsilon)$ is contained
in ${\Cal U}_i$ for some $i$. 

\msk


{\bf Proof:} If not, then for every $n\in{\Bbb N}$ there is an $x_n\in X$ 
whose $1/n$-neighborhood is contained in no ${\Cal U}_i$; that is, for 
every $i\in I$, there is an $x_{n,i}$ with
$d(x_n,x_{n,i})<1/n$ and $x_{n,i}\notin{\Cal U}_i$, so $x_{n,i}\in C_i=X\setminus {\Cal U}_i$, 
a closed set. 
But since $X$ is compact, there is a convergent subsequence of the $x_n$; $x_{n_k}\ra y\in X$.

\msk

[Proof: if not, then no point is the limit of a subsequence, so for every $x\in X$ there is
an $\epsilon(x)>0$ and an $N=N(x)$ so that $n\geq N$ implies $x_n\notin N_d(x,\epsilon(x))$.
But these neighborhoods cover $X$, so a finite number of them do; for any $n$ greater than
the maximum of the associated $N(x)$'s $x_n$ lies in none of the neighborhoods, a contradiction,
since $x_n\in X=$ the union of these neighborhoods.]

\msk

But then since $d(x_{n_k},y)\ra 0$ and $d(x_{n_k},x_{n_k,i})\ra 0$, for every $i$
the $x_{n_k,i}$ also converge to $y$; since the $x_{n_k,i}$ all lie in the closed
set $C_i$, so does $y$. So $y\in C_i$ for all $i$, so $y\notin {\Cal U}_i$ for all $i$,
a contradiction, since the ${\Cal U}_i$ cover $X$. So some $\epsilon>0$, a 
{\it Lebesgue number} for the covering, must exist.


\vfill
\end