
\magnification=2000
\overfullrule=0pt
\parindent=0pt

\nopagenumbers

\input amstex

%\voffset=-.6in
%\hoffset=-.5in
%\hsize = 7.5 true in
%\vsize=10.4 true in

\voffset=1.8 true in
\hoffset=-.6 true in
\hsize = 10.2 true in
\vsize=8 true in

\input colordvi

\def\cltr{\Red}		  % Red  VERY-Approx PANTONE RED

\loadmsbm

\input epsf

\def\ctln{\centerline}
\def\u{\underbar}
\def\ssk{\smallskip}
\def\msk{\medskip}
\def\bsk{\bigskip}
\def\hsk{\hskip.1in}
\def\hhsk{\hskip.2in}
\def\dsl{\displaystyle}
\def\hskp{\hskip1.5in}

\def\lra{$\Leftrightarrow$ }
\def\ra{\rightarrow}
\def\mpto{\logmapsto}
\def\pu{\pi_1}
\def\mpu{$\pi_1$}
\def\sig{\Sigma}
\def\msig{$\Sigma$}
\def\ep{\epsilon}
\def\sset{\subseteq}
\def\del{\partial}
\def\inv{^{-1}}
\def\wtl{\widetilde}
%\def\lra{\Leftrightarrow}
\def\del{\partial}
\def\delp{\partial^\prime}
\def\delpp{\partial^{\prime\prime}}
\def\sgn{{\roman{sgn}}}
\def\wtih{\widetilde{H}}
\def\bbz{{\Bbb Z}}
\def\bbr{{\Bbb R}}
\def\rtar{$\Rightarrow$}

\def\cltr{\Red}		  % Red  VERY-Approx PANTONE RED
\def\cltb{\Blue}		  % Blue  Approximate PANTONE BLUE-072
\def\cltg{\PineGreen}	  % ForestGreen  Approximate PANTONE 349



{\bf Covering spaces:}

\msk


The projective plane ${\Bbb R}P^2$
has $\pu = {\Bbb Z}_2$ . It is also the quotient of the simply-connected
space $S^2$ by the antipodal map, which, together with the identity map,
forms a group of homeomorphisms of $S^2$ which is isomorphic to ${\Bbb Z}_2$.
The fact that ${\Bbb Z}_2$ has this dual role to play in describing 
${\Bbb R}P^2$ is no accident; codifying this relationship requires the 
notion of a covering space.

\msk

The quotient map $q:S^2\ra {\Bbb R}P^2$ is an example of a {\it covering map}.
A map $p:E\ra B$ is called a covering map if for every point $x\in B$, there
is a neighborhood ${\Cal U}$ of $x$ (an
{\it evenly covered neighborhood}) so that $p^{-1}({\Cal U})$ 
is a disjoint union ${\Cal U}_\alpha$ of open sets in $E$, each mapped
homeomorphically onto ${\Cal U}$ by (the restriction of) $p$ . $B$ is
called the {\it base space} of the covering; $E$ is called the {\it total
space}. 

\msk

The quotient map $q$ is an example; (the image of) the complement
of a great circle in $S^2$ will be an evenly covered neighborhood
of any point it contains. 

\msk

The disjoint union of 42 copies of a space,
each mapping homeomorphically to a single copy, is an example of a 
{\it trivial covering}. 

\msk

The famous 
exponential map $p:{\Bbb R}\ra S^1$ given by $t\mapsto e^{2\pi it} = 
(\cos (2\pi t),\sin (2\pi t))$. The image $J\subseteq S^1$ of any interval 
$(a,b)$ of length
less than 1 will have inverse image the disjoint union of the
intervals $(a+n,b+n)$ for $n\in{\Bbb Z}$ .

\vfill
\eject

We can build many finite-sheeted (every point
inverse is finite) coverings of a bouquet of two circles, by 
assembling $n$ points over the vertex, and then, on either side (the red/blue
sides?),
connecting the points by $n$ (oriented) arcs, one with one red/blue arcs going
in/out of
each vertex. By choosing orientations on each 1-cell of the bouquet,
we can build a covering map by sending the vertices above to the
vertex, and the arcs to the one cells, homeomorphically, respecting 
the orientations. We can build infinite-sheeted coverings in much 
the same way.


\msk

\leavevmode


\epsfxsize=3in
\ctln{{\epsfbox{cov1.ai}}}


\msk

Covering spaces of more ``interesting'' graphs can be assembled similarly.

\msk

\leavevmode


\epsfxsize=3in
\ctln{{\epsfbox{cov2.ai}}}


\msk


\vfill
\end







