



\magnification=2000
\overfullrule=0pt
\parindent=0pt

\nopagenumbers

\input amstex

%\voffset=-.6in
%\hoffset=-.5in
%\hsize = 7.5 true in
%\vsize=10.4 true in

\voffset=1.8 true in
\hoffset=-.6 true in
\hsize = 10.2 true in
\vsize=8 true in

\input colordvi



\def\cltr{\Red}		  % Red  VERY-Approx PANTONE RED
\def\cltb{\Blue}		  % Blue  Approximate PANTONE BLUE-072
\def\cltg{\PineGreen}	  % ForestGreen  Approximate PANTONE 349
\def\cltp{\DarkOrchid}	  % DarkOrchid  No PANTONE match
\def\clto{\Orange}	  % Orange  Approximate PANTONE ORANGE-021
\def\cltpk{\CarnationPink}	  % CarnationPink  Approximate PANTONE 218
\def\clts{\Salmon}	  % Salmon  Approximate PANTONE 183
\def\cltbb{\TealBlue}	  % TealBlue  Approximate PANTONE 3145
\def\cltrp{\RoyalPurple}	  % RoyalPurple  Approximate PANTONE 267
\def\cltp{\Purple}	  % Purple  Approximate PANTONE PURPLE

\def\cgy{\GreenYellow}     % GreenYellow  Approximate PANTONE 388
\def\cyy{\Yellow}	  % Yellow  Approximate PANTONE YELLOW
\def\cgo{\Goldenrod}	  % Goldenrod  Approximate PANTONE 109
\def\cda{\Dandelion}	  % Dandelion  Approximate PANTONE 123
\def\capr{\Apricot}	  % Apricot  Approximate PANTONE 1565
\def\cpe{\Peach}		  % Peach  Approximate PANTONE 164
\def\cme{\Melon}		  % Melon  Approximate PANTONE 177
\def\cyo{\YellowOrange}	  % YellowOrange  Approximate PANTONE 130
\def\coo{\Orange}	  % Orange  Approximate PANTONE ORANGE-021
\def\cbo{\BurntOrange}	  % BurntOrange  Approximate PANTONE 388
\def\cbs{\Bittersweet}	  % Bittersweet  Approximate PANTONE 167
%\def\creo{\RedOrange}	  % RedOrange  Approximate PANTONE 179
\def\cma{\Mahogany}	  % Mahogany  Approximate PANTONE 484
\def\cmr{\Maroon}	  % Maroon  Approximate PANTONE 201
\def\cbr{\BrickRed}	  % BrickRed  Approximate PANTONE 1805
\def\crr{\Red}		  % Red  VERY-Approx PANTONE RED
\def\cor{\OrangeRed}	  % OrangeRed  No PANTONE match
\def\paru{\RubineRed}	  % RubineRed  Approximate PANTONE RUBINE-RED
\def\cwi{\WildStrawberry}  % WildStrawberry  Approximate PANTONE 206
\def\csa{\Salmon}	  % Salmon  Approximate PANTONE 183
\def\ccp{\CarnationPink}	  % CarnationPink  Approximate PANTONE 218
\def\cmag{\Magenta}	  % Magenta  Approximate PANTONE PROCESS-MAGENTA
\def\cvr{\VioletRed}	  % VioletRed  Approximate PANTONE 219
\def\parh{\Rhodamine}	  % Rhodamine  Approximate PANTONE RHODAMINE-RED
\def\cmu{\Mulberry}	  % Mulberry  Approximate PANTONE 241
\def\parv{\RedViolet}	  % RedViolet  Approximate PANTONE 234
\def\cfu{\Fuchsia}	  % Fuchsia  Approximate PANTONE 248
\def\cla{\Lavender}	  % Lavender  Approximate PANTONE 223
\def\cth{\Thistle}	  % Thistle  Approximate PANTONE 245
\def\corc{\Orchid}	  % Orchid  Approximate PANTONE 252
\def\cdo{\DarkOrchid}	  % DarkOrchid  No PANTONE match
\def\cpu{\Purple}	  % Purple  Approximate PANTONE PURPLE
\def\cpl{\Plum}		  % Plum  VERY-Approx PANTONE 518
\def\cvi{\Violet}	  % Violet  Approximate PANTONE VIOLET
\def\clrp{\RoyalPurple}	  % RoyalPurple  Approximate PANTONE 267
\def\cbv{\BlueViolet}	  % BlueViolet  Approximate PANTONE 2755
\def\cpe{\Periwinkle}	  % Periwinkle  Approximate PANTONE 2715
\def\ccb{\CadetBlue}	  % CadetBlue  Approximate PANTONE (534+535)/2
\def\cco{\CornflowerBlue}  % CornflowerBlue  Approximate PANTONE 292
\def\cmb{\MidnightBlue}	  % MidnightBlue  Approximate PANTONE 302
\def\cnb{\NavyBlue}	  % NavyBlue  Approximate PANTONE 293
\def\crb{\RoyalBlue}	  % RoyalBlue  No PANTONE match
%\def\cbb{\Blue}		  % Blue  Approximate PANTONE BLUE-072
\def\cce{\Cerulean}	  % Cerulean  Approximate PANTONE 3005
\def\ccy{\Cyan}		  % Cyan  Approximate PANTONE PROCESS-CYAN
\def\cpb{\ProcessBlue}	  % ProcessBlue  Approximate PANTONE PROCESS-BLUE
\def\csb{\SkyBlue}	  % SkyBlue  Approximate PANTONE 2985
\def\ctu{\Turquoise}	  % Turquoise  Approximate PANTONE (312+313)/2
\def\ctb{\TealBlue}	  % TealBlue  Approximate PANTONE 3145
\def\caq{\Aquamarine}	  % Aquamarine  Approximate PANTONE 3135
\def\cbg{\BlueGreen}	  % BlueGreen  Approximate PANTONE 320
\def\cem{\Emerald}	  % Emerald  No PANTONE match
%\def\cjg{\JungleGreen}	  % JungleGreen  Approximate PANTONE 328
\def\csg{\SeaGreen}	  % SeaGreen  Approximate PANTONE 3268
\def\cgg{\Green}	  % Green  VERY-Approx PANTONE GREEN
\def\cfg{\ForestGreen}	  % ForestGreen  Approximate PANTONE 349
\def\cpg{\PineGreen}	  % PineGreen  Approximate PANTONE 323
\def\clg{\LimeGreen}	  % LimeGreen  No PANTONE match
\def\cyg{\YellowGreen}	  % YellowGreen  Approximate PANTONE 375
\def\cspg{\SpringGreen}	  % SpringGreen  Approximate PANTONE 381
\def\cog{\OliveGreen}	  % OliveGreen  Approximate PANTONE 582
\def\pars{\RawSienna}	  % RawSienna  Approximate PANTONE 154
\def\cse{\Sepia}		  % Sepia  Approximate PANTONE 161
\def\cbr{\Brown}		  % Brown  Approximate PANTONE 1615
\def\cta{\Tan}		  % Tan  No PANTONE match
\def\cgr{\Gray}		  % Gray  Approximate PANTONE COOL-GRAY-8
\def\cbl{\Black}		  % Black  Approximate PANTONE PROCESS-BLACK
\def\cwh{\White}		  % White  No PANTONE match


\loadmsbm

\input epsf

\def\ctln{\centerline}
\def\u{\underbar}
\def\ssk{\smallskip}
\def\msk{\medskip}
\def\bsk{\bigskip}
\def\hsk{\hskip.1in}
\def\hhsk{\hskip.2in}
\def\dsl{\displaystyle}
\def\hskp{\hskip1.5in}

\def\lra{$\Leftrightarrow$ }
\def\ra{\rightarrow}
\def\mpto{\logmapsto}
\def\pu{\pi_1}
\def\mpu{$\pi_1$}
\def\sig{\Sigma}
\def\msig{$\Sigma$}
\def\ep{\epsilon}
\def\sset{\subseteq}
\def\del{\partial}
\def\inv{^{-1}}
\def\wtl{\widetilde}
%\def\lra{\Leftrightarrow}
\def\del{\partial}
\def\delp{\partial^\prime}
\def\delpp{\partial^{\prime\prime}}
\def\sgn{{\roman{sgn}}}
\def\wtih{\widetilde{H}}
\def\bbz{{\Bbb Z}}
\def\bbr{{\Bbb R}}
\def\bbq{{\Bbb Q}}
\def\bbc{{\Bbb C}}
\def\hdsk{\hskip.7in}
\def\hdskb{\hskip.9in}
\def\hdskc{\hskip1.1in}
\def\hdskd{\hskip1.3in}



{\bf Homology with coefficients:} The chain complexes that we have dealt with so far have
had elements which are $\bbz$-linear combinations of basis elements (which are
themselves singular simplices or equivalence classes of them); that is, the chain
groups have been free abelian groups. Formally, though, there is no reason to restrict
ourselves to $\bbz$ coefficients; all that is required is to add
coefficients and take the negative of a chain (i.e., the negatives of its
coefficients). Any abelian group $G$ can be used as coefficient group; the
(singular or simplicial) chain complex $C_*(X;G)$ consists of the groups
$C_n(X;G)=\{\sum g_i\sigma_i : \sigma_i:\Delta^i\ra X\}$ of formal $G$-linear
combinations of (singular or actual) $i$-simplices in $X$. Defining boundary maps
as before, we write $\del_i g\sigma = \sum (-1)^jg \sigma|_{\text{faces}}$;
$(-1)^j$ isn't thought of as an integer, but as altering the coefficient
assigned to a face, between $g$ and $-g$. Our formulas then go through unchanged,
to show that $\del^2=0$, so $C_*(X;G)$ is a chain complex. Its homology groups 
are called the (simplicial or singular) {\it homology groups of $X$ with coefficients in $G$},
denoted $H_*(X;G)$. ``Ordinary'' homology, in this context, would be denoted $H_*(X;\bbz)$

\ssk

The machinery we have built up to work with ordinary homology carries over to homology
with coefficients; none of our proofs really used the fact that our coefficient group was $\bbz$.
With one exception; our computations of simplicial homology groups, via Smith normal form,
specifically used row and column operations over $\bbz$. But this points out the fact that
it is really linear algebra that is required to carry out computations; if we choose a coefficient
group which is (the additive group of) a \u{field} $F$, then we can treat our chain groups
$C_n(X;F)$ as vector spaces over $F$, and the boundary maps are linear
transformations, where we know that $B_i=\text{im}\ \del_{i+1}$ is a subvector space of 
$Z_i=\ker\del_i$, and $H_i(X;F)=Z_i/B_i$ is a vector space over $F$, i.e., 
$H_i(X;F)\cong F^{n_i}$ for some $n_i$ (possibly infinite). 

\vfill
\eject

[We use the field structure to \u{prove} this, but never need the 
multiplicative structure to present it. There \u{is} something to prove
here, though: if $z$ is a cycle, so is $az$ for any $a\in F$ (formally, this is multiplying 
coeffiecients), and the same is true for boundaries. But the point is that we are imposing
the vector space structure from the ``outside'' (the computations don't care), and noting
that from this point of view $\del(az)=a\del z$, i.e., that $\del$ is a linear transformation.
This \u{helps} us do computations, but isn't \u{required} to do them.] Popular coefficient
groups to use are $\bbz_n,\bbq,\bbr$, and $\bbc$.

\ssk

We can also introduce
reduced homology with coefficients, by augmenting a chain complex with an evaluation map 
$C_0(X;G)\ra G\ra 0$, which takes the sum of the coefficients of a $0$-chain. 
We can also extend these ideas to cellular homology. Working through
the computations again, we find, for a CW-complex $X$, that 

\ssk

\ctln{$H_n(X^{(n)},X^{(n-1)};G)\cong 
\wtih_n(\vee S^n;G)\cong \oplus G$,}

\ssk

one summand for each $n$-cell, and we can construct cellular 
homology with coefficients and show that $H_n^{CW}(X;G)\cong H_n(X;G)$ in the exact same way 
(the same of course, is true of simplicial homology with coefficients). The only point to really
make is that the computation of the cellular boundary maps 

\ssk

\ctln{$H_n(X^{(n)},X^{(n-1)};G)\ra H_{n-1}(X^{(n-1)},X^{(n-2)};G)$}

\ssk

again amounts to computing
the maps 

\ssk

\ctln{$f_*:G = H_{n-1}(S^{n-1};G)\ra H_{n-1}(S^{n-1};G)=G$}

\ssk

induced by the attaching map 

\ssk

\ctln{$f:S^{n-1}\ra X^{(n-1)}\ra X^{(n-1)}/X^{(n-2)}\cong \vee S^{n-1}\ra S^{n-1}$}

\ssk

of an $n$-cell; 
if this map has degree $m$, then $f_*:G\ra G$ is multiplication by $m$. 

\vfill
\eject

So, for
example, we can compute the homology groups 
$H_k(\bbr P^n;\bbz_2)$ via cellular homology by noting that using the standard CW-structure
with one cell in each dimension, where before the cellular boundary maps $\bbz\ra\bbz$ were either
$0$ or multiplication by $2$, now the maps $\bbz_2\ra \bbz_2$ are all $0$ (since in $\bbz_2$ 
multiplication by $2$ \u{is} $0$). So in every computation $\ker = \bbz_2$ and $\text{im}\ = 0$,
so $H_k(\bbr P^n;\bbz_2)\cong \bbz_2$ for all $0\leq k\leq n$, and is $0$ otherwise.

\msk

Coefficients do introduce one additional feature; a homomorphism of groups

$\varphi:G\ra H$ induces chain maps on complexes $C_n(X;G)\ra C_n(X;H)$, via

$\sum g_i\sigma_i\mapsto \sum \varphi(g_i)\sigma_i$. Even more, a short exact
sequence of coefficient groups 

$0\ra K\ra G\ra H\ra 0$ induces SESs of chain
complexes 

$0\ra C_*(X;K)\ra C_*(X;G)\ra C_*(X;H)\ra 0$, giving LEHSs 
interweaving the homology groups of $X$ with coefficients in $G,H$, and $K$.
E.g., the SES 

\ssk

\ctln{$0\ra \bbz \buildrel{\times m}\over\ra \bbz\ra \bbz_m\ra 0$}

\ssk

gives the LEHS

\ssk

\ctln{$\cdots\ra H_n(X)\buildrel{(\times m)_*}\over \ra H_n(X)\ra H_n(X;\bbz_m)\ra H_{n-1}(X)\buildrel{(\times m)_*}\over \ra H_{n-1}(X)\ra\cdots$.}

\msk

Using the meta-fact that an exact sequence 

\ssk

\ctln{$A\buildrel{\alpha}\over\ra B\buildrel{\beta}\over\ra C\buildrel{\gamma}\over\ra D\buildrel{\delta}\over\ra E$}

\ssk

yields the short exact sequence

\ctln{$0\ra B/\text{im}\ \alpha\buildrel{\overline{\beta}}\over\ra C\buildrel{\gamma}\over\ra\ker\delta\ra 0$}

\ssk

(since $\text{im}\ \alpha = \ker\beta$ and $\text{im}\ \gamma = \ker\delta$), we get

\vfill
\eject

\hfill ${ }$
\vskip.1in

\ctln{$0\ra H_n(X)/\{m[z]: [z]\in H_n(X)\}\ra H_n(X;\bbz_m)\ra \{[z]\in H_{n-1}(X) : m[z]=0\}\ra 0$}

\ssk

is exact. If $m=p=$ prime, then knowing the homology of $X$ allows us to compute $H_n(X;\bbz_p)$, since we can compute the two
groups on the ends (hence their dimensions over $\bbz_p$), which allows us to compute the dimension of $H_n(X;\bbz_p)$,
hence compute $H_n(X;\bbz_p)$. This last part follows since the alternating sum of the dimensions over $\bbz_p$ of the terms in this
exact sequence is $0$, for the exact same reason this is true about ranks over $\bbz$: such a number is invariant under taking 
homology (the proof for rank over $\bbz$ goes through unchanged).

\msk

Homology with coefficients can, in some instances, tell us things that ordinary homology can't. For example,
$X=\bbr P^2$ has reduced homology $\bbz_2$ in dimension 1 only, and so the quotient map 
$q:X\ra X/X^{(1)}\cong S^2$ induces the trivial map on all homology groups, using $\bbz$ coefficients. But
using $\bbz_2$-coefficients, the LES of the pair $(X^{(1)},X)$ gives, in part,

\ssk

\ctln{$\cdots \ra 0=\wtih_2(X^{(1)};\bbz_2)\ra \wtih_2(X;\bbz_2)\buildrel{q_*}\over\ra \wtih_2(X/X^{(1)};\bbz_2)\ra\cdots$}

\ssk

so $q_*:\bbz_2=\wtih_2(X;\bbz_2)\ra \wtih_2(X/X^{(1)};\bbz_2)$ is injective, hence non-trivial, so $q$ is not
a homotopically trivial map. Which is something that could not be concluded from the induced map on 
ordinary homology.

\vfill
\eject

As another example, homology with $\bbz_2$-coefficients play a role in a proof of the
\crr{{\bf Borsuk-Ulam Theorem}: For every map $f:S^n\ra \bbr^n$, there is an $x\in S^n$ with $f(x)=f(-x)$.}
As part of our proof, we need: if $f:S^n\ra S^n$ is an odd map (i.e., $f(-x)=-f(x)$), then $f$ has odd degree.
The idea is that $f$ induces a map $g:\bbr P^n\ra \bbr P^n$ (first take the composition $S^n\ra S^n\ra \bbr P^n$,
and note that it factors through a map from $\bbr P^n$) satisfying $g\circ q = q\circ f$. 

\ssk

$q:S^n\ra \bbr P^n$ is a (2-sheeted) covering space map. There is a short exact sequence of chain complexes

\ssk

\ctln{$0\ra C_n(\bbr P^n;\bbz_2)\buildrel{\tau}\over\ra C_n(S^n;\bbz_2)\buildrel{q_\#}\over\ra C_n(\bbr P^n;\bbz_2)\ra 0$}

\ssk

where $\tau$ is the {\it transfer map} $\tau(\sigma) = \wtl{\sigma}_1+\wtl{\sigma}_2$, the sum of the two lifts
of a singular simplex into $\bbr P^n$, to $S^n$; the lifts exist by the lifting criterion,
since the domain, $\Delta^n$, of $\sigma$ is contractible. (This last statement also shows that $q_\#$ is surjective.)
$q_\#(\wtl{\sigma}_1+\wtl{\sigma}_2)=\sigma+\sigma=0$, since we are using $\bbz_2$-coefficients; OTOH, if ,
$q_\#(\sum \sigma_i)=0$, then the $\sigma_i$ must occur as pairs of lifts of singular simplices in $\bbr P^n$,
giving exactness at the middle term. Finally, the transfer map is injective, since 
$\tau(\sum\sigma_i)=0$ means that the $\wtl{\sigma}_{i,j}$ must pair off. They can't do this
as $\wtl{\sigma}_{i,1},\wtl{\sigma}_{i,2}$ (they don't agree as maps,
since they send the basepoint to different points), so they must pair with different initial indices. Then
$\sigma_{i_1}=q\circ\wtl{\sigma}_{i_1,j_1}=q\circ\wtl{\sigma}_{i_2,j_2}=\sigma_{i_2}$
means that we can eliminate $\sigma_{i_1}+\sigma_{i_2}=0$ from the sum; finitely many repetitions
give $\sum\sigma_i=0$.

\vfill
\eject

We therefore get a long exact homology sequence

\ssk

$\cdots 0\ra
H_n(\bbr P^n;\bbz_2)\buildrel{\tau_*}\over\ra
H_n(S^n;\bbz_2)\buildrel{q_*}\over\ra
H_n(\bbr P^n;\bbz_2)\buildrel{\del}\over\ra
H_{n-1}(\bbr P^n;\bbz_2)\ra
0\cdots$

$\cdots 0\ra H_i(\bbr P^n;\bbz_2)\buildrel{\del}\over\ra
H_{i-1}(\bbr P^n;\bbz_2)\ra
0\cdots$

$\cdots 0\ra
H_1(\bbr P^n;\bbz_2)\buildrel{\del}\over\ra
H_0(\bbr P^n;\bbz_2)\buildrel{\tau_*}\over\ra
H_0(S^n;\bbz_2)\buildrel{q_*}\over\ra
H_{0}(\bbr P^n;\bbz_2)\ra
0$

\ssk

(the two $0$'s at top are $\wtih_{n+1}(\bbr P^n;\bbz_2)$ and $\wtih_{n-1}(S^n;\bbz_2)$, 
the rest are similar) which, plugging in the known values for the groups is 

\ssk

$0\ra\bbz_2\buildrel{\tau_*}\over\ra\bbz_2\buildrel{q_*}\over\ra\bbz_2\buildrel{\del}\over\ra
\bbz_2\ra 0\cdot\cdot\ 0\ra \bbz_2\buildrel{\del}\over\ra \bbz_2\ra 0\cdot\cdot\ 0\ra\bbz_2
\buildrel{\del}\over\ra\bbz_2\buildrel{\tau_*}\over\ra\bbz_2\buildrel{q_*}\over\ra
\bbz_2\ra 0$

\ssk

But for the initial part, $\tau_*$ is then injective, hence surjective, and $\del$ is surjective, hence injective,
so $q_*$ is the zero map. In the middle part we have isomorphisms, while the final part again is a zero map
between two isomorphisms.

\msk


As we have remarked, an odd map $f:S^n\ra S^n$ induces a map $g:\bbr P^n\ra \bbr P^n$, which
in turn induce maps between chain groups in two short
exact transfer sequences. These maps are chain maps; by definition they commute
with the induced map on chains from the quotient map, and they commute with the transfer map since
the two lifts of $g\circ\sigma$ are $f\circ \wtl{\sigma}_i$, since these \u{are} lifts and lifts are
unique. So their sums are the same, implying that $\tau\circ g = f\circ\tau$.

\ssk

These therefore descend to maps between the corresponding two long exact transfer sequences. 
This, it turns out, allows us to pull ourselves up by our bootstraps, allowing
us to infer information about the map $g_*:H_n(\bbr P^n;\bbz_2)\ra H_n(\bbr P^n;\bbz_2)$
from information about the map $g_*:H_0(\bbr P^n;\bbz_2)\ra H_0(\bbr P^n;\bbz_2)$ .

\vfill
\eject

In the commutative squares

\ssk


$\displaystyle
\CD
H_i(\bbr P^n;\bbz_2)@>\del>> H_{i-1}(\bbr P^n;\bbz_2)\\
@Vg_*VV @Vg_*VV\\
H_i(\bbr P^n;\bbz_2)@>\del>> H_{i-1}(\bbr P^n;\bbz_2)\\
\endCD
$\hskip.1in , $1\leq i\leq n$,\hskip.1in
$\displaystyle
\CD
H_n(\bbr P^n;\bbz_2)@>\tau_*>>H_n(S^n;\bbz_2)\\
@Vg_*VV @Vf_*VV\\
H_n(\bbr P^n;\bbz_2)@>\tau_*>>H_n(S^n;\bbz_2)\\
\endCD
$

\ssk

the horizontal maps are isomorphisms, and for the first, the vertical arrow on the right is an isomorphism
(by induction; the base case is $H_0$ which amounts to saying that $g$ induces a bijection on path components),
so the arrow on the left is an isomorphism. For the second, the vertical arrow on the left is an isomorphism
by the argument just given, so the arrow on the right is an isomorphism. But this arrow is the induced map on 
top-dimensional homology of an $n$-sphere, and so by our discussion on cellular homology is multiplication,
in $\bbz_2$, by the degree of the map $f$. If this degree were even, then the map on $\bbz_2$ would be the
$0$ map; therefore, the degree is odd. Which finishes the proof of the preliminary result.

\ssk

For the proof of Borsuk-Ulam, given $f:S^n\ra \bbr^n$, suppose that $f(x)\neq f(-x)$ for every $x$.
Then the fcn $g(x)=f(x)-f(-x)$ never takes the value $0$, so the fcn $h:S^n\ra S^{n-1}$ given by
$h(x)=g(x)/||g(x)||$ is cts. But $h(-x)=-h(x)$, so the fcn $k=h|{S^{n-1}}:S^{n-1}\ra S^{n-1}$
is an odd fcn, and so has odd degree, and, in particular, induces a non-trivial map on the level of
$H_{n-1}(S^{n-1})$. But this map factors through $H_{n-1}(S^n)=0$, since $k=h\circ\iota:S^{n-1}\ra S^n\ra S^{n-1}$,
a contradiction. So there must be an $x$ with $f(x)=f(-x)$. So, e.g, there are somewhere on Earth antipodal points 
that have both the exact same level of background radiation and the same annual rainfall (or any two other 
contiouously varying quantities you care to name).





\vfill
\end

