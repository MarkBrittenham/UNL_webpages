



\magnification=2000
\overfullrule=0pt
\parindent=0pt

\nopagenumbers

\input amstex

%\voffset=-.6in
%\hoffset=-.5in
%\hsize = 7.5 true in
%\vsize=10.4 true in

\voffset=1.8 true in
\hoffset=-.6 true in
\hsize = 10.2 true in
\vsize=8 true in

\input colordvi



\def\cltr{\Red}		  % Red  VERY-Approx PANTONE RED
\def\cltb{\Blue}		  % Blue  Approximate PANTONE BLUE-072
\def\cltg{\PineGreen}	  % ForestGreen  Approximate PANTONE 349
\def\cltp{\DarkOrchid}	  % DarkOrchid  No PANTONE match
\def\clto{\Orange}	  % Orange  Approximate PANTONE ORANGE-021
\def\cltpk{\CarnationPink}	  % CarnationPink  Approximate PANTONE 218
\def\clts{\Salmon}	  % Salmon  Approximate PANTONE 183
\def\cltbb{\TealBlue}	  % TealBlue  Approximate PANTONE 3145
\def\cltrp{\RoyalPurple}	  % RoyalPurple  Approximate PANTONE 267
\def\cltp{\Purple}	  % Purple  Approximate PANTONE PURPLE

\def\cgy{\GreenYellow}     % GreenYellow  Approximate PANTONE 388
\def\cyy{\Yellow}	  % Yellow  Approximate PANTONE YELLOW
\def\cgo{\Goldenrod}	  % Goldenrod  Approximate PANTONE 109
\def\cda{\Dandelion}	  % Dandelion  Approximate PANTONE 123
\def\capr{\Apricot}	  % Apricot  Approximate PANTONE 1565
\def\cpe{\Peach}		  % Peach  Approximate PANTONE 164
\def\cme{\Melon}		  % Melon  Approximate PANTONE 177
\def\cyo{\YellowOrange}	  % YellowOrange  Approximate PANTONE 130
\def\coo{\Orange}	  % Orange  Approximate PANTONE ORANGE-021
\def\cbo{\BurntOrange}	  % BurntOrange  Approximate PANTONE 388
\def\cbs{\Bittersweet}	  % Bittersweet  Approximate PANTONE 167
%\def\creo{\RedOrange}	  % RedOrange  Approximate PANTONE 179
\def\cma{\Mahogany}	  % Mahogany  Approximate PANTONE 484
\def\cmr{\Maroon}	  % Maroon  Approximate PANTONE 201
\def\cbr{\BrickRed}	  % BrickRed  Approximate PANTONE 1805
\def\crr{\Red}		  % Red  VERY-Approx PANTONE RED
\def\cor{\OrangeRed}	  % OrangeRed  No PANTONE match
\def\paru{\RubineRed}	  % RubineRed  Approximate PANTONE RUBINE-RED
\def\cwi{\WildStrawberry}  % WildStrawberry  Approximate PANTONE 206
\def\csa{\Salmon}	  % Salmon  Approximate PANTONE 183
\def\ccp{\CarnationPink}	  % CarnationPink  Approximate PANTONE 218
\def\cmag{\Magenta}	  % Magenta  Approximate PANTONE PROCESS-MAGENTA
\def\cvr{\VioletRed}	  % VioletRed  Approximate PANTONE 219
\def\parh{\Rhodamine}	  % Rhodamine  Approximate PANTONE RHODAMINE-RED
\def\cmu{\Mulberry}	  % Mulberry  Approximate PANTONE 241
\def\parv{\RedViolet}	  % RedViolet  Approximate PANTONE 234
\def\cfu{\Fuchsia}	  % Fuchsia  Approximate PANTONE 248
\def\cla{\Lavender}	  % Lavender  Approximate PANTONE 223
\def\cth{\Thistle}	  % Thistle  Approximate PANTONE 245
\def\corc{\Orchid}	  % Orchid  Approximate PANTONE 252
\def\cdo{\DarkOrchid}	  % DarkOrchid  No PANTONE match
\def\cpu{\Purple}	  % Purple  Approximate PANTONE PURPLE
\def\cpl{\Plum}		  % Plum  VERY-Approx PANTONE 518
\def\cvi{\Violet}	  % Violet  Approximate PANTONE VIOLET
\def\clrp{\RoyalPurple}	  % RoyalPurple  Approximate PANTONE 267
\def\cbv{\BlueViolet}	  % BlueViolet  Approximate PANTONE 2755
\def\cpe{\Periwinkle}	  % Periwinkle  Approximate PANTONE 2715
\def\ccb{\CadetBlue}	  % CadetBlue  Approximate PANTONE (534+535)/2
\def\cco{\CornflowerBlue}  % CornflowerBlue  Approximate PANTONE 292
\def\cmb{\MidnightBlue}	  % MidnightBlue  Approximate PANTONE 302
\def\cnb{\NavyBlue}	  % NavyBlue  Approximate PANTONE 293
\def\crb{\RoyalBlue}	  % RoyalBlue  No PANTONE match
%\def\cbb{\Blue}		  % Blue  Approximate PANTONE BLUE-072
\def\cce{\Cerulean}	  % Cerulean  Approximate PANTONE 3005
\def\ccy{\Cyan}		  % Cyan  Approximate PANTONE PROCESS-CYAN
\def\cpb{\ProcessBlue}	  % ProcessBlue  Approximate PANTONE PROCESS-BLUE
\def\csb{\SkyBlue}	  % SkyBlue  Approximate PANTONE 2985
\def\ctu{\Turquoise}	  % Turquoise  Approximate PANTONE (312+313)/2
\def\ctb{\TealBlue}	  % TealBlue  Approximate PANTONE 3145
\def\caq{\Aquamarine}	  % Aquamarine  Approximate PANTONE 3135
\def\cbg{\BlueGreen}	  % BlueGreen  Approximate PANTONE 320
\def\cem{\Emerald}	  % Emerald  No PANTONE match
%\def\cjg{\JungleGreen}	  % JungleGreen  Approximate PANTONE 328
\def\csg{\SeaGreen}	  % SeaGreen  Approximate PANTONE 3268
\def\cgg{\Green}	  % Green  VERY-Approx PANTONE GREEN
\def\cfg{\ForestGreen}	  % ForestGreen  Approximate PANTONE 349
\def\cpg{\PineGreen}	  % PineGreen  Approximate PANTONE 323
\def\clg{\LimeGreen}	  % LimeGreen  No PANTONE match
\def\cyg{\YellowGreen}	  % YellowGreen  Approximate PANTONE 375
\def\cspg{\SpringGreen}	  % SpringGreen  Approximate PANTONE 381
\def\cog{\OliveGreen}	  % OliveGreen  Approximate PANTONE 582
\def\pars{\RawSienna}	  % RawSienna  Approximate PANTONE 154
\def\cse{\Sepia}		  % Sepia  Approximate PANTONE 161
\def\cbr{\Brown}		  % Brown  Approximate PANTONE 1615
\def\cta{\Tan}		  % Tan  No PANTONE match
\def\cgr{\Gray}		  % Gray  Approximate PANTONE COOL-GRAY-8
\def\cbl{\Black}		  % Black  Approximate PANTONE PROCESS-BLACK
\def\cwh{\White}		  % White  No PANTONE match


\loadmsbm

\input epsf

\def\ctln{\centerline}
\def\u{\underbar}
\def\ssk{\smallskip}
\def\msk{\medskip}
\def\bsk{\bigskip}
\def\hsk{\hskip.1in}
\def\hhsk{\hskip.2in}
\def\dsl{\displaystyle}
\def\hskp{\hskip1.5in}

\def\lra{$\Leftrightarrow$ }
\def\ra{\rightarrow}
\def\mpto{\logmapsto}
\def\pu{\pi_1}
\def\mpu{$\pi_1$}
\def\sig{\Sigma}
\def\msig{$\Sigma$}
\def\ep{\epsilon}
\def\sset{\subseteq}
\def\del{\partial}
\def\inv{^{-1}}
\def\wtl{\widetilde}
%\def\lra{\Leftrightarrow}
\def\del{\partial}
\def\delp{\partial^\prime}
\def\delpp{\partial^{\prime\prime}}
\def\sgn{{\roman{sgn}}}
\def\wtih{\widetilde{H}}
\def\bbz{{\Bbb Z}}
\def\bbr{{\Bbb R}}
\def\hdsk{\hskip.7in}
\def\hdskb{\hskip.9in}
\def\hdskc{\hskip1.1in}
\def\hdskd{\hskip1.3in}




{\bf Homology and homotopy groups:} There are connections between homology groups and
the fundamental (and higher) homotopy groups, provided by what is known as the
{\it Hurewicz map} $H:\pi_n(X,x_0)\ra H_n(X)$ . For $n=1$ (higher $n$ are similar)
the idea is that elements of $\pu(X)$ are loops, which can be thought of as maps
$\gamma:S^1\ra X$ (or more precisely, mapping into the path component containing 
$x_0$), inducing a map $\gamma_*:\bbz = H_1(S^1)\ra H_1(X)$ . 
We define $H([\gamma])=\gamma_*(1)$ . Because homotopic maps give the same induced
map on homology, this really is well-defined map on homotopy
classes, i.e. from $\pu(X)$ to $H_1(X)$. [A different view: 
a loop $\gamma:(I,\del I)\ra (X,x_0)$ defines a singular 1-chain which, being a loop,
has zero boundary, so is a 1-cycle. Since based homotopic maps give homologous
chains (essentially by the same homotopy invariance property above), we get a 
well-defined map $\pu(X,x_0)\ra H_1(X)$.

Since as 1-chains, the concatenation $\gamma*\delta$ of two loops is homologous
to the sum $\gamma+\delta$ - the map $K:I\times I\ra X$ given by $K(s,t)=(\gamma*\delta)(s)$,
after crushing the left and right vertical boundaries to points, can be thought of as
a singular 2-simplex with boundary $\gamma + \delta - (\gamma*\delta)$ - the map 
$H$ is a homomorphism.

\vfill
\eject

When $X$ is path-connected, this map $H:\pu(X)\ra H_1(X)$ is onto . [When it isn't it maps onto the 
summand of $H_1(X)$ corresponding to the path component containing our chosen basepoint.]
To see this, note that any cycle $z\in Z_1(X)$ can be represented as a sum of singular
1-simplices $\sum \sigma_i^1$ , i.e. we can (by reversing the orientations on simplices
to make coefficient positive, and then writing a multiple of a simplex as a sum of
simplices) assume all coefficients in our sum are 1. Then 
$0 = \del z = \sum (\sigma_i^1(0,1) - \sigma_i^1(1,0))$ means that, starting with any positive term, 
we can match it with a 
negative term to cancel that term, which is paired with a postive term, having a matching negative term, etc.,
until the initial positive term is cancelled. This sub-chain represents a collection of paths which 
concatenate to a loop, so $z=$ (this loop) + (the remaining terms) . Induction implies that
$z$ can be written as a sum of (sums of paths forming loops), which is (as above) homologous  to the
sum of loops. Choosing paths from the start of these loops to our chosen basepoint (which is the only
place where we use path connectedness, we can concatenate the based loops $\overline{\gamma}*\sigma*\gamma$
to a single based loop $\eta$, which under $H$ is sent to a chain homologous to $z$.
So $H[\eta] = [z]$ .

\vfill
\eject

Since $H_1(X)$ is abelian (and $\pu(X)$ need not be), the kernel of $H$ contains the
commutator subgroup $[\pu(X),\pu(X)]$ . 
We now show that, if $X$ is path connected,
$H$ induces an isomorphism $H_1(X) \cong \pu(X)/[\pu(X),\pu(X)]$ . 
To show this, it remains to show that  ker$(H)\subseteq [\pu(X),\pu(X)]$ . 
Or put differently, the ineduced map from $\pu(X)_{ab} = \pu(X)/[\pu(X),\pu(X)]$ 
(i.e., $\pu(X)$, written using additive notation) to $H_1(X)$  is injective.
So suppose $[\gamma]\in \pu(X)$ and, thought of as a singular 1-simplex, 
$\gamma = \del w$ for some 2-simplex $w=\sum a_i\sigma_i^2$ . 
As before, we may assume that all $a_i=1$, by reversing orientation and writing multiples as 
sums. By adding ``tails'' from each image of a vertex of each $\sigma_i^2$ to our chosen 
basepoint $x_0$, we may assume that the image of every face of $\Delta^2$, under
the $\sigma_i$ , is a loop at $x_0$ (by essentially replacing each $\sigma_i$ with a $\tau_i$
which first collapses little triangle at each vertex to arcs, maps the resulting central triangle
via $\sigma_i$, and the arcs via the paths). 

\ssk

Once we have made this slight alteration, the equation
$\displaystyle \gamma = \del w = \sum_{i=1}^n \sum_{j=0}^2 \del_j \sigma_i = 0$
makes sense (and is true) in both ($C_1(X)$ hence $Z_1(X)$ hence) $H_1(X)$ and
$\pu(X)_{ab}$, the first essentially by definition and the second because 
all of the $\del_j \sigma_i$ are loops at $x_0$ and, in 
$\pu(X)$, $(\del_0\sigma_i)\overline{\del_1\sigma_i}(\del_2\sigma_i)$ is null-homotopic,
so is trivial in $\pu(X)$. Written additively, this means that in
$\pu(X)_{ab}$ , $\del_0\sigma_i - \del_1\sigma_i + \del_2\sigma_i = 0$. 
So $\gamma = 0$ in $\pu(X)_{ab}$ , as desired.


\vfill
\eject

The Hurewicz map $H:\pu(X)\ra H_1(X)$ induces, when $X$ is path-connected,
an isomorphism from $\pu(X)/[\pu(X),\pu(X)]$ to $H_1(X)$ . 
This result can be used in two ways; knowing a (presentation for) $\pu(X)$
allows us to compute $H_1(X)$, by writing the relators additively, giving
$H_1(X)$ as the free abelian group on the generators, modulo the kernel 
of the ``presentation matrix'' given by the resulting linear equations. Conversely,
knowing $H_1(X)$ provides information about $\pu(X)$. For example,
a calculation on the way to invariance of domain implied that for every
knot $K$ in $S^3$ (i.e., the image of an embedding $h:S^1\hookrightarrow S^3$),
$H_1(S^3\setminus K) \cong \bbz$ . This implies that the abelianization of 
$G_K = \pu(S^3\setminus K)$ (i.e., the largest abelian quotient of $G_K$)  is $\bbz$.
But this in turn implies that for every integer $n\geq 2$, there is a \u{unique}
surjective homomorphism $G_K\ra \bbz_n$, since such a homomorphism must factor 
through the abelianization, and there is exactly one surjective homomorphism
$\bbz \ra \bbz_n$ ! Consequently, there is a unique (normal) subroup (the kernel
of this homomorphism) $K_n\sset G_K$ with quotient $\bbz_n$ . Using the Galois
correspondence, there is a (unique) covering space $X_n$ of $X=S^3\setminus K$
corresponding to $K_n$, called the $n$-fold cyclic covering of $K$ . This space is determined
by $K$ and $n$, and so its homology groups are determined by the same data.
And even though homology cannot distinguish between two knot complements,
$K$, $K^\prime$, it might be the case that homology \u{can} distinguish between 
their cyclic coverings. Consequently, 
if $H_1(X_n)\not\cong H_1(X_n^\prime)$, then $K$ and $K^\prime$ have non-homeomorphic
complement, and so represent ``different'' embeddings, hence different knots.
In practice, one can compute presentations for $\pu(X_n)$ (in several different ways),
and so one can compute $H_1(X_n)$, providing an effective way to use homology
to distinguish knots! This approach was ultimately formalized (by Alexander) into a polynomial
invariant of knots, known as the Alexander polynomial.

\msk

Computing the homology of the cyclic coverings can be done in several ways. The
Reidemeister-Schreier method will allow one to compute a presentation for the
kernel of a homomorphism $\varphi:G\ra H$, given a presentation of $G$ and a
{\it transversal} of the map, which is a representative of each coset of $G$ modulo
the kernel. Abelianizing this will give homology computation. Another approach
uses {\it Seifert surfaces}, orientable surfaces with $\del \Sigma = K$, to cut
$S^3\setminus K$ open along. Writing $S^3\setminus K = (S^3\setminus N(\Sigma))\cup N(\Sigma)$
allows us to use Mayer-Vietoris to compute homology. But the cyclic covering spaces
can be built by ``unwinding'' this view of $S^3\setminus K$; instead of gluing the two
ends of $N(K)$ to the same $S^3\setminus N(\Sigma)$, we can take $n$ copies
of $S^3\setminus N(\Sigma)$ and glue them together in a circle. Mayer-Vietoris
again tells us how to compute the homology of the resulting space. Details may be found on the accompanying
pages taken from Rolfsen's ``Knots and Links''.


\vfill
\end

