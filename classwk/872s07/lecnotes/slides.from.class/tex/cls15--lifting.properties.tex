
\magnification=2000
\overfullrule=0pt
\parindent=0pt

\nopagenumbers

\input amstex

%\voffset=-.6in
%\hoffset=-.5in
%\hsize = 7.5 true in
%\vsize=10.4 true in

\voffset=1.8 true in
\hoffset=-.6 true in
\hsize = 10.2 true in
\vsize=8 true in

\input colordvi

\def\cltr{\Red}		  % Red  VERY-Approx PANTONE RED

\loadmsbm

\input epsf

\def\ctln{\centerline}
\def\u{\underbar}
\def\ssk{\smallskip}
\def\msk{\medskip}
\def\bsk{\bigskip}
\def\hsk{\hskip.1in}
\def\hhsk{\hskip.2in}
\def\dsl{\displaystyle}
\def\hskp{\hskip1.5in}

\def\lra{$\Leftrightarrow$ }
\def\ra{\rightarrow}
\def\mpto{\logmapsto}
\def\pu{\pi_1}
\def\mpu{$\pi_1$}
\def\sig{\Sigma}
\def\msig{$\Sigma$}
\def\ep{\epsilon}
\def\sset{\subseteq}
\def\del{\partial}
\def\inv{^{-1}}
\def\wtl{\widetilde}
%\def\lra{\Leftrightarrow}
\def\del{\partial}
\def\delp{\partial^\prime}
\def\delpp{\partial^{\prime\prime}}
\def\sgn{{\roman{sgn}}}
\def\wtih{\widetilde{H}}
\def\bbz{{\Bbb Z}}
\def\bbr{{\Bbb R}}
\def\rtar{$\Rightarrow$}

\def\cltr{\Red}		  % Red  VERY-Approx PANTONE RED
\def\cltb{\Blue}		  % Blue  Approximate PANTONE BLUE-072
\def\cltg{\PineGreen}	  % ForestGreen  Approximate PANTONE 349



{\bf Lifting properties:}

\msk

Covering spaces of a (suitably nice) space $X$ have a very close relationship
to $\pu(X,x_0)$. 

\ssk 

\cltr{{\bf Homotopy Lifting Property:} If $p:\wtl{X}\ra X$ is a covering map, 
$H:Y\times I\ra X$ is a homotopy, $H(y,0)=f(y)$, and
$\wtl{f}:Y\ra \wtl{X}$ is a {\it lift} of $f$ (i.e., $p\circ \wtl{f}=f$),
then there is a unique lift $\wtl{H}$ of $H$ with $\wtl{H}(y,0)=\wtl{f}(y)$ .}

\msk

In particular, applying this property in the case $Y=\{*\}$, where a homotopy

$H:\{*\}\times I\ra X$ is really just a a path $\gamma:I\ra X$,
we have the 

\cltr{{\bf Path Lifting Property}: Given 
a covering map $p:\wtl{X}\ra X$, a path 
$\gamma:I\ra X$ with $\gamma(0)=x_0$, and a point 
$\wtl{x}_0\in p^{-1}(x_0)$, there is a unique path $\wtl{\gamma}$
lifting $\gamma$ with $\wtl{\gamma}(0)=\wtl{x}_0$.} 

An immediate consequence: 

\msk  

\cltb{If $p:(\wtl{X},\tilde{x}_0)\ra (X,x_0)$ is a covering map, then the 
induced homomorphism} 

\cltb{$p_*:\pu(\wtl{X},\wtl{x}_0)\ra \pu(X,x_0)$ is injective.}

\msk

{\bf Proof:} $\gamma:(I,\del I)\ra (\wtl{X},\wtl{x}_0)$ a 
loop with $p_*([\gamma])=1$ in $\pu(X,x_0)$. There is

$H:(I\times I,\del I\times I)\ra (X,x_0)$ interpolating
between $p\circ\gamma$ and the constant 
path. By homotopy lifting, there is a homotopy $\wtl{H}$ from $\gamma$ to 
the lift of the constant map at $x_0$. The vertical sides 
$s\mapsto \wtl{H}(0,s),\wtl{H}(1,s)$ are also lifts of the 
constant map, beginning from 
$\wtl{H}(0,0),\wtl{H}(1,0)=\gamma(0)=\gamma(1)=\wtl{x}_0$, so
are the constant map at $\wtl{x}_0$. So the lift at the 
bottom is the constant map at $\wtl{x}_0$. So $\wtl{H}$
represents a null-homotopy of $\gamma$, so $[\gamma]=1$
in $\pu(\wtl{X},\wtl{x}_0)$.


\vfill
\eject

Even more, $p_*(\pu(\wtl{X},\wtl{x}_0)))\sset \pu(X,x_0)$
is precisely the elements given by loops at $x_0$, 
whose lifts to paths starting at $\wtl{x}_0$, are loops. 
If $\gamma$ lifts
to a loop $\wtl{\gamma}$, then $p\circ\wtl{\gamma}=\gamma$, so
$p_*([\wtl{\gamma}])=[\gamma]$ . If 
$p_*([\wtl{\gamma}])=[\gamma]$, then $\gamma\simeq p\circ\wtl{\gamma}$
rel endpoints; the homotopy lifts to 
a homotopy b/w the lift of $\gamma$ at 
$\wtl{x}_0$ and the lift of $p\circ\wtl{\gamma}$ at $\wtl{x}_0$
(which is $\wtl{\gamma}$, since $\wtl{\gamma}(0)=\wtl{x}_0$ and
lifts are unique). So the lift of $\gamma$ is a loop, as desired.

\msk

{\bf Proof} of H.L.P.:  lift maps a little bit at a time! Cover $X$ by
evenly covered open sets ${\Cal U}_i$. For each fixed
$y\in Y$, since $I$ is compact and the sets $H^{-1}({\Cal U}_i)$ form an
open cover of $Y\times I$,
the Tube Lemma provides an open neighborhood 
${\Cal V_y}$ of $y$ in $Y$ and finitely many $p^{-1}{\Cal U}_{i}$ which
cover ${\Cal V_y}\times I$ . 

\msk

To define $\wtl{H}(y,t)=\wtl{H}_y(t)$, cut $\{y\}\times I$ into pieces $I_j$,
each mapping into some ${\Cal U}_{j}$ under $H$. Starting from the left,
we have (inductively) a lift $\wtl{H}_y(t_j)$ of the left endpoint $t_j$ of $I_j$ to $\wtl{X}$,
and a homeo $h_j:{\Cal U}_{j}\ra$ the component of its inverse image of ${\Cal U}_j$
containing $\wtl{H}_y(t_j)$. Then define $\wtl{H}_y$ on $I_j$
to be $h_j\circ {H}_y$. By induction, $\wtl{H}(y,t)$ is defined for all $t$ 
(and $y$). This definition is independent of the partition $\{y\}\times I$,
by the usual process of taking the union of the partitions, and noticing that
the choice of $h_j$ is unique. (If we change the open cover,
we can compare using the intersections; the choice of $h_j$'s will be the
same.)
$\wtl{H}$ is a lift of $H$ since 
$p\circ \wtl{H} = p\circ (h_j\circ H) = (p\circ h_j)\circ H=I\circ H = H$.
$\wtl{H}$ is continuous since for $y$ near $y_0$ we can use the same partitions 
and the same open cover (because of our tube lemma condition), which means 
that we use the same maps $h_j$ to lift; the pasting lemma implies continuity.

\vfill
\eject

So, for example, if we build a 5-sheeted cover of the bouquet of 2 circles, 
as before, (after choosing a maximal tree upstairs) 
we can read off the images of the generators of the fundamental group
of the total space; we have labelled each edge by the generator it
traces out downstairs, and for each edge outside of the maximal tree
chosen, we read from basepoint out the tree to one end, across the edge,
and then back to the basepoint in the tree. In our example, this
gives:


\msk

\ctln{$<ab,aaab^{-1}, baba^{-1},baa,ba^{-1}bab^{-1},bba^{-1}b^{-1} | >$}

\msk

\leavevmode


\epsfxsize=3in
\ctln{{\epsfbox{lift1.ai}}}


\bsk

This is (from its construction) a copy of the free group on 6 letters,
in the free group $F(a,b)$ . In a similar way, by explicitly building
a covering space, we find that the fundamental group of a closed 
surface of genus 3 is a subgroup of the fundamental group of the 
closed surface of genus 2. 

\vfill
\eject

The cardinality of a point inverse $p\inv(y)$ is, by the evenly
covered property, constant on (small) open sets, so the set of 
points of $x$ whose point inverses have any given cardinality
is open. Consequently, if $X$ is connected, this number
is constant over all of $X$, and is called the number of {\it sheets}
of the covering $p:\wtl{X}\ra X$ . 

\bsk


The number of sheets of a covering map can also be determined 
from the fundamental groups:

\msk

\cltr{{\bf Proposition:} If $X$ and $\wtl{X}$ are 
path-connected, then the number of sheets of a covering map equals
the index of the subgroup $H=p_*(\pu(\wtl{X},\wtl{x}_0)$ in 
$G=\pu(X,x_0)$ .}

\msk

{|bf Proof:} Choose loops $\{g_\alpha=[\gamma_\alpha]\}$, one in 
each of the (right) cosets of $H$ in $G$. Lift
them to loops based at $\wtl{x}_0$; they will have distinct
endpoints. (If $\wtl{\gamma}_1(1)=\wtl{\gamma}_2(1)$, then 
by uniqueness of lifts, $\gamma_1*\overline{\gamma_2}$ lifts to 
$\wtl{\gamma}_1*\overline{\wtl{\gamma}_2}$, so it
lifts to a loop, so $\gamma_1*\overline{\gamma_2}$ represents
an element of $p_*(\pu(\wtl{X},\wtl{x}_0)$, so they are
in the same coset.)
Conversely, every point in $p\inv(x_0)$ is the endpoint of one of these
lifts, since we can construct a path $\wtl{\gamma}$
from $\wtl{x}_0$ to any such point $y$, giving a loop
$\gamma=p\circ \wtl{\gamma}$ representing an element $g\in\pu(X,x_0)$.
But then $g=hg_\alpha$ for some $h\in H$ and $\alpha$, 
so $\gamma$ is homotopic rel endpoints to $\eta*\gamma_\alpha$ for some loop
$\eta$ representing $h$. But then lifting these based at $\wtl{x}_0$, by homotopy
lifting, $\wtl{\gamma}$ is homotopic rel endpoints to $\wtl{\eta}$, which is a 
loop, followed by the lift $\wtl{\gamma}_\alpha$ of $\gamma_\alpha$
starting at $\wtl{x}_0$. So $\wtl{\gamma}$ and 
$\wtl{\gamma}_\alpha$ have the same value at 1.

\ssk

Therefore, lifts of representatives of coset representatives of $H$ in $G$ give
a 1-to-1 correspondence, given by $\wtl{\gamma}(1)$, with $p\inv{x_0}$.
In particular, they have the same cardinality.

\vfill
\eject

The path lifting property (because $\pi([0,1],0)=\{1\}$) is a special
case of the \cltr{{\bf lifting criterion}: If 
$p:(\wtl{X},\wtl{x}_0)\ra (X,x_0)$ is a covering map, and 
$f:(Y,y_0)\ra (X,x_0)$ is a map, where
$Y$ is path-connected and locally path-connected, then there is a unique lift 
$\wtl{f}:(Y,y_0)\ra (\wtl{X},\wtl{x}_0)$ of $f$ (i.e., 
$f=p\circ\wtl{f}$) $\Leftrightarrow$ 
$f_*(\pu(Y,y_0))\sset p_*(\pu(\wtl{X},\wtl{x}_0))$ . }


\msk

If $\wtl{f}$ exists, then $f_*=p_*\circ\wtl{f}_*$, so 
$f_*(\pu(Y,y_0)) = p_*(\wtl{f}_*(\pu(Y,y_0)))\sset p_*(\pu(\wtl{X},\wtl{x}_0))$.

\ssk

Conversely, if $f_*(\pu(Y,y_0))\sset p_*(\pu(\wtl{X},\wtl{x}_0))$,
we will use path lifting to build the lift.
Given $y\in Y$,
choose a path $\gamma$ in $Y$ from $y_0$ to $y$ and lift the path $f\circ\gamma$ 
in $X$ to a path $\wtl{f\circ\gamma}$ with 
$\wtl{f\circ\gamma}(0)=\wtl{x}_0$ . Then define
$\wtl{f}(y)=\wtl{f\circ\gamma}(1)$ . \cltb{\underbar{If} well-defined and continuous}, 
this is our required lift, 
since $(p\circ\wtl{f})(y) = p(\wtl{f}(y))=p(\wtl{f\circ\gamma}(1))
=p\circ\wtl{f\circ\gamma}(1) = (f\circ\gamma)(1) = f(\gamma(1)) = f(y)$. 
For well-defined, if $\eta$ is a path from 
$y_0$ to $y$, then $\gamma*\overline{\eta}$ is a loop, 
so $f\circ(\gamma*\overline{\eta})=(f\circ\gamma)*\overline{(f\circ\eta)}$
is a loop, giving an element of 
$f_*(\pu(Y,y_0))\sset p_*(\pu(\wtl{X},\wtl{x}_0))$, and so
lifts to a loop based at $\wtl{x}_0$.
So $f\circ\gamma$ and $f\circ\eta$ lift,
starting at $\wtl{x}_0$, to have the same value at 1. So $\wtl{f}$ is
well-defined.  Continuity comes from the 
evenly covered property of $p$. Given $y\in Y$,
and  a nbhd $\wtl{\Cal U}$ of 
$\wtl{f}(y)$ in $\wtl{X}$, 
we want a nbhd ${\Cal V}$ of $y$ with 
$\wtl{f}({\Cal V})\sset\wtl{\Cal U}$. Choose an evenly covered 
nbhd ${\Cal U}_y$ for $f(y)$, the sheet 
$\wtl{\Cal U}_y$ over ${\Cal U}_y$ which contains $\wtl{f}(y)$,
and set ${\Cal W}=\wtl{\Cal U}\cap \wtl{\Cal U}_y$. $p$ is a homeo from ${\Cal W}$ to the 
open set $p({\Cal W})\sset X$. Then if we set ${\Cal V}^\prime= f\inv(p({\Cal W}))$
this is open and contains $y$, and so contains a path-connected open 
nbhd ${\Cal V}$ of $y$. Then for every point $z\in {\Cal V}$ we compute $\wtl{f}(z)$ 
by a path
$\gamma$ from $y_0$ to $z$ which first goes to $y$ and then, \cltb{{\it in} ${\Cal V}$},
from $y$ to $z$. Then by unique path lifting, 
since $f({\Cal V})\sset {\Cal U}_y$ , $f\circ\gamma$ lifts to 
the concatenation of a path from $\wtl{x}_0$ to $\wtl{f}(y)$ and a 
path \cltb{{\it in} $\wtl{\Cal U}_y$} from $\wtl{f}(y)$ to $\wtl{f}(z)$.
So $\wtl{f}(z)\in\wtl{\Cal U}$.

\ssk


\vfill
\end


Because $\wtl{f}$ is built by lifting paths, and path
lifting is unique, the last statement of the proposition follows.
