
\magnification=2000
\overfullrule=0pt
\parindent=0pt

\nopagenumbers

\input amstex

%\voffset=-.6in
%\hoffset=-.5in
%\hsize = 7.5 true in
%\vsize=10.4 true in

\voffset=1.8 true in
\hoffset=-.6 true in
\hsize = 10.2 true in
\vsize=8 true in

\input colordvi

\def\cltr{\Red}		  % Red  VERY-Approx PANTONE RED

\loadmsbm

\input epsf

\def\ctln{\centerline}
\def\u{\underbar}
\def\ssk{\smallskip}
\def\msk{\medskip}
\def\bsk{\bigskip}
\def\hsk{\hskip.1in}
\def\hhsk{\hskip.2in}
\def\dsl{\displaystyle}
\def\hskp{\hskip1.5in}

\def\lra{$\Leftrightarrow$ }
\def\ra{\rightarrow}
\def\mpto{\logmapsto}
\def\pu{\pi_1}
\def\mpu{$\pi_1$}
\def\sig{\Sigma}
\def\msig{$\Sigma$}
\def\ep{\epsilon}
\def\sset{\subseteq}
\def\del{\partial}
\def\inv{^{-1}}
\def\wtl{\widetilde}
%\def\lra{\Leftrightarrow}
\def\del{\partial}
\def\delp{\partial^\prime}
\def\delpp{\partial^{\prime\prime}}
\def\sgn{{\roman{sgn}}}
\def\wtih{\widetilde{H}}
\def\bbz{{\Bbb Z}}
\def\bbr{{\Bbb R}}
\def\rtar{$\Rightarrow$}

\def\cltr{\Red}		  % Red  VERY-Approx PANTONE RED
\def\cltb{\Blue}		  % Blue  Approximate PANTONE BLUE-072
\def\cltg{\PineGreen}	  % ForestGreen  Approximate PANTONE 349

{\bf Seifert-van Kampen Theorem:}

\msk

If $X={\Cal U}\cup{\Cal V}$ with ${\Cal U},{\Cal V}$ and ${\Cal U}\cap{\Cal V}={\Cal W}$ open and path-connected, and
$x_0\in{\Cal W}$, then $\pi_1(X,x_0)$ is the pushout of $\pi_1({\Cal U})$ and $\pi_1({\Cal V})$ along $\pi_1({\Cal W})$.

\msk

The basic idea: a proof in two (and a half) parts.

\ssk

The inclusions $i_{\Cal U}:{\Cal U}\ra X$ and $i_{\Cal V}:{\Cal V}\ra X$ induce a homomorphism

\ssk

$\varphi:\pi_1({\Cal U})*\pi_1({\Cal V})\ra\pi_1(X)$. 

\ssk

First step: show that $\varphi$ is surjective
(using a Lebesgue number argument!). We also have inclusion-induced homomorphisms from the
maps 

\ssk

$j_{\Cal U}:{\Cal W}\ra {\Cal U}$ and $j_{\Cal V}:{\Cal W}\ra {\Cal U}$. 

\ssk

Second step: show that 

\ssk

$\ker(\varphi)=\langle \{{j_{\Cal U}}_*(\gamma){j_{\Cal V}}_*(\gamma^{-1}) : \gamma\in\pi_1(W)\}\rangle^N$

\ssk

(using a Lebesgue number argument!). Therfore, $\varphi$ induces an isomorphism

$\vartheta:\pi_1({\Cal U})*_{\pi_1({\Cal W})}\pi_1({\Cal V})\ra\pi_1(X)$, as desired.

\msk

The reliance upon a decomposition of $X$ into open sets in the theorem is dictated by our use of Lebesgue numbers. In practice,
we sidestep this (typically annoying) condition, by decomposing into \u{closed} sets $X=C\cup D$,
but insist that these sets have open neighborhoods ${\Cal U},{\Cal V}$ so that ${\Cal U},{\Cal V},{\Cal U}\cap{\Cal V}$
deformation retract to $C,D,C\cap D$ respectively. The closed sets therefore have (essentially)
the same fundamental groups as the open sets, and the analogous result follows.
Later we will show how we can always arrange this hypothesis for most ``reasonable''
closed subsets of ``reasonable'' spaces.

\vfill
\eject

{\bf Proof, part one:}

\msk

To show that $\varphi:\pi_1({\Cal U})*\pi_1({\Cal V})\ra\pi_1(X)$ is surjective, we wish, given a loop

$\gamma:(I,\del i)\ra(X,x_0)$, to show that $\gamma$ is homotopic rel endpoints to the concatenation of
loops which each map into either ${\Cal U}$ or ${\Cal V}$. But ${\Cal U},{\Cal V}$ form an open cover of $X$,
so there is a Lebesgue number $1/n$ for the cover $\gamma^{-1}({\Cal U}),\gamma^{-1}({\Cal V})$ of $I$.
Partitioning $I$ into $n$ equal pieces, and writing $\gamma_i=\gamma|_{[{{i-1}\over{n}},{{i}\over{n}}]}$,
each $\gamma_i$ maps into ${\Cal U}$ or ${\Cal V}$. As in our proof of $\pi_1(S^1)\cong \bbz$, we amalgamate
the subintervals until the subsets they map into alternate between the two as we traverse the interval $I$.


\msk

Calling the resulting intervals $I_j=[z_{j-1},z_j]$, $j=1,\ldots,m$, we then have that 

$\gamma(z_j)\in{\Cal U}\cap{\Cal V}$ for every $j$. Since ${\Cal U}\cap{\Cal V}$ is path-connected,
we can find a path $\alpha_j$ in ${\Cal U}\cap{\Cal V}$ from $z_j$ to $x_0$. 
We will recycle the notation $\gamma_j=\gamma|_{[z_{j-1},z_j]}$; then

\msk

$\gamma\simeq\gamma_0*\cdots *\gamma_m\simeq
\gamma_0*(\alpha_1*\overline{\alpha_1})*\gamma_1*\cdots *\gamma_{m-1}*(\alpha_{m-1}*\overline{\alpha_{m-1}})*\gamma_m$

$\simeq 
(\gamma_0*\alpha_1)*(\overline{\alpha_1}*\gamma_1*\alpha_2)*\cdots *(\overline{\alpha_{m-2}}*
\gamma_{m-1}*\alpha_{m-1})*(\overline{\alpha_{m-1}}*\gamma_m)$

$=\eta_0*\cdots *\gamma_m$, 

\ssk

which is a concatenation of loops (based at $x_0$) which alternately map into
${\Cal U}$ and ${\Cal V}$. 

\msk

To be completely pedantic, if we let $\omega_j$=$\eta_j$ with its codomain changed from $X$ to ${\Cal U}$ or ${\Cal V}$
as appropriate (the $\omega_i$ are continuous, since restriction of codomain preserves continuity (using subspace 
topologies)), then $[\omega_0]\cdot \cdots \cdot [\omega_m]\in \pi_1({\Cal U})*\pi_1({\Cal V})$, and

\ssk

$\varphi([\omega_0]\cdot \cdots \cdot [\omega_m])=[\eta_0]\cdot \cdots \cdot [\eta_m]=[\eta_0*\cdots *\eta_m]=[\gamma]$,
so $\varphi$ is surjective, as desired.

\vfill
\eject

{\bf Proof, part two:}

\msk

It remains to show that $\ker(\varphi)=\langle \{{j_{\Cal U}}_*(\gamma){j_{\Cal V}}_*(\gamma^{-1}) : \gamma\in\pi_1(W)\}\rangle^N={\Cal H}$.

\msk

The containment $\supseteq$ follows by showing that $\varphi({j_{\Cal U}}_*(\gamma){j_{\Cal V}}_*(\gamma^{-1}))=1$
in $\pi_1(X)$; but $\varphi({j_{\Cal U}}_*(\gamma){j_{\Cal V}}_*(\gamma^{-1}))
={i_{\Cal U}}_*{j_{\Cal U}}_*(\gamma)\cdot {i_{\Cal V}}_*{j_{\Cal V}}_*(\gamma^{-1}))
=(i_{\Cal U}\circ j_{\Cal U})_*(\gamma)\cdot (i_{\Cal V}\circ j_{\Cal V})_*(\gamma^{-1})$.
But since both $(i_{\Cal U}\circ j_{\Cal U})$ and $(i_{\Cal V}\circ j_{\Cal V})$ are equal to the
inclusion map $\iota:{\Cal U}\cap{\Cal V}\ra X$, 
we have $\varphi({j_{\Cal U}}_*(\gamma){j_{\Cal V}}_*(\gamma^{-1}))=\iota_*(\gamma)\cdot \iota_*(\gamma^{-1})
=\iota_*(\gamma\cdot\gamma^{-1})=\iota_*(1)=1$.

\msk

For the opposite containment, suppose that
$\varphi([\gamma_1]\cdot \cdots \cdot [\gamma_n])=[\gamma_1*\cdots *\gamma_n]=1$
where each $\gamma_i$ maps into ${\Cal U}$ or ${\Cal V}$. Then we have a homotopy
rel endpoints $H:I\times I\ra X$ from $\gamma_1*\cdots *\gamma_n$ to the constant
map at $x_0$. We wish to show that $[\gamma_1]\cdot \cdots \cdot [\gamma_n]$ is equal, 
in $\pi_1({\Cal U})*\pi_1({\Cal V})$, to a product of conjugates of elements of the 
form ${j_{\Cal U}}_*(\gamma){j_{\Cal V}}_*(\gamma^{-1})$. 

\msk

As before, we find a Lebesgue number $\epsilon >0$ for the open cover $H^{-1}({\Cal U}),H^{-1}({\Cal V})$ of $I\times I$,
so that there is an $N$ so that every ${{1}\over{N}}\times{{1}\over{N}}$ subsquare in $I\times I$ maps
into ${\Cal U}$ or ${\Cal V}$ under $H$. Partitioning $I\times I$ into $N^2$ squares, these squares form $N$ horizontal
strips, each of height $1/N$. 

\ssk

The proof proceeds by showing that each of the loops
$\alpha_i:t\mapsto H(t,(N-i)/N)$ lies in ${\Cal H}$, by induction on $i$; the initial case $i=0$ is the constant loop,
which is immediate, while the case $i=N$ is our desired result. For the inductive step we show that if the
bottom of one of our horizontal strips lies in ${\Cal H}$ then the top does, as well.

\msk

The one delicate point in what follows is that $\gamma_1*\cdots *\gamma_n$ is given as an explicit product of loops in 
${\Cal U},{\Cal V}$, and we must remember in our deformations to always be dealing with loops into these two sets, in 
order to show that \underbar{in} \underbar{the} \underbar{free} \underbar{product} $\gamma_1*\cdots *\gamma_n$
is a product of conjugates of the form we desire. 
\vfill
\eject

We have our domain cut into an $n\times n$ grid, so that each subsquare maps into ${\Cal U}$ or ${\Cal V}$; pick one for 
each. As before, we amalgamate horizontally adjacent squares if they are labeled the same; then each horizontal strip
is cut into rectangles which alternate their label. The vertical edges of the rectangles then map into
${\Cal U})\cap{\Cal V}$, and so their endpoints do, as well. Join each of these endpoints to our basepoint by 
paths $\eta_{i,j}$. Then the top and bottoms of the strips are

$\gamma_{i,1}*\cdots \gamma_{i,m}\simeq 
\gamma_{i,1}*(\eta_{i,1}*\overline{\eta_{i,1}})*\gamma_{i,2}*\cdots *\gamma_{i,m-1}*(\eta_{i,m-1}*\overline{\eta_{i,m-1}})*\gamma_{i,m}$

$\simeq
(\gamma_{i,1}*\eta_{i,1})*(\overline{\eta_{i,1}}*\gamma_{i,2}*\eta_{i,2})*\cdots *\gamma_{i,m-1}*\eta_{i,m-1})*(\overline{\eta_{i,m-1}}*\gamma_{i,m})$
is a product of loops in ${\Cal U})$ and ${\Cal V}$.

\msk

Our \underbar{real} inductive hypothesis is that the bottom of thr strip has a partition and collection of paths to the basepoint in
${\Cal U}\cap{\Cal V}$ from those partition points so that the resulting loops (as above), as an element of 
$\pi_1({\Cal U}))*\pi_1({\Cal V})$, lies in ${\Cal H}$. (Note that this is immediate for the bottom of the square; it is literally
written as products of loops and their inverses (since we go up and down each path).) 
The first point is that the partitioning \cltr{coming from the strip and the
paths chosen there} \underbar{also} express the bottom edge as an element of ${\Cal H}$. This is because we can add the paths from 
one to the other without changing the element (the added paths map into ${\Cal U}\cap{\Cal V}$, so change nothing, up to homotopy), but then 
when we change perspectives between the two sets of paths some of the sub-edges may change their label, and these subedges then map into 
${\Cal U}\cap{\Cal V}$. But this change of label is precisely the same as inserting an element of the form 
$\alpha={j_{\Cal U}}_*(\gamma)^{-1}{j_{\Cal V}}_*(\gamma)$ (or its inverse) into our group element; multiplication by
$\alpha$ literally removes ${j_{\Cal U}}_*(\gamma)$ and replaces it with ${j_{\Cal V}}_*(\gamma)$; its inverse does the reverse.

\vfill
\eject

Dealing with the horizontal strip itself is more straightforward. The top of the strip is homotopic to the bottom, and is equal, 
in $\pi_1({\Cal U}))*\pi_1({\Cal V})$ to the bottom strip with pairs of loops inserted, namely (top path)*(vertical edge)*(bottom path)
and its inverse. These inserted loops all map into ${\Cal U}\cap{\Cal V}$, but it is \underbar{thought} \underbar{of} as one copy
mapping into ${\Cal U}$ and the other into ${\Cal V}$, to match the subset that each subrectangle maps into. Again, this 
amounts to inserting an element of ${\Cal H}$ into an element of $\pi_1({\Cal U}))*\pi_1({\Cal V})$ which, by the inductive
hypothesis, also lies in ${\Cal H}$. Therefore the top, partitioned as the subrectangles dictate, lies in ${\Cal H}$.

\msk

This insertion process leaves us in ${\Cal H}$, because:

\ssk

If $uv\in{\Cal H}$ and $w\in{\Cal H}$, then $uwv=(uwu^{-1})(uv)\in{\Cal H}$, since ${\Cal H}$ is normal, so 

$uwu^{-1}\in{\Cal H}$.

\msk

By induction, we are done: $\ker(\varphi)\subseteq{\Cal H}$, so $\ker(\varphi)={\Cal H}$, as desired.

\vfill
\end

