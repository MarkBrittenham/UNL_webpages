



\magnification=2000
\overfullrule=0pt
\parindent=0pt

\nopagenumbers

\input amstex

%\voffset=-.6in
%\hoffset=-.5in
%\hsize = 7.5 true in
%\vsize=10.4 true in

\voffset=1.8 true in
\hoffset=-.6 true in
\hsize = 10.2 true in
\vsize=8 true in

\input colordvi



\def\cltr{\Red}		  % Red  VERY-Approx PANTONE RED
\def\cltb{\Blue}		  % Blue  Approximate PANTONE BLUE-072
\def\cltg{\PineGreen}	  % ForestGreen  Approximate PANTONE 349
\def\cltp{\DarkOrchid}	  % DarkOrchid  No PANTONE match
\def\clto{\Orange}	  % Orange  Approximate PANTONE ORANGE-021
\def\cltpk{\CarnationPink}	  % CarnationPink  Approximate PANTONE 218
\def\clts{\Salmon}	  % Salmon  Approximate PANTONE 183
\def\cltbb{\TealBlue}	  % TealBlue  Approximate PANTONE 3145
\def\cltrp{\RoyalPurple}	  % RoyalPurple  Approximate PANTONE 267
\def\cltp{\Purple}	  % Purple  Approximate PANTONE PURPLE

\def\cgy{\GreenYellow}     % GreenYellow  Approximate PANTONE 388
\def\cyy{\Yellow}	  % Yellow  Approximate PANTONE YELLOW
\def\cgo{\Goldenrod}	  % Goldenrod  Approximate PANTONE 109
\def\cda{\Dandelion}	  % Dandelion  Approximate PANTONE 123
\def\capr{\Apricot}	  % Apricot  Approximate PANTONE 1565
\def\cpe{\Peach}		  % Peach  Approximate PANTONE 164
\def\cme{\Melon}		  % Melon  Approximate PANTONE 177
\def\cyo{\YellowOrange}	  % YellowOrange  Approximate PANTONE 130
\def\coo{\Orange}	  % Orange  Approximate PANTONE ORANGE-021
\def\cbo{\BurntOrange}	  % BurntOrange  Approximate PANTONE 388
\def\cbs{\Bittersweet}	  % Bittersweet  Approximate PANTONE 167
%\def\creo{\RedOrange}	  % RedOrange  Approximate PANTONE 179
\def\cma{\Mahogany}	  % Mahogany  Approximate PANTONE 484
\def\cmr{\Maroon}	  % Maroon  Approximate PANTONE 201
\def\cbr{\BrickRed}	  % BrickRed  Approximate PANTONE 1805
\def\crr{\Red}		  % Red  VERY-Approx PANTONE RED
\def\cor{\OrangeRed}	  % OrangeRed  No PANTONE match
\def\paru{\RubineRed}	  % RubineRed  Approximate PANTONE RUBINE-RED
\def\cwi{\WildStrawberry}  % WildStrawberry  Approximate PANTONE 206
\def\csa{\Salmon}	  % Salmon  Approximate PANTONE 183
\def\ccp{\CarnationPink}	  % CarnationPink  Approximate PANTONE 218
\def\cmag{\Magenta}	  % Magenta  Approximate PANTONE PROCESS-MAGENTA
\def\cvr{\VioletRed}	  % VioletRed  Approximate PANTONE 219
\def\parh{\Rhodamine}	  % Rhodamine  Approximate PANTONE RHODAMINE-RED
\def\cmu{\Mulberry}	  % Mulberry  Approximate PANTONE 241
\def\parv{\RedViolet}	  % RedViolet  Approximate PANTONE 234
\def\cfu{\Fuchsia}	  % Fuchsia  Approximate PANTONE 248
\def\cla{\Lavender}	  % Lavender  Approximate PANTONE 223
\def\cth{\Thistle}	  % Thistle  Approximate PANTONE 245
\def\corc{\Orchid}	  % Orchid  Approximate PANTONE 252
\def\cdo{\DarkOrchid}	  % DarkOrchid  No PANTONE match
\def\cpu{\Purple}	  % Purple  Approximate PANTONE PURPLE
\def\cpl{\Plum}		  % Plum  VERY-Approx PANTONE 518
\def\cvi{\Violet}	  % Violet  Approximate PANTONE VIOLET
\def\clrp{\RoyalPurple}	  % RoyalPurple  Approximate PANTONE 267
\def\cbv{\BlueViolet}	  % BlueViolet  Approximate PANTONE 2755
\def\cpe{\Periwinkle}	  % Periwinkle  Approximate PANTONE 2715
\def\ccb{\CadetBlue}	  % CadetBlue  Approximate PANTONE (534+535)/2
\def\cco{\CornflowerBlue}  % CornflowerBlue  Approximate PANTONE 292
\def\cmb{\MidnightBlue}	  % MidnightBlue  Approximate PANTONE 302
\def\cnb{\NavyBlue}	  % NavyBlue  Approximate PANTONE 293
\def\crb{\RoyalBlue}	  % RoyalBlue  No PANTONE match
%\def\cbb{\Blue}		  % Blue  Approximate PANTONE BLUE-072
\def\cce{\Cerulean}	  % Cerulean  Approximate PANTONE 3005
\def\ccy{\Cyan}		  % Cyan  Approximate PANTONE PROCESS-CYAN
\def\cpb{\ProcessBlue}	  % ProcessBlue  Approximate PANTONE PROCESS-BLUE
\def\csb{\SkyBlue}	  % SkyBlue  Approximate PANTONE 2985
\def\ctu{\Turquoise}	  % Turquoise  Approximate PANTONE (312+313)/2
\def\ctb{\TealBlue}	  % TealBlue  Approximate PANTONE 3145
\def\caq{\Aquamarine}	  % Aquamarine  Approximate PANTONE 3135
\def\cbg{\BlueGreen}	  % BlueGreen  Approximate PANTONE 320
\def\cem{\Emerald}	  % Emerald  No PANTONE match
%\def\cjg{\JungleGreen}	  % JungleGreen  Approximate PANTONE 328
\def\csg{\SeaGreen}	  % SeaGreen  Approximate PANTONE 3268
\def\cgg{\Green}	  % Green  VERY-Approx PANTONE GREEN
\def\cfg{\ForestGreen}	  % ForestGreen  Approximate PANTONE 349
\def\cpg{\PineGreen}	  % PineGreen  Approximate PANTONE 323
\def\clg{\LimeGreen}	  % LimeGreen  No PANTONE match
\def\cyg{\YellowGreen}	  % YellowGreen  Approximate PANTONE 375
\def\cspg{\SpringGreen}	  % SpringGreen  Approximate PANTONE 381
\def\cog{\OliveGreen}	  % OliveGreen  Approximate PANTONE 582
\def\pars{\RawSienna}	  % RawSienna  Approximate PANTONE 154
\def\cse{\Sepia}		  % Sepia  Approximate PANTONE 161
\def\cbr{\Brown}		  % Brown  Approximate PANTONE 1615
\def\cta{\Tan}		  % Tan  No PANTONE match
\def\cgr{\Gray}		  % Gray  Approximate PANTONE COOL-GRAY-8
\def\cbl{\Black}		  % Black  Approximate PANTONE PROCESS-BLACK
\def\cwh{\White}		  % White  No PANTONE match


\loadmsbm

\input epsf

\def\ctln{\centerline}
\def\u{\underbar}
\def\ssk{\smallskip}
\def\msk{\medskip}
\def\bsk{\bigskip}
\def\hsk{\hskip.1in}
\def\hhsk{\hskip.2in}
\def\dsl{\displaystyle}
\def\hskp{\hskip1.5in}

\def\lra{$\Leftrightarrow$ }
\def\ra{\rightarrow}
\def\mpto{\logmapsto}
\def\pu{\pi_1}
\def\mpu{$\pi_1$}
\def\sig{\Sigma}
\def\msig{$\Sigma$}
\def\ep{\epsilon}
\def\sset{\subseteq}
\def\del{\partial}
\def\inv{^{-1}}
\def\wtl{\widetilde}
%\def\lra{\Leftrightarrow}
\def\del{\partial}
\def\delp{\partial^\prime}
\def\delpp{\partial^{\prime\prime}}
\def\sgn{{\roman{sgn}}}
\def\wtih{\widetilde{H}}
\def\bbz{{\Bbb Z}}
\def\bbr{{\Bbb R}}
\def\bbq{{\Bbb Q}}
\def\bbc{{\Bbb C}}
\def\hdsk{\hskip.7in}
\def\hdskb{\hskip.9in}
\def\hdskc{\hskip1.1in}
\def\hdskd{\hskip1.3in}
\def\Hom{\text{Hom}}
\def\Ext{\text{Ext}}
\def\larr{\leftarrow}



{\bf The Cup Product:} The one big difference between homology and cohomology is that
cohomology can be endowed with a ``natural'' product, making cohomology, specifically
$\oplus_n H^n(X;R)$ into a ring. (Any group can be given ``unnatural'' products, like the
product of any two elements are $0$.) 

\ssk

In order to multiply cochains we will need to multiply their coefficients, and so do to 
this right we need to use a ring for our coefficients, instead of just an abelian group. On the other hand,
most of our coefficient groups have been the additve groups of rings, anyway. Popular choices are $\bbz, \bbz_n$ for 
various $n$ ($n=2$ is popular), and $\bbq$.

\msk

The basic idea is that cochains $\varphi\in \Hom(C_k(X),R),\psi\in\Hom(C_\ell(X),R)$ can be used to build
a $(k+\ell)$-cochain $\varphi\cup\psi$, defined on a singular $(k+\ell)$-simplex $\sigma:\Delta^{k+\ell}=[v_0,\ldots,v_{k+\ell}]\ra X$
by 

\ssk

\ctln{$(\varphi\smile\psi)(\sigma)=\varphi(\sigma|_{[v_0,\ldots,v_k]})\cdot \psi(\sigma|_{[v_{k},\ldots,v_{k+\ell}]})$}

\ssk

where the multiplication on the right takes place in $R$. [Technically, these restricted maps have the wrong
domains; they aren't the \u{standard} $k$- and $\ell$-simplices. But we just pre-compose with the
``obvious'' maps from the standard simplices.] This {\it cup product} will induce a product on cohomology, 
by the following fact:

\bsk

... wait for it ...

\vfill
\eject

$\delta(\varphi\smile\psi)=\delta\varphi\smile\psi+(-1)^k\varphi\smile\delta\psi$ .

\ssk

This is essentially a routine computation. For $\sigma:\Delta^{k+\ell+1}=[v_0,\ldots ,v_{k+\ell+1}]\ra X$,

\ssk

$(\delta\varphi\smile\psi)(\sigma)=\delta\varphi(\sigma|_{[v_0,\ldots ,v_{k+1}]})\psi(\sigma|_{[v_{k+1},\ldots ,v_{k+\ell+1}]})$

\hskip.2in $=\varphi(\del\sigma|_{[v_0,\ldots ,v_{k+1}]})\psi(\sigma|_{[v_{k+1},\ldots ,v_{k+\ell+1}]})$

\hskip.2in $=\varphi(\sum_{i=0}^{k+1}(-1)^i\sigma|_{[v_0,\ldots ,\hat{v_i},\ldots ,v_{k+1}]})\psi(\sigma|_{[v_{k+1},\ldots ,v_{k+\ell+1}]})$

\hskip.2in $=\sum_{i=0}^{k+1}(-1)^i\varphi(\sigma|_{[v_0,\ldots ,\hat{v_i},\ldots ,v_{k+1}]})\psi(\sigma|_{[v_{k+1},\ldots ,v_{k+\ell+1}]})$
= (*), while

\ssk

$(-1)^k(\varphi\smile\delta\psi)(\sigma)
=(-1)^k\varphi(\sigma|_{[v_0,\ldots ,v_{k}]})\delta\psi(\sigma|_{[v_{k},\ldots ,v_{k+\ell+1}]})$

\hskip.2in$=(-1)^k\varphi(\sigma|_{[v_0,\ldots ,v_{k}]})\psi(\del\sigma|_{[v_{k},\ldots ,v_{k+\ell+1}]})$

\hskip.2in $=(-1)^k\varphi(\sigma|_{[v_0,\ldots ,v_{k}]})\psi(\sum_{i=k}^{k+\ell+1}(-1)^{i-k}\sigma|_{[v_{k},\ldots,\hat{v_i},\ldots ,v_{k+\ell+1}]})$

\hskip.2in $=\sum_{i=k}^{k+\ell+1}(-1)^{i}\varphi(\sigma|_{[v_0,\ldots ,v_{k}]})\psi(\sigma|_{[v_{k},\ldots,\hat{v_i},\ldots ,v_{k+\ell+1}]})$
= (**). But then


\ssk

$\delta(\varphi\smile\psi)(\sigma)
=\varphi\smile\psi(\del\sigma)=(\varphi\smile\psi)(\sum_{i=0}^{k+\ell+1}(-1)^i\sigma|_{[v_0,\ldots,\hat{v_i},\ldots,v_{k+\ell+1}]})$

\hskip.2in$=\sum_{i=0}^{k+\ell+1}(-1)^i\varphi\smile\psi(\sigma|_{[v_0,\ldots,\hat{v_i},\ldots,v_{k+\ell+1}]})$

\hskip.2in$=\sum_{i=0}^{k}(-1)^i\varphi\smile\psi(\sigma|_{[v_0,\ldots,\hat{v_i},\ldots,v_{k+\ell+1}]})
+\sum_{i=k+1}^{k+\ell+1}(-1)^i\varphi\smile\psi(\sigma|_{[v_0,\ldots,\hat{v_i},\ldots,v_{k+\ell+1}]})$

\hskip.2in$=\sum_{i=0}^{k}(-1)^i\varphi(\sigma|_{[v_0,\ldots,\hat{v_i},\ldots,v_{k+1}]})\psi(\sigma|_{[v_{k+1},\ldots,v_{k+\ell+1}]})$

\hfill $+\sum_{i=k+1}^{k+\ell+1}(-1)^i\varphi(\sigma|_{[v_0,\ldots,v_k]})\psi(\sigma|_{[v_{k},\ldots,\hat{v_i},\ldots,v_{k+\ell+1}]})$

\hskip.2in$\buildrel{\text{!}}\over{=}\sum_{i=0}^{k+1}(-1)^i\varphi(\sigma|_{[v_0,\ldots,\hat{v_i},\ldots,v_{k+1}]})\psi(\sigma|_{[v_{k+1},\ldots,v_{k+\ell+1}]})$

\hfill $+\sum_{i=k}^{k+\ell+1}(-1)^i\varphi(\sigma|_{[v_0,\ldots,v_k]})\psi(\sigma|_{[v_{k},\ldots,\hat{v_i},\ldots,v_{k+\ell+1}]})$

= (*) + (**), since

\hskip.5in $(-1)^{k+1}\varphi(\sigma|_{[v_0,\ldots,\ldots,v_{k},\hat{v_{k+1}}]})\psi(\sigma|_{[v_{k+1},\ldots,v_{k+\ell+1}]})$

\hskip2in $+(-1)^k\varphi(\sigma|_{[v_0,\ldots,v_k]})\psi(\sigma|_{[\hat{v_k},v_{k+1},\ldots,v_{k+\ell+1}]}) = 0$.

\vfill
\eject

\ctln{$\delta(\varphi\smile\psi)=\delta\varphi\smile\psi+(-1)^k\varphi\smile\delta\psi$}

\ssk

This tells us several things. first, if $\varphi$ and $\psi$ are both cocycles, then 
$\delta\varphi = 0$ and $\delta\psi = 0$, so $\delta(\varphi\smile\psi) = 0\smile\psi \pm \varphi \pm 0 = 0\pm 0 =0$, 
so $\varphi\smile\psi$ is also a cocycle. therefore, the map

\ssk

\ctln{$\cup: C^k(X;R)\times C^\ell(X;R)\ra C^{k+\ell}(X;R)$}

\ssk

induces a map 

\ssk

\ctln{$Z^k\times Z^\ell\ra Z^{k+\ell}\ra H^{k+\ell}(X;R)$}

\ssk

and if either $\varphi=\delta f$ or $\psi=\delta g$, then, e.g., 
$\delta(f\smile\psi)= (\delta f)\smile\psi \pm f\smile\delta\psi = \varphi\smile\psi + f\smile 0 = \varphi\smile\psi$
(assuming that $\psi\in Z^\ell$), and similarly $\delta((-1)^k\varphi\smile g) = \varphi\smile\psi$, so 
$B^k\times Z^\ell \cup Z^k\times B^\ell$ maps into $B^{k+\ell}$, and so there is an induced map

\ssk

\ctln{$\smile:H^k(X;R)\times H^\ell(X;R)\ra H^{k+\ell}(X;R)$}

\ssk

which is what we call the {\it cup product} on cohomology.

\ssk

This product turns $H^*(X;R) = \oplus_n H^n(X;R)$ into (what we will call a {\it graded}) ring;
it is associative and distributive since it is on the level of cochains;

\ssk

$(\varphi\smile\psi)\smile\theta(\sigma) 
= (\varphi(\sigma|_{[v_0,\ldots,v_k]})\psi(\sigma|_{[v_k,\ldots,v_{k+\ell}]}))\theta(\sigma|_{[v_{k+\ell},\ldots,v_{k+\ell+m}]})$

$= \varphi(\sigma|_{[v_0,\ldots,v_k]})(\psi(\sigma|_{[v_k,\ldots,v_{k+\ell}]})\theta(\sigma|_{[v_{k+\ell},\ldots,v_{k+\ell+m}]}))
=(\varphi\smile\psi)\smile\theta(\sigma)$

\ssk

and distributivity is similar. If the coefficient ring $R$ has an identity $1$, then so does $H^*(X;R)$; the
class $[1]\in H^0(X;R)$ which sends each singular $0$-simplex to $1$ is the identity element.

\vfill
\eject

If we were to work with the simplicial cochain complex, we could define the exact same product, and so the
restriction of singular cochains to simplical ones can be viewed as a ring homomorphism, and so the
isomorphism between singular and simplicial cohomology is in fact a ring isomorphism. The product structure
is also ``natural'' with respect to the maps induced by continuous maps,
so $f:X\ra Y$ induces a ring homomorphism $f^*:H^*(Y;R)\ra H^*(X;R)$. Taking this to its logical conclusion,
any homotopy equivalence induces a ring isomorphism between the respective cohomology rings.

\msk

The cup product is not quite commutative; the precise statement is that, if $R$ is commutative, for $\varphi\in H^k(X;R)$
and $\psi\in H^{\ell}(X;R)$, then 

\ctln{$\varphi\smile\psi = (-1)^{k\ell}\psi\smile\varphi\in H^{k+\ell}(X;R)$.}

 We omit the proof.
Such a product is, in some circles, called {\it graded commutative}.

\ssk

As a sample computation of the cup product for a space, we look at the closed orientable surfaces of genus $g\geq 1$, $F_g$.
By universal coefficients, since $H_*(F_g;\bbz)$ is free abelian, all $\Ext$ groups will be $0$, so we have 
$H^*(F_g;R)\cong R$ in dimensions $0$ and $2$, and $\cong R^{2g}$ in dimension $1$; all other groups are $0$. 
So the only non-trivial
cup products will occur between $1$-dimensional classes. Thinking in terms of cellular cohomology, using the standard 
CW-structure on $F_g$ as the quotient of a $2g$-gon $D$ with edges identified in the pattern 
$a_1,b_1,\overline{a_1},\overline{b_1},a_2,b_2,\ldots$. Identifying $H^1(F_g;R)\cong \Hom(H_1(F_g),R)$ 
with $R^{2g}$ as an assignment of elements of $R$ to each of the standard basis elements $a_i,b_i$ of $H_1(F_g)$,
the cup product $\varphi\smile\psi$ can be identified with the value it assigns to the generator $[D/\del D]$ of $H^2(F_g;R)$.
It suffices to compute the cup products among the basis elements $\alpha_i=a_i^*,\beta_j=b_j^*$ dual to our basis for $H_1(F_g;R)$.

\ssk

We didn't actually \u{define} cup products for cellular cohomology (except through its isomorphism with singular 
and simplicial cohomology), but we can see by the isomorphisms that simplicially, writing $F_g$ as
a $\Delta$-complex by cutting $D$ into $2g$ 2-simplices by coning each edge to a vertex in the center of $D$, we have the same
basis for $H_1(F_g;R)$ and hence for $H^1(F_g;R)$, and the generator for $H_2(F_g;R)$ is the (oriented) sum of the
$2g$ 2-simplices formed (since these add up to $D$). So to compute cup products, it suffices to determine what value
of each of $\alpha_i\cup\alpha_j,\alpha_i\cup\beta_j,\beta_i\cup\alpha_j,\beta_i\cup\beta_j$ take on these sums of
simplices. 

\ssk

Note that on the level of cochains, we must also assign the $\alpha_i,\beta_j$ values
on the 1-simplices added in the interior of $D$. In order to be sure we are describing a cocycle, the resulting values must 
sum to zero around every one of the 2-simplices. The figures give one set of choices:

\ssk

\ctln{\vbox{\hsize=4in
\leavevmode
\epsfxsize=4in
\epsfbox{cls31f1.eps}}}

\ssk

With these in hand, the rest is just a bunch of calculations.

\vfill
\eject

\ctln{\vbox{\hsize=4in
\leavevmode
\epsfxsize=4in
\epsfbox{cls31f2.eps}}}

\ssk

$\alpha_i\smile\beta_i(A_i)=\alpha_i\smile\beta_i([z,w_i,v_i])=\alpha_i[z,w_i]\beta_i[w_i,v_i]=1\cdot 0=0$ , 

$\alpha_i\smile\beta_i(B_i)=\alpha_i\smile\beta_i([z,x_i,w_i])=\alpha_i[z,x_i]\beta_i[x_i,w_i]=1\cdot -1=-1$ ,

$\alpha_i\smile\beta_i(C_i)=\alpha_i\smile\beta_i([z,y_i,x_i])=\alpha_i[z,y_i]\beta_i[y_i,x_i]=0\cdot 0=0$ , and

$\alpha_i\smile\beta_i(D_i)=\alpha_i\smile\beta_i([z,v_{i+1},y_i])=\alpha_i[z,v_{i+1}]\beta_i[v_{i+1},y_i]=0\cdot 1=0$ .

All other 2-simplices have the value $0$ since all of their edges are labeled $0$ by both $\alpha_i$ and $\beta_i$.
So, summing, $\alpha_i\smile\beta_i[D]=-1$. 

[$\beta_i\smile\alpha_i = 1$ follows by another computation or graded commutativity.] 

Similar computations establish that 

\ssk

$\alpha_i\smile\alpha_j = \alpha_i\smile\beta_j = \beta_i\smile\alpha_j = \beta_i\smile\beta_j = \alpha_i\smile\alpha_i = \beta_i\smile\beta_i = 0$
for all $i,j$.

\msk

This shows, for example, that $F_2$ and $S^2\vee S^1\vee S^1\vee S^1\vee S^1$ are not homotopy equivalent,
even though they have isomorphic homology and cohomology groups (for all coefficients!). This is because the ring structure
of 

$H^*(S^2\vee S^1\vee S^1\vee S^1\vee S^1;R)$ is different; all of the cup products of $1$-dim'l classes are $0$.
[Think of the 2-sphere as $\del$(3-simplex), and note that the duals of the homology classes from the $S^1$s
can be given values $0$ on the 1-simplices of the 2-sphere. Since cup products are computed around the boundaries
of the 2-simplices, they are all $0$.]

\vfill
\eject

{\bf The Cap Product:} There is also a product which mixes cohomology and homology, and is defined in a similar way. The {\it cap product}
of a singular chain $\sigma\in C_n(X;R)$ and a singular cochain $\varphi\in C^k(X;R)$ produces a singular chain
$\sigma\frown\varphi\in C_{n-k}(X;R)$ defined by, letting $\sigma:[v_0,\ldots,v_n]\ra R$,

\ssk

\ctln{$\sigma\frown\varphi=\varphi(\sigma|_{[v_0,\ldots,v_k]})\sigma|_{[v_k,\ldots,v_{k+n}]}$}

\ssk

where $[v_k,\ldots,v_{k+n}]$ is identified with the ``standard'' $(n-k)$-simplex in the obvious way. We extend this
definition to $n$-chains $R$-linearly. A very similar
computation to the one just carried out establishes that 

\ssk

\ctln{$\del(\sigma\frown\varphi)=(-1)^n(\del\sigma\frown\varphi-\sigma\frown\delta\varphi)$}

\ssk

As before, this implies that the cap product of a cycle and a cocycle is a cycle, and if either $z$ is a boundary or $\varphi$
is a coboundary then $z\frown\varphi$ is a boundary, so we get an induced map

\ssk

\ctln{$\frown:H_n(X;R)\times H^k(X;R)\ra H_{n-k}(X;R)$}

\ssk

which is $R$-linear in each coordinate, which we call the {\it cap product}.

\msk

The cap product is also natural with respect to continuous maps, although in an odd way:
given $f:X\ra Y$ we have homomorphisms 

\ssk

\ctln{$f_*:H_*(X;R)\ra H_*(Y;R)$ and $f^*:H^*(Y;R)\ra H^*(X;R)$, and}

\ssk

\ctln{$f_*([z]\cap f^*[\varphi])=f_*[z]\cap [\varphi]$}

\ssk

The proof is similar to our argument for cup products.


\vfill
\eject

The two products, cup and cap, have relative versions, which we will not explore. There is
also a very concise expression relating the two products together:

\msk

If $z\in H_n(X;R),\varphi\in H^k(X;R)$, and $\psi\in H^\ell(X;R)$, then

\hfill $z\frown(\varphi\smile\psi) = (z\frown\varphi)\frown\psi\in H_{n-k-\ell}(X;R)$ .

\msk

The proof is immediate; the formula holds on the level of chains and cochains.
\vfill
\end

