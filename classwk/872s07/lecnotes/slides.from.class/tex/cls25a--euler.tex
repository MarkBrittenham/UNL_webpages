



\magnification=2000
\overfullrule=0pt
\parindent=0pt

\nopagenumbers

\input amstex
\input amssym

%\voffset=-.6in
%\hoffset=-.5in
%\hsize = 7.5 true in
%\vsize=10.4 true in

\voffset=1.8 true in
\hoffset=-.6 true in
\hsize = 10.2 true in
\vsize=8 true in

\input colordvi



\def\cltr{\Red}		  % Red  VERY-Approx PANTONE RED
\def\cltb{\Blue}		  % Blue  Approximate PANTONE BLUE-072
\def\cltg{\PineGreen}	  % ForestGreen  Approximate PANTONE 349
\def\cltp{\DarkOrchid}	  % DarkOrchid  No PANTONE match
\def\clto{\Orange}	  % Orange  Approximate PANTONE ORANGE-021
\def\cltpk{\CarnationPink}	  % CarnationPink  Approximate PANTONE 218
\def\clts{\Salmon}	  % Salmon  Approximate PANTONE 183
\def\cltbb{\TealBlue}	  % TealBlue  Approximate PANTONE 3145
\def\cltrp{\RoyalPurple}	  % RoyalPurple  Approximate PANTONE 267
\def\cltp{\Purple}	  % Purple  Approximate PANTONE PURPLE

\def\cgy{\GreenYellow}     % GreenYellow  Approximate PANTONE 388
\def\cyy{\Yellow}	  % Yellow  Approximate PANTONE YELLOW
\def\cgo{\Goldenrod}	  % Goldenrod  Approximate PANTONE 109
\def\cda{\Dandelion}	  % Dandelion  Approximate PANTONE 123
\def\capr{\Apricot}	  % Apricot  Approximate PANTONE 1565
\def\cpe{\Peach}		  % Peach  Approximate PANTONE 164
\def\cme{\Melon}		  % Melon  Approximate PANTONE 177
\def\cyo{\YellowOrange}	  % YellowOrange  Approximate PANTONE 130
\def\coo{\Orange}	  % Orange  Approximate PANTONE ORANGE-021
\def\cbo{\BurntOrange}	  % BurntOrange  Approximate PANTONE 388
\def\cbs{\Bittersweet}	  % Bittersweet  Approximate PANTONE 167
%\def\creo{\RedOrange}	  % RedOrange  Approximate PANTONE 179
\def\cma{\Mahogany}	  % Mahogany  Approximate PANTONE 484
\def\cmr{\Maroon}	  % Maroon  Approximate PANTONE 201
\def\cbr{\BrickRed}	  % BrickRed  Approximate PANTONE 1805
\def\crr{\Red}		  % Red  VERY-Approx PANTONE RED
\def\cor{\OrangeRed}	  % OrangeRed  No PANTONE match
\def\paru{\RubineRed}	  % RubineRed  Approximate PANTONE RUBINE-RED
\def\cwi{\WildStrawberry}  % WildStrawberry  Approximate PANTONE 206
\def\csa{\Salmon}	  % Salmon  Approximate PANTONE 183
\def\ccp{\CarnationPink}	  % CarnationPink  Approximate PANTONE 218
\def\cmag{\Magenta}	  % Magenta  Approximate PANTONE PROCESS-MAGENTA
\def\cvr{\VioletRed}	  % VioletRed  Approximate PANTONE 219
\def\parh{\Rhodamine}	  % Rhodamine  Approximate PANTONE RHODAMINE-RED
\def\cmu{\Mulberry}	  % Mulberry  Approximate PANTONE 241
\def\parv{\RedViolet}	  % RedViolet  Approximate PANTONE 234
\def\cfu{\Fuchsia}	  % Fuchsia  Approximate PANTONE 248
\def\cla{\Lavender}	  % Lavender  Approximate PANTONE 223
\def\cth{\Thistle}	  % Thistle  Approximate PANTONE 245
\def\corc{\Orchid}	  % Orchid  Approximate PANTONE 252
\def\cdo{\DarkOrchid}	  % DarkOrchid  No PANTONE match
\def\cpu{\Purple}	  % Purple  Approximate PANTONE PURPLE
\def\cpl{\Plum}		  % Plum  VERY-Approx PANTONE 518
\def\cvi{\Violet}	  % Violet  Approximate PANTONE VIOLET
\def\clrp{\RoyalPurple}	  % RoyalPurple  Approximate PANTONE 267
\def\cbv{\BlueViolet}	  % BlueViolet  Approximate PANTONE 2755
\def\cpe{\Periwinkle}	  % Periwinkle  Approximate PANTONE 2715
\def\ccb{\CadetBlue}	  % CadetBlue  Approximate PANTONE (534+535)/2
\def\cco{\CornflowerBlue}  % CornflowerBlue  Approximate PANTONE 292
\def\cmb{\MidnightBlue}	  % MidnightBlue  Approximate PANTONE 302
\def\cnb{\NavyBlue}	  % NavyBlue  Approximate PANTONE 293
\def\crb{\RoyalBlue}	  % RoyalBlue  No PANTONE match
%\def\cbb{\Blue}		  % Blue  Approximate PANTONE BLUE-072
\def\cce{\Cerulean}	  % Cerulean  Approximate PANTONE 3005
\def\ccy{\Cyan}		  % Cyan  Approximate PANTONE PROCESS-CYAN
\def\cpb{\ProcessBlue}	  % ProcessBlue  Approximate PANTONE PROCESS-BLUE
\def\csb{\SkyBlue}	  % SkyBlue  Approximate PANTONE 2985
\def\ctu{\Turquoise}	  % Turquoise  Approximate PANTONE (312+313)/2
\def\ctb{\TealBlue}	  % TealBlue  Approximate PANTONE 3145
\def\caq{\Aquamarine}	  % Aquamarine  Approximate PANTONE 3135
\def\cbg{\BlueGreen}	  % BlueGreen  Approximate PANTONE 320
\def\cem{\Emerald}	  % Emerald  No PANTONE match
%\def\cjg{\JungleGreen}	  % JungleGreen  Approximate PANTONE 328
\def\csg{\SeaGreen}	  % SeaGreen  Approximate PANTONE 3268
\def\cgg{\Green}	  % Green  VERY-Approx PANTONE GREEN
\def\cfg{\ForestGreen}	  % ForestGreen  Approximate PANTONE 349
\def\cpg{\PineGreen}	  % PineGreen  Approximate PANTONE 323
\def\clg{\LimeGreen}	  % LimeGreen  No PANTONE match
\def\cyg{\YellowGreen}	  % YellowGreen  Approximate PANTONE 375
\def\cspg{\SpringGreen}	  % SpringGreen  Approximate PANTONE 381
\def\cog{\OliveGreen}	  % OliveGreen  Approximate PANTONE 582
\def\pars{\RawSienna}	  % RawSienna  Approximate PANTONE 154
\def\cse{\Sepia}		  % Sepia  Approximate PANTONE 161
\def\cbr{\Brown}		  % Brown  Approximate PANTONE 1615
\def\cta{\Tan}		  % Tan  No PANTONE match
\def\cgr{\Gray}		  % Gray  Approximate PANTONE COOL-GRAY-8
\def\cbl{\Black}		  % Black  Approximate PANTONE PROCESS-BLACK
\def\cwh{\White}		  % White  No PANTONE match


\loadmsbm

\input epsf

\def\ctln{\centerline}
\def\u{\underbar}
\def\ssk{\smallskip}
\def\msk{\medskip}
\def\bsk{\bigskip}
\def\hsk{\hskip.1in}
\def\hhsk{\hskip.2in}
\def\dsl{\displaystyle}
\def\hskp{\hskip1.5in}

\def\lra{$\Leftrightarrow$ }
\def\ra{\rightarrow}
\def\mpto{\logmapsto}
\def\pu{\pi_1}
\def\mpu{$\pi_1$}
\def\sig{\Sigma}
\def\msig{$\Sigma$}
\def\ep{\epsilon}
\def\sset{\subseteq}
\def\del{\partial}
\def\inv{^{-1}}
\def\wtl{\widetilde}
%\def\lra{\Leftrightarrow}
\def\del{\partial}
\def\delp{\partial^\prime}
\def\delpp{\partial^{\prime\prime}}
\def\sgn{{\roman{sgn}}}
\def\wtih{\widetilde{H}}
\def\bbz{{\Bbb Z}}
\def\bbr{{\Bbb R}}
\def\hdsk{\hskip.7in}
\def\hdskb{\hskip.9in}
\def\hdskc{\hskip1.1in}
\def\hdskd{\hskip1.3in}




{\bf Euler characteristics:} Given a finite $\Delta$-complex $X$ (that is, having a 
finite number of simplices), the number of simplices in each dimension (which is
also the rank of each of the groups in the simplicial chain complex) are not an invariant
of the space, but, it turns out, the alternating sum $\chi(X)=\sum (-1)^i \dim C_i^\Delta(X)$
\u{is} a topological invariant, called the {\it Euler characteristic} of $X$. This also
is a consequence of the isomrphism between singular and simplicial homology (since it
implies that the simplicial homology groups are topological invariants), together with
the following result:

\ssk

{\bf Proposition:} If $0\ra C_n\ra C_{n-1}\ra \cdots \ra C_1\ra C_0 \ra 0$ is a chain complex of abelian
groups, with each chain group having finite rank, and having homology groups $H_i=H_i({\Cal C})$,
then $\sum (-1)^i {\text rank}(C_i)=\sum (-1)^i {\text rank}(H_i)$.

\ssk

The proof consists of noting that if we set $Z_i=\ker \del_i\subseteq C_i$ and
$B_i={\text im} \del_{i+1}\subseteq Z_i$, then by one of the Noether isomorphism
theorems $B_{i-1}\cong C_i/Z_i$, while $H_i=Z_i/B_i$ by definition, so 

\ssk

${\text rank}(B_{i-1})={\text rank}(C_i) - {\text rank}(Z_i)$, so 
${\text rank}(C_i)={\text rank}(Z_i)+{\text rank} (B_{i-1})$, 

while 
${\text rank}(H_i)={\text rank}(Z_i)-{\text rank}(B_i)$,
so 

\ssk

$\sum (-1)^i {\text rank}(C_i)-\sum (-1)^i {\text rank}(H_i)$

$=\sum (-1)^i [({\text rank}(Z_i)+{\text rank} (B_{i-1}))-({\text rank}(Z_i)-{\text rank}(B_i))]$

$=\sum (-1)^i [{\text rank} (B_{i-1})+{\text rank}(B_i)]$

$=\sum (-1)^i[{\text rank}(B_i)-(-1)^{i-1}{\text rank} (B_{i-1})]$

$=\sum (-1)^i{\text rank}(B_i)-\sum (-1)^i{\text rank}(B_i)=0$, as desired.

\msk

Therefore $\chi(X)=\sum (-1)^i {\text rank}(H_i^\Delta(X))=\sum (-1)^i {\text rank}(H_i(X))$
depends only on $X$, not on a particular $\Delta$-complex structure. 

\vfill
\eject

As a special case,
since $\chi(D^n)=1$ (since the only non-trivial homnology group of $D^n$ is $H_0(D^n)=\bbz$),
we obtain Euler's formula: if a 2-disk is triangulated with $v$ vertices, $e$ edges, and $f$
faces, then $v-e+f=1$. Since the singular homology groups are invariant under homotopy
equivalence, we also have that every contractible finite $\Delta$-complex has Euler characteristic
$1$. So a quick (but only partially successful) way to show that a connected $\Delta$-complex $X$
is not contractible is to show that $\chi(X)\neq 1$.

\msk

We can also extend these results to finite CW-complexes; the alternating sum of the number of cells
in each dimension is an invariant of the underlying topological space. To show this, we introduce
yet ``another'' homology theory, {\it celluluar homology}. The interesting feature of this
is that the chain groups are singular homology groups! Specifically, if $X$ is a CW-complex with
$k$-skeleta $X^{(k)}$, then the relative singular homology group
$H_k(X^{(k)},X^{(k-1)})\cong \wtih_k(X^{(k)}/X^{(k-1)})\cong\wtih_k(\vee S^k)\cong \oplus\bbz$,
with one $\bbz$ summand for each $k$-cell of $X$. From the exact sequence of the triple
$(X^{(k)},X^{(k-1)},X^{(k-2)})$ we have a connecting homomorphism 

$d_k: H_k(X^{(k)},X^{(k-1)})\ra H_{k-1}(X^{(k-1)},X^{(k-2)})$, which as with the sequence
for a pair, ``really'' takes a relative cycle $[z]$, under

\ssk

\ctln{$C_k(X^{(k)},X^{(k-1)})\buildrel{\del}\over\ra C_{k-1}(X^{(k-1)})\buildrel{p}\over\ra 
C_{k-1}(X^{(k-1)},X^{(k-2)})$}

\ssk

 to  the coset of $[\del z]$. Applying this twice, 

\ssk

\ctln{$H_k(X^{(k)},X^{(k-1)})\ra H_{k-1}(X^{(k-1)},X^{(k-2)})\ra H_{k-2}(X^{(k-2)},X^{(k-3)})$,}

\ssk

is therefore zero, since it amounts to taking the ordinary boundary twice. So we have a chain
complex $\{C_n^{CW}(X),d_n\}=\{H_n(X^{(n)},X^{(n-1)}),d_n\}$, called the {\it cellular chain
complex} of $X$, whose homology groups are the {\it cellular homology groups of} $X$. 

\vfill
\eject

The ``Euler characteristic'' of the cellular complex is the alternating sum of the number
of $n$-cells in $X$, which by the homological argument above is the same as the alternating 
sum of the ranks
of the cellular homology groups. As with simplicial homology, cellular homology is defined
in terms of a particular CW-structure on $X$, but again, we can show that it is independent
of this structure, by showing that cellular homology is isomorphic to singular homology. To do
this, we first need some basic facts:

\ssk

(a) $H_n(X^{(n+1)})\cong H_n(X)$, from the long exact sequence of the pair,

\ssk

\ctln{$0=H_{n+1}(X,X^{(n+1)})\ra H_n(X^{(n+1)})\ra H_n(X)\ra H_{n}(X,X^{(n+1)})=0$,}

\ssk

since
$H_{i}(X,X^{(n+1)})\cong \wtih_i(X/X^{(n+1)})$ and

\ssk

(b) If $Y$ is a connected CW-complex with $Y^{(n)}=$ a point, then $\wtih_i(Y)=0$ for
$i\leq n$. 

\ssk

For finite dimensional $Y$ ($Y=Y^{(k)}$ for some $k$), this follows by induction,
using the long exact sequence for the pair $(Y^{(j+1)},Y^{(j)})$, since 

\ssk

\ctln{$H_{i+1}(Y^{(j+1)},Y^{(j)})\ra \wtih_i(Y^{(j)})\ra \wtih_i(Y^{(j+1)})\ra H_i(Y^{(j+1)},Y^{(j)})$}

\ssk

is $0\ra \wtih_i(Y^{(j)})\ra \wtih_i(Y^{(j+1)})\ra 0$ for $i\neq j,j+1$, so 
$\wtih_i(Y^{(j)})\cong \wtih_i(Y^{(j+1)})$ for $i<j$ and $i>j+1$, and 
$\wtih_{j}(Y^{(j)})\ra \wtih_{j}(Y^{(j+1)})$ is surjective. On the other hand, 
$\wtih_i(Y^{(n)})=\wtih(\{*\})=0$, so
for $i\leq n$
\hskip.3in
$\wtih_i(Y)=\wtih_i(Y^{(k)})\cong \cdots \cong \wtih_i(Y^{(n+1)})\twoheadleftarrow\wtih_i(Y^{(n)})=0$,
\hskip.3in
so $\wtih_i(Y)=0$. 
The above argument also shows that for any CW-complex $X$, $\wtih_i(X^{(n)})=0$
for $i>n$, since $0=\wtih_i(X^{(0)})\cong \cdots \cong \wtih_i(X^{(n-1)})\cong\wtih_i(X^{(n)})$.

\vfill\eject

For the infinite dimensional case, we recycle an old argument to show that
since any cycle $[z]$ in $\wtih_i(Y)$ is a finite union of singular simplices, and a compact 
set meets only finitely
many cells, we can think of $z$ as a chain in some $Y^{(k)}$, where by the above it is a boundary,
so it is a boundary in $Y$, so $[z]=0$, so $\wtih_i(Y)=0$.

\ssk

The same argument above \u{also} shows that for $i<n$, $\wtih_i(X^{(n)})\cong\wtih_i(X)$ (under the 
inclusion-induced homomorphism); for finite-dimensional complexes this requires only
$\wtih_i(X)=\wtih_i(X^{(k)})\cong \cdots \cong \wtih_i(X^{(n+1)})\cong \wtih_i(X^{(n)})$,
while for infinite-dimensional complexes the same final argument shows injectivity, and
a parallel argument [any representative of $[z]\in \wtih_i(X)$ is really a chain in some
$X^{(r)}$, so is in the image of $\wtih_i(X^{(n)})\buildrel{\cong}\over\ra 
\wtih_i(X^{(r)})\ra\wtih_i(X)$] proves surjectivity. 

\vfill
\eject

With these facts in hand, we proceed to prove that $H_n^{CW}(X)\cong H_n(X)$. The basic idea is 
that $H_n^{CW}(X)$ is computed from 

\ssk

$H_{n+1}(X^{n+1},X^{n})\buildrel{d_{n+1}}\over\ra 
H_{n}(X^{n},X^{n-1})\buildrel{d_{n}}\over\ra H_{n-1}(X^{n-1},X^{n-2})$, \hfill which is really

\ctln{$H_{n+1}(X^{n+1},X^{n})\buildrel{\del_{n+1}}\over\ra 
\wtih_n(X^{n}) \buildrel{p_*}\over\ra
H_{n}(X^{n},X^{n-1})\buildrel{\del_{n}}\over\ra 
\wtih_{n-1}(X^{n-1})\buildrel{p_*}\over\ra
H_{n-1}(X^{n-1},X^{n-2})$}

\ssk

built from three different LESs of pairs! (This is, however, \u{not} exact.)

\ssk

$\wtih_n(X)\cong \wtih_n(X^{(n+1)})$, but $\wtih_n(X^{(n+1)})$ is part of a long exact sequence

\ssk

\ctln{$H_{n+1}(X^{(n+1)},X^{(n)}) 
\buildrel{\del_{n+1}}\over\ra \wtih_{n}(X^{(n)}) 
\buildrel{\iota_*}\over\ra \wtih_n(X^{(n+1)})
\ra H_{n}(X^{(n+1)},X^{(n)})$}


\ssk

with $H_{n}(X^{(n+1)},X^{(n)})\cong \wtih_{n}(X^{(n+1)}/X^{(n)})=0$ (its $n$-skeleton is a point),
so 

\ssk

$\wtih_n(X^{(n+1)})\cong \wtih_{n}(X^{(n)})/\ker(\iota_*)=
\wtih_{n}(X^{(n)})/{\text im}(\del_{n+1})$. 

\ssk

The LES of the pair

\ssk

\ctln{$\cdots \ra 0=\wtih_{n-1}(X^{(n-2)})\ra \wtih_{n-1}(X^{n-1}) \buildrel{p_*}\over\ra
H_{n-1}(X^{n-1},X^{n-2})\ra \cdots$}

\ssk

implies that the second map $p_*$ is injective, so $\ker d_n=\ker \del_n$. 

The LES sequence of the pair

\ssk

\ctln{$\cdots \ra 0=\wtih_{n}(X^{(n-1)})\ra \wtih_{n}(X^{n}) \buildrel{p_*}\over\ra
H_{n}(X^{n},X^{n-1})\ra \cdots$}

\ssk

implies that the first map $p_*$ is injective, so $p_*$ maps ${\text im}\ \del_{n+1}$
isomorphically to ${\text im}\ d_{n+1}= p_*({\text im}\ \del_{n+1})$, and $\wtih_n(X^{(n)})$
isomorphically to ${\text im}\ p_*=\ker\del_n=\ker d_n$. Consequently, $p_*:\wtih_n(X^{(n)})\ra \ker d_n$ induces an isomorphism

\ssk

\ctln{$\wtih_n(X)\cong \wtih_n(X^{(n+1)})\cong
\wtih_n(X^{(n)})/{\text im}\ \del_{n+1} \buildrel{\cong}\over\longrightarrow 
\ker d_n/{\text im}\ d_{n+1}=H_n^{CW}(X)$.}

\vfill
\eject

As before, this isomorphism immediately leads to some useful facts, both about cellular and singular
homology, that are much tougher to establish without the isomorphism:

\msk

The cellular homology groups depend only on the underlying topological space, not on the CW structure.

\ssk

The Euler characteristic of a finite CW-complex is well-defined.

\ssk

If a CW-complex $X$ has no $k$-cells, then $H_k(X)=0$ (since the $k$-th cellular chain group is $0$).

\ssk

More generally, if a CW-complex $X$ has $r$ $k$-cells, then $H_k(X)$ has a generating set with at most $r$ elements.

\ssk

If $X$ is $n$-dimensional, then $H_n(X)$ is free abelian 

(since $H_n^{CW}(X)=\ker d_n\subseteq C_n^{CW}(X)$, 
since $C_{n+1}^{CW}(X)=0$, so ${\text im}\ d_{n+1}=0$).

\ssk

If a CW-complex $X$ has no $k-2$- and $k$-cells, then $H_{k-1}(X)$ is the free abelian group on the 
$k-1$-cells of $X$ (since the chain complex is $0\ra C_{k-1}^{CW}(X)\ra 0$ at that point).

\msk

Together, the third and sixth facts give a much quicker way to compute the homology groups 
of spheres
$S^n$ (for $n\geq 2$, anyway), for example, since $S^n$ has a CW-structure with one $0$-cell
and one $n$-cell. Another quick collection of examples is $S^n\times S^m$ with $n=m\geq 2$
(1 0-cell, 2 $n$-cells, and 1 $2n$-cell) or $|n-m|\geq 2$ and $n,m\geq 2$ ( 1 each of 0-,
$n$-, $m$-, and $(n+m)$-cells).

\vfill
\eject

More involved computations require a better understanding of what the boundary maps

\ssk

\ctln{$H_k(X^{(k)},X^{(k-1)})\buildrel{d_n}\over\longrightarrow H_{k-1}(X^{(k-1)},X^{(k-2)})$}

\ssk

are. These groups have as bases, essentially, the $k$-cells $\{e^n_\alpha\}$ and $(k-1)$-cells $\{e^{n-1}_\beta\}$
of $X$. In terms of these bases, 
letting $\varphi:D^n\ra X^{(n)}$ be the characteristic map of $e^n_\alpha$,
$d_n(e^n_\alpha)=\sum n_{\alpha\beta}e^{n-1}_\beta$, where
$n_{\alpha\beta}$ counts how many times the attaching map $f=\varphi|_{\del D^n}:S^{n-1}\ra X^{(n-1)}$ 
of $e^n_\alpha$ ``passes over''
$e^{n-1}_\beta$, in the following sense:
taking the composition

\msk

$\bbz=\wtih_{n-1}(S^{n-1})\buildrel{f_*}\over\ra \wtih_{n-1}(X^{(n-1)}) 
\buildrel{p_*}\over\ra H_{n-1}(X^{(n-1)},X^{(n-2)})$

\hfill $\cong \wtih_{n-1}(X^{(n-1)}/X^{(n-2)})\cong \wtih_{n-1}(\vee_\beta S^{n-1})
\cong\oplus_\beta\wtih_{n-1}(S^{n-1})=\oplus_\beta\bbz
\buildrel{{\text proj}_\beta}\over\longrightarrow \bbz$

\msk

sends $1$ to $n_{\alpha\beta}$. (We omit the proof, but what else could it be....?)

\msk

$n_{\alpha\beta}$ is, then, what $1$ gets sent to under the map on $\wtih_{n-1}$ induced
by the map 

\ssk

\ctln{$S^{n-1}\buildrel{f}\over\ra X^{(n-1)}\ra X^{(n-1)}/X^{(n-2)}\cong\vee_\beta S^{n-1}
\buildrel{p_\beta}\over\ra S^{n-1}$}

\ssk

which is, at least in principal, computable, given enough information
about the attaching maps of our CW-complex $X$.
This number $n_{\alpha\beta}$ is called the {\it degree} of the map
$S^{n-1}\ra S^{n-1}$. [It is in fact true that maps $f:S^{n}\ra S^{n}$
are \u{determined} up to homotopy by their degree, but we will (probably)
not prove this.] For example,
for a homeomorphism $g:S^{n-1}\ra S^{n-1}$, its degree is either $1$ or $-1$ (since
the induced map on $\wtih_{n-1}$ is an isomorphism).




\vfill
\end

