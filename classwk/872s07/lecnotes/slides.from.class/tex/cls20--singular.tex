
\magnification=2000
\overfullrule=0pt
\parindent=0pt

\nopagenumbers

\input amstex

%\voffset=-.6in
%\hoffset=-.5in
%\hsize = 7.5 true in
%\vsize=10.4 true in

\voffset=1.8 true in
\hoffset=-.6 true in
\hsize = 10.2 true in
\vsize=8 true in

\input colordvi

\def\cltr{\Red}		  % Red  VERY-Approx PANTONE RED

\loadmsbm

\input epsf

\def\ctln{\centerline}
\def\u{\underbar}
\def\ssk{\smallskip}
\def\msk{\medskip}
\def\bsk{\bigskip}
\def\hsk{\hskip.1in}
\def\hhsk{\hskip.2in}
\def\dsl{\displaystyle}
\def\hskp{\hskip1.5in}

\def\lra{$\Leftrightarrow$ }
\def\ra{\rightarrow}
\def\mpto{\logmapsto}
\def\pu{\pi_1}
\def\mpu{$\pi_1$}
\def\sig{\Sigma}
\def\msig{$\Sigma$}
\def\ep{\epsilon}
\def\sset{\subseteq}
\def\del{\partial}
\def\inv{^{-1}}
\def\wtl{\widetilde}
%\def\lra{\Leftrightarrow}
\def\del{\partial}
\def\delp{\partial^\prime}
\def\delpp{\partial^{\prime\prime}}
\def\sgn{{\roman{sgn}}}
\def\wtih{\widetilde{H}}
\def\bbz{{\Bbb Z}}
\def\bbr{{\Bbb R}}
\def\rtar{$\Rightarrow$}

\def\cltr{\Red}		  % Red  VERY-Approx PANTONE RED
\def\cltb{\Blue}		  % Blue  Approximate PANTONE BLUE-072
\def\cltg{\PineGreen}	  % ForestGreen  Approximate PANTONE 349




The simplicial homology groups are, in the end, fairly routine to calculate from a 
$\Delta$-complex structure. But! The calculations
\u{depend} on the $\Delta$ structure! This is not a group defined from the space
$X$; it depends on the space \u{and} a $\Delta$ structure on it. 
Choosing a different structure on $X$, maybe we would get
different groups! We \u{should} denote our groups by $H_i^\Delta(X)$, to 
acknowledge this dependence on the structure.

\bsk

But we don't {\it want} a group that depends on this structure. We want groups that
just depend on the topological space $X$, i.e., which are topological invariants.
These groups $H_i^\Delta(X)$ \u{are} topological invariants,
but we will need to take a very roundabout route to show this. We
now define another sequence $H_i(X)$ of groups, the {\it singular homology
groups}, which are readily seen to be topological invariants;
but this definition will also make it very unclear how to really compute them! 
Then we will show that for $\Delta$-complexes these two sequences of groups
are really the same. In so doing, we will have built a sequence of topological
invariants that for a large class of spaces are fairly routine to compute. Then
all we will need to show is that they also capture useful information about
a space (i.e., we can prove useful theorems with them!).

\msk

And the basic idea behind defining them is that, with simplicial homology,
we have already done all of the hard work. What we do is, as before, build a 
sequence of (free) abelian groups, the chain groups $C_n(X)$, 
and (linear) boundary maps between them,
with consecutive maps composing to 0. Then, as before, the homology groups are
kernels mod images, i.e., cycles mod boundaries. And, as before, the basis
elements for each of our chain groups $C_n(X)$ will be the $n$-simplices
in $X$. But now $X$ is \u{any} topological space. So how do we get $n$-simplices
in such a space? We do the only thing we can; we {\it map} them in. 

\vfill
\eject

We work with {\it singular $n$-chains}, that is,
formal (finite) linear combinations $\sum a_i\sigma_i$, where $a_i\in{\Bbb Z}$
and the $\sigma_i$ are {\it singular simplices}, that is,
maps $\sigma_i:\Delta^n\ra X$.
The boundary maps are exactly as before; the alternating sum of
the restrictions of $\sigma_i$ to the $n+1$ faces of $\Delta^n$. 
The same proof as before (interpreting
the faces as restrictions of the map $\sigma_i$, instead of as physical faces)
shows that the composition of two successive boundaries is 0,
and so all of the machinery is in place to define the {\it singular homology
groups} $H_i(X)$ as the kernel of $\del_i$ modulo the image of $\del_{i+1}$ = $Z_i(X)/B_i(X)$ .
They are, by definition, groups defined using the topological space $X$ as input,
and so are topological invariants. Elements are equivalence classes of $i$-cycles,
where $z_1\simeq z_2$ if $z_1-z_2=\del w$ for some $(i+1)$-chain $w$ . We say that $z_1$ and $z_2$ are
{\it homologous}.

\msk

Singular homology groups are quick to define, but what do they measure?
We are trying to mimic simplicial homology, but because a
general topological space $X$ cannot be built out of simplices,
we do the next best thing; we study $X$ by \u{mapping} simplices \u{in}. 
This is true of simplicial homology, too, except that we restrict
ourselves to a very few special singular simplices (the characteristic maps of the
simplices for $X$). In the end an $n$-cycle $\sum a_i\sigma_i^n$, since the 
faces of the $\sigma_i$ must match up precisely, in order to cancel, 
can be thought of as a map of an $n$-complex into $X$, made by gluing the
$n$-simplices $\sigma_i$ together \u{before} mapping in. 
The integer coefficients can really be interpreted as taking multiple copies of 
$\Delta^n$ and gluing them together along their boundaries (the signs tell us
the underlying orientations). The idea is that this $n$-complex is being
mapped ``around a hole'', unless it \u{extends} to a map of an $(n+1)$-complex
into $X$ (having our $n$-complex as boundary). So singular homology really
is trying to detect holes, it is just doing it with maps.....

\vfill
\eject

The ``fun'' with singular homology groups comes when you try to \u{compute} them!
$C_n(X) = \{\sum a_i\sigma_i$ : $a_i\in {\Bbb Z}$ and $\sigma_i:\Delta^n\ra X$ 
is continuous$\}$ is typically a \u{huge} group, since there will be immense
numbers of maps $\Delta^n\ra X$ . The only space for which this is not true is
the one-point space $*$; then there is, for each $n$, exactly one (distinct)
map $\sigma_n :\Delta^n\ra *$ ; the constant map. Therefore each face of $\Delta^n$
gives the same restriction map $\sigma^{n-1}$, and so the boundary maps can 
be directly computed (they depend on the parity of the number $n+1$ of faces 
an $n$-simplex has). We find that $\del_{2n}=Id$ and $\del_{2n-1}=0$ . So in 
computing homology groups, we either have kernel everything ($\del_i=0$) and
image everything ($\del_{i+1}=Id$) or kernel nothing ($\del_i=Id$) and
image nothing ($\del_{i+1}=0$), so in both cases $H_i(*)=0$ . Except for $i=0$;
then $\del_0=0$ (by definition) and $\del_1=0$, so $H_0(*)={\Bbb Z}$ .
But other than this example (and, well, OK, spaces with the discrete topology;
it's the same calculation as above for every point!), computing singular 
homology from the definition is quite a chore! So we need to build
some labor-saving devices, namely, some theorems to help us break the problem
of computing these groups into smaller, more manageable pieces.

\msk

First set of manageable pieces: if we decompose $X$ into its path components, $X=\bigcup X_\alpha$,
then $H_i(X) \cong \bigoplus H_i(X_\alpha)$ for every $i$. This is because every singular simplex,
since $\Delta^i$ is path-connected, maps into some $X_\alpha$ . So $C_i(X) \cong \bigoplus C_i(X_\alpha)$.
Since the boundary of a simplex mapping into $X_\alpha$ consists of simplices in $X_\alpha$, the 
boundary maps respect the decompositions of the chain groups, so 
$B_i(X) \cong \bigoplus B_i(X_\alpha)$ and $Z_i(X) \cong \bigoplus Z_i(X_\alpha)$, and so 
the quotients are $H_i(X) \cong \bigoplus H_i(X_\alpha)$ . 

\vfill
\eject

So, if we wish to, we can focus on computing homology groups for path-connected spaces $X$. For such a space, 
$H_0(X)\cong {\Bbb Z}$, generated by any map of a 0-simplex (= a point) into $X$. This is because any pair
of 0-simplices are homologous; given any two points $x,y\in X$, there is a path $\gamma: I\ra X$ from $x$ to $y$,
This path can be interpreted as a singular 1-simplex, and $\del\gamma = y-x$ . So $H_0(X)$ is generated
by a single point $[x]$ . No multiple of this point is null-homologous, because for any 1-chain $\sum n_i \sigma_i$,
the sum of the coefficients of its boundary is 0 (since this is true for each singular 1-simplex), and any 0-chain
$\sum n_i [x_i]$ is homologous to $(\sum n_i)[x]$ by the above argument.

\msk

Techincal aside: the fact that $H_0(*)={\Bbb Z}$ is annoying to some,
and often requires treating 0-dimensional homology as a special case. 
But since the boundary of a singular 1-simplex is always of the form $v-w$, we find that the 
image of $\del_1$ is always contained in the subgroup of $C_0(X)$ consisting
of chains whose coefficients sum to 0. This means that we can, for free, 
{\it augment} the singular chain complex by a map
$\cdots \ra C_1(X) {\del_1\atop\ra}C_0)X) {\alpha\atop \ra} {\Bbb Z} \ra 0$
where $\alpha$ is the map $\alpha(\sum a_i\sigma_i^0) = \sum a_i$ . This 
is still a chain complex (compositions of consecutive maps are 0); the resulting
homology groups are called {\it reduced} homology $\widetilde{H}_i(X)$ . 
All that this does is remove one copy of ${\Bbb Z}$ from 
$H_0$; $\widetilde{H}_0(X)\oplus {\Bbb Z} \cong H_0(X)$ . All other
homology groups are unchanged. There is a reduced relative homology 
as well, since we can augment with the same map (1-simplices always have 2 ends!),
but in this case it has (essentially) no effect; $\widetilde{H}_i(X,A)\cong H_i(X,A)$
for all $i$ \u{unless} $A=\emptyset$, in which case we lose the ${\Bbb Z}$ in
dimension 0 that we expect to. 

\vfill
\eject

{\bf Continuous maps:} Perhaps the most important property of the fundamental group is that a continuouos map 
between spaces induces a homomorphism between groups. (This implied, for instance,
that homeomorphic spaces have isomorphic \mpu ). The same is true for homology groups, 
for essentially the same reason. Given a map $f:X\ra Y$, there is an induced map $f_\#:C_n(X)\ra C_n(Y)$
defined by postcomposition; for a singular simplex $\sigma$, $f_\#(\sigma) = f\circ\sigma$, and we extend
the map linearly. Since $f\circ(g|_A) = (f\circ g)|_A$ (postcomposition commutes with restriction of the domain),
$f_\#$ commutes with $\del$ : $f_\#(\del \sigma) = \del(f_\#(\sigma))$. A homomorphism between
chain complexes (i.e., a sequence of such maps, one for each chain group) which commutes with the 
boundaries maps in this way, is called a {\it chain map}.
A chain map, such as $f_\#$, therefore, takes cycles to cycles,
and boundaries to boundaries, and so $f_\#:Z_i(X)\ra Z_i(Y)$ (which is linear, hence a homomorphism)
induces a homomorphism $f_*:H_i(X)\ra H_i(Y)$ by $f_*[z] = [f_\#(z)]$ . 
Since it is defined by composition with singular simplices, it is 
immediate that, for a map $g:Y\ra Z$, $(g\circ f)_*=g_*\circ f_*$ . And since the identity map $I:X\ra X$
satisfies $I_\#=Id$, so $I_*=Id$, homeomorphic spaces have isomorphic homology groups.

\vfill
\eject

{\bf Homotopic maps:} Another important property of \mpu\ is that homotopic maps give the same
induced map (after correcting for basepoints). This is also true for homology;
if $f\simeq g:X\ra Y$, then $f_*=g_*$ . The proof, however, is not as straightforward
as for \mpu. It requires some new technology; the chain homotopy.
A chain homotopy $H$ between the chain maps $f_\#,g_\#:C_*(X)\ra C_*(Y)$ 
is a sequence of homomorphisms $H_i:C_i(X)\ra C_{i+1}(Y)$ satisfying
$H_{i-1}\del_i+\del_{i+1}H_i = f_\#-g_\#:C_i(X)\ra C_i(Y)$ . The existence of $H$
implies that $f_*=g_*$; for an $i$-cycle $z$ (with $\del_i(z)=0$) we have

\ssk

$f_*[z]-g_*[z] = [f_\#(z)-g_\#(z)] = [H_{i-1}\del_i(z)+\del_{i+1}H_i(z)] = [H_{i-1}(0)+\del_{i+1}(w)]$

$=[\del_{i+1}(w)]=0$.

\ssk

And the existence of a homotopy between $f$ and $g$ implies the existence of a 
chain homotopy between $f_\#$ and $g_\#$ . This is because the homotopy 
gives a map $H:X\times I\ra Y$, which induces a map $H_\#:C_{i+1}(X\times I)\ra C_{i+1}(Y)$ .
Then we pull, from our back pocket, a {\it prism map}
$P:C_i(X)\ra C_{i+1}(X\times I)$; the composition $H_\#\circ P$ will be our chain homotopy.
The prism map takes a (singular) $i$-simplex $\sigma$ and sends it to a sum of singular $(i+1)$-simplices
in $X\times I$. And the way we define it is to take the $i$-simplex $\Delta^i$, and cut
$\Delta^i\times I$ (i.e., a {\it prism}) into a sum of $(i+1)$-simplices. Using the
map $\sigma^\prime = \sigma\times Id : \Delta^i\times I\ra X\times I$, 
restricted to each of these $(i+1)$-simplices,
yields the prism map. There are many ways of decomposing a prism into simplices,
but we need to be careful to choose one which restricts well to each of 
the \u{faces} of $\Delta^i$,
in order to get the chain homotopy property we require. What 
this requires is that the
decomposition, when restricted to any face of $\Delta^i$ (which we think of as a copy
of $\Delta^{i-1}$), is the same as the decomposition we would have applied to a prism over
an $(i-1)$-simplex. After some exploration, we are led to the following formulation:

\vfill
\eject

If we write $\Delta^n\times\{0\}=[v_0,\ldots ,v_n]$ and 
$\Delta^n\times\{1\}=[w_0,\ldots ,w_n]$, then we can decompose
$\Delta^n\times I$ as the (n+1)-simplices $[v_0,\ldots ,v_i,w_i,\ldots ,w_n]$. 
We then define $P(\sigma) = \sum (-1)^i \sigma^\prime|_{[v_0,\ldots ,v_i,w_i,\ldots ,w_n]}$.
A ``routine calculation'' verifies that 

\ssk

\ctln{$(\del P+P\del)(\sigma) = \sigma^\prime|_{[w_0,\ldots ,w_n]}-\sigma^\prime|_{[v_0,\ldots v_n]}$}

\ssk

Composing with $H_{\#}$ yields our result.

\msk

Consequently, for example, homotopy equivalent spaces have isomporphic
(reduced) homology groups; homotopy equivalences induce isomorphisms.
So all contractible spaces have trivial reduced homology
in all dimensions, since they are all homotopy to a point. If we think of a cell complex as
a collection of disks glued together, this lends some hope that we can compute
their homology groups, since we can compute the homology of the building blocks. 
Our next goal is to make turn this idea into action; but we need another tool, to
frame our answer in the best way possible.

\vfill
\end