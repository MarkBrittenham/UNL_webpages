
\magnification=2000
\overfullrule=0pt
\parindent=0pt

\nopagenumbers

\input amstex

%\voffset=-.6in
%\hoffset=-.5in
%\hsize = 7.5 true in
%\vsize=10.4 true in

\voffset=1.8 true in
\hoffset=-.6 true in
\hsize = 10.2 true in
\vsize=8 true in

\input colordvi

\def\cltr{\Red}		  % Red  VERY-Approx PANTONE RED

\loadmsbm

\input epsf

\def\ctln{\centerline}
\def\u{\underbar}
\def\ssk{\smallskip}
\def\msk{\medskip}
\def\bsk{\bigskip}
\def\hsk{\hskip.1in}
\def\hhsk{\hskip.2in}
\def\dsl{\displaystyle}
\def\hskp{\hskip1.5in}

\def\lra{$\Leftrightarrow$ }
\def\ra{\rightarrow}
\def\mpto{\logmapsto}
\def\pu{\pi_1}
\def\mpu{$\pi_1$}
\def\sig{\Sigma}
\def\msig{$\Sigma$}
\def\ep{\epsilon}
\def\sset{\subseteq}
\def\del{\partial}
\def\inv{^{-1}}
\def\wtl{\widetilde}
%\def\lra{\Leftrightarrow}
\def\del{\partial}
\def\delp{\partial^\prime}
\def\delpp{\partial^{\prime\prime}}
\def\sgn{{\roman{sgn}}}
\def\wtih{\widetilde{H}}
\def\bbz{{\Bbb Z}}
\def\bbr{{\Bbb R}}
\def\rtar{$\Rightarrow$}

\def\cltr{\Red}		  % Red  VERY-Approx PANTONE RED
\def\cltb{\Blue}		  % Blue  Approximate PANTONE BLUE-072
\def\cltg{\PineGreen}	  % ForestGreen  Approximate PANTONE 349

{\bf The Fundamental Theorem of Algebra:}

\msk

\cltr{Every non-constant polynomial
has a complex root: for every $f(z)=a_nz^n+\cdots +a_0$ with $n\geq 1$ and $a_n\neq 0$,
$a_i\in{\Bbb C}$, there is a $z_0\in{\Bbb C}$ with $f(z_0)=0$.}

\ssk

{\bf Proof:} Thinking of ${\Bbb C}=\bbr^2$, if not, then $f$ is a map $\bbr^2\ra \bbr^2\setminus\{0\}$.
We can divide through by $a_n$ without affecting this, and assume that $f$ is monic.

\ssk

Setting \cltr{$\gamma_m(t)=f(m\cos(2\pi t),m\sin(2\pi t))$}, then $\gamma_m:S^1\ra\bbr^2\setminus\{0\}$
extends to a map $\Gamma_m:D^2\ra\bbr^2\setminus\{0\}$, as $\Gamma_n(x)=f(mx)$, so $\gamma_m$ is 
null-homotopic for all $m$. 

\ssk

But $\bbr^2\setminus\{0\}$ def. retracts to the unit circle (by $r(z)=z/|z|$), so 
$\pi_1(\bbr^2\setminus\{0\})\cong \bbz$, and by the above all of the $[\gamma_m]$
represent $0$ in $\bbz$, and so $r_*[\gamma_m]=[r\circ\gamma_m]=0$, as well. 
But for large $m$ we can compute
$w(r\circ\gamma_m)=n$; since $n\geq 1$, this is a contradiction. 

\msk

\cltr{$\gamma_m(t)=f(me^{2\pi it})=
m^n(e^{2\pi nit}+{{a_{n-1}}\over{m}}e^{2\pi (n-1)it}+\cdots +{{a_0}\over{m^n}})= m^n(e^{2\pi nit}+R(m,t))$},

\ssk

so $r\circ\gamma_m(t)=(e^{2\pi nit}+R(m,t))/|e^{2\pi nit}+R(m,t)|$.
But as $m\ra \infty$, $R(m,t)\ra 0$

uniformly in $t$;
$|R(m,t)|=|{{a_{n-1}}\over{m}}e^{2\pi (n-1)it}+\cdots +{{a_0}\over{m^n}}|\leq {{|a_{n-1}|}\over{m}}+\cdots +{{|a_0|}\over{m^n}}\ra 0$ .


\msk

So for large enough $m$ $|R(m,t)|<{{1}\over{2}}$ for all $t$, 

and then 
for every $s\in I$, 
$|e^{2\pi nit}+sR(m,t)|\neq 0$, since 

$|e^{2\pi nit}+sR(m,t)|\geq |e^{2\pi nit}|-s|R(m,t)|\geq |e^{2\pi nit}|-|R(m,t)|\geq {{1}\over{2}}$.

\msk

Then the homotopy
$H(t,s)=(e^{2\pi nit}+sR(m,t))/|e^{2\pi nit}+sR(m,t)|$ is well-defined and
continuous, $H:I\times I\ra S^1$, and defines a homotopy from 
$\alpha:t\mapsto e^{2\pi nit}$ 

(at $s=0$) to $r\circ\gamma_m$ (at $s=1$).
Since $w(\alpha)=n$, for large enough $m$,
$w(r\circ\gamma_m)=n$. 

\msk

This contradiction implies that $f$ must have a root, as
desired.

\vfill
\end

