

\magnification=2000
\overfullrule=0pt
\parindent=0pt

\nopagenumbers

\input amstex

%\voffset=-.6in
%\hoffset=-.5in
%\hsize = 7.5 true in
%\vsize=10.4 true in

\voffset=1.8 true in
\hoffset=-.6 true in
\hsize = 10.2 true in
\vsize=8 true in

\input colordvi

\def\cltr{\Red}		  % Red  VERY-Approx PANTONE RED

\loadmsbm

\input epsf

\def\ctln{\centerline}
\def\u{\underbar}
\def\ssk{\smallskip}
\def\msk{\medskip}
\def\bsk{\bigskip}
\def\hsk{\hskip.1in}
\def\hhsk{\hskip.2in}
\def\dsl{\displaystyle}
\def\hskp{\hskip1.5in}

\def\lra{$\Leftrightarrow$ }
\def\ra{\rightarrow}
\def\mpto{\logmapsto}
\def\pu{\pi_1}
\def\mpu{$\pi_1$}
\def\sig{\Sigma}
\def\msig{$\Sigma$}
\def\ep{\epsilon}
\def\sset{\subseteq}
\def\del{\partial}
\def\inv{^{-1}}
\def\wtl{\widetilde}
%\def\lra{\Leftrightarrow}
\def\del{\partial}
\def\delp{\partial^\prime}
\def\delpp{\partial^{\prime\prime}}
\def\sgn{{\roman{sgn}}}
\def\wtih{\widetilde{H}}
\def\bbz{{\Bbb Z}}
\def\bbr{{\Bbb R}}
\def\rtar{$\Rightarrow$}

\def\cltr{\Red}		  % Red  VERY-Approx PANTONE RED
\def\cltb{\Blue}		  % Blue  Approximate PANTONE BLUE-072
\def\cltg{\PineGreen}	  % ForestGreen  Approximate PANTONE 349





{\bf Higher homotopy groups:} Fundamental groups are a remarkably powerful
tool for studying spaces; they capture a great deal of the global
structure of a space, and so they are very good a detecting
between homotopy-inequivalent spaces. In theory! \cltr{{\bf But}} in practice,
they suffer from the fact that deciding whether two groups are 
isomorphic or not is, in general, undecideable.... 

\msk

There are also ``higher'' homotopy groups beyond the fundamental group \mpu ,
(hence the name pi-{\it one}); elements are homotopy classes, rel boundary, 
of based maps \hfill

$\gamma:(I^n,\del I^n)\ra(X,x_0)$. Multiplication is again by
concatenation. We glue two $n$-cubes side-by-side and then
reprarametrize into a single cube: 

\ssk

$\gamma*\eta(t_1,\ldots,t_n)
=\cases
\gamma(2t_1,t_2\ldots,t_n), & \text{if}\ t_1\leq 1/2\cr
\eta(2t_1-1,t_2\ldots,t_n)& \text{if}\ t_1\geq 1/2\cr 
\endcases$

\ssk

The elements can be interpreted as based homotopy classes of maps 
$\gamma:(S^n,1)\ra(X,x_0)$, by crushing $\del I^n$ to a point.
Like \mpu, it describes, essentially, maps of $S^n$ into
$X$ which don't extend to maps of $D^{n+1}$, i.e., it turns the ``$n$-dimensional
holes'' of $X$ into a group.

\ssk

The (well, an) inverse reverses the first coordinate; a homotopy to the
constant map is built by applying the homotopy in the $\pi_1$ case 
for each tuple $(t_2,\ldots,t_n)$.

\vfill
\eject

One feature of the higher ($n\geq 2$) homotopy groups is that they are 
all {\it abelian}: $\gamma*\eta$ is homotopic, rel boundary, to $\eta*\gamma$.
The homotopy may be obtained by ``spinning'' the middle boundary between
$\gamma$ and $\eta$ around one of the the coordinates.

\msk

\leavevmode


\epsfxsize=4in
\ctln{{\epsfbox{higher.homotopy.ai}}}

\msk

As with $\pi_1$, a continuous map induces homomorphisms of higher 
homotopy groups: $f_*:\pi_n(X,x_0)\ra\pi_n(Y,y_0)$ by post-composition with $f$. 

\ssk

Another distinctive feature
of the higher case is that a covering map $p:\wtl{X}\ra X$ induces an
{\it isomorphism} on $\pi_n$; any map $\gamma:S^n\ra X$ lifts, by the lifting
criterion, giving surjectivity, and homotopy lifting gives injectivity
as usual. As before, a homotopy equivalence induces isomorphisms of
the higher homotopy groups, as well.

\msk

But unlike \mpu , where we have a chance to compute it from simpler pieces
via Seifert-van Kampen, nobody, for example knows what all of the 
homotopy groups $\pi_n(S^2)$ are (except that nearly all of them are
non-trivial!). For particular spaces (except for contractible ones!)
they are, as a result, notoriously difficult to compute. But this doesn't 
stop anyone from using them!, thanks
in large part to 

\ssk

\cltr{Whitehead's Theorem: any map between CW-complexes that
induces an isomorphism on all homotopy groups is a homotopy equivalence.}

\ssk

Proving this result would take us too far afield, however.



\vfill
\end


