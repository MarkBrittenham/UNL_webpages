
\magnification=2000
\overfullrule=0pt
\parindent=0pt

\nopagenumbers

\input amstex

%\voffset=-.6in
%\hoffset=-.5in
%\hsize = 7.5 true in
%\vsize=10.4 true in

\voffset=2 true in
\hoffset=-.6 true in
\hsize = 10.2 true in
\vsize=8 true in

\input colordvi

\loadmsbm

\input epsf

\def\ctln{\centerline}
\def\u{\underbar}
\def\ssk{\smallskip}
\def\msk{\medskip}
\def\bsk{\bigskip}
\def\hsk{\hskip.1in}
\def\hhsk{\hskip.2in}
\def\dsl{\displaystyle}
\def\hskp{\hskip1.5in}

\def\lra{$\Leftrightarrow$ }
\def\ra{\rightarrow}
\def\mpto{\logmapsto}
\def\pu{\pi_1}
\def\mpu{$\pi_1$}
\def\sig{\Sigma}
\def\msig{$\Sigma$}
\def\ep{\epsilon}
\def\sset{\subseteq}
\def\del{\partial}
\def\inv{^{-1}}
\def\wtl{\widetilde}
%\def\lra{\Leftrightarrow}
\def\del{\partial}
\def\delp{\partial^\prime}
\def\delpp{\partial^{\prime\prime}}
\def\sgn{{\roman{sgn}}}
\def\wtih{\widetilde{H}}
\def\bbz{{\Bbb Z}}
\def\bbr{{\Bbb R}}
\def\rtar{$\Rightarrow$}

{\bf $\pi_1(S^1)\cong\bbz$ : A proof in two parts}

\msk

Show $\bbz$ surjects onto $\pi_1(S^1)$, and show that the 
surjection is injective.

\msk

${\Cal U}_- = \{(x,y)\in S^1 : y < \epsilon\}$, 
${\Cal U}_+ = \{(x,y)\in S^1 : y > -\epsilon\}$ open cover of $S^1\subseteq \bbr^2={\Bbb C}$

\ssk

$\gamma:(I,\del I)\ra (S^1,1)$ , $1/n$ a Lebesgue number for $\gamma^{-1}({\Cal U}_-),
\gamma^{-1}({\Cal U}_+)$ ; $I_j=[{{j}\over{n}},{{j+1}\over{n}}]$

\ssk

So $\gamma(I_j)\subseteq {\Cal U}_\pm$ ; pick one and label each subinterval $+$ or $-$ .
If consecutive intervals have the same sign \rtar\ (**) take union, label with the same
sign. Induction \rtar\ eventually consecutive intervals have opposite signs.

\ssk

${\Cal U}_\pm\cong\bbr$ and $\pi_1(\bbr)=1$ \rtar\ $\gamma_j=\gamma|_{I_j}\simeq$ arc in 
${\Cal U}_\pm$ between endpoints. Replace each $\gamma_j$ with the arc, get loop
$\simeq\gamma$ that is a (finite) concatenation of arcs $\alpha_j$.

\ssk

If $\alpha_j(0),\alpha_j(1)$ lie in same component of ${\Cal U}_-\cap{\Cal U}_+$,
then $\alpha_j$ maps into ${\Cal U}_-\cap{\Cal U}_+$ ; change label and apply
(**) to lower number of intervals.

\ssk

Eventually, all $\alpha_j$ ``cross'' their ${\Cal U}_\pm$; after a small homotopy,
we may assume \hfill

$\alpha_j(\del I_j)\subseteq \{-1,1\}$. Reparametrize so that 
$I_j=[{{j}\over{m}},{{j+1}\over{m}}]$, then \hfill

$\alpha_j(t)=(\pm\cos(m\pi t),\pm\sin(m\pi t))$ for some choices of $\pm$'s.
No backtracking \rtar\ $m$ is even and 
$\alpha_{j}(t)=(\cos(m\pi t),\sin(m\pi t))$ for all $j$ 
or 
$\alpha_{j}(t)=(\cos(m\pi t),-\sin(m\pi t))$ for all $j$.


\ssk

Therefore $\alpha\simeq\gamma$ is $\alpha(t)=(\cos(m\pi t),\sin(m\pi t))$
for some (positive, negativce, or zero) $m$. The map $\varphi:\bbz\ra\pi_1(S^1)$
given by
$m\mapsto[\beta_m:t\mapsto(\cos(m\pi t),\sin(m\pi t))]$ is surjective, by the
above, and a homomorphism, since $\beta_m*\beta_n\simeq\beta_{m+n}$.

\vfill\eject

Second part: $\varphi$ is injective. I.e., $\beta_m\simeq\beta_n$ \rtar\ $m=n$.

\msk

Introduce {\it winding number} $w:\pi_1(S^1)\ra \bbz$. Let $\gamma:(I,\del I)\ra(S^1,1)$.

\ssk

${\Cal U}_{x+}=\{(x,y)\in S^1 : x>0\},{\Cal U}_{x-}=\{(x,y)\in S^1 : x<0\}$,

${\Cal U}_{y+}=\{(x,y)\in S^1 : y>0\},{\Cal U}_{y-}=\{(x,y)\in S^1 : y<0\}$ : cover
of $S^1$.

\ssk

$1/n$ = Lebesgue number for their inverse images under $\gamma$, 
$I_j=[{{j}\over{n}},{{j+1}\over{n}}]$ (again). 
$\gamma({{j}\over{n}}),\gamma({{j+1}\over{n}})$ both lie in one of the sets
\rtar\ there is a well-defined (signed) angle $\theta_j$ (strictly between 
$-\pi$ and $\pi$) between them. Define $w(\gamma)=\sum\theta_j$. Then show:

\msk

(1): $w(\gamma)$ is independent of the partition of $I$ used to compute it.
Typical trick: show each is the same as the number computed using the 
union of the two partitions (by noting that it is unchanged when a 
single point is added to the partition).

\ssk

(2): If $\gamma\simeq \beta$, then $w(\gamma)=w(\beta)$. Another standard trick:
the homotopy is a concatenation of ``small'' homotopies. Choose a 
Lebesgue number $1/m$ for the inverse images of the sets under the homotopy
$H$, and partition $I\times I$ into an $m$-by-$m$ grid, with vertices
$(x_i,x_j)$. Each small square maps into one of our sets, so if
$\theta_{i,j}=$angle from $H(x_i,x_j)$ to $H(x_{i+1},x_j)$,
and $\phi_{i,j}=$angle from $H(x_i,x_j)$ to $H(x_i,x_{j+1})$,
then (walking around a grid square)
$\theta_{i,j}+\phi_{i+1,j}=\phi_{i,j}+\theta_{i,j+1}$ . This in turn implies
that $w(H|_{I\times x_j})=\sum \theta_{i,j}=\sum \theta_{i,j+1}=w(H|_{I\times x_{j+1}})$.
(telescoping sum!). So $w(\gamma)=w(\beta)$.

\ssk

(3): $w(\beta_n)=n$ (direct computation). 

\msk

So if $\beta_m\simeq\beta_n$, then $m=n$.
So $\varphi(m)=\varphi(n)$ \rtar\ $m=n$, as desired. 




\vfill
\end