
\input amstex


\magnification=1200


\loadmsbm

\input colordvi

\nopagenumbers
\parindent=0pt

\def\cgy{\GreenYellow}     % GreenYellow  Approximate PANTONE 388
\def\cyy{\Yellow}	  % Yellow  Approximate PANTONE YELLOW
\def\cgo{\Goldenrod}	  % Goldenrod  Approximate PANTONE 109
\def\cda{\Dandelion}	  % Dandelion  Approximate PANTONE 123
\def\capr{\Apricot}	  % Apricot  Approximate PANTONE 1565
\def\cpe{\Peach}		  % Peach  Approximate PANTONE 164
\def\cme{\Melon}		  % Melon  Approximate PANTONE 177
\def\cyo{\YellowOrange}	  % YellowOrange  Approximate PANTONE 130
\def\coo{\Orange}	  % Orange  Approximate PANTONE ORANGE-021
\def\cbo{\BurntOrange}	  % BurntOrange  Approximate PANTONE 388
\def\cbs{\Bittersweet}	  % Bittersweet  Approximate PANTONE 167
%\def\creo{\RedOrange}	  % RedOrange  Approximate PANTONE 179
\def\cma{\Mahogany}	  % Mahogany  Approximate PANTONE 484
\def\cmr{\Maroon}	  % Maroon  Approximate PANTONE 201
\def\cbr{\BrickRed}	  % BrickRed  Approximate PANTONE 1805
\def\crr{\Red}		  % Red  VERY-Approx PANTONE RED
\def\cor{\OrangeRed}	  % OrangeRed  No PANTONE match
\def\paru{\RubineRed}	  % RubineRed  Approximate PANTONE RUBINE-RED
\def\cwi{\WildStrawberry}  % WildStrawberry  Approximate PANTONE 206
\def\csa{\Salmon}	  % Salmon  Approximate PANTONE 183
\def\ccp{\CarnationPink}	  % CarnationPink  Approximate PANTONE 218
\def\cmag{\Magenta}	  % Magenta  Approximate PANTONE PROCESS-MAGENTA
\def\cvr{\VioletRed}	  % VioletRed  Approximate PANTONE 219
\def\parh{\Rhodamine}	  % Rhodamine  Approximate PANTONE RHODAMINE-RED
\def\cmu{\Mulberry}	  % Mulberry  Approximate PANTONE 241
\def\parv{\RedViolet}	  % RedViolet  Approximate PANTONE 234
\def\cfu{\Fuchsia}	  % Fuchsia  Approximate PANTONE 248
\def\cla{\Lavender}	  % Lavender  Approximate PANTONE 223
\def\cth{\Thistle}	  % Thistle  Approximate PANTONE 245
\def\corc{\Orchid}	  % Orchid  Approximate PANTONE 252
\def\cdo{\DarkOrchid}	  % DarkOrchid  No PANTONE match
\def\cpu{\Purple}	  % Purple  Approximate PANTONE PURPLE
\def\cpl{\Plum}		  % Plum  VERY-Approx PANTONE 518
\def\cvi{\Violet}	  % Violet  Approximate PANTONE VIOLET
\def\parp{\RoyalPurple}	  % RoyalPurple  Approximate PANTONE 267
\def\cbv{\BlueViolet}	  % BlueViolet  Approximate PANTONE 2755
\def\cpe{\Periwinkle}	  % Periwinkle  Approximate PANTONE 2715
\def\ccb{\CadetBlue}	  % CadetBlue  Approximate PANTONE (534+535)/2
\def\cco{\CornflowerBlue}  % CornflowerBlue  Approximate PANTONE 292
\def\cmb{\MidnightBlue}	  % MidnightBlue  Approximate PANTONE 302
\def\cnb{\NavyBlue}	  % NavyBlue  Approximate PANTONE 293
\def\crb{\RoyalBlue}	  % RoyalBlue  No PANTONE match
%\def\cbb{\Blue}		  % Blue  Approximate PANTONE BLUE-072
\def\cce{\Cerulean}	  % Cerulean  Approximate PANTONE 3005
\def\ccy{\Cyan}		  % Cyan  Approximate PANTONE PROCESS-CYAN
\def\cpb{\ProcessBlue}	  % ProcessBlue  Approximate PANTONE PROCESS-BLUE
\def\csb{\SkyBlue}	  % SkyBlue  Approximate PANTONE 2985
\def\ctu{\Turquoise}	  % Turquoise  Approximate PANTONE (312+313)/2
\def\ctb{\TealBlue}	  % TealBlue  Approximate PANTONE 3145
\def\caq{\Aquamarine}	  % Aquamarine  Approximate PANTONE 3135
\def\cbg{\BlueGreen}	  % BlueGreen  Approximate PANTONE 320
\def\cem{\Emerald}	  % Emerald  No PANTONE match
\def\cjg{\JungleGreen}	  % JungleGreen  Approximate PANTONE 328
\def\csg{\SeaGreen}	  % SeaGreen  Approximate PANTONE 3268
\def\cgg{\Green}	  % Green  VERY-Approx PANTONE GREEN
\def\cfg{\ForestGreen}	  % ForestGreen  Approximate PANTONE 349
\def\cpg{\PineGreen}	  % PineGreen  Approximate PANTONE 323
\def\clg{\LimeGreen}	  % LimeGreen  No PANTONE match
\def\cyg{\YellowGreen}	  % YellowGreen  Approximate PANTONE 375
\def\cspg{\SpringGreen}	  % SpringGreen  Approximate PANTONE 381
\def\cog{\OliveGreen}	  % OliveGreen  Approximate PANTONE 582
\def\pars{\RawSienna}	  % RawSienna  Approximate PANTONE 154
\def\cse{\Sepia}		  % Sepia  Approximate PANTONE 161
\def\cbr{\Brown}		  % Brown  Approximate PANTONE 1615
\def\cta{\Tan}		  % Tan  No PANTONE match
\def\cgr{\Gray}		  % Gray  Approximate PANTONE COOL-GRAY-8
\def\cbl{\Black}		  % Black  Approximate PANTONE PROCESS-BLACK
\def\cwh{\White}		  % White  No PANTONE match


\voffset=-.5in
\hoffset=-.3in
\hsize = 7.2 true in
\vsize=10.2 true in

%\voffset=1.2in
%\hoffset=-.5in
%\hsize = 10.2 true in
%\vsize=8 true in

\overfullrule=0pt


\def\ctln{\centerline}
\def\u{\underbar}
\def\ssk{\smallskip}
\def\msk{\medskip}
\def\bsk{\bigskip}
\def\hsk{\hskip.1in}
\def\hhsk{\hskip.2in}
\def\dsl{\displaystyle}
\def\hskp{\hskip1.5in}

\def\delx{{{\partial}\over{\partial x}}}
\def\dely{{{\partial}\over{\partial y}}}

\def\moda{\medspace {\underset a\to \equiv} \medspace}
\def\modb{\medspace {\underset b\to \equiv} \medspace}
\def\modc{\medspace {\underset c\to \equiv} \medspace}

\def\lra{$\Leftrightarrow$ }


\ctln{\bf Math 445 Number Theory}

\ssk

\ctln{December 3, 2004}

\msk

{\bf Elliptic curves:} $f(x,y) = y^2-(ax^3+bx^2+cx+d) = y^2-q(x)$
${\Cal C}_f({\Bbb R})$ is an {\it elliptic curve} if $f$ has no linear factors
and ${\Cal C}_f({\Bbb R})$ has no singular points. 

\ssk

Verifying this, over ${\Bbb R}$ can be hard! But if we work over ${\Bbb C}$, we have


\ssk

{\it Fact:} ${\Cal C}_f({\Bbb C})$ is an elliptic curve (which implies that
${\Cal C}_f({\Bbb R})$ is) $\Leftrightarrow$ $q(x)$ has no repeated root.

\ssk

An elliptic curve is a cubic curve. So two points on the curve $A,B$ can be used
to find a third, $C$, as $C$ = the other point lying on $L\cap{\Cal C}_f({\Bbb R})$,
where $L$ = the line through $A$ and $B$ . This can be used to define a 
\underbar{product} on ${\Cal C}_f({\Bbb R})$ , $C=AB$ . (If $A=B$, we
can use $L$ = the tangent line through $A$.) This product, unfortunately, is not
very well-behaved; for example it isn't associative. An example: of $AA=B$, then 
$AB=A$, so $A(AB) = AA = B$. But $(AA)B = BB$ = the third point on the tangent
line through $B$, which is can't be $A$, since then the line through $A$ and $B$
is tangent at both $A$ and $B$, so the cubic equation $f(x,mx+r)=0$ has
two double roots!

\msk

But this can be remedied, by introducing a second binary operation, $+$, defined
as follows. Let $\underline{0}\in {\Cal C}_f({\Bbb R})$ be any point, and define,
for $A,B\in {\Cal C}_f({\Bbb R})$, $A+B = \underline{0}(AB)$ . This addition
\underbar{is} associative, and in fact, turns ${\Cal C}_f({\Bbb R})$ into an 
abelian group! In particular, we have

\ssk

$A+B=B+A$ (since $AB = C = BA$ is the third point on the line through $A,B$)

\ssk

$A+\underline{0} = A$ (since if 
$A\underline{0} = C$, then $A+\underline{0} = \underline{0}(A\underline{0}) = 
\underline{0}C = A$, since $\underline{0},A,C$ are the three points of some 
$L\cap{\Cal C}_f({\Bbb R})$

\ssk

For every $A$ there is exactly one $B$ with $A+B = \underline{0}$ ; 
$A+B = \underline{0}(AB) = \underline{0}$ means that the line through 
$\underline{0}$ and $AB$ is tangent at $\underline{0}$. There is only
only such line, so $AB$ must be $\underline{0}\underline{0}$. So
$B$ = $A(AB) = A(\underline{0}\underline{0})$ is determined by $A$,
and we can check that in fact $A+B = \underline{0}(AB) = \underline{0}(\underline{0}\underline{0}) = \underline{0}$ .

\ssk

Associativity is the fun one! See the second page.....

\ssk

\msk

But what does this mean? It means that  an elliptic curve ${\Cal C}_f({\Bbb R}$ 
forms an (abelian) group under this addition! And if $\underline{0}$ is chosen
with rational coordinates (assuming ${\Cal C}_f({\Bbb R}$ has a rational point),
then the chord-and-tangent claculations in the addition will always give rational
points when starting from rational points. That is, ${\Cal C}_f({\Bbb Q}$ is \underbar{also}
an abelian group under this operation! 

\ssk

For the case of elliptic curves, with polynomial $f(x,y) = y^2-(ax^3+bx^2+cx+d)$, 
a particularly nice choice for $\underline{0}$ is the ``point at infinity'', since it simplifies 
many calculations. A formal approach to this requires us to projectivize everything,
which means to think, instead of $f$, of the homogeneous polynomial 
$F(x,y) = y^2z-(ax^3+bx^2z+cxz^2+dz^3)$, which has solution 
$(0,1,0)$, 
which ``represents'' vertical lines in the plane. But the upshot of choosing 
$\underline{0}$ at infinity is that if $A=(a_1,a_2)$, then $\underline{0}A=(a_1,-a_2)$
(since the line from $A$ to ``vertical lines'' is the vertical line through $A$ !).
This allows us to write \underbar{formulas} for $A+B = \underline{0}(AB)$
and $2A=\underline{0}(AA)$ . For the ``normalized'' polynomials \hhsk
$y^2=x^3+ax+b$ \hhsk , if $A=(a_1,a_2)$ and $B=(b_1,b_2)$, then
a little computation with chords and tangents reveals:

\ssk

$\dsl A+B = ({{m^2-b}\over{a}}-a_1-b_1,-(a_2+m({{m^2-b}\over{a}}-2a_1-b_1))$ , where
$\dsl m={{b_2-a_2}\over{b_1-a_1}}$ .

$\dsl 2A = ({{M^2-b}\over{a}}-2a_1,-(a_2+m({{M^2-b}\over{a}}-3a_1))$ , where
$\dsl M={{3a_1^2+2aa_1+b}\over{2a_2}}$

\ssk

Note that, in the first case, when $a_1=b_1$, and in the second case, when $a_2=0$, that the 
resulting point is the point at infinity (the line used in the calculation is a vertical line). So we 
must treat $[0:1:0]$ (as it is usually written) as a (rational) point on the curve!

\vfill\eject

$A+(B+C)= (A+B)+C$ : this is the fun one! This {\it says} that
\hhsk
$\underline{0}(A(\underline{0}(BC))) = \underline{0}((\underline{0}(AB))C)$
\hhsk, so we need to show that 
\hhsk $A(\underline{0}(BC)) =(\underline{0}(AB))C$ . And how do you
show this?! Well, we use a little

\ssk

{\it Lemma:} If $f(x,y),g(x,y)$ are cubic polynomials, and 
$P_1,\ldots,P_9\in{\Cal C}_f({\Bbb R}\cap{\Cal C}_g({\Bbb R}$, with
$P_1,P_2,P_3$ lying on a line $L$ (which is {\it not} contained in
${\Cal C}_f({\Bbb R}$), then there is a quadratic 
polynomial $q(x,y)$ with $P_4,\ldots,P_9\in{\Cal C}_q({\Bbb R}$ .

\ssk

And the point to this result is that, typically, you can't expect 6 points
chosen at random to lie on a quadratic (i.e., on a conic section). so this 
is really saying something.

\ssk

Setting the proof of this aside for the moment, to show associativity,
start with a cubic curve ${\Cal C}_f({\Bbb R}$ (which contains no line), and
set 

\ssk

$P_1=B,P_4=AB,P_7=A$ (all on a line $L_1$ : $L_1(x,y)=0$)

$P_2=B,P_5=\underline{0},P_8=\underline{0}(BC)$ (on a line $L_2(x,y)=0$)

$P_3=C,P_6-\underline{0}(AB),P_9=(\underline{0}(AB))C$ (on a line $L_3(x,y)=0$

\ssk

These points all lie on ${\Cal C}_f({\Bbb R}$ (since $A,B,C,\underline{0}$ do), and they
also lie on ${\Cal C}_g({\Bbb R}$ , where $g(x,y)=L_1(x,y)L_2(x,y)L_3(x,y)$ . Furthermore,
$P_1,P_2,P_3$ lie on a line $L$. In the generic case, where all 9 of these points are 
distinct, the lemma lets us conclude that the remaining 6 points $P_4,\ldots,P_9$ lie on a
quadratic. But! $P_4,P_5,P_6$ \underbar{also} lie on a line$L^\prime$ , so 
$L^\prime\subseteq{\Cal C}_fq{\Bbb R}$, since $L$ hits the quadratic in 
$3>2$=degree$(q)$ points. So, $q$ is really a product of linear functions, implying
that $P_7,P_8,P_9$ lie on a line, since otherwise one of these lies on $L^\prime$,
implying that it hits ${\Cal C}_f({\Bbb R}$ in $4>3$=degree$(f)$ points, 
so $L^\prime\subseteq{\Cal C}_f({\Bbb R}$, a contradiction. But this
means that $P_7P_8=P_9$, i.e., $A(\underline{0}(BC))=(\underline{0}(AB))C$ !

\ssk

If these 9 points are not all distinct, we appeal to ``continuity'', by finding a nearby
generic situation; the limits of 3 sequences of points lying on lines is 3 points on a line.
The details of this can (sort of) be found in the text.....

\vfill
\end


























