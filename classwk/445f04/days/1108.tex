
\input amstex


\magnification=1200


\loadmsbm

\input colordvi

\nopagenumbers
\parindent=0pt

\def\cgy{\GreenYellow}     % GreenYellow  Approximate PANTONE 388
\def\cyy{\Yellow}	  % Yellow  Approximate PANTONE YELLOW
\def\cgo{\Goldenrod}	  % Goldenrod  Approximate PANTONE 109
\def\cda{\Dandelion}	  % Dandelion  Approximate PANTONE 123
\def\capr{\Apricot}	  % Apricot  Approximate PANTONE 1565
\def\cpe{\Peach}		  % Peach  Approximate PANTONE 164
\def\cme{\Melon}		  % Melon  Approximate PANTONE 177
\def\cyo{\YellowOrange}	  % YellowOrange  Approximate PANTONE 130
\def\coo{\Orange}	  % Orange  Approximate PANTONE ORANGE-021
\def\cbo{\BurntOrange}	  % BurntOrange  Approximate PANTONE 388
\def\cbs{\Bittersweet}	  % Bittersweet  Approximate PANTONE 167
%\def\creo{\RedOrange}	  % RedOrange  Approximate PANTONE 179
\def\cma{\Mahogany}	  % Mahogany  Approximate PANTONE 484
\def\cmr{\Maroon}	  % Maroon  Approximate PANTONE 201
\def\cbr{\BrickRed}	  % BrickRed  Approximate PANTONE 1805
\def\crr{\Red}		  % Red  VERY-Approx PANTONE RED
\def\cor{\OrangeRed}	  % OrangeRed  No PANTONE match
\def\paru{\RubineRed}	  % RubineRed  Approximate PANTONE RUBINE-RED
\def\cwi{\WildStrawberry}  % WildStrawberry  Approximate PANTONE 206
\def\csa{\Salmon}	  % Salmon  Approximate PANTONE 183
\def\ccp{\CarnationPink}	  % CarnationPink  Approximate PANTONE 218
\def\cmag{\Magenta}	  % Magenta  Approximate PANTONE PROCESS-MAGENTA
\def\cvr{\VioletRed}	  % VioletRed  Approximate PANTONE 219
\def\parh{\Rhodamine}	  % Rhodamine  Approximate PANTONE RHODAMINE-RED
\def\cmu{\Mulberry}	  % Mulberry  Approximate PANTONE 241
\def\parv{\RedViolet}	  % RedViolet  Approximate PANTONE 234
\def\cfu{\Fuchsia}	  % Fuchsia  Approximate PANTONE 248
\def\cla{\Lavender}	  % Lavender  Approximate PANTONE 223
\def\cth{\Thistle}	  % Thistle  Approximate PANTONE 245
\def\corc{\Orchid}	  % Orchid  Approximate PANTONE 252
\def\cdo{\DarkOrchid}	  % DarkOrchid  No PANTONE match
\def\cpu{\Purple}	  % Purple  Approximate PANTONE PURPLE
\def\cpl{\Plum}		  % Plum  VERY-Approx PANTONE 518
\def\cvi{\Violet}	  % Violet  Approximate PANTONE VIOLET
\def\parp{\RoyalPurple}	  % RoyalPurple  Approximate PANTONE 267
\def\cbv{\BlueViolet}	  % BlueViolet  Approximate PANTONE 2755
\def\cpe{\Periwinkle}	  % Periwinkle  Approximate PANTONE 2715
\def\ccb{\CadetBlue}	  % CadetBlue  Approximate PANTONE (534+535)/2
\def\cco{\CornflowerBlue}  % CornflowerBlue  Approximate PANTONE 292
\def\cmb{\MidnightBlue}	  % MidnightBlue  Approximate PANTONE 302
\def\cnb{\NavyBlue}	  % NavyBlue  Approximate PANTONE 293
\def\crb{\RoyalBlue}	  % RoyalBlue  No PANTONE match
%\def\cbb{\Blue}		  % Blue  Approximate PANTONE BLUE-072
\def\cce{\Cerulean}	  % Cerulean  Approximate PANTONE 3005
\def\ccy{\Cyan}		  % Cyan  Approximate PANTONE PROCESS-CYAN
\def\cpb{\ProcessBlue}	  % ProcessBlue  Approximate PANTONE PROCESS-BLUE
\def\csb{\SkyBlue}	  % SkyBlue  Approximate PANTONE 2985
\def\ctu{\Turquoise}	  % Turquoise  Approximate PANTONE (312+313)/2
\def\ctb{\TealBlue}	  % TealBlue  Approximate PANTONE 3145
\def\caq{\Aquamarine}	  % Aquamarine  Approximate PANTONE 3135
\def\cbg{\BlueGreen}	  % BlueGreen  Approximate PANTONE 320
\def\cem{\Emerald}	  % Emerald  No PANTONE match
\def\cjg{\JungleGreen}	  % JungleGreen  Approximate PANTONE 328
\def\csg{\SeaGreen}	  % SeaGreen  Approximate PANTONE 3268
\def\cgg{\Green}	  % Green  VERY-Approx PANTONE GREEN
\def\cfg{\ForestGreen}	  % ForestGreen  Approximate PANTONE 349
\def\cpg{\PineGreen}	  % PineGreen  Approximate PANTONE 323
\def\clg{\LimeGreen}	  % LimeGreen  No PANTONE match
\def\cyg{\YellowGreen}	  % YellowGreen  Approximate PANTONE 375
\def\cspg{\SpringGreen}	  % SpringGreen  Approximate PANTONE 381
\def\cog{\OliveGreen}	  % OliveGreen  Approximate PANTONE 582
\def\pars{\RawSienna}	  % RawSienna  Approximate PANTONE 154
\def\cse{\Sepia}		  % Sepia  Approximate PANTONE 161
\def\cbr{\Brown}		  % Brown  Approximate PANTONE 1615
\def\cta{\Tan}		  % Tan  No PANTONE match
\def\cgr{\Gray}		  % Gray  Approximate PANTONE COOL-GRAY-8
\def\cbl{\Black}		  % Black  Approximate PANTONE PROCESS-BLACK
\def\cwh{\White}		  % White  No PANTONE match


\voffset=-.6in
\hoffset=-.5in
\hsize = 7.5 true in
\vsize=10.6 true in

%\voffset=1.2in
%\hoffset=-.5in
%\hsize = 10.2 true in
%\vsize=8 true in

\overfullrule=0pt


\def\ctln{\centerline}
\def\u{\underbar}
\def\ssk{\smallskip}
\def\msk{\medskip}
\def\bsk{\bigskip}
\def\hsk{\hskip.1in}
\def\hhsk{\hskip.2in}
\def\dsl{\displaystyle}

\def\lra{$\Leftrightarrow$ }


\ctln{\bf Math 445 Number Theory}

\smallskip

\ctln{November 8, 2004}

\medskip


$x$ has a repeating continued fraction expansion $x=[a_0,\ldots,a_n,\overline{b_0,\ldots,b_m}]$
$\Leftrightarrow$ $x=r+s\sqrt{t}$ for some $r,s\in{\Bbb Q}$ , $t\in{\Bbb Z}$ . \hsk
Last time: enough to show this for $\alpha=[\overline{b_0,\ldots,b_m}] = [b_0,\ldots,b_m,\alpha]$ . 
Then for $\dsl [b_0,\ldots,b_m] =
{{h_m^\prime}\over{k_m^\prime}}$ , $\dsl \alpha = 
{{h_m^\prime\alpha + h_{m-1}^\prime}\over{k_m^\prime\alpha + k_{m-1}^\prime}}$ , so
$k_m^\prime\alpha^2 + k_{m-1}^\prime\alpha = h_m^\prime\alpha + h_{m-1}^\prime$ ,
so $\alpha$ is the solution of a quadratic equation with rational coefficients, so $\alpha = r_0+s_0\sqrt{t}$ ,
as desired. \hhsk
The converse ($\Leftarrow$) direction follows an argument parallel to one of your homework
questions; our further explorations will not need this direction.

\ssk

In what follows, for $\dsl x={{a+\sqrt{d}}\over{b}}$ , it will be useful to have the
notation $\dsl x^\prime = {{a-\sqrt{d}}\over{b}}$ for the {\it conjugate} of $x$, that
is, the other root of the quadratic having $x$ as root. Our main result on periodic
continued fractions is:
\hhsk
\crr{If $x=\sqrt{n}+\lfloor \sqrt{n}\rfloor$, then $x=[\overline{a_0,\ldots , a_k}]$ is {\it purely periodic}.}

\ssk

To see this, note that $x^\prime = \lfloor \sqrt{n}\rfloor - \sqrt{n}$ , so $-1<x^\prime < 0$ .
If we set $x=[a_0,\ldots , a_i+x_i] = [a_0,\ldots , a_i,\zeta_i]$ (so $\dsl \zeta_i={{1}\over{x_i}}$  
and $a_{i+1} = \lfloor\zeta_i \rfloor$) then from our homework we know that (since
$\sqrt{n} = [b_0,b_1,\ldots] = [a_0-\lfloor\sqrt{n}\rfloor,a_1,a_2,\ldots]$) \hsk
\crr{$\dsl x_i = {{\sqrt{n}-m_i}\over{q_i}}$} and 
$\dsl \zeta_{i+1} = {{q_i}\over{\sqrt{n}-m_i}} = {{\sqrt{n}+m_i}\over{q_{i+1}}}$ . 

So
$x_{i+1} = \zeta_{i+1}-a_{i+1}$ , where $q_iq_{i+1} = n-m_i^2$ (which, inductively, defines
$q_{i+1}$) , 
$\dsl a_{i+1} = \lfloor\zeta_{i+1}\rfloor$, so 
$\dsl  {{\sqrt{n}+m_i}\over{q_{i+1}}} = a_{i+1} + {{\sqrt{n}-m_{i+1}}\over{q_{i+1}}}$ , and so
$m_{i+1} = a_{i+1}q_{i+1} - m_i$ (which, inductively, defines $m_{i+1}$) . 
In other words, \crr{the formulas 
$\dsl q_{i+1} = {{n-m_i^2}\over{q_i}}$ , 
$\dsl a_{i+1}=\lfloor{{\sqrt{n}+m_i}\over{q_{i+1}}}\rfloor$ , 
and $m_{i+1} = a_{i+1}q_{i+1} - m_i$ allow us to inductively 
define each of these symbols, starting from 
$m_0=\lfloor\sqrt{n}\rfloor$ and $q_0=1$ .}

\ssk

The key to the proof is that $-1< \zeta_i^\prime  <0$ for all $i$; the proof may be found
at the end of the day's notes. This implies that
$\dsl \lfloor{{-1}\over{\zeta_{i+1}^\prime}}\rfloor = \lfloor a_i-\zeta_i^\prime\rfloor = a_i$, since
$a_i<a_i-\zeta_i^\prime < a_i+1$ . So $a_i$ can be recovered from $\zeta_{i+1}$ .

\ssk

We know, from homework, that the continued fraction for $\sqrt{n}$ and therefore 
for $\sqrt{n}+\lfloor\sqrt{n}\rfloor$ (since they agree in all but the first term), becomes
periodic; past a certain point $k$, there is an $m> 0$ with $a_{k+s+m} = a_{k+s}$ for all $s\geq 0$.
That is, $\zeta_k = \zeta_{k+m}$ .
Let $k$ and $m$ be the smallest such numbers (i.e., $k$ = place where periodicity starts, 
$m$=length of the shortesst period). We {\it claim:} $k=0$ . But this is just because if $k>0$, then
$\zeta_k = \zeta_{k+m}$ $\Rightarrow$ $\zeta_k^\prime = \zeta_{k+m}^\prime$ $\Rightarrow$
$\dsl a_{k-1} = \lfloor{{-1}\over{\zeta_{k}^\prime}}\rfloor = 
\lfloor{{-1}\over{\zeta_{k+m}^\prime}}\rfloor = a_{k+m-1}$ $\Rightarrow$
$\dsl {{1}\over{\zeta_{k-1}-a_{k-1}}} = \zeta_k =  \zeta_{k+m} = {{1}\over{\zeta_{k+m-1}-a_{k+m-1}}}
= {{1}\over{\zeta_{k+m-1}-a_{k-1}}}$ $\Rightarrow$ $\zeta_{k-1} = \zeta_{(k-1)+m}$ , contradicting
our choice of $k$ . 
\hhsk
So $k=0$ ; and so there is an $m>0$ so that $a_{m+s} = a_s$ for all $s\geq 0$ . So
$\sqrt{n}+\lfloor\sqrt{n}\rfloor = [\overline{a_0,\ldots,a_{m-1}}] = [a_0,\overline{a_1,\ldots,a_{m-1},a_0}]$ .
Note that $a_0=2\lfloor\sqrt{n}\rfloor$, so 
\hhsk
$\sqrt{n} = [\lfloor\sqrt{n}\rfloor,\overline{a_1,\ldots,a_{m-1},2\lfloor\sqrt{n}\rfloor}]$.

\msk

Now back to Pell's Equation! We know that if $|N|<\sqrt{n}$, then every solution to $x^2-ny^2=N$
has $\dsl {{x}\over{y}}$ = a convergent of $\sqrt{n}$. But as we have just seen,
$\sqrt{n}+\lfloor\sqrt{n}\rfloor = [\overline{2\lfloor\sqrt{n}\rfloor,a_1,\ldots,a_{m-1}}]$ , and this 
will allow us to shed light on $h_i^2-nk_i^2$, to understand Pell's equation better.

\ssk

$\sqrt{n}+\lfloor\sqrt{n}\rfloor = [\overline{2\lfloor\sqrt{n}\rfloor,a_1,\ldots,a_{m-1}}]$
means \crr{(with $a_0=\lfloor\sqrt{n}\rfloor$) that
$\dsl \sqrt{n}  = [a_0,\overline{a_1,\ldots,a_{m-1},2a_0}] $}

\ssk

Wherever we choose to stop the continued fraction expansion of $\sqrt{n}$, 
$\dsl \sqrt{n} = [a_0,\ldots,a_s,\zeta_{s+1}] = $

$\dsl [a_0,\ldots,a_s,{{\sqrt{n}+m_s}\over{q_{s+1}}}]$, we find that 

\crr{$\dsl \sqrt{n} = 
{{{{\sqrt{n}+m_s}\over{q_{s+1}}}h_s+h_{s-1}}\over{{{\sqrt{n}+m_s}\over{q_{s+1}}}k_s+k_{s-1}}}
= {{(\sqrt{n}+m_s)h_s+q_{s+1} h_{s-1}}\over{(\sqrt{n}+m_s)k_s+q_{s+1}k_{s-1}}}$ . }
Using this, we can calculate what $h_s^2-nk_s^2$ equals; we will do this next time.

\vfill\eject

Proof of $\dsl -1<\zeta_{1}^\prime < 0$: Note that $\dsl \zeta_i = {{\sqrt{n}+m_{i-1}}\over{q_i}}$ , so 

$\dsl \zeta_{i+1} = {{1}\over{\zeta_i-a_i}} = {{1}\over{{{\sqrt{n}+m_{i-1}}\over{q_i}} - a_i}}
= {{q_i}\over{\sqrt{n}+m_{i-1} - a_iq_i}}
 = {{q_i\sqrt{n}-(m_{i-1}-a_iq_{i+1})q_i}\over{n-(m_{i-1}-a_iq_i)^2}}$ . Then

$\dsl \zeta_i^\prime = {{-\sqrt{n}+m_{i-1}}\over{q_i}}$, and 

$\dsl {{1}\over{\zeta_i^\prime-a_i}} = 
{{1}\over{{{-\sqrt{n}+m_{i-1}}\over{q_i}} - a_i}} = {{q_i}\over{(m_{i-1}-a_iq_i)-\sqrt{n}}} = 
{{q_i((m_{i-1}-a_iq_i)+\sqrt{n}}\over{(m_{i-1}-a_iq_i)^2-n}} = $

$\dsl {{-q_i\sqrt{n}-(m_{i-1}-a_iq_{i+1})q_i}\over{n-(m_{i-1}-a_iq_i)^2}} = \zeta_{i+1}^\prime$ .

But $x=\zeta_0$, so $-1<\zeta_0^\prime<0$ ; then we have, by induction, 
$-1<\zeta_i^\prime$ $\Rightarrow$ $\zeta_i^\prime - a_i < -1$ $\Rightarrow$
$\dsl -1<{{1}\over{\zeta_i^\prime - a_i}} = \zeta_{i+1}^\prime < 0$ .

\vfill\end


Note that since $\dsl \zeta_0={{\sqrt{n}+\lfloor\sqrt{n}\rfloor}\over{1}}$, $m_0=\lfloor\sqrt{n}\rfloor$
and $q_1=1$ . Then since $\zeta_0=\zeta_m=\zeta_{2m} = \cdots$ , we have
$q_{mt+1} = 1$ for all $t\geq 0$.


[a_0,a_1,\ldots,a_{m-1}+x_{m-1}] = [a_0,a_1,\ldots,a_{m-1},{{1}\over{x_{m-1}}}]
= [a_0,a_1,\ldots,a_{m-1},\sqrt{n}+a_0]$  . So

$\dsl \sqrt{n} = {{(\sqrt{n}+a_0)h_{m-1} + h_{m-2}}\over{(\sqrt{n}+a_0)k_{m-1} + k_{m-2}}}$, 
so  $\sqrt{n}((\sqrt{n}+a_0)k_{m-1} + k_{m-2}) = (\sqrt{n}+a_0)h_{m-1} + h_{m-2}$ , so 
$\sqrt{n}(a_0k_{m-1} + k_{m-2} -h_{m-1}) = a_0h_{m-1}+h_{m-2}-nk_{m-1}$, which implies
that $\sqrt{n}\in{\Bbb Q}$, a contradiction, unless $a_0k_{m-1} + k_{m-2} -h_{m-1} = 0$ (so 
$a_0h_{m-1}+h_{m-2}-nk_{m-1} = 0$. 
So $h_{m-1} = a_0k_{m-1} + k_{m-2}$ and $nk_{m-1} = a_0h_{m-1}+h_{m-2}$ , so 
$h_{m-1}^2-nk_{m-1}^2 = (a_0k_{m-1} + k_{m-2})h_{m-1} - (a_0h_{m-1}+h_{m-2})k_{m-1}
= h_{m-1}k_{m-2} - h_{m-2}k_{m-1} = (-1)^m$

So $h_{m-1}^2-nk_{m-1}^2 = (-1)^m$ . If $m$ is even, then we have found a solution to
$x^2-ny^2=1$ . If $m$ is odd, then apply the same reasoning, except with \underbar{two}
periods of the continued fraction: 

$\sqrt{n} = [a_0,\ldots,a_{m-1},a_m,\ldots,a_{2m-1},\sqrt{n}+a_0]$, and the same argument shows
that $h_{2m-1}^2-nk_{2m-1}^2 = (-1)^{2m} = 1$ . In general, taking $t$ periods, we get
$h_{tm-1}^2-nk_{tm-1}^2 = (-1)^{tm}$ . So we have shown that $x^2-ny^2=1$ always has a solution;
$x=h_{2m-1}, y=k_{2m-1}$ where $m$ = the period of the continued fraction of $\sqrt{n}$ , will
always work.




$h_s^2-nk_s^2 = (-1)^{s-1}\beta$ . So the only $N$ with $|N|\leq \sqrt{n}$ for which $x^2-ny^2=N$
can be solved are those for which $N = (-1)^{s-1}\beta$ where 
$\dsl \zeta_{s+1} = {{\sqrt{n}+\alpha}\over{beta}}$ . 

\msk

This is best illustrated with an example! Taking $n=19$, we have

$a_0=4$ , $\dsl x_0=\sqrt{19}-4, \zeta_1 = {{\sqrt{19}+4}\over{3}}$ , 

$a_1=2$ , $\dsl x_1={{\sqrt{19}-2}\over{3}}, \zeta_2 = {{\sqrt{19}+2}\over{5}}$ , 

$a_2=1$ , $\dsl x_2={{\sqrt{19}-3}\over{5}}, \zeta_3 = {{\sqrt{19}+3}\over{2}}$ ,

$a_3=3$ , $\dsl x_3={{\sqrt{19}-3}\over{2}}, \zeta_4 = {{\sqrt{19}+3}\over{5}}$ ,

$a_4=1$ , $\dsl x_4={{\sqrt{19}-2}\over{5}}, \zeta_5 = {{\sqrt{19}+2}\over{3}}$ ,

$a_5=2$ , $\dsl x_5={{\sqrt{19}-4}\over{3}}, \zeta_6 = {{\sqrt{19}+4}\over{1}}$ ,

$a_6=8$ , $\dsl x_6=\sqrt{19}-4 = x_0$ , and we can compute

\ssk

$h_0=4, h_1=9, h_2=13, h_3=48, h_4=61, h_5=170, h_6=1421, \ldots$

$k_0=1, k_1=2, k_2=3, k_3=11, k_4=14, k_5=39, k_6=325, \ldots$

\ssk

From our analysis above, $(h_5)^2-19(k_5)^2= (-1)^6=1$ , so
$(170,39)$ is a solution to $x^2-19y^2=1$ . Also, the values of $(-1)^{s-1}\beta$
are $-3, 5, -2, 5, -3, 1, -3, 5, -2, 5, \ldots$, so among $-4, -3, \ldots, 3, 4$ , the only $N$
for which $x^2-19y^2=N$ has a solution are $N=-3,-2,$ and 1.  By continuing to compute
convergents, we can find infnitely many solutions for each such equation.










\vfill\end









\ctln{$\displaystyle {{a}\over{b}} = q_1+{{r_1}\over{b}}$ , 
$\displaystyle {{b}\over{r_1}} = q_2+{{r_2}\over{r_1}}$  , $\ldots$ , 
$\displaystyle {{r_i}\over{r_{i+1}}} = q_{i+2}+{{r_{i+2}}\over{r_{i+1}}}$ , 
$\displaystyle {{r_{n-2}}\over{1}} = q_{n}+0$}

then we can use them to express $\displaystyle {{a}\over{b}}$ as a {\it continued fraction}:

\ssk

$\displaystyle {{a}\over{b}}$ = 
$\displaystyle q_1+{\displaystyle {r_1}\over{b}}$ = 
$\displaystyle q_1+{\displaystyle {1}\over{\displaystyle {b}\over{\displaystyle r_1}}}$ = 
$\displaystyle q_1+{\displaystyle {1}\over{\displaystyle q_2+{\displaystyle {r_2}\over{\displaystyle r_1}}}}$ = 
$\cdots$ = 
$\displaystyle q_1+{\displaystyle {1}\over{\displaystyle q_2+{\displaystyle {1}\over{q_3+{\displaystyle {1}\over{\displaystyle \cdots + {\displaystyle {1}\over{\displaystyle q_n}}}}}}}}$ 


$\displaystyle \sum_{n=1}^{\infty} $

\Big({{}\over{}}\Big)


\cfrac q_1+{{1}\over{{q_2+{{1}\over{{q_3+ {{1}\over{{\cdots +{{1}\over{q_n}}}}}}}}}}}}}\endcfrac

+\cfrac{1}{q_3+ \cfrac{1}{\cdots +\cfrac{1}{q_n}}}}}}}}
