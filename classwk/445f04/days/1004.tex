
\input amstex


\magnification=1400


\loadmsbm

\nopagenumbers
\parindent=0pt

\voffset=-.6in
\hoffset=-.4in
\hsize = 7.5 true in
\vsize=10.6 true in

%\voffset=1.2in
%\hoffset=-.5in
%\hsize = 10.2 true in
%\vsize=8 true in

\overfullrule=0pt


\def\ctln{\centerline}
\def\u{\underbar}
\def\ssk{\smallskip}
\def\msk{\medskip}
\def\bsk{\bigskip}
\def\hsk{\hskip.1in}
\def\hhsk{\hskip.2in}

\def\lra{$\Leftrightarrow$ }


\ctln{\bf Math 445 Number Theory}

\smallskip

\ctln{October 4, 2004}

\medskip

%We now know alot about the existence (and numbers of) $n^{\text th}$ roots modulo (odd) prime powers $p^k$, of integers
%$a$ relatively prime to $p$. But can we say more? Of course....

%\ssk

{\it Proposition:} If $f$ is a polynomial with integer coefficients and $(M,N)=1$, then the congruence equation
$f(x)\equiv 0\pmod{MN}$ has a solution $\Leftrightarrow$ the equations \hsk $f(x)\equiv 0\pmod{M}$ \hsk and \hsk $f(x)\equiv 0\pmod{N}$
\hsk both do.

\ssk

The direction ($\Rightarrow$) is immediate; $MN|f(x)$ implies $M|f(x)$ and $N|f(x)$, since $M,N|MN$ . The point to ($\Leftarrow$) 
is that the solutions we know of to each of the two equations might be {\it different}: $f(x_1)\equiv 0\pmod{M}$ \hsk and \hsk $f(x_2)\equiv 0\pmod{N}$ .
What we wish to show is that a single number solves {\it both}, since then $M|f(x_0)$  and $N|f(x_0)$ , and then $(M,N)=1$ implies that
$MN|f(x_0)$ . 

\ssk

To do this, we use the fact that $f$ is a polynomial, since then if $a\equiv b\pmod{n}$ , then $f(a)\equiv f(b)b\pmod{n}$ . So if we suppose that
we have found $a$ and $b$ with $f(a)\equiv 0\pmod{M}$ \hsk and \hsk $f(b)\equiv 0\pmod{N}$ , then any $x$ satisfying both 
$x\equiv a\pmod{M}$ and $x\equiv b\pmod{N}$ will satisfy both $f(x)\equiv 0\pmod{M}$ and $f(x)\equiv 0\pmod{N}$ simultaneously,
as desired. So it is enough to show that for any $a,b$ , there is an $x$ which simultaneously satisfies

\ctln{$x\equiv a\pmod{M}$ \hsk and \hsk $x\equiv b\pmod{N}$}

But since $(M,N)=1$, this is true by the Chinese Remainder Theorem. In fact, finding $x$ is a matter of solving
$x=a+Mi$ , $x=b+Nj$ , so we need $a+Mi=b+Nj$ , so $b-a = Mi-Nj$ . But since $(M,N)=1$, we can use the Euclidean 
algorithm to write $1=MI_0+NJ_0$, and then $i=(b-a)I_0,j=-(b-a)J_0$ will work, allowing us to solve for $x$. In fact, since the only other
$I,J$ which will work are $I=I_0+kN, J=J_0-kM$ , we find that our solution $x$ is unique modulo $MN$ . 

\ssk

For any pair of solutions $a,b$ to $f(a)\equiv 0\pmod{M}$ and $f(b)\equiv 0\pmod{N}$ there is a unique corresponding
$x$ mod $MN$ (with $x\equiv a\pmod{M}$ and $x\equiv b\pmod{N}$) satisfying $f(x)\equiv 0\pmod{MN}$. Iintroducing 
the notation $S(n)$ = the number of solutions, mod $n$, to the equation $f(x)\equiv 0\pmod{n}$ , we then have
shown that $S(MN)=S(M)S(N)$ whenever $(M,N)=1$ . So by induction, whenever $N_1,\ldots N_k$ 
are relatively prime, $S(N_1\cdots N_k) = S(N_1)\cdots S(N_k)$ . 

\ssk

So  if $N=p_1^{k_1}\cdots p_r^{k_r}$ is the prime factorization of the 
odd number $N$, then for any $(a,N)=1$ (so $(a,p_i)=1$ for each $i$) we have
\hsk $x^n\equiv a\pmod{N}$ has solutions $\Leftrightarrow$ $x^n\equiv a\pmod{p_i^{k_i}}$ does for every $i$ ,
\hsk and we know how to determine when that occurs.

\ssk

{\bf Quadratic Residues:} If $x^2\equiv a\pmod{n}$ has a solution, $a$ is a {\it quadratic residue} modulo $n$ . If it
doesn't, $a$ is a {\it quadratic non-residue} modulo $n$ . Euler's Criterion gives us a test: 
if $p$ is a prime, then $a$ is a quadratic residue mod $n$ $\Leftrightarrow$ $a^{{p-1}\over{2}}\equiv 1\pmod{p}$.
But this may require a lot of calculation if $p$ is large; our next task is to find a quicker way. 
%This method isknown as {\it quadratic resiprocity}. 

\ssk

To talk about things in a compact manner, we introduce the {\it Legendre symbol}; for $p$ an odd prime, 

\vskip-.1in

\ctln{$\displaystyle \Big({{a}\over{p}}\Big) = \cases 0 &\text{if }p|a \cr
1 &\text{if } a \text{ is a quadratic residue mod } p\cr
-1 &\text{if } a \text{ is a quadratic non-residue mod } p\cr
\endcases$}

\ssk

By Euler's criterion, this really means $\displaystyle \Big({{a}\over{p}}\Big) \equiv a^{{p-1}\over{2}}\pmod{p}$ , 
but our goal is to find a {\it quicker} way to compute it!

\vfill\end







