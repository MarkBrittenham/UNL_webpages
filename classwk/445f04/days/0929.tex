
\input amstex


\magnification=1400


\loadmsbm

\nopagenumbers
\parindent=0pt

\voffset=-.6in
\hoffset=-.4in
\hsize = 7.5 true in
\vsize=10.5 true in

%\voffset=1.2in
%\hoffset=-.5in
%\hsize = 10.2 true in
%\vsize=8 true in

\overfullrule=0pt


\def\ctln{\centerline}
\def\u{\underbar}
\def\ssk{\smallskip}
\def\msk{\medskip}
\def\bsk{\bigskip}

\def\lra{$\Leftrightarrow$ }


\ctln{\bf Math 445 Number Theory}

\smallskip

\ctln{September 29, 2004}

\medskip

Last time: \hskip.2in If $m$ is  an odd prime and $(a,m)=1$, then 

\hfill$x^2\equiv a\pmod{m}$ has $\displaystyle \cases 2 \text{ solutions,} &\text{if }a^{{m-1}\over{2}}\equiv 1\cr
                                                                                                     0 \text{ solutions, }&\text{if }\displaystyle a^{{m-1}\over{2}}\equiv -1\cr\endcases$

\msk

The only fact we really needed to know about the prime $m$, though, was that there was a primitive root, $b$, mod $m$. This, it
turns out, is true somewhat more generally, and will allow us to extend our result above, suitably modified. What is in fact true is:

\msk

{\it Theorem:} If $p$ is an odd prime and $k\geq 1$, then $m=p^k$ has a primitive root, i.e., there is an integer $b$ with 
ord$_{p^k}(b) = \Phi(p^k) = p^{k-1}(p-1)$ .

\msk

To see this, start with a primitive root $b$ modulo $p$ , i.e., ord$_p(b)=p-1$ , and consider the collection of integers

\ssk

\ctln{$A$ = $\{b+pk : 0\leq k\leq p-1\}$}

\ssk

We claim that for all but at most one $a\in A$, ord$_{p^2}(a) = p(p-1)$ . To see this, note that since $(a,p)=(b,p)=1$, $(a,p^2)=1$, 
so $a^{\Phi(p^2)} = a^{p(p-1)}\equiv 1\pmod{p^2}$  by Euler's Theorem, so ord$_{p^2}(a)|p(p-1)$. But 
$a^k\equiv 1\pmod{p^2}$ implies $a^k \equiv 1\pmod{p}$ and $a\equiv b\pmod{p}$, so $p-1|$ord$_{p^2}(a)$, so 
ord$_{p^2}(a)$ = $p-1$ or $p(p-1)$ . Our claim asserts that there is at most one $a$ where it is $p-1$.

\msk

So, suppose there are two! 

\ssk

Suppose $(b+k_1p)^{p-1}\equiv 1\equiv (b+k_2p)^{p-1}$, mod $p^2$ with $0\leq k_2<k_1\leq p-1$. Then 

\ssk

$p^2|(b+k_1p)^{p-1}-(b+k_2p)^{p-1}$ = 
$[(b+k_1p)-(b+k_2p)]\cdot [(b+k_1p)^{p-2}+(b+k_1p)^{p-3}(b+k_2p) + \cdots +(b+k_1p)(b+k_2p)^{p-3}+(b+k_2p)^{p-2}]$
= $p(k_1-k_2)$(stuff)

\ssk

So $p|(k_1-k_2)$(stuff) , so $p|(k_1-k_2)$ or $p|$(stuff) . But $0<k_1-k_2<p-1$ , so the first is impossible. And, mod $p$ , 

\ssk

stuff = $(b+k_1p)^{p-2}+(b+k_1p)^{p-3}(b+k_2p) + \cdots +(b+k_1p)(b+k_2p)^{p-3}+(b+k_2p)^{p-2}$

\hfill $\equiv(b)^{p-2}+(b)^{p-3}(b) + \cdots +(b)(b)^{p-3}+(b)^{p-2} = (p-1)b^{p-2}$

\ssk

and since $p\not |(p-1)$ , $p\not | b$ , $p$ can't divide this stuff, either. This gives us a contractiction, so there is at most
one value of  $0\leq k\leq p-1$ for which ord$_{p^2}(b+kp)=p-1$ . So for all of the others, ord$_{p^2}(b+kp)=p(p-1)$ , 
i.e., $b+kp$ is a primitive root modulo $p^2$ . 

\msk

Next time we will see that we get all other higher powers for free......



\vfill\end







