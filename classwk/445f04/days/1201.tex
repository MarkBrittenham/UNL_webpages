
\input amstex


\magnification=1400


\loadmsbm

\input colordvi

\nopagenumbers
\parindent=0pt

\def\cgy{\GreenYellow}     % GreenYellow  Approximate PANTONE 388
\def\cyy{\Yellow}	  % Yellow  Approximate PANTONE YELLOW
\def\cgo{\Goldenrod}	  % Goldenrod  Approximate PANTONE 109
\def\cda{\Dandelion}	  % Dandelion  Approximate PANTONE 123
\def\capr{\Apricot}	  % Apricot  Approximate PANTONE 1565
\def\cpe{\Peach}		  % Peach  Approximate PANTONE 164
\def\cme{\Melon}		  % Melon  Approximate PANTONE 177
\def\cyo{\YellowOrange}	  % YellowOrange  Approximate PANTONE 130
\def\coo{\Orange}	  % Orange  Approximate PANTONE ORANGE-021
\def\cbo{\BurntOrange}	  % BurntOrange  Approximate PANTONE 388
\def\cbs{\Bittersweet}	  % Bittersweet  Approximate PANTONE 167
%\def\creo{\RedOrange}	  % RedOrange  Approximate PANTONE 179
\def\cma{\Mahogany}	  % Mahogany  Approximate PANTONE 484
\def\cmr{\Maroon}	  % Maroon  Approximate PANTONE 201
\def\cbr{\BrickRed}	  % BrickRed  Approximate PANTONE 1805
\def\crr{\Red}		  % Red  VERY-Approx PANTONE RED
\def\cor{\OrangeRed}	  % OrangeRed  No PANTONE match
\def\paru{\RubineRed}	  % RubineRed  Approximate PANTONE RUBINE-RED
\def\cwi{\WildStrawberry}  % WildStrawberry  Approximate PANTONE 206
\def\csa{\Salmon}	  % Salmon  Approximate PANTONE 183
\def\ccp{\CarnationPink}	  % CarnationPink  Approximate PANTONE 218
\def\cmag{\Magenta}	  % Magenta  Approximate PANTONE PROCESS-MAGENTA
\def\cvr{\VioletRed}	  % VioletRed  Approximate PANTONE 219
\def\parh{\Rhodamine}	  % Rhodamine  Approximate PANTONE RHODAMINE-RED
\def\cmu{\Mulberry}	  % Mulberry  Approximate PANTONE 241
\def\parv{\RedViolet}	  % RedViolet  Approximate PANTONE 234
\def\cfu{\Fuchsia}	  % Fuchsia  Approximate PANTONE 248
\def\cla{\Lavender}	  % Lavender  Approximate PANTONE 223
\def\cth{\Thistle}	  % Thistle  Approximate PANTONE 245
\def\corc{\Orchid}	  % Orchid  Approximate PANTONE 252
\def\cdo{\DarkOrchid}	  % DarkOrchid  No PANTONE match
\def\cpu{\Purple}	  % Purple  Approximate PANTONE PURPLE
\def\cpl{\Plum}		  % Plum  VERY-Approx PANTONE 518
\def\cvi{\Violet}	  % Violet  Approximate PANTONE VIOLET
\def\parp{\RoyalPurple}	  % RoyalPurple  Approximate PANTONE 267
\def\cbv{\BlueViolet}	  % BlueViolet  Approximate PANTONE 2755
\def\cpe{\Periwinkle}	  % Periwinkle  Approximate PANTONE 2715
\def\ccb{\CadetBlue}	  % CadetBlue  Approximate PANTONE (534+535)/2
\def\cco{\CornflowerBlue}  % CornflowerBlue  Approximate PANTONE 292
\def\cmb{\MidnightBlue}	  % MidnightBlue  Approximate PANTONE 302
\def\cnb{\NavyBlue}	  % NavyBlue  Approximate PANTONE 293
\def\crb{\RoyalBlue}	  % RoyalBlue  No PANTONE match
%\def\cbb{\Blue}		  % Blue  Approximate PANTONE BLUE-072
\def\cce{\Cerulean}	  % Cerulean  Approximate PANTONE 3005
\def\ccy{\Cyan}		  % Cyan  Approximate PANTONE PROCESS-CYAN
\def\cpb{\ProcessBlue}	  % ProcessBlue  Approximate PANTONE PROCESS-BLUE
\def\csb{\SkyBlue}	  % SkyBlue  Approximate PANTONE 2985
\def\ctu{\Turquoise}	  % Turquoise  Approximate PANTONE (312+313)/2
\def\ctb{\TealBlue}	  % TealBlue  Approximate PANTONE 3145
\def\caq{\Aquamarine}	  % Aquamarine  Approximate PANTONE 3135
\def\cbg{\BlueGreen}	  % BlueGreen  Approximate PANTONE 320
\def\cem{\Emerald}	  % Emerald  No PANTONE match
\def\cjg{\JungleGreen}	  % JungleGreen  Approximate PANTONE 328
\def\csg{\SeaGreen}	  % SeaGreen  Approximate PANTONE 3268
\def\cgg{\Green}	  % Green  VERY-Approx PANTONE GREEN
\def\cfg{\ForestGreen}	  % ForestGreen  Approximate PANTONE 349
\def\cpg{\PineGreen}	  % PineGreen  Approximate PANTONE 323
\def\clg{\LimeGreen}	  % LimeGreen  No PANTONE match
\def\cyg{\YellowGreen}	  % YellowGreen  Approximate PANTONE 375
\def\cspg{\SpringGreen}	  % SpringGreen  Approximate PANTONE 381
\def\cog{\OliveGreen}	  % OliveGreen  Approximate PANTONE 582
\def\pars{\RawSienna}	  % RawSienna  Approximate PANTONE 154
\def\cse{\Sepia}		  % Sepia  Approximate PANTONE 161
\def\cbr{\Brown}		  % Brown  Approximate PANTONE 1615
\def\cta{\Tan}		  % Tan  No PANTONE match
\def\cgr{\Gray}		  % Gray  Approximate PANTONE COOL-GRAY-8
\def\cbl{\Black}		  % Black  Approximate PANTONE PROCESS-BLACK
\def\cwh{\White}		  % White  No PANTONE match


\voffset=-.5in
\hoffset=-.3in
\hsize = 7.2 true in
\vsize=10.2 true in

%\voffset=1.2in
%\hoffset=-.5in
%\hsize = 10.2 true in
%\vsize=8 true in

\overfullrule=0pt


\def\ctln{\centerline}
\def\u{\underbar}
\def\ssk{\smallskip}
\def\msk{\medskip}
\def\bsk{\bigskip}
\def\hsk{\hskip.1in}
\def\hhsk{\hskip.2in}
\def\dsl{\displaystyle}
\def\hskp{\hskip1.5in}

\def\delx{{{\partial}\over{\partial x}}}
\def\dely{{{\partial}\over{\partial y}}}

\def\moda{\medspace {\underset a\to \equiv} \medspace}
\def\modb{\medspace {\underset b\to \equiv} \medspace}
\def\modc{\medspace {\underset c\to \equiv} \medspace}

\def\lra{$\Leftrightarrow$ }


\ctln{\bf Math 445 Number Theory}

\ssk

\ctln{December 1, 2004}

\msk


We can now apply our geometric approach to more general polynomial
equations, in particular to {\it cubic} equations. $f(x,y)$ has
rational coefficients, and the line $y=mx+r$ meets ${\Cal C}_f({\Bbb R})$
in two rational solutions, then $p(x)=f(x,mx+r)$ is a cubic polynomial
with rational coefficients and two rational roots, and so,
as before, the third root is also rational, and gives a third 
rational point on ${\Cal C}_f({\Bbb R})$. But in this case there
are three ways to find such lines:

\ssk

(a): find two distinct rational points, and the line through them,

(b): find a double point $(x_0,y_0)$ in ${\Cal C}_f({\Bbb R})$, then any line
with rational slope through $(x_0,y_0)$ will give $f(x,mx+r)$ has $x_0$ as
a double root,

(c): find a rational point $(x_0,y_0)$, then for the tangent line to the graph 
of ${\Cal C}_f({\Bbb R})$, $f(x,mx+r)$ has $x_0$ as a double root.

\ssk

Taken in turn, these can in many cases find infinitely many rational 
points on a cubic curve.

\msk

For example, on the curve $x^3+y^3=9$, starting from the 
point $(2,1)$, with $f(x,y)=x^3+y^3-9$, we can compute 
$f_x(2,1) = 12 , f_y(2,1) = 3$, and so the tangent line is
$(12,3)\bullet(x-2,y-1)=0$ so $y=9-4x$, and so $x^3+(9-4x)^3-9 = (x-2)^2(180-63x)$,
giving a new solution $(20/7,-17/7)$ . Repeatedly using their tangent lines, 
we can find further solutions. 

\ssk

A double point example: $f(x,y)=y^2-x^3+2x^2=0$ has $f_x=-3x^2+4x$, $f_y=2y$,
and all three are 0 at $(0,0)$. If we look at the lines through $(0,0)$ with rational
slope, $y=mx$, and solve $m^2x^2-x^3+2x^2 = x^2((m^2+2)-x) = $ gives
$x=m^2+2$ and $y=m^3+2m$.

\msk

Why do tangent lines $y=mx+b$ give double roots of $f(x,mx+b)=0$ at
the point of tangency? This is just a little (multivariate) calculus. If $(a,b)$
is our rational point, then the equation for its tangent line is

\ssk

$f_x(a,b)(x-a)+f_y(a,b)(y-b)=0$ , and so we wish to solve 

$\dsl p(x) = f(x,-{{f_x(a,b)}\over{f_y(a,b)}}(x-a)+b) = 0$, which has
$p(a)=0$ and 

$\dsl p^\prime(a) = f_x(a,b) + f_y(a,b)L^\prime(a) = f_x(a,b) + f_y(a,b)(-{{f_x(a,b)}\over{f_y(a,b)}}) = 0$ , as desired.

\ssk

Integer points on ${\Cal C}_f({\Bbb R})$, $f(x,y)=x^3+y^3-M$ ? 
$x^3+y^3=M=(x+y)(x^2-xy+y^2) = AB$, then 
$\dsl |M|\geq |B| = |x^2-xy+y^2| = (x-{{y}\over{2}})^2+{{3}\over{4}}y^2\geq {{3}\over{4}}y^2$
so $\dsl |y|\leq {{2}\over{\sqrt{3}}}\sqrt{|M|}$ (and, by symmetry, 
$\dsl |x|\leq {{2}\over{\sqrt{3}}}\sqrt{|M|}$), so we can check for integer solutions, by hand.

\vfill
\end


























