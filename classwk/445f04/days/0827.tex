
\input amstex
\loadmsbm

\nopagenumbers
\parindent=-20pt
\voffset=-.6in

\magnification=1400

\def\ctln{\centerline}
\def\u{\underbar}
\def\zzz{{\Bbb Z}}

\ctln{\bf Math 445 Number Theory}

\medskip

\ctln{August 27, 2004}

\bigskip

An integer $p\geq 2$ is {\it prime} if the only $a|p$ are
$\pm 1$ and $\pm p$ .

\medskip

{\bf Fundamental Theorem of Arithmetic}: Every integer is a product 
of primes, unique up to re-ordering.

\smallskip

Because: if there is an $n$ which isn't, then there is a {\it smallest}
one; then it isn't prime (else $n=p$ is the product), so $n=ab$ with
$1<a,b<n$, so each is a product of primes, so $n$ is a product of \u{their}
products, a contradiction.

Uniqueness: need {\it If $p$ is prime and $p|a_1\cdots a_n$, then $p|a_i$
for some $i$.} Then if $n=p_1\cdots p_k = q_1\cdots q_l$, then $p_1|q_i$
for some $i$, so $p_1=q_i$, so $p_2\cdots p_k = q_1\cdots q_{i-1}q_{i+1}\cdots
q_l$. Continuing, we can pair all the $p$'s with $q$'s. [Better? If not
always unique, there is a {\it smallest} number without unique factorization;
structure proof as before.]

\medskip

Completely factoring a number a la FTA has two parts; find factors, and 
decide when they are prime. But how do you decide that a number $N\geq 2$ 
is prime?

\medskip

(1) $a|b$ implies $|a|\leq |b|$. So check that no $1<a<N$ divides $N$ .

\smallskip

(2) $N=ab$ implies $|a|\leq \sqrt{|N|}$ or $|b|\leq \sqrt{|N|}$ . So 
check that no $1<a\leq\sqrt{N}$ divides $N$ .

\smallskip

(3) A prime factorization $N=p_1\cdots p_k$ with $p_1\leq p_2\leq \ldots \leq p_k$
is unique. Then (if $k\geq 2$, i.e., N is not prime) 

\smallskip

\ctln{$p_1^2$ $\leq$ $p_1^k$ = $p_1\cdots p_1$ $\leq$ $p_1\cdots p_k$ = $N$, so $p_1\leq
\sqrt{N}$}

\smallskip

So check that no {\it prime} $p$, $1<p \leq\sqrt{N}$ divides $N$ .

\bigskip

Almost every other primality (or factoring) test involves {\it Fermat's
Little Theorem}.

\bigskip

{\it{\bf Typo on Introduction sheet:}}

\medskip

(2) $d$ = gcd$(a,b)$ is the smallest {\it positive} number that can be written as $d=ax+by$ with $x,y\in\zzz$ .
({\bf Not} with $a,b\in\zzz$ ...) There is a similar typo in property (4).

\vfill\end







