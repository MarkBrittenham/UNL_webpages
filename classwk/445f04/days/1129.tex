
\input amstex


\magnification=1400


\loadmsbm

\input colordvi

\nopagenumbers
\parindent=0pt

\def\cgy{\GreenYellow}     % GreenYellow  Approximate PANTONE 388
\def\cyy{\Yellow}	  % Yellow  Approximate PANTONE YELLOW
\def\cgo{\Goldenrod}	  % Goldenrod  Approximate PANTONE 109
\def\cda{\Dandelion}	  % Dandelion  Approximate PANTONE 123
\def\capr{\Apricot}	  % Apricot  Approximate PANTONE 1565
\def\cpe{\Peach}		  % Peach  Approximate PANTONE 164
\def\cme{\Melon}		  % Melon  Approximate PANTONE 177
\def\cyo{\YellowOrange}	  % YellowOrange  Approximate PANTONE 130
\def\coo{\Orange}	  % Orange  Approximate PANTONE ORANGE-021
\def\cbo{\BurntOrange}	  % BurntOrange  Approximate PANTONE 388
\def\cbs{\Bittersweet}	  % Bittersweet  Approximate PANTONE 167
%\def\creo{\RedOrange}	  % RedOrange  Approximate PANTONE 179
\def\cma{\Mahogany}	  % Mahogany  Approximate PANTONE 484
\def\cmr{\Maroon}	  % Maroon  Approximate PANTONE 201
\def\cbr{\BrickRed}	  % BrickRed  Approximate PANTONE 1805
\def\crr{\Red}		  % Red  VERY-Approx PANTONE RED
\def\cor{\OrangeRed}	  % OrangeRed  No PANTONE match
\def\paru{\RubineRed}	  % RubineRed  Approximate PANTONE RUBINE-RED
\def\cwi{\WildStrawberry}  % WildStrawberry  Approximate PANTONE 206
\def\csa{\Salmon}	  % Salmon  Approximate PANTONE 183
\def\ccp{\CarnationPink}	  % CarnationPink  Approximate PANTONE 218
\def\cmag{\Magenta}	  % Magenta  Approximate PANTONE PROCESS-MAGENTA
\def\cvr{\VioletRed}	  % VioletRed  Approximate PANTONE 219
\def\parh{\Rhodamine}	  % Rhodamine  Approximate PANTONE RHODAMINE-RED
\def\cmu{\Mulberry}	  % Mulberry  Approximate PANTONE 241
\def\parv{\RedViolet}	  % RedViolet  Approximate PANTONE 234
\def\cfu{\Fuchsia}	  % Fuchsia  Approximate PANTONE 248
\def\cla{\Lavender}	  % Lavender  Approximate PANTONE 223
\def\cth{\Thistle}	  % Thistle  Approximate PANTONE 245
\def\corc{\Orchid}	  % Orchid  Approximate PANTONE 252
\def\cdo{\DarkOrchid}	  % DarkOrchid  No PANTONE match
\def\cpu{\Purple}	  % Purple  Approximate PANTONE PURPLE
\def\cpl{\Plum}		  % Plum  VERY-Approx PANTONE 518
\def\cvi{\Violet}	  % Violet  Approximate PANTONE VIOLET
\def\parp{\RoyalPurple}	  % RoyalPurple  Approximate PANTONE 267
\def\cbv{\BlueViolet}	  % BlueViolet  Approximate PANTONE 2755
\def\cpe{\Periwinkle}	  % Periwinkle  Approximate PANTONE 2715
\def\ccb{\CadetBlue}	  % CadetBlue  Approximate PANTONE (534+535)/2
\def\cco{\CornflowerBlue}  % CornflowerBlue  Approximate PANTONE 292
\def\cmb{\MidnightBlue}	  % MidnightBlue  Approximate PANTONE 302
\def\cnb{\NavyBlue}	  % NavyBlue  Approximate PANTONE 293
\def\crb{\RoyalBlue}	  % RoyalBlue  No PANTONE match
%\def\cbb{\Blue}		  % Blue  Approximate PANTONE BLUE-072
\def\cce{\Cerulean}	  % Cerulean  Approximate PANTONE 3005
\def\ccy{\Cyan}		  % Cyan  Approximate PANTONE PROCESS-CYAN
\def\cpb{\ProcessBlue}	  % ProcessBlue  Approximate PANTONE PROCESS-BLUE
\def\csb{\SkyBlue}	  % SkyBlue  Approximate PANTONE 2985
\def\ctu{\Turquoise}	  % Turquoise  Approximate PANTONE (312+313)/2
\def\ctb{\TealBlue}	  % TealBlue  Approximate PANTONE 3145
\def\caq{\Aquamarine}	  % Aquamarine  Approximate PANTONE 3135
\def\cbg{\BlueGreen}	  % BlueGreen  Approximate PANTONE 320
\def\cem{\Emerald}	  % Emerald  No PANTONE match
\def\cjg{\JungleGreen}	  % JungleGreen  Approximate PANTONE 328
\def\csg{\SeaGreen}	  % SeaGreen  Approximate PANTONE 3268
\def\cgg{\Green}	  % Green  VERY-Approx PANTONE GREEN
\def\cfg{\ForestGreen}	  % ForestGreen  Approximate PANTONE 349
\def\cpg{\PineGreen}	  % PineGreen  Approximate PANTONE 323
\def\clg{\LimeGreen}	  % LimeGreen  No PANTONE match
\def\cyg{\YellowGreen}	  % YellowGreen  Approximate PANTONE 375
\def\cspg{\SpringGreen}	  % SpringGreen  Approximate PANTONE 381
\def\cog{\OliveGreen}	  % OliveGreen  Approximate PANTONE 582
\def\pars{\RawSienna}	  % RawSienna  Approximate PANTONE 154
\def\cse{\Sepia}		  % Sepia  Approximate PANTONE 161
\def\cbr{\Brown}		  % Brown  Approximate PANTONE 1615
\def\cta{\Tan}		  % Tan  No PANTONE match
\def\cgr{\Gray}		  % Gray  Approximate PANTONE COOL-GRAY-8
\def\cbl{\Black}		  % Black  Approximate PANTONE PROCESS-BLACK
\def\cwh{\White}		  % White  No PANTONE match


\voffset=-.5in
\hoffset=-.3in
\hsize = 7.2 true in
\vsize=10.2 true in

%\voffset=1.2in
%\hoffset=-.5in
%\hsize = 10.2 true in
%\vsize=8 true in

\overfullrule=0pt


\def\ctln{\centerline}
\def\u{\underbar}
\def\ssk{\smallskip}
\def\msk{\medskip}
\def\bsk{\bigskip}
\def\hsk{\hskip.1in}
\def\hhsk{\hskip.2in}
\def\dsl{\displaystyle}
\def\hskp{\hskip1.5in}

\def\delx{{{\partial}\over{\partial x}}}
\def\dely{{{\partial}\over{\partial y}}}

\def\moda{\medspace {\underset a\to \equiv} \medspace}
\def\modb{\medspace {\underset b\to \equiv} \medspace}
\def\modc{\medspace {\underset c\to \equiv} \medspace}

\def\lra{$\Leftrightarrow$ }


\ctln{\bf Math 445 Number Theory}

\ssk

\ctln{November 29, 2004}

\msk

{\bf Equations of higher degree:}

\msk

Our geometric approach to finding rational solutions to 
quadratic equations can be applied to higher degree equations
as well. If $f(x,y)$ is a polynomial of two variables, 
with total degree $d$, we will use the notation
\hhsk
${\Cal C}_f({\Bbb R}) = \{(x,y)\in{\Bbb R}^2 : f(x,y)=0\}$ 

\ssk

Our goal is to find the rational points ($x,y\in{\Bbb Q}$) in 
${\Cal C}_f({\Bbb R})$.

\msk

Since, as before, the line through two rational solutions has rational 
slope, we can try to search for rational solutions, given one 
solution, by looking at lines with rational slope. The most generic
equation for a line is $ax+by+c=0$, but we will typically think of it as
$y=mx+r$ . A point lying on a line $L$ and on ${\Cal C}_f({\Bbb R})$
satisfies both $f(x,y)=0$ and $y=mx+r$, so it satisfies $p(x)=f(x,mx+r)=0$ .
Since this is a polynomial in $x$ of degree at most $d$, it has 
at most $d$ roots, unless it is identically 0. But if $p(x)$ is identically 0,
then $y=mx+r$ implies $f(x,y)=0$ . So $L\subseteq {\Cal C}_f({\Bbb R})$.
So we have:

\ssk

{\it Theorem:} If $f(x,y)$ is a polynomial of degree $d$, and the line $L$
intersects ${\Cal C}_f({\Bbb R})$ in more than $d$ points, then 
$L\subseteq {\Cal C}_f({\Bbb R})$.

\ssk

In fact, even more is true: using polynomial long division (thinking
of $f(x,y)$ as a polynomial in $y$ with coefficients being polynomials 
in $x$), if $L$, given by $ax+by+c=0$ meets ${\Cal C}_f({\Bbb R})$
in more than $d$=degree$(f)$ points, then $f(x,y)=(ax+by+c)k(x,y)$
for some polynomial $k$. And perhaps just as important for our
purposes, if $f$ has rational coefficients, and $a,b,c$ are rational,
then $k$ has rational coefficients. The same is true if we have
integer coefficients.

\msk

This can be further refined if we introduce the {\it multplicity} of
a root of $f(x,y)=0$ . Building in analogy with the one variable case:
$x=1$ is a multiple root of $f(x)=x^3-x^2-x+1$ means $(x-1)^2|f(x)$
( in this case, $f(x)=(x-1)^2(x+1)$ ), which in turn means $f(1)=0$ 
and $f^\prime(1)=0$ .
In the two-variable case, the multiplicity $M$ of a solution $(a,b)$
to $f(x,y)=0$ is the largest $M$ so that
$\dsl (\delx)^i(\dely)^jf(a,b)=0$ for all $i+j\leq M$ . Then the count 
of roots of $f(x,y)=0$ can include their multiplicity, and the result above
is still true. A point on ${\Cal C}_f({\Bbb R})$ with multiplicity 1 is
called {\it smooth}, a point with multiplicity 2 is a {\it double point},
etc. A point with multiplicity greater than 1 is called {\it singular}.
If all points are smooth, then ${\Cal C}_f({\Bbb R})$ is called
smooth.

\vfill\end

























