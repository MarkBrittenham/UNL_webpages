
\input amstex


\magnification=1200


\loadmsbm

\input colordvi

\nopagenumbers
\parindent=0pt

\def\cgy{\GreenYellow}     % GreenYellow  Approximate PANTONE 388
\def\cyy{\Yellow}	  % Yellow  Approximate PANTONE YELLOW
\def\cgo{\Goldenrod}	  % Goldenrod  Approximate PANTONE 109
\def\cda{\Dandelion}	  % Dandelion  Approximate PANTONE 123
\def\capr{\Apricot}	  % Apricot  Approximate PANTONE 1565
\def\cpe{\Peach}		  % Peach  Approximate PANTONE 164
\def\cme{\Melon}		  % Melon  Approximate PANTONE 177
\def\cyo{\YellowOrange}	  % YellowOrange  Approximate PANTONE 130
\def\coo{\Orange}	  % Orange  Approximate PANTONE ORANGE-021
\def\cbo{\BurntOrange}	  % BurntOrange  Approximate PANTONE 388
\def\cbs{\Bittersweet}	  % Bittersweet  Approximate PANTONE 167
%\def\creo{\RedOrange}	  % RedOrange  Approximate PANTONE 179
\def\cma{\Mahogany}	  % Mahogany  Approximate PANTONE 484
\def\cmr{\Maroon}	  % Maroon  Approximate PANTONE 201
\def\cbr{\BrickRed}	  % BrickRed  Approximate PANTONE 1805
\def\crr{\Red}		  % Red  VERY-Approx PANTONE RED
\def\cor{\OrangeRed}	  % OrangeRed  No PANTONE match
\def\paru{\RubineRed}	  % RubineRed  Approximate PANTONE RUBINE-RED
\def\cwi{\WildStrawberry}  % WildStrawberry  Approximate PANTONE 206
\def\csa{\Salmon}	  % Salmon  Approximate PANTONE 183
\def\ccp{\CarnationPink}	  % CarnationPink  Approximate PANTONE 218
\def\cmag{\Magenta}	  % Magenta  Approximate PANTONE PROCESS-MAGENTA
\def\cvr{\VioletRed}	  % VioletRed  Approximate PANTONE 219
\def\parh{\Rhodamine}	  % Rhodamine  Approximate PANTONE RHODAMINE-RED
\def\cmu{\Mulberry}	  % Mulberry  Approximate PANTONE 241
\def\parv{\RedViolet}	  % RedViolet  Approximate PANTONE 234
\def\cfu{\Fuchsia}	  % Fuchsia  Approximate PANTONE 248
\def\cla{\Lavender}	  % Lavender  Approximate PANTONE 223
\def\cth{\Thistle}	  % Thistle  Approximate PANTONE 245
\def\corc{\Orchid}	  % Orchid  Approximate PANTONE 252
\def\cdo{\DarkOrchid}	  % DarkOrchid  No PANTONE match
\def\cpu{\Purple}	  % Purple  Approximate PANTONE PURPLE
\def\cpl{\Plum}		  % Plum  VERY-Approx PANTONE 518
\def\cvi{\Violet}	  % Violet  Approximate PANTONE VIOLET
\def\parp{\RoyalPurple}	  % RoyalPurple  Approximate PANTONE 267
\def\cbv{\BlueViolet}	  % BlueViolet  Approximate PANTONE 2755
\def\cpe{\Periwinkle}	  % Periwinkle  Approximate PANTONE 2715
\def\ccb{\CadetBlue}	  % CadetBlue  Approximate PANTONE (534+535)/2
\def\cco{\CornflowerBlue}  % CornflowerBlue  Approximate PANTONE 292
\def\cmb{\MidnightBlue}	  % MidnightBlue  Approximate PANTONE 302
\def\cnb{\NavyBlue}	  % NavyBlue  Approximate PANTONE 293
\def\crb{\RoyalBlue}	  % RoyalBlue  No PANTONE match
%\def\cbb{\Blue}		  % Blue  Approximate PANTONE BLUE-072
\def\cce{\Cerulean}	  % Cerulean  Approximate PANTONE 3005
\def\ccy{\Cyan}		  % Cyan  Approximate PANTONE PROCESS-CYAN
\def\cpb{\ProcessBlue}	  % ProcessBlue  Approximate PANTONE PROCESS-BLUE
\def\csb{\SkyBlue}	  % SkyBlue  Approximate PANTONE 2985
\def\ctu{\Turquoise}	  % Turquoise  Approximate PANTONE (312+313)/2
\def\ctb{\TealBlue}	  % TealBlue  Approximate PANTONE 3145
\def\caq{\Aquamarine}	  % Aquamarine  Approximate PANTONE 3135
\def\cbg{\BlueGreen}	  % BlueGreen  Approximate PANTONE 320
\def\cem{\Emerald}	  % Emerald  No PANTONE match
\def\cjg{\JungleGreen}	  % JungleGreen  Approximate PANTONE 328
\def\csg{\SeaGreen}	  % SeaGreen  Approximate PANTONE 3268
\def\cgg{\Green}	  % Green  VERY-Approx PANTONE GREEN
\def\cfg{\ForestGreen}	  % ForestGreen  Approximate PANTONE 349
\def\cpg{\PineGreen}	  % PineGreen  Approximate PANTONE 323
\def\clg{\LimeGreen}	  % LimeGreen  No PANTONE match
\def\cyg{\YellowGreen}	  % YellowGreen  Approximate PANTONE 375
\def\cspg{\SpringGreen}	  % SpringGreen  Approximate PANTONE 381
\def\cog{\OliveGreen}	  % OliveGreen  Approximate PANTONE 582
\def\pars{\RawSienna}	  % RawSienna  Approximate PANTONE 154
\def\cse{\Sepia}		  % Sepia  Approximate PANTONE 161
\def\cbr{\Brown}		  % Brown  Approximate PANTONE 1615
\def\cta{\Tan}		  % Tan  No PANTONE match
\def\cgr{\Gray}		  % Gray  Approximate PANTONE COOL-GRAY-8
\def\cbl{\Black}		  % Black  Approximate PANTONE PROCESS-BLACK
\def\cwh{\White}		  % White  No PANTONE match


\voffset=-.7in
\hoffset=-.5in
\hsize = 7.5 true in
\vsize=10.5 true in

%\voffset=1.2in
%\hoffset=-.5in
%\hsize = 10.2 true in
%\vsize=8 true in

\overfullrule=0pt


\def\ctln{\centerline}
\def\u{\underbar}
\def\ssk{\smallskip}
\def\msk{\medskip}
\def\bsk{\bigskip}
\def\hsk{\hskip.1in}
\def\hhsk{\hskip.2in}
\def\dsl{\displaystyle}
\def\hskp{\hskip1.5in}

\def\moda{\medspace {\underset a\to \equiv} \medspace}
\def\modb{\medspace {\underset b\to \equiv} \medspace}
\def\modc{\medspace {\underset c\to \equiv} \medspace}

\def\lra{$\Leftrightarrow$ }


\ctln{\bf Math 445 Number Theory}

\ssk

\ctln{November 19, 2004}

\msk

\crr{{\it Theorem:} If $abc$ is square-free, then $ax^2+by^2+cz^2=0$ has a 
(non-trfvial!) solution
$x,y,z\in{\Bbb Z}$ \lra\  $a,b,c$ do not all have the same sign, and each of 
the equations }

\crr{\hhsk $w^2\equiv -ab\pmod{c} ,w^2\equiv -ac\pmod{b}, w^2\equiv -bc\pmod{a}$
\hhsk have solutions.}

\msk

($\Rightarrow$ :) WOLOG $x,y,z$ have no common factor. If $(c,x)>1$, then choosing
some prime $p|c,x$ we have $p|-ax^2-cz^2=by^2$ but $p\not |b$, so $p|y$. Then 
$p^2|ax^2+by^2=-cz^2$, so either $p^2|c$ or $p|z$ (both contradictions) . so $(c,x)=1$.
Choosing $u$ so that $ux\equiv 1\pmod{c}$ we have, mod $c$, 
$0\equiv (u^2b)(ax^2+by^2) = (ab)(ux)^2+(uby)^2\equiv ab+(uby)^2$ , so $w^2=(uby)^2\equiv -ab$ .
A similar argument establishes the other two congruences.

\msk

So, for example, \crr{$35x^2+23y^2 -6z^2=0$ has no integer solutions},
because $35\cdot 23\cdot -6 = -2\cdot 3\cdot 5\cdot 7\cdot 23$ is square-free 
and $w^2\equiv -23\cdot -6 = 138\pmod{35}$ has no solutions,
since $\dsl \Big({{138}\over{5}}\Big) = \dsl \Big({{3}\over{5}}\Big) = -1$, 
so $w^2\equiv 138\pmod{5}$ has no solutions.
\hhsk On the other hand, \hhsk
\crr{$5x^2+7y^2=13z^2$ has integer solutions}, since 
$\dsl \Big({{91}\over{5}}\Big)=\Big({{65}\over{7}}\Big)=\Big({{-35}\over{13}}\Big)=1$ , as we can readily compute;
they are, respectively, $\dsl \Big({{1}\over{5}}\Big)=1 , \Big({{2}\over{7}}\Big)=(-1)^6=1$ , and 
$\dsl \Big({{4}\over{13}}\Big) = \Big({{2}\over{13}}\Big)^2=1$ .

\msk

And if $abc$ is not square-free? If $d^2|$ one of $a,b,c$, say $d^2|a$, then we write
$a=d^2a^\prime$ and if $ax^2+by^2+cz^2=0$ , then $a^\prime(dx)^2+by^2+cz^2=0$ 
so $a^\prime X^2+bY^2+cZ^2=0$ has a solution. Conversely, if $a^\prime X^2+bY^2+cZ^2=0$,
then $a^\prime d^2X^2+bd^2Y^2+cd^2Z^2=0 = aX^2+b(dY)^2+c(dZ)^2$ , so $ax^2+by^2+cz^2=0$
has  solution. So we can test for solutions to $ax^2+by^2+cz^2=0$ by checking $a^\prime X^2+bY^2+cZ^2=0$ ,
with $a^\prime bc = abc/d^2 < abc$ .
And if $d|$ two of $a,b,c$, say $d|a,b$, then $a=dA,b=dB$ and if 
$ax^2+by^2+cz^2=0$ , then $Adx^2+Bdy^2+cz^2=0$ so $Ad^2x^2+Bd^2y^2+cdz^2=0 = A(dx)^2+B(dy)^2+(cd)z^2=0$ 
with $AB(cd) = abc/d < abc$ . Conversely, if $AX^2+BY^2+(cd)Z^2=0$, then 
$AdX^2+BdY^2+cd^2Z^2=0 = aX^2+bY^2+c(dZ)^2=0$ so $ax^2+by^2+cz^2=0$
has a solution. So by induction, we can test whether $ax^2+by^2+cz^2=0$ has solutions
by testing if some $a^\prime x^2+b^\prime y^2+c^\prime z^2=0$ , with $a^\prime b^\prime c^\prime$ square-free,
has solutions. 


\msk

\crr{If we actually want to \underbar{find} the solutions}, we can use an approach from geometry. We'll 
start by illustrating this with an equation we already know how to solve: $x^2+y^2-z^2=0$ . If we
write this as $\dsl \Big({{x}\over{z}}\Big)^2 + \Big({{y}\over{z}}\Big)^2 = 1$, we find ourselves 
looking for {\it rational} solutions to $a^2+b^2=1$ , i.e., rational points on the unit circle. 

\ssk

The key idea is to look at how {\it lines} intersect the circle $x^2+y^2-1=0$ . If we set $y=rx+s$ and plug in,
we have a quadratic equation $x^2+(rx+s)^2-1=0$ in $x$, describing the $x$-coordinates of the 
points of intersection of line and circle. If we know one of these points $(x_0,y_0)$, then $(x-x_0)|(x^2+(rx+s)^2-1)$,
and so the \underbar{other} factor of $x^2+(rx+s)^2-1$ is also linear, and setting it equal to 0 gives the $x$-coordinate
of the \underbar{other} point of intersection. But the \underbar{real} key idea is that if $x_0,y_0$ and $r$ 
are all rational (i.e., we know a rational point on the circle, e.g., $(1,0)$) then the other point of 
intersection has rational coordinates, because that other linear factor has rational coefficients. 
Conversely, the slope of a line between points with 
rational coordinates is rational; this means that this process will find \underbar{all} rational points
on the unit circle.

\msk

Putting this into practice, if we start with $(x_0,y_0)=(1,0)$ , which is a solution to  $x^2+y^2=1$, and look at the line
through $(1,0)$ with rational slope $r$, having equation $y=r(x-1) = rx-r$ , and plug in, we need to solve
$x^2+r^2(x^2-2x+1)-1=0 = (1+r^2)x^2-2r^2x+(r^2-1) = (x-1)((r^2+1)x-(r^2-1))$, so $x=1$ (our original
solution) or $\dsl x={{r^2-1}\over{r^2+1}}$, which implies (by plugging into $y=rx-r$) that $\dsl y={{2r}\over{r^2+1}}$ . If we 
write $\dsl r={{u}\over{v}}$ and simplify, we have $\dsl (x,y) = ({{u^2-v^2}\over{u^2+v^2}},{{2uv}\over{u^2+v^2}})$,
giving solutions $(u^2-v^2,2uv,u^2+v^2)$ to $x^2+y^2=z^2$ . Which are all of the Pythagorean
triples, as we have seen before!

\vfill
\end

\msk

This geometric process works for any equation 
$ax^2+by^2+cz^2=0$ , i.e., $aX^2+bY^2+c=0$ , for which we
know a single rational solution, to find all rational solutions to $aX^2+bY^2+c=0$ 
(and hence all integer solutions to $ax^2+by^2+cz^2=0$). For example, knowing that 
$2x^2+3y^2=5$ has solution $(x_0,y_0)=(1,1)$ we find, using the line 
$y=r(x-1)+1$ with rational slope $r$ through $(1,1)$, that 

\ssk

$0=2x^2+3y^2-5 = 2x^2+3(rx-r+1)^2-5 = 2x^2+3(r^2x^2-2r^2x+2rx+r^2-2r+1)-5 =
(2+3r^2)x^2+(6r-6r^2)x+(3r^2-6r-2) = ((x-1)((2+3r^2)x-(3r^2-6r-2))$ , so $x=1$ or
$\dsl x={{3r^2-6r-2}\over{3r^2+2}}$ , giving $\dsl y=rx-r+1 = r{{3r^2-6r-2}\over{3r^2+2}}-r+1 = 
-{{3r^2+4r-2}\over{3r^2+2}}$ . Setting $\dsl r={{u}\over{v}}$ , we get, as before,

$2(3u^2-6uv-2v^2)^2+3(3u^2+4uv-2v^2)^2=5(3u^2+2v^2)$ . For example, setting
$u=11,v=4$, we have $2(67)^2+3(507)^2=5(395)^2$.

\msk

The hard part: finding the first solution! For the special situation $x^2+y^2=nz^2$, we know from 
a homework problem awhile back that if $X^2+Y^2=n$ has a rational solution, then it 
has an {\it integer} solution, which we can look for by an exhaustive search among
$0\leq X,Y\leq \sqrt{n}$ ! Note that our newest result gives us a quick criterion to decide
if $x^2+y^2=n$ has an integer solution, when $n$ is square-free; 
we need $\dsl\Big({{-1}\over{n}}\Big) = 1$ . The interested reader can check that this 
really is equaivalent to the (slightly more long-winded) answer we found before...


\vfill
\end



























