
\magnification=1400
\nopagenumbers

\parindent=-20pt

\voffset=-.6in
\vsize=10in

\overfullrule=0pt

\def\ctln{\centerline}
\def\u{\underbar}


\input epsf

\ctln{\bf Math 445 Number Theory}

\medskip

\ctln{August 23 and 25, 2004}

\bigskip

Number theory is about {\it finding} and {\it explaining}
patterns in numbers.

\bigskip

Ulam Sprial:

\medskip

\vbox{\hsize=1.6in

\leavevmode

\epsfxsize=1.6in
\epsfbox{ulam.ai}}


\vskip-1.6in

\hskip2.6in
\vbox{\hsize=2.2in Place the natural numbers in a
rectangular spiral. The primes tend to fall on certain
diagonal lines with more frequency than it seems they should?

\medskip

This means: for certain values of $\alpha,\beta,\gamma$, the sequences
$n^2+\alpha$ , $n^2+n+\beta$ , $n^2-n+\gamma$ have more primes than
we {\it expect} them to.

\medskip

{\it Why?} We don't yet know...}

\vskip.3in


Modulus:  $a\equiv b\pmod{n}$ means $a$ and $b$ leave the same remainder
when you divide by $n$ (i.e., $n$ evenly divides $b-a$ ; we write $n|b-a$).

\medskip

$317 = 11^2+14^2$, but $319$ cannot be expressed that way. In fact, 

\ctln{if $n=a^2+b^2$, then $n\equiv 0, 1,$ or $2\pmod{4}$}

We will explore {\it which ones} are a sum of two squares later on.

\bigskip

Similarly, if $n=a^3+b^3+c^3$, then $n\not\equiv 4,5\pmod{9}$ . A conjecture
(of ``Waring type'') states that 

\ctln{if $n\not\equiv 4,5\pmod{9}$, then $n=a^3+b^3+c^3$}

This is still unresolved.

\bigskip

Egyptian fractions:

Any rational number $m/n$ can be written as a sum of reciprocals $1/a$ of
integers. In fact, by repeatedly subtracting the largest reciprocal that we
can from whatever is left, we find that 

\ctln{$\displaystyle {{m}\over{n}} = {{1}\over{a_1}} + \cdots + {{1}\over{a_k}}$}

with $a_1<a_2< \ldots < a_k$ and $k\leq n$. But not every fraction $3/n$ can 
be expressed as a sum of {\it two} reciprocals (e.g, 3/7). However, it is
conjectured (the Erd\"os-Strauss Conjecture) that

\ctln{every fraction $\displaystyle {{4}\over{n}}$ is the sum of at most 3 reciprocals.}

This has been verified to $n=10^{14}$, but still remains open.

\vfill\end







