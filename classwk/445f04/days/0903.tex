
\input amstex


%\load amsbm

\nopagenumbers
\parindent=-20pt
\voffset=-.6in

\magnification=1400

\def\ctln{\centerline}
\def\u{\underbar}
\def\ssk{\smallskip}
\def\msk{\medskip}
\def\bsk{\bigskip}


\ctln{\bf Math 445 Number Theory}

\medskip

\ctln{September 3, 2004}

\bigskip

{\it Miller-Rabin Test:} Given a number $N$, and a base $a$, compute $N-1 = 2^k\cdot d$ , with $d$ odd. Then compute

\ssk

\noindent{$a_0=a^d\pmod{N}$ , $a_1=a^{2d} = (a^d)^2\pmod{N}$ , $a_2 = (a_1)^2\pmod{N}$ , $\ldots$ , $a_k=a^{2^kd} = a_{k-1}^2\pmod{N}$}

\ssk

If $a_0=1$ or $a_i\equiv -1\pmod{N}$ for some $i\leq k-1$, then $N$ passes the test; it is either prime or a {\it strong
pseudoprime} to the base $a$. If not, then $N$ is definitely not prime.

\msk

Monier and Rabin in 1980 showed that a composite number $N$ is a strong pseudoprime for at most 1/4 of possible
bases $a$. So if $N$ passes this test for $m$ randomly chosen bases $a_1,\ldots ,a_m$, then $N$ has only a 1 in $4^m$
chance of {\it not} being prime. That is, multiple Miller-Rabin tests are very good at ferreting out non-primes.

\bsk

If this test tells us that a number $N$ is composite, how do we find its factors? The most straightforward 
approach; test divide all numbers less than $\sqrt{N}$, or better, all {\it primes} less than $\sqrt{N}$; 
eventually you will find a factor. But this requires on the order of $\sqrt{N}$ steps, which is far too
large. 

\msk

A different method uses the fact that if $N=ab$ and $a_1,\ldots a_n$ are chosen at random, $a$ is more 
likely to divide one of the $a_i$ (or rather (for later efficiency), one of the differences $a_i-a_j$), than $N$
is. This can be tested for by computing gcd's, $d=(a_i-a_j,N)$; this number is $1<d<N$ if $a$ (or some other
factor) divides $a_i-a_j$ but $N$ does not, and finds us a proper factor, $d$, of $N$. The probability that
$a$ divides none of the differences is approximately $1-1/a$ for each difference, and so is approximately

\ssk

\noindent{$\displaystyle (1-{{1}\over{a}})^{n\choose{2}} = ((1-{{1}\over{a}})^a)^{{n(n-1)}\over{2a}} 
\approx ((1-{{1}\over{a}})^a)^{{n^2}\over{2a}} \approx ((1-{{1}\over{a}})^a)^{{n^2}\over{2a}} \approx (e^{-1})^{{n^2}\over{2a}} = 
e^{{-n^2}\over{2a}}$}

\ssk

\noindent which is small when $n^2\approx a\leq\sqrt{N}$, i.e., $n\approx N^{1/4}$. The problem with this method,
however, is that it requires $n(n-1)/2 \approx \sqrt{N}$ calculations, and so is no better than trial division! We 
will rectify this by choosing the $a_i$ {\it pseudo-randomly} (which will also explain the use of differences). This
will lead us to the Pollard $\rho$ method for factoring.


\vfill\end







