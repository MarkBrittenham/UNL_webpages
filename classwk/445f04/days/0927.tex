
\input amstex


\magnification=1400


\loadmsbm

\nopagenumbers
\parindent=0pt

\voffset=-.6in
\hoffset=-.4in
\hsize = 7.5 true in
\vsize=10.5 true in

%\voffset=1.2in
%\hoffset=-.5in
%\hsize = 10.2 true in
%\vsize=8 true in

\overfullrule=0pt


\def\ctln{\centerline}
\def\u{\underbar}
\def\ssk{\smallskip}
\def\msk{\medskip}
\def\bsk{\bigskip}

\def\lra{$\Leftrightarrow$ }


\ctln{\bf Math 445 Number Theory}

\smallskip

\ctln{September 24, 2004}

\medskip

Last time we found the result {\it "If $p$ is an odd prime then $x^2\equiv -1\pmod{p}$ has a solution \lra $p\equiv 1\pmod{4}$"} useful.
Now we will explore such equations more generally.

\ssk

When does the equation $x^n\equiv a\pmod{m}$ have a solution?

\ssk

We will find it useful to first deal with the warm-up problem {\it When does $nx\equiv a\pmod{m}$ have a solution?} For this, we have
$nx\equiv a\pmod{m}$ \lra $m|nx-a$ \lra $a=nx-my$ for some $x,y$ \lra $(n,m)|a$. Further, if $nx_0\equiv a\pmod{n}$, then a complete 
set of incongruent solutions is given by (setting $k=(n,m)$)

\ssk

\ctln{$\displaystyle  x_0,x_0+{{m}\over{k}}, \ldots ,x_0+(k-1){{m}\over{k}}$ , \hskip.2in since $\displaystyle m|n{{m}\over{k}}=m{{n}\over{k}}$}

\ssk

So there are in fact $(n,m)$ solutions, if there are any.

\msk

Turning now to the main question, (*) $x^n\equiv a\pmod{m}$, we begin by supposing $m$ is prime, so that there is a primitive root $r$ mod $m$ , i.e., 
ord$_m(r)=m-1$ . Then either $m|a$ (so $a\equiv 0$ and $x=0$ solves (*)) or $(a,m)=1$ . In the latter case, $a=r^s$ for some $s$. Since $(a,m)=1$,
any possible solution to (*) must have $(x,m)=1$, as well, and so we can write $x=r^t$ for some $t$. So the equation that we {\it really} wish to solve is

\ssk

\ctln{(**) $(r^t)^n\equiv r^s\pmod{m}$ \hskip.2in (where we wish to solve for $t$).}

\ssk

But this means we wish to solve $(r^{nt-s}\equiv 1\pmod{m}$ , which, since ord$_m(r)=m-1$, means $m-1|nt-s$, i.e., 
$nt\equiv s\pmod{m-1}$ . But as we have just seen, this has a solution (and we know how many) \lra $(n,m-1)|s$ .
Translating this back into information about $a$, we find that $s=(n,m-1)q$ so $a=r^s=r^{(n,m-1)q}$, so, mod $m$,

\ssk

\ctln{$\displaystyle a^{{m-1}\over{(n,m-1)}} = (r^{(n,m-1)q})^{{m-1}\over{(n,m-1}} = r^{(m-1)q} = (r^{m-1})^q\equiv 1^q = 1$}

\ssk

Conversely, if $\displaystyle a^{{m-1}\over{(n,m-1)}}\equiv 1$, then $\displaystyle r^{s{{m-1}\over{(n,m-1)}}}\equiv 1$ .
Therefore ord$\displaystyle _m(b)=m-1|s{{m-1}\over{(n,m-1)}}$, so $\displaystyle (m-1){{s}\over{(n,m-1)}} = (m-1)y$ , so 
$\displaystyle {{s}\over{(n,m-1)}} = y$ is an integer. So $(n,m-1)|s$, which means (**) has a solution, and we can follow the
argument back up from there to see that (*) has a solution. So we find:

\msk

If $m$ is prime and $(a,m)=1$, then 

\hfill $x^n\equiv a\pmod{m}$ has $\displaystyle \cases (n,m-1) \text{ solutions,} &\text{if }a^{{m-1}\over{(n,m-1)}}\equiv 1\cr
                                                                                                     0 \text{ solutions, }&\text{if }\displaystyle a^{{m-1}\over{(n,m-1)}}\not\equiv 1\cr\endcases$

\ssk

Specializing to $n=2$, we have Euler's Criterion:

\ssk

If $m$ is  an odd prime and $(a,m)=1$, then 

\hfill$x^2\equiv a\pmod{m}$ has $\displaystyle \cases 2 \text{ solutions,} &\text{if }a^{{m-1}\over{2}}\equiv 1\cr
                                                                                                     0 \text{ solutions, }&\text{if }\displaystyle a^{{m-1}\over{2}}\equiv -1\cr\endcases$

\msk

So for example, by checking that $13^2=169\equiv -1\pmod{17}$, so $13^8\equiv 1\pmod{17}$, we find that
$x^2\equiv 13\pmod{17}$ has (two) solutions.

\vfill\end







