
\input amstex


\magnification=1400


\loadmsbm

\nopagenumbers
\parindent=0pt

\voffset=-.6in
\hoffset=-.4in
\hsize = 7.5 true in
\vsize=10 true in

%\voffset=1.2in
%\hoffset=-.5in
%\hsize = 10.2 true in
%\vsize=8 true in

\overfullrule=0pt


\def\ctln{\centerline}
\def\u{\underbar}
\def\ssk{\smallskip}
\def\msk{\medskip}
\def\bsk{\bigskip}


\ctln{\bf Math 445 Number Theory}

\medskip

\ctln{September 20, 2004}

\bigskip

Finishing our proof that for $n$ prime, there is an $a$ with ord$_n(a)=n-1$ : we introduce the notation 
$p^k||N$, which means that $p^k|N$ but $p^{k+1}\not |N$ .

\ssk

For each prime $p_i$ dividing $n-1$, $1\leq i\leq s$, we let $p_i^{k_i}||n-1$ . 
Then the equation 
(*) $x^{p_i^{k_i}}\equiv 1\pmod{n}$ has $p_i^{k_i}$ solutions, while
(\dag) $x^{p_i^{k_i-1}}\equiv 1\pmod{n}$ has only $p_i^{k_i-1}<p_i^{k_i}$ solutions; 
pick a solution, $a_i$ to (*) which is not a solution to (\dag) . 
[In particular, ord$_n(a_i)=p_i^{k_i}$.] 
Then set $a=a_1\cdots a_s$ . 
Then a computation yields that, mod $n$, 
$\displaystyle a^{{n-1}\over{p_i}} \equiv a_i^{{n-1}\over{p_i}}\not\equiv 1$, since otherwise
ord$\displaystyle _n(a_i)|{{n-1}\over{p_i}}$, and so 
ord$\displaystyle _n(a_i)|\gcd(p_i^{k_i},{{n-1}\over{p_i}})=p_i^{k_i-1}$ , a contradiction.
So $p_i^{k_i}||$ord$_n(a)$ for every $i$, so 
$n-1|$ord$_n(a)$, so ord$_n(a)=n-1$.

\msk

This result is fine for theoretical purposes (and we will use it many times), but it is somewhat less than satisfactory for computational
purposes; this process of {\it finding} such an $a$ would be very laborious.

\bsk

{\it Pythagorian triples:} If $a^2+b^2=c^2$, then we call $(a,b,c)$ a Pythagorean triple. Their connection to right triangles is
well-known, and so it is of interest to know what the triples are! It is fairly straighforward to generate a lot of them (e,g, via
$(n+1)^2=n^2+(2n+1)$, so any odd square $k^2=2n+1$ can be used to build one). But to find them all takes a bit more work:

\msk

A Pythagorean triple $(a,b,c)$ is {\it primitive} if the three numbers share no common factor. This is equivalent, in this case, to
$(a,b)=(a,c)=(b,c)=1$ . Then by considering the equation mod 4, we can see that for a primitive triple, $c$ must be odd,
$a$ (say) even and $b$ odd. If we then write the equation as $a^2=c^2-b^2=(c+b)(c-b)$, we find that we have factored
$a^2$ in two different ways. Since $a,b+c$ and $b-c$ are all even, we can write $(a/2)^2=[(c+b)/2]^2[(c-b)/2]^2$ But
because $(c+b)/2 +(c-b)/2 = c$ and  $(c+b)/2 -(c-b)/2 = b$, $\gcd((c+b)/2,(c-b)/2)=1$ . Then we can apply:

\msk

{\it Proposition:} If $(x,y)=1$ and $xy=c^2$, then $x=u^2,y=v^2$ for some integers $u,v$ .

\msk

This allows us to write $(c+b)/2=u^2$ and $(c-b)/2=v^2$ , so $c=u^2+v^2$ and $b=u^2-v^2$ . Also, $(a/2)^2=u^2v^2=(uv)^2$ , so 
$a=2uv$ . So we find that if
$a^2+b^2=c^2$ is a primitive Pythagorean triple (with the parity information above), then 

\ssk

\ctln{$a=2uv$ , $b=u^2-v^2$ , and $c=u^2+v^2$ for some integers $u,v$ .}

\ssk

Note that such a triple {\it is} a Pythagorean triple; these formulas therefore describe {\it all} primitive
Pythagorean triples.


\vfill\end







