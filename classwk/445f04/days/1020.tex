
\input amstex


\magnification=1300


\loadmsbm

\input colordvi

\nopagenumbers
\parindent=0pt

\def\cgy{\GreenYellow}     % GreenYellow  Approximate PANTONE 388
\def\cyy{\Yellow}	  % Yellow  Approximate PANTONE YELLOW
\def\cgo{\Goldenrod}	  % Goldenrod  Approximate PANTONE 109
\def\cda{\Dandelion}	  % Dandelion  Approximate PANTONE 123
\def\capr{\Apricot}	  % Apricot  Approximate PANTONE 1565
\def\cpe{\Peach}		  % Peach  Approximate PANTONE 164
\def\cme{\Melon}		  % Melon  Approximate PANTONE 177
\def\cyo{\YellowOrange}	  % YellowOrange  Approximate PANTONE 130
\def\coo{\Orange}	  % Orange  Approximate PANTONE ORANGE-021
\def\cbo{\BurntOrange}	  % BurntOrange  Approximate PANTONE 388
\def\cbs{\Bittersweet}	  % Bittersweet  Approximate PANTONE 167
%\def\creo{\RedOrange}	  % RedOrange  Approximate PANTONE 179
\def\cma{\Mahogany}	  % Mahogany  Approximate PANTONE 484
\def\cmr{\Maroon}	  % Maroon  Approximate PANTONE 201
\def\cbr{\BrickRed}	  % BrickRed  Approximate PANTONE 1805
\def\crr{\Red}		  % Red  VERY-Approx PANTONE RED
\def\cor{\OrangeRed}	  % OrangeRed  No PANTONE match
\def\paru{\RubineRed}	  % RubineRed  Approximate PANTONE RUBINE-RED
\def\cwi{\WildStrawberry}  % WildStrawberry  Approximate PANTONE 206
\def\csa{\Salmon}	  % Salmon  Approximate PANTONE 183
\def\ccp{\CarnationPink}	  % CarnationPink  Approximate PANTONE 218
\def\cmag{\Magenta}	  % Magenta  Approximate PANTONE PROCESS-MAGENTA
\def\cvr{\VioletRed}	  % VioletRed  Approximate PANTONE 219
\def\parh{\Rhodamine}	  % Rhodamine  Approximate PANTONE RHODAMINE-RED
\def\cmu{\Mulberry}	  % Mulberry  Approximate PANTONE 241
\def\parv{\RedViolet}	  % RedViolet  Approximate PANTONE 234
\def\cfu{\Fuchsia}	  % Fuchsia  Approximate PANTONE 248
\def\cla{\Lavender}	  % Lavender  Approximate PANTONE 223
\def\cth{\Thistle}	  % Thistle  Approximate PANTONE 245
\def\corc{\Orchid}	  % Orchid  Approximate PANTONE 252
\def\cdo{\DarkOrchid}	  % DarkOrchid  No PANTONE match
\def\cpu{\Purple}	  % Purple  Approximate PANTONE PURPLE
\def\cpl{\Plum}		  % Plum  VERY-Approx PANTONE 518
\def\cvi{\Violet}	  % Violet  Approximate PANTONE VIOLET
\def\parp{\RoyalPurple}	  % RoyalPurple  Approximate PANTONE 267
\def\cbv{\BlueViolet}	  % BlueViolet  Approximate PANTONE 2755
\def\cpe{\Periwinkle}	  % Periwinkle  Approximate PANTONE 2715
\def\ccb{\CadetBlue}	  % CadetBlue  Approximate PANTONE (534+535)/2
\def\cco{\CornflowerBlue}  % CornflowerBlue  Approximate PANTONE 292
\def\cmb{\MidnightBlue}	  % MidnightBlue  Approximate PANTONE 302
\def\cnb{\NavyBlue}	  % NavyBlue  Approximate PANTONE 293
\def\crb{\RoyalBlue}	  % RoyalBlue  No PANTONE match
%\def\cbb{\Blue}		  % Blue  Approximate PANTONE BLUE-072
\def\cce{\Cerulean}	  % Cerulean  Approximate PANTONE 3005
\def\ccy{\Cyan}		  % Cyan  Approximate PANTONE PROCESS-CYAN
\def\cpb{\ProcessBlue}	  % ProcessBlue  Approximate PANTONE PROCESS-BLUE
\def\csb{\SkyBlue}	  % SkyBlue  Approximate PANTONE 2985
\def\ctu{\Turquoise}	  % Turquoise  Approximate PANTONE (312+313)/2
\def\ctb{\TealBlue}	  % TealBlue  Approximate PANTONE 3145
\def\caq{\Aquamarine}	  % Aquamarine  Approximate PANTONE 3135
\def\cbg{\BlueGreen}	  % BlueGreen  Approximate PANTONE 320
\def\cem{\Emerald}	  % Emerald  No PANTONE match
\def\cjg{\JungleGreen}	  % JungleGreen  Approximate PANTONE 328
\def\csg{\SeaGreen}	  % SeaGreen  Approximate PANTONE 3268
\def\cgg{\Green}	  % Green  VERY-Approx PANTONE GREEN
\def\cfg{\ForestGreen}	  % ForestGreen  Approximate PANTONE 349
\def\cpg{\PineGreen}	  % PineGreen  Approximate PANTONE 323
\def\clg{\LimeGreen}	  % LimeGreen  No PANTONE match
\def\cyg{\YellowGreen}	  % YellowGreen  Approximate PANTONE 375
\def\cspg{\SpringGreen}	  % SpringGreen  Approximate PANTONE 381
\def\cog{\OliveGreen}	  % OliveGreen  Approximate PANTONE 582
\def\pars{\RawSienna}	  % RawSienna  Approximate PANTONE 154
\def\cse{\Sepia}		  % Sepia  Approximate PANTONE 161
\def\cbr{\Brown}		  % Brown  Approximate PANTONE 1615
\def\cta{\Tan}		  % Tan  No PANTONE match
\def\cgr{\Gray}		  % Gray  Approximate PANTONE COOL-GRAY-8
\def\cbl{\Black}		  % Black  Approximate PANTONE PROCESS-BLACK
\def\cwh{\White}		  % White  No PANTONE match


\voffset=-.6in
\hoffset=-.5in
\hsize = 7.5 true in
\vsize=10.6 true in

%\voffset=1.2in
%\hoffset=-.5in
%\hsize = 10.2 true in
%\vsize=8 true in

\overfullrule=0pt


\def\ctln{\centerline}
\def\u{\underbar}
\def\ssk{\smallskip}
\def\msk{\medskip}
\def\bsk{\bigskip}
\def\hsk{\hskip.1in}
\def\hhsk{\hskip.2in}

\def\lra{$\Leftrightarrow$ }


\ctln{\bf Math 445 Number Theory}

\smallskip

\ctln{October 20, 2004}

\medskip

For $Q$ odd and $(A,Q)=1$, if $Q=q_1\cdots q_k$ is the prime factorization of $Q$, then the {\it Jacobi symbol}
$\Big({{A}\over{Q}}\Big)$ is defined to be 
$\Big({{A}\over{Q}}\Big)$ = $\Big({{A}\over{q_1}}\Big)\cdots\Big({{A}\over{q_k}}\Big)$ .

\msk

The use of the same notation as for Legendre symbols should cause no confusion, and is in fact
deliberate; if $Q$ is prime, then both symbols are equal to one another. 
Straight from the definition, some basic properties:

\ssk

If $(A,Q) = 1 = (B,Q)$ then $\Big({{AB}\over{Q}}\Big) = \Big({{A}\over{Q}}\Big)\Big({{B}\over{Q}}\Big)$

\ssk

If $(A,Q) = 1 = (A,Q^\prime)$ then $\Big({{A}\over{QQ^\prime}}\Big) = \Big({{A}\over{Q}}\Big)\Big({{A}\over{Q^\prime}}\Big)$

\ssk

If $(PP^\prime,QQ^\prime) = 1$ then $\Big({{P^\prime P^2}\over{Q^\prime Q^2}}\Big) = \Big({{P^\prime}\over{Q^\prime}}\Big)$

\msk

{\bf Warning!} If $Q$ is not prime, then $\Big({{A}\over{Q}}\Big)=1$ does {\it not} mean that $x^2\equiv A\pmod{Q}$ has a solution.
For example, $\Big({{2}\over{9}}\Big) = (\Big({{2}\over{3}}\Big))^2 = 1$ , but $x^2\equiv 2\pmod{9}$ has no solution,
because $x^2\equiv 2\pmod{3}$ has none. But $\Big({{A}\over{Q}}\Big)=-1$ does mean that $x^2\equiv A\pmod{Q}$ has
{\it no} solution, because $\Big({{A}\over{Q}}\Big)=-1$ implies $\Big({{A}\over{q_i}}\Big)=-1$ for some prime factor of $Q$, so 
$x^2\equiv A\pmod{q_i}$ has no solution.

\msk

Some less basic properties:

\ssk

\crr{If $Q$ is odd, then $\Big({{-1}\over{Q}}\Big) = (-1)^{{Q-1}\over{2}}$} : \hskip.2in 
If $Q=q_1\cdots q_k$ is the prime factorization, then $\Big({{-1}\over{Q}}\Big) = \Big({{-1}\over{q_1}}\Big)\cdots \Big({{-1}\over{q_k}}\Big)
= (-1)^{{q_1-1}\over{2}}\cdots (-1)^{{q_k-1}\over{2}} = (-1)^{\sum_{i=1}^k{{q_i-1}\over{2}}}$, and this equals$(-1)^{{Q-1}\over{2}}$,
provided, mod 2, $\sum_{i=1}^k{{q_i-1}\over{2}}\equiv {{Q-1}\over{2}} = {{q_1\cdots q_k-1}\over{2}}$ . 
This in turn can be established by induction; the inductive step is

${{q_1\cdots q_kq_{k+1}-1}\over{2}} = 
(q_{k+1}-1){{q_1\cdots q_k-1}\over{2}} + {{q_1\cdots q_k-1}\over{2}} + {{q_{k+1}-1}\over{2}}
\equiv (q_{k+1}-1){{q_1\cdots q_k-1}\over{2}} +{{q_{k+1}-1}\over{2}}+ \sum_{i=1}^{k}{{q_i-1}\over{2}}
\equiv (q_{k+1}-1){{q_1\cdots q_k-1}\over{2}} + \sum_{i=1}^{k+1}{{q_i-1}\over{2}}
\equiv \sum_{i=1}^{k+1}{{q_i-1}\over{2}}$, since $Q$ is odd, so $q_{k+1}-1$ is even.

\ssk

\crr{If $Q$ is odd, then $\Big({{2}\over{Q}}\Big) = (-1)^{{Q^2-1}\over{8}}$} : \hskip.2in 
as before, $\Big({{2}\over{Q}}\Big) = \Big({{2}\over{q_1}}\Big)\cdots \Big({{2}\over{q_k}}\Big)$

$= (-1)^{{q_1^2-1}\over{8}}\cdots (-1)^{{q_k^2-1}\over{8}} = (-1)^{\sum_{i=1}^k{{q_i^2-1}\over{8}}}$
and this equals$(-1)^{{Q^2-1}\over{8}}$,
provided, mod 2, 

$\sum_{i=1}^k{{q_i^2-1}\over{8}}\equiv {{Q^2-1}\over{8}} = {{q_1^2\cdots q_k^2-1}\over{8}}$ ,
i.e., mod 16, $\sum_{i=1}^k(q_i^2-1)\equiv {{Q^2-1}\over{8}} = {{q_1^2\cdots q_k^2-1}\over{8}}$. 
This can also be established by induction; the inductive step is

$q_1^2\cdots q_{k+1}^2-1 = q_{k+1}^2q_1^2\cdots q_k^2-1 = (q_{k+1}^2-1)(q_1^2\cdots q_k^2-1) +(q_1^2\cdots q_k^2-1)+(q_{k+1}^2-1)
\equiv (q_{k+1}^2-1)+ (q_1^2\cdots q_k^2-1)\equiv (q_{k+1}^2-1)+ \sum_{i=1}^{k}(q_i^2-1) = \sum_{i=1}^{k+1}(q_i^2-1)$ , since
both $(q_{k+1}^2-1)$ and $(q_1^2\cdots q_k^2-1)$ are multiples of 8, so $(q_{k+1}^2-1)(q_1^2\cdots q_k^2-1)$ is divisible by 64, hence by 16.

\msk

Finally, \crr{if $P$ and $Q$ are both odd, and $(P,Q)=1$, then 
$\Big({{P}\over{Q}}\Big)\Big({{Q}\over{P}}\Big) = (-1)^{({{P-1}\over{2}})({{Q-1}\over{2}})}$} : \hskip.2in 
if $P=p_1\cdots p_r$ and $Q=q_1\cdots q_s$ are their prime factorizations, then 
$\Big({{P}\over{Q}}\Big)\Big({{Q}\over{P}}\Big) = \Big({{p_1\cdots p_r}\over{Q}}\Big)\Big({{Q}\over{p_1\cdots p_r}}\Big)$
= 
$\Big({{p_1\over{Q}}\Big)\cdots \Big({{p_r}\over{Q}}\Big) \Big({{Q}\over{p_1}}\Big)\cdots \Big({{Q}\over{p_r}}\Big)}$
= 

$[( \Big({{p_1}\over{q_1}}\Big)\cdots\Big({{p_1}\over{q_s}}\Big) )\cdots ( \Big({{p_r}\over{q_1}}\Big)\cdots  \Big({{p_r}\over{q_s}}\Big) )]
[( \Big({{q_1}\over{p_1}}\Big)\cdots\Big({{q_s}\over{p_1}}\Big) )\cdots ( \Big({{q_1}\over{p_r}}\Big)\cdots \Big({{q_s}\over{p_r}}\Big) )]$ 
= 

$\prod_{i,j}\Big({{p_i}\over{q_j}}\Big)\Big({{q_j}\over{p_i}}\Big)$
=
$\prod_{i,j} (-1)^{{{p_i-1}\over{2}}{{q_j-1}\over{2}}}
=(-1)^{\sum_{i,j}{{p_i-1}\over{2}}{{q_j-1}\over{2}}} = (-1)^{(\sum_{i=1}^r{{p_i-1}\over{2}})(\sum_{j=1}^s{{q_j-1}\over{2}})}$ .


This equals  $(-1)^{({{P-1}\over{2}})({{Q-1}\over{2}})}$, provided, mod 2, 
$(\sum_{i=1}^r{{p_i-1}\over{2}})(\sum_{j=1}^s{{q_j-1}\over{2}})\equiv ({{P-1}\over{2}})({{Q-1}\over{2}})$. But our first proof above 
established this, for each of the two parts, and so it is also true for their product!

\vfill\end







