
\input amstex


\magnification=1400


\loadmsbm

\nopagenumbers
\parindent=0pt

\voffset=-.6in
\hoffset=-.4in
\hsize = 7.5 true in
\vsize=10 true in

%\voffset=1.2in
%\hoffset=-.5in
%\hsize = 10.2 true in
%\vsize=8 true in

\overfullrule=0pt


\def\ctln{\centerline}
\def\u{\underbar}
\def\ssk{\smallskip}
\def\msk{\medskip}
\def\bsk{\bigskip}


\ctln{\bf Math 445 Number Theory}

\medskip

\ctln{September 17, 2004}

\bigskip

Fermat numbers $\displaystyle 2^{2^n}+1$ ; known prime only for $n=0,1,2,3,4$ . Part of the interest in them is

\ssk

{\it Fact (Gauss):} A regular $n$-gon can be constructed by compass and straight-edge $\Leftrightarrow$ $n= 2^kd$ where
$d$ is a product of distinct Fermat primes. 

\ssk

So the fact that we know of only 5 Fermat primes means we only know of 32 regular $n$-gons with an odd number of sides that
can be so constructed. If there is another one, it has more than a billion sides!

\bsk

Lucas' Theorem has a rather strong converse:

\ssk

{\it Theorem:} If $p$ is prime, then there is an $a$ with $(a,p)=1$ so that for every prime $q$ 
with $q|n-1$, $\displaystyle a^{{p-1}\over{q}}\not\equiv 1\pmod{p}$ .

\ssk

Note that $a^{p-1}\equiv 1\pmod{p}$ is always true, because $p$ is prime. In effect, what this theorem
says is that ord$_p(a) = p-1$ (which in the language of groups says that the group of units in ${\Bbb Z}_p$ is
cyclic, when $p$ is prime). In order to prove this theorem, 
we need a bit of machinery:

\msk

{\it Lagrange's Theorem:} If $f(x)$ is a polynomial with integer coefficients, of degree $n$, and $p$ is prime,
then the equation $f(x)\equiv 0 \pmod{p}$ has at most $n$ mutually incongruent solutions, unless $f(x)\equiv 0 \pmod{p}$
for \underbar{all} $x$.

\msk

To see this, do what you would do if you were proving this for real or complex roots; given a solution $a$, write
$f(x)=(x-a)g(x)+r$ with $r$=constant (where we understand this equation to have coefficients in ${\Bbb Z}_p$) 
using polynomial long division. This makes sense because ${\Bbb Z}_p$ is a {\it field}, so division by non-zero
elements works fine. Then $0=f(a)=(a-a)g(a)+r=r$ means $r=0$ in ${\Bbb Z}_p$, so $f(x)=(x-a)g(x)$ with $g(x)$
a polynomial with degree $n-1$ . Structuring this as an induction argument, we can assume that $g(x)$ has at most
$n-1$ roots, so $f$ has at most ($a$ and the roots of $g$, so) $n$ roots, because, 
{\it since $p$ is prime}, if $f(b)=(b-a)g(b)\equiv 0\pmod{p}$, then either $b-a\equiv 0$ 
(so $a$ and $b$ are congruent mod $p$), or $g(b)=0$, so $b$ is among the roots of $g$.

\msk

This in turn leads us to 

\ssk

{\it Corollary:} If $p$ is prime and $d|p-1$ , then the equation $x^d-1\equiv 0\pmod{p}$ has
{\it exactly} $d$ solutions mod $p$.

\msk

This is because, writing $p-1=ds$, $f(x)=x^{p-1}-1\equiv 0$ has exactly $p-1$ solutions (namely, 1 through $p-1$), and
$x^{p-1} = (x^d-1)(x^{d(s-1)}+x^{d(s-2)}+\cdots +x^d+1) = (x^d-1)g(x)$ . But $g(x)$ has {\it at most} $d(s-1)=(p-1)-d$
roots, and $x^d-1$ has at most $d$ roots, and together (since $p$ is prime) they make up the $p-1$ roots of $f$. So in
order to have enough, they both must have {\it exactly} that many roots.

\msk

This in turn will allow us to find our $a$ ....


\vfill\end







