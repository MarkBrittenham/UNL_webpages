
\input amstex


%\load amsbm

\nopagenumbers
\parindent=-20pt
\voffset=-.6in

\magnification=1400

\def\ctln{\centerline}
\def\u{\underbar}


\ctln{\bf Math 445 Number Theory}

\medskip

\ctln{September 1, 2004}

\bigskip

{\it Fermat's Little Theorem}: If $(a,n)=1$ and 
$a^{n-1}\not\equiv 1 \pmod{n}$, then $n$ is not prime.

\medskip

This is a very effective test, mostly because we can, in fact,
effectively compute $a^{n-1} \pmod{n}$, by successive squaring.
The idea: write $n-1$ as a sum of powers of 2, by repeatedly 
subtracting the highest power of 2 less than what remains
after doing prior subtractions. E.g.,

\smallskip

$78 = 64+14$ , $14=8+6$ , $6=4+2$ , so $78 = 2^6+2^3+2^2+2^1$ 

\smallskip

Then we can compute $a^{78}=a^{2^6}\cdot a^{2^3}\cdot a^{2^2}\cdot a^{2^1}$ , mod $79$, 
by first computing each factor (mod $79$), using $a^{2^k}=a^{2^{k-1}\cdot 2}=(a^{2^{k-1}})^2$
to proceed in stages. In this way we can compute $a^{n-1}$ , mod $n$ , with under $2\log_2(n)$
multiplications.

\medskip

But pseudoprimes exist; Carmichael numbers exist. (There are, in fact, infinitely many of them.)
We need a better test! Which we get from:

\medskip

Fact (Euler): If $p$ is prime and $a^2\equiv 1\pmod{p}$, 


\hfill then $a\equiv 1\pmod{p}$ or $a\equiv -1\pmod{p}$ .

\smallskip

Proof: $p|a^2-1 = (a-1)(a+1)$ ......

\medskip

This means that if we suspect that if $n$ is prime, we can test more thoroughly; 
set $n-1=2^k\cdot d$ with $d$ odd (by repeatedly dividing $n-1$ by 2 until what 
is left is odd). Then look, mod $n$ at

\medskip

\ctln{$a^d$ , $a^{2d}$ , $a^{2^2d}$ , $\ldots$ , $a^{2^kd} = a^{n-1}$}

\medskip

If $n$ is prime, the last number is $1$, and, by Euler, the number {\it just before} we first
start seeing $1$'s must be $-1$. If if {\it don't} see this pattern, then $n$ cannot be prime.

\medskip

This is the basis for our next test, the Miller-Rabin test.







\vfill\end







