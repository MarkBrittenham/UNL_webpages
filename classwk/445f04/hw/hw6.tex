\input amstex
\magnification=1200

\define\ctln{\centerline}
\define\ssk{\smallskip}
\define\msk{\medskip}
\define\bsk{\bigskip}

\overfullrule=0pt
\nopagenumbers


%(NZM, Problem ) 

\ctln{\bf Math 445 Homework 6}

\msk

\ctln{Due Wednesday, October 27}

\bsk

\item{26.} Show that if $p$ is an odd prime and $a$ is a primitive root
mod $p$, then $\displaystyle \Big({{a}\over{p}}\Big) = -1$ .

\bsk

\item{27.} [Pepin's Theorem] Show that the Fermat number
$\displaystyle F_n=2^{2^n}+1$ , for $n\geq 1$, is prime 
$\Leftrightarrow$ $\displaystyle 3^{{F_n-1}\over{2}}\equiv -1\pmod{F_n}$ .


\bsk

\item{28.}  The primes $p$ for which $x^2\equiv 13\pmod{p}$ has solutions 
consists precisely of those primes 
lying in certain congruence classes mod $13$ ; which ones?

\msk

\item{} [Hint: if you think of the classes as being represented by
$-6,\ldots,0,\ldots ,6$ then you can recycle a lot of your work....]

\bsk

\item{29.} [NZM, p. 148, \#\ 3.3.15] Show that if $p\geq 7$ 
is an odd prime, then 
$\displaystyle \Big({{n}\over{p}}\Big) = \Big({{n+1}\over{p}}\Big)$ for 
at least one of $n=2,4,5$, or 8.

\msk

\item{} [Hint: it might help to express this in terms of 
$\displaystyle \Big({{n}\over{p}}\Big)\Big({{n+1}\over{p}}\Big)$

\bsk

\item{30.} Compute $\displaystyle \Big({{35}\over{149}}\Big)$ , 
$\displaystyle \Big({{39}\over{145}}\Big)$ , and 
$\displaystyle \Big({{280}\over{351}}\Big)$ .



\vfill\end

