\input amstex
\magnification=1200

\define\ctln{\centerline}
\define\ssk{\smallskip}
\define\msk{\medskip}
\define\bsk{\bigskip}

\overfullrule=0pt
\nopagenumbers


%(NZM, Problem ) 

\ctln{\bf Math 445 Homework 5}

\msk

\ctln{Due Wednesday, October 6}

\bsk

\item{21.} Show that if an integer $n$ can be expressed as the sum of the squares of two {\it rational} numbers

\ssk

\ctln{$\displaystyle n = ({{a}\over{b}})^2 + ({{c}\over{d}})^2$ ,}

\ssk

\item{} then $n$ can be expressed as the sum of the squares of two {\it integers}.

\msk

\item{} (Hint: Not directly! Show that $n$ has the correct prime factorization....)

\bsk

\item{22.} [NZM, p. 106, \#\ 2.8.8] Determine how many solutions (mod 17) each of the following 
congruence equations has:

\msk

\hskip1in (a) $x^{12}\equiv 16$ (mod 17) \hskip1in (b) $x^{48}\equiv 9$ (mod 17)

\ssk


\hskip1in (c) $x^{20}\equiv 13$ (mod 17) \hskip1in (d) $x^{11}\equiv 9$ (mod 17)


\bsk

\item{23.}  If $p$ is a prime, and $p\equiv 3$ (mod 4), show that the congruence equation

\ssk

\item{} $x^4\equiv a$ (mod $p$) has a solution $\Leftrightarrow$ $x^2\equiv a$ (mod $p$) does.


\bsk

\item{24.} [NZM, p.107, \#\ 2.8.18] Show that if $a,b$ are both primitive roots of 1 modulo the {\bf odd}
prime $p$, then $ab$ is {\it not} a primitive root of 1 modulo $p$.

\msk

\item{} (Hint: there is a specific, smaller, number $k$ for which we can guarantee $(ab)^k\equiv 1$ (mod $p$) ...)



\bsk

\item{25.} [NZM, p.106, \#\ 2.8.2] Find a primitive root modulo 23.




\vfill\end



\bsk

\item{16.} If $p$ is a prime, and $p\equiv 3$ (mod 4), show that the congruence equation

\ssk

\item{} $x^4\equiv a$ (mod $p$) has a solution $\Leftrightarrow$ $x^2\equiv a$ (mod $p$) does.



[NZM p.60, \# 57]  