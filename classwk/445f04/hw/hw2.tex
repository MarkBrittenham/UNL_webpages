\input amstex
\magnification=1200

\define\ctln{\centerline}
\define\ssk{\smallskip}
\define\msk{\medskip}
\define\bsk{\bigskip}

\overfullrule=0pt
\nopagenumbers


%(NZM, Problem ) 

\ctln{\bf Math 445 Homework 2}

\msk

\ctln{Due Wednesday, September 15}

\bsk

\item{6.} [The ``only if'' part of the characterization of Charmichael numbers.]

\ssk

\item{} Show that if $N=p_1\cdots p_k$ is a product of distinct primes and
 $(p_i-1)|(N-1)$, for every $i$, then $N$ is a pseudoprime to every base $a$ 
satisfying $(a,N)=1$. 

\ssk

\item{} (Hint: Show that a number $\equiv 1\pmod{p_i}$ for every $i$ is $\equiv 1\pmod{N}$ .)

\bsk

\item{7.} Show that if $n=pq$ with $p<q$ and $p,q$ both prime,
then it is not possible for $q-1$ to divide $n-1$. 
(Consequently, Carmichael numbers must have at least three prime factors...)

\ssk

\item{} (Hint: If it did, then show that the other factor would have to be too big....)

\bsk

\item{8.} Show that 2465, 2821, and 6601 are Carmichael numbers.

\bsk

\item{9.} (NZM, Problem 2.4.9) [For a pseudoprime, failing the 
Miller-Rabin test \underbar{finds} factors.]

\ssk

\item{} Show that if $x^2\equiv 1$ (mod $n$) and $x \not\equiv \pm 1$ (mod $n$), then
$1<(x-1,n)<n$ and $1<(x+1,n)<n$ .

\bsk

\item{10.} Show that $n$ = 3277 = 29$\times$113 is a strong pseudoprime to the base 2.

\item{} [Do the calculations by hand....]

\vfill\end



\bsk

\item{10.}
