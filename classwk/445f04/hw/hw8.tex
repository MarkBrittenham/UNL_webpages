\input amstex
\magnification=1200

\define\ctln{\centerline}
\define\ssk{\smallskip}
\define\msk{\medskip}
\define\bsk{\bigskip}
\define\dsl{\displaystyle}

\overfullrule=0pt
\nopagenumbers

\documentstyle{amsppt}


%(NZM, Problem ) 

\ctln{\bf Math 445 Homework 8}

\msk

\ctln{Due Wednesday, November 10}

\bsk

\item{36.} Let $h_n/k_n$ (as usual) denote the $n^{\roman th}$ convergent of the continued
fraction expansion of the irrational number $x$. Show by example that it need {\bf not} be true
that

\ssk

\ctln{$\displaystyle  \Bigl| x-{{a}\over{b}}\Bigr| < \Bigl| x-{{h_n}\over{k_n}}\Bigr|$ 
implies $b\geq k_{n+1}$}

\bsk



\item{37.} [NZM, p.344, \# 7.6.3] Show that for any $c>2$, there are only finitely
many pairs of integers $a,b$ with $\dsl |\sqrt{2}-{{a}\over{b}}| < {{1}\over{b^c}}$ .

\bsk

\item{38.} [NZM, p. 333, \# 7.3.6] Let $p$ be prime and suppose
$u^2\equiv -1\pmod{p}$ (so $p\equiv 1\pmod{4}$). 
Let $[a_0,\ldots ,a_n]$ be the continued fraction expansion of 
$\dsl {{u}\over{p}}$ , and let $i$ be the largest integer with $k_i\leq \sqrt{p}$ . Show that
$\dsl |{{h_i}\over{k_i}}-{{u}\over{p}}| < {{1}\over{k_i\sqrt{p}}}$, and $|h_ip-k_iu|<\sqrt{p}$ .
Setting $x=k_i$ and $y=h_ip-uk_i$, show that $p|x^2+y^2$ and $x^2+y^2<2p$, so 
$x^2+y^2=p$ .

\bsk

\item{39.} Show that for $n$ a positive integer that is not a perfect square (translation: 
the continued fraction expansion of $\sqrt{n}$ never terminates), that at every stage 
of the continued fraction expansion of $x = \sqrt{n}$

\ssk

\ctln{$x$ = $[a_0,a_1,\ldots ,a_{k-1},a_k+x_k]$}

\ssk

\item{} $x_k$ is always of the form $\displaystyle x_k = {{\sqrt{n}-a}\over{b}}$ , where 
$a,b\in{\Bbb N}$ and $b|n-a^2$ . Conclude that the continued fraction expansion of $\sqrt{n}$ will 
eventually repeat, with a period of length at most $n\lfloor \sqrt{n}\rfloor$.

\ssk

\item{} Hint: by induction! In the inductive step, write 
$\displaystyle {{b}\over{\sqrt{n}-a}} = {{\sqrt{n}+a}\over{c}}$, and then 
find the fractional part of this. For the second half, how long must you wait 
before the $x_k$'s {\it must} repeat themselves?

\vfill\end






Show that if $a,b\in{\Bbb Z}$, $b\geq 1$, and $(a,b)=1$, then the continued fraction expansion of 
$a/b$ has length at most $b$ . 

\ssk

\item{} Hint: this is really a question about the Euclidean algorithm....








\bsk

\item{10.}
