%\baselineskip=18pt plus 2pt

\magnification=1200

\parindent=50pt

\def\ni{\noindent}
\def\ctln{\centerline}
\def\msk{\medskip}
\def\ssk{\smallskip}
\def\bsk{\bigskip}

\def\iit{\itemitem}
\def\htp{\hskip10pt}
\def\vtp{\vskip.02in}
\def\hsk{\hskip.2in}
%\input shorthand

\nopagenumbers

\ctln{\bf Math 445 (845) Introduction to the Theory of Numbers}

%\vtp

\ctln{\bf Section 001}

\smallskip

\ni{\bf Lecture:} MWF 2:30 - 3:20 \htp Avery Hall (AvH) 118

\msk

\ni{\bf Instructor:} Mark Brittenham

%\smallskip

\ni{\bf Office:} Avery Hall (AvH) 317
%\smallskip

\ni{\bf Telephone:} (47)2-7222

%\ssk

\ni{\bf E-mail:} mbritten@math.unl.edu

\ni{\bf WWW:} http://www.math.unl.edu/$\sim$mbritten/

\ni{\bf WWW pages for this class:} http://www.math.unl.edu/$\sim$mbritten/classwk/445f04/

\ssk

\ni(There you will find copies of nearly every handout from class, lists of homework 
problems assigned, dates for exams, etc.)

\smallskip

\ni{\bf Office Hours:} (tentatively) Mo 1:00-2:00, We 11:00-12:00, 
Th 11:00 - 12:00, and Fr 1:00-2:00, and whenever you can find me in my office and I'm not 
horrendously busy. You are also quite welcome to make an appointment
for any other time; this is easiest to arrange just before or 
after class, or via email.

\ssk

\ni{\bf Text:} {\it An Introduction to the Theory of Numbers}, 
by I. Niven, H. Zuckerman, and H. Montgomery (5th edition, John Wiley and Sons).

\msk

\ni This course, as its name is meant to imply, is intended to introduce you to 
the theory of numbers, that is, the theory of the integers and their properties. 
The topics we will cover will be determined partly by the interests of those attending;
likely topics include primality testing, quadratic reciprocity, arithmetic functions, 
continued fractions and/or Diophantine equations.

\msk

\ni{\bf Homework} will be assigned approximately weekly, and collected
one week after it is assigned. 
It is an essential ingredient to the course - as with almost all of 
mathematics, we learn best by doing (again and again and ...). Cooperation 
with other students on these assignments is acceptable, and even 
encouraged. However, you must write up solutions on your own - after 
all, you get to bring only one brain to exams (and it can't be someone 
else's). For the same reason, I also recommend that you try working 
each problem on your own, first. The homework grades will 
count 40\% toward your final grade.
Late homework may be marked as turned in but not graded.

\ssk

\ni{\bf Midterm exams} will be given two times during the 
semester - the specific dates will be announced in class 
well in advance of each exam. At least one of them will 
be a take-home exam. Each exam will count 15\% 
toward your grade. You can take a 
make-up exam only if there are compelling reasons (a doctor SAYS 
you were sick, jury duty, etc.) for you to miss an exam. Make-up 
exams tend to be harder than the originals (because make-up exams 
are harder to write!). 


\ssk

\ni Finally, there will be a regularly scheduled {\bf final exam} on 
Monday, December 13, from 1:00pm to 3:00pm.
It will cover the entire course, with a slight emphasis 
on material covered after the last midterm exam. It will count the 
remaining 30\% toward your grade.

\msk

\ni {\bf Your course grade} will be calculated numerically using the above scales,
and will be converted to a letter grade based partly on the overall average of the
class. However, a score of 90\% or better will guarantee some kind of {\bf A}, 80\%
or better at least some sort of {\bf B}, 70\% or better at least a flavor of 
{\bf C}, and 60\% or 
better at least a {\bf D}.

\bsk 

\ni\hskip.2in In mathematics, new concepts continually rely upon the mastery
of old ones; it is therefore essential that you thoroughly understand each 
new topic before moving on. Our classes are an important opportunity for you to ask
questions; to make \underbar{sure} that you are understanding concepts correctly.
Speak up! It's \underbar{your} education at stake. Make every effort to resist
the temptation to put off work, and to fall behind. Every topic has to be gotten 
through, not around. And it's a lot easier to read 50 pages in a week than it is
in a day. Try to do some work on your  mathematics class(es) every day. You'd
be amazed at what the back of your brain can do with a problem, if you give it 
enough time!
{\bf Class attendance} is probably your best way to insure that you will keep 
up with the material, and make sure that you understand all of the
concepts.

\msk

\ni{\bf Departmental Grading Appeals Policy:} Students who believe their
academic evaluation has been prejudiced or capricious have recourse for appeals 
to (in order) the instructor, the departmental chair, the departmental appeals 
committee, and 
the college appeals committee.

\msk

\ctln{\bf Some important academic dates}

\ssk

{\bf Aug. 23} First day of classes.

{\bf Sept. 6} Labor Day - no classes.

{\bf Sept. 3} Last day to withdraw from a course without a {\bf `W'}.

{\bf Oct. 15} Last day to change to or from P/NP.

{\bf Oct. 18-19} Fall break - no classes.

{\bf Nov. 12} Last day to withdraw from a course.

{\bf Nov. 24} Student holiday - no classes.

{\bf Nov. 25-28} Thanksgiving Vacation - no classes.

{\bf Dec. 11} Last day of classes.

{\bf Dec. 13-18} Final exam week.

\vfill

\end

\vfill\eject
