



\documentclass[12pt]{article}
\usepackage{amsfonts}
\usepackage{amssymb}
\usepackage{dutchcal}

\textheight=8.9in
\textwidth=6.5in
\voffset=-1in
\hoffset=-.5in
\parindent=-10pt

\begin{document}

\def\ctln{\centerline}
\def\msk{\medskip}
\def\bsk{\bigskip}
\def\ssk{\smallskip}
\def\ra{\rightarrow}
\def\ubr{\underbar}
\def\sset{\subseteq}
\def\hsk{\hskip10pt}

\def\mt{{\mathcal T}}
\def\mb{{\mathcal B}}
\def\mbp{{\mathcal B}^\prime}
\def\mbpp{{\mathcal B}^{\prime\prime}}
\def\ms{{\mathcal S}}
\def\mu{{\mathcal U}}
\def\mv{{\mathcal V}}
\def\mp{{\mathcal P}}
\def\mtp{{\mathcal T}^\prime}
\def\mtpp{{\mathcal T}^{\prime\prime}}

\def\bbr{{\mathbb R}}
\def\bbz{{\mathbb Z}}
\def\bbq{{\mathbb Q}}
\def\bbn{{\mathbb N}}
\def\spc{$~$\hskip.15in$~$}
\def\bbd{{\mathbb D}}
\def\bbc{{\mathbb C}}

\def\up{U^\prime}
\def\upp{U^{\prime\prime}}
\def\vp{V^\prime}
\def\vpp{V^{\prime\prime}}
\def\wp{W^\prime}
\def\wpp{W^{\prime\prime}}

\def\finv{f^{-1}}
\def\ginv{g^{-1}}
\def\hinv{h^{-1}}

\def\sset{\subseteq}
\def\lra{$\Leftrightarrow$}
\def\smin{\setminus}
\def\rta{$\Rightarrow$}
\def\rtta{\Rightarrow}

\def\ep{\epsilon}
\def\pu{\pi_1}
\def\mpu{$\pi_1$}
\def\del{\partial}
\def\sig{\Sigma}
\def\msig{$\Sigma$}
\def\wtl{\widetilde}


%%\UseAMSsymbols

\ctln{\bf Math 872 Exam 1 Topics}

\msk

{\bf Homotopy Theory.}

\ssk

Motivation: understand \ubr{all} continuous functions $f:X\ra Y$,
since it is functions to/from `model' spaces that allow us to 
explore a space. 

\ssk

E.g., paths = $\gamma:I=[0,1]\ra X$. How many `essentially distinct'
paths are there from $(-1,0)$ to $(1,0)$ in $\bbr^2\smin\{(0,0)\}$ ?
What is inessetial? \ubr{Deformations}.

\ssk

Two maps $f,g:X\ra Y$ are {\it homotopic} if one can be deformed 
to the other (through continuous maps). Formally, there is a
cts map $H:X\times I\ra Y$ so that $H(x,0)=f(x)$ and $H(x,1)=g(x)$
for all $x\in X$. We write: $f\simeq g$ (via $H$).

\ssk

Note: $\gamma_x:t\mapsto H(x,t)$ is a cts path in $Y$, for every $x$.

\ssk

Notation: $f:(X,A)\ra (Y,B)$ means $A\sset X$, $B\sset Y$ and $f(A)\sset B$.

\ssk

Two maps $f,g:(X,A)\ra (Y,B)$ are homotopic rel $A$ if 
$H:X\times I\ra Y$ also satisfies $H(a,t)=f(a)=g(a)$ for all $a\in A$, $t\in I$.
[So, in part, $f|_A=g|_A$ .]

\ssk

Basic example: any two maps $f,g:X\ra \bbr^n$ are homotopic, via a
{\it straight-line homotopy}: $H(x,t)=(1-t)f(x)+tg(x)$.

\msk

Homotopy is an \ubr{equivalence} \ubr{relation}: $f\simeq f$  (via $H(x,t)=f(x)$),
$f\simeq g$ implies $g\simeq f$ (via $K(x,t)=H(x,1-t)$); 
$f\simeq g$ and $g\simeq h$ implies $f\simeq h$ (via doubling the speed;
$M(x,t)=H(x,2t)$ for $t\leq 1/2$ and $=K(x,2t-1)$ for $t\geq 1/2$).

\ssk

This allows us to introduce a new notion of equivalence of topological spaces.
$X$ and $Y$ are {\it homotopy equivalent} [we write $X\simeq Y$]
if there are $f:X\ra Y$ and $g:Y\ra X$ so that $g\circ f\simeq {\textrm Id}_X$
and $f\circ g\simeq{\textrm Id}_Y$ . 

\ssk

Homotopy equivalence is an equivalence relation! Note: a homeomorphism is a
homotopy equivalence! [$g\circ f = {\textrm Id}_X\simeq {\textrm Id}_X$].

\msk

{\bf The homotopy viewpoint.}

\ssk

The basic idea is that homotopy equivalence  (= `h.e.') 
allows us to move past/around
`unimportant' differences in spaces. For example, 
$\bbr^2\smin\{(0,0)\}\cong S^1\times\bbr\simeq S^1\times I\simeq S^1$
means that maps into $\bbr^2\smin\{(0,0)\}$ `behave like' maps
into $S^1$ (which we can more readily understand?).

\ssk

{\it Algebraic topology} seeks to understand topological spaces through
algebraic invariants. An algebraic invariant assigns to each space $X$ an 
algebraic object $A(X)$ and to each map $f:X\ra Y$ a 
homomorphism $A(f):A(X)\ra A(Y)$. If $X$ and $Y$ are the `same', then 
$A(X)$ and $A(X)$ will be isomorphic. Usually, `same' means homeomorphic,
but we will often find that homotopy equivalent spaces will 
same the same invariants, due the the methods that we use to build 
them.

\ssk

This can be both bad and good, `homotopy invariance' of a invariant means
that it will not be able to distinguish h.e. spaces that are not homeomorphic.
But it also means that when computing an algebraic invariant, we can replace
a space $X$ with $Y\simeq Y$, which may streamline a computation.

\msk

A {\it retraction} of $X$ onto $A\sset X$ is a map $r:X\ra A$ so that $r(a)=a$
for all $a\in A$. [$A$ is a {\it retract} of $X$]. $A$ is a {\it deformation
retract} of $X$ if $\iota\circ r:X\ra A\ra X$ is $\simeq{\textrm Id}_X$
[$r$ is a {\it deformation retraction}]. and $r$ is a {\it strong
deformation retraction} if $\iota\circ r:(X,A)\ra (X,A)$ is $\simeq{\textrm Id}_X$
rel $A$ (i.e., $H(a,t)=a$ for all $a\in A$). We write $X\searrow A$. 

\ssk

For example, $r:\bbr^n\searrow\{\vec{0}\}$, since $\iota\circ r\simeq{\textrm Id}_{\bbr^n}$
via a straight-line homotopy 

\hfill $H(x,t)=(1-t)\iota\circ r(\vec{x})+t{\textrm Id}_{\bbr^n}(\vec{x})=t\vec{x}$ .

\ssk

A space $X$ is {\it contractible} if $X\simeq\{*\}$.

\ssk

Mapping cylinders:  If $f:X\ra Y$, then $M_f = X\times I\coprod Y/\sim$, where
$(x,1)\sim f(x)$. [Idea: we glue $X\times\{1\}$ to $Y$ using $f$.] Then since 
$X\times I\searrow X\times\{1\}$, we have $M_f\searrow Y$.

\ssk

Fact: $f:X\ra Y$ is a homotopy equivalence \lra\ $M_f\searrow X\times\{0\}$.
This means that $X\simeq Y$ \lra\ there is a space $Z$ with $X,Y\sset Z$ and 
$Z\searrow X$, $Z\searrow Y$.

\msk

{\bf The Fundamental Group.}

\ssk

Idea: find the essentially distinct paths between points in $X$.
How? Turn this into a group! How? The concatenation $\gamma*\eta$ of two paths is
a path. But: only if the first ends where the second begins (so that, by
the Pasting Lemma, the resulting map is cts). So we
either have a \ubr{partial} multiplication (= groupoid!), or we
focus on \ubr{loops} $\gamma:(I,\partial I)\ra (X,x_0)$ based at a
fixed point $x_0$ 9we'll do the second).

\ssk

Elements of the {\it fundamental group} $\pi_1(X,x_0)$ `are' 
loops; the inverse will be the reverse $\overline{\gamma}(t)=\gamma(1-t)$,
since $\gamma*\overline{\gamma}\simeq c_{x_0}$, and the 
identity element will be the constant map $c_{x_0}$.
But! to make $\gamma*\overline{\gamma}$ equal $c_{x_0}$, we need to work with
{\it homotopy classes} of loops. So elements really are 
equivalence classes $[\gamma]$ of loops, under $\simeq$ rel $\partial I$.

\ssk

Then by building homotopies (mostly working on the domain $I$,
i.e., building $K=\gamma\circ H:I\times I\ra I\ra X$)
we can see that $[\gamma][\eta] = [\gamma*\eta]$ is well defined,
$[\gamma]^{-1}=[\overline{\gamma}]$ is the inverse, and 
$([\gamma][\eta])[\omega] = [\gamma]([\eta][\omega])$, so 
under $*$, $\pi_1(X,x_0)$ is a group. [Most of the proofs that
needed maps (like $(\gamma*\eta)*\omega$ and $\gamma*(\eta*\omega)$
(which are the same concatenations, except at 4,4, and 2 times speed, versus 
2,4, and 4 times speed) are homotopic can be given `picture' proofs, 
in addition to explicit analytic formulas.

\ssk

Given a map $f:(X,x_0)\ra (Y,y_0)$, we get an induced map
$f_*:\pi_1(X,x_0)\ra \pi_1(Y,y_0)$ via $f_*[\gamma]=[f\circ\gamma]$.
This is well-defined, and a homomorphism.

\ssk

Basic computations: $\pi_1(\{*\},*) = \{1\}$, as are 
$\pi_1(\bbr^n,\vec{0})$ and $\pi_1([0,1]^n,x_0)$ for any
$x_0$. More generally, any contractible space has trivial fundamental
group.

\msk

Since $(f\circ g)_*=f_*\circ g_*$, and $({\textrm Id}_X)_*={\textrm Id}_{\pi_1(X,x_0)}$,
then $X\cong Y$ via $f$ implies $f_*:\pi_1(X,x_0)\ra \pi_1(Y,f(x_0))$
is an isomorphism.

\ssk

More generally, if $f:X\ra Y$ is a h.e., then $f_*$ is an isomorphism,
but, because of basepoint issues, the inverse of $f_*$ is generally
\ubr{not} $g_*$ for $g$ a homotopy inverse. Th is is because under
a homotopy $H:X\times I\ra X$ of $g\circ f$ to ${\textrm Id}$,
the basepoint $x_0$ traces out a path $\eta$ from $g(f(x_0))=x_1$ to $x_0$,
and $[g\circ f\circ\gamma] = [\overline{\eta}*\gamma *\eta]$.
This map $[\gamma]\mapsto [\overline{\eta}*\gamma *\eta]$ from
$\pi_1(X,x_0)$ to $\pi_1(X,x_1)$ is a {\it change of basepoint 
isomorphism}, which we might call $\eta_*$ ? The fact that homotopies
can drag basepoints around will be a theme we will return to 
many times moving forward.

\ssk

If $X$ is path connected, then, up to isomorphism, $\pi_1(X,x_0)$ 
is independent of $x_0$ (we can always find a path to effect an 
isomorphism), and so we will often write $pi_1(X)$, when $X$ is 
path-connected, when we only care about the abstract group.

\msk

$\pi_1(S^1,(1,0))\cong\bbz$. The main ingredients:

\ssk

Writing $S^1\sset\bbc$ and $\gamma_n(t)=e^{2\pi int}$ is the loop
traversing $S^1$ $n$ times counterclockwise at uniform speed,
then (1) every loop $\gamma$ at $(1,0)$ is $\simeq\gamma_n$ for some $n$.

\ssk

We define $w:\pi_1(S^1,(1,0))\ra\bbz$ by $w[\gamma] = n$ if
$[\gamma]=[\gamma_n]$. This is well-defined:
(2) if $\gamma_n\simeq\gamma_m$ rel endpoints, then $n=m$.

\ssk

$w$ is a bijective homomorphism!

\ssk

The proof of (1) amounted to making a general $\gamma$ progressively nicer, 
via homotopy. This involved 

\ssk

{\it Lebesgue Number Theorem}: If $(X,d)$ is a compact metric space and
$\{U_\alpha\}$ is an open covering of $X$, then there is an $\epsilon>0$
so that for every $x\in X$ there is an $\alpha=\alpha(x)$ so that 
we have $N_d(x,\epsilon)\sset U_\alpha$.

\ssk

Then by covering $S^1$ by the `top 2/3rds' and `bottom 2/3rds' subsets
and taking inverse images under $\gamma:(I,\partial I)\ra (S^1,(1,0))$,
the LNT will partition $I$ into finitely many intervals each mapping into
top or bottom. Creating subpaths by restricting to each subinterval, and inserting
`hairs' to points $(1,0),(-1,0)$ in the intersection of top and bottom,
we can then homotope the subpaths to standard paths $t\mapsto e^{\pm 2\pi it}$.
Cancelling pairs the reverse direction give us our `normal forms'
$\gamma_n$.

\ssk

The proof of (2) amounted to using an `extra' coordinate $(\cos t,\sin t,t)$
to keep track of how many times we wind around the circle. To do this correctly,
we really use the map $p:t\mapsto(\cos t,\sin t,t)\mapsto(\cos t,\sin t)$ and
then \ubr{lift} paths $\gamma:I\ra S^1$ to paths 
$\widetilde{\gamma}:I\ra \bbr$ with $\gamma = p\circ\widetilde{\gamma}$. 
This agin uses the LNT to partition $I$ into subintervals mapping into top and bottom,
and the fact that the inverse image of top and bottom are a disjoint
union of open sets mapped homeomorphically under $p$ to the top and bottom.
[This is the {\it evenly covered property}.]

\ssk

More than this, homotopies $H:I\times I\ra S^1$ can also be lifted; this enables us to
show that loops homotopic rel endpoints, when lifted both starting at the same point,
will end at the same point. Since $\gamma_n$ when lifted starting at $0$ will
end at $n$, the result follows.

\msk

{\bf Applications.}
This single computation has many applications! First, there is no
retraction $r:\bbd^2\ra \partial\bbd^2$. This is because if there were one,
then $r_*:\pi_1(\bbd^2,(1,0))\ra\pi_1(S^1,(1,0))$ would be a surjection,
which is impossible.

\ssk

This in turn gives the {\it Brouwer Fixed Point Theorem}: Every countinuous
map $f:\bbd^2\ra \bbd^2$ has a fixed point. For if not, we can then 
manufacture a retraction $r:\bbd^2\ra \partial\bbd^2$.

\ssk

Finally, we can prove the {\it Fundamental Theorem of Algebra}: Every non-constant
polynomial $p$ has a complex root. For if not, then for large enough $N$ the map

\ctln{$t\mapsto f(Ne^{2\pi it})\mapsto f(Ne^{2\pi it})/||f(Ne^{2\pi it})||$}

from 
$I$ to $S^1$ is homotopic to both $c_{(1,0)}=\gamma_0$ and $\gamma_n$ for $n$ = the
degree of $f$, a contradiction.

\bsk

{\bf Group presentations.}

{\it Free groups:} $\Sigma$ = a set; a {\it reduced word} on \msig\ is a (formal)
product $a_1^{\ep_1}\cdots a_n^{\ep_n}$ with $a_i\in\sig$ and $\ep_i=\pm 1$,
and either $a_i\neq a_{i+1}$ or $\ep_i\neq -\ep_{i+1}$ for every $i$. (I.e., no
$aa^{-1},a^{-1}a$ in the product.)

\ssk

The free group $F(\sig)$ = the set of reduced words, with multiplication = concatenation 
followed by reduction; remove all possible $aa^{-1},a^{-1}a$ from the site of concatenation.
Identity element = the empty word, 
$(a_1^{\ep_1}\cdots a_n^{\ep_n})^{-1} = a_n^{-\ep_n}\cdots a_1^{-\ep_1}$. 
$F(\sig)$ is generated by \msig, with no relations among the generators
other than the ``obvious'' ones.

\ssk

Important property of free groups: any function $f:\sig\ra G$ , $G$ a group, extends
uniquely to a homomorphism $\phi: F(\sig)\ra G$.

\ssk

If $R\sset F(\sig)$, then $<R>^N$ = normal subgroup generated by $R$ 
= $\displaystyle \{\prod_{i=1}^n g_i r_i g_i^{-1} : n\in\bbn_0 , g_i\in F(\sig) , r_i\in R\}$
= smallest normal subgroup containing $R$.

$F(\sig)/<R>^N$ = the group with {\it presentation} $<\sig | R>$ .
Any homom $\varphi:F(\sig)\ra G$ with $F(R)=\{1_G\}$ induces 
$\overline{\varphi}:<\sig | R>\ra G$ .

If $G_1=<\sig_1 | R_1>$ and $G_2=<\sig_2 | R_2>$, then their {\it free product}
$G_1*G_2 = <\sig_1\coprod\sig_2 | R_1\cup R_2>$ .
Any pair of
homoms $\phi_i:G_i\ra G$ extends uniquely to a homom $\phi:G_1*G_2\ra G$

\ssk

Gluing groups: given groups $G_1=<\sig_1 | R_1>$ and 
$G_2=<\sig_2 | R_2>$, a group $H$ and homomorphisms $\phi_1 : H\ra G_i$,
the largest group ``generated'' by
$G_1$ and $G_2$, in which $\phi_1(h)=\phi_2(h)$ for all $h\in H$ is
$G_1*_HG_2 = <\sig_1\coprod\sig_2 | R_1\cup R_2\cup\{\phi_1(h)(\phi_2(h))^{-1} : h\in H\}>$ .

Important special cases : $G*_H\{1\} = G/<\phi(H)>^N = <\sig | R\cup \phi(H)>$ , and
$G_1*_\{1\}G_2 \cong G_1*G_2$ .

\ssk

{\bf Seifert-van Kampen Theorem.}
If we express a topological space as the union $X=X_1\cup X_2$, then we have 
inclusion-induced homomorphisms 

\ctln{$j_{1*}: \pu(X_1)\ra \pu(X)$ , $j_{2*}: \pu(X_2)\ra \pu(X)$}

This in turn gives a homomorphism $\phi:\pi(X_1)*\pu(X_2)\ra \pu(X)$ . Under the
hypotheses
\hskip.2in $X_1,X_2$ are open, and $X_1,X_2,X_1\cap X_2$ are path-connected
\hskip.2in (choose a basepoint in $X_1\cap X_2$ and)
this homom is onto, and has kernel 
$H=<i_{1*}(\gamma)(i_{2*}(\gamma))^{-1} : \gamma\in\pu(A) >^N$, so we have
the {\it Seifert - van Kampen Theorem:}  $\pu(X)\cong \pu(X_1)*_{\pu(A)}\pu(X_2)$ . 

[Why? Lebesgue number theorem! Any loop into $X$, using the inverse images of 
$X_1,X_2$ as an open cover, can be partitioned into subloops alternately mapping into
$X_1,X_2$, which makes $\phi$ surjective. Partitioning a null homotopy, using $H$ to 
change which of $\pi(X_1),\pu(X_2)$ a loop lies in, yields the result.]

Generalization (of sorts): if $X=C\cup D$ closed sets, with $C,D$ having nbhds $U,V$ which 
deformation retract to $C,D$ (and $U\cap V$ def retracts to $C\cap D=A$, then 
$\pu(X)\cong \pu(C)*_{\pu(A)}\pu(D)$.

\ssk

{\bf Applications.} 

{\it Fundamental groups of graphs:} Choosing a maximal tree $T$ in a graph $\Gamma$,
$\Gamma \simeq \Gamma/T = $a bouquet of circles, which by SvK has fundamental gorup free on the number
of loops.

{\it Gluing on a 2-disk:} If $X$ is a topological space and $f:\del \bbd^2\ra X$ is continuous, then we
can construct the quotient space $Z=(X\coprod \bbd^2)/\{x\sim f(x) : x\in\del\bbd^2\}$,
the result of gluing $\bbd^2$ to $X$ along $f$. Then SvK (with some delicacy choosing the basepoint),
treating $f$ as a loop in $X$, gives
$\pu(Z)\cong \pu(X)*_\bbz\{1\} = \pu(X_2)/<\bbz>^N \cong \pu(X_2)/<[f]>^N$ .
So the effect of gluing on a 2-disk onto a space, on the fundamental group, is to add a new relator, 
namely the word represented by the attaching map (adjusting for basepoint).
All of this applies equally well to attaching several 
2-disks; each adds a new relator. 
This in turn opens up huge possibilities for the computation of $\pu(X)$. For example, for cell complexes
(see below!),
we can inductively compute \mpu\ by starting with the 1-skeleton, with free fundamental group, and 
attaching the 2-cells one by one, which each add a relator to the presentation of $\pu(X)$  .

Knot complements $X=S^3\setminus K$ deformation retract onto a 2-complex which can be built from a planar diagram of the 
knot. From this, a presentation for $\pu(X)$ can be built, with a generator for each strand of the
knot diagram, and a (length 4) relator for each crossing, expressing the relation that
the overstrand conjugates one understrand at the crossing to the other. (The particular form
of the relator ($xax^{-1}=b$ or $x^{-1}ax=b$ is determined by an orientation for the knot.)

\msk

{\bf CW complexes:} The ``right'' spaces to do algebraic topology on.
The basic idea: CW complexes are built inductively, by gluing 
disks onto lower-dimensional strata. $X=\bigcup X^{(n)}$, where
$X^{(0)}$ = a disjoint union of points, and, inductively,
$X^{(n)}$ is built from $X^{(n-1)}$ by gluing $n$-disks $D^n_i$
along their boundaries. 
$X=\bigcup X^{(n)}$
is given the {\it weak topology}; that is,  $C\subseteq X$ is closed \lra\
$C\cap X^{(n)}$ is closed for all $n$. 
Each disk $D^n_i$ has a {\it characteristic map} $\phi_i:D^n_i\ra X$
given by 
$D^n_i\ra X^{(n-1)}\cup(\coprod D^n_i)\ra X^{(n)}\subseteq X$.
$f:X\ra Y$ is cts \lra\ $f\circ \phi_i:D^n_i\ra X\ra Y$ is cts for 
all $D^n_i$. 

A {\it CW pair} $(X,A)$ is a CW complex $X$ and a {\it subcomplex}
$A$. If $(X,A)$ is a CW pair, then
$X/A$ admits a CW structure whose cells are $[A]$ and the cells of 
$X$ not in $A$. We can glue two CW complexes $X,Y$ along isomorphic
subcomplexes $A\subseteq X,Y$, yielding $X\cup_AY$.

Perhaps the most important property of CW complexes (for algebraic topology,
anyway) is the {\it homotopy extension property}; given a CW pair
$(X,A)$, a map $f:X\ra Y$, and a homotopy $H:A\times I\ra Y$ such that
$H|_{A\times 0}=f|_A$, there is a homotopy (extension)
$K:X\times I\ra Y$ with $K|_{A\times I}=H$. This is because 
$B=X\times\{0\}\cup A\times I$ is a retract of $X\times I$; $K$ is the 
composition of this retraction and the ``obvious'' map from $B$ to $Y$.
Consequence: if $(X,A)$ is a CW pair and 
$A$ is contractible, then $X/A\simeq X$.

\msk

{\bf Covering spaces:} 
A map $p:E\ra B$ is called a covering map if for every point $x\in B$, there
is a neighborhood $\mu$ of $x$ (an
{\it evenly covered neighborhood}) so that $p^{-1}(\mu)$ 
is a disjoint union $\mu_\alpha$ of open sets in $E$, each mapped
homeomorphically onto $\mu$ by (the restriction of) $p$ . $B$ is
called the {\it base space} of the covering; $E$ is called the {\it total
space}. 

The disjoint union of 42 copies of a space,
each mapping homeomorphically to a single copy, is an example of a 
{\it trivial covering}. 
The famous 
exponential map $p:\bbr\ra S^1$ given by $t\mapsto e^{2\pi it} = 
(\cos (2\pi t),\sin (2\pi t))$ is a covering map.
We can build many finite-sheeted (every point
inverse is finite) coverings of a bouquet of two circles, by 
assembling $n$ points over the vertex, and then, on either side (the red/blue
sides?),
connecting the points by $n$ (oriented) arcs, one with one red/blue arcs going
in/out of
each vertex. 
Covering spaces of more ``interesting'' graphs can be assembled similarly.
Our basic theme: covering spaces of a (suitably nice) space $X$ have a very close relationship
to $\pu(X,x_0)$. 

\ssk

{\bf Homotopy Lifting Property:} If $p:\wtl{X}\ra X$ is a covering map, 
$H:Y\times I\ra X$ is a homotopy, $H(y,0)=f(y)$, and
$\wtl{f}:Y\ra \wtl{X}$ is a {\it lift} of $f$ (i.e., $p\circ \wtl{f}=f$),
then there is a unique lift $\wtl{H}$ of $H$ with $\wtl{H}(y,0)=\wtl{f}(y)$ .

The idea: using the Lebesgue Number Theorem, we build the homotopy a little 
bit at a time, using inverse images of evenly-covered neighborhoods.
In particular, applying this property in the case $Y=\{*\}$, we get the

{\bf Path Lifting Property}: Given 
a covering map $p:\wtl{X}\ra X$, a path 
$\gamma:I\ra X$ with $\gamma(0)=x_0$, and a point 
$\wtl{x}_0\in p^{-1}(x_0)$, there is a unique path $\wtl{\gamma}$
lifting $\gamma$ with $\wtl{\gamma}(0)=\wtl{x}_0$.

\ssk

An immediate consequence: 
If $p:(\wtl{X},\tilde{x}_0)\ra (X,x_0)$ is a covering map, then the 
induced homomorphism
$p_*:\pu(\wtl{X},\wtl{x}_0)\ra \pu(X,x_0)$ is injective.
Even more, $p_*(\pu(\wtl{X},\wtl{x}_0)))\sset \pu(X,x_0)$
is precisely the elements given by loops at $x_0$, 
whose lifts to paths starting at $\wtl{x}_0$, are loops. 

The cardinality of a point inverse $p^{-1}(y)$ is, by the evenly
covered property, constant on (small) open sets, so the set of 
points of $x$ whose point inverses have any given cardinality
is open. Consequently, if $X$ is connected, this number
is constant over all of $X$, and is called the number of {\it sheets}
of the covering $p:\wtl{X}\ra X$ . 
If $X$ and $\wtl{X}$ are 
path-connected, then the number of sheets of a covering map equals
the index of the subgroup $H=p_*(\pu(\wtl{X},\wtl{x}_0)$ in 
$G=\pu(X,x_0)$ . The idea: loops representing elements in the same coset have lifts
at $\wtl{x}_0$ which end at the same point.

\ssk

The path lifting property (because $\pi([0,1],0)=\{1\}$) is actually a special
case of a more general {\bf lifting criterion}: If 
$p:(\wtl{X},\wtl{x}_0)\ra (X,x_0)$ is a covering map, and 
$f:(Y,y_0)\ra (X,x_0)$ is a map, where
$Y$ is path-connected and locally path-connected, then there is a lift 
$\wtl{f}:(Y,y_0)\ra (\wtl{X},\wtl{x}_0)$ of $f$ (i.e., 
$f=p\circ\wtl{f}$) $\Leftrightarrow$ 
$f_*(\pu(Y,y_0))\sset p_*(\pu(\wtl{X},\wtl{x}_0))$ . 
Furthermore, two lifts of $f$ which agree at a single point are equal.

\ssk

{\bf Universal covering spaces}:
A covering space $(\wtl{X},\wtl{x}_0)$ determines a subgroup of $\pu(X,x_0)$. Does it go the
other way? A particularly
important covering space to identify 
is one which is simply connected. Such a covering 
is essentially unique: if $X$ is {\it locally path-connected} and has two connected, simply connected
covering spaces $p_1:X_1\ra X$ and $p_2:X_2\ra X$, then
there is a homeomorphism $h:X_1\ra X_2$ with $p_2\circ h = p_1$.

Not every (locally path-connected) space $X$ has a universal covering; a 
(further) necessary condition is that $X$ be {\it semi-locally simply connected}:
every point $x\in X$ has a nbhd $x\in\mu\sset X$ with $\iota_*:\pu(U,x)\ra\pi(X,x)$ is trivial.
Conversely, every path connected, locally path connected, and
semi-locally simply connected (S-LSC) space $X$ has a universal covering.
$\widetilde{X}$ is the space whose
points are (equivalence classes $[\gamma]$ of)
based paths $\gamma:(I,0)\ra (X,x_0)$, where two paths are equivalent
if they are homotopic rel endpoints. The projection map is
$p([\gamma])=\gamma(1)$. 

\msk

This in turn is the key to building covering spaces corresponding to any subgroup $H$ of 
$\pu(X)$. [This can, alternatively, be done by mimicking the construction above, 
except paths $\gamma,\eta$ are equivalent when $[\gamma*\overline{\eta}]\in H$.]
The key to this is the {\it deck transformation group
(Deckbewegungsgruppe)}
of a covering space $p:\wtl{X}\ra X$; this is the set of all
homeomorphisms $h:\wtl{X}\ra\wtl{X}$ such that $p\circ h = p$.

By definition, these $h$ permute each of the point inverses
of $p$. Since $h$ is a lift of the projection map
$p$, by the lifting criterion $h$ is determined by which point in $p^{-1}(x_0)$ 
it takes the basepoint 
$\wtl{x}_0$ of $\wtl{X}$ to. A deck transformation sending
$\wtl{x}_0$ to $\wtl{x}_1$ exists $\Leftrightarrow$
$p_*(\pu(\wtl{X},\wtl{x}_0)=p_*(\pu(\wtl{X},\wtl{x}_1)$
[we need one inclusion to give $h$, and the opposite inclusion
to ensure it is a bijection]. 

In general, these two groups $p_*(\pu(\wtl{X},\wtl{x}_0),p_*(\pu(\wtl{X},\wtl{x}_1)$
are always {\it conjugate}, by the projection of a path from 
$\wtl{x}_0$ to $\wtl{x}_1$.
Paths in $\wtl{X}$ from $\wtl{x}_0$ to $\wtl{x}_1$ are in 1-to-1
corresp with the cosets of $H=p_*(\pu(\wtl{X},\wtl{x}_0)$ in 
$\pu({X},{x}_0)$; so deck transformations are in 1-to-1 
corresp with cosets whose representatives conjugate 
$H$ to itself. The set of such elements in $G$ is called the 
{\it normalizer of $H$ in $G$}, and denoted $N_G(H)$ or simply
$N(H)$. The deck transformation group is therefore
isomorphic to the group $N(H)/H$ under
$h\mapsto$ the coset with representative the projection of the path from 
$\wtl{x}_0$ to $h(\wtl{x}_0)$. 

\ssk

Applying this to the universal covering space
$p:\wtl{X}\ra X$, in this case $H=\{1\}$, so $N(H)=\pu(X,x_0)$.
So the deck transformation group is isomorphic to $\pu(X,x_0)$. 
For example, this gives the quickest possible proof 
that $\pu(S^1)\cong \bbz$, since $\bbr$ is a 
contractible covering space, whose deck transformations
are the translations by integer distances. 

Thus $\pu(X)$ acts on its universal cover as a group of
homeomorphisms. And since this action is {\it simply transitive}
on point inverses [there is exactly one (that's the simple
part) deck transformation carrying any one point in a point 
inverse to any other one (that's the transitive part)], the 
quotient map from $\wtl{X}$ to the orbits of this action \underbar{is}
the projection map $p$ to $X$.

But! Given $G=\pu(X,x_0)$ and its 
action on a univ cover $\wtl{X}$, we can, instead of modding out by $G$,
mod out by any \underbar{subgroup} $H$ of $G$, to build $X_H=\wtl{X}/H$. 
This is a space with $\pi_1(X_H)\cong H$, having $\wtl{X}$ as univ covering.
And since the quotient (covering) map $p_G:\wtl{X}\ra X=\wtl{X}/G$ factors through $\wtl{X}/H$,
we have an induced map $p_H:\wtl{X}/H\ra X$, which is a covering map.
So every subgroup of $G$ is the
fundamental group of a covering of $X$. Even more:

{\bf The Galois correspondence:} 
Two coverings
$p_1:X_1\ra X$ , $p_2:X_2\ra X$ are {\it isomorphic} if there is a homeo
$h:X_1\ra X_2$ with $p_1=p_2\circ h$. Isomorphic coverings give,
under projection, conjugate subgroups of $\pu(X,x_0)$.
For a path-connected, locally path-connected, semi-locally simply-connected space $X$,  
the image of the induced homomorphism on \mpu\ 
gives a one-to-one correspondence between 
[isomorphism classes of (connected) coverings of $X$] and 
[conjugacy classes of subgroups of $\pu(X)$].

\ssk

So, for example, if you have a group $G$ that you are interested in, you know of a (nice enough) 
space $X$ with $\pu(X)\cong G$, and you know enough about the coverings of $X$, then you can
gain information about the subgroup structure of $G$. 

For example, a free group $F(\Sigma)$ is \mpu\ of a bouquet of circles $X$. Since every covering
of $X$ is a graph, we have: every
subgroup of a free group is free. 
A subgroup $H$ of index $n$ in $F(\Sigma)$ corresponds to a $n$-sheeted covering $\wtl{X}$ of $X$. If
$|\Sigma| = m$, then $\wtl{X}$ will have $n$ vertices and $nm$ edges. Collapsing a maximal
tree, having $n-1$ edges, to a point, leaves a bouquet of $nm-n+1$ circles, so $H\cong F(nm-n+1)$.

Given a free group
$G=F(a_1,\ldots a_n)$ and a collection of words $w_1,\ldots w_m\in G$,
we can determine the rank and ndex of the subgroup it $H$ they
generate by building the corresponding cover. The idea is
to start with a bouquet of $m$ circles, each subdivided 
and labelled to spell
out the words $w_i$. Then we repeatedly identify edges sharing
on common vertex if they are labelled precisely the same (same
letter {\it and} same orientation). This process is known
as {\it folding}. When done, we have (by adding trees if needed),
constructed the covering corresponding to $<w_1,\ldots w_m>\sset G$.

With work, this same process can be applied to subgroups of finitely presented
groups (to be certain it stops, one usually needs a priori knowledge that the 
subgroup has finite index). In so doing, it yields a presentation for the subgroup!

\msk

{\bf Postscript: why care about covering spaces?} The preceding discussion
probably makes it clear that covering places play a central role in
(combinatorial) group theory. It also plays a role in embedding 
problems; a common scenario is to have a map $f:Y\ra X$ which is 
injective on \mpu , and we wish to know if we can lift $f$ to a 
finite-sheeted covering so that the lifted map $\widetilde{f}$ is 
homotopic to an embedding. Information that is easier to obtain 
in the case of an embedding can then be passed down to gain information
abut the original map $f$



\vfill
\end{document}



