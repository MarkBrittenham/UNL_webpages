


\documentclass[12pt]{article}
\usepackage{amsfonts}
\usepackage{amssymb}
\usepackage{dutchcal}

\textheight=10in
\textwidth=6.5in
\voffset=-1in
\hoffset=-1in

\begin{document}

\def\ctln{\centerline}
\def\msk{\medskip}
\def\bsk{\bigskip}
\def\ssk{\smallskip}
\def\hsk{\hskip.3in}
\def\ra{\rightarrow}
\def\ubr{\underbar}

\def\mt{{\mathcal T}}
\def\mb{{\mathcal B}}
\def\ms{{\mathcal S}}
\def\mu{{\mathcal U}}
\def\mv{{\mathcal V}}

\def\bbr{{\mathbb R}}
\def\bbz{{\mathbb Z}}
\def\bbq{{\mathbb Q}}
\def\spc{$~$\hskip.15in$~$}

\def\sset{\subseteq}
\def\del{\partial}
\def\lra{$\Leftrightarrow$}
\def\bra{$\Rightarrow$}
\def\wtl{\widetilde}


%%\UseAMSsymbols

\ctln{\bf Math 872 Problem Set 5}

\msk

Starred (*) problems are due Thursday, March 5.


\begin{description}


\item{32.} Is the subgroup $H$ of $F(a,b)$ generated by the elements $\{aba,bba,bab,abb\}\sset F(a,b)$
isomorphic to a free group on 4 letters? That is, if you build a covering space $p:\wtl{X}\ra S^1\vee S^1$
with $p_*(\pi_1(\wtl{X}))=H$ (and you should!), is $\pi_1(\wtl{X})$ free on four generators?

\msk

\item{33.} [Hatcher, p.79, \# 8] Let $p:\wtl{X}\ra X$ and $q:\wtl{Y}\ra Y$ be covering spaces of path-connected,
locally path connected spaces $X$ and $Y$ with $\wtl{X}$ and $\wtl{Y}$ simply-connected.
Show that if $X$ and $Y$ are homotopy equivalent, then $\wtl{X}$ and $\wtl{Y}$ are homotopy
equivalent. 

\msk

\item{\spc} [Note: the problem statement in Hatcher points to an earlier problem that might help...]

\msk

\item{(*)} 34. [Hatcher, p.79, problem \# 4, sort of] Build a connected, simply connected covering 
space $p:\wtl{X}\ra X$ of the wedge of a circle and a 2-sphere $X = S^1\vee S^2$ . 

\item{\spc} [Note: problem \# 39 might give you some guidance on what parts of it should look like...]

\msk

\item{(*)} 35. Show that the fundamental group of the closed
orientable surface $\Sigma_2$ of genus 2 is not abelian. 

\msk

\item{\spc} [Hint: to show that for some pair of loops $\gamma,\eta$ 
that $\gamma*\eta*\overline{\gamma}*\overline{\eta}$ isn't homotopically trivial, show that it (at least once!)
does not lift to a loop in \underbar{some} covering space of $\Sigma_2$.]

\msk

\item{36.} A group $G$ is called {\it residually finite} if for every $g\in G$ with $g\neq 1$, there is a
finite group $H$ and a homomorphism $\varphi:G\ra H$ with $\varphi(g)\neq 1$. Show that $G$ is residually
finite \ubr{if} \ubr{and} \ubr{only} \ubr{if} for 
some (equivalently, any!) CW-complex $X$ with $\pi_1(X,x_0)\cong G$ and any loop $\gamma:I\ra X$ at $x_0$ with 
$1\neq [\gamma]\in\pi_1(X)$, there is a finite-sheeting covering space $p:\wtl{X}\ra X$ and basepoint
$\wtl{x_0}$ over $x_0$ so that $\gamma$ does \ubr{not} lift to a loop at $\wtl{x_0}$.

\msk

\item{37.} Show that if a group $G$ acts freely ($x=gx$ $\Rightarrow$ $g=1$)
and properly discontinuously (for all $x\in X$ there is a nbhd $\mu$ of $x$ such that
$\{g\ :\ g(\mu)\cap\mu\neq\emptyset\}$ is finite)
on a space $X$, then the quotient map $p:X\ra X/G=X/\{x\sim gx \textrm{\ for all\ } g\in G\}$
given by $p(x)=[x]$
is a covering map. In particular if $X$ is Hausdorff and $G$ is a finite group acting freely on $X$,
then $p:X\ra X/G$ is a covering map. 

\msk

\item{\spc} [Pointless remark: some people would write our quotient 
space as $G\backslash X$, since $G$ is acting on the left, and so is being quotiented out from the left,
although the Wikipedia entry on the matter, 
{\it http://en.wikipedia.org/wiki/Group{\_}action},
agrees with us in this. Besides, as one usually learns when TeXing things, TeX doesn't like $\backslash$
as a symbol, it asks what the macro ``$\backslash$X[next symbol]'' is supposed to mean ...]

\msk

\item{(*)} 38. [Hatcher, p.79, problem \# 9] Show that if a path-connected, locally path-connected space
$X$ has $\pi_1(X)$ finite, then every map $f:X\ra S^1$ is homotopic to a constant map.

\msk

\item{39.} [Hatcher, p.80, problem \# 15] Suppose $p:\wtl{X}\ra X$ is a simply connected covering space of $X$,
and $A\sset X$ is a path-connected, locally path-connected subspace of $X$, with $\wtl{A}]\sset\wtl{X}$ a 
path component of $p^{-1}(A)$. Show that $p_*(\pi_1(\wtl(A))\sset \pi_1(A)$ is the kernel of the inclusion-induced 
homomorphism $\pi_1(A)\ra\pi_1(X)$. 



\end{description}
\vfill

\end{document}
