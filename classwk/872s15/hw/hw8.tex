


\documentclass[12pt]{article}
\usepackage{amsfonts}
\usepackage{amssymb}
\usepackage{dutchcal}

\textheight=10in
\textwidth=6.5in
\voffset=-1in
\hoffset=-1in

\begin{document}

\def\ctln{\centerline}
\def\msk{\medskip}
\def\bsk{\bigskip}
\def\ssk{\smallskip}
\def\hsk{\hskip.3in}
\def\ra{\rightarrow}
\def\ubr{\underbar}

\def\mt{{\mathcal T}}
\def\mb{{\mathcal B}}
\def\ms{{\mathcal S}}
\def\mu{{\mathcal U}}
\def\mv{{\mathcal V}}

\def\bbr{{\mathbb R}}
\def\bbz{{\mathbb Z}}
\def\bbq{{\mathbb Q}}
\def\spc{$~$\hskip.15in$~$}

\def\sset{\subseteq}
\def\del{\partial}
\def\lra{$\Leftrightarrow$}
\def\bra{$\Rightarrow$}
\def\wtl{\widetilde}


%%\UseAMSsymbols

\ctln{\bf Math 872 Problem Set 8}

\msk

Starred (*) problems are due Thursday, April 16.


\begin{description}

\item{(*)} 53. [Hatcher, p.132, \#12] Show that for chain maps $f,g$ between chain
complexes ${\mathcal A} = \{A_n\},{\mathcal B}=\{B_n\}$, the relation 
``$f$ and $g$ are chain homotopic'' is an equivalence relation.

\msk

\item{54.} [Hatcher, p.133, \# 27(a)] Let $f:(X,A)\ra (Y,B)$ be a map of pairs such that
both $f:X\ra Y$ and $f:A\ra B$ are homotopy equivalences. Show that
the induced map $f_*:H_n(X,A)\ra H_n(Y,B)$ is an isomorphism for all $n$.

\msk

\item{55.} [Hatcher, p.132, \#15] Show that if $A\subseteq X$, then the inclusion map
$i:A\ra X$ induces an isomorphism on homology groups \lra\ 
$H_n(X,A)=0$ for all $n\geq 0$.

\msk

\item{(*)} 56. Show that if a short exact sequence
\hskip.2in
$0\ra A{\buildrel{\alpha}\over\ra} B{\buildrel{\beta}\over\ra} C\ra 0$
\hskip.2in
{\it splits}, that is, there is a map $\gamma:B \ra A$ with
$\gamma\circ\alpha=I$, then the map $\varphi:B\ra A\oplus C$
given by $b\mapsto (\gamma(b),\beta(b))$, is an isomorphism.

\msk

\item{\spc} [This is part of the Splitting Lemma, proved in Hatcher, p.147. Splitting is equivalent 
to the existence of $\delta:C\ra B$ satisfying
$\beta\circ\delta = I$, but this is irrelevant to the question above.]

\msk

\item{(*)} 57. Show that if $A\subseteq X$ and $r:X\ra A$ is a retraction, then
for every $n$,

\item{\spc} $H_n(X)\cong H_n(A)\oplus H_n(X,A)$. 

\ssk

\item{\spc} [Hint: show that the (piece of) the long exact homology sequence

\item{\spc} $H_n(A)\ra H_n(X)\ra H_n(X,A)$ is ``really'' 

\item{\spc} $0\ra H_n(A)\ra H_n(X)\ra H_n(X,A)\ra 0$, and splits.]

\msk

\item{58.} [Hatcher, p.156, \# 12] 
$S^1\times S^1/[S^1\times\{*\}\cup\{*\}\times S^1$ is homeomorphic to $S^2$.
Show that the quotient map $q:S^1\times S^1\ra S^2$ induces an isomorphism on $H_2$,
so $q$ is not null-homotopic. Show, conversely (using covering spaces) that 
any map $p:S^2\ra S^1\times S^1$ \underbar{is} null-hmotopic.

\msk

\item{59.} Find examples of spaces and subspaces
$A_0\subseteq X_0$ and $A_1\subseteq X_1$ so that
$H_*(X_0)\cong H_*(X_1)$ and $H_*(A_0)\cong H_*(A_1)$,
but $H_*(X_0,A_0)\not\cong H_*(X_1,A_1)$ . (If you want to make
it more challenging, find examples with all of the spaces path-connected?
Note that Problem \#54 gives a hint on how \underbar{not} to solve this problem...)




\end{description}
\vfill

\end{document}


Show that if $A\subseteq X$ and the identity map 
$I:X\ra X$ is homotopic 
to a map $f:X\ra X$ with $f(X)\subseteq A$, then
for every $n$, $H_n(A)\cong H_n(X)\oplus H_{n+1}(X,A)$.

\item{} (So $A$ has more ``holes'' than $X$ does...)




[Hatcher, p.131, \#9]

\msk



