
\magnification=1400
\overfullrule=0pt
\parindent=0pt

\nopagenumbers

\input amstex

%\voffset=-.6in
%\hoffset=-.5in
%\hsize = 7.5 true in
%\vsize=10.4 true in

%\voffset=1.8 true in
%\hoffset=-.6 true in
%\hsize = 10.2 true in
%\vsize=8 true in

\input colordvi

\def\cltr{\Red}		  % Red  VERY-Approx PANTONE RED

\loadmsbm

\input epsf

\def\ctln{\centerline}
\def\u{\underbar}
\def\ssk{\smallskip}
\def\msk{\medskip}
\def\bsk{\bigskip}
\def\hsk{\hskip.1in}
\def\hhsk{\hskip.2in}
\def\dsl{\displaystyle}
\def\hskp{\hskip1.5in}

\def\lra{$\Leftrightarrow$ }
\def\ra{\rightarrow}
\def\mpto{\logmapsto}
\def\pu{\pi_1}
\def\mpu{$\pi_1$}
\def\sig{\Sigma}
\def\msig{$\Sigma$}
\def\ep{\epsilon}
\def\sset{\subseteq}
\def\del{\partial}
\def\inv{^{-1}}
\def\wtl{\widetilde}
%\def\lra{\Leftrightarrow}
\def\del{\partial}
\def\delp{\partial^\prime}
\def\delpp{\partial^{\prime\prime}}
\def\sgn{{\roman{sgn}}}
\def\wtih{\widetilde{H}}
\def\bbz{{\Bbb Z}}
\def\bbr{{\Bbb R}}
\def\rtar{$\Rightarrow$}

\def\cltr{\Red}		  % Red  VERY-Approx PANTONE RED
\def\cltb{\Blue}		  % Blue  Approximate PANTONE BLUE-072
\def\cltg{\PineGreen}	  % ForestGreen  Approximate PANTONE 349


{\bf An extended example:}

\ctln{\bf Wirtinger presentations for knot complements.}

\msk

A {\it knot} $K$ is (the image of) an embedding $h:S^1\hookrightarrow\bbr^3$. Wirtinger 
gave a prescription for taking a planar projection of $K$ and producing a presentation
of $\pi_1(\bbr^3\setminus K)=\pi_1(X)$. The idea: think of $K$ as lying on the projection
plane, except near the crossings, where it arches under itself. 
We build a CW-complex $Y\subseteq X$ that $X$ deformation retracts to. A presentation for $\pi_1(Y)$
gives us $\pi_1(X)$.

\msk

\vbox{\hsize=2in
\leavevmode
\epsfxsize=3in
\epsfbox{wirtinger1a.ai}}

\msk

\hskip1.5in
\vbox{\hsize=3in
\leavevmode
\epsfxsize=3in
\epsfbox{wirtinger2.ai}}

\msk


To build $Y$, glue rectangles arching under the strands of $K$ to a horzontal plane lying just above the projection plane of $K$. 
At the crossing, the rectangle is glued to the rectngle arching under the over-strand. $X$ deformation retracts to $Y$; the top half 
of $\bbr^3$ deformation retracts to the top plane, the parts of $X$ inside the tubes formed by the rectangles
radially retract to the boundaries of the tubes, and the bottom part of $X$ vertically retracts onto $Y$.
Formally, we should really keep a ``slab'' above the plane, to give us a place to run arcs to a fixed
basepoint in the interior of the slab.

\msk

We think of $Y$ as being built up from the slab $C$, by gluing on annuli $A_i\cong S^1\times I$, one for each rectangle $R_i$ glued on;
the rectangle $S_i$ lying above $R_i$ in the bottom of the slab $C$ is the other half of the annulus. Then we glue on the 2-disks $D_j$,
one for each crossing of the knot projection. A little thought shows that there are as many annuli as disks;
the annuli correspond to the unbroken strands of the knot projetion, which each have two ends, and each crossing is where
two ends terminate (so there are two ends for every $A_i$ and two ends for every $D_j$, so there are half as
many of each as there are total number of ends). To make sure that all of our interections are path connected,
and to formally use a single basepoint in all of our computations, we join every one of the annuli and disks to a basepoint lying
in the slab by a collection of (disjoint) paths.

\ssk

Now starting with the slab (with $\pi_1=1$), add the $A_i$ one at a time; each has $\pi_1=\bbz$,
generated by a loop which travels once around the $S^1$-direction, and its intersection with $C\cup$ the
previously glued on annuli is the rectangle $S_i$, which is simply connected. So, inductively,
$\pi_1(C\cup A_1\cup\cdots \cup A_i)\cong \pi_1(C\cup A_1\cup\cdots \cup A_{i-1})*\pi_1(A_i)\cong F(i-1)*\bbz\cong F(i)$ is the 
free group on $i$ letters, so, adding all $n$ (say) of the annuli yields $F(n)$. Then we glue on the $n$ 2-disks $D_j$;
these add $n$ relators to the presentation $\langle x_1,\ldots,x_n | \rangle$. To determine the relators, choose specific 
generators for our $\pi_1(A_i)$,  by {\it orienting} the knot
(choosing a direction to travel around it) and choosing the loop which goes counter-clockwise around the annulus,
when you face in the direction of the orientation. Going around the boundary of the 2-disk
$D_j$ spells out the word $x_rx_sx_r^{-1}x_t^{-1}$ or  $x_rx_s^{-1}x_r^{-1}x_t$ reading counter-clockwise,
depending on orientations.
Carrying this out for every 2-disk completes the presentation of $\pi_1(Y)\cong \pi_1(X)$.


\ctln{\vbox{\hsize=2in
\leavevmode
\epsfxsize=2.5in
\epsfbox{wirtinger3.ai}}}

\ssk

With practice, it becomes completely routine to read off a presentation for the 
fundamental group of $\bbr^3\setminus K$ from a projection of $K$. For example, from the projection above, we have

\ssk

$\pi_1(\bbr^3\setminus K)\cong \langle x_1,\ldots ,x_8 | x_8x_1=x_2x_8, x_2x_7=x_8x_2, x_5x_8=x_1x_5,
x_1x_5=x_6x_1, x_3x_6=x_7x_3 ,x_7x_2=x_3x_7 ,x_3x_2=x_2x_4, x_7x_4=x_5x_7\rangle$




\vfill
\end

