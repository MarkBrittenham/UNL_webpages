\documentclass[12pt]{article}

\usepackage{amsmath, amsthm, amssymb}
\usepackage{dutchcal}

\usepackage[all]{xy}

\textheight=10.1in
\textwidth=7.4in
\voffset=-1.35in
\hoffset=-1in

\def\ctln{\centerline}
\def\nidt{\noindent}
\def\msk{\medskip}
\def\bsk{\bigskip}
\def\ssk{\smallskip}
\def\hsk{\hskip.3in}
\def\ra{\rightarrow}
\def\ubr{\underbar}
\def\im{\textrm{Im }}
\def\krr{\textrm{Ker }}

\def\mt{{\mathcal T}}
\def\mb{{\mathcal B}}
\def\ms{{\mathcal S}}
\def\mu{{\mathcal U}}
\def\mv{{\mathcal V}}

\def\bbr{{\mathbb R}}
\def\bbz{{\mathbb Z}}
\def\bbq{{\mathbb Q}}
\def\spc{$~$\hskip.15in$~$}

\def\sset{\subseteq}
\def\del{\partial}
\def\lra{$\Leftrightarrow$}
\def\bra{$\Rightarrow$}
\def\wtl{\widetilde}
\def\sxs{short exact sequence}
\def\sxss{short exact sequences}
\def\lxs{long exact sequence}
\def\lxss{long exact sequences}

\pagenumbering{gobble}

\begin{document}


\ctln{\bf The zen of long exact sequences}


\bsk

Extracting information from a \lxs\ amounts to exploiting knowledge
of the identities of some of the groups and/or some of the maps involved.
The general setup we find coming from the conversion of \sxss\
of chain maps to \lxss\ in homology is that every third group comes from the same chain complex.
So we generally expect things to come in threes; we know/control every third
group (when we control one) and/or every third map (when we control one).

\msk

A typical situation is that we are building a new space/chain complex out of two others
that we ``understand'', meaning that we know the groups for two of the three spaces 
and the maps running between them.
In the \lxs\

\ssk

\ctln{$\xymatrix{
\ar[r]&
A_n \ar[r]^{\alpha_n}&
B_n \ar[r]^{\beta_n} &
C_n \ar[r]^{\gamma_n}&
A_{n-1}\ar[r]^{\alpha_{n-1}}&
B_{n-1}\ar[r]&
}$}

\ssk

\nidt if we know the groups $A_k$ and $B_k$ and the maps $\alpha_k$ between them,
then we can harvest the \lxs\ to give a collection of \sxss\ by noting that
$\krr \beta_n = \im \alpha_n$ (which we know) and $\im \gamma_n = \krr \alpha_{n-1}$
(which we know). So $\gamma_n$ maps onto $\krr \alpha_{n-1}$, making

\ssk

\ctln{$\xymatrix{
A_n \ar[r]^{\alpha_n}&
B_n \ar[r]^{\beta_n} &
C_n \ar[r]^{\gamma_n}&
\krr {\alpha_{n-1}} \ar[r]&
0
}$}

\ssk

\nidt exact, and on the other end, since by one of those Noether theorems, 
the induced map $\overline{\beta}_n:B_n/\krr\beta_n\ra C_n$ is injective
(with the same image as $\beta_n$), we have

\ssk

\ctln{$\xymatrix{
0 \ar[r]&
B_n/\im \alpha_n \ar[r]^{\overline{\beta}_n} &
C_n \ar[r]^{\gamma_n}&
\krr {\alpha_{n-1}} \ar[r]&
0
}$}

\ssk

\nidt is (short) exact. The groups on the ends are ones that, in principle, we know how to compute from the data
$\xymatrix{A_n \ar[r]^{\alpha_n}&B_n}$. On the other hand, knowledge of the end groups is not
enough to determine the middle group $C_n$; but if we know that the sequence 
splits (which is always true, for example, if the rightmost group $\krr {\alpha_{n-1}}$ is free
abelian (which, in turn, is true, for example, if $A_{n-1}$ is free abelian)), then the middle
group is the direct sum of the outer groups.

\ssk

A similar analysis applies when we `know' $\xymatrix{B_n \ar[r]^{\beta_n} & C_n}$, or we know 
$\xymatrix{C_n \ar[r]^{\gamma_n}&A_{n-1}}$ .

\msk

Another situation that we can often engineer is to know that every third map (the $\alpha_k$,
for example) is always injective, or always surjective, or always the $0$ map. [We have
already encountered topological conditions which make this happen.] As it happens, each of these 
sort of imply the other, in fact: in the exact sequence

\ssk

\ctln{$\xymatrix{\ar[r]&
B_{n+1} \ar[r]^{\beta_{n+1}} &
C_{n+1} \ar[r]^{\gamma_{n+1}}&
A_n \ar[r]^{\alpha_n}&
B_n \ar[r]^{\beta_n} &
C_n \ar[r]^{\gamma_n}&
A_{n-1}\ar[r]&}$}

\ssk

\nidt if $\alpha_n$ is injective, then 
$\{0\} = \krr\alpha_n=\im\gamma_{n+1}$, so $\gamma_{n+1}$ is the $0$ map,
so $\krr\gamma_{n+1} = C_{n+1} = \im\beta_{n+1}$, so $\beta_{n+1}$ is surjective.
similarly, if $\alpha_n$ is surjective, then $\beta_n$ is the $0$ map, and so
$\gamma_n$ is injective, while if $\alpha_n$ is the $0$ map, then $\beta_n$ is
injective and $\gamma_{n+1}$ is surjective. So if all of the $\alpha_k$
have the same character, then every third map (somewhere) in the \lxs\ is
the $0$ map, and so the map that precedes it is surjective and the map that
follows it is injective. This enables us, again, to harvest the \lxs\ to create a
collection of \lxss, which will be one of

\ssk


\hskip1in{$\xymatrix{
0 \ar[r]&
A_n \ar[r]^{\alpha_n} \ar[r]&
B_n \ar[r]^{\beta_n} &
C_n \ar[r]&
0
}$ , or}

\hskip2in{$\xymatrix{
0 \ar[r]&
C_{n+1} \ar[r]^{\gamma_{n+1}}&
A_n \ar[r]^{\alpha_n} \ar[r]&
B_n \ar[r]&
0
}$ , or}

\hskip3in{$\xymatrix{
0 \ar[r]&
B_n \ar[r]^{\beta_n} &
C_n \ar[r]^{\gamma_n}&
A_{n-1} \ar[r]&
0
}$}



\ssk

\nidt If we treat the $A_n$ and $B_n$ as `known', and the $C_n$ as unknown,
then in the first case $C_n\cong B_n/\im\alpha_n$, while in the second
case $C_{n+1}\cong \im\gamma_{n+1}\cong\krr\alpha_{n}$. The third case, 
again, poses an `extension problem', and so without furher information about
$B_n$, $A_{n-1}$, and the maps we cannot definitively determine $C_n$.

\ssk

And recognizing a $0$ map can happen many ways. One of the groups being $0$ is the 
quickest, but, for example, the only map from a torsion group (all elements have
finite order) to a torsion-free group (all non-zero elements have infinite order!)
is the $0$ map.



\end{document}




