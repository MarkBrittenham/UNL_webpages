


\documentclass[12pt]{article}
\usepackage{amsfonts}
\usepackage{amssymb}
\usepackage{dutchcal}

\textheight=10in
\textwidth=6.5in
\voffset=-1in
\hoffset=-1in

\begin{document}

\def\ctln{\centerline}
\def\msk{\medskip}
\def\bsk{\bigskip}
\def\ssk{\smallskip}
\def\hsk{\hskip.3in}
\def\ra{\rightarrow}
\def\ubr{\underbar}

\def\mt{{\mathcal T}}
\def\mb{{\mathcal B}}
\def\ms{{\mathcal S}}
\def\mu{{\mathcal U}}
\def\mv{{\mathcal V}}

\def\bbr{{\mathbb R}}
\def\bbz{{\mathbb Z}}
\def\bbq{{\mathbb Q}}
\def\spc{$~$\hskip.15in$~$}

\def\sset{\subseteq}
\def\del{\partial}
\def\lra{$\Leftrightarrow$}
\def\bra{$\Rightarrow$}
\def\wtl{\widetilde}


%%\UseAMSsymbols

\ctln{\bf Math 872 Problem Set 7}

\msk

Starred (*) problems are due Thursday, April 9.


\begin{description}

\item{48.} [Hatcher, p.131, \#9] Compute the simplicial homology
groups of the $\Delta$-complex $X$ obtained from the $n$-simplex
$\Delta^n=[v_0,\ldots v_n]$ by identifying every face of the
same dimension to one another (respecting the implied orientations
from the ordering of the vertices, so, e.g, $[v_1,v_2,v_4]$ is glued
to $[v_0,v_3,v_4]$ via the homeomorphism that sends $v_0$ to $v_1$,
$v_2$ to $v_3$, etc.). Thus $X$ has a single $k$-simplex for each $k\leq n$.

\msk

\item{49.} Show that the Smith normal form of the matrix 
$\left( \begin{array}{cc} 2 & 0 \\ 0 & 3 \end{array} \right)$
is 
$\left( \begin{array}{cc} 1 & 0 \\ 0 & 6 \end{array} \right)$. Explain why this 
implies that $\bbz_2\oplus\bbz_3\cong\bbz_6$ .

\msk

\item{(*)} 50. Find the Smith normal form of the matrix
$\left( \begin{array}{ccc} 3 & 5 & 9 \\ 1 & -1 & 1 \\ 5 & 5 & 11 \end{array} \right)$

\msk


\item{50.} Find the Smith normal form of the matrix
$\left( \begin{array}{ccc} 2 & 2 & 5 \\ 2 & 3 & 1 \\ 5 & 2 & 6 \end{array} \right)$

\msk

\item{\bf (*)} 51. Regarding the $n$-simplex $X=\Delta^n$ as a $\Delta$-complex 
in the natural way, show that if $A\subseteq X$ is a subcomplex with $H_{n-1}(A)\neq 0$,
then $A=\del \Delta^n$. (Hint: show that an $(n-1)$-cycle for $A$ (and hence for $X$)
must either be 0 or have non-zero coefficient for \underbar{every}
$(n-1)$-dimensional face of $X$.)

\msk

\item{\bf (*)} 52. [Hatcher, p.132, \#11] Show that if $A\subseteq X$ is a retract of $X$, 
then the inclusion map
$\iota:A\ra X$ induces an injection on all singular homology groups. 





\end{description}
\vfill

\end{document}



\msk



