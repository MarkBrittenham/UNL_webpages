

\documentclass[12pt]{article}
\usepackage{amsfonts}
\usepackage{amssymb}
\usepackage{dutchcal}

\textheight=10in
\textwidth=6.5in
\voffset=-1in
\hoffset=-1in

\begin{document}

\def\ctln{\centerline}
\def\msk{\medskip}
\def\bsk{\bigskip}
\def\ssk{\smallskip}
\def\hsk{\hskip.3in}
\def\ra{\rightarrow}
\def\ubr{\underbar}

\def\mt{{\mathcal T}}
\def\mb{{\mathcal B}}
\def\ms{{\mathcal S}}
\def\mu{{\mathcal U}}
\def\mv{{\mathcal V}}

\def\bbr{{\mathbb R}}
\def\bbz{{\mathbb Z}}
\def\spc{$~$\hskip.15in$~$}

\def\sset{\subseteq}
\def\del{\partial}
\def\lra{$\Leftrightarrow$}
\def\bra{$\Rightarrow$}



%%\UseAMSsymbols

\ctln{\bf Math 872 Problem Set 1}

\msk

Starred (*) problems are due Thursday, January 29.


\begin{description}

\item{1.} Show that the cone on a circle $cS^1 = S^1\times I/\sim$,
where $(x,1)\sim(y,1)$ for $x,y\in S^1$, with the quotient topology,
is homeomorphic to the unit disk $D^2=\{\vec{x}\in\bbr^2\ :\ ||\vec{x}||\leq 1\}$.

\msk

\item{2.} If $x_0\in A\sset X$ and $\pi_1(X,x_0)=\{1\}$, show that, for any space $Y$, if
a cts map $f:A\ra Y$ extends to a map $g:X\ra Y$, then $f_*:\pi_1(A,x_0)\ra\pi_1(Y,f(x_0))$
is the trivial map.

\msk

\item{(*)} 3. [Hatcher, p.38, \#5] Show that $\pi_1(X,x_0)=\{1\}$ for
every $x_0\in X$ if and only if every cts map $f:S^1\ra X$ extends
to a map $F:D^2\ra X$ (i.e., for some $F$, we have $F\circ\iota = f$).

\msk

\item{4.} [Hatcher, p.38, \#2] Show that, in general, if 
$\alpha,\beta:I\ra X$ are paths which are {\it homotopic rel endpoints},
$\alpha(0)=\beta(0)=x_0$, $\alpha(1)=\beta(1)=x_1$, then their associated change of basepoint
maps are equal: 
\hsk ${\alpha}_*={\beta}_*:\pi_1(X,x_0)\ra \pi_1(X,x_1)$.

\msk

\item{\spc} [N.B.: Hatcher calls these maps $\beta_\alpha$ and (sorry...) $\beta_\beta$, for
reasons I don't understand. Maybe $\beta$ = `basepoint'?]

\msk

\item{5.} [Hatcher, p.38, \#3] Show that for a path-connected space $X$, 
$\pi_1(X)$ is abelian \lra\ the change of basepoint maps
are all independent of path, i.e.,  

\hskip.3in for $\alpha,\beta:I\ra X$ with $\alpha(0)=\beta(0)=x_0$ and $\alpha(1)=\beta(1)=x_1$,
we always have ${\alpha}_*={\beta}_*:\pi_1(X,x_0)\ra \pi_1(X,x_1)$. 

\ssk

\item{\spc} [Useful notation: our hypothesis can be expressed symbolically by saying that $\alpha,\beta$ are maps
of triples; $\alpha,\beta:(I,0,1)\ra(X,x_0,x_1)$.]

\msk

\item{(*)} 6. Show that every homomorphism $\varphi:\bbz\ra \pi_1(X,x_0)$ can be realized as 
the induced homomorphism $\varphi=f_*$ of a continuous map $f:(S^1,(1,0))\ra (X,x_0)$. 
[Hint: Look at $\varphi(1)\in\pi_1(X,x_0)$.]

\msk

\item{(*)} 7. [Hatcher, p.39, \# 13] If $x_0\in A\subseteq X$ and $A$ is path-connected, show that the inclusion-induced map
$\iota_*:\pi_1(A,x_0)\ra\pi_1(X,x_0)$ is surjective \lra\ for every path $\gamma:I\ra X$ with endpoints in
$A$, $\gamma(0),\gamma(1)\in A$, $\gamma$ is homotopic rel endpoints to a path in $A$, that is,
$\gamma\simeq\alpha$ rel $\del I$ with $\alpha:I\ra A\subseteq X$.

\msk

\item{8.} A {\it topological group} is a space $G$ with continuous maps $G\times G\ra G$ and $G\ra G$, denoted
$(g,h)\mapsto g\cdot h$ and $g\mapsto g^{-1}$, which (together with an $e\in G$) make $G$ a group. Show that
for loops $\alpha,\beta:(I,\del I)\ra (G,e)$, the loop $\gamma(t)=\alpha(t)\cdot\beta(t)$ is homotopic,
rel endpoints, to \underbar{both} $\alpha*\beta$ and $\beta*\alpha$. Conclude that for any
topological group $G$, $\pi_1(G,e)$ is \underbar{abelian}.

\ssk

\item{\spc} [Hint: $\alpha*\beta(t) = (\alpha*c_e)(t)\cdot(c_e*\beta)(t)$.]


\end{description}
\vfill

\end{document}
