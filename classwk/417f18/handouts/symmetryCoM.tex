

\magnification=1200


\input epsf


\def\ctln{\centerline}
\def\ssk{\smallskip}
\def\msk{\medskip}
\def\bsk{\bigskip}
\def\ubr{\underbar}
\def\ovln{\overline}
\def\bbr{{\rm I\hskip-2pt R}}

\parindent=0pt

\centerline{\bf Symmetries fix the center of mass}

\medskip

Central (sorry...) to our study of symmetry is the fact that
any rigid motion $R$ of the plane (although this can be extended to higher
dimensions) which carries a region $Q$ to itself, $R(Q)=Q$, must satisfy
$R(\overline{x},\overline{y})=(\overline{x},\overline{y})$, where
$(\overline{x},\overline{y})$ is the 
\ubr{center} \underbar{of} \underbar{mass} (or, more properly, 
\underbar{centroid}) of the region $Q$.

\medskip

This follows from a bit of multivariable calculus and linear algebra.
Recall (!) that the coordinates $(\overline{x},\overline{y})$ of the centroid 
are those values for which

\smallskip

\ctln{$\int\int_Q x-\overline{x}\ dA = 0$ and $\int\int_Q y-\overline{y}\ dA = 0$}

\ssk

[The idea: $Q$ will `balance' on the vertical and horizontal lines through the centroid.
$x-\ovln{x}$ and $y-\ovln{y}$ are the `signed' distances from points in $Q$ to these lines.]

\msk

This in turn implies that $Q$ will balance on \ubr{any} line through the centroid. Seeing
this amounts to working out what the signed distance is from a point in $Q$ to a line
passing through the centroid, and noting that the integral of that distance, over $Q$, 
is \ubr{also} $0$. Such a line can be expressed, parametrically, as 
$L(t)=(\ovln{x},\ovln{y})+t(a,b)$, where, for convenience, we will assume that
$a^2+b^2=1$ (i.e., our direction vector has length $1$). Finding the signed distance 
we need amounts to writing the vector $\vec{v}=(x-\ovln{x},y-\ovln{y})$ as a linear combination
of the (orthonormal) vectors $\vec{w}_1=(a,b)$ and $\vec{w}_2=(b,-a)$, and taking the 
$(b,-a)$-coordinate (see figure). 

\msk

\leavevmode

\epsfxsize=3in
\ctln{\epsfbox{symm.ai}}

\msk

But \ubr{this} can be done by multiplying
$\vec{v}$ by the \ubr{inverse} of the matrix with columns $\vec{w}_1,\vec{w}_2$,
which, since these vectors are orthonormal, is the \ubr{transpose}. So, 
computing:

\msk

$\left(\matrix{a&b\cr b&-a\cr}\right)^T = \left(\matrix{a&b\cr b&-a\cr}\right)$, 
and $\left(\matrix{a&b\cr b&-a\cr}\right)\left(\matrix{x-\ovln{x}\cr y-\ovln{y}\cr}\right) =
\left(\matrix{a[x-\ovln{x}]+b[y-\ovln{y}]\cr b[x-\ovln{x}]-a[y-\ovln{y}]\cr}\right)$.

\ssk

So (check that!) $\left(\matrix{x-\ovln{x}\cr y-\ovln{y}\cr}\right) = 
(a[x-\ovln{x}]+b[y-\ovln{y}])\left(\matrix{a\cr b\cr}\right) 
+ (b[x-\ovln{x}]-a[y-\ovln{y}])\left(\matrix{b\cr -a\cr}\right)$, 

\ssk

making $b[x-\ovln{x}]-a[y-\ovln{y}]$ the signed distance from $(x,y)$ to the line. But then

\ssk

$\int\int_Q b[x-\ovln{x}]-a[y-\ovln{y}\ dA = b\int\int_Q x-\ovln{x}\ dA - a\int\int_Q y-\ovln{y}\ dA
=b(0)-a(0) = 0$, as desired.

\bsk

This enables us to establish our main result. Any rigid motion $R$ carries lines to lines, and so it 
carries a line $L$ through the centroid of $Q$ to some other line $L^\prime$. But because $R$ is an isometry, it 
preserves distance, and so the distance from a point $(x,y)\in Q$ to $L$ will be equal to 
the distance from $R(x,y)$ to $L^\prime$. [Ah, this is because an isometry will \ubr{also} preserve \ubr{angle};
this is because the Law of Cosines will let you compute (the cosine of) an angle using length measurements
alone: $c^2=a^2+b^2-2ab\cos(C)$ let's you compute $C$. Or: under an isometry the lengths in a triangle won't
change, so the angles won't, either. So the orthogonal projection of a point will be taken to 
the orthogonal projection.] [In addition, $R$ will either preserve all of the signs (``orientation-preserving'')
or reverse all of the signs (``orientation-reversing'').]
This in turn means that if you integrate the distance to $L^\prime$
over the region $Q$, you will still get $0$; it is the same as integrating the distance to $L$, by thinking 
of the isometry $R$ as a change of variables function taking the region $Q$ to the region $Q$ (!).

\msk

But this means that $L^\prime$ \ubr{is} a line through the centroid! Some parallel line $L^{\prime\prime}$
must pass through the
centroid, but our integral for $L^{\prime\prime}$ will differ from the integral for $L^\prime$ by
the distance between the lines times the area of $Q$ (since at every point in $Q$ the two integrands
will differ by the distance betweeen the lines). The only way for the integral for $L^{\prime\prime}$ to give $0$, 
therefore, is to have $L^\prime=L^{\prime\prime}$. So our symmetry $R$ must carry a line
through the centroid of $Q$ to a line through the centroid.

\msk

But! Now take \ubr{two} lines through the centroid (which determine the centroid!) $L_1,L_2$; 
so $L_1\cap L_2 =(\ovln{x},\ovln{y})$. Then $R(L_1)$ and $R(L_2)$ are both lines through the centroid,
so $\{(\ovln{x},\ovln{y})\}\subseteq R(L_1)\cap R(L_2)=R(L_1\cap L_2)=R(\ovln{x},\ovln{y})$. 
So $R(\ovln{x},\ovln{y}) = (\ovln{x},\ovln{y})$, and so $R$ fixes the centroid of $Q$.

\bsk

So by employing double integrals to compute centroids, the (orthogonal) projection from a point to a line, 
the change of basis formula from linear algebra, the Law of Cosines, and a change of variables
for double integrals, we achieve our result. This all generalizes to higher dimensions; lines
(and distances to them) are replaced, in $\bbr^n$, with $(n-1)$-dimensional subspaces (and
distances to them), and you
then need to use $n$ `hyperplanes' to determine the centroid, but most
everything else stays the same...


\vfill
\end
