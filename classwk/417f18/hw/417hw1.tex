
\documentclass[12pt]{article}
\usepackage{amsfonts}
\usepackage{amssymb}
\usepackage{dutchcal}

\textheight=10in
\textwidth=6.5in
\voffset=-1in
\hoffset=-1in

\begin{document}

\def\ctln{\centerline}
\def\msk{\medskip}
\def\bsk{\bigskip}
\def\ssk{\smallskip}
\def\hsk{\hskip.3in}
\def\ra{\rightarrow}
\def\ubr{\underbar}

\def\mt{{\mathcal T}}
\def\mb{{\mathcal B}}
\def\ms{{\mathcal S}}
\def\mu{{\mathcal U}}
\def\mv{{\mathcal V}}

\def\bbr{{\mathbb R}}
\def\bbz{{\mathbb Z}}
\def\spc{$~$\hskip.15in$~$}

\def\sset{\subseteq}
\def\del{\partial}
\def\lra{$\Leftrightarrow$}
\def\bra{$\Rightarrow$}



%%\UseAMSsymbols

\ctln{\bf Math 417 Problem Set 1}

\msk

Starred (*) problems are due Friday, Augsut 31.


\begin{description}


\item{(*) 1.} Some linear algebra(!) shows that that rotation $R(\theta)$ by angle $\theta$ 
around the origin, 
and reflection $M(\theta)$ in the line through the origin making angle $\theta$,
are linear transformations, given in matrix terms as multiplication by 

\ssk

\ctln{
$R(\theta) = \left( \begin{array}{cc} \cos(\theta) & -\sin(\theta) \\ \sin(\theta) & \cos(\theta) \end{array} \right)$ 
and
$M(\theta) = \left( \begin{array}{cc} \cos(2\theta) & \sin(2\theta) \\ \sin(2\theta) & -\cos(2\theta) \end{array} \right)$
}

\ssk

\item{\spc} Using matrix multiplication, show that $M(\theta)\circ M(\psi)$ is a rotation, and 
$M(\theta)\circ R(\psi)$ and $R(\psi)\circ M(\theta)$ are both reflections, 
and determine which angle they rotate or reflect by.

\ssk

\item{\spc} [You can show yourself that the matrices are correct, since their columns are the vectors that
$R(\theta)$ and $M(\theta)$ send the `standard' basis vectors of $\mathbb{R}^2$ to.]

\msk

\item{(*) 2.} (Gallian, p.38, \#14) If we build a rhombus $R$ (a quadrilateral
with all four sides having equal length) by gluing two equilateral
triangles together along a pair of sides, describe the symmetries
of $R$ in terms of rotations and reflections.

\msk

\item{3.} Describe the symmetries of a (right circular) cylinder (of height $h$ and radius $r$).

\msk

\item{4.} Show that the set $G=\{1,5,9,13\}$ forms a group, with group 
multiplication
being multiplication modulo 16. (One approach: build the `Cayley' table!
This helps to see why some needed properties hold.)

\msk

\item{(*) 5.} (Gallian, p.55, \#18) Which elements $x\in D_4$ = the group
of symmetries of a regular 4-gon (i.e., square) satisfy $x^2=e$ ?
Which satisfy $x=y^2$ for some $y\in D_4$ ? 

\item{\spc} [Problem \#1 can help you decide what an element $y^2$
can look like...]

\msk

\item{6.} (Gallian, p.57, \#48) Show that the collection of all $3\times 3$
matrices 

\ssk

\ctln{$\left( \begin{array}{ccc} 1 & a & b \\ 0 & 1 & c \\ 0 & 0 & 1 \end{array} \right)$}

\ssk

\item{\spc} with $a,b,c\in \bbr$ forms a group under matrix multiplication.

\ssk

\item{\spc} [This group is known as the {\it Heisenberg group}, and arises in the study of
one-dimensional quantum systems. You may find your row reduction prowess useful
in finding inverses!]

\msk


\end{description}
\vfill

\end{document}


