
\documentclass[12pt]{article}
\usepackage{amsfonts}
\usepackage{amssymb}
\usepackage{dutchcal}

\textheight=10in
\textwidth=6.5in
\voffset=-1in
\hoffset=-1in

\begin{document}

\def\ctln{\centerline}
\def\msk{\medskip}
\def\bsk{\bigskip}
\def\ssk{\smallskip}
\def\hsk{\hskip.3in}
\def\ra{\rightarrow}
\def\ubr{\underbar}

\def\mt{{\mathcal T}}
\def\mb{{\mathcal B}}
\def\ms{{\mathcal S}}
\def\mu{{\mathcal U}}
\def\mv{{\mathcal V}}

\def\bbr{{\mathbb R}}
\def\bbz{{\mathbb Z}}
\def\spc{$~$\hskip.15in$~$}

\def\sset{\subseteq}
\def\del{\partial}
\def\lra{$\Leftrightarrow$}
\def\bra{$\Rightarrow$}



%%\UseAMSsymbols

\ctln{\bf Math 417 Problem Set 6}

\msk

Starred (*) problems are due Friday, October 5.


\begin{description}

\item{35.} Show that the alternating group $A_8\leq S_8$ contains
an element of order $15$. Does it have an element of order $14$ ? Of order $16$ ?

\msk

\item{(*) 36.} Show that every element of $S_n$ can be written 
as a product of transpositions of the form $(1,k)$ for $2\leq k\leq n$. (Assume that
$n>1$ so that you don't have to worry about the philosophical challenges of $S_1=\{()\}$...)

\msk

\item{\spc} [Hint: why is it enough to show that this is true for transpositions?]

\msk

\item{37.} (Gallian, p.115, \#46) Show that in the symmetric group $S_7$,
there is \underbar{no} element $x\in S_7$ so that $x^2=(1,2,3,4)$. On
the other hand, find two distinct elements $y\in S_7$ so that 
$y^3=(1,2,3,4)$ .

\msk

\item{(*) 38.} Show that the function $f:\bbr\rightarrow\bbr$ given by $f(x)=e^x$, thought of as a function
frm the group $(\bbr,+,0)$ of real numbers under addition to the group $(\bbr^+,*,1)$ of positive real 
numbers under multiplication, is an isomorphism of groups.  

\msk

\item{39.} Suppose that $G$ is a {\it dihedral group} (i.e., a group 
of symmetries of some regular $n$-gon), and define the function $\varphi:G\rightarrow H = (\{-1,1\},*,1)$
to the group $H$ (isomorphic to $\bbz_2$) by $\varphi(\textrm{rotation})=1$ 
and $\varphi(\textrm{reflection})=-1$. Show that $\varphi$ is a homomorphism. 

\msk

\item{\spc} [Hint: Problem Set \# 1 !]

\msk

\item{40.} Show that if $G_1,G_2$ are groups, $H_1\leq G_1$ is a subgroup of $G_1$, and 
$\varphi:G_1\rightarrow G_2$ is a homomorphism, then $H_2=\{\varphi(h)\ :\ h\in H_1\}$
(the {\it image} of $H_1$) is a subgroup of $G_2$.

\msk

\item{41.} Show that the function $\varphi:\bbz_{16}^*\rightarrow \bbz_{16}^*$
given by $\varphi(a) = a^3$ is an isomorphism. What about $a\mapsto a^5$ ? Or $a\mapsto a^7$ ?

\msk

\item{(*) 42.} (Gallian, p.133, \# 32) Suppose that $\varphi:(\bbz_{50},+,0)\rightarrow(\bbz_{50},+,0)$
is an isomorphism and $\varphi(7) = 13$. Show that, for all $x$, $\varphi(x)=kx$ for a certain $k$, 
and find $k$ !


\end{description}
\vfill

\end{document}


