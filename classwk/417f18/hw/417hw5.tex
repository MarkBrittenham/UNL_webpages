
\documentclass[12pt]{article}
\usepackage{amsfonts}
\usepackage{amssymb}
\usepackage{dutchcal}

\textheight=10in
\textwidth=6.5in
\voffset=-1in
\hoffset=-1in

\begin{document}

\def\ctln{\centerline}
\def\msk{\medskip}
\def\bsk{\bigskip}
\def\ssk{\smallskip}
\def\hsk{\hskip.3in}
\def\ra{\rightarrow}
\def\ubr{\underbar}

\def\mt{{\mathcal T}}
\def\mb{{\mathcal B}}
\def\ms{{\mathcal S}}
\def\mu{{\mathcal U}}
\def\mv{{\mathcal V}}

\def\bbr{{\mathbb R}}
\def\bbz{{\mathbb Z}}
\def\spc{$~$\hskip.15in$~$}

\def\sset{\subseteq}
\def\del{\partial}
\def\lra{$\Leftrightarrow$}
\def\bra{$\Rightarrow$}



%%\UseAMSsymbols

\ctln{\bf Math 417 Problem Set 5}

\msk

Starred (*) problems are due Friday, September 28.


\begin{description}

\item{(*) 28.} (Gallian, p.88, \#24, sort of) Show that if $G$ is a group with $a,b\in G$
and $ab=ba$, then $\langle b\rangle\leq C_G(a)$ = the centralizer of $a$ in $G$.

\msk

\item{29.} Show that if $G$ is a group, $A,b\in G$ and $ab\in Z(G)$ [the center of $G$],
then $ab=ba$ (i.e., $a$ and $b$ commute).

\msk

\item{(*) 30.} (Gallian, p.86, \#17) If $a\in G$ and $|a|<\infty$, then 
complete the following statement: 

\ssk

\item{\spc} ``$|a^2| = |a^{12}|$ if and only if $\_\_\_\_\_\_\_\_\_\_$ .''

\ssk

\item{\spc} Explain why your statement is true.

\msk

\item{31.} (Gallian, p.87, \#14) Suppose that $G$ is a \ubr{cyclic} group that has exactly 
three subgroups: $G$, $\{e\}$, and a subgroup of order $7$. What is $|G|$ ?
Is there anything special about the number $7$ ?

\msk


\item{32.} (Gallian, p.112, \#3) Write each of the following permutations as a product of disjoint
cycles:

\item{\spc} (a) $(1\ 2\ 3\ 5)(4\ 1\ 3)$

\item{\spc} (b) $(1\ 3\ 2\ 5\ 6)(2\ 3)(4\ 6\ 5\ 1\ 2)$

\item{\spc} (c) $(1 2)(1\ 3)(2\ 3)(1\ 4\ 2)$

\msk

\item{33.} (Gallian, p.114, \#32) If $\beta=(1\ 2\ 3)(1\ 4\ 5)$, express $\beta^{99}$ as a 
product of disjoint cycles.

\msk

\item{(*) 34.} Show that if $\alpha\in S_n$ has $|\alpha|$ \underbar{odd}, then $\alpha$ is
an \underbar{even} permutation!

\ssk

\item{\spc} [Hint: Imagine that you have expressed $\alpha$ as a product of disjoint cycles...]


\end{description}
\vfill

\end{document}


