
\documentclass[12pt]{article}
\usepackage{amsfonts}
\usepackage{amssymb}
\usepackage{dutchcal}

\textheight=10in
\textwidth=6.5in
\voffset=-1in
\hoffset=-1in

\begin{document}

\def\ctln{\centerline}
\def\msk{\medskip}
\def\bsk{\bigskip}
\def\ssk{\smallskip}
\def\hsk{\hskip.3in}
\def\ra{\rightarrow}
\def\ubr{\underbar}

\def\mt{{\mathcal T}}
\def\mb{{\mathcal B}}
\def\ms{{\mathcal S}}
\def\mu{{\mathcal U}}
\def\mv{{\mathcal V}}

\def\bbr{{\mathbb R}}
\def\bbz{{\mathbb Z}}
\def\spc{$~$\hskip.15in$~$}

\def\sset{\subseteq}
\def\del{\partial}
\def\lra{$\Leftrightarrow$}
\def\bra{$\Rightarrow$}



%%\UseAMSsymbols

\ctln{\bf Math 417 Problem Set 2}

\msk

Starred (*) problems are due Friday, September 7.


\begin{description}


\item{7.} (Gallian, p.56, \#31) Show that for any group $G$, its
`Cayley' table is a {\it Latin square}: every group element appears
exactly once in each row and column of the table.

\msk

\item{8.} (Gallian, p.24, \#19) Show that $\textrm{gcd}(n,ab)=1$ 

\item{\spc} \hskip2in if and only
if $\textrm{gcd}(n,a)=1$ and $\textrm{gcd}(n,b)=1$.

\ssk

\item{\spc} [This is what `makes' $\bbz_n^*$ a group under multiplication;
the product of two numbers relatively prime to $n$ is a number relatively prime to $n$.]

\msk

\item{(*) 9.} Use the Euclidean algorithm to find the inverses of the 
elements $2$, $5$, and $7$ in the group $G=(\bbz_{141}^*,\cdot,1)$.

\msk

\item{(*) 10.} Find the inverse of the element
$\left( \begin{array}{cc} 2 & 6 \\ 3 & 5 \end{array} \right)$ in
$GL_2(\bbz_{11})$. 

\msk

\item{11.} (Gallian, p.57, \#42) Suppose that $F_1=M(\theta)$
and $F_2=M(\psi)$ (in Gallian's/our notation) are reflections 
in lines through the origin of slope $\theta$ and $\psi$, with $\theta\neq\psi$, and 
$F_1\circ F_2 = F_2\circ F_1$. Show that then 
$F_1\circ F_2 = R(\pi)$ is rotation by angle $\pi$.

\ssk

\item{\spc} [Your results from Problem \#1 might help!]

\msk

\item{(*) 12.} (Gallian, p.57, \#34) Prove that if $G$ is a group and
$a,b\in G$ then $(ab)^2=a^2b^2$ if and only if $ab=ba$ .

\msk

\item{13.} (Gallian, p.58, \#47) Suppose that $G$ is a group and, for every $x\in G$,
we have $x^2=e$. Show that for every $a,b\in G$ we have $ab=ba$ (that is, $G$ is abelian!).

\end{description}
\vfill

\end{document}


