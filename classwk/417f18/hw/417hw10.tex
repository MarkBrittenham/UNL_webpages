
\documentclass[12pt]{article}
\usepackage{amsfonts}
\usepackage{amssymb}
\usepackage{dutchcal}

\textheight=10in
\textwidth=6.5in
\voffset=-1in
\hoffset=-1in

\begin{document}

\def\ctln{\centerline}
\def\msk{\medskip}
\def\bsk{\bigskip}
\def\ssk{\smallskip}
\def\hsk{\hskip.3in}
\def\ra{\rightarrow}
\def\ubr{\underbar}

\def\mt{{\mathcal T}}
\def\mb{{\mathcal B}}
\def\ms{{\mathcal S}}
\def\mu{{\mathcal U}}
\def\mv{{\mathcal V}}

\def\bbr{{\mathbb R}}
\def\bbz{{\mathbb Z}}
\def\bbq{{\mathbb Q}}
\def\spc{$~$\hskip.15in$~$}

\def\sset{\subseteq}
\def\del{\partial}
\def\lra{$\Leftrightarrow$}
\def\bra{$\Rightarrow$}



%%\UseAMSsymbols

\ctln{\bf Math 417 Problem Set 10}

\msk

Starred (*) problems are due Friday, November 16.


\begin{description}

\item{66.} (Gallian, p209, \# 51) Let $N$ be a normal subgroup of a group $G$. 
Show that every subgroup $K$ of $G/N$ has the form
$H/N$, where $H$ is a subgroup of $G$. [Hint: Think about the homomorphism
$\varphi:G\ra G/N$ .

\msk

\item{(*) 67.} (Gallian, p.191, \# ) If $G$ is a group, $H\lhd G$ is a normal subgroup, and 
$K\leq G$ is a subgroup, then $HK=\{hk\ :\ h\in H, k\in K\}$ is a subgroup of $G$.
(See Example 5 on p.175 for an explanation why.) Show that if, in addition, $K$ is a \ubr{normal}
subgroup of $G$, then $HK$ is a normal subgroup.

\msk

\item{68.} (Gallian, p.168, \# 17) Show that if
$G\oplus H$ is a cyclic group, then $G$ and $H$ are
both cyclic. [Hint: A group \ubr{isomorphic}
to a cyclic group is cyclic!]

\msk

\item{69.} (Gallian, p.170, \# 59) Let $p$ be a prime. Prove that $\bbz_p\oplus \bbz_p$ has exactly $p+1$
distinct subgroups of order $p$.

\msk

\item{(*) 70.} Show that $2$ is \ubr{not} a generator for the group $\bbz_{31}^*$ of units 
modulo $31$, but that $3$ \ubr{is}. If, using $\bbz_{31}^*$ and $a=3$ as the basis for a 
(very weak!) Diffie-Hellman key exchange, if Alice chooses $n=5$ and Bob chooses $m=11$ 
to build a shared key, what information do they send to one another and what is that key?

\msk

\item{71.} In the group $S_{10}$ the elements $a=(1,2,3)(4,5)(8,9)$ and $b=(2,4,8)(1,10)(3,7)$
are conjugate. Find at least two distinct conjugating elements $x$ (so that
$xa=bx$).

\msk

\item{72.} Find a matrix $X\in GL(2,\bbz)$ so that 
$X\left(\begin{array}{cc} 2 & 1 \\ 1 & 1 \end{array}\right) = \left(\begin{array}{cc} 2 & -1 \\ -1 & 1 \end{array}\right)X$ .

\msk

\item{(*) 73.} Find a matrix $\left(\begin{array}{ccc} a & b \\ c & d \end{array}\right)\in GL(2,\bbz_7)$ so that 

\ssk


\ctln{$\left(\begin{array}{ccc} a & b \\ c & d \end{array}\right)\left(\begin{array}{ccc} 2 & 3 \\ 4 & 5 \end{array}\right) = 
\left(\begin{array}{ccc} 3 & 4 \\ 5 & 4 \end{array}\right)\left(\begin{array}{ccc} a & b \\ c & d \end{array}\right)$}

\ssk

\item{\spc} in the group $GL(2,\bbz_7)$ .

\ssk

\item{\spc} [Note that we can multiply $a,b,c$, and $d$, in a solution, by $u\in\bbz_7^*$, and still have a solution. This allows you to 
\ubr{assume} that, for example, either $a=0$ or $a=1$ . This can lower your work factor....]


\end{description}
\vfill

\end{document}

\left(\begin{array}{ccc} a & b \\ c & d \end{array}\right)
