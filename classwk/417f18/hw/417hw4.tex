
\documentclass[12pt]{article}
\usepackage{amsfonts}
\usepackage{amssymb}
\usepackage{dutchcal}

\textheight=10in
\textwidth=6.5in
\voffset=-1in
\hoffset=-1in

\begin{document}

\def\ctln{\centerline}
\def\msk{\medskip}
\def\bsk{\bigskip}
\def\ssk{\smallskip}
\def\hsk{\hskip.3in}
\def\ra{\rightarrow}
\def\ubr{\underbar}

\def\mt{{\mathcal T}}
\def\mb{{\mathcal B}}
\def\ms{{\mathcal S}}
\def\mu{{\mathcal U}}
\def\mv{{\mathcal V}}

\def\bbr{{\mathbb R}}
\def\bbz{{\mathbb Z}}
\def\spc{$~$\hskip.15in$~$}

\def\sset{\subseteq}
\def\del{\partial}
\def\lra{$\Leftrightarrow$}
\def\bra{$\Rightarrow$}



%%\UseAMSsymbols

\ctln{\bf Math 417 Problem Set 4}

\msk

Starred (*) problems are due Friday, September 21.


\begin{description}


\item{21.} If $G$ is a group,
and $H\subseteq G$ is a subset of $G$ so that, whenever
$a,b\in H$ we have $a^{-1}b^{-1}\in H$, is this enough
to guarantee that $H$ is a subgroup of $G$? If yes, 
explain why! If not, give an example which shows that
it doesn't work.

\msk

[Hint: if $a\in H$, start listing other elements that you can
\ubr{guarantee} are in $H$ ...]

\msk

\item{22.} (Gallian, p.70, \#34) Show that if $G$ is a 
group and $H,K\subseteq G$ are subgroups of $G$, then 
their intersection $H\cap K$ is also a subgroup of $G$.
Does this extend to the intersection of \underbar{any}
number of subgroups of $G$ ?

\msk

\item{(*) 23.} (Gallian, p.71, \#46) Suppose that $G$ is a group
and $g\in G$ has $|g|=5$. Show that the centralizer of $g$,
$C(g)=C_G(g)=\{x\in G\ :\ xg=gx\}$, is equal to the
centralizer of $g^3$, $C_G(g^3)$.

\ssk

[Hint: show that anything that commutes with $g$ must commute 
with $g^3$, \ubr{and} \ubr{vice} \ubr{versa}! What, if anything,
is special about the numbers $5$ and $3$ in this problem?]

\msk

\item{24.} (Gallian, p.73, \#66) Let $G = GL_2(\bbr)$ = the
$2\times 2$ invertible matrices, under matrix multiplication, and
let $H=\{A\in GL_2(\bbr)\ :\ \textrm{det}(A) = 2^k\ \textrm{for some}\ k\in\bbz\}$.
Show that $H$ is a subgroup of $G$.

\msk

\item{(*) 25.} If $G$ is an \ubr{abelian} group and $n\in\bbz$, show that 
$H_n = \{g\in G\ :\ g=x^n\ \textrm{for some}\ x\in G\}$ (i.e., the set
of $n$-th powers of elements of $G$) is a subgroup
of $G$. Give an example where this \ubr{fails} if $G$ is \ubr{not} abelian.
\msk

\item{(*) 26.} (Gallian, p.72, \#53) Consider the element 
$A = \left( \begin{array}{cc} 1 & 1 \\ 0 & 1 \end{array} \right)\in SL(2,\bbz)$
What is the order of $A$ ? If we instead view 
$A = \left( \begin{array}{cc} 1 & 1 \\ 0 & 1 \end{array} \right)\in SL(2,\bbz_n)$
for an integer $n\geq 2$, what is the order of $A$ ?

\msk

\item{27.} (Gallian, p.86, \#15) Let $G$ be an abelian group and let 
$H = \{g \in G\ :\ |g|$ divides $12\}$.
Prove that $H$ is a subgroup of $G$. Is there anything special about $12$
here? Would your proof be valid if $12$ were replaced by some other
positive integer? Why or why not?


\end{description}
\vfill

\end{document}


