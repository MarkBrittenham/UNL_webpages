
\magnification=1200
\def\ctln{\centerline}
\def\ssk{\smallskip}	
\def\bsk{\bigskip}
\def\cgg{\ForestGreen}
\def\cbb{\Blue}
\def\cbr{\BrickRed}

\input epsf.tex

\input colordvi

Yamada: turning a projection for a knot/link into a braid, without changing the number of Seifert circles:

\ssk

The idea: if the Seifert circles are all parallel, forming a ``bullseye'', then the center of the bullseye is an axis 
around which the knot/link runs, making the knot/link a braid. We want to find a projection having this
property, without changing the number of Seifert circles for the projection.

\ssk

Starting with the knot/link, find the Seifert circles. \cbr{(Red segments represent the twisted bands.)}
The goal: find a new projection with fewer distinct families of parallel circles among the Seifert circles.

\bsk

\ctln{\epsfxsize=4in
\epsfbox{yamada_pic1.ai}}

\bsk

The idea: find a an \cgg{arc, missing the circles and the segments,} which joins together two different families
of parallel circles having the same (clockwise or counter-clockwise) orientation.

\vfill\eject

Use this arc to \cbb{extend the two sets of Seifert disks} one over/under one another, to make their boundaries
become a single set of parallel circles.

\bsk

\ctln{\epsfxsize=5in
\epsfbox{yamada_pic2.ai}}

\bsk

Problem: our original surface is {\bf not} the one that Seifert's algorithm would build for the resulting new projection! 
\cbb{[The new circles pass over/under some of our old twisted bands.]} But if we run Seifert's algorithm on the new
projection, we {\bf do} get the same Seifert circles! We just have ( a lot) more twisted bands. Basically, this is because the
added \cbb{blue} arcs are oriented in the same way as the boundaries of the Seifert disks they are running parallel to.

\vfill\eject



\ctln{\epsfxsize=5in
\epsfbox{yamada_pic3.ai}}

\bsk

Finding the arc to amalgamate disks along: if the Seifert circles are {\bf not} all parallel, then there is a complementary 
region of the Seifert circles which is neither a disk nor an annulus (i.e., we see at least three distinct Seifert circles while 
standing in the middle). At most one of the associated Seifert disks $D$ may contain all of the others (i.e., all others
are nested within it). If we choose the opposite disk (on the projection 2-sphere) for the circle corresponding to $D$,
then none of the disks are nested. [This switch of Seifert disk can be thought of as  choosing a different point at infinity for the
2-sphere, to put the projection in the plane, which does not affect the Seifert circles.] When the disks are not nested, a twisted band 
between them implies that the orientations of their boundaries are opposite. Since we see at least three disks, two of them
must have the same orientation; we choose these to join by our \cgg{green arc}.

\bsk

\ctln{\epsfxsize=2in
\epsfbox{yamada_pic4.ai}}


\vfill\end


