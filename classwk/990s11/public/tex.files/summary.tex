


\input amstex
\magnification=1100

\vsize=8.9in
\voffset=-.35in
\loadmsbm

\input epsf

\define\dl{\displaystyle}
\define\ctln{\centerline}
\define\ssk{\smallskip}
\define\msk{\medskip}
\define\bsk{\bigskip}
\define\bbr{{\Bbb R}}
\define\bbz{{\Bbb Z}}
\define\bbc{{\Bbb C}}
\define\ii{{\italic i}}
\define\ubr{\underbar}
\define\lra{$\Leftrightarrow$}
\define\ra{\rightarrow}
%\define\ssk{\vskip-10pt}
\define\Arg{\text{Arg}}
\define\Log{\text{Log}}
\define\Arcsin{\text{Arcsin}}
\define\del{\partial}

\overfullrule=0pt
\parindent=0pt
%\parskip=10pt

%\nopagenumbers

\ctln{\bf Knot Theory from the Combinatorial Point of View}

\msk

\ctln{Highlights of ideas/topics/results}

\bsk

Knot theory is essentially about the simplest non-trivial
instance of the embedding problem.

\ssk

$S^1\hookrightarrow \bbr^2$ : Jordan curve theorem says all are isotopic (all bound disks)

$[0,1]\hookrightarrow \bbr^n$ : draw embedding back along itself to (near) $0$

$S^1\hookrightarrow \bbr^3$ (or $S^3$) : interesting things happen!

\ssk

Avoid `wild' embeddings: embedding is \ubr{smooth} or \ubr{polygonal}. Two knots are the `same' if
they are isotopic = homotopic through embeddings.

\ssk

Every knot has (many) planar projections = 4-valent graph indicating which strand passes
over/under the other at each vertex.

{\it Knot diagrammatics.} Reidemeister moves. Two diagrams of the same knot/link can be turned on into 
another by a finite sequence of three moves: R1 = remove a kink, R2= remove a pair of cancelling 
crossings, R3 = pass a strand over/under a crossing.

\ssk

As a result, knot theory can be treated as the study of properties of knot/link diagrams
that remain invariant under Reidemeister moves.

\msk

{\it Crossing changes.} Given a knot/link diagram $D$, by changing at most half of the crossings
(start at a point, and retrace the knot always going under what you have already drawn),
$D$ can be transformed into the diagram of an unknot/unlink.

\ssk

Alternatively (no pun intended), we can change crossings in $D$ so that the crossings
alternate over/under. This is essentially beacause the dual of the underlying 4-valent
graph can be 2-colored red/blue: looking out from a red vertex, change crossings in $D$ so that
the knot rises while moving right. 

\msk

{\it Knot invariants.} Goal: construct functions $I:\{$knots$\}\rightarrow\{$something$\}$ so that 
$K_1=K_2$ implies $I(K_1)=I(K_2)$. Then: if $I(K_1)\neq I(K_2)$, then $K_1$ and $K_2$ are 
different knots!

\ssk

Lots of knot invariants can be defined via diagrammatics, often as a minimum:

crossing number = $c(K)$ = minimum number of crossings over all diagrams of $K$

unknotting number = $u(K)$ = minimum number of crossing changes needed to change $K$ to unknot

bridge number = $br(K)$ = minimum number of maxima in a projection $K\hookrightarrow \bbr^3 \rightarrow \bbr$

\ssk

Basic idea: build values from knot diagrams that are invariant under Reidemeister moves.

\ssk

Motivating goal: build computable knot invariants that shed light (= bound, and/or often compute!) invariants
like $c(K),u(K),br(K)$.

\ssk

{\it $n$-colorability.} A knot diagram $D$ consists of a collection of unbroken strands running
from undercrossing to undercorssing. If $D$ has $k$ crossings, it has $k$ strands. An $n$-coloring
of $D$ is a (non-constant) function from the strands of $D$ to $\bbz_n$ so that, at each crossing, with
the $i$-th strand crossing over (and the $i$-th and $k$-th ending), we have $a_i+a_i=a_j+a_k$ mod $n$.
$n$-colorability is invariant under R-moves, so defines a ($2$-valued `yes/no') knot invariant.

\ssk

These equations can be assembled into a matrix $M$ with entries $-1,0,2$; a knot is $p$-colorable ($p$ prime) iff
$p$ divides the determinant of (any) $(k-1)\times(k-1)$ minor of this matrix. We call the absolute value of
this determinant
the {\it determinant of the knot $K$}; it is itself invariant under R-moves, so is a knot invariant.
The number of diagonal entries divisible by $p$ = the dimension of the solution space of $p$-colorings
is also an invariant of $K$.
$\det(K)$ can be computed by putting the matrix into Smith normal form, by doing row and column
operations over $\bbz$ on $M$. The resulting diagonal matrix will have a $0$ on the diagonal. 

\ssk

$K$ is 2-colorable \lra\ $K$ has more than one component. (So for a knot, $\det(K)$ is odd.)

For $K$ an alternating knot (with alternating diagram $D$), 
$\det(K)$ = the number of maximal trees in the `checkerboard' graph $\Gamma$ for $D$: 2-color
the dual to $D$, pick a color to use as vertices, and add an edge b/w vertices if the two
complementary regions of $D$ the represent share a corner. This can make for much quicker
computations of $\det(K)$, e.g., for an alternating pretzel knot $K_{p,q,r}$, $p,q,r>0$, 
$\det(K_{p,q,r})=pq+pr+qr$.

\ssk

We showed the $\det(K)$ was a knot invariant by actually showing that it was $\Delta_K(-1)$ for another
invariant, the Alexander polynomial $\Delta_K(t)$. If we orient $K$, then the crossings fall into two
types, right-handed (positive) and left-handed (negative), distinguished, putting the arrows at the top,
by the overstrand (SW-NE for $+$, SE-NW for $-$). Then we can refine the matrix $M$ by using the entries
in the figure below, yielding the {\it Alexander matrix}:


\ssk

\ctln{\epsfxsize=1.5in
\epsfbox{alex.ai}}

\ssk

The determinant of (any) $(k-1)\times(k-1)$ minor is a polynomial; it is invariant, up to 
multiples of $\pm t^n$, under R-moves, and is called the {\it Alexander polynomial} $\Delta_K(t)$ 
of the oriented knot/link $K$. It is a remarkably good invariant, capable of distinguishing
all but a few pairs of knots in the standard (through 10 crossings) knot tables. Reversing the
orientation on \ubr{all} components has the effect of replacing $t$ with $t^{-1}$.


\msk

{\it Alternate views.} $n$-colorings assign elements of $\bbz_n$ to strands of the knot diagram.
This is really a group presentation in disguise: $\pi_1(S^3\setminus K)$ has a presentation, the 
Wirtinger presentation, with generators $\leftrightarrow$ strands of $K$ and relations $\leftrightarrow$
crossings of $K$. Orient $K$, and draw an arrow (= generator) under each strand going right to left. At each
crossing, the arrows match in pairs head-to-tail; setting the two length-two words (reading along the
arrows) equal gives the relation corresponding to the crossing. 

\ssk

A presentation is good for building homomorphisms \ubr{out} of a group; we assign values to the generators
and check to see if those assignments turn the relators into true statements in the target group.
We have an assignment to the gens of $\pi_1(S^3\setminus K)$ in $\bbz_n$, but the relations don't hold.
But if we instead send the gen to the symmetry of the regular $n$-gon that is reflection
in the line through the center that passes through the (cyclically ordered) vertex labeled
by our element of $\bbz_n$, then the relation \ubr{is} satisfied. So $p$-colorings correspond to 
(surjective) homoms $\pi_1(S^3\setminus K)\rightarrow D_{2p}$ (the dihedral group) 
sending Wirtinger generators to reflections.

\ssk

The Alexander polynomial has its own interpretation in terms of $G=\pi_1(S^3\setminus K)$. $G$
has abelianization $\bbz^{|K|}$, $|K|$ = the number of components of $K$, and an orientation on
$K$ then determines a more or less canonical map $G\rightarrow \bbz$. The kernel of this
map is a subgroup $H\subseteq G$, which by Galois correspondence corresponds to a covering space 
$X$ of $S^3\setminus K$. $G/H=\bbz$ acts on this space (by deck transformations) and so acts on 
$H_1(X)$, making this group a $\bbz[\bbz]$-module. (In group-theoretic terms, $\bbz=G_{ab}$ = 
the abelianization of $G$, so $H=[G,G]=G^\prime$ = the commutator subgroup, so 
$H_1(X)=[G,G]_{ab}=G^\prime/G^{\prime\prime}$.) This module happens to be finitely-generated, 
so has a presentation matrix over $\bbz[\bbz]$ = the Laurent polynomial ring in 
a variable $t$. The determinant
of this presentation matrix \ubr{is} the Alexander polynomial (up to multiplication by $\pm t^n$).
This gives an answer to a basic question: When is $\Delta_K(t)=1$? Ans: when this matrix is
invertible, i.e., when $G^\prime = G^{\prime\prime}$, i.e., when $G^\prime$ is `perfect'.
(Unfortunately, there are non-trivial knots, e.g., Kinoshita-Terasaka, for which this is true...)

\ssk

{\it Fox calculus.} The compuation of $\Delta_K(t)$ via the group can be carried out using Fox 
derivatives. Given a presentation, we take the derivative of a relator by accumulating terms in an 
(initially empty) sum as follows: reading $r_j$ from left
to right, each occurance of the generator $x_i$ occurs as
$r_j=ux_iv$ or $r_j=ux_i^{-1}v$. In the first case we add $u$, and in the 
second case we add $-ux_i^{-1}$. The resulting sum should be thought of
as an element in the group ring of the free group $F(X)$ generated by
the symbols $X$, and is denoted $\dl {{\del r_j}\over{\del x_i}}$. 
The canonical homomorphism $\psi$ from $F(X)$ to $G$ (sending each $x_i$ to $x_i$),
followed by our chosen homomomorphism $\phi$ from $G$ to $\bbz=<t|>$, induces ring
homomorphisms between their group rings; their composition sends each
$\dl {{\del r_j}\over{\del x_i}}$ to a (Laurent) polynomial in $t$ with
coefficients in $\bbz$. 
These polynomials, when assembled into a `Jacobian' matrix

\ssk

\ctln{$\dl J=\big({{\del r_j}\over{\del x_i}}\big)^{\psi\phi}$,}

\ssk

yields a matrix which depends 
upon the presentation for $G$ chosen. But any two (finite) presentations of $G$ can
be transformed one to the other by Teitze transformations, and the effect on 
the Jacobian matrix under these moves can be codified. In particular, the 
resulting moves on the Jacobian do not change (for J an $n\times m$ matrix)
the {\it ideal (in $\bbz[\bbz]$) generated by the 
$(m-k)\times(m-k)$ minor determinants} of $J$. Then taking the
ideal $I_1$ generated by the $(m-1)\times(m-1)$ minors, $I_1$ is principal,
and a generator for $I_1$ is the Alexander polynomial $\Delta_K(t)$ (well defined up
to multiplication by a unit $\pm t^n$ in $\bbz[\bbz]$).

\ssk

The essential point is that any presentation, and values of the generators which yield the
`canonical' homomorphisnm to $\bbz$, can be used to compute the Alexander polynomial.
For example, the Dehn presentation, where the generators correspond to the 
complementary regions of a diagram and the relators again come from the crossings (but 
yield difsferent relations!), can be used. Computing the Jacobian, we end up with 
a formulation which is identical to Alexander's original approach. 
The (p,q) torus knots have a presentation $<a,b | a^p=b^q>$, which yields
a very quick computation of their Alexander polynomials, and a proof that $K_{p,q}=K_{r,s}$
\lra\ $\{|p|,|q|\}=\{|r|,|s|\}$. 2-bridge knots have a presentation with 2 generators and
1 relator (by applying Teitze transformations to a Wirtinger presentation in `bridge
position'); the Fox calculus then yields a quick computation of $\Delta_K(t)$.

\ssk

Using Fox calculus anda  pair of `dual' presentations, Fox and Torres showed the most important property
of $\Delta_K(t)$: up to multiplication by $\pm t^n$, $\Delta_K(t)=\Delta_K(t^{-1})$ (originally
due to Seifert).

\msk

{\it Skein relations.} Together with $\Delta_K(1)=\pm 1$ (from the original matrix version), 
these facts allow us to
`normalize' the Alexander polynomial; multiplication by an appropriate power yields
$\Delta_K(1)=\pm 1$ and $\Delta_K(t)=\Delta_K(t^{-1})$, and so
$\Delta_K(t)=\nabla_K(t^{1/2}-t^{-1/2})=\nabla_K(z)$ for some polynomial $\nabla$.
$\nabla_K(z)$ = the (Alexander-)Conway polynomial. Conway anticipated (it took awhile 
for the world to appreciate) a new wave of knot invariants by showing that this
normalized polynomial satisfies a {\it skein relation}: if three (oriented) diagrams differ at
a single crossing, where we have a right-handed ($K_+$), left-handed ($K_-$) and no
crossing ($K_0$) [but the orientation is respected], then 
$\nabla_{K_+}(z)-\nabla_{K_-}(z)=z\nabla_{K_0}(z)$. 

\ssk

This identity was in fact anticipated in Alexander's original work (but the lack of normalization
made it difficult as a computational tool). The skein relation enables computation of Conway
(and Alexander - it is essentially the same polynomial) polynomials by induction, since crossing
changes lead to the unknot and crossing removals yield diagrams with fewer crossings! But more than
that, it leads to effective proofs of properties of these polynomials by induction, as well.
For example, every Laurent polynomial $p$ satisfying $p(1)=1$ and $p(t)=p(t^{-1}$ is the 
alexander polynomial of some knot $K$. Also, if $L$ is the mirror image of $K$, then 
$\nabla_L(z)=\nabla_K(-z)$.

\ssk

Still more: the Alexander polynomial is invariant under 
{\it mutation}. If you can find a loop (think: 2-sphere in space, 
a disk above and below the projection plane) in the projection plane meeting the knot $K$ in 
four points (separating the knot into two 2-tangles), then any rotation of one of the
tangles (i.e., rotate the 3-ball) which brings the 4 points back to themselves
without leaving any fixed yields a new knot (a {\it mutant} of $K$), which has the
same Alexander-Conway polynomial as $K$.

\msk

{\it The Jones revolution:} 1985 was a very big year for knot theory, and skein relations played
an important role. In 1984 Vaughan Jones introdues (via operator algebras) a new
Laurent polynomial knot invariant, $V_L(t)$. In the end, it can be defined by a skein relation;
$tV_{L-+}(t)-t^{-1}V_{L_-}(t)=(t^{1/2}-t^{-1/2}V_{L_0}(t)$ (together with the normalization
$V_{O}(t)=1$ for O=unknot). [Note: the literature contains many minor but equivalent (under some
change of variables) variations on this relation.] 

\ssk

Perhaps the most direct and combinatorial
way to see that such a thing defines a knot invariant is via the Kauffman bracket $<D>$,
which is defined as a `state sum'. Starting with an unoriented knot diagram $D$, the
vicinity of each crossing can be decorated the A- and B-'channels': with the overstrand running
SW-NE, the top and bottom are A's, and the two sides are B's. We now form a sum over all states
(= choice of A- and B- channel at each crossing), where a state yields the summand 
$A^\#(\text{A-chans})B^\#(\text{B-chans})C^{s(D)-1}$, where $s(D) = $ the number of loops
created when every crossing of $D$ is removed and the ends are glued to make an open channel
according to the state (`clear' the A- or B-channel at each crossing). This sum is $<D>$.

\ssk

Note that this could be thought of as an inductive computation as well: at a single crossing of
the diagram $D$,
$<D>=A<$clear A-crossing$>+B<$clear B-crossing$>$.
Checking the effect of the Reidemeister moves on this state sum reveals that R2 and R3 leave it
invariant, \ubr{provided} we set $B=A^{-1}$ and $C=-(A^2+A^{-2})$. But the R1 move
will always multiply $<D>$ by $(-A^3)^{\pm 1}$, depending upon which R1 move is done.
But by (temporarily!) orienting the knot and computing the {\it writhe} $W(D)$=\#(positive
crossings)$-$\#(negative crossings) [which is also invariant under R2 and R3], we find that 
$J_K(A)=(-A^3)^{-w(D)}<D>$ is invariant under all of the Reidemeister moves, and so defines 
a Laurent polynomial invariant.

\ssk

We can, by comparing $<D>$ when we change a positive crossing to a negative one (essentially,
interchange the A- and B-crossings), find a skein relation for $P$: 
$A^4J_{K_+}(A)-A^{-4}J_{K_-}(A)=(A^2-a^{-2})J_{K_0}(A)$, and $J_O(A)=1$. Comparison
of this relation with the skein relation for $V_K(t)$, and induction, yields that
$V_K(t)=J_K(t^{1/4})$. [Again, note that the literature contains many different
formulations of this (again, equivalent under change of variable), with
slightly different skein relations!]

\ssk

Again, inductive arguments using the skein relation yields interesting results.
If $L$ is the mirror image of $K$, then $J_L(A)=J_K(A^{-1}$. Unlike the Alexander 
polynomial, however, this may not equal $J_K(A)$, allowing the Jones polynomial 
to often distinguish a knot from its mirror image (e.g., the trefoil knot).
If you reverse \ubr{all} orientations of a link $L$, yields $-L$, the writhe is 
unchanged, so $J_{-L}(A)=J_L(A)$. Reversing the orientation on some components
only changes the writhe, and so $J_L(A)$ only changes by a power of $A$. [This is
in contradistinction with the Alexander polynomial, where a partial reversal of
orientation can have unexpected effects.] 

\ssk

A major open problem in knot theory is: does the Jones polynomial detect the unknot?
I.e., if $K$ is a knot and $J_K(A)=1$, is $K$ = unknot? Neither the Jones nor the Alexander
polynomials determine one another; there are knots the can be distinguished by
one but not the other.

\msk

The Jones polynomial distinguished itself early on by being able to shed light on
the crossing number of a knot. For any Laurent polynomial $p$, define the {\it span
of $p$} to be the difference between the highest index of a non-zero coefficient
and the lowest. Then for any link $L$, the span of $J_K(A)$ is $\leq 4c(K)$. But
Thistlethwiate, Murasugi, and Kauffman all showed that if $L$  is an alternating
link (i.e., possesses an alternating diagram), then this inequality is in fact
an equality. Moreover, if $L$ is {\it prime} (see below), and has no alternating
projection, then the inequality is strict.

\ssk

A knot is prime if it cannot be decomposed as a non-trivial connected sum. The 
connected sum $K_1\sharp K_2$ of two (oriented) knots $K_1,K_2$ 
is obtained by drawing diagrams of the
knots with two oppositely oriented strands side-by-side, and cutting and splicing
the two strands together, respecting the orientation. A connected sum is trivial if
one of the knots is the unknot (giving the other knot as the connected sum). Via
skein relations, we can show that the Alexander and Jones polynomials of a 
connected sum is the product of the polynomials of the summands. This and the 
result above helps to motivate the (still open) conjecture:
$c(K_1\sharp K_2)=c(K_1)+c(K_2)$.

\msk


{\it HOMFLY/HOMFLYPT/MYTHFLOP/LYMPH-TOFU.} Fast on the heels of the Jones polynomial
came a 2-variable generalization, which also generalizes the Alexander polynomial.
In the end it can be defined by a skein relation [again, there are many 
equivalent formulations] 

\ctln{$\ell P_{K_+}(\ell,m)+\ell^{-1}P_{K_-}(\ell,m)+m P_{K_0}(\ell,m)=0$}

Proving this is a well-defined invariant is a very delicate induction on 
the `distance' to a (well-decorated!) unlink, and showing that the result is
invariant under the extra decorations. The Alexander polynomial and Jones
polynomial can be recovered from $P_K$ by substituting values for $\ell$ and
$m$, since their skein relations can be recovered that way.

\ssk

Like the other polynomials, the skein relation gives us a way to establish properties
by induction. For example, if $L$ is $K$ with the opposite orientation, then 
$P_L(\ell,m)=P_K(\ell^{-1},m)$. The HOMFLYPT polynomial of a connected sum is 
the product of their polynomials. The the HOMFLYPT polynomial is 
invariant under mutation. But it is not a complete knot invariant: Kanenobu
gave a family examples of pairs of inequivalent knots with the same 
HOMFLYPT polynomial. (The inequivalence comes from studying the Alexander matrix.)

\ssk

Like the Jones polynomial, the highest and lowest degrees of each of the variables
of $P_K$ yields information about more classical invariants. Using the form 
of the polynomial with skein relation [at least in this case the different
variants had the good sense to use different letters for their variables...]
$v^{-1} P_{K_+}(v,z)-vP_{K_-}(v,z)=zP_{K_0}(v,z)$, we write the
degree range in the $v$ variable as $e$ to $E$ and the degree range in the $z$
variable as $m$ to $M$ (low to high). For an oriented diagram $D$ for $K$, we
can remove all of the crossings while respecting the orientation, yielding a 
particular state for ther diagram. The loops for this state are called the
{\it Seifert circles} for $D$ (for reasons coming later...), and we denote the
number of them by $s(D)$. Recalling that $w(D)$ denotes the writhe of the
diagram, Morton showed that $M\leq c(d)-s(D)+1$ and 

\ctln{$w(D)-(s(D)-1)\leq e\leq E\leq w(D)+(s(D)-1)$}

The latter inequalities implies that 
the $v$-span of $P_K$ = $E-e\leq 2(s(D)-1)$ .

\ssk

This inequality becomes more interesting when paired with the result of Yamada.
Alexander showed that every link can be expressed as the closure of a braid:
a braid is a set of strands in $I^2\times I$ which run monotonicly from the
top of the box to the bottom. The closure matches the points at the top to the 
bottom, around the outside of the box, with no additional crossings.
The minimum number of strands need to express a knot $K$ as (the closure of)
a braid is the {\it braid index} $\beta(K)$. Yamada showed that $\beta(K)$ 
is equal to the minimum of $s(D)$ over all diagrams of $K$, so Morton's
inequality then becomes $\beta(K)\geq (\text{span}_v P_K 1)/2$. This inequality
is currently the best way to determine the braid index of a knot; it is
an equality with remarkable frequency.

\ssk

Morton's other inequality is also an equality with remarkable frequency. It 
estimates the {\it canoical genus} of a knot. Given an oriented knot, the
Seifert circles for the diagram can be joined together with half-twisted
bands to create an oriented embedded surface $\Sigma$ whose boundary is $K$, called
a {\it Seifert surface} for $K$. [This algorithm for building a Seifert
surface is called {\it SEifert's algorithm}.] 
The quantity $s(D)-c(D)$ is the Euler characteristic
of $\Sigma$, which (for a knot) is equal to $1-2g(\Sigma)$, where $g(\Sigma)$ is its
genus. So Morton's inequality translates to $\text{maxdeg}_z P_K\leq 2g(\Sigma)$
for every surface for $K$ built by Seifert's algorithm. The minimum of these
genera is called the {\it canonical genus} $g_c(K)$ of $K$, and is a knot invariant;
Morton then gives a lower bound for $g_c(K)$, which is again an equality with
remarkable frequency.

\ssk

Side note: The span of the Alexander polynomial provides a lower bound for a 
similar knot invariant. The {\it genus} of $K$, $g(K)$, is the minimum over all 
Seifert surface $\Sigma$ for $K$ (whether built by Seifert's
algorithm or not...) of the genus of $\Sigma$. Seifert's approach to the
Alexander polynomial (which we did not explore) involves using a pairing
on the first homology of a Seifert surface, which ends up taking a determinant
of a $2g(\Sigma)\times 2g(\Sigma)$ matrix, yielding this 
$\text{span}\Delta_K(t)\leq 2g(K)$. Crowell showed that for alternating knots,
this inequality is an equality.

\ssk

It is a classical result of Seifert that $g(K_1\sharp K_2)=g(K_1)+g(K_2)$, proved
using a classical cut-and-paaste argument looking at the circles and arc of
intersection of a genus-minimizing Seifert surface for the connected sum 
with the sphere defining the conncted sum, and eliminating the circles of 
intersection. 

\msk

{\it Vassiliev invariants.} Skein relations compare positive and negative
crosssing with the link obtained by erasing the crossing. But in some sense
this erasion \ubr{isn't} what lies halfway between the two types of crossing; 
what does lie halfway is an \ubr{actual} crossing, i.e., a singularity. 
In 1990 Vassiliev initiated a study of knot invariants of such singular knots,
as a way to understand invariants of embedded knots. As reformulated by
Birman, Lin, Bar-Natan and others, this has a combinatorial formulation
(Vassiliev's approach was more homological). 

\ssk

Given a knot invariant $I$, with values in an abelain group, We can extend the
invariant to knots with a finite number of singularities (sometimes thought of
as embeddings of rigid-vertex 4-valent graphs) by (inductively) breaking
a singularity at $p$ to a $+$ and $-$ crossing, and defining
$I(K_p)=I(K_{p+})-I(K_{p-})$. $I$ is a {\it finite type} invariant of order $m$
if the extension $I$ is identically zero for any singular knot with more 
than $m$ crossings. 

\ssk

Many finite type invariants can be extracted from the polynomial invariants we
have built thus far. The coefficient $c_i(K)$ of $z^i$ in the Conway polynomial
is an invariant of order $\leq i$, since from the skein relation 
$I(K_p)=z$(something), so by induction the Conway polynomial of a singular 
knot with more than $i$ double points will have trivial coefficients up to
$z^i$. In a similar vein if we set $t=e^x$ in the Jone's polynomial and then 
expand as a power series, each of the coefficients of $\nabla_K(e_x)$ is 
a finite type invariant.

\msk

But many of the classical invariants are not finite type. Work of Dean and Trapp
show that starting with a tangle $T$ and summing with a tangle $T_n$ consisting
of $n$ full twists in two strands, yielding a sequence of knots $K_n$, then
for any invariant $I$ of order $k$, $n\mapsto I(K_n)$ is a poynomial of degree
at most $k$. Specific choices of tangle $T$ then show that $c(K),u(K),g_c(K),g(K),br(K)$,
and $\beta(K)$ are not finite type, since their values do not behave like those
of a polynomial.

\ssk

A finite type invariant of order $k$
is determined by a finite collection of data, called an 
{\it actuality table}. The idea behind this is that once
we have $k$ singularities, crossings in a diagram can be changed  
without changing the invariant, so all that really matters is the
relative order around a circle (= the domain of
a singular embedding (i.e., immersion!)) that the pairs of points
corresponding to singularities occur. This can be codified using the
idea of {\it chord diagrams}. Pairs of points mapping to a singularity
can be codified by drawing a (straight) arc between the points. A
singular knot with $k$ double points then has $k$ chords. A realization
of a chord diagram is an immersion whose double points yield the
diagram; all diagrams are realizable. 

\ssk

If a chord diagram has a
chord that meets no other chord (it is `isolated'), then it has a realization for 
which that singular point meets a vertical line in the projection plane
that meets the immersion nowhere else. Turning this double 
point into $+$ and $-$ crossings yields the same knot/link, and
so every finite type invariant ttakes the value $0$ on that
singular knot. This motivates our construction of the table.

\ssk

For each chord diagram with $\leq k$ chords, choose a specific 
realization of it, and record it and the value of $I$ on that
realization, with the proviso that we use the 
above realization (and value $0$) for diagrams with an isolated 
chord. Then this information is enough to compute $I$ for
any singular knot. The proof is by induction, starting with the
\ubr{largest} number of singularities. Knots with $>k$ 
double points have invariant $0$. Using the fact that 
two knots with the same chord diagram can be deformed on to the 
other (adding singularities to pass strands through one
another) without moving any of the chords we started with,
$I$ on a diagram is determined by the value of the representative 
and the values of $I$ on knots with larger number of singularities,
which, by induction, are themselves determined by the data
in the actuality table. So by, induction, we are done.

\ssk

But the data in an actuality table, constructed at random, need not come
from a finite type invariant; it might be inconsistent. Vassiliev
showed that there is a set of 4-term relations (4T), which come from
resolving the singularities of diagram which resembles a R3 move
but two of the three crossings are singularities, which are satisfied
by any finite type invariant. These can be translated into 4-term
relations on the values assigned
to representative of chord diagrams, which differ in the relative positions of 
pairs of chords in diagrams that are otherwise identical. 
Kontsevich showed, conversely, that an actuality table satisfying
these 4-term relations defines a finite type invariant of order $\leq k$.

\msk

Finite type invariants have in recent years helped to understand several
of the invariants we have looked at. For example, despite the fact that
the Alexander and Jones polynomials do not determine the other, a `colored'
version of the Jones polynomial determines the coefficients of the 
Alexander polynomial; the (first) proof used finite type invariants to show this.
There is a move on knot diagrams (in addition to the Reidemeister moves) which
completely decribes all pairs of knots that agree for all $\bbz$-valued
finite type invariants of order $\leq n$, called a `clasper' move.
The question of how many finite type invariants (taking values in the same 
abelian group) of order $\leq n$ there are is a active subject of current
research. [Such invariants form a module (think: vector space) under addition
of values; the result above implies that it is finite-dimensional.]










\vfill\end


