\magnification=1200

\nopagenumbers

\input amstex
\loadmsbm

\def\ctln{\centerline}
\def\ssk{\smallskip}
\def\msk{\medskip}
\def\bsk{\bigskip}
\def\nidt{\noindent}
\def\del{\partial}
\def\bbr{{\Bbb R}}
\def\ra{\rightarrow}
\def\lra{$\Leftrightarrow$}


\ctln{\bf Math 856 Homework 2}

\ssk

\ctln{Starred {\bf (*)} problems are due Friday, October 9}

\bsk

\nidt {\bf (*) 9:} [Lee, problem 2-1 (part)] Using the charts on the circle $S^1$ given 
by steriographic projection, compute the local coordinate representations of the functions

\ssk

\ctln{$f_n:S^1\rightarrow S^1$ given by $f_n(z)=z^n$ (in complex coordinates)}

\ssk

and use this to demonstrate that each $f_n$ is $C^\infty$.

\bsk

\nidt {\bf 10:} We know that if $C,D\subseteq M$ are disjoint closed sets of the smooth manifold $M$,
then there exists a smooth function $f:M\ra [0,1]$ with $C\subseteq f^{-1}(0)$ and $D\subseteq f^{-1}(1)$.
But we can in fact make these containments {\it equalities}:

\ssk

(a) Show that it suffices to build a smooth function $g:M\ra [0,1]$ with $C=g^{-1}(0)$.

\ssk

(b) Build a countable cover $\{U_i\}$ of $M\setminus C$ by open sets of the form $h_i^{-1}(B(x_i,1)$ for
a collection of coordinate charts $h_i=(x^1,\ldots,x^n)$ with image containing $B(x_i,2)$. Build $C^\infty$ functions $g_i:M\ra \bbr$ which are $>0$
in $U_i$ and $=0$ on $M\setminus U_i$. Note that $\overline{U_i}$ is compact; for each $i$, let

\ssk

\ctln{$\displaystyle \alpha_i = \sup_{x\in \overline{U_i};j\leq i;m\leq i;k_1,\ldots 
k_m\leq n}\{ {{\del^m g_j}\over{\del x^{k_1}\cdots \del x^{k_m}}}(x)\}$ .}

\ssk

\noindent Show that the function $g=\sum g_i/(\alpha_i 2^i)$ is $C^\infty$ and $C=g^{-1}(0)$ .

\bsk

\nidt {\bf 11:} [Lee, problem 2-6] For $M$ a (smooth) manifold, let $C(M)$ denote the
set of continuous functions from $M$ to $\bbr$, thought of as an algebra (i.e., a ring and a vector
space over $\bbr$) with scalar multiplication by $\bbr$, and pointwise addition and multiplication.
Let $C^\infty(M)$ be the subalgebra of smooth functions. If $F:M\ra N$ is continuous,
let $F^*:C(N)\ra C(M)$ be given by $F^*(f)=f\circ F$.

\ssk

(a) Show that $F^*$ is a linear map.

\ssk

(b) Show that $F$ is smooth \lra\ $F^*(C^\infty(N))\subseteq C^\infty(M)$ .

\ssk

(c) Suppose $F$ is a homeomorphism. Show that $F$ is a diffeomorphism

\noindent \lra\ $F^*:C^\infty(N)\ra C^\infty(M)$ is an isomorphism.

\bsk

\nidt {\bf (*) 12:} [Lee, problem 2-17] Find an example of a (non-closed: it can't be done if the set is
closed! This is what the Tietze Extension Theorem says...) subset $A$ of a smooth manifold $M$, 
and a smooth function $f:A\rightarrow \bbr$ which admits
{\bf no} extension to a smooth function $\tilde{f}:M\rightarrow \bbr$.

\ssk

(Recall that $f$ is called smooth if for every $x\in A$ there is a neighborhood $x\in{\Cal U}$ and a smooth
extension of $f|_{A\cap {\Cal U}}$ to the neighborhood ${\Cal U}$.

\bsk

\item{\bf 13.} Giving $M_1\times M_2$ the product smooth structure, show that $f:N\ra M_1\times M_2$
is smooth \lra\ the maps $p_1\circ f:N\ra M_1$, $p_2\circ f:N\ra M_2$ are smooth, where
$p_1,p_2$ are the projections onto the first and second factors, respectively. Show, moreover, that
the product smooth structure is the only smooth structure on $M_1\times M_2$ with this property.

\bsk

\item{\bf (*) 14.} For $a\in M$, let ${\Cal F_a}\subseteq C^\infty(M)$ denote the smooth 
functions satisfying $f(a)=0$. and let $L:{\Cal F}_a\ra \bbr$ be a linear operator
satisfying $L(fg)=0$ for all $f,g\in {\Cal F}_a$. Show that there is a unique
derivation $X\in T_aM$ satisfying $X|_{{\Cal F}_a}=L$ .

\ssk

\item{} (N.B. This leads to still another characterization of tangent vectors, has
the vector space of linear maps $X:{\Cal F}_a/W\ra \bbr$, where 
$W = {\Cal F}_a^2$ = the ideal generated by products $fg$ for $f,g\in{\Cal F}_a$ .)

\bsk

\item{\bf 15.} Using our `standard' charts on the sphere $S^n$ (domains are the $2n$ open
hemispheres, and maps ignore the coordinate whose sign is being restricted), show that the
map $F:S^3\ra S^2$ given by 

\ssk

\ctln{$F(x,y,z,w)=(2xz+2yw,2yz-2xw,x^2+y^2-z^2-w^2)$}

\ssk

\item{} is smooth (here we write points of $S^3$ as points of 
$\bbr^4$ satisfying $x^2+y^2+z^2+w^2=1$). [Check also that it maps into $S^2$ !]

\ssk

\item{} [N.B.: This is a fairly famous map, known as the {\it Hopf map}. 
Writing points in $\bbr^4$ as pairs of complex numbers $(z_1,z_2)$, this map
can be expressed as $F(z_1,z_2) = (2z_1\overline{z_2},|z_1|^2 - |z_2|^2)$. (This was stolen straight
from Wikipedia.) 
If $F(z_1,z_2)=F(w_1,w_2)$, then $z_1=\lambda w_1$ and $z_2=\lambda w_2$ for 
some complex number $\lambda$ with $|\lambda|=1$. This implies that every point inverse is a
`round' circle.]


\vfill
\end
