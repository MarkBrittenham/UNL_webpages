\magnification=1200

\nopagenumbers

\input amstex
\loadmsbm

\def\ctln{\centerline}
\def\ssk{\smallskip}
\def\msk{\medskip}
\def\bsk{\bigskip}
\def\nidt{\noindent}
\def\del{\partial}
\def\bbr{{\Bbb R}}
\def\cla{{\Cal A}}
\def\clb{{\Cal B}}
\def\clc{{\Cal C}}



\ctln{\bf Math 856 Problem Set 1}

\ssk

\ctln{Starred (*) problems to be handed in Friday, September 18}

\bsk

\item{\bf (*) 1.} You've probably heard that a connected, locally path-connected
space $X$ is path connected; the set of points reachable from $x\in X$ by a path is open
(and its complement is also open). So a connected manifold is path connected.

\ssk

\item{} Show, further, that a connected manifold $M$ is {\it arcwise connected},
that is, for every pair of points $x,y\in M$ there is a
{\it one-to-one} path $\gamma:[0,1]\rightarrow M$ with $\gamma(0)=x,\gamma(1)=y$.

\ssk

\item{} (There is a theorem, due to Hahn and Mazurkiewicz (circa 1914), which says 
that a Hausdorff space is path connected iff it is
arcwise connected (which is kind of funny, since the term ``Hausdorff''
wasn't really introduced until the 1920s?); but for manifolds you can 
give a much more elementary proof...)

\msk

\item{\bf 2.} Show that every topological $n$-manifold has a countable basis consisting of
open sets homeomorphic to $\bbr^n$. [Hint: start with any old countable basis....]

\msk

\item{\bf 3.} (Lee, p. 28, problem 1-4) If $0\leq k\leq \min\{m,n\}$, show that the set 
$R_k\subseteq M(m\times n,{\Bbb R})$ of $m$-by-$n$ matrices 
with rank $\geq k$ is an open subset of $M(m\times n,{\Bbb R})
\cong {\Bbb R}^{mn}$ (and therefore admits a smooth structure).
({\it Hint:} look at Lee's linear algebra appendix...)

\ssk

\item{} Note: This implies that the space $GL(n,\bbr)$ of invertible $n\times n$ matrices
is a smooth manifold, of dimension $n^2$.

\msk

\item{\bf (*) 4.} 
We will call two $C^\infty$ atlases $\cla$ and $\clb$ for a manifold $M$ {\it equivalent}
if their union $\cla\cup\clb=\clc$ is also a $C^\infty$ atlas for $M$. Show that
equivalence is an equivalence relation!

\msk

\item{\bf 5.} We say that two charts 
$\phi :U\rightarrow {\Bbb R}^n$ , $\psi :V\rightarrow {\Bbb R}^n$,
$U,V\subseteq M^n$ are \underbar{$C^\infty$-related}
if $\psi\circ\phi^{-1}:\phi(U\cap V)\rightarrow \psi(U\cap V)$
and $\phi\circ\psi^{-1}:\psi(U\cap V)\rightarrow \phi(U\cap V)$
are both $C^\infty$. Show that the relation `` is $C^\infty$-related
to '' is {\bf not} an equivalence relation.
(Hint: $M^n = {\Bbb R}$ will suffice for an example...)

\msk

\item{\bf 6.} Show that ${\Bbb R}$ has uncountably many distinct smooth structures.
[(Perhaps) show first that it is enough to find uncountably many charts, with intersecting domains 
and ranges, no two of which are $C^\infty$-related to one another.]

\msk

\item{\bf (*) 7.}
If $M$ and $N$ are smooth manifolds, show that $M\times N$ and $N\times M$,
with the (two) product smooth structures 
$\{(U_\alpha\times V_\beta,h_\alpha\times k_\beta)\}$
are diffeomorphic. [I.e., exhibit (and verify) a diffeomorphism!]

\msk

\item{\bf 8:} Show that a function $f:M^n\rightarrow N^m$ is $C^\infty$ $\Leftrightarrow$ 
$g\circ f:M^n\rightarrow {\Bbb R}$ is $C^\infty$ for {\it every} $C^\infty$ function
$g:N^m\rightarrow {\Bbb R}$. 
\vfill
\end
