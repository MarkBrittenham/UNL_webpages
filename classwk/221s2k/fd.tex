%\baselineskip=18pt plus 2pt

\magnification=1200

\parindent=50pt

\def\ni{\noindent}
\def\ctln{\centerline}
\def\msk{\medskip}
\def\ssk{\smallskip}
\def\bsk{\bigskip}

\def\iit{\itemitem}
\def\htp{\hskip10pt}
\def\vtp{\vskip.02in}
\def\hsk{\hskip.2in}
%\input shorthand

\nopagenumbers

\ctln{\bf Math 221 (821) Differential Equations}

%\vtp

\ctln{\bf Section 002}

\smallskip

\ni{\bf Lecture:} TuTh 12:30-1:45 \htp Oldfather Hall (OldH) 204

\msk

\ni{\bf Instructor:} Mark Brittenham

%\smallskip

\ni{\bf Office:} Oldfather Hall (OldH) 819
%\smallskip

\ni{\bf Telephone:} (47)2-7222

%\ssk

\ni{\bf E-mail:} mbritten@math.unl.edu

\ni{\bf WWW:} http://www.math.unl.edu/\~{ }mbritten/

\ni{\bf WWW pages for this class:} http://www.math.unl.edu/\~{ }mbritten/classwk/221s2k/

\ssk

\ni(There you will find copies of nearly every handout from class, lists of homework 
problems assigned, dates for exams, etc.)

\smallskip

\ni{\bf Office Hours:} (tentatively) Mo 1:30-2:30, Tu 2:00 - 3:00, We 10:00-11:00, 
and Th 9:30-10:30, and whenever you can find me in my office and I'm not 
horrendously busy. You are also quite welcome to make an appointment
for any other time; this is easiest to arrange just before or 
after class.

\ssk

\ni{\bf Text:} {\it Elementary Differential Equations}, 
by Boyce and DiPrima (6th edition, John Wiley and Sons).

\msk

\ni This course, as the name is meant to imply, is intended to introduce 
you to some of the problems, techniques, and applications of differential
equations (i.e., problems involving an (unknown) function and some of its
derivatives). Developing and solving such equations is a fundamental part
of many science and engineering problems. We will explore several different
approaches to differential equations, which depend on different interpretations
of the word `solving'. The course will focus on analytical methods (finding a formula),
qualitative methods (understanding the basic shape of the {\it graph} of
a solution), and numerical methods (finding approximate solutions, largely
with the help of a computer).

\ssk

\hskip-.2in Our basic goal will be to work through some or all of each of the following chapters:

\ssk

\ni\hsk Ch. 1, Introduction

\ni\hsk Ch. 2, First Order Differential Equations

\ni\hsk Ch. 3, Second Order Linear Equations

\ni\hsk Ch. 6, The Laplace Transform

\ni\hsk Ch. 7, Systems of First Order Linear Equations

\ni\hsk Ch. 9, Nonlinear Differential Equations and Stability

\ssk

plus the odd section from the remaining chapters.

\msk

\ni{\bf Homework} will be assigned from each section, as we finish it. 
It is an essential ingredient to the course - as with almost all of 
mathematics, we learn best by doing (again and again and ...). Cooperation 
with other students on these assignments is acceptable, and even 
encouraged. However, you should make sure you are understanding the
process of finding the solution, on your own - after 
all, you get to bring only one brain to exams (and it can't be someone 
else's). For the same reason, I also recommend that you try working 
each problem on your own, first. Homework will not be collected, and therefore, not 
graded (this is largely because our textbook has the 
interesting feature of providing the answers to {\it every} problem, at the 
end of the book); but it is probably the most important way to make 
sure that you are understanding the material..

\ssk

\ni In addition, we will have one significantly larger assignment. This {\bf project}
will be assigned near the end of February, and be due several weeks later. 
You may choose to work on the project in groups of up to three, with one write-up
turned in for the group, or you may choose to work on it individually. It will 
count 10\% towards your final grade.

\ssk

\ni {\bf Quizzes} will be given each Thursday, during weeks that do not 
also contain 
an exam (in {\it our} class...) or the first day of classes. Each will 
typically consist of one
question (modelled on a homework problem) from the material covered through
the previous Tuesday. Your lowest two quiz grades will be dropped before computing your
final quiz average, which will constitute 20 \% of your grade. A missed quiz will 
count as zero (and will therefore be the first grade dropped); a make-up quiz can 
be arranged only under the most unusual of circumstances.

\ssk

\ni{\bf Midterm exams} will be given three times during the 
semester, approximately every four weeks - the specific dates will 
be announced in class well in advance (likely candidates: early February, 
end of March, mid-April). Each exam will count 15\% toward your grade. 
You can take a 
make-up exam only if there are compelling reasons (a doctor SAYS 
you were sick, jury duty, etc.) for you to miss an exam. Make-up 
exams tend to be harder than the originals (because make-up exams 
are harder to write!). 

\ssk

\ni Finally, there will be a regularly scheduled {\bf final exam} on 
Monday, May 1, from 1:00pm to 3:00pm.
It will cover the entire course, with a slight emphasis 
on material covered after the last midterm exam. It will count the 
remaining 25\% toward your grade.


\msk

\ni {\bf Your course grade} will be calculated numerically using the above percentages,
and will be converted to a letter grade based partly on the overall average of the
class. However, a score of 90\% or better will guarantee some kind of {\bf A}, 80\%
or better at least some sort of {\bf B}, 70\% or better at least a flavor of 
{\bf C}, and 60\% or 
better at least a {\bf D}.

\bsk 

And now the obligatory pep talk:

\ssk

\ni\hskip.2in In mathematics, new concepts continually rely upon the mastery
of old ones; it is therefore essential that you thoroughly understand each 
new topic before moving on. Our classes are an important opportunity for you to ask
questions; to make \underbar{sure} that you are understanding concepts correctly.
Speak up! It's \underbar{your} education at stake. Make every effort to resist
the temptation to put off work, and to fall behind. Every topic has to be gotten 
through, not around. And it's alot easier to read 50 pages in a week than it is
in a day. Try to do some mathematics every single day. (I do.)
{\bf Class attendance} is probably your best way to insure that you will keep 
up with the material, and make sure that you understand all of the
concepts. I will not be taking attendance; I expect that you will simply 
see the wisdom of attending class, for yourselves.

\msk

\ni{\bf Departmental Grading Appeals Policy:} The Department of 
Mathematics and Statistics does not tolerate discrimination  
or harassment on the basis of race, gender, religion, or sexual orientation. 
If you believe you have been subject to such discrimination or harassment,  
in this or any other math course, please contact the department. 
If, for this or any other reason, you believe your grade was assigned 
incorrectly or capriciously, then appeals may be made to (in order)  
the instructor, the department chair, the department grading appeals  
committee, the college grading appeals committee, and the university  
grading appeals committee. 

\msk

\ctln{\bf Some important academic dates}

\ssk

{\bf Jan. 10} First day of classes.

{\bf Jan. 17} Martin Luther King Day - no classes.

{\bf Jan. 21} Last day to withdraw from a course without receiving a {\bf `W'}.

{\bf Mar. 3} Last day to change to or from P/NP.

{\bf Mar. 12-19} Spring break - no classes.

{\bf Apr. 7} Last day to withdraw from a course.

{\bf Apr. 29} Last day of classes.

\vfill

\end

\vfill\eject
