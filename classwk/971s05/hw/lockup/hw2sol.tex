

\magnification=1200
\overfullrule=0pt
\parindent=0pt
\nopagenumbers

\input amstex 

\loadmsbm

\input epsf

\def\ctln{\centerline}
\def\u{\underbar}
\def\ssk{\smallskip}
\def\msk{\medskip}
\def\bsk{\bigskip}
\def\hsk{\hskip.1in}
\def\hhsk{\hskip.2in}
\def\dsl{\displaystyle}
\def\hskp{\hskip1.5in}

\def\lra{$\Leftrightarrow$ }
\def\pu{\pi_1}
\def\mpu{$\pi_1$}
\def\ra{\rightarrow}
\def\del{\partial}
\def\inv{^{-1}}


%%Homework: p.39: 5*,13,15,19,20* p.53: 6,7,8*


\ctln{\bf Math 971 Algebraic Topology}

\ssk

\ctln{Homework \#\ 2 Solutions}

\msk

A continuous surjection from 
a compact space to a Hausdorff space is (a closed map hence) a quotient map.
(Which makes building induced maps a whole lot more straightforward.)

\ssk

If $\gamma:I\ra X$ is a path in $X$ beginning at $x_0$ and ending at $x_1$, then 
the induced change of basepoint isomorphism $\widehat{\gamma}:\pu(X,x_0)\ra \pu(X,x_1)$
is $\widehat{\gamma}([\eta]) = [\overline{\gamma}*\eta*\gamma]$ \hhsk (to get 
the basepoints to work out right).

\bsk

{\bf (p.38, \#\ 5):} Every map $\gamma:S^1\ra X$ is homotopic to a constant 
\lra\ every map $\gamma:S^1\ra X$ extends to a map $\Gamma:{\Bbb D}^2\ra X$
\lra\ $\pu(X,x_0)=\{1\}$ for every $x_0\in X$.

\ssk

We will prove (1) $\Rightarrow$ (2) $\Rightarrow$ (3) $\Rightarrow$ (1). 
We will think of $S^1$ as the unit circle in ${\Bbb R}^2={\Bbb C}$, and ${\Bbb D}^2$
as the unit disk in ${\Bbb R}^2$.

\ssk

(1) $\Rightarrow$ (2): Given $\gamma:S^1\ra X$, by (1), there is a 
homotopy $H:S^1\times I\ra X$ with $H(z,0)=\gamma(z)$ and $H(z,1)=x_0$
for some $x_0\in X$. If we define a map $h:S^1\times I\ra{\Bbb D}^2$ by 
$h(z,t)= h((x,y),t) = (1-t)z = ((1-t)x,(1-t)y)$. As a function of 3 variables, it is continuous,
so restricting domain and range it is cts. The map $h$ factors through the quotient space
$Z=(S^1\times I)/(S^1\times\{1\})$, since on the 1 end $h$ is 0. The
resulting map $\overline{h}:Z\ra {\Bbb D}^2$ is a cts bijection from (quotient of compact, hence)
compact to Hausdorff, so it is a homeomorphism. $H$ also factors through
$Z$, since on the 1 end $H$ is constant; call the resulting map $\overline{H}$.
Then $\Gamma = \overline{H}\circ\overline{h}^{-1}:{\Bbb D}^2\ra X$ is the required map 
extending $\gamma$.

\ssk

(2) $\Rightarrow$ (3): Given an element $[\gamma]\in \pu(X,x_0)$, 
$\gamma:(I,\del I)\ra (X,x_0)$ factors through the (quotient) map
$f:I\ra S^1$ given by $f(t)=e^{2\pi it}$, to give a map $g:S^1\ra X$.
By hypothesis, this map extends to a map $G:{\Bbb D}^2\ra X$. 
If we define a map $K:I\times I\ra {\Bbb D}^2$ by
$K(t,s) = (s,0)+(1-s)e^{2\pi it} = (s+(1-s)\cos(2\pi t),(1-s)\sin(2\pi t))$,
then $K(t,0)=f(t)$ and $K(t,1)=(1,0)$. Then 
$H=G\circ K:I\times I\ra X$ has $H(t,0)=\gamma(t)$ and $H(t,1) = G(1,0)=x_0$,
and $H(0,s)=H(1,s) = K(1,0) = x_0$. So $H$ represents a
homotopy, rel basepoint, from $\gamma$ to the constant map.
So  $\pu(X,x_0)=\{1\}$ .

\ssk

(3) $\Rightarrow$ (1): Given $\gamma:S^1\ra X$, composing with the 
(quotient) map $f$ above gives a based loop $g=\gamma\circ f:(I,\del I)\ra (X,x_0)$,
where $x_0=\gamma(1,0)$. By hypothesis, this map is null-homotopic,
so there is a map $H:I\times I\ra X$ with $H(t,0)=g(t)$, and 
$H(t,1)=H(0,s)=H(1,s)=x_0$ for all $t,s\in I$. This map factors
through the (quotient) map $f\times Id:I\times I\ra S^1\times I$ to give
an induced map $\overline{H}:S^1\times I\ra X$ with 
$\overline{H}(z,0)=\gamma(z)$ and $\overline{H}(z,1)=x_0$. So $\gamma$
is homotopic to a constant map.

\ssk

\hhsk $X$ is simply-connected \lra\ all maps $S^1\ra X$ are homotopic to one another:

\ssk

($\Rightarrow$): Given two maps $g,h:S^1\ra X$, composing them with the map $p:I\ra S^1$,
$p(t)=(\cos(2\pi t),\sin(2\pi t)$ gives us a pair of based loops $\gamma,\eta$, based at
$g(1,0)=x_0$ and $h(1,0)=x_1$ respectively. By hypothesis, each one represents the trivial 
element in $\pu(X,x_\epsilon)$, so there are homotopies $G,H:I\times I\ra X$ between these
loops and their respective constant maps. Because these maps are constant on $I\times \del I$,
they factor through the map $f$ above to induce maps $G^\prime,H^\prime:S^1\times I\ra X$
with restriction to $S^1\times\{1\}$ the (appropriate) constant map. Since $X$ is 0-connected,
there is a path $\delta:I\ra X$ with $\delta(0)=x_0$ and $\delta(1)=x_1$. Then defining $K:S^1\times I\ra X$
by $K(x,t)=\delta(t)$ we have a continuous map (since $K\inv({\Cal U}) = S^1\times\delta\inv({\Cal U})$).
And finally, defining $R:S^1\times I\ra X$ by

$\displaystyle R(x,t) = 
\cases
G^\prime(x,3t)&\text{, if $t\leq 1/3$}\cr
K(x,3t-1)&\text{, if $1/3\leq t\leq 2/3$}\cr
H^\prime(x,3-3t)&\text{, if $t\geq 2/3$}\cr
\endcases$
\hhsk defines a homotopy from $g$ to $h$ .

\ssk

($\Leftarrow$): We wish to show both that $X$ is path-connected and $\pu(X)=\{1\}$ . For path
connected, given $x_0,x_1\in X$ for the constant maps $g,h:S^1\ra X$ constant at these points,
the hypothesis implies that there is a homotopy $H:S^!\times I\ra X$ between them. Then the 
path $\gamma:I\ra X$ given by $\gamma(t)=H((1,0),t)$ has $\gamma(0)=H((1,0),0)=g(1,0)=x_0$
and $\gamma(1)=H((1,0),1)=h(1,0)=x_1$. So $X$ is path connected. And since every map $g:S^1\ra X$
is homotopic to any constant map, by (1) $\Rightarrow$ (2) $\Rightarrow$ (3) above,
$\pu(X,x_0)=\{1\}$ for every $x_0$, so $X$ is 1-connected. So $X$ is simply-connected.

\msk

{\bf (p.39, \#\ 20):} If $H:X\times I\ra X$ is a cts homotopy
from $H(x,0)=x$ to $H(x,1)=x$, then the loop defined by
$\gamma(t)=H(x_0,t)$ represents an element in the center of $\pu(X,x_0)$.

\ssk

By Lemma 1.19 of the text, 
the change of basepoint isomorphism
$\widehat{\gamma}:\pu(X,x_0)\ra \pu(X,x_0)$ given by
$\widehat{\gamma}[\eta] = [\overline{\gamma}*\eta*\gamma]$
satisfies $H_{0*}=\widehat{\gamma}\circ H_{1*}$. But
since $H_0=H_1=Id$, so their induced homomorphisms
are $Id$, we have $Id=\widehat{\gamma}\circ Id$, so 
$\widehat{\gamma} =Id$ . But this means that
for all $\eta$, $[\overline{\gamma}*\eta*\gamma]=[\eta]$,
so $[\gamma][\eta]=[\eta][\gamma]$ for every $[\eta]\in\pu(X,x_0)$.
So $[\gamma]$ commutes with every element of $\pu(X,x_0)$,
so it is central.

\msk

{\bf (p.53, \#\ 8):} Compute $\pu(X)$ where $X$ is obtained from 
two copies of the torus $S^1\times S^1$ by identifying the
circle $S^1\times\{x_0\}$ on one with the corresponding 
circle on the other.

\ssk

A cheap way to do this is to identify $X$ as a product space
itself. $X$ is the quotient space of $S^1\times S^1\times \{1,2\}$
where we identify $(x,x_0,1)$ with $(x,x_0,2)$ . But this is the 
same as taking the product of $S^1$ with the quotient $Z$ of 
$S^1\times \{1,2\}$ where we identify $(x_0,1)$ with $(x_0,2)$ .
But $Z$ is a bouquet of two circles; giving each copy of $S^1$ a 
cell structure with vertex $x_0$ and one 1-cell, $Z$ then has one vertex
and two 1-cells, which is what a bouquet of two circles is.
Then we have that $\pu(Z)=<a,b\ | > = F(a,b)$ is free on two generators,
so $\pu(X) = \pu(S^1\times Z)\cong \pu(S^1)\times \pu(Z) = {\Bbb Z}\times F(a,b)$ .

\ssk

Or if you prefer a cell structure approach, each torus can be given a cell 
structure with one 0-cell, two 1-cells (one of which, with the vertex, is the circle
$S^1\times \{x_0\}$), and one 2-cell whose boundary spells out
the commutator of the two 1-cells. $X$ therefore has one 0-cell, three
1-cells (since one from each torus have been identified), and two 2-cells.
Thinking of this as gluing two 2-cells to a bouquet of 3 circles, whose 
boundaries map to $[a,b]$ and $[b,c]$, we have
$\pu(X)\cong <a,b,c\ |\ aba^{-1}b^{-1},bcb^{-1}c^{-1}>$ .

\msk

The motivated student can verify that these two groups are in fact 
isomorphic!

\vskip.15in


\vbox{\hsize=4in

\leavevmode

\epsfxsize=2in
\epsfbox{hw2f1.ai} 
\hskip.5in
\epsfxsize=2in
\epsfbox{hw2f2.ai}}



\vfill
\end


\vskip.15in




\vbox{\hsize=3in

\leavevmode

\epsfxsize=3in
}








\vbox{\hsize=5in

\leavevmode

\epsfxsize=5in
\epsfbox{hw1f1.ai}}




\vfill
\eject


\vbox{\hsize=5in

\leavevmode

\epsfxsize=5in
\epsfbox{hw1f2.ai}}













