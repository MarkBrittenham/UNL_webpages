

\magnification=1200
\overfullrule=0pt
\parindent=0pt



\input epsf

\def\ctln{\centerline}
\def\u{\underbar}
\def\ssk{\smallskip}
\def\msk{\medskip}
\def\bsk{\bigskip}
\def\hsk{\hskip.1in}
\def\hhsk{\hskip.2in}
\def\dsl{\displaystyle}
\def\hskp{\hskip1.5in}

\def\lra{$\Leftrightarrow$ }
\def\pu{\pi_1}
\def\mpu{$\pi_1$}




\ctln{\bf Math 971 Algebraic Topology}

\ssk

\ctln{Homework \#\ 1 Solutions}

\msk

{\bf (p.19, \#\ 14):} Given $v,e,f>0$ with $v-e+f=2$, build a cell structure on $S^2$ with
$v$ 0-cells, $e$ 1-cells, and $f$ 2-cells. 

\msk

Here is one way to more or less systematically do it. Starting from the 
smallest case, $(v,e,f) = (1,0,1)$ [we always need at least one  
top- and bottom-dimensional
cell, each] as a 2-cell with its  boundary quotiented out to a point, we can proceed
to the cases $(1,n,n+1)$ by adding a bouquet of circles off of our 0-cell. 
Each new loop cuts out a new 2-cell from our original one, so the edges 
and faces each increase by 1 each time. Then we can choose one of the 
1-cells, and continually cut it into pieces, each time creating one more 
vertex and edge, to build the cases $(1+m,n+m,n+1)$ . This covers
all cases, except for $f=1,v>1$ (since increasing $v$, above, required at least
one $e$, which you don't get unless $f>1$); this we can handle, for example,
by starting from $(2,1,1)$ as an arc in the sphere, and continually subdividing the 
arc. Formally, we should probably describe the gluing maps for the 2-cells, 
but these should be evident from the pictures. 
\hhsk See the pictures on the accompanying page.

\msk

{\bf (p.38, \#\ 10):} Since $\pu(X\times Y,(x_0,y_0))\cong \pu(X,x_0)\times \pu(Y,y_0)$,
elements represented by loops $a(t)=(\gamma(t),y_0)$ and $b(s)=(x_0,\delta(s))$,
with $\gamma : I\rightarrow X$ , $\delta : I\rightarrow Y$ , commute. Construct
an explicit homotopy.

\msk

We wish to build a based homotopy $H:I\times I\rightarrow X\times Y$
between $a*b$ and $b*a$ (see the pictures on the accompanying page).
The basic idea is really to write down the only map we can, that has a
vague chance of looking like a homotopy! Define
$K:I\times I\rightarrow X\times Y$ by $K(t,s)=(\gamma(t),\delta(s))$ .
This is continuous, because 

\ctln{$(t,s)\longmapsto t\longmapsto \gamma(t)$
and $(t,s)\longmapsto s\longmapsto \delta(s)$}

 both are. Since 
$\gamma(0)=\gamma(1)=x_0$ and $\delta(0)=\delta(1)=y_0$, this
homotopy, on its boundary, has $a$'s and $b$'s (as in the figure).
But fundamentally (no pun intended) it is what we want, just without
some constant maps (the vertical sides that we want for $H$) inserted.
But since a loop followed by the constant map is homotopic to the loop,
this is something that we can fix. A formal approach involves grafting on 
some auxiliary homotopies to the one we have built, and using the fact 
that the resulting domain is still homeomorphic to $I\times I$ . (Writing 
this homeo explicitly is tedious but not hard.)

\msk

Actually, in the end, it appears I used a pair of ``constant'' (i.e., ignore the last factor)  homotopies,
$(t,s)\longmapsto (a*b)(t)$ and $(t,s)\longmapsto (b*a)(t)$ , 
together with homeos that map the upper right and
lower left portions of the last figure to a standard rectangle $I\times I$,
and pasting things together with the Pasting Lemma to assure continuity.
And as Susan (H.) has pointed out, if we shave a bit off of that picture,
we get a function (represented by the last picture) that, with patience, 
we can really write down:

\bsk

$\displaystyle H(t,s) = 
\cases {
a(t),&if $t+s\leq 1/2$\cr
b(t),&if $t\geq s+1/2$\cr
b(t),&if $s\geq t+1/2$\cr
a(t),&if $s+t\geq 3/2$\cr
K(s+t-{{1}\over{2}},s-t+{{1}\over{2}})&otherwise\cr
}$


\vfill
\eject


\vbox{\hsize=5in

\leavevmode

\epsfxsize=5in
\epsfbox{hw1f1.ai}}




\vfill
\eject


\vbox{\hsize=5in

\leavevmode

\epsfxsize=5in
\epsfbox{hw1f2a.ai}}

\vfill
\end






















