


\magnification=1200
\parindent=0pt
\nopagenumbers

\input amstex

\loadmsbm

\def\ctln{\centerline}
\def\ssk{\smallskip}
\def\msk{\medskip}
\def\bsk{\bigskip}
\def\hsk{\hskip10pt}
\def\hhsk{\hskip.5in}

\def\ra{\rightarrow}
\def\bbr{{\Bbb R}}
\def\bbz{{\BbbZ}}
\def\sset{\subseteq}
\def\cat{{\Cal T}}
\def\catpr{{\Cal T}^\prime}


{\bf \hhsk UNL Mathematics Sample Qualifying Exam I \hhsk Math 970/971}

\msk

Do three of the problems from section A and three questions from section B. If
you work more than the required number of problems, make sure that you clearly
mark which problems you want to have counted. If you have doubts about the wording of a 
problem or about what results may be assumed without proof, please ask for
clarification. In no case should you interpret a problem in such a way that it becomes
trivial.

\bsk

{\bf Section A: Point Set Topology.}

\msk

{\bf 1.} A subset $D\sset X$ is called {\it dense} if the closure, in $X$, of $D$ is $X$.
Show that if $A \subseteq X$ and $D\subseteq X$ is dense in $X$, that 
$D\cap A$ {\it need not} be dense in $A$.
Show that if $A$ is open in $X$, then $D\cap A$ {\it is} dense in $A$.

\msk

{\bf 2.}  Show that if $A,B$ are disjoint compact subsets of the Hausdorff space $X$, 
then there are disjoint open sets ${\Cal U},{\Cal V}\subseteq X$ with 
$A\subseteq {\Cal U}$ , $B\subseteq{\Cal V}$ .

\msk

{\bf 3.} Let $Y$ be a topological space, and
for each natural number $n$, let $X_n$ be a connected
subspace of $Y$.  Suppose that $X_{n+1} \subseteq X_n$
for every $n$.  Must $\cap X_n$ also be connected?

\msk

{\bf 4.} A function $f:X \rightarrow Y$ is called {\it open}
if for every open subset $U$ of $X$, the set $f(U)$ is
open in $Y$.

{\bf a.}  Give an example of a continuous function that is
not open.

{\bf b.}  Let $p:X \rightarrow Y$ be an open continuous
function, and let $A$ be open in $X$.  Show that if
$q:A \rightarrow p(A)$ is the restriction of $p$, then
$q$ is also open.




\bsk

{\bf Section B: Homotopy and Homology.}

\msk

{\bf 5.} Let $x,y,z$ be three distinct points in the 2-dimensional
torus $T$ of genus 1. 
Compute $\pi_1(T\setminus\{x,y,z\})$ .

\msk

{\bf 6.} Show that the M\"obius band $M$ does not admit a retraction onto 
its boundary circle $S=\partial M$.

\msk

{\bf 7.} Use covering space theory to find two non-conjugate subgroups of 
index 4 in the free group $F(a,b)$ on two letters.

\msk

{\bf 8.} Let $X$ = the union of ${\Bbb R}P^2$ and $S^2$ with one point in 
each identified. Find the universal covering $\widetilde{X}$ of $X$ and 
compute its homology groups.



\vfill
\end