
\magnification=1200
\overfullrule=0pt
\parindent=0pt

\nopagenumbers

\input amstex

\voffset=-.6in
\hoffset=-.5in
\hsize = 7.5 true in
\vsize=10.4 true in

%\voffset=1.4in
%\hoffset=-.5in
%\hsize = 10.2 true in
%\vsize=8 true in

\input colordvi

\loadmsbm

\input epsf

\def\ctln{\centerline}
\def\u{\underbar}
\def\ssk{\smallskip}
\def\msk{\medskip}
\def\bsk{\bigskip}
\def\hsk{\hskip.1in}
\def\hhsk{\hskip.2in}
\def\dsl{\displaystyle}
\def\hskp{\hskip1.5in}

\def\lra{$\Leftrightarrow$ }
\def\ra{\rightarrow}
\def\mpto{\logmapsto}
\def\pu{\pi_1}
\def\mpu{$\pi_1$}
\def\sig{\Sigma}
\def\msig{$\Sigma$}
\def\ep{\epsilon}
\def\sset{\subseteq}
\def\del{\partial}
\def\inv{^{-1}}
\def\wtl{\widetilde}



\ctln{\bf Math 971 Algebraic Topology}

\ssk

\ctln{February 10, 2005}

\msk

\bsk

{\bf Postscript:  why should we care?} The role of the fundamental group
in distinguishing spaces has already been touched upon; if two 
(path-connected) spaces have non-isomorphic fundamental groups, then
the spaces are not homeomorphic, and even not homotopy equivalent.
It is one of the most basic, and in many cases the best such invariant
we have in our arsenal
(hence the name ``fundamental''). As we have seen with the circle, it
captures the notion of how many times a loop ``winds around'' in a space.
And the idea of using paths to understand a space is very basic; we 
explore a space by mapping familiar objects into it. (This is 
a theme we keep returning to in this course.) The concepts we 
have introduced play a role in analysis, for instance with the notion
of a path integral; the invariance of the integral under homotopies
rel endpoints is an important property, related to Green's Theorem
and (locally) conservative vector fields. And the \underbar{space}
of all paths in $X$ plays an important (theoretical, although
pprobably not practical) role in what we will do next.

\bsk

{\bf Covering spaces:} We can motivate our next topic by looking more
closely at one of our examples above. The projective plane ${\Bbb R}P^2$
has $\pu = {\Bbb Z}_2$ . It is also the quotient of the simply-connected
space $S^2$ by the antipodal map, which, together with the identity map,
forms a group of homeomorphisms of $S^2$ which is isomorphic to ${\Bbb Z}_2$.
The fact that ${\Bbb Z}_2$ has this dual role to play in describing 
${\Bbb R}P^2$ is no accident; codifying this relationship requires the 
notion of a covering space.

\msk

The quotient map $q:S^2\ra {\Bbb R}P^2$ is an example of a {\it covering map}.
A map $p:E\ra B$ is called a covering map if for every point $x\in B$, there
is a neighborhood ${\Cal U}$ of $x$ (an
{\it evenly covered neighborhood}) so that $p^{-1}({\Cal U})$ 
is a disjoint union ${\Cal U}_\alpha$ of open sets in $E$, each mapped
homeomorphically onto ${\Cal U}$ by (the restriction of) $p$ . $B$ is
called the {\it base space} of the covering; $E$ is called the {\it total
space}. The quotient map $q$ is an example; (the image of) the complement
of a great circle in $S^2$ will be an evenly covered neighborhood
of any point it contains. The disjoint union of 43 copies of a space,
each mapping homeomorphically to a single copy, is an example of a 
{\it trivial covering}. As a last example, we have the famous 
exponential map $p:{\Bbb R}\ra S^1$ given by $t\mapsto e^{2\pi it} = 
(\cos (2\pi t),\sin (2\pi t))$. The image of any interval $(a,b)$ of length
less than 1 will have inverse image the disjoint union of the
intervals $(a+n,b+n)$ for $n\in{\Bbb Z}$ .

\msk

OK, maybe not the last. We can build many finite-sheeted (every point
inverse is finite) coverings of a bouquet of two circles, say, by 
assembling $n$ points over the vertex, and then, on either side,
connecting the points by $n$ (oriented) arcs, one each going in and out of
each vertex. By choosing orientations on each 1-cell of the bouquet,
we can build a covering map by sending the vertices above to the
vertex, and the arcs to the one cells, homeomorphically, respecting 
the orientations. We can build infinite-sheeted coverings in much 
the same way.

\msk

\leavevmode


\epsfxsize=3in
\ctln{{\epsfbox{0208f1.ai}}}


\bsk

Covering spaces of a (suitably nice) space $X$ have a very close relationship
to $\pu(X,x_0)$. The basis for this relationship is the

\msk 

{\bf Homotopy Lifting Property:} If $p:\wtl{X}\ra X$ is a covering map, 
$H:Y\times I\ra X$ is a homotopy, $H(y,0)=f(y)$, and
$\wtl{f}:Y\ra \wtl{X}$ is a {\it lift} of $f$ (i.e., $p\circ \wtl{f}=f$),
then there is a unique lift $\wtl{H}$ of $H$ with $\wtl{H}(y,0)=\wtl{f}(y)$ .

\msk

The {\bf proof} of this we will defer to next time, to give us sufficient time to ensure we
finish it!

\msk

 In particular, applying this property in the case $Y=\{*\}$, where a homotopy
$H:\{*\}\times I\ra X$ is really just a a path $\gamma:I\ra X$,
we have the {\bf Path Lifting Property}: ``given 
a covering map $p:\wtl{X}\ra X$, a path 
$\gamma:I\ra X$ with $\gamma(0)=x_0$, and a point 
$\wtl{x}_0\in p^{-1}(x_0)$, there is a unique path $\wtl{\gamma}$
lifting $\gamma$ with $\wtl{\gamma}(0)=\wtl{x}_0$ .'' One of the 
immediate consequences of this is one of the cornerstones of covering space
theory:

\bsk  

If $p:(\wtl{X},\tilde{x}_0)\ra (X,x_0)$ is a covering map, then the 
induced homomorphism $p_*:\pu(\wtl{X},\wtl{x}_0)\ra \pu(X,x_0)$ is
injective.

\msk

{\bf Proof:} Suppose $\gamma:(I,\del I)\ra (\wtl{X},\wtl{x}_0)$ is a 
loop $p_*([\gamma])=1$ in $\pu(X,x_0)$. So there is a homotopy
$H:(I\times I,\del I\times I)\ra (X,x_0)$ between $p\circ\gamma$ and the constant 
path. By homotopy lifting, there is a homotopy $\wtl{H}$ from $\gamma$ to 
the lift of the constant map at $x_0$. The vertical sides 
$s\mapsto \wtl{H}(0,s),\wtl{H}(1,s)$ are also lifts of the 
constant map, beginning from 
$\wtl{H}(0,0),\wtl{H}(1,0)=\gamma(0)=\gamma(1)=\wtl{x}_0$, so
are the constant map at $\wtl{x}_0$. Consequently, the lift at the 
bottom is the constant map at $\wtl{x}_0$. So $\wtl{H}$
represents a null-homotopy of $\gamma$, so $[\gamma]=1$
in $\pu(\wtl{X},\wtl{x}_0)$. So $\pu(\wtl{X},\wtl{x}_0)=\{1\}$ .

\msk

Even more, the image $p_*(\pu(\wtl{X},\wtl{x}_0)))\sset \pu(X,x_0)$
is precisely the elements whose representatives are loops at $x_0$, 
which when
lifted to paths starting at $\wtl{x}_0)$, are loops. 
For if $\gamma$ lifts
to a loop $\wtl{\gamma}$, then $p\circ\wtl{\gamma}=\gamma$, so
$p_*([\wtl{\gamma}])=[\gamma]$ . Conversely, if 
$p_*([\wtl{\gamma}])=[\gamma]$, then $\gamma$ and $p\circ\wtl{\gamma}$
are homotopic rel endpoints, and so the homotopy lifts to 
a homotopy rel endpoints between the lift of $\gamma$ at 
$\wtl{x}_0$, and the lift of $p\circ\wtl{\gamma}$ at $\wtl{x}_0$
(which is $\wtl{\gamma}$, since $\wtl{\gamma}(0)=\wtl{x}_0$ and
lifts are unique). So the lift of $\gamma$ is a loop, as desired.

\vfill
\end
