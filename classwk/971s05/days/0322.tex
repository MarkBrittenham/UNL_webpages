

\magnification=1200
\overfullrule=0pt
\parindent=0pt

\nopagenumbers

\input amstex

\voffset=-.6in
\hoffset=-.5in
\hsize = 7.5 true in
\vsize=10.4 true in

%\voffset=1.4in
%\hoffset=-.5in
%\hsize = 10.2 true in
%\vsize=8 true in

\input colordvi

\def\cltr{\Red}		  % Red  VERY-Approx PANTONE RED
\def\cltb{\Blue}		  % Blue  Approximate PANTONE BLUE-072
\def\cltg{\PineGreen}	  % ForestGreen  Approximate PANTONE 349
\def\cltp{\DarkOrchid}	  % DarkOrchid  No PANTONE match
\def\clto{\Orange}	  % Orange  Approximate PANTONE ORANGE-021
\def\cltpk{\CarnationPink}	  % CarnationPink  Approximate PANTONE 218
\def\clts{\Salmon}	  % Salmon  Approximate PANTONE 183
\def\cltbb{\TealBlue}	  % TealBlue  Approximate PANTONE 3145
\def\cltrp{\RoyalPurple}	  % RoyalPurple  Approximate PANTONE 267
\def\cltp{\Purple}	  % Purple  Approximate PANTONE PURPLE

\def\cgy{\GreenYellow}     % GreenYellow  Approximate PANTONE 388
\def\cyy{\Yellow}	  % Yellow  Approximate PANTONE YELLOW
\def\cgo{\Goldenrod}	  % Goldenrod  Approximate PANTONE 109
\def\cda{\Dandelion}	  % Dandelion  Approximate PANTONE 123
\def\capr{\Apricot}	  % Apricot  Approximate PANTONE 1565
\def\cpe{\Peach}		  % Peach  Approximate PANTONE 164
\def\cme{\Melon}		  % Melon  Approximate PANTONE 177
\def\cyo{\YellowOrange}	  % YellowOrange  Approximate PANTONE 130
\def\coo{\Orange}	  % Orange  Approximate PANTONE ORANGE-021
\def\cbo{\BurntOrange}	  % BurntOrange  Approximate PANTONE 388
\def\cbs{\Bittersweet}	  % Bittersweet  Approximate PANTONE 167
%\def\creo{\RedOrange}	  % RedOrange  Approximate PANTONE 179
\def\cma{\Mahogany}	  % Mahogany  Approximate PANTONE 484
\def\cmr{\Maroon}	  % Maroon  Approximate PANTONE 201
\def\cbr{\BrickRed}	  % BrickRed  Approximate PANTONE 1805
\def\crr{\Red}		  % Red  VERY-Approx PANTONE RED
\def\cor{\OrangeRed}	  % OrangeRed  No PANTONE match
\def\paru{\RubineRed}	  % RubineRed  Approximate PANTONE RUBINE-RED
\def\cwi{\WildStrawberry}  % WildStrawberry  Approximate PANTONE 206
\def\csa{\Salmon}	  % Salmon  Approximate PANTONE 183
\def\ccp{\CarnationPink}	  % CarnationPink  Approximate PANTONE 218
\def\cmag{\Magenta}	  % Magenta  Approximate PANTONE PROCESS-MAGENTA
\def\cvr{\VioletRed}	  % VioletRed  Approximate PANTONE 219
\def\parh{\Rhodamine}	  % Rhodamine  Approximate PANTONE RHODAMINE-RED
\def\cmu{\Mulberry}	  % Mulberry  Approximate PANTONE 241
\def\parv{\RedViolet}	  % RedViolet  Approximate PANTONE 234
\def\cfu{\Fuchsia}	  % Fuchsia  Approximate PANTONE 248
\def\cla{\Lavender}	  % Lavender  Approximate PANTONE 223
\def\cth{\Thistle}	  % Thistle  Approximate PANTONE 245
\def\corc{\Orchid}	  % Orchid  Approximate PANTONE 252
\def\cdo{\DarkOrchid}	  % DarkOrchid  No PANTONE match
\def\cpu{\Purple}	  % Purple  Approximate PANTONE PURPLE
\def\cpl{\Plum}		  % Plum  VERY-Approx PANTONE 518
\def\cvi{\Violet}	  % Violet  Approximate PANTONE VIOLET
\def\clrp{\RoyalPurple}	  % RoyalPurple  Approximate PANTONE 267
\def\cbv{\BlueViolet}	  % BlueViolet  Approximate PANTONE 2755
\def\cpe{\Periwinkle}	  % Periwinkle  Approximate PANTONE 2715
\def\ccb{\CadetBlue}	  % CadetBlue  Approximate PANTONE (534+535)/2
\def\cco{\CornflowerBlue}  % CornflowerBlue  Approximate PANTONE 292
\def\cmb{\MidnightBlue}	  % MidnightBlue  Approximate PANTONE 302
\def\cnb{\NavyBlue}	  % NavyBlue  Approximate PANTONE 293
\def\crb{\RoyalBlue}	  % RoyalBlue  No PANTONE match
%\def\cbb{\Blue}		  % Blue  Approximate PANTONE BLUE-072
\def\cce{\Cerulean}	  % Cerulean  Approximate PANTONE 3005
\def\ccy{\Cyan}		  % Cyan  Approximate PANTONE PROCESS-CYAN
\def\cpb{\ProcessBlue}	  % ProcessBlue  Approximate PANTONE PROCESS-BLUE
\def\csb{\SkyBlue}	  % SkyBlue  Approximate PANTONE 2985
\def\ctu{\Turquoise}	  % Turquoise  Approximate PANTONE (312+313)/2
\def\ctb{\TealBlue}	  % TealBlue  Approximate PANTONE 3145
\def\caq{\Aquamarine}	  % Aquamarine  Approximate PANTONE 3135
\def\cbg{\BlueGreen}	  % BlueGreen  Approximate PANTONE 320
\def\cem{\Emerald}	  % Emerald  No PANTONE match
%\def\cjg{\JungleGreen}	  % JungleGreen  Approximate PANTONE 328
\def\csg{\SeaGreen}	  % SeaGreen  Approximate PANTONE 3268
\def\cgg{\Green}	  % Green  VERY-Approx PANTONE GREEN
\def\cfg{\ForestGreen}	  % ForestGreen  Approximate PANTONE 349
\def\cpg{\PineGreen}	  % PineGreen  Approximate PANTONE 323
\def\clg{\LimeGreen}	  % LimeGreen  No PANTONE match
\def\cyg{\YellowGreen}	  % YellowGreen  Approximate PANTONE 375
\def\cspg{\SpringGreen}	  % SpringGreen  Approximate PANTONE 381
\def\cog{\OliveGreen}	  % OliveGreen  Approximate PANTONE 582
\def\pars{\RawSienna}	  % RawSienna  Approximate PANTONE 154
\def\cse{\Sepia}		  % Sepia  Approximate PANTONE 161
\def\cbr{\Brown}		  % Brown  Approximate PANTONE 1615
\def\cta{\Tan}		  % Tan  No PANTONE match
\def\cgr{\Gray}		  % Gray  Approximate PANTONE COOL-GRAY-8
\def\cbl{\Black}		  % Black  Approximate PANTONE PROCESS-BLACK
\def\cwh{\White}		  % White  No PANTONE match


\loadmsbm

\input epsf

\def\ctln{\centerline}
\def\u{\underbar}
\def\ssk{\smallskip}
\def\msk{\medskip}
\def\bsk{\bigskip}
\def\hsk{\hskip.1in}
\def\hhsk{\hskip.2in}
\def\dsl{\displaystyle}
\def\hskp{\hskip1.5in}

\def\lra{$\Leftrightarrow$ }
\def\ra{\rightarrow}
\def\mpto{\logmapsto}
\def\pu{\pi_1}
\def\mpu{$\pi_1$}
\def\sig{\Sigma}
\def\msig{$\Sigma$}
\def\ep{\epsilon}
\def\sset{\subseteq}
\def\del{\partial}
\def\inv{^{-1}}
\def\wtl{\widetilde}
\def\lra{\Leftrightarrow}
\def\del{\partial}
\def\delp{\partial^\prime}
\def\delpp{\partial^{\prime\prime}}



\ctln{\bf Math 971 Algebraic Topology}

\ssk

\ctln{March 22, 2005}

\msk

Singular homology groups are very quick to define, but what do they measure?
The basic idea is that we are trying to mimic simplicial homology, but because a
general topological space $X$ cannot be thought of as being built out of simplices,
we do the next best thing; we study the space by \u{mapping} simplices \u{in}. Formally,
this is what we did with simplicial homology anyway, except that we restricted 
ourselves to a very few special singular simplices (the characteristic maps of the
building blocks for $X$). In the end an $n$-cycle $\sum a_i\sigma_i^n$, since the 
faces of the $\sigma_i$ must match up precisely, in order to cancel in the sum, 
can be thought of as a map of an $n$-complex into $X$, made by gluing the
$n$-simplices $\sigma_i$ together \u{before} mapping in. The fact that faces
cancel really means that these simplices restrict to the same maps on their faces.
The integer coefficients can really be interpreted as taking multiple copies of 
$\Delta^n$ and gluing them together along their boundaries (the signs tell us
the underlying orientations). The idea is that this $n$-complex is being
mapped ``around a hole'', unless it \u{extends} to a map of an $(n+1)$-complex
into $X$ (having our $n$-complex as boundary). So singular homology really
is trying to detect holes, it is just doing it with maps.....

\msk

The ``fun'' with singular homology groups, though, comes when you try to compute them. 
$C_n(X) = \{\sum a_i\sigma_i$ : $a_i\in {\Bbb Z}$ and $\sigma_i:\Delta^n\ra X$ 
is continuous$\}$ is typically a \u{huge} group, since there will be immense
numbers of maps $\Delta^n\ra X$ . About the only space for which this is not true is
the one-point space $*$; then there is, for each $n$, exactly one (distinct)
map $\sigma_n :\Delta^n\ra *$ ; the constant map. Therefore each face of $\Delta^n$
gives the same restriction map $\sigma^{n-1}$, and so the boundary maps can 
be dirctly computed (the depend on the parity of the number $n+1$ of faces 
an $n$-simplex has). We find that $\del_{2n}=Id$ and $\del_{2n-1}=0$ . so in 
computing homology groups, we either have kernel everything ($\del_i=0$) and
image everything ($\del_{i+1}=Id$) or kernel nothing ($\del_i=Id$) and
image nothing ($\del_{i+1}=0$), so in both cases $H_i(*)=0$ . Except for $i=0$;
then $\del_0=0$ (by definition) and $\del_1=0$, so $H_0(*)={\Bbb Z}$ .
But other than this example (and, well, OK, spaces with the discrete topology;
it's the same calculation as above for every point!), computing singular 
homology from the definition is quite a chore! so we need to build
some labor-saving devices, namely, some theorems to help us break the problem
of computing these groups into smaller, more managable pieces.

\msk

First set of managable pieces: if we decompose $X$ into its path components, $X=\bigcup X_\alpha$,
then $H_i(X) \cong \bigoplus H_i(X_\alpha)$ for every $i$. This is because every singular simplex,
since $\Delta^i$ is path-connected, maps into some $X_\alpha$ . So $C_i(X) \cong \bigoplus C_i(X_\alpha)$.
Since the boundary of a simplex mapping into $X_\alpha$ consists of simplices in $X_\alpha$, the 
boundary maps respect the decomposistions of the chain groups, so 
$B_i(X) \cong \bigoplus B_i(X_\alpha)$ and $Z_i(X) \cong \bigoplus Z_i(X_\alpha)$, and so 
the quotients are $H_i(X) \cong \bigoplus H_i(X_\alpha)$ . 

\msk

So, if we wish to, we can focus on computing homology groups for path-connected spaces $X$. For such a space, 
$H_0(X)\cong {\Bbb Z}$, generated by any map of a 0-simplex (= a point) into $X$. This is because any pair
of 0-simplices are homologous; given any two points $x,y\in X$, there is a path $\gamma: I\ra X$ from $x$ to $y$,
This path can be interpreted as a singular 1-simplex, and $\del\gamma = y-x$ . So $H_0(X)$ is generated
by a single point $[x]$ . No multiple of this point is null-homologous, because for any 1-chain $\sum n_i \sigma_i$,
the sum of the coefficients of its boundary is 0 (since this is true for each singular 1-simplex), and any 0-chain
$\sum n_i [x_i]$ is homologous to $(\sum n_i)[x]$ by the above argument.

\msk

A small techincal aside: the fact that $H_0(*)={\Bbb Z}$ is annoying to some,
and often requires treating 0-dimensional homology as a special case. 
But since the boundary of a singular 1-simplex is always of the form $v-w$, we find that the 
image of $\del_1$ is always contained in the subgroup of $C_0(X)$ consisting
of chains whose coefficients sum to 0. This means that we can, for free, 
{\it augment} the singular chain complex by a map
$\cdots \ra C_1(X) {\del_1\atop\ra}C_0)X) {\alpha\atop \ra} {\Bbb Z} \ra 0$
where $\alpha$ is the map $\alpha(\sum a_i\sigma_i^0) = \sum a_i$ . This 
is still a chain complex (compositions of consecutive maps are 0); the resulting
homology groups are called {\it reduced} homology $\widetilde{H}_i(X)$ . 
The only affect this really has is to remove one copy of ${\Bbb Z}$ from 
$H_0$; $\widetilde{H}_0(X)\oplus {\Bbb Z} \cong H_0(X)$ . All other
homology groups are unchanged. 


\vfill
\end