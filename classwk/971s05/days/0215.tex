
\magnification=1200
\overfullrule=0pt
\parindent=0pt

\nopagenumbers

\input amstex

\voffset=-.6in
\hoffset=-.5in
\hsize = 7.5 true in
\vsize=10.4 true in

%\voffset=1.4in
%\hoffset=-.5in
%\hsize = 10.2 true in
%\vsize=8 true in

\input colordvi

\loadmsbm

\input epsf

\def\ctln{\centerline}
\def\u{\underbar}
\def\ssk{\smallskip}
\def\msk{\medskip}
\def\bsk{\bigskip}
\def\hsk{\hskip.1in}
\def\hhsk{\hskip.2in}
\def\dsl{\displaystyle}
\def\hskp{\hskip1.5in}

\def\lra{$\Leftrightarrow$ }
\def\ra{\rightarrow}
\def\mpto{\logmapsto}
\def\pu{\pi_1}
\def\mpu{$\pi_1$}
\def\sig{\Sigma}
\def\msig{$\Sigma$}
\def\ep{\epsilon}
\def\sset{\subseteq}
\def\del{\partial}
\def\inv{^{-1}}
\def\wtl{\widetilde}



\ctln{\bf Math 971 Algebraic Topology}

\ssk

\ctln{February 15, 2005}

\msk

The {\bf proof} of the homotopy lifting property follows a pattern 
that we will become 
very familiar with: we lift maps a little bit at a time. For every $x\in X$
there is an open set ${\Cal U}_x$ evenly covered by $p$ . For each fixed
$y\in Y$, since $I$ is compact and the sets $H^{-1}({\Cal U}_x)$ form an
open cover of $Y\times I$, then since $I$ is compact, 
the Tube Lemma provides an open neighborhood 
${\Cal V}$ of $y$ in $Y$ and finitely many $p^{-1}{\Cal U}_{x}$ whose union
covers ${\Cal V}\times I$ . 

\msk

To define $\wtl{H}(y,t)$, we (using a Lebesgue number argument) cut the
interval $\{y\}\times I$ into finitely many pieces, the $i$th mapping into 
 ${\Cal U}_{x_i}$ under $H$. $\wtl{f}(y)$ is in one of the evenly covered
sets ${\Cal U}_{x_1\alpha_1}$, and the restricted map 
$p^{-1}:{\Cal U}_{x_1}\ra {\Cal U}_{x_1\alpha_1}$ following $H$ restricted
to the first interval lifts $H$ along the first interval to a map 
we will call $\wtl{H}$. We then have 
lifted $H$ at the end of the first interval = the beginning of the second, 
and we continue as before. In this way we can define $\wtl{H}$ for all
$(y,t)$ . To show that this is independent of the choices we have
made along the way, we imagine two ways of cutting up the interval 
$\{y\}\times I$ using evenly covered neighborhoods ${\Cal U}_{x_i}$
and ${\Cal V}_{w_j}$, and take intersections of both sets of intervals
to get a common refinement of both sets, covered by the intersections
${\Cal U}_{x_i}\cap {\Cal V}_{w_j}$, and imagine building $\wtl{H}$ using
the refinement. At the start, at $\wtl{f}(y)$, we are in 
${\Cal U}_{x_1\alpha_1}\cap {\Cal V}_{w_1\beta_1}$. Because at the 
start of the lift $(y,0)$ we lift to the same point, and $p^{-1}$ restricted 
to this intersection agrees with $p^{-1}$ restricted to each of the two
pieces, we get the same lift acroos the first refined subinterval. This
process repeats itself across all of the subintervals, showing that
the lift is independent of the choices made. This also shows that
the lift is unique; once we have decided what $\wtl{H}(y,0)$, the
rest of the values of the $\wtl{H}$ are determined by the requirement
of being a lift. also, once we know the map is well-defined, we can see
that it is continuous, since for any $y$, we can make the same choices
across the entire open set $V$ given by the Tube Lemma, and find
that $\wtl{H}$, restricted to ${\Cal V}\times(a_i-\delta,b_i+\delta)$
(for a small delta; we could wiggle the endpoints in the construction
without changing the resulting function, by its well-definedness)
is $H$ estricted to this set followed by $p^{-1}$ restriced in domain 
and range, so this composition is continuous. So $\wtl{H}$
is locally continuous, hence continuous.

\msk

So, for example, if we build a 5-sheeted cover of the bouquet of 2 circles, 
as before, (after choosing a maximal tree upstairs) 
we can read off the images of the generators of the fundamental group
of the total space; we have labelled each ede by the ereator it
traces out downstairs, and for each ede outside of the maximal tree
chosen, we read from basepoint out the tree to one end, across the edge,
and then back to the basepoint in the tree. In our example, this
gives:


\msk

\ctln{$<ab,aaab^{-1}, baba^{-1},baa,ba^{-1}bab^{-1},bba^{-1}b^{-1} | >$}

\msk

\leavevmode


\epsfxsize=3in
\ctln{{\epsfbox{0208f2.ai}}}


\bsk

This is (from its construction) a copy of the free group on 6 letters,
in the free group $F(a,b)$ . In a similar way, by explicitly building
a covering space, we find that the fundamental group of a closed 
surface of genus 3 is a subgroup of the fundamental group of the 
closed surface of genus 2. 

\msk

The cardinality of a point inverse $p\inv(y)$ is, by the evenly
covered property, constant on (small) open sets, so the set of 
points of $x$ whose point inverses have any given cardinality
is open. Consequently, if $X$ is connected, this number
is constant over all of $X$, and is called the number of {\it sheets}
of the covering $p:\wtl{X}\ra X$ . 

\vfill
\end

