
\magnification=1100
\overfullrule=0pt
\parindent=0pt

\nopagenumbers

\input amstex

\voffset=-.6in
\hoffset=-.5in
\hsize = 7.5 true in
\vsize=10.4 true in

%\voffset=1.4in
%\hoffset=-.5in
%\hsize = 10.2 true in
%\vsize=8 true in

\input colordvi

\def\cltr{\Red}		  % Red  VERY-Approx PANTONE RED
\def\cltb{\Blue}		  % Blue  Approximate PANTONE BLUE-072
\def\cltg{\PineGreen}	  % ForestGreen  Approximate PANTONE 349
\def\cltp{\DarkOrchid}	  % DarkOrchid  No PANTONE match
\def\clto{\Orange}	  % Orange  Approximate PANTONE ORANGE-021
\def\cltpk{\CarnationPink}	  % CarnationPink  Approximate PANTONE 218
\def\clts{\Salmon}	  % Salmon  Approximate PANTONE 183
\def\cltbb{\TealBlue}	  % TealBlue  Approximate PANTONE 3145
\def\cltrp{\RoyalPurple}	  % RoyalPurple  Approximate PANTONE 267
\def\cltp{\Purple}	  % Purple  Approximate PANTONE PURPLE

\def\cgy{\GreenYellow}     % GreenYellow  Approximate PANTONE 388
\def\cyy{\Yellow}	  % Yellow  Approximate PANTONE YELLOW
\def\cgo{\Goldenrod}	  % Goldenrod  Approximate PANTONE 109
\def\cda{\Dandelion}	  % Dandelion  Approximate PANTONE 123
\def\capr{\Apricot}	  % Apricot  Approximate PANTONE 1565
\def\cpe{\Peach}		  % Peach  Approximate PANTONE 164
\def\cme{\Melon}		  % Melon  Approximate PANTONE 177
\def\cyo{\YellowOrange}	  % YellowOrange  Approximate PANTONE 130
\def\coo{\Orange}	  % Orange  Approximate PANTONE ORANGE-021
\def\cbo{\BurntOrange}	  % BurntOrange  Approximate PANTONE 388
\def\cbs{\Bittersweet}	  % Bittersweet  Approximate PANTONE 167
%\def\creo{\RedOrange}	  % RedOrange  Approximate PANTONE 179
\def\cma{\Mahogany}	  % Mahogany  Approximate PANTONE 484
\def\cmr{\Maroon}	  % Maroon  Approximate PANTONE 201
\def\cbr{\BrickRed}	  % BrickRed  Approximate PANTONE 1805
\def\crr{\Red}		  % Red  VERY-Approx PANTONE RED
\def\cor{\OrangeRed}	  % OrangeRed  No PANTONE match
\def\paru{\RubineRed}	  % RubineRed  Approximate PANTONE RUBINE-RED
\def\cwi{\WildStrawberry}  % WildStrawberry  Approximate PANTONE 206
\def\csa{\Salmon}	  % Salmon  Approximate PANTONE 183
\def\ccp{\CarnationPink}	  % CarnationPink  Approximate PANTONE 218
\def\cmag{\Magenta}	  % Magenta  Approximate PANTONE PROCESS-MAGENTA
\def\cvr{\VioletRed}	  % VioletRed  Approximate PANTONE 219
\def\parh{\Rhodamine}	  % Rhodamine  Approximate PANTONE RHODAMINE-RED
\def\cmu{\Mulberry}	  % Mulberry  Approximate PANTONE 241
\def\parv{\RedViolet}	  % RedViolet  Approximate PANTONE 234
\def\cfu{\Fuchsia}	  % Fuchsia  Approximate PANTONE 248
\def\cla{\Lavender}	  % Lavender  Approximate PANTONE 223
\def\cth{\Thistle}	  % Thistle  Approximate PANTONE 245
\def\corc{\Orchid}	  % Orchid  Approximate PANTONE 252
\def\cdo{\DarkOrchid}	  % DarkOrchid  No PANTONE match
\def\cpu{\Purple}	  % Purple  Approximate PANTONE PURPLE
\def\cpl{\Plum}		  % Plum  VERY-Approx PANTONE 518
\def\cvi{\Violet}	  % Violet  Approximate PANTONE VIOLET
\def\clrp{\RoyalPurple}	  % RoyalPurple  Approximate PANTONE 267
\def\cbv{\BlueViolet}	  % BlueViolet  Approximate PANTONE 2755
\def\cpe{\Periwinkle}	  % Periwinkle  Approximate PANTONE 2715
\def\ccb{\CadetBlue}	  % CadetBlue  Approximate PANTONE (534+535)/2
\def\cco{\CornflowerBlue}  % CornflowerBlue  Approximate PANTONE 292
\def\cmb{\MidnightBlue}	  % MidnightBlue  Approximate PANTONE 302
\def\cnb{\NavyBlue}	  % NavyBlue  Approximate PANTONE 293
\def\crb{\RoyalBlue}	  % RoyalBlue  No PANTONE match
%\def\cbb{\Blue}		  % Blue  Approximate PANTONE BLUE-072
\def\cce{\Cerulean}	  % Cerulean  Approximate PANTONE 3005
\def\ccy{\Cyan}		  % Cyan  Approximate PANTONE PROCESS-CYAN
\def\cpb{\ProcessBlue}	  % ProcessBlue  Approximate PANTONE PROCESS-BLUE
\def\csb{\SkyBlue}	  % SkyBlue  Approximate PANTONE 2985
\def\ctu{\Turquoise}	  % Turquoise  Approximate PANTONE (312+313)/2
\def\ctb{\TealBlue}	  % TealBlue  Approximate PANTONE 3145
\def\caq{\Aquamarine}	  % Aquamarine  Approximate PANTONE 3135
\def\cbg{\BlueGreen}	  % BlueGreen  Approximate PANTONE 320
\def\cem{\Emerald}	  % Emerald  No PANTONE match
%\def\cjg{\JungleGreen}	  % JungleGreen  Approximate PANTONE 328
\def\csg{\SeaGreen}	  % SeaGreen  Approximate PANTONE 3268
\def\cgg{\Green}	  % Green  VERY-Approx PANTONE GREEN
\def\cfg{\ForestGreen}	  % ForestGreen  Approximate PANTONE 349
\def\cpg{\PineGreen}	  % PineGreen  Approximate PANTONE 323
\def\clg{\LimeGreen}	  % LimeGreen  No PANTONE match
\def\cyg{\YellowGreen}	  % YellowGreen  Approximate PANTONE 375
\def\cspg{\SpringGreen}	  % SpringGreen  Approximate PANTONE 381
\def\cog{\OliveGreen}	  % OliveGreen  Approximate PANTONE 582
\def\pars{\RawSienna}	  % RawSienna  Approximate PANTONE 154
\def\cse{\Sepia}		  % Sepia  Approximate PANTONE 161
\def\cbr{\Brown}		  % Brown  Approximate PANTONE 1615
\def\cta{\Tan}		  % Tan  No PANTONE match
\def\cgr{\Gray}		  % Gray  Approximate PANTONE COOL-GRAY-8
\def\cbl{\Black}		  % Black  Approximate PANTONE PROCESS-BLACK
\def\cwh{\White}		  % White  No PANTONE match


\loadmsbm

\input epsf

\def\ctln{\centerline}
\def\u{\underbar}
\def\ssk{\smallskip}
\def\msk{\medskip}
\def\bsk{\bigskip}
\def\hsk{\hskip.1in}
\def\hhsk{\hskip.2in}
\def\dsl{\displaystyle}
\def\hskp{\hskip1.5in}

\def\lra{$\Leftrightarrow$ }
\def\ra{\rightarrow}
\def\mpto{\logmapsto}
\def\pu{\pi_1}
\def\mpu{$\pi_1$}
\def\sig{\Sigma}
\def\msig{$\Sigma$}
\def\ep{\epsilon}
\def\sset{\subseteq}
\def\del{\partial}
\def\inv{^{-1}}
\def\wtl{\widetilde}
\def\lra{\Leftrightarrow}
\def\del{\partial}
\def\delp{\partial^\prime}
\def\delpp{\partial^{\prime\prime}}
\def\sgn{{\roman{sgn}}}
\def\wtih{\widetilde{H}}
\def\bbz{{\Bbb Z}}
\def\bbr{{\Bbb R}}



\ctln{\bf Math 971 Algebraic Topology}

\ssk

\ctln{April 26, 2005}

\msk


The Hurewicz map $H:\pu(X)\ra H_1(X)$ induces, when $X$ is path-connected,
an isomorphism from $\pu(X)/[\pu(X),\pu(X)]$ to $H_1(X)$ . 
This result can be used in two ways; knowing a (presentation for) $\pu(X)$
allows us to compute $H_1(X)$, by writing the relators additively, giving
$H_1(X)$ as the free abelian group on the generators, modulo the kernel 
of the ``presentation matrix'' given by the resulting linear equations. Conversely,
knowing $H_1(X)$ provides information about $\pu(X)$. For example,
a calculation on the way to invariance of domain implied that for every
knot $K$ in $S^3$ (i.e., the image of an embedding $h:S^1\hookrightarrow S^3$),
$H_1(S^3\setminus K) \cong bbz$ . This implies that the abelianization of 
$G_K = \pu(S^3\setminus K)$ (i.e., the largest abelian quotient of $G_K$  is $\bbz$.
But this in turn implies that for every integer $n\geq 2$, there is a \u{unique}
surjective homomorphism $G_K\ra \bbz_n$, since such a homomorphism must factor 
through the abelianization, and there is exactly one surjective homomorphism
$\bbz \ra \bbz_n$ ! Consequently, there is a unique (normal) subroup (the kernel
of this homomorphism) $K_n\sset G_K$ with quotient $\bbz_n$ . Using the Galois
correspondence, there is a (unique) covering space $X_n$ of $X=S^3\setminus K$
corresponding to $K_n$, called the $n$-fold cyclic covering of $K$ . This space is determined
by $K$ and $n$, and so its homology groups are determined by the same data.
And even though homology cannot distinguish between two knot complements,
$K$, $K^\prime$, it might be the case that homology \u{can} distinguish between 
their cyclic coverings. Consequently, 
if $H_1(X_n)\not\cong H_1(X_n^\prime)$, then $K$ and $K^\prime$ have non-homeomorphic
complement, and so represent ``different'' embeddings, hence different knots.
In practice, one can compute presentations for $\pu(X_n)$ (in several different ways),
and so one can compute $H_1(X_n)$, providing an effective way to use homology
to distinguish knots! This approach was ultimately formalized (by Alexander) into a polynomial
invariant of knots, known as the Alexander polynomial.

\msk

Computing the homology of the cyclic coverings can be done in several ways. The
Reidemeister-Schreier method will allow one to compute a presentation for the
kernel of a homomorphism $\varphi:G\ra H$, given a presentation of $G$ and a
{\it transversal} of the map, which is a representative of each coset of $G$ modulo
the kernel. Abelianizing this will give homology computation. Another approach
uses {\it Seifert surfaces}, orientable surfaces with $\del \Sigma = K$, to cut
$S^3\setminus K$ open along. Writing $S^3\setminus K = (S^3\setminus N(\Sigma))\cup N(\Sigma)$
allows us to use Mayer-Vietoris to compute homology. But the cyclic covering spaces
can be built by ``unwinding'' this view of $S^3\setminus K$; instead of gluing the two
ends of $N(K)$ to the same $S^3\setminus N(\Sigma)$, we can take $n$ copies
of $S^3\setminus N(\Sigma)$ and glue them together in a circle. Mayer-Vietoris
again tells us how to compute the homology of the resulting space. Details may be found on the accompanying
pages taken from Rolfsen's ``Knots and Links''.




\vfill
\end


A map of pairs $f: (X,A) \ra (Y,B)$ (meaning that $f(A)\subseteq B$)
induces (by postcomposition) a map of relative homology $f_*:H_i(X,A)\ra H_i(Y,B)$ , just as with 
absolute homology.
We also get a homotopy-invariance result: if $f,g: (X,A) \ra (Y,B)$ are maps of pairs
which are {\it homotopic as maps of pairs},
i.e., there is a map $(X\times I,A\times I)\ra (Y,B)$ which is $f$ on one end and $g$
on the other, then $f_*=g_*$ . The proof is identical to the one given before; the prism map 
$P$ sends chains in $A$ to chains in $A$, so induces a map $C_i(X\times I,A\times I)\ra C_{i+1}(X,A)$
which does precisely what we want.








\vfill
\end
