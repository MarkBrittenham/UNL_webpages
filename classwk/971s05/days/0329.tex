

\magnification=1200
\overfullrule=0pt
\parindent=0pt

\nopagenumbers

\input amstex

\voffset=-.6in
\hoffset=-.5in
\hsize = 7.5 true in
\vsize=10.4 true in

%\voffset=1.4in
%\hoffset=-.5in
%\hsize = 10.2 true in
%\vsize=8 true in

\input colordvi

\def\cltr{\Red}		  % Red  VERY-Approx PANTONE RED
\def\cltb{\Blue}		  % Blue  Approximate PANTONE BLUE-072
\def\cltg{\PineGreen}	  % ForestGreen  Approximate PANTONE 349
\def\cltp{\DarkOrchid}	  % DarkOrchid  No PANTONE match
\def\clto{\Orange}	  % Orange  Approximate PANTONE ORANGE-021
\def\cltpk{\CarnationPink}	  % CarnationPink  Approximate PANTONE 218
\def\clts{\Salmon}	  % Salmon  Approximate PANTONE 183
\def\cltbb{\TealBlue}	  % TealBlue  Approximate PANTONE 3145
\def\cltrp{\RoyalPurple}	  % RoyalPurple  Approximate PANTONE 267
\def\cltp{\Purple}	  % Purple  Approximate PANTONE PURPLE

\def\cgy{\GreenYellow}     % GreenYellow  Approximate PANTONE 388
\def\cyy{\Yellow}	  % Yellow  Approximate PANTONE YELLOW
\def\cgo{\Goldenrod}	  % Goldenrod  Approximate PANTONE 109
\def\cda{\Dandelion}	  % Dandelion  Approximate PANTONE 123
\def\capr{\Apricot}	  % Apricot  Approximate PANTONE 1565
\def\cpe{\Peach}		  % Peach  Approximate PANTONE 164
\def\cme{\Melon}		  % Melon  Approximate PANTONE 177
\def\cyo{\YellowOrange}	  % YellowOrange  Approximate PANTONE 130
\def\coo{\Orange}	  % Orange  Approximate PANTONE ORANGE-021
\def\cbo{\BurntOrange}	  % BurntOrange  Approximate PANTONE 388
\def\cbs{\Bittersweet}	  % Bittersweet  Approximate PANTONE 167
%\def\creo{\RedOrange}	  % RedOrange  Approximate PANTONE 179
\def\cma{\Mahogany}	  % Mahogany  Approximate PANTONE 484
\def\cmr{\Maroon}	  % Maroon  Approximate PANTONE 201
\def\cbr{\BrickRed}	  % BrickRed  Approximate PANTONE 1805
\def\crr{\Red}		  % Red  VERY-Approx PANTONE RED
\def\cor{\OrangeRed}	  % OrangeRed  No PANTONE match
\def\paru{\RubineRed}	  % RubineRed  Approximate PANTONE RUBINE-RED
\def\cwi{\WildStrawberry}  % WildStrawberry  Approximate PANTONE 206
\def\csa{\Salmon}	  % Salmon  Approximate PANTONE 183
\def\ccp{\CarnationPink}	  % CarnationPink  Approximate PANTONE 218
\def\cmag{\Magenta}	  % Magenta  Approximate PANTONE PROCESS-MAGENTA
\def\cvr{\VioletRed}	  % VioletRed  Approximate PANTONE 219
\def\parh{\Rhodamine}	  % Rhodamine  Approximate PANTONE RHODAMINE-RED
\def\cmu{\Mulberry}	  % Mulberry  Approximate PANTONE 241
\def\parv{\RedViolet}	  % RedViolet  Approximate PANTONE 234
\def\cfu{\Fuchsia}	  % Fuchsia  Approximate PANTONE 248
\def\cla{\Lavender}	  % Lavender  Approximate PANTONE 223
\def\cth{\Thistle}	  % Thistle  Approximate PANTONE 245
\def\corc{\Orchid}	  % Orchid  Approximate PANTONE 252
\def\cdo{\DarkOrchid}	  % DarkOrchid  No PANTONE match
\def\cpu{\Purple}	  % Purple  Approximate PANTONE PURPLE
\def\cpl{\Plum}		  % Plum  VERY-Approx PANTONE 518
\def\cvi{\Violet}	  % Violet  Approximate PANTONE VIOLET
\def\clrp{\RoyalPurple}	  % RoyalPurple  Approximate PANTONE 267
\def\cbv{\BlueViolet}	  % BlueViolet  Approximate PANTONE 2755
\def\cpe{\Periwinkle}	  % Periwinkle  Approximate PANTONE 2715
\def\ccb{\CadetBlue}	  % CadetBlue  Approximate PANTONE (534+535)/2
\def\cco{\CornflowerBlue}  % CornflowerBlue  Approximate PANTONE 292
\def\cmb{\MidnightBlue}	  % MidnightBlue  Approximate PANTONE 302
\def\cnb{\NavyBlue}	  % NavyBlue  Approximate PANTONE 293
\def\crb{\RoyalBlue}	  % RoyalBlue  No PANTONE match
%\def\cbb{\Blue}		  % Blue  Approximate PANTONE BLUE-072
\def\cce{\Cerulean}	  % Cerulean  Approximate PANTONE 3005
\def\ccy{\Cyan}		  % Cyan  Approximate PANTONE PROCESS-CYAN
\def\cpb{\ProcessBlue}	  % ProcessBlue  Approximate PANTONE PROCESS-BLUE
\def\csb{\SkyBlue}	  % SkyBlue  Approximate PANTONE 2985
\def\ctu{\Turquoise}	  % Turquoise  Approximate PANTONE (312+313)/2
\def\ctb{\TealBlue}	  % TealBlue  Approximate PANTONE 3145
\def\caq{\Aquamarine}	  % Aquamarine  Approximate PANTONE 3135
\def\cbg{\BlueGreen}	  % BlueGreen  Approximate PANTONE 320
\def\cem{\Emerald}	  % Emerald  No PANTONE match
%\def\cjg{\JungleGreen}	  % JungleGreen  Approximate PANTONE 328
\def\csg{\SeaGreen}	  % SeaGreen  Approximate PANTONE 3268
\def\cgg{\Green}	  % Green  VERY-Approx PANTONE GREEN
\def\cfg{\ForestGreen}	  % ForestGreen  Approximate PANTONE 349
\def\cpg{\PineGreen}	  % PineGreen  Approximate PANTONE 323
\def\clg{\LimeGreen}	  % LimeGreen  No PANTONE match
\def\cyg{\YellowGreen}	  % YellowGreen  Approximate PANTONE 375
\def\cspg{\SpringGreen}	  % SpringGreen  Approximate PANTONE 381
\def\cog{\OliveGreen}	  % OliveGreen  Approximate PANTONE 582
\def\pars{\RawSienna}	  % RawSienna  Approximate PANTONE 154
\def\cse{\Sepia}		  % Sepia  Approximate PANTONE 161
\def\cbr{\Brown}		  % Brown  Approximate PANTONE 1615
\def\cta{\Tan}		  % Tan  No PANTONE match
\def\cgr{\Gray}		  % Gray  Approximate PANTONE COOL-GRAY-8
\def\cbl{\Black}		  % Black  Approximate PANTONE PROCESS-BLACK
\def\cwh{\White}		  % White  No PANTONE match


\loadmsbm

\input epsf

\def\ctln{\centerline}
\def\u{\underbar}
\def\ssk{\smallskip}
\def\msk{\medskip}
\def\bsk{\bigskip}
\def\hsk{\hskip.1in}
\def\hhsk{\hskip.2in}
\def\dsl{\displaystyle}
\def\hskp{\hskip1.5in}

\def\lra{$\Leftrightarrow$ }
\def\ra{\rightarrow}
\def\mpto{\logmapsto}
\def\pu{\pi_1}
\def\mpu{$\pi_1$}
\def\sig{\Sigma}
\def\msig{$\Sigma$}
\def\ep{\epsilon}
\def\sset{\subseteq}
\def\del{\partial}
\def\inv{^{-1}}
\def\wtl{\widetilde}
\def\lra{\Leftrightarrow}
\def\del{\partial}
\def\delp{\partial^\prime}
\def\delpp{\partial^{\prime\prime}}
\def\sgn{{\roman{sgn}}}



\ctln{\bf Math 971 Algebraic Topology}

\ssk

\ctln{March 29, 2005}

\msk


The main tool we will use turns a family of short exact sequences of chain maps
between three chain complexes into a single {\it long exact homology sequence}.
Given chain complexes ${\Cal A}=(A_n,\del)$ , 
${\Cal B}=(B_n,\del^\prime)$ , and ${\Cal C}=(C_n,\del^{\prime\prime})$
and short exact sequences of chain maps (i.e., 
$\del^\prime i_n  = i_n\del $ , $\del^{\prime\prime}j_n = j_n\del^\prime $)
\hhsk

$0\ra A_n{i_n\atop \ra}B_n{j_n\atop \ra}C_n\ra 0$
\hhsk
there is a general result which provides us with a long exact sequence

\ctln{$\cdots {\del \atop \ra} H_n({\Cal A}) {i_{*}\atop \ra} H_n({\Cal B})
{j_{*}\atop \ra} H_n({\Cal C}) {\del\atop\ra} H_{n-1}({\Cal A}) {i_{*}\atop\ra} \cdots$}

Most of the work is in defining the ``boundary'' map $\del$. Given an 
element $[z]\in H_n({\Cal C})$, a representative $z\in C_n$ satisfies 
$\del^{\prime\prime}(z)=0$. But $j_n$ is onto, so there is a $b\in B_n$ with
$j_n(b)=z$, Then $ i_{n-1}\del^\prime(b) = \del^{\prime\prime}j_n(b)
=0$, so $\del^\prime(b)\in\ker(j_{n-1}=$im$(a_{n-1})$. So there is an $a\in A_{n-1}$
with $i_{n-1}(a)=\del^\prime(b)$ . But then 
$i_{n-2}\del (a) = \del^\prime i_{n-1}(a)=\del^\prime\del^\prime(b)=0$,
so, since $i_{n-2}$ is injective, $\del a=0$, so $a\in Z_{n-1}({\Cal A})$, and
so represents a homology class $[a]\in H_n({\Cal A})$. We define
$\del([z])=[a]$ . 


To show that this is well-defined, we need to show that the
class $[a]$ we end up with is independent of the choices made along the 
way. The choice of $a$ was not really a choice; $i_{n-1}$ is, by assumption, 
injective. For $b$, if $j_n(b)=z=j_n(b^\prime)$, then
$j_n(b-b^\prime)=0$, so $b-b^\prime=i_n(w)$ for some $w\in A_n$. Then 
$\del^\prime b^\prime = \del^\prime b - \del^\prime i_n(w) = 
\del^\prime b - i_{n-1} \del(w)$, so choosing $a^\prime = a-\del(w)$ we have
$i_{n-1}(a^\prime)=\del^\prime(b^\prime)$. But then
$[a^\prime]=[a-\del w] =[a] -[del w] =[a]$. Finally, there is actually a choice
of $z$ ; if $[z]=[z^\prime]$, then $z^\prime = z+\del^{\prime\prime}w$
for some $w\in C_{n+1}$; but then choosing $b^\prime,w^\prime$ with 
$j_n(b^\prime)=z^\prime$ , $j_{n+1}(w^\prime)=w$ , we have 

$\del^{\prime\prime}w=\del^{\prime\prime}j_{n+1}(w^\prime) = j_n\del^\prime(w^\prime)$ ,
so 

$z^\prime= z+\del^{\prime\prime}w= j_n(b+\del^\prime w^\prime)$, so we may choose
$b^\prime = b+\del^\prime w^\prime$ (since the result is independent of this choice!),
then since $\del^\prime b^\prime = \del^\prime b$ everything continues the same.

\msk

Now to exactness! We need to show three (types of) equalities, which means six
containments. Three (image contained in kernel) 
are shown basically by showing that compositions of
two consecutive homomorphisms are trivial. $j_ni_n=0$ 
immediately implies $j_*i_*=0$ . From the definition of $\del$,
$i_*\del[z] = [i_n(a)] = [\del^\prime(b)] = 0$, and 
$\del j_*[z] = \del[j_n(z)] = [a]$, where $i_{n-1}(a)=\del^\prime(z) = 0$,
so $a=0$ (since $i_{n-1}$ is injective), so $[a]=0$. 

\ssk

For the opposite containments,
if $j_*[z]=[j_n(z)]=0$, then $j_n(z)=\del^{\prime\prime}w$ for some $w$. 
Since $j_{n+1}$ is onto, $w=j_{n+1}(b)$ for some $b$. Then 
$j_n(z-\delp b) = \delpp w-\delpp j_{n+1} b = 0$, so 
$z=\delp b = i_n(a)$ for some $a$, so $i_*[a] = [z-\delp b] = [z]$ . 
So $\ker j_*\subseteq$im$i_*$ . If $i_*[z]=0$, then $i_n(z)=\delp w$ for some $w\in B_{n+1}$.
Setting $c=j_{n+1}(w)$, then $\delpp c = j_n \delp w - i_n i_n(Z) = 0$, so 
$[c]\in h_{n+1}({\Cal C})$, and computing $\del [c]$ we find that we can choose $w$ for the 
first step and $z$ for the second step, so $\del [c] = [z]$ . So $\ker j_n\subseteq$im$\del$ .
Finally, if $\del [z] = 0$, then $z=j_n(b)$ for some $b$, and $\delp b = i_{n-1}(a)$ with
$[a]=0$, i.e., $a=\del w$ for some $w$. So $\delp b = i_{n-1}\del w = \delp i_n w$ But
then $\delp (b-i_n w) = 0$, and 
$j_n(b-i_n w) = z-0 = z$, so $z\in$im$(j_n)$, so $[z]\in$im$(j_*)$ . So 
$\ker\del\subseteq$im$(j_n)$ . Which finishes the proof!

\msk

Now all we need are some new chain complexes. To start, we build the singular chain complex
of a pair $(X,A)$ , i.e., of a space $X$ and a subspace $A\subseteq X$ .
Since as abelian groups we can think of 
$C_n(A)$ as a subgroup of $C_n(X)$ (under the injective homomorphism induced by the 
inclusion $i:A\ra X$) we can set $C_n(X,A)= C_n(X)/C_n(A)$ . Since the
boundary map $\del_n:C_n(X)\ra C_{n-1}(X)$ satisfies
$\del_n(C_n(A)\subseteq C_{n-1}(A)$ (the boundary of a map into $A$ maps into $A$),
we get an induced boundary map $\del_n:C_n(X,A)\ra C_{n-1}(X,A)$ . These
groups and maps $(C_n(X,A),\del_n)$ form a chain complex, whose homology groups 
are the {\it singluar relative homology groups of the pair} $(X,A)$ . 






\vfill
\end