

\magnification=1200
\overfullrule=0pt
\parindent=0pt

\nopagenumbers

\input amstex

\voffset=-.6in
\hoffset=-.5in
\hsize = 7.5 true in
\vsize=10.4 true in

%\voffset=1.4in
%\hoffset=-.5in
%\hsize = 10.2 true in
%\vsize=8 true in

\input colordvi

\loadmsbm

\input epsf

\def\ctln{\centerline}
\def\u{\underbar}
\def\ssk{\smallskip}
\def\msk{\medskip}
\def\bsk{\bigskip}
\def\hsk{\hskip.1in}
\def\hhsk{\hskip.2in}
\def\dsl{\displaystyle}
\def\hskp{\hskip1.5in}

\def\lra{$\Leftrightarrow$ }
\def\ra{\rightarrow}
\def\mpto{\logmapsto}
\def\pu{\pi_1}
\def\mpu{$\pi_1$}
\def\sig{\Sigma}
\def\msig{$\Sigma$}
\def\ep{\epsilon}
\def\sset{\subseteq}
\def\del{\partial}




\ctln{\bf Math 971 Algebraic Topology}

\ssk

\ctln{February 3, 2005}

\msk

\msk

We now turn our attention to proving Seifert - van Kampen; understanding
the kernel of the map $\phi : \pu(X_1)*\pu(X_2)\ra\pu(X)$ , 
under the hypotheses
that $X_1,X_2$ are open, $A=X_1\cap X_2$ is path-connected, and the
basepoint $x_0\in A$ . So we start with a product $g = g_1\cdots g_n$ 
of loops alternately in $X_1$ and $X_2$, which when thought of in $X$
is null-homotopic. We wish to show that $g$ can be expressed as a 
product of conjugates of elements of the form $i_{1*}(a)(i_{2*}(a))^{-1}$
(and their inverses). The basic idea is that a ``big'' homotopy can be
viewed as a large number of ``little'' homotopies, which
we essentially deal with one at a time, and we find out how
little ``little'' is by using the same Lebesgue number agument that we
used before.

\msk

Specifically, if $H$ is the homotopy, rel basepoint, from 
$\gamma_1*\cdots *\gamma_n$,
where $\gamma_i$ is a based loop representing $g_i$, and the constant
loop, then, as before, $\{H^{-1}(X_1),H^{-1}(X_2)\}$ is an open cover 
of $I\times I$, and so has a Lebesgue number $\ep$. If we cut
$I\times I$ into subsquares, with length $1/N$ on a side, where $1/N<\ep$,
then each subsquare maps into either $X_1$ or $X_2$. The idea is to 
think of this as a collection of horizontal strips, each cut into squares.
Arguing by induction, starting from the bottom (where our conclusion
will be obvious), we will argue that if the bottom of the strip
can be expressed as an element of the group 

$N = <i_{1*}(\gamma)(i_{2*}(\gamma))^{-1} : 
\gamma\in\pu(A) >^N\sset \pu(X_1)*\pu(X_2)$ 

(i.e., as a product
of conjugates of such loops), then so can the 
top of the strip. 

\msk

\leavevmode

\epsfxsize=4in
\ctln{{\epsfbox{0203f1.ai}}}

\msk

And to do \underbar{this}, we work as before. We have a strip of squares,
each mapping into either $X_1$ or $X_2$. If adjacent squares map into the
same subpace, amalgamate them into a single larger rectangle. Continuing
in this way, we can break the strip into subrectangles which
alternately map into $X_1$ or $X_2$. This means that the vertical arcs
in between map into $X_1\cap X_2 = A$, and represent paths $\eta_i$ in 
$A$. Their endpoints also map into $A$,
and so can be joined by paths ($\delta_i$  on the top, $\epsilon_i$ 
on the bottom) in $A$ to the basepoint. The top of the strip is
homotopic, rel basepoint, to 

$(\alpha_1*\delta_1)*(\overline{\delta_1}*\alpha_2*\delta_2)*\cdots *
(\overline{\delta_{k-1}}*\alpha_k)$

each grouping mapping into either $X_1$ or $X_2$.
The rectangles demonstrate that each grouping is homotopic, rel basepoint,
to the product of loops

$(\overline{\delta_i}*\eta_i*\epsilon_i)*
(\overline{\epsilon_i}*\beta_i*\epsilon_{i+1})*
(\overline{\epsilon_{i+1}}*\overline{\eta_{i+1}}*\delta_{i+1})
=a_i b_i a_{i+1}^{-1}$

where this is thought of as a product in either $\pu(X_1)$ or
$\pu(X_2)$. The point is that when strung together, this appears
to give $(b_1a_2^{-1})(a_2b_2a_3^{-1})\cdots (a_kb_k)$ , with lots
of cancellation, but in reality, the terms $a_i^{-1}a_i$ represent
elements of $N$, since the two ``cancelling'' 
factors are thought of as living in the
different groups $\pu(X_1),\pu(X_2)$. The remaining terms, if we
delete these ``cancelling'' pairs, is 
$b_1\cdots b_k = 
\beta_1*\epsilon_1*\cdots *\overline{\epsilon_i}*\beta_i*\epsilon_{i+1}
*\cdots * \overline{\epsilon_k}*\beta_k$, which is homotopic
rel endpoints to $\beta_1*\cdots *\beta_k$, which, by induction, can 
be represented as a product which lies in $N$. 

\msk




\leavevmode


\epsfxsize=3.5in
\ctln{{\epsfbox{0203f2.ai}}}

\bsk

So, we can obtain the element represented by the top of the strip by 
inserting elements of $N$ into the bottom, which 
is a word having a representation as an 
element of $N$. The final problem to overcome is that the
insertions represented by the vertical arcs might not be occuring
where we want them to be! But this doesn't matter; inserting a word $w$
in the middle of another $uv$ (to get $uwv$) 
is the same as multiplying $uv$ by a conjugate of $w$;
$uwv = (uv)(v^{-1}wv)$, so since the bottom of the strip is 
in $N$, and we obtain the top of the strip by inserting elements
of $N$ into the bottom, the top is represented by a product of 
conjugates of elements of $N$, so (since $N$ is normal) is in $N$.
And a \underbar{final} final point; the subrectangles may not have 
cut the bottom of the strip up into the same pieces that the inductive
hypothesis used to express the bottom as an element of $N$. It didn't even 
cut it into loops; we added paths at the break points to make that happen.
The inductive hypothesis would have, in fact, added its own extra paths,
at possibly different points!
But if we add \underbar{both} sets of paths, 
and cut the loop up into even more pieces, 
then we end up with a loop, which we have expressed as a product in
$\pu(X_1)*\pu(X_2)$ in two (possibly different) ways, since the 
two points of view will have interpreted pieces as living in 
different subspaces. But when this happens, it must be because
the subloop really lives in $X_1\cap X_2=A$. Moving from one
to the other amounts to repeatedly changing ownership between the 
two sets, which in $\pu(X_1)*\pu(X_2)$ means \underbar{inserting}
an element of $N$ into the product (that is literally what elements
of $N$ do). But as before, these insertions can be collected at one
end as products of conjugates. So if one of the elements is in $N$, 
the other one is, too.

\msk

Which completes the proof! 

\vfill
\end








