

\magnification=1200
\overfullrule=0pt
\parindent=0pt

\nopagenumbers

\input amstex

\voffset=-.6in
\hoffset=-.5in
\hsize = 7.5 true in
\vsize=10.4 true in

%\voffset=1.4in
%\hoffset=-.5in
%\hsize = 10.2 true in
%\vsize=8 true in

\input colordvi

\loadmsbm

\input epsf

\def\ctln{\centerline}
\def\u{\underbar}
\def\ssk{\smallskip}
\def\msk{\medskip}
\def\bsk{\bigskip}
\def\hsk{\hskip.1in}
\def\hhsk{\hskip.2in}
\def\dsl{\displaystyle}
\def\hskp{\hskip1.5in}

\def\lra{$\Leftrightarrow$ }
\def\ra{\rightarrow}
\def\mpto{\logmapsto}
\def\pu{\pi_1}
\def\mpu{$\pi_1$}
\def\sig{\Sigma}
\def\msig{$\Sigma$}
\def\ep{\epsilon}
\def\sset{\subseteq}
\def\del{\partial}




\ctln{\bf Math 971 Algebraic Topology}

\ssk

\ctln{February 3, 2005}

\msk


{\it Gluing on a 2-disk:} If $X$ is a topological 
space and $f:\del {\Bbb D}^2\ra X$ is continuous, then we
can construct the quotient space $Z=(X\coprod 
{\Bbb D}^2)/\{x\sim f(x) : x\in\del{\Bbb D}^2\}$,
the result of gluing ${\Bbb D}^2$ to $X$ along $f$. 
We can use Seifert - van Kampen to compute \mpu\ 
of the resulting space, although if we
wish to be careful with basepoints $x_0$ 
(e.g., the image of $f$ might not contain $x_0$, and/or we
may wish to glue several disks on, in remote parts of $X$),
we should also include a rectangle $R$, the mapping 
cylinder of a path $\gamma$ running from 
$f(1,0)$ to $x_0$, glued to 
${\Bbb D}^2$ along the arc from $(1/2,0)$ to $(1,0)$ (see figure). 
This space $Z_+$ deformation retracts to $Z$, but it
is technically simpler to do our calculations 
with the basepoint $y_0$ lying above $x_0$.
If we write $D_1 = \{x\in {\Bbb D}^2 : ||x||<1\}\cup(R\setminus X)$ 
and $D_2 = \{x\in {\Bbb D}^2 : ||x||>1/3\}\cup R$ , 
then we can write $Z_+=D_+\cup(X\cup D_2) = X_1\cup X_2$.
But since $X_1\simeq *$ , $X_2\simeq X$ 
(it is essentially the mapping cylinder of 
the maps $f$ and $\gamma$ )
and $X_1\cap X_2 = \{x\in {\Bbb D}^2 : 
1/3<||x||<1\}\cap(R\setminus X)\sim S^1$, we find that 

\ctln{$\pu(Z,y_0)\cong \pu(X_2,y_0)*_{\Bbb Z}\{1\} = 
\pu(X_2)/<{\Bbb Z}>^N \cong 
\pu(X_2)/<[\overline{\delta}*\overline{\gamma}*f*\gamma*\delta]>^N$}

If we then use $\delta$ as a path for a change of 
basepoint isomorphism, and then a homotopy
equivalence from $X_2$ to $X$ (fixing $x_0$), we 
have, in terms of group presentations, 
if $\pu(X,x_0)=<\sig | R>$ , then $\pu(Z) = 
<\sig | R\cup\{[\overline{\gamma}*f*\gamma]\}>$ . 
So the effect of gluing on a 2-disk on the fundamental 
group is to add a new relator, 
namely the word represented by the attaching map 
(adjusting for basepoint). 
All of this applies equally well to attaching several 
2-disks; each adds a new relator. 

\msk


\leavevmode

\epsfxsize=5.4in
\ctln{{\epsfbox{0201f1.ai}}}



\bsk

The inherent complications above derived from needing open
sets can be legislated away, by introducing additional
hypotheses:

\msk

{\bf Theorem:} If $X=X_1\cup X_2$ is a union of closed
sets $X_1,X_2$, with $A=X_1\cap X_2$ path-connected, 
and if $X_1,X_2$ have open neighborhood ${\Cal U}_1,{\Cal U}_2$
so that ${\Cal U}_1,{\Cal U}_2,{\Cal U}_1\cap{\Cal U}_2$
deformation retract onto $X_1,X_2,A$ respectively, then 
$\pu(X)\cong \pu(X_1)*_{\pu(A)}\pu(X_2)$ as before.

\msk

The hypotheses are satisfied, for example, if $X_1.X_2$ are subcomplexes of the
cell complex $X$.

\msk


This in turn opens up huge possibilities for the 
computation of $\pu(X)$. For example, for cell complexes,
we can inductively compute \mpu\ by starting with 
the 1-skeleton, with free fundamental group, and 
attaching the 2-cells one by one, which each add 
a relator to the presentation of $\pu(X)$ . 
[{\bf Exercise:}
(Hatcher, p.53, \#\ 6) Attaching $n$-cells, for 
$n\geq 3$, has no effect on \mpu .] 
For example, the 2-sphere $S^2$ can be thought of as a
2-disk with a 2-disk attached, along a circle, and so has
$\pu(S^2)\cong \{1\}_{\Bbb Z}\{1\}=\{1\}$ . 
We can also compute the fundamental group of 
any compact surface:

\msk

The {\it real projective plane} ${\Bbb R}P^2$ is the quotient of 
the 2-sphere $S^2$ by the antipodal map $x\mapsto -x$; 
it can also be thought of as the upper hemisphere, with 
identification only along the boundary. This in turn can be 
interpreted as a 2-disk glued to a circle, whose boundary
wraps around the circle twice. So $\pu({\Bbb R}P^2)\cong 
<a | a^2>\cong {\Bbb Z}_2 = {\Bbb Z}/2{\Bbb Z}$ .
A surface $F$ of genus 2 can be given a cell structure with 1 0-cell,
4 1-cells, and 1 2-cell, as in the figure, as in the first of the 
figures below. The fundamental group
of the 1-skeleton is therefore free of rank 4, and $\pu(F)$ has
a presentation with 4 generators and 1 relator. Reading the attaching
map from the figure, the presentation is $<a,b,c,d | [a,b][c,d] >$ .

\msk



\leavevmode

\epsfxsize=5.6in
\ctln{{\epsfbox{0201f2.ai}}}

\msk

Giving it a different cell structure, as in the second figure, with 2 0-cells,
6 1-cels, and 2 2-cells, after choosing a maximal tree, we can read off the
two relators from the 2-cells to arrive at a different presentation
$\pu(F) = <a,b,c,d,e | aba^{-1}eb^{-1},cde^{-1}c^{-1}d^{-1}>$ . A posteriori,
these two presentations describe isomorphic groups.

\bsk

Using the same technology, we can also see that, in general, 
any group is the fundamental group of some 2-complex $X$;
starting with a presentation $G = <\sig | R>$, build $X$ by starting
with a bouquet of $|\sig|$ circles, and attach $|R|$ 2-disks along
loops which represent each of the generators of $R$. (This works just
as well for infinite sets $\sig$ and/or $R$; essentially the same proofs
as above apply.)

\bsk

{\bf Understanding that darn kernel.}

\msk

We now turn our attention to proving Seifert - van Kampen; understanding
the kernel of the map $\phi : \pu(X_1)*\pu(X_2)\ra\pu(X)$ , 
under the hypotheses
that $X_1,X_2$ are open, $A=X_1\cap X_2$ is path-connected, and the
basepoint $x_0\in A$ . So we start with a product $g = g_1\cdots g_n$ 
of loops alternately in $X_1$ and $X_2$, which when thought of in $X$
is null-homotopic. We wish to show that $g$ can be expressed as a 
product of conjugates of elements of the form $i_{1*}(a)(i_{2*}(a))^{-1}$
(and their inverses). The basic idea is that a ``big'' homotopy can be
viewed as a large number of ``little'' homotopies, which
we essentially deal with one at a time, and we find out how
little ``little'' is by using the same Lebesgue number agument that we
used before.

\msk

Specifically, if $H$ is the homotopy, rel basepoint, from 
$\gamma_1*\cdots *\gamma_n$,
where $\gamma_i$ is a based loop representing $g_i$, and the constant
loop, then, as before, $\{H^{-1}(X_1),H^{-1}(X_2)\}$ is an open cover 
of $I\times I$, and so has a Lebesgue number $\ep$. If we cut
$I\times I$ into subsquares, with length $1/N$ on a side, where $1/N<\ep$,
then each subsquare maps into either $X_1$ or $X_2$. The idea is to 
think of this as a collection of horizontal strips, each cut into squares.
Arguing by induction, starting from the bottom (where our conclusion
will be obvious), we will argue that if the bottom of the strip
can be expressed as an element of the group 

$N = <i_{1*}(\gamma)(i_{2*}(\gamma))^{-1} : 
\gamma\in\pu(A) >^N\sset \pu(X_1)*\pu(X_2)$ 

(i.e., as a product
of conjugates of such loops), then so can the 
top of the strip. 

\msk





\leavevmode

\epsfxsize=4in
\ctln{{\epsfbox{0203f1.ai}}}

\msk

And to do \underbar{this}, we work as before. We have a strip of squares,
each mapping into either $X_1$ or $X_2$. If adjacent squares map into the
same subpace, amalgamate them into a single larger rectangle. Continuing
in this way, we can break the strip into subrectangles which
alternately map into $X_1$ or $X_2$. This means that the vertical arcs
in between map into $X_1\cap X_2 = A$, and represent paths $\eta_i$ in 
$A$. Their endpoints also map into $A$,
and so can be joined by paths ($\delta_i$  on the top, $\epsilon_i$ 
on the bottom) in $A$ to the basepoint. The top of the strip is
homotopic, rel basepoint, to 

$(\alpha_1*\delta_1)*(\overline{\delta_1}*\alpha_2*\delta_2)*\cdots *
(\overline{\delta_{k-1}}*\alpha_k)$

each grouping mapping into either $X_1$ or $X_2$.
The rectangles demonstrate that each grouping is homotopic, rel basepoint,
to the product of loops

$(\overline{\delta_i}*\eta_i*\epsilon_i)*
(\overline{\epsilon_i}*\beta_i*\epsilon_{i+1})*
(\overline{\epsilon_{i+1}}*\overline{\eta_{i+1}}*\delta_{i+1})
=a_i b_i a_{i+1}^{-1}$

where this is thought of as a product in either $\pi(X_1)$ or
$\pu(X_2)$. The point is that when strung together, this appears
to give $(b_1a_2^{-1})(a_2b_2a_3^{-1})\cdots (a_kb_k)$ , with lots
of cancellation, but in reality, the terms $a_i^{-1}a_i$ represent
elements of $N$, since the two ``cancelling'' 
factors are thought of as living in the
different groups $\pu(X_1),\pu(X_2)$. The remaining terms, if we
delete these ``cancelling'' pairs, is 
$b_1\cdots b_k = 
\beta_1*\epsilon_1*\cdots *\overline{\epsilon_i}*\beta_i*\epsilon_{i+1}
*\cdots * \overline{\epsilon_k}*\beta_k$, which is homotopic
rel endpoints to $\beta_1*\cdots *\beta_k$, which, by induction, can 
be represented as a product which lies in $N$. 

\msk




\leavevmode


\epsfxsize=3.5in
\ctln{{\epsfbox{0203f2.ai}}}

\bsk

So, we can obtain the element represented by the top of the strip by 
inserting elements of $N$ into the bottom, which 
is a word having a representation as an 
element of $N$. The final problem to overcome is that the
insertions represented by the vertical arcs might not be occuring
where we want them to be! But this doesn't matter; inserting a word $w$
in the middle of another $uv$ (to get $uwv$) 
is the same as multiplying $uv$ by a conjugate of $w$;
$uwv = (uv)(v^{-1}wv)$, so since the bottom of the strip is 
in $N$, and we obtain the top of the strip by inserting elements
of $N$ into the bottom, the top is represented by a product of 
conjugates of elements of $N$, so (since $N$ is normal) is in $N$.
And a \underbar{final} final point; the subrectangles may not have 
cut the bottom of the strip up into the same pieces that the inductive
hypothesis used to express the bottom as an element of $N$. It didn't even 
cut it into loops; we added paths at the break points to make that happen.
The inductive hypothesis would have, in fact, added its own extra paths,
at possibly different points!
But if we add \underbar{both} sets of paths, 
and cut the loop up into even more pieces, 
then we end up with a loop, which we have expressed as a product in
$\pu(X_1)*\pu(X_2)$ in two (possibly different) ways, since the 
two points of view will have interpreted pieces as living in 
different subspaces. But when this happens, it must be because
the subloop really lives in $X_1\cap X_2=A$. Moving from one
to the other amounts to repeatedly changing ownership between the 
two sets, which in $\pu(X_1)*\pu(X_2)$ means \underbar{inserting}
an element of $N$ into the product (that is literally what elements
of $N$ do). But as before, these insertions can be collected at one
end as products of conjugates. So if one of the elements is in $N$, 
the other one is, too.

\msk

Which completes the proof! 

\bsk

{\bf Postscript:  why should we care?} The role of the fundamental group
in distinguishing spaces has already been touched upon; if two 
(path-connected) spaces have non-isomorphic fundamental groups, then
the spaces are not homeomorphic, and even not homotopy equivalent.
It is one of the most basic, and in many cases the best such invariant
we have in our arsenal
(hence the name ``fundamental''). As we have seen with the circle, it
captures the notion of how many times a loop ``winds around'' in a space.
And the idea of using paths to understand a space is very basic; we 
explore a space by mapping familiar objects into it. (This is 
a theme we keep returning to in this course.) The concepts we 
have introduced play a role in analysis, for instance with the notion
of a path integral; the invariance of the integral under homotopies
rel endpoints is an important property, related to Green's Theorem
and (locally) conservative vector fields. And the \underbar{space}
of all paths in $X$ plays an important (theoretical, although
pprobably not practical) role in what we will do next.

\bsk

{\bf Covering spaces:} We can motivate our next topic by looking more
closely at one of our examples above. The projective plane ${\Bbb R}P^2$
has $\pu = {\Bbb Z}_2$ . It is also the quotient of the simply-connected
space $S^2$ by the antipodal map, which, together with the identity map,
forms a group of homeomorphisms of $S^2$ which is isomorphic to ${\Bbb Z}_2$.
The fact that ${\Bbb Z}_2$ has this dual role to play in describing 
${\Bbb R}P^2$ is no accident; codifying this relationship requires the 
notion of a covering space.

\msk

The quotient map $q:S^2\ra {\Bbb R}P^2$ is an example of a {\it covering map}.
A map $p:E\ra B$ is called a covering map if for every point $x\in B$, there
is a neighborhood ${\Cal U}$ of $x$ (an
{\it evenly covered neighborhood}) so that $p^{-1}({\Cal U})$ 
is a disjoint union ${\Cal U}_\alpha$ of open sets in $E$, each mapped
homeomorphically onto ${\Cal U}$ by (the restriction of) $p$ . $B$ is
called the {\it base space} of the covering; $E$ is called the {\it total
space}. The quotient map $q$ is an example; (the image of) the complement
of a great circle in $S^2$ will be an evenly covered neighborhood
of any point it contains. The disjoint union of 43 copies of a space,
each mapping homeomorphically to a single copy, is an example of a 
{\it trivial covering}. As a last example, we have the famous 
exponential map $p:{\Bbb R}\ra S^1$ given by $t\mapsto e^{2\pi it} = 
(\cos (2\pi t),\sin (2\pi t))$. The image of any interval $(a,b)$ of length
less than 1 will have inverse image the disjoint union of the
intervals $(a+n,b+n)$ for $n\in{\Bbb Z}$ .

\msk

OK, maybe not the last. We can build many finite-sheeted (every point
inverse is finite) coverings of a bouquet of two circles, say, by 
assembling $n$ points over the vertex, and then, on either side,
connecting the points by $n$ (oriented) arcs, one each going in and out of
each vertex. By choosing orientations on each 1-cell of the bouquet,
we can build a covering map by sending the vertices above to the
vertex, and the arcs to the one cells, homeomorphically, respecting 
the orientations. We can build infinite-sheeted coverings in much 
the same way.

\msk

\leavevmode


\epsfxsize=3in
\ctln{{\epsfbox{0208f1.ai}}}


\bsk

Covering spaces of a (suitably nice) space $X$ have a very close relationship
to $\pu(X,x_0)$. The basis for this relationship is the

\msk 

{\bf Homotopy Lifting Property:} If $p:\widetilde{X}\ra X$ is a covering map, 
$H:Y\times I\ra X$ is a homotopy, $H(y,0)=f(y)$, and
$\widetilde{f}:Y\ra \widetilde{X}$ is a {\it lift} of $f$ (i.e., $p\circ \widetilde{f}=f$),
then there is a unique lift $\widetilde{H}$ of $H$ with $\widetilde{H}(y,0)=\widetilde{f}(y)$ .

\msk

The {\bf proof} of this property follows a pattern that we will become 
very familiar with: we lift maps a little bit at a time. For every $x\in X$
there is an open set ${\Cal U}_x$ evenly covered by $p$ . For each fixed
$y\in Y$, since $I$ is compact and the sets $H^{-1}({\Cal U}_x)$ form an
open cover of $Y\times I$, then since $I$ is compact, 
the Tube Lemma provides an open neighborhood 
${\Cal V}$ of $y$ in $Y$ and finitely many $p^{-1}{\Cal U}_{x}$ whose union
covers ${\Cal V}\times I$ . 

\msk

To define $\widetilde{H}(y,t)$, we (using a Lebesgue number argument) cut the
interval $\{y\}\times I$ into finitely many pieces, the $i$th mapping into 
 ${\Cal U}_{x_i}$ under $H$. $\widetilde{f}(y)$ is in one of the evenly covered
sets ${\Cal U}_{x_1\alpha_1}$, and the restricted map 
$p^{-1}:{\Cal U}_{x_1}\ra {\Cal U}_{x_1\alpha_1}$ following $H$ restricted
to the first interval lifts $H$ along the first interval to a map 
we will call $\widetilde{H}$. We then have 
lifted $H$ at the end of the first interval = the beginning of the second, 
and we continue as before. In this way we can define $\widetilde{H}$ for all
$(y,t)$ . To show that this is independent of the choices we have
made along the way, we imagine two ways of cutting up the interval 
$\{y\}\times I$ using evenly covered neighborhoods ${\Cal U}_{x_i}$
and ${\Cal V}_{w_j}$, and take intersections of both sets of intervals
to get a common refinement of both sets, covered by the intersections
${\Cal U}_{x_i}\cap {\Cal V}_{w_j}$, and imagine building $\widetilde{H}$ using
the refinement. At the start, at $\widetilde{f}(y)$, we are in 
${\Cal U}_{x_1\alpha_1}\cap {\Cal V}_{w_1\beta_1}$. Because at the 
start of the lift $(y,0)$ we lift to the same point, and $p^{-1}$ restricted 
to this intersection agrees with $p^{-1}$ restricted to each of the two
pieces, we get the same lift acroos the first refined subinterval. This
process repeats itself across all of the subintervals, showing that
the lift is independent of the choices made. This also shows that
the lift is unique; once we have decided what $\widetilde{H}(y,0)$, the
rest of the values of the $\widetilde{H}$ are determined by the requirement
of being a lift. also, once we know the map is well-defined, we can see
that it is continuous, since for any $y$, we can make the same choices
across the entire open set $V$ given by the Tube Lemma, and find
that $\widetilde{H}$, restricted to ${\Cal V}\times(a_i-\delta,b_i+\delta)$
(for a small delta; we could wiggle the endpoints in the construction
without changing the resulting function, by its well-definedness)
is $H$ estricted to this set followed by $p^{-1}$ restriced in domain 
and range, so this composition is continuous. So $\widetilde{H}$
is locally continuous, hence continuous.

\bsk

 In particular, applying this in the case $Y=\{*\}$, where a homotopy
$H:\{*\}\times I\ra X$ is generally thought of as a path $\gamma:I\ra X$,
we have the {\bf Path Lifting Property}: ``given 
a covering map $p:\widetilde{X}\ra X$, a path 
$\gamma:I\ra X$ with $\gamma(0)=x_0$, and a point 
$\widetilde{x}_0\in p^{-1}(x_0)$, there is a unique path $\widetilde{\gamma}$
lifting $\gamma$ with $\widetilde{\gamma}(0)=\widetilde{x}_0$ .'' One of the 
immediate consequences of this is one of the cornerstones of covering space
theory:

\bsk  

%%
%%
%%Homework: p.39: 5*,13,15,19,20* p.53: 6,7,8*
%%
%%

If $p:(\widetilde{X},\tilde{x}_0)\ra (X,x_0)$ is a covering map, then the 
induced homomorphism $p_*:\pu(\widetilde{X},\widetilde{x}_0)\ra \pu(X,x_0)$ is
injective.

\msk

{\bf Proof:} Suppose $\gamma:(I,\del I)\ra (\widetilde{X},\widetilde{x}_0)$ is a 
loop $p_*([\gamma])=1$ in $\pu(X,x_0)$. So there is a homotopy
$H:(I\times I,\del I\times I)\ra (X,x_0)$ between $p\circ\gamma$ and the constant 
path. By homotopy lifting, there is a homotopy $\widetilde{H}$ from $\gamma$ to 
the lift of the constant map at $x_0$. The vertical sides 
$s\mapsto \widetilde{H}(0,s),\widetilde{H}(1,s)$ are also lifts of the 
constant map, beginning from 
$\widetilde{H}(0,0),\widetilde{H}(1,0)=\gamma(0)=\gamma(1)=\widetilde{x}_0$, so
are the constant map at $\widetilde{x}_0$. Consequently, the lift at the 
bottom is the constant map at $\widetilde{x}_0$. So $\widetilde{H}$
represents a null-homotopy of $\gamma$, so $[\gamma]=1$
in $\pu(\widetilde{X},\widetilde{x}_0)$. So $\pu(\widetilde{X},\widetilde{x}_0)=\{1\}$ .

\msk

Even more, the image $p_*(\pu(\widetilde{X},\widetilde{x}_0))\sset \pu(X,x_0)$
is precisely the elements whose representatives are loops at $x_0$, 
which when
lifted to paths starting at $\widetilde{x}_0)$, are loops. 
For if $\gamma$ lifts
to a loop $\widetilde{\gamma}$, then $p\circ\widetilde{\gamma}=\gamma$, so
$p_*([\widetilde{\gamma}])=[\gamma]$ . Conversely, if 
$p_*([\widetilde{\gamma}])=[\gamma]$, then $\gamma$ and $p\circ\widetilde{\gamma}$
are homotopic rel endpoints, and so the homotopy lifts to 
a homotopy rel endpoints between the lift of $\gamma$ at 
$\widetilde{x}_0$, and the lift of $p\circ\widetilde{\gamma}$ at $\widetilde{x}_0$
(which is $\widetilde{\gamma}$, since $\widetilde{\gamma}(0)=\widetilde{x}_0$ and
lifts are unique). So the lift of $\gamma$ is a loop, as desired.

\msk

So, for example, if we build a 5-sheeted cover of the bouquet of 2 circles, 
as above, (after choosin maximal tree upstairs) 
we can read off the images of the generators of the fundamental group
of the total space; we have labelled each ede by the ereator it
traces out downstairs, and for each ede outside of the maximal tree
chosen, we read from basepoint out the tree to one end, across the edge,
and then back to the basepoint in the tree. In our example, this
gives:


\msk

\ctln{$<ab,aaab^{-1}, baba^{-1},baa,ba^{-1}bab^{-1},bba^{-1}b^{-1} | >$}

\msk

\leavevmode


\epsfxsize=3in
\ctln{{\epsfbox{0208f2.ai}}}


\bsk

This is (from its construction) a copy of the free group on 6 letters,
in the free group $F(a,b)$ . In a similar way, by explicitly building
a covering space, we find that the fundamental group of a closed 
surface of genus 3 is a subgroup of the fundamental group of the 
closed surface of genus 2. 


\vfill
\end









