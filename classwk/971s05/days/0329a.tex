

\magnification=1200
\overfullrule=0pt
\parindent=0pt

\nopagenumbers

\input amstex

\voffset=-.6in
\hoffset=-.5in
\hsize = 7.5 true in
\vsize=10.4 true in

%\voffset=1.4in
%\hoffset=-.5in
%\hsize = 10.2 true in
%\vsize=8 true in

\input colordvi

\def\cltr{\Red}		  % Red  VERY-Approx PANTONE RED
\def\cltb{\Blue}		  % Blue  Approximate PANTONE BLUE-072
\def\cltg{\PineGreen}	  % ForestGreen  Approximate PANTONE 349
\def\cltp{\DarkOrchid}	  % DarkOrchid  No PANTONE match
\def\clto{\Orange}	  % Orange  Approximate PANTONE ORANGE-021
\def\cltpk{\CarnationPink}	  % CarnationPink  Approximate PANTONE 218
\def\clts{\Salmon}	  % Salmon  Approximate PANTONE 183
\def\cltbb{\TealBlue}	  % TealBlue  Approximate PANTONE 3145
\def\cltrp{\RoyalPurple}	  % RoyalPurple  Approximate PANTONE 267
\def\cltp{\Purple}	  % Purple  Approximate PANTONE PURPLE

\def\cgy{\GreenYellow}     % GreenYellow  Approximate PANTONE 388
\def\cyy{\Yellow}	  % Yellow  Approximate PANTONE YELLOW
\def\cgo{\Goldenrod}	  % Goldenrod  Approximate PANTONE 109
\def\cda{\Dandelion}	  % Dandelion  Approximate PANTONE 123
\def\capr{\Apricot}	  % Apricot  Approximate PANTONE 1565
\def\cpe{\Peach}		  % Peach  Approximate PANTONE 164
\def\cme{\Melon}		  % Melon  Approximate PANTONE 177
\def\cyo{\YellowOrange}	  % YellowOrange  Approximate PANTONE 130
\def\coo{\Orange}	  % Orange  Approximate PANTONE ORANGE-021
\def\cbo{\BurntOrange}	  % BurntOrange  Approximate PANTONE 388
\def\cbs{\Bittersweet}	  % Bittersweet  Approximate PANTONE 167
%\def\creo{\RedOrange}	  % RedOrange  Approximate PANTONE 179
\def\cma{\Mahogany}	  % Mahogany  Approximate PANTONE 484
\def\cmr{\Maroon}	  % Maroon  Approximate PANTONE 201
\def\cbr{\BrickRed}	  % BrickRed  Approximate PANTONE 1805
\def\crr{\Red}		  % Red  VERY-Approx PANTONE RED
\def\cor{\OrangeRed}	  % OrangeRed  No PANTONE match
\def\paru{\RubineRed}	  % RubineRed  Approximate PANTONE RUBINE-RED
\def\cwi{\WildStrawberry}  % WildStrawberry  Approximate PANTONE 206
\def\csa{\Salmon}	  % Salmon  Approximate PANTONE 183
\def\ccp{\CarnationPink}	  % CarnationPink  Approximate PANTONE 218
\def\cmag{\Magenta}	  % Magenta  Approximate PANTONE PROCESS-MAGENTA
\def\cvr{\VioletRed}	  % VioletRed  Approximate PANTONE 219
\def\parh{\Rhodamine}	  % Rhodamine  Approximate PANTONE RHODAMINE-RED
\def\cmu{\Mulberry}	  % Mulberry  Approximate PANTONE 241
\def\parv{\RedViolet}	  % RedViolet  Approximate PANTONE 234
\def\cfu{\Fuchsia}	  % Fuchsia  Approximate PANTONE 248
\def\cla{\Lavender}	  % Lavender  Approximate PANTONE 223
\def\cth{\Thistle}	  % Thistle  Approximate PANTONE 245
\def\corc{\Orchid}	  % Orchid  Approximate PANTONE 252
\def\cdo{\DarkOrchid}	  % DarkOrchid  No PANTONE match
\def\cpu{\Purple}	  % Purple  Approximate PANTONE PURPLE
\def\cpl{\Plum}		  % Plum  VERY-Approx PANTONE 518
\def\cvi{\Violet}	  % Violet  Approximate PANTONE VIOLET
\def\clrp{\RoyalPurple}	  % RoyalPurple  Approximate PANTONE 267
\def\cbv{\BlueViolet}	  % BlueViolet  Approximate PANTONE 2755
\def\cpe{\Periwinkle}	  % Periwinkle  Approximate PANTONE 2715
\def\ccb{\CadetBlue}	  % CadetBlue  Approximate PANTONE (534+535)/2
\def\cco{\CornflowerBlue}  % CornflowerBlue  Approximate PANTONE 292
\def\cmb{\MidnightBlue}	  % MidnightBlue  Approximate PANTONE 302
\def\cnb{\NavyBlue}	  % NavyBlue  Approximate PANTONE 293
\def\crb{\RoyalBlue}	  % RoyalBlue  No PANTONE match
%\def\cbb{\Blue}		  % Blue  Approximate PANTONE BLUE-072
\def\cce{\Cerulean}	  % Cerulean  Approximate PANTONE 3005
\def\ccy{\Cyan}		  % Cyan  Approximate PANTONE PROCESS-CYAN
\def\cpb{\ProcessBlue}	  % ProcessBlue  Approximate PANTONE PROCESS-BLUE
\def\csb{\SkyBlue}	  % SkyBlue  Approximate PANTONE 2985
\def\ctu{\Turquoise}	  % Turquoise  Approximate PANTONE (312+313)/2
\def\ctb{\TealBlue}	  % TealBlue  Approximate PANTONE 3145
\def\caq{\Aquamarine}	  % Aquamarine  Approximate PANTONE 3135
\def\cbg{\BlueGreen}	  % BlueGreen  Approximate PANTONE 320
\def\cem{\Emerald}	  % Emerald  No PANTONE match
%\def\cjg{\JungleGreen}	  % JungleGreen  Approximate PANTONE 328
\def\csg{\SeaGreen}	  % SeaGreen  Approximate PANTONE 3268
\def\cgg{\Green}	  % Green  VERY-Approx PANTONE GREEN
\def\cfg{\ForestGreen}	  % ForestGreen  Approximate PANTONE 349
\def\cpg{\PineGreen}	  % PineGreen  Approximate PANTONE 323
\def\clg{\LimeGreen}	  % LimeGreen  No PANTONE match
\def\cyg{\YellowGreen}	  % YellowGreen  Approximate PANTONE 375
\def\cspg{\SpringGreen}	  % SpringGreen  Approximate PANTONE 381
\def\cog{\OliveGreen}	  % OliveGreen  Approximate PANTONE 582
\def\pars{\RawSienna}	  % RawSienna  Approximate PANTONE 154
\def\cse{\Sepia}		  % Sepia  Approximate PANTONE 161
\def\cbr{\Brown}		  % Brown  Approximate PANTONE 1615
\def\cta{\Tan}		  % Tan  No PANTONE match
\def\cgr{\Gray}		  % Gray  Approximate PANTONE COOL-GRAY-8
\def\cbl{\Black}		  % Black  Approximate PANTONE PROCESS-BLACK
\def\cwh{\White}		  % White  No PANTONE match


\loadmsbm

\input epsf

\def\ctln{\centerline}
\def\u{\underbar}
\def\ssk{\smallskip}
\def\msk{\medskip}
\def\bsk{\bigskip}
\def\hsk{\hskip.1in}
\def\hhsk{\hskip.2in}
\def\dsl{\displaystyle}
\def\hskp{\hskip1.5in}

\def\lra{$\Leftrightarrow$ }
\def\ra{\rightarrow}
\def\mpto{\logmapsto}
\def\pu{\pi_1}
\def\mpu{$\pi_1$}
\def\sig{\Sigma}
\def\msig{$\Sigma$}
\def\ep{\epsilon}
\def\sset{\subseteq}
\def\del{\partial}
\def\inv{^{-1}}
\def\wtl{\widetilde}
\def\lra{\Leftrightarrow}
\def\del{\partial}
\def\delp{\partial^\prime}
\def\delpp{\partial^{\prime\prime}}
\def\sgn{{\roman{sgn}}}



\ctln{\bf Math 971 Algebraic Topology}

\ssk

\ctln{March 29, 2005}

\msk

The main tool we will use turns a family of short exact sequences of chain maps
between three chain complexes into a single {\it long exact homology sequence}.
Given chain complexes ${\Cal A}=(A_n,\del)$ , 
${\Cal B}=(B_n,\del^\prime)$ , and ${\Cal C}=(C_n,\del^{\prime\prime})$
and short exact sequences of chain maps (i.e., 
$\del^\prime i_n  = i_n\del $ , $\del^{\prime\prime}j_n = j_n\del^\prime $)
\hhsk

$0\ra A_n{i_n\atop \ra}B_n{j_n\atop \ra}C_n\ra 0$
\hhsk
there is a general result which provides us with a long exact sequence

\ctln{$\cdots {\del \atop \ra} H_n({\Cal A}) {i_{*}\atop \ra} H_n({\Cal B})
{j_{*}\atop \ra} H_n({\Cal C}) {\del\atop\ra} H_{n-1}({\Cal A}) {i_{*}\atop\ra} \cdots$}

Most of the work is in defining the ``boundary'' map $\del$. Given an 
element $[z]\in H_n({\Cal C})$, a representative $z\in C_n$ satisfies 
$\del^{\prime\prime}(z)=0$. But $j_n$ is onto, so there is a $b\in B_n$ with
$j_n(b)=z$, Then $ i_{n-1}\del^\prime(b) = \del^{\prime\prime}j_n(b)
=0$, so $\del^\prime(b)\in\ker(j_{n-1}=$im$(a_{n-1})$. So there is an $a\in A_{n-1}$
with $i_{n-1}(a)=\del^\prime(b)$ . But then 
$i_{n-2}\del (a) = \del^\prime i_{n-1}(a)=\del^\prime\del^\prime(b)=0$,
so, since $i_{n-2}$ is injective, $\del a=0$, so $a\in Z_{n-1}({\Cal A})$, and
so represents a homology class $[a]\in H_n({\Cal A})$. We define
$\del([z])=[a]$ . 


To show that this is well-defined, we need to show that the
class $[a]$ we end up with is independent of the choices made along the 
way. The choice of $a$ was not really a choice; $i_{n-1}$ is, by assumption, 
injective. For $b$, if $j_n(b)=z=j_n(b^\prime)$, then
$j_n(b-b^\prime)=0$, so $b-b^\prime=i_n(w)$ for some $w\in A_n$. Then 
$\del^\prime b^\prime = \del^\prime b - \del^\prime i_n(w) = 
\del^\prime b - i_{n-1} \del(w)$, so choosing $a^\prime = a-\del(w)$ we have
$i_{n-1}(a^\prime)=\del^\prime(b^\prime)$. But then
$[a^\prime]=[a-\del w] =[a] -[del w] =[a]$. Finally, there is actually a choice
of $z$ ; if $[z]=[z^\prime]$, then $z^\prime = z+\del^{\prime\prime}w$
for some $w\in C_{n+1}$; but then choosing $b^\prime,w^\prime$ with 
$j_n(b^\prime)=z^\prime$ , $j_{n+1}(w^\prime)=w$ , we have 

$\del^{\prime\prime}w=\del^{\prime\prime}j_{n+1}(w^\prime) = j_n\del^\prime(w^\prime)$ ,
so 

$z^\prime= z+\del^{\prime\prime}w= j_n(b+\del^\prime w^\prime)$, so we may choose
$b^\prime = b+\del^\prime w^\prime$ (since the result is independent of this choice!),
then since $\del^\prime b^\prime = \del^\prime b$ everything continues the same.

\msk

Now to exactness! We need to show three (types of) equalities, which means six
containments. Three (image contained in kernel) 
are shown basically by showing that compositions of
two consecutive homomorphisms are trivial. $j_ni_n=0$ 
immediately implies $j_*i_*=0$ . From the definition of $\del$,
$i_*\del[z] = [i_n(a)] = [\del^\prime(b)] = 0$, and 
$\del j_*[z] = \del[j_n(z)] = [a]$, where $i_{n-1}(a)=\del^\prime(z) = 0$,
so $a=0$ (since $i_{n-1}$ is injective), so $[a]=0$. 

\ssk

For the opposite containments,
if $j_*[z]=[j_n(z)]=0$, then $j_n(z)=\del^{\prime\prime}w$ for some $w$. 
Since $j_{n+1}$ is onto, $w=j_{n+1}(b)$ for some $b$. Then 
$j_n(z-\delp b) = \delpp w-\delpp j_{n+1} b = 0$, so 
$z=\delp b = i_n(a)$ for some $a$, so $i_*[a] = [z-\delp b] = [z]$ . 
So $\ker j_*\subseteq$im$i_*$ . If $i_*[z]=0$, then $i_n(z)=\delp w$ for some $w\in B_{n+1}$.
Setting $c=j_{n+1}(w)$, then $\delpp c = j_n \delp w - i_n i_n(Z) = 0$, so 
$[c]\in h_{n+1}({\Cal C})$, and computing $\del [c]$ we find that we can choose $w$ for the 
first step and $z$ for the second step, so $\del [c] = [z]$ . So $\ker j_n\subseteq$im$\del$ .
Finally, if $\del [z] = 0$, then $z=j_n(b)$ for some $b$, and $\delp b = i_{n-1}(a)$ with
$[a]=0$, i.e., $a=\del w$ for some $w$. So $\delp b = i_{n-1}\del w = \delp i_n w$ But
then $\delp (b-i_n w) = 0$, and 
$j_n(b-i_n w) = z-0 = z$, so $z\in$im$(j_n)$, so $[z]\in$im$(j_*)$ . So 
$\ker\del\subseteq$im$(j_n)$ . Which finishes the proof!

\msk

Now all we need are some new chain complexes. To start, we build the singular chain complex
of a pair $(X,A)$ , i.e., of a space $X$ and a subspace $A\subseteq X$ .
Since as abelian groups we can think of 
$C_n(A)$ as a subgroup of $C_n(X)$ (under the injective homomorphism induced by the 
inclusion $i:A\ra X$) we can set $C_n(X,A)= C_n(X)/C_n(A)$ . Since the
boundary map $\del_n:C_n(X)\ra C_{n-1}(X)$ satisfies
$\del_n(C_n(A)\subseteq C_{n-1}(A)$ (the boundary of a map into $A$ maps into $A$),
we get an induced boundary map $\del_n:C_n(X,A)\ra C_{n-1}(X,A)$ . These
groups and maps $(C_n(X,A),\del_n)$ form a chain complex, whose homology groups 
are the {\it singluar relative homology groups of the pair} $(X,A)$ . To be a cycle
in relative homology, you need to have a representative $z$ with $\del z\in C_{n-1}(A)$,
i.e., you are a chain with boundary in $A$. To be a boundary, you need
$z=\del w +a$ for some $w\in C_{n+1}(X)$ and $a\in C_n(A)$ , i.e., you {\it cobound}
a chain in $A$ ($\del w = z-a$). Note that the relative homology of the pair $(X,\emptyset)$
is just the ordinary homology of $X$; we aren't modding out by anything.

\ssk

%There is a reduced relative homology 
%as well, since we can augment with the same map (1-simplices always have 2 ends!),
%but in this case it has (essentially) no effect; $\widetilde{H}_i(X,A)\cong H_i(X,A)$
%for all $i$ \u{unless} $A=\emptyset$, in which case we lose the ${\Bbb Z}$ in
%dimension 0 that we expect to. 

%\ssk

The inclusion $i_n$ and projection  $p_n$ maps give us short exact sequences \hhsk
$0\ra C_n(A)\ra C_n(X) \ra C_n(X,A)\ra 0$ \hhsk, and since the boundary on chains
in $X$ restricts to the boundary on $A$ and induces the boundary on $(X,A)$,
$i_n$ and $p_n$ are chain maps. So we get a long exact homology sequence

\ssk

\ctln{$\cdots \ra H_n(A) \ra H_n(X) \ra H_n(X,A) \ra H_{n-1}(A) \ra H_{n-1}(X) \ra \cdots$}

\ssk

There is also a long exact sequence of a triple $(X,A,B)$ , where by triple we
mean $B\sset A\sset X$ . From the short exact sequences 
\hhsk $0\ra C_n(A,B) \ra C_n(X,B) \ra C_n(X,A)\ra 0$ \hhsk
(i.e., $0\ra C_n(A)/C_n(B) \ra C_n(X)/C_n(B) \ra C_n(X)/C_n(A)\ra 0$) we get the
long exact sequence 

\ssk

\ctln{$\cdots \ra H_n(A,B) \ra H_n(X,B) \ra H_n(X,A) \ra H_{n-1}(A,B) 
\ra H_{n-1}(X,B) \ra \cdots$}

\msk

A map of pairs $f: (X,A) \ra (Y,B)$ (meaning that $f(A)\subseteq B$)
induces (by postcomposition) a map of relative homology $f_*:H_i(X,A)\ra H_i(Y,B)$ , just as with 
absolute homology.
We also get a homotopy-invariance result: if $f,g: (X,A) \ra (Y,B)$ are maps of pairs
which are {\it homotopic as maps of pairs},
i.e., there is a map $(X\times I,A\times I)\ra (Y,B)$ which is $f$ on one end and $g$
on the other, then $f_*=g_*$ . The proof is identical to the one given before; the prism map 
$P$ sends chains in $A$ to chains in $A$, so induces a map $C_i(X\times I,A\times I)\ra C_{i+1}(X,A)$
which does precisely what we want.

\bsk

But the big result that allows us to get our homology machine really running is what is
known as {\it excision}. To motivate it, let's try to imagine that we are trying to 
generalize Seifert - van Kampen. We start with $X=A\cup B$, and we want to try to 
express the homology of $X$ in terms of that of $A$, $B$, and $A\cap B$. With our
new-found tool of long exact homology sequences, we might try to first build
a short exact sequence out of the chain complexes $C_*(A\cap B), C_*(A), C_*(B)$, and $C_*(X)$.
If we take our cue from the \u{proof} of S-vK, we might think of chains in $X$ as sums of 
chains in $A$ and $B$, except that we mod out by chains in $A\cap B$. Putting this into action,
we might try the sequence

\ssk

\ctln{$0\ra C_n(A\cap B) \ra C_n(A)\oplus C_n(B) \ra C_n(X) \ra 0$}

\ssk

where $j_n:C_n(A)\oplus C_n(B) \ra C_n(X)$ is defined as $j_n(a,b)=a+b$ . In order to get exactness
at the middle term (i.e., image = the kernel of this map, which is $\{ (x,-x) : x\in C_n(A)\cap C_n(B)\}$),
we set $i_n:C_n(A\cap B) \ra C_n(A)\oplus C_n(B)$ to be $i_n(x) = (x,-x)$ , since
$C_n(A\cap B) = C_n(A)\cap C_n(B)$ ! $i_n$ is then injective, and we certainly have that
this sequence is exact at the middle term. But, in general, $j_n$ is far from surjective! The image of $j_n$
is the set of $n$-chains that can be expressed as sums of chains in $A$ and $B$. Which of course
not every chain in $X$ can be; singular simplices in $X$ need not map entirely
into either $A$ or $B$. 

\msk

We can solve this by \u{replacing} $C_n(X)$ with the image of $j_n$, calling it, say,
$C_n^{\{A,B\}}(X)$ ... [Note: these \u{would} form a chain complex.]
Then we have a short exact sequence, and hence a long exact homology sequence.
But it involves a ``new'' homology group $H_n^{{A,B}\}}(X)$ . The \u{point} is that, like S-vK,
under the right conditions, this new homology is the same as $H_n(X)$ !

\msk

Starting from scratch, the idea is that, starting with an {\it open cover} $\{{\Cal U}_\alpha\}$
of $X$, we build the {\it chain groups subordinate to the cover} 
$C_n^{Cal U}(X) = \{\sum a_i \sigma_i^n : \sigma_i : n:\Delta^n\ra X , \sigma_i^n(\Delta^n)\subseteq {\Cal U}_\alpha$
for some $\alpha\}$ $\subseteq$ $C_n(X)$ . 
Since the face of any simplex mapping into ${\Cal U}_\alpha$ also maps into ${\Cal U}_\alpha$,
our ordinary boundary maps induce boundary maps on these groups, turning
$(C_n^{\Cal U}(X),\partial_n)$ into a chain complex. Our main result is that the inclusion
$i$ of these groups into $C_n(X)$ induces an isomophism on homology. And to show this, we once
again use the notion of a chain homotopy.

\msk

{\bf Theorem:} There is a chain map $b:C_n(X)\ra C_n^{\Cal U}(X)$ so that $i\circ b$ and $b\circ i$ are both chain 
homotopic to the identity. $i$ consequently induces isomorphisms on homology.

\msk

And the key to building $b$ (and the chain homotopies) is what is know as the {\it barycentric subdivision map}.
The idea is really the same as for S-vK; we cut our singular simplices up into tiny enough
pieces so that (via the Lebesgue number theorem) 
each piece maps into \u{some} ${\Cal U}_\alpha$ . Unlike S-vK, though, we want to do 
this in a more structured way, so that the cutting up process is ``compatible'' with our
boundary maps. And the best way to describe this cutting up is through {\it barycentric
coordinates}. Recall that an $n$-simplex is the set of convex linar combinations
$\sum x_i v_i$ with $x_i\geq 0$ and $\sum x_i=1$ . The map which sends \u{an}
$n$-simplex to \u{the} $n$-simplex $\Delta^n$ is literally the map 
$\sum x_i v_i \mapsto (x_0,\ldots ,x_n)$ . These are the barycentric coordinates of an $n$-simplex.
Since, formally, all singular simplices are considered to have $\Delta^n$ for their
domain, we can describe barycentric subdivision by describing how to cut up $\Delta^n$.
The idea is to build a process the is compatible with the boundary map, so that the
subdivision, when restricted to a sub-simplex, is the subdivision of that sub-simplex.
A 1-simplex $[v_0,v_1]$ is subdivided by adding the barycenter $w=(v+0+v_1)/2$ as a vertex,
cutting $[v_0,v_1]$ into two 1-simplices ,$[v_0,w]$,$[w,v_1]$ . A 2-simplex 
$[v_0,v_1,v_2]$ will, to be compatible withe boundary map,
have its boundary cutinto 6 1-simplices; using the barycenter $(v_0+v_1+v_2)/3$
we can cone off each of these 1-simplices to subdivide $[v_0,v_1,v_2]$
into 6 2-simplices. Taking the cue that $2=(1+1)!$ , $6=(2+1)!$ is probably no accident, we
might expect that an $n$-simplex will be cut into $(n+1)!$ $n$-simplices. 
Note that this is thew number of ways of ordering the vertices of our simplex. 
And following the ``pattern'' of our two test cases, where each new simplex was the convex
span of vertices chosen as (vertex) , (barycenter of a 1-simplex having (vertex) as a vertex),
(barycenter of a 2-simplex containing the previous 2 vertices), etc., we are led to the idea that
the barycentric subdivision of an $n$-simplex $[v_0,\ldots , v_n]$ is the 
$(n+1)!$ $n$-simplices, 

\ctln{$[v_{\alpha(0)},(v_{\alpha(0)}+v_{\alpha(1)})/2,(v_{\alpha(0)}+v_{\alpha(1)}+v_{\alpha(2)})/3,\ldots,
(v_{\alpha(0)}+\cdots v_{\alpha(n)})/(n+1)]$}

one for every permutation $\alpha$ of $\{0,\ldots ,n\}$ . And since we want to take into account
\u{orientations} as well, the natural thing to do is to define the barycentric subdivision of a singular
$n$-simplex $\sigma:[v_0,\ldots ,v_n]\ra X$ to be

\ctln{$\displaystyle S(\sigma) = \sum_\alpha (-1)^{\sgn (\alpha)} \sigma
|_{[v_{\alpha(0)},(v_{\alpha(0)}+v_{\alpha(1)})/2,(v_{\alpha(0)}+v_{\alpha(1)}+v_{\alpha(2)})/3,\ldots,
(v_{\alpha(0)}+\cdots v_{\alpha(n)})/(n+1)]}$}

where the sum is taken over all permutations of $\{0,\ldots ,n\}$ .
This (extending linearly over the chain group) is the subdivision operator, $S:C_n(X)\ra C_n(X)$ . 
A ``routine'' calculation establishes that $\del S = S\del$ , i.e., it is a chain map
(i.e., it behaves well on the boundary of our simplices). The point to this operator is that all of the 
subsimplices in the sum are a definite \u{factor} smaller than the original simplex. In fact,
if the diameter of $[v_0,\ldots ,v_n]$ is $d$ (the largest distance between points, which will,
because it is the convex span of the vertices, be the largest distance between vertices), then
every individual simplex in $S(\sigma)$ will have diameter at most $nd/(n+1)$ (the result of a little
Euclidean geometry and induction). So by \u{repeatedly} applying the subdivision operator
$S$ to a singular simplex, we will obtain a singular chain $S^k(\sigma)$,
which is ``really'' $\sigma$ written as a sum of tiny simplices, whose singular simplices 
have image as small as we want. Or put more succinctly, if $\{{\Cal U}_\alpha\}$ is an open cover of $X$
and $\sigma:\Delta^n\ra X$ is a singular $n$-simplex, then choosing a Lebesgue number $\epsilon$ for
the open cover $\sigma^{-1}({\Cal U}_\alpha)$ of the compact metric space $\Delta^n$, and choosing 
a $k$ with $d(n/(n+1))^k<\epsilon$, we find that $S^k(\sigma)$ is a sum of singular simplices
each of which maps into one of the ${\Cal U}_\alpha$, i.e., $S^k(\sigma)\in C_n^{\Cal U}(X)$.

\msk

This, in turn, allows us to define our chain map $b:C_n(X)\ra C_n^{\Cal U}(X)$; 
given a chain $\sum a_i \sigma_i$, we can, for each $i$ find a $k_i$ with $S^{k_i}(\sigma_i)\in C_n^{\Cal U}(X)$;
then define $b(\sum a_i \sigma_i) = \sum a_i S^{k_i}(\sigma_i)$ . If we want to make sure this is well-defined,
always choose the \u{smallest} $k_i\geq 0$ which works. Now that we have our putative homotopy-inverse,
it suffices to show that $i\circ b$ and $b\circ i$ are chain homotopic to the identity. First note that
$b\circ i$ \u{is} the identity; for every singular simplex in $C_n^{\Cal U}(X)$, 
thinking of it as lying in $C_n(X)$ we find that the corresponding $k_i=0$,
and $b$ does nothing to it. So all that we need to do is to come up with the chain homotopy
$H:C_n(X)\ra C_{n+1}(X)$ with $\del_{n+1}H+H\del_n = I-i\circ b$ . Again, we really only need to define
$H$ for a single singular $n$-simplex $\sigma$ (and extend linearly). And in the end, we only need to show how 
to define, for any $k$, an $H_k$ so that $(\del_{n+1}H_k+H_k\del_n)(\sigma) = (I-i\circ S^k)(\sigma)$ 
(and then set $H=H_k$ on that simplex, for the appropriate $k$).

\msk

And to do \u{that}, we define a map $R:C_n(X)\ra C_{n+1}(X\times I)$; when followed by the projection-induced
map $p_\# : C_{n+1}(X\times I)\ra C_{n+1}(X)$, we get a map $T:C_n(X)\ra C_{n+1}(X)$ from which we will build
$H_k$ as $H_k$ = $\sum TS^j$, where the sum is taken over $j=0,\ldots k-1$. Once we define $T$ (!) and show that
$\del T + T\del = I-S$, we will have
$\del H_k+H_k\del = \sum \del TS^j + TS^j \del = \sum (\del T+T\del) S^j =\sum (S^j-S^{j+1}) = I-S^k$
(since the last sum telescopes). And defining $R$, is, formally, just another particular sum.
Setting up some notation,
thinking of $\Delta^n\times I$ , as before, as having vertices $\{v_0,\ldots v_n\}$ on the 0-end and 
$\{w_0,\ldots ,w_n\}$ on the 1-end,  $N=\{0,\ldots ,n\}$, $\Pi(Q)$ = the group of permuations of $Q$,
and $\sigma^\prime = \sigma\times I:\Delta^n\times I\ra X\times I$), we have

\ssk

$\displaystyle R(\sigma) = 
\sum_{A\subseteq N}\sum_{\pi\in\Pi(N\setminus A)}\big\{ (-1)^{|A|}(-1)^{\sgn(\pi)}\prod_{j\in N\setminus A}(-1)^j \big\}$

\hfill $\displaystyle  \sigma^\prime
|_{[v_{i_0},\ldots ,v_{i_j},(w_{i_0}+\cdots w_{i_j})/(j+1),
(w_{i_0}+\cdots w_{i_j}+w_{\pi(i_{j+1}})/(j+2),
\ldots ,
(w_{i_0}+\cdots w_{i_j}+w_{\pi(i_{j+1})}+\cdots w_{\pi(i_n)})/(n+1)]}$

\ssk

where we sum over all \u{non-empty} subsets of $\{0,\ldots n\}$ (with the induced ordering on vertices
from the ordering on $\{0,\ldots ,n\}$).
Intuitively, this map ``interpolates'' between the simplex $[v_0,\ldots v_n]$ and the 
barycentric subdivision on $w_0,\ldots ,w_n$, by taking the (signed sums of the) convex spans of
simplices on the bottom (0) and simplices on the top (1). Again, a ``routine'' calculation will 
establish that $\del T + T\del = I-S$ , as desired. [At any rate, I verified it for n=1,2; the formula
for the sign of each simplex was determined by working backwards from these examples.]

\bsk

And the point to all of these calculations was that if $\{{\Cal U}_\alpha\}$ is an open cover of $X$, then the 
inclusions $i_n:C_n^{\Cal U}(X)\ra C_n(X)$ induce isomorphisms on homology. This gives us two
big theorems. The first is

\msk

{\bf Mayer-Vietoris Sequence}: If $X={\Cal U}\cup{\Cal V}$ is the union of two open sets, then
the short exact sequences \hhsk 
$0\ra C_n({\Cal U}\cap {\Cal V}) \ra C_n({\Cal U})\oplus C_n({\Cal V}) \ra C_n^{\{ {\Cal U},{\Cal V}\}}(X)\ra 0$
\hhsk , together with the isomorphism above, give the long exact sequence

\ctln{$\cdots \ra H_n({\Cal U}\cap {\Cal V}) {\buildrel{(i_{{\Cal U}*},-i_{{\Cal V}*})}\over\ra}
H_n({\Cal U})\oplus H_n({\Cal V}) {\buildrel{j_{{\Cal U}*}+j_{{\Cal V}*}}\over\ra}H_n(X)
{\buildrel \del\over\ra} H_{n-1}({\Cal U}\cap {\Cal V}) \ra \cdots$}

\msk

And just like Seifert - van Kampen, we can replace open sets by sets $A,B$ having neigborhoods which deformation
retract to them, and whose intersection deformation retracts to $A\cap B$. For example,
subcomplexes $A,B\sset X$ of a CW-complex, with $A\cup B = X$ have homology satisfying a long
exact sequence

\ssk

\ctln{$\cdots \ra H_n(A\cap B) {\buildrel{(i_{A*},-i_{B*})}\over\ra}
H_n(A)\oplus H_n(B) {\buildrel{j_{A*}+j_{B*}}\over\ra}H_n(X)
{\buildrel \del\over\ra} H_{n-1}(A\cap B) \ra \cdots$}

\ssk

And this is also true for reduced homology; we just augment the chain complexes used above with the 
short exact sequence \hhsk $0\ra {\Bbb Z}\ra {\Bbb Z}\oplus {\Bbb Z} \ra {\Bbb Z} \ra 0$ , 
where the first non-trivial map is $a\mapsto (a,-a)$ and the second is $(a,b)\mapsto a+b$ .

\msk

And now we can do some meaningful calculations!  An $n$-sphere $S^n$
is the union $S^n_+\cup S^n_-$ of its upper and lower hemispheres, each of which 
is contractible, and have intersection $S^n_+\cap S^n_-=S^{n-1}_0$ the equatorial
$(n-1)$-sphere. So Mayer-Vietoris gives us the exact sequence
\hhsk $\cdots \ra \widetilde{H}_k(S^n_+)\oplus \widetilde{H}_k(S^n_-) \ra \widetilde{H}_k(S^n)
\ra \widetilde{H}_{k-1}(S^{n-1}_0) \ra \widetilde{H}_{k-1}(S^n_+)\oplus \widetilde{H}_{k-1}(S^n_-)\ra \cdots$ 
hhtp , i.e, \hhsk
$0 \ra \widetilde{H}_k(S^n)\ra \widetilde{H}_{k-1}(S^{n-1}_0) \ra  0$ \hhsk 
i.e., $\widetilde{H}_k(S^n)\cong \widetilde{H}_{k-1}(S^{n-1})$ for every $k$ and $n$. 
So by induction, 

\ssk

\ctln{$\widetilde{H}_k(S^n)\cong\widetilde{H}_{k-n}(S^0)\cong    
\cases
{\Bbb Z}, & if $k=n$\cr
0, & otherwise $ $\cr 
\endcases$}

\msk

And this, in turn, let's us prove a fairly sizable theorem:

\ssk

{\bf Brouwer Fixed Point Theorem}: For every $n$, every
map $f:{\Bbb D}^n\ra {\Bbb D}^n$ has a fixed point.

\ssk

{\bf Proof:} If $f(x)\neq x$ for every $x$, then is with the $n=2$ case
that you may have seen before, we can construct a retraction
$r:{\Bbb D}^n \ra \del {\Bbb D}^n = S^{n-1}$ by setting
$r(x)$ = the (first) point past $f(x)$ where the ray from $f(x)$ to $x$ meets
$\del {\Bbb D}^n$ . This function is continuous, and is the 
identity on the boundary. So from our of your problem sets, the 
inclusion-induced map $i_*: H_{n-1}(S^n) \ra H_{n-1}({\Bbb D}^n)$ 
is injective. But this is impossible, since the first group is ${\Bbb Z}$ and the
second is $0$ .

\bsk

The second result that this machinery gives us is what is properly known as {\it excision}:

\msk

If $B\sset A\sset X$ and cl$_X(B)\sset$ int$_X(A)$, then for every $k$ the inclusion-induced map 
$H_k(X\setminus B,A\setminus B)\ra H_k(X,A)$ is an isomorphism. 

\msk

An equivalent formulation of this is that if $A,B\sset X$ and int$_X(A)\cup$ int$_X(B) = X$, then the
inclusion-induced map $H_k(B,A\cap B)\ra H_k(X,A)$ is an isomorphism. [From first to second
statement, set $B^\prime = X\setminus B$ .] A proof of this second result follows from a relative
version of the barycentric subdivision process; 







\vfill
\end