
\magnification=1200
\overfullrule=0pt
\parindent=0pt

\nopagenumbers

\input amstex

\voffset=-.6in
\hoffset=-.5in
\hsize = 7.5 true in
\vsize=10.4 true in

%\voffset=1.4in
%\hoffset=-.5in
%\hsize = 10.2 true in
%\vsize=8 true in

\input colordvi

\loadmsbm

\input epsf

\def\ctln{\centerline}
\def\u{\underbar}
\def\ssk{\smallskip}
\def\msk{\medskip}
\def\bsk{\bigskip}
\def\hsk{\hskip.1in}
\def\hhsk{\hskip.2in}
\def\dsl{\displaystyle}
\def\hskp{\hskip1.5in}

\def\lra{$\Leftrightarrow$ }
\def\ra{\rightarrow}
\def\mpto{\logmapsto}
\def\pu{\pi_1}
\def\mpu{$\pi_1$}
\def\sig{\Sigma}
\def\msig{$\Sigma$}
\def\ep{\epsilon}
\def\sset{\subseteq}
\def\del{\partial}
\def\inv{^{-1}}
\def\wtl{\widetilde}



\ctln{\bf Math 971 Algebraic Topology}

\ssk

\ctln{February 15, 2005}

\msk

The {\bf proof} of the homotopy lifting property follows a pattern 
that we will become 
very familiar with: we lift maps a little bit at a time. For every $x\in X$
there is an open set ${\Cal U}_x$ evenly covered by $p$ . For each fixed
$y\in Y$, since $I$ is compact and the sets $H^{-1}({\Cal U}_x)$ form an
open cover of $Y\times I$, then since $I$ is compact, 
the Tube Lemma provides an open neighborhood 
${\Cal V}$ of $y$ in $Y$ and finitely many $p^{-1}{\Cal U}_{x}$ whose union
covers ${\Cal V}\times I$ . 

\msk

To define $\wtl{H}(y,t)$, we (using a Lebesgue number argument) cut the
interval $\{y\}\times I$ into finitely many pieces, the $i$th mapping into 
 ${\Cal U}_{x_i}$ under $H$. $\wtl{f}(y)$ is in one of the evenly covered
sets ${\Cal U}_{x_1\alpha_1}$, and the restricted map 
$p^{-1}:{\Cal U}_{x_1}\ra {\Cal U}_{x_1\alpha_1}$ following $H$ restricted
to the first interval lifts $H$ along the first interval to a map 
we will call $\wtl{H}$. We then have 
lifted $H$ at the end of the first interval = the beginning of the second, 
and we continue as before. In this way we can define $\wtl{H}$ for all
$(y,t)$ . To show that this is independent of the choices we have
made along the way, we imagine two ways of cutting up the interval 
$\{y\}\times I$ using evenly covered neighborhoods ${\Cal U}_{x_i}$
and ${\Cal V}_{w_j}$, and take intersections of both sets of intervals
to get a common refinement of both sets, covered by the intersections
${\Cal U}_{x_i}\cap {\Cal V}_{w_j}$, and imagine building $\wtl{H}$ using
the refinement. At the start, at $\wtl{f}(y)$, we are in 
${\Cal U}_{x_1\alpha_1}\cap {\Cal V}_{w_1\beta_1}$. Because at the 
start of the lift $(y,0)$ we lift to the same point, and $p^{-1}$ restricted 
to this intersection agrees with $p^{-1}$ restricted to each of the two
pieces, we get the same lift acroos the first refined subinterval. This
process repeats itself across all of the subintervals, showing that
the lift is independent of the choices made. This also shows that
the lift is unique; once we have decided what $\wtl{H}(y,0)$, the
rest of the values of the $\wtl{H}$ are determined by the requirement
of being a lift. also, once we know the map is well-defined, we can see
that it is continuous, since for any $y$, we can make the same choices
across the entire open set $V$ given by the Tube Lemma, and find
that $\wtl{H}$, restricted to ${\Cal V}\times(a_i-\delta,b_i+\delta)$
(for a small delta; we could wiggle the endpoints in the construction
without changing the resulting function, by its well-definedness)
is $H$ estricted to this set followed by $p^{-1}$ restriced in domain 
and range, so this composition is continuous. So $\wtl{H}$
is locally continuous, hence continuous.

\msk

So, for example, if we build a 5-sheeted cover of the bouquet of 2 circles, 
as before, (after choosing a maximal tree upstairs) 
we can read off the images of the generators of the fundamental group
of the total space; we have labelled each ede by the ereator it
traces out downstairs, and for each ede outside of the maximal tree
chosen, we read from basepoint out the tree to one end, across the edge,
and then back to the basepoint in the tree. In our example, this
gives:


\msk

\ctln{$<ab,aaab^{-1}, baba^{-1},baa,ba^{-1}bab^{-1},bba^{-1}b^{-1} | >$}

\msk

\leavevmode


\epsfxsize=3in
\ctln{{\epsfbox{0208f2.ai}}}


\bsk

This is (from its construction) a copy of the free group on 6 letters,
in the free group $F(a,b)$ . In a similar way, by explicitly building
a covering space, we find that the fundamental group of a closed 
surface of genus 3 is a subgroup of the fundamental group of the 
closed surface of genus 2. 

\msk

The cardinality of a point inverse $p\inv(y)$ is, by the evenly
covered property, constant on (small) open sets, so the set of 
points of $x$ whose point inverses have any given cardinality
is open. Consequently, if $X$ is connected, this number
is constant over all of $X$, and is called the number of {\it sheets}
of the covering $p:\wtl{X}\ra X$ . It can also be determined 
from the fundamenatal groups:

\msk

{\bf Proposition:} If $X$ and $\wtl{X}$ are 
path-connected, then the number of sheets of a covering map equals
the index of the subgroup $H=p_*(\pu(\wtl{X},\wtl{x}_0)$ in 
$G=\pu(X,x_0)$ . 

\msk

To see this, choose loops $\{\gamma_\alpha\}$ representing representatives $\{g_\alpha\}$ of each of the (right) cosets of $H$ in $G$. Then if we lift
each of them to loops based at $\wtl{x}_0$, they will have distinct
endpoints; if $\wtl{\gamma}_1(1)=\wtl{\gamma}_2(1)$, then 
by uniqueness of lifts, $\gamma_1*\overline{\gamma_2}$ lifts to 
$\wtl{\gamma}_1*\overline{\wtl{\gamma}_2}$, so it
lifts to a loop, so $\gamma_1*\overline{\gamma_2}$ represents
an element of $p_*(\pu(\wtl{X},\wtl{x}_0)$, so $g_1=g_2$.
Conversely, every point in $p\inv(x_0)$ is the endpoint of on of these
lifts, since we can construct a path $\wtl{\gamma}$
from $\wtl{x}_0$ to any such point $y$, giving a loop
$\gamma=p\circ \wtl{\gamma}$ representing an element $g\in\pu(X,x_0)$.
But then $g=hg_\alpha$ for some $h\in H$ and $\alpha$, 
so $\gamma$ is homotopic rel endpoints to $\eta*\gamma_\alpha$ for some loop
$\eta$ representing $h$. But then lifting these based at $\wtl{x}_0$, by hmotopy
lifting, $\wtl{\gamma}$ is homotopic rel endpoints to $\wtl{\eta}$, which is a 
loop, followed by the lift $\wtl{\gamma}_\alpha$ of $\gamma_\alpha$
starting at $\wtl{x}_0$. So $\wtl{\gamma}$ and 
$\wtl{\gamma}_\alpha$ have the same value at 1.

\ssk

Therefore, lifts of representatives of coset representatives of $H$ in $G$ give
a 1-to-1 correspondence, given by $\wtl{\gamma}(1)$, with $p\inv{x_0}$.
In particular, they have the same cardinality.

\msk

The path lifting property (because $\pi([0,1],0)=\{1\}$) is actually a special
case of a more general {\bf lifting criterion}: If 
$p:(\wtl{X},\wtl{x}_0)\ra (X,x_0)$ is a covering map, and 
$f:(Y,y_0)\ra (X,x_0)$ is a map, where
$Y$ is path-connected and locally path-connected, then there is a lift 
$\wtl{f}:(Y,y_0)\ra (\wtl{X},\wtl{x}_0)$ of $f$ (i.e., 
$f=p\circ\wtl{f}$) $\Leftrightarrow$ 
$f_*(\pu(Y,y_0))\sset p_*(\pu(\wtl{X},\wtl{x}_0))$ . 
Furthermore, two lifts of $f$ which agree at a single point are equal.

\msk

If the lift exists, then $f=p\circ\wtl{f}$ implies 
$f_*=p_*\circ\wtl{f}_*$, so 
$f_*(\pu(Y,y_0)) = p_*(\wtl{f}_*(\pu(Y,y_0)))\sset p_*(\pu(\wtl{X},\wtl{x}_0))$ , as desired.
Conversely, if $f_*(\pu(Y,y_0))\sset p_*(\pu(\wtl{X},\wtl{x}_0))$,
then we wish to build the lift of $f$. Not wishing to waste our work on the
special case, we will {\it use} path lifting to do it! Given $y\in Y$,
choose a path $\gamma$ in $Y$ from $y_0$ to $y$ and use path
lifting in $X$ to lift the path $f\circ\gamma$ to a path $\wtl{f\circ\gamma}$ with 
$\wtl{f\circ\gamma}(0)=\wtl{x}_0$ . Then define
$\wtl{f}(y)=\wtl{f\circ\gamma}(1)$ . Provided we show that
this is well-defined and continuous, it is our required lift, 
since $(p\circ\wtl{f})(y) = p(\wtl{f}(y))=p(\wtl{f\circ\gamma}(1))
=p\circ\wtl{f\circ\gamma})(1) = (f\circ\gamma)(1) = f(\gamma(1)) = f(y)$. 
To show that it
is well-defined, if $\eta$ is any other path from 
$y_0$ to $y$, then $\gamma*\overline{\eta}$ is a loop in $Y$, 
so $f\circ(\gamma*\overline{\eta})=(f\circ\gamma)*\overline{(f\circ\eta)}$
is a loop in $X$ representing an element of 
$f_*(\pu(Y,y_0))\sset p_*(\pu(\wtl{X},\wtl{x}_0))$, and
so lifts to a loop in $\wtl{X}$ based at $\wtl{x}_0$.
Consequently, as before, $f\circ\gamma$ and $f\circ\eta$ lift to paths
starting at $\wtl{x}_0$ with the same value at 1. So $\wtl{f}$ is
well-defined.  To show that $\wtl{f}$ is continuous, we use the 
evenly covered property of $p$. Given $y\in Y$,
and  a neighborhood $\wtl{\Cal U}$ of 
$\wtl{f}(y)$ in $\wtl{X}$, 
we wish to find a nbhd ${\Cal V}$ of $y$ with 
$\wtl{f}({\Cal V})\sset\wtl{\Cal U}$. Choosing an evenly covered 
neighborhood ${\Cal U}_y$ for $f(y)$, choose the sheet 
$\wtl{\Cal U}_y$ over ${\Cal U}_y$ which contains $\wtl{f}(y)$,
and set ${\Cal W}=\wtl{\Cal U}\cap \wtl{\Cal U}_y$. This is open in 
$\wtl{X}$, and $p$ is a homeomorphism from this 
set to the open set $p({\Cal W})\sset X$. Then if we set ${\Cal V}^\prime= f\inv(p({\Cal W})$
this is an open set containing $y$, and so contains a path-connected open 
set ${\Cal V}$ containing $y$. Then is for every point $z\in {\Cal V}$ we build a path
$\gamma$
from $y_0$ to $z$ by concatenating a path from $y_0$ to $y$ with a path {\it in} ${\Cal V}$
from $y$ to $z$, then by unique path lifting, 
since $f({\Cal V}\sset {\Cal U}_y$ , $f\circ\gamma$ lifts to 
the concatenation of a path from $\wtl{x}_0$ to $\wtl{f}(y)$ and a 
path {\it in} $\wtl{\Cal U}_y$ from $\wtl{f}(y)$ to $\wtl{f}(z)$.
So $\wtl{f}(z)\in\wtl{\Cal U}$.

\ssk

Because $\wtl{f}$ is built by lifting paths, and path
lifting is unique, the last statement of the proposition follows.

\bsk

{\bf Universal covering spaces}: As we shall see, a particularly
important covering space to identify is one which is simply
connected. One thing we can see from the lifting crierion is
that such a covering is essentially unique:

\msk

If $X$ is locally path-connected, and has two connected, simply connected
coverings $p_1:X_1\ra X$ and $p_2:X_2\ra X$, then choosing
basepoints $x_i, i=0,1,2$ , since 
$p_{1*}(\pu(X_1,x_1)) = p_{2*}(\pu(X_2,x_2))=\{1\}\sset \pu(X,x_0)$,
the lifting criterion with each projection playing the role of $f$ in turn
gives us maps $\wtl{p}_1:(X_1,x_1)\ra (X_2,x_2)$ and 
$\wtl{p}_2:(X_2,x_2)\ra (X_1,x_1)$ with $p_2\circ\wtl{p}_1=p_1$
and $p_1\circ\wtl{p}_2=p_2$. Consequently, 
$p_2\circ\wtl{p}_1\circ \wtl{p}_2 = p_1\circ\wtl{p}_2=p_2$
and similarly, 
$p_1\circ\wtl{p}_2\circ \wtl{p}_1 =p_2\circ\wtl{p}_1=p_1$.
So $\wtl{p}_1\circ \wtl{p}_2:(X_2,x_2)\ra (X_2,x_2)$, for example,
is a lift of $p_2$ to the covering map $p_2$. But so is the identity map! By
uniqueness, therefore, $\wtl{p}_1\circ \wtl{p}_2=Id$ . Similarly,
$\wtl{p}_2\circ \wtl{p}_1=Id$. So $(X_1,x_1)$ and $(X_2,x_2)$ 
are homeomorphic. So up to homeomorphism, a space can have
only one connected, simply-connected covering space. It is known
as the {\it universal covering} of the space $X$. 

\msk

Not every (locall path-connected) space $X$ has a universal covering; a 
(further) necessary condition is that $X$ be {\it semi-locally simply connected}.
The idea is that If $p:\wtl{X}\ra X$ is the universal cover, then for every 
point $x\in X$, we have an evenly-covered neighborhood ${\Cal U}$ of $x$.
The inclusion $i:{\Cal U}\ra X$, by definition, lifts to $\wtl{X}$, so
$i_*(\pu({\Cal U},x))\sset p_*(\pu(\wtl{X},\wtl{x}) = \{1\}$, so
$i_*$ is the trivial map. Consequently, every loop in ${\Cal U}$ is 
null-homotopic in $X$. This is semi-local simple connectivity;
every point has a neighborhood whose inclusion-induced homomorphism
is trivial. Not all spaces have this property; the most famous is the 
Hawaiian earrings 
$\displaystyle X=\bigcup_{n}\{x\in {\Bbb R}^2 :  ||x-(1/n,0)||=1/n\}$ .
The point $(0,0)$ has no such neighborhood. We shall see later that this
property is also sufficient to guarantee the existence of a universal 
cover.

\msk

One reason for the importance of the universal cover is that it gives
us a unified approach to building \underbar{all} connected covering
spaces of $X$. The basis for this is the {\it deck transformation group}
of a covering space $p:\wtl{X}\ra X$; this is the set of all
homeomorphisms $h:\wtl{X}\ra\wtl{X}$ such that $p\circ h = p$.
These homeomorphisms, by definition, permute each of the point inverses
of $p$. In fact, since $h$ can be thought of as a lift of the projection
$p$, by the lifting criterion $h$ is determined by which point in the 
inverse image of the basepoint $x_0$ it takes the basepoint 
$\wtl{x}_0$ of $\wtl{X}$ to. A deck transformation sending
$\wtl{x}_0$ to $\wtl{x}_1$ exists $\Leftrightarrow$
$p_*(\pu(\wtl{X},\wtl{x}_0)=p_*(\pu(\wtl{X},\wtl{x}_1)$
[we need one inclusion to give the map $h$, and the opposite inclusion
to ensure it is a bijection (because its inverse exists)]. These two groups
are in general {\it conjugate}, by the projection of a path from 
$\wtl{x}_0$ to $\wtl{x}_1$; this can be seen by following the change
of basepoint isomorphism down into $G=\pu(X,x_0)$. As we have seen, paths
in $\wtl{X}$ from $\wtl{x}_0$ to $\wtl{x}_1$ are in 1-to-1
correspondence with the cosets of $H=p_*(\pu(\wtl{X},\wtl{x}_0)$ in 
$p_*(\pu({X},{x}_0)$; so deck transformations are in 1-to-1 
correspondence with cosets whose representatives conjugate 
$H$ to itself. The set of such elements in $G$ is called the 
{\it normalizer of $H$ in $G$}, and denoted $N_G(H)$ or simply
$N(H)$. The deck transformation group is therefore
in 1-to-1 correspondence with the group $N(H)/H$ under
$h\mapsto$ the coset represented by the projection of the path from 
$\wtl{x}_0$ to $h(\wtl{x}_0)$. And since $h$ is essentially built
by lifting paths, it follows quickly that this map is a
homomorphism, hence an isomorphism.

\msk

In particular, applying this to the universal covering space
$p:\wtl{X}\ra X$, since in this case $H=\{1\}$, so $N(H)=\pu(X,x_0)$,
its deck transformation group is isomorphic to $\pu(X,x_0)$. 
Thus $\pu(X)$ acts on its universal cover as a group of
homeomorphisms. And since this action is {\it simply transitive}
on point inverses [there is exactly one (that's the simple
part) deck transformation carrying any one point in a point 
inverse to any other one (that's the transitive part)], the 
quotient map from $\wtl{X}$ to the orbits of this action \underbar{is}
the projection map $p$. The evenly covered property of $p$ implies
that $X$ does have the quotient topology under this action.

\msk

So every space it $X$ the quotient of its universal cover (if it has
one!) by its fundamental group $\pu(X,x_0)$, realized as the group
of deck transformations. And the quotient map is the covering 
projection. In general, a quotient by a group action need not be 
a covering map; the action must be {\it free} - the only group element
which fixes a point is $Id$ (this s from the uniqueness of lifts) - 
and {\it properly discontinuous}


\vfill
\end