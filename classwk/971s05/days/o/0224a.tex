


\magnification=1200
\overfullrule=0pt
\parindent=0pt

\nopagenumbers

\input amstex

\voffset=-.6in
\hoffset=-.5in
\hsize = 7.5 true in
\vsize=10.4 true in

%\voffset=1.4in
%\hoffset=-.5in
%\hsize = 10.2 true in
%\vsize=8 true in

\input colordvi

\def\cltr{\Red}		  % Red  VERY-Approx PANTONE RED
\def\cltb{\Blue}		  % Blue  Approximate PANTONE BLUE-072
\def\cltg{\PineGreen}	  % ForestGreen  Approximate PANTONE 349
\def\cltp{\DarkOrchid}	  % DarkOrchid  No PANTONE match
\def\clto{\Orange}	  % Orange  Approximate PANTONE ORANGE-021
\def\cltpk{\CarnationPink}	  % CarnationPink  Approximate PANTONE 218
\def\clts{\Salmon}	  % Salmon  Approximate PANTONE 183
\def\cltbb{\TealBlue}	  % TealBlue  Approximate PANTONE 3145
\def\cltrp{\RoyalPurple}	  % RoyalPurple  Approximate PANTONE 267
\def\cltp{\Purple}	  % Purple  Approximate PANTONE PURPLE

\def\cgy{\GreenYellow}     % GreenYellow  Approximate PANTONE 388
\def\cyy{\Yellow}	  % Yellow  Approximate PANTONE YELLOW
\def\cgo{\Goldenrod}	  % Goldenrod  Approximate PANTONE 109
\def\cda{\Dandelion}	  % Dandelion  Approximate PANTONE 123
\def\capr{\Apricot}	  % Apricot  Approximate PANTONE 1565
\def\cpe{\Peach}		  % Peach  Approximate PANTONE 164
\def\cme{\Melon}		  % Melon  Approximate PANTONE 177
\def\cyo{\YellowOrange}	  % YellowOrange  Approximate PANTONE 130
\def\coo{\Orange}	  % Orange  Approximate PANTONE ORANGE-021
\def\cbo{\BurntOrange}	  % BurntOrange  Approximate PANTONE 388
\def\cbs{\Bittersweet}	  % Bittersweet  Approximate PANTONE 167
%\def\creo{\RedOrange}	  % RedOrange  Approximate PANTONE 179
\def\cma{\Mahogany}	  % Mahogany  Approximate PANTONE 484
\def\cmr{\Maroon}	  % Maroon  Approximate PANTONE 201
\def\cbr{\BrickRed}	  % BrickRed  Approximate PANTONE 1805
\def\crr{\Red}		  % Red  VERY-Approx PANTONE RED
\def\cor{\OrangeRed}	  % OrangeRed  No PANTONE match
\def\paru{\RubineRed}	  % RubineRed  Approximate PANTONE RUBINE-RED
\def\cwi{\WildStrawberry}  % WildStrawberry  Approximate PANTONE 206
\def\csa{\Salmon}	  % Salmon  Approximate PANTONE 183
\def\ccp{\CarnationPink}	  % CarnationPink  Approximate PANTONE 218
\def\cmag{\Magenta}	  % Magenta  Approximate PANTONE PROCESS-MAGENTA
\def\cvr{\VioletRed}	  % VioletRed  Approximate PANTONE 219
\def\parh{\Rhodamine}	  % Rhodamine  Approximate PANTONE RHODAMINE-RED
\def\cmu{\Mulberry}	  % Mulberry  Approximate PANTONE 241
\def\parv{\RedViolet}	  % RedViolet  Approximate PANTONE 234
\def\cfu{\Fuchsia}	  % Fuchsia  Approximate PANTONE 248
\def\cla{\Lavender}	  % Lavender  Approximate PANTONE 223
\def\cth{\Thistle}	  % Thistle  Approximate PANTONE 245
\def\corc{\Orchid}	  % Orchid  Approximate PANTONE 252
\def\cdo{\DarkOrchid}	  % DarkOrchid  No PANTONE match
\def\cpu{\Purple}	  % Purple  Approximate PANTONE PURPLE
\def\cpl{\Plum}		  % Plum  VERY-Approx PANTONE 518
\def\cvi{\Violet}	  % Violet  Approximate PANTONE VIOLET
\def\clrp{\RoyalPurple}	  % RoyalPurple  Approximate PANTONE 267
\def\cbv{\BlueViolet}	  % BlueViolet  Approximate PANTONE 2755
\def\cpe{\Periwinkle}	  % Periwinkle  Approximate PANTONE 2715
\def\ccb{\CadetBlue}	  % CadetBlue  Approximate PANTONE (534+535)/2
\def\cco{\CornflowerBlue}  % CornflowerBlue  Approximate PANTONE 292
\def\cmb{\MidnightBlue}	  % MidnightBlue  Approximate PANTONE 302
\def\cnb{\NavyBlue}	  % NavyBlue  Approximate PANTONE 293
\def\crb{\RoyalBlue}	  % RoyalBlue  No PANTONE match
%\def\cbb{\Blue}		  % Blue  Approximate PANTONE BLUE-072
\def\cce{\Cerulean}	  % Cerulean  Approximate PANTONE 3005
\def\ccy{\Cyan}		  % Cyan  Approximate PANTONE PROCESS-CYAN
\def\cpb{\ProcessBlue}	  % ProcessBlue  Approximate PANTONE PROCESS-BLUE
\def\csb{\SkyBlue}	  % SkyBlue  Approximate PANTONE 2985
\def\ctu{\Turquoise}	  % Turquoise  Approximate PANTONE (312+313)/2
\def\ctb{\TealBlue}	  % TealBlue  Approximate PANTONE 3145
\def\caq{\Aquamarine}	  % Aquamarine  Approximate PANTONE 3135
\def\cbg{\BlueGreen}	  % BlueGreen  Approximate PANTONE 320
\def\cem{\Emerald}	  % Emerald  No PANTONE match
%\def\cjg{\JungleGreen}	  % JungleGreen  Approximate PANTONE 328
\def\csg{\SeaGreen}	  % SeaGreen  Approximate PANTONE 3268
\def\cgg{\Green}	  % Green  VERY-Approx PANTONE GREEN
\def\cfg{\ForestGreen}	  % ForestGreen  Approximate PANTONE 349
\def\cpg{\PineGreen}	  % PineGreen  Approximate PANTONE 323
\def\clg{\LimeGreen}	  % LimeGreen  No PANTONE match
\def\cyg{\YellowGreen}	  % YellowGreen  Approximate PANTONE 375
\def\cspg{\SpringGreen}	  % SpringGreen  Approximate PANTONE 381
\def\cog{\OliveGreen}	  % OliveGreen  Approximate PANTONE 582
\def\pars{\RawSienna}	  % RawSienna  Approximate PANTONE 154
\def\cse{\Sepia}		  % Sepia  Approximate PANTONE 161
\def\cbr{\Brown}		  % Brown  Approximate PANTONE 1615
\def\cta{\Tan}		  % Tan  No PANTONE match
\def\cgr{\Gray}		  % Gray  Approximate PANTONE COOL-GRAY-8
\def\cbl{\Black}		  % Black  Approximate PANTONE PROCESS-BLACK
\def\cwh{\White}		  % White  No PANTONE match


\loadmsbm

\input epsf

\def\ctln{\centerline}
\def\u{\underbar}
\def\ssk{\smallskip}
\def\msk{\medskip}
\def\bsk{\bigskip}
\def\hsk{\hskip.1in}
\def\hhsk{\hskip.2in}
\def\dsl{\displaystyle}
\def\hskp{\hskip1.5in}

\def\lra{$\Leftrightarrow$ }
\def\ra{\rightarrow}
\def\mpto{\logmapsto}
\def\pu{\pi_1}
\def\mpu{$\pi_1$}
\def\sig{\Sigma}
\def\msig{$\Sigma$}
\def\ep{\epsilon}
\def\sset{\subseteq}
\def\del{\partial}
\def\inv{^{-1}}
\def\wtl{\widetilde}
\def\lra{\Leftrightarrow}



\ctln{\bf Math 971 Algebraic Topology}

\ssk

\ctln{February 24, 2005}

\msk


Every space it $X$ the quotient of its universal cover (if it has
one!) by its fundamental group $G=\pu(X,x_0)$, realized as the group
of deck transformations. And the quotient map is the covering 
projection. 
So $X\cong \wtl{X}/G$ . In general, a quotient of a 
space $Z$ by a group action $G$ 
need not be 
a covering map; the action must be {\it properly discontinuous}, 
which means that for every point 
 $z\in Z$, there is a neighborhood ${\Cal U}$ of $x$ so that $g\neq 1$ $\Rightarrow$
${\Cal U}\cap g{\Cal U}=\emptyset$ (the group action carries sufficiently
small neighborhoods off of themselves). The evenly covered neighborhoods
provide these for the universal cover. And conversely, the quotient of a space by a 
p.d. group action is a covering space. 

\msk

But! Given $G=\pu(X,x_0)$ and its 
action on a universal cover $\wtl{X}$, we can, instead of quotienting out by $G$,
quotient out by any \underbar{subgroup} $H$ of $G$, to build $X_H=\wtl{X}/H$. 
This is a space with fundamental group $H$, having $\wtl{X}$ as universal covering.
And since the quotient (covering) map $p_G:\wtl{X}\ra X=\wtl{X}/G$ factors through $\wtl{X}/H$,
we get an induced map $p_H:\wtl{X}/H\ra X$, which is a covering map; open sets with
trivial inclusion-induced homomorphism lift homeomorphically to $\wtl{X}$,
hence homeomorphically to $\wtl{X}/H$; taking lifts to each point inverse of $x\in X$
verifies the evenly covering property for $p_H$ . So every subgroup of $G$ is the
fundamental group of a covering of $X$. 

\ssk

We can further refine this to give the {\it Galois correspondence}. Two covering spaces
$p_1:X_1\ra X$ , $p_2:X_2\ra X$ are {\it isomorphic} if there is a homeomorphism
$h:X_1\ra X_2$ with $p_1=p_2\circ h$. Choosing basepoints $x_1,x_2$ mapping to $x_0\in X$,
this implies that, if $h(x_1)=x_2$, then $p_{1*}(\pu(X_1,x_1)) = p_{2*}(h_*(\pu(X_1,x_1))) = 
p_{2*}(\pu(X_2,x_2))$ . On the other hand, our homeomorphism $h$ need not take our
chosen basepoints to one another; if $h(x_1)=x_3$, then $p_{1*}(\pu(X_1,x_1)) = p_{2*}(\pu(X_2,x_3))$.
But  $p_{2*}(\pu(X_2,x_2))$ and $p_{2*}(\pu(X_2,x_3))$ are isomorphic, via a change 
of basepoint isomorphism $\widehat{\eta}$ , where $\eta$ is a path in $X_2$ from $x_2$ to $x_3$.
But such a path projects to $X$ has a loop at $x_0$, and since the change of basepoint isomorphism
is by ``conjugating'' by the path $\eta$, the resulting groups $p_{2*}(\pu(X_2,x_2))$ and $p_{2*}(\pu(X_2,x_3))$
are conjugate, by $p_2\circ \eta$ . So, without reference to basepoints, isomorphic coverings give,
under projection, conjugate subgroups of $\pu(X,x_0)$ . But conversely, given covering spaces
$X_1,X_2$ whose subgroups $p_{1*}(\pu(X_1,x_1))$ and $p_{2*}(\pu(X_2,x_2))$ are conjugate,
lifting a loop $\gamma$ representing the conjugating element to a loop $\wtl{\gamma}$ in
$X_2$ starting at $x_2$ gives, as its terminal endpoint, a point $x_3$ with 
$p_{1*}(\pu(X_1,x_1)) = p_{2*}(\pu(X_2,x_3))$ (since it conjugates back!), and so, by the lifting criterion,
there is an isomorphism $h:(X_1,x_1) \ra (X_2,x_3)$. So conjugate subgroups give isomorphic coverings.
Thus, for a path-connected, locally path-connected, semi-locally simply-connected space $X$,  
the image of the induced homomorphism on \mpu\ 
gives a one-to-one correspondence between 
[isomorphism classes of (connected) coverings of $X$] and 
[conjugacy classes of subgroups of $\pu(X)$].

\msk

So, for example, if you have a group $G$ that you are interested in, you know of a (nice enough) 
space $X$ with $\pu(X)\cong G$, and you know enough about the covering of $X$, then you can
gain information about the subgroup structure of $G$. For example, and in some respects as
motivation for all of this machinery!, a free group $F(\Sigma)$ is \mpu\ of a bouquet of circles $X$. 
Any covering space $\wtl{X}$ of $X$ is a union of vertices and edges, so is a graph.  Collapsing
a maximal tree to a point, $\wtl{X}$ is $\simeq$ a bouquet of circles, so has free \mpu . So, every
subgroup of a free group is free. (That is a lot shorter than the original, group-theoretic, proof...) 
A subgroup $H$ of index $n$ in $F(\Sigma)$ corresponds to a $n$-sheeted covering $\wtl{X}$ of $X$. If
$|\Sigma| = m$, then $\wtl{X}$ will have $n$ vertices and $nm$ edges. Collapsing a maximal
tree, having $n-1$ edges to a point, leaves a bouquet of $nm-n+1$ circles, so $H\cong F(nm-n+1)$.
For example, for $m=3$, index $n$ subgroups are free on $2n+1$ generators, so every free subgroup
on 4 generators has infinite index in $F(3)$. Try proving that directly!

\msk

{\it Kurosh Subgroup Theorem}: If $H < G_1*G_2$ is a subgroup of
a free product, then $H$ is (isomorphic to) a free product of a
collection of conjugates of subgroups of $G_1$ and $G_2$ and a 
free froup. The proof is to build a space by taking 2-complexes
$X_1$ and $X_2$ with $\pu$'s isomorphic to $G_1,G_2$ and join
their basepoints by an arc. The covering space of this space $X$
corresponding to $H$ consists of spaces that cover $X_1,X_2$
(giving, after basepoint considerations, the conjugates)
connected by a collection of arcs (which, suitably interpreted,
gives the free group).

\msk

{\it Residually finite groups}: $G$ is said to be residually finite if for every $g\neq 1$ there is a 
finite group $F$ and a homomorphism $\varphi: G\ra F$ with $\varphi(g)\neq 1$ in $F$. This 
amounts to saying that $g\notin$ the (normal) subgroup $\ker(\varphi)$, which amounts to
saying that a loop corresponding to $g$ does \underbar{not} lift to a loop in the finite-sheeted
covering space corresponding to $\ker(\varphi)$. So residual finiteness of a group can be
verified by building coverings of a space $X$ with $\pu(X)=G$. For example, free groups can be
shown to be residually finite in this way. 

\msk

{\it Ranks of free (sub)groups:} A free group on $n$
generators is isomorphic to a free group on $m$ generators
$\lra$ $n=m$; this is because the abelianizations of the two 
groups are ${\Bbb Z}^n,{\Bbb Z}^m$. The (minimal) number of 
generators for a free group is called its {\it rank}.
Given a free group
$G=F(a_1,\ldots a_n)$ and a collection of words $w_1,\ldots w_m\in G$,
we can determine the rank and ndex of the subgroup it $H$ they
generate by building the corresponding cover. The idea is
to start with a bouquet of $m$ circles, each subdivided 
and labelled to spell
out the words $w_i$. Then we repeatedly identify edges sharing
on common vertex if they are labelled precisely the same (same
letter {\it and} same orientation). This process is known
as {\it folding}. One can inductively show that the (obvious)
maps from these graphs to the bouquet of $n$ circles $X_n$ both
have image $H$ under the induced maps on \mpu ; the graphs
are in fact homotopy equivalent, and the map for the unfolded graph
factors through the one for the folded graph. We continue until there
is no more folding to be done; the resulting graph $X$ is what is 
known (in combinatorics) as a {\it graph covering}; the map to $X_n$
is locally injective. If this map is a covering map, then our subgroup
$H$ has finite index (equal to the degree of the
covering) and we can compute the rank of $H$ (and a basis!) from this
index as above. If not, then the map is not locally surjective at
some vertices; if we graft trees onto these vertices, we can extend the map
to an (infinite-sheeted) covering map without changing the homotopy
type of the graph. $H$ therefore has infinite index in $G$, and its
rank can be computed from $H\cong \pu(X)$. An example of this procedure
is given below.

\msk

{\bf Postscript: why care about covering spaces?} The preceding discussion
probably makes it clear that covering places play a central role in
(combinatorial) group theory. It also plays a role in embedding 
problems; a common scenario is to have a map $f:Y\ra X$ which is 
injective on \mpu , and we wish to know if we can lift $f$ to a 
finite-sheeted covering so that the lifted map $\widetilde{f}$ is 
homotopic to an embedding. Information that is easier to obtain 
in the case of an embedding can then be passed down to gain information
abut the original map $f$. And covering spaces underlie the 
theory of analytic continuation in complex analysis; starting
with a domain $D\subseteq {\Bbb C}$, what analytic continuation really
builds is an (analytic) function from a covering space of $D$ to ${\Bbb C}$.
For example, the logarithm is really defined as a map from 
the universal cover of ${\Bbb C}\setminus\{0\}$ to ${\Bbb C}$. 
The various ``branches'' of the logarithm refer to which sheet
in this cover you are in.

\bsk

{\bf Homology theory:} 


\vfill
\end