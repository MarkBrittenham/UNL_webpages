
\magnification=1200
\overfullrule=0pt
\parindent=0pt

\nopagenumbers

\input amstex

\voffset=-.6in
\hoffset=-.5in
\hsize = 7.5 true in
\vsize=10.4 true in

%\voffset=1.4in
%\hoffset=-.5in
%\hsize = 10.2 true in
%\vsize=8 true in

\input colordvi

\loadmsbm

\input epsf

\def\ctln{\centerline}
\def\u{\underbar}
\def\ssk{\smallskip}
\def\msk{\medskip}
\def\bsk{\bigskip}
\def\hsk{\hskip.1in}
\def\hhsk{\hskip.2in}
\def\dsl{\displaystyle}
\def\hskp{\hskip1.5in}

\def\lra{$\Leftrightarrow$ }
\def\ra{\rightarrow}
\def\mpto{\logmapsto}
\def\pu{\pi_1}
\def\mpu{$\pi_1$}
\def\sig{\Sigma}
\def\msig{$\Sigma$}
\def\ep{\epsilon}
\def\sset{\subseteq}
\def\del{\partial}
\def\inv{^{-1}}
\def\wtl{\widetilde}



\ctln{\bf Math 971 Algebraic Topology}

\ssk

\ctln{February 15, 2005}

\msk

The number of sheets of a covering map can also be determined 
from the fundamental groups:

\msk

{\bf Proposition:} If $X$ and $\wtl{X}$ are 
path-connected, then the number of sheets of a covering map equals
the index of the subgroup $H=p_*(\pu(\wtl{X},\wtl{x}_0)$ in 
$G=\pu(X,x_0)$ . 

\msk

To see this, choose loops $\{\gamma_\alpha\}$ representing representatives $\{g_\alpha\}$ of each of the (right) cosets of $H$ in $G$. Then if we lift
each of them to loops based at $\wtl{x}_0$, they will have distinct
endpoints; if $\wtl{\gamma}_1(1)=\wtl{\gamma}_2(1)$, then 
by uniqueness of lifts, $\gamma_1*\overline{\gamma_2}$ lifts to 
$\wtl{\gamma}_1*\overline{\wtl{\gamma}_2}$, so it
lifts to a loop, so $\gamma_1*\overline{\gamma_2}$ represents
an element of $p_*(\pu(\wtl{X},\wtl{x}_0)$, so $g_1=g_2$.
Conversely, every point in $p\inv(x_0)$ is the endpoint of on of these
lifts, since we can construct a path $\wtl{\gamma}$
from $\wtl{x}_0$ to any such point $y$, giving a loop
$\gamma=p\circ \wtl{\gamma}$ representing an element $g\in\pu(X,x_0)$.
But then $g=hg_\alpha$ for some $h\in H$ and $\alpha$, 
so $\gamma$ is homotopic rel endpoints to $\eta*\gamma_\alpha$ for some loop
$\eta$ representing $h$. But then lifting these based at $\wtl{x}_0$, by hmotopy
lifting, $\wtl{\gamma}$ is homotopic rel endpoints to $\wtl{\eta}$, which is a 
loop, followed by the lift $\wtl{\gamma}_\alpha$ of $\gamma_\alpha$
starting at $\wtl{x}_0$. So $\wtl{\gamma}$ and 
$\wtl{\gamma}_\alpha$ have the same value at 1.

\ssk

Therefore, lifts of representatives of coset representatives of $H$ in $G$ give
a 1-to-1 correspondence, given by $\wtl{\gamma}(1)$, with $p\inv{x_0}$.
In particular, they have the same cardinality.

\msk

The path lifting property (because $\pi([0,1],0)=\{1\}$) is actually a special
case of a more general {\bf lifting criterion}: If 
$p:(\wtl{X},\wtl{x}_0)\ra (X,x_0)$ is a covering map, and 
$f:(Y,y_0)\ra (X,x_0)$ is a map, where
$Y$ is path-connected and locally path-connected, then there is a lift 
$\wtl{f}:(Y,y_0)\ra (\wtl{X},\wtl{x}_0)$ of $f$ (i.e., 
$f=p\circ\wtl{f}$) $\Leftrightarrow$ 
$f_*(\pu(Y,y_0))\sset p_*(\pu(\wtl{X},\wtl{x}_0))$ . 
Furthermore, two lifts of $f$ which agree at a single point are equal.

\msk

If the lift exists, then $f=p\circ\wtl{f}$ implies 
$f_*=p_*\circ\wtl{f}_*$, so 
$f_*(\pu(Y,y_0)) = p_*(\wtl{f}_*(\pu(Y,y_0)))\sset p_*(\pu(\wtl{X},\wtl{x}_0))$ , as desired.
Conversely, if $f_*(\pu(Y,y_0))\sset p_*(\pu(\wtl{X},\wtl{x}_0))$,
then we wish to build the lift of $f$. Not wishing to waste our work on the
special case, we will {\it use} path lifting to do it! Given $y\in Y$,
choose a path $\gamma$ in $Y$ from $y_0$ to $y$ and use path
lifting in $X$ to lift the path $f\circ\gamma$ to a path $\wtl{f\circ\gamma}$ with 
$\wtl{f\circ\gamma}(0)=\wtl{x}_0$ . Then define
$\wtl{f}(y)=\wtl{f\circ\gamma}(1)$ . Provided we show that
this is well-defined and continuous, it is our required lift, 
since $(p\circ\wtl{f})(y) = p(\wtl{f}(y))=p(\wtl{f\circ\gamma}(1))
=p\circ\wtl{f\circ\gamma})(1) = (f\circ\gamma)(1) = f(\gamma(1)) = f(y)$. 
To show that it
is well-defined, if $\eta$ is any other path from 
$y_0$ to $y$, then $\gamma*\overline{\eta}$ is a loop in $Y$, 
so $f\circ(\gamma*\overline{\eta})=(f\circ\gamma)*\overline{(f\circ\eta)}$
is a loop in $X$ representing an element of 
$f_*(\pu(Y,y_0))\sset p_*(\pu(\wtl{X},\wtl{x}_0))$, and
so lifts to a loop in $\wtl{X}$ based at $\wtl{x}_0$.
Consequently, as before, $f\circ\gamma$ and $f\circ\eta$ lift to paths
starting at $\wtl{x}_0$ with the same value at 1. So $\wtl{f}$ is
well-defined.  To show that $\wtl{f}$ is continuous, we use the 
evenly covered property of $p$. Given $y\in Y$,
and  a neighborhood $\wtl{\Cal U}$ of 
$\wtl{f}(y)$ in $\wtl{X}$, 
we wish to find a nbhd ${\Cal V}$ of $y$ with 
$\wtl{f}({\Cal V})\sset\wtl{\Cal U}$. Choosing an evenly covered 
neighborhood ${\Cal U}_y$ for $f(y)$, choose the sheet 
$\wtl{\Cal U}_y$ over ${\Cal U}_y$ which contains $\wtl{f}(y)$,
and set ${\Cal W}=\wtl{\Cal U}\cap \wtl{\Cal U}_y$. This is open in 
$\wtl{X}$, and $p$ is a homeomorphism from this 
set to the open set $p({\Cal W})\sset X$. Then if we set ${\Cal V}^\prime= f\inv(p({\Cal W})$
this is an open set containing $y$, and so contains a path-connected open 
set ${\Cal V}$ containing $y$. Then is for every point $z\in {\Cal V}$ we build a path
$\gamma$
from $y_0$ to $z$ by concatenating a path from $y_0$ to $y$ with a path {\it in} ${\Cal V}$
from $y$ to $z$, then by unique path lifting, 
since $f({\Cal V}\sset {\Cal U}_y$ , $f\circ\gamma$ lifts to 
the concatenation of a path from $\wtl{x}_0$ to $\wtl{f}(y)$ and a 
path {\it in} $\wtl{\Cal U}_y$ from $\wtl{f}(y)$ to $\wtl{f}(z)$.
So $\wtl{f}(z)\in\wtl{\Cal U}$.

\ssk

Because $\wtl{f}$ is built by lifting paths, and path
lifting is unique, the last statement of the proposition follows.

\bsk

{\bf Universal covering spaces}: As we shall see, a particularly
important covering space to identify is one which is simply
connected. One thing we can see from the lifting crierion is
that such a covering is essentially unique:

\msk

If $X$ is locally path-connected, and has two connected, simply connected
coverings $p_1:X_1\ra X$ and $p_2:X_2\ra X$, then choosing
basepoints $x_i, i=0,1,2$ , since 
$p_{1*}(\pu(X_1,x_1)) = p_{2*}(\pu(X_2,x_2))=\{1\}\sset \pu(X,x_0)$,
the lifting criterion with each projection playing the role of $f$ in turn
gives us maps $\wtl{p}_1:(X_1,x_1)\ra (X_2,x_2)$ and 
$\wtl{p}_2:(X_2,x_2)\ra (X_1,x_1)$ with $p_2\circ\wtl{p}_1=p_1$
and $p_1\circ\wtl{p}_2=p_2$. Consequently, 
$p_2\circ\wtl{p}_1\circ \wtl{p}_2 = p_1\circ\wtl{p}_2=p_2$
and similarly, 
$p_1\circ\wtl{p}_2\circ \wtl{p}_1 =p_2\circ\wtl{p}_1=p_1$.
So $\wtl{p}_1\circ \wtl{p}_2:(X_2,x_2)\ra (X_2,x_2)$, for example,
is a lift of $p_2$ to the covering map $p_2$. But so is the identity map! By
uniqueness, therefore, $\wtl{p}_1\circ \wtl{p}_2=Id$ . Similarly,
$\wtl{p}_2\circ \wtl{p}_1=Id$. So $(X_1,x_1)$ and $(X_2,x_2)$ 
are homeomorphic. So up to homeomorphism, a space can have
only one connected, simply-connected covering space. It is known
as the {\it universal covering} of the space $X$. 

\msk

Not every (locall path-connected) space $X$ has a universal covering; a 
(further) necessary condition is that $X$ be {\it semi-locally simply connected}.
The idea is that If $p:\wtl{X}\ra X$ is the universal cover, then for every 
point $x\in X$, we have an evenly-covered neighborhood ${\Cal U}$ of $x$.
The inclusion $i:{\Cal U}\ra X$, by definition, lifts to $\wtl{X}$, so
$i_*(\pu({\Cal U},x))\sset p_*(\pu(\wtl{X},\wtl{x}) = \{1\}$, so
$i_*$ is the trivial map. Consequently, every loop in ${\Cal U}$ is 
null-homotopic in $X$. This is semi-local simple connectivity;
every point has a neighborhood whose inclusion-induced homomorphism
is trivial. Not all spaces have this property; the most famous is the 
Hawaiian earrings 
$\displaystyle X=\bigcup_{n}\{x\in {\Bbb R}^2 :  ||x-(1/n,0)||=1/n\}$ .
The point $(0,0)$ has no such neighborhood. 

\msk

On the other hand, if a space $X$ is path connected, locally path connected, and
semi-locally simply connected (S-LSC), then it has a universal covering;
we describe a general construction. The idea is that a covering
space should have the path lifting and homotopy
lifting properties, and the universal 
cover can be characterized as the only covering space for 
which {\it only} null-homotopic loops lift to loops. So we build a 
space and a map which \underbar{must} have these properties.
We do this by making a space $\widetilde{X}$ whose
points are (equivalence classes of) $[\gamma]$
based paths $\gamma:(I,0)\ra (X,x_0)$, where two paths are equivalent
if they are homotopic rel endpoints! The projection map is
$p([\gamma])=\gamma(1)$. The S-LSCness is what guarantees that this is a 
covering map; choosing $x\in X$, a path $\gamma_0$ from $x_0$ to $x$,
and a  neighborhood ${\Cal U}$ of $x$ guaranteed by S-LSC, paths from 
$x_0$ to points in ${\Cal U}$ are based equivalent to $\gamma*\gamma_0*\eta$
where $\gamma$ is a based loop at $x_0$ and $\eta$ is a path in ${\Cal U}$.
But by simple connectivity, a path in ${\Cal U}$ is determined up to homotopy
by its endpoints, and so, with $\gamma$ fixed, these paths are in one-to-one
correspondence with ${\Cal U}$. So $p\inv({\Cal U}$ is a disjoint union,
indexed by $\pu(X,x_0)$, of sets in bijective correspondence with ${\Cal U}$.
The appropriate topology on $\widetilde{X}$, essentially given as a basis
by triples $\gamma*,gamma_0,{\Cal U}$ as above, make $p$ a covering map.
Note that the inverse image of 
the basepoint $x_0$ is the equivalence classes of \underbar{loops} at $x_0$,
i.e., $\pu(X,x_0)$. A path $\gamma$ lifts to the path of paths
$[\gamma_t]$, where $\gamma_t(s)=\gamma(ts)$, and so the only 
loop in $X$ which lifts to a loop in $\widetilde{X}$ has endpoint
$[\gamma]=[c_{x_0}]$, i.e., $[\gamma]=1$ in $\pu(X,x_0)$. This
implies that $p_*(\pu(\widetilde{X},[c_{x_0}]))=\{1\}$, so 
$\pu(\widetilde{X},[c_{x_0}])=\{1\}$ . \hhsk However, nobody in their
right minds would go about building $\widetilde{X}$ in this way, in general!
Before describing how to do it ``right'', though, we should perhaps see why
we should want to?

\msk

One reason for the importance of the universal cover is that it gives
us a unified approach to building \underbar{all} connected covering
spaces of $X$. The basis for this is the {\it deck transformation group}
of a covering space $p:\wtl{X}\ra X$; this is the set of all
homeomorphisms $h:\wtl{X}\ra\wtl{X}$ such that $p\circ h = p$.
These homeomorphisms, by definition, permute each of the point inverses
of $p$. In fact, since $h$ can be thought of as a lift of the projection
$p$, by the lifting criterion $h$ is determined by which point in the 
inverse image of the basepoint $x_0$ it takes the basepoint 
$\wtl{x}_0$ of $\wtl{X}$ to. A deck transformation sending
$\wtl{x}_0$ to $\wtl{x}_1$ exists $\Leftrightarrow$
$p_*(\pu(\wtl{X},\wtl{x}_0)=p_*(\pu(\wtl{X},\wtl{x}_1)$
[we need one inclusion to give the map $h$, and the opposite inclusion
to ensure it is a bijection (because its inverse exists)]. These two groups
are in general {\it conjugate}, by the projection of a path from 
$\wtl{x}_0$ to $\wtl{x}_1$; this can be seen by following the change
of basepoint isomorphism down into $G=\pu(X,x_0)$. As we have seen, paths
in $\wtl{X}$ from $\wtl{x}_0$ to $\wtl{x}_1$ are in 1-to-1
correspondence with the cosets of $H=p_*(\pu(\wtl{X},\wtl{x}_0)$ in 
$p_*(\pu({X},{x}_0)$; so deck transformations are in 1-to-1 
correspondence with cosets whose representatives conjugate 
$H$ to itself. The set of such elements in $G$ is called the 
{\it normalizer of $H$ in $G$}, and denoted $N_G(H)$ or simply
$N(H)$. The deck transformation group is therefore
in 1-to-1 correspondence with the group $N(H)/H$ under
$h\mapsto$ the coset represented by the projection of the path from 
$\wtl{x}_0$ to $h(\wtl{x}_0)$. And since $h$ is essentially built
by lifting paths, it follows quickly that this map is a
homomorphism, hence an isomorphism.

\msk

In particular, applying this to the universal covering space
$p:\wtl{X}\ra X$, since in this case $H=\{1\}$, so $N(H)=\pu(X,x_0)$,
its deck transformation group is isomorphic to $\pu(X,x_0)$. 
For example, this gives the quickest possible proof 
that $\pu(S^1)\cong {\Bbb Z}$, since ${\Bbb R}$ is a 
contractible covering space, whose deck transformations
are the translations by integer distances. 
Thus $\pu(X)$ acts on its universal cover as a group of
homeomorphisms. And since this action is {\it simply transitive}
on point inverses [there is exactly one (that's the simple
part) deck transformation carrying any one point in a point 
inverse to any other one (that's the transitive part)], the 
quotient map from $\wtl{X}$ to the orbits of this action \underbar{is}
the projection map $p$. The evenly covered property of $p$ implies
that $X$ does have the quotient topology under this action.

\msk

So every space it $X$ the quotient of its universal cover (if it has
one!) by its fundamental group $G=\pu(X,x_0)$, realized as the group
of deck transformations. And the quotient map is the covering 
projection. So $X|cong \wtl{X}/G$ . In general, a quotient of a 
space $Z$ by a group action $G$ 
need not be 
a covering map; the action must be {\it properly discontinuous}, 
which means that for every point 
 $z\in Z$, there is a neighborhood ${\Cal U}$ of $x$ so that $g\neq 1$ $\Rightarrow$
${\Cal U}\cap g{\Cal U}=\emptyset$ (the group action carries sufficiently
small neighborhoods off of themselves). The evenly covered neighborhoods
provide these for the universal cover. And conversely, the quotient of a space by a 
p.d. group action is a covering space. 

\msk

But! Given $G=\pu(X,x_0)$ and its 
action on a universal cover $\wtl{X}$, we can, instead of quotienting out by $G$,
quotient out by any \underbar{subgroup} $H$ of $G$, to build $X_H=\wtl{X}/H$. 
This is a space with fundamental group $H$, having $\wtl{X}$ as universal covering.
And since the quotient (covering) map $p_G:\wtl{X}\ra X=\wtl{X}/G$ factors through $\wtl{X}/H$,
we get an induced map $p_H:\wtl{X}/H\ra X$, which is a covering map; open sets with
trivial inclusion-induced homomorphism lift homeomorphically to $\wtl{X}$,
hence homeomorphically to $\wtl{X}/H$; taking lifts to each point inverse of $x\in X$
verifies the evenly covering property for $p_H$ . So every subgroup of $G$ is the
fundamental group of a covering of $X$. 

\ssk

We can further refine this to give the {\it Galois correspondence}. Two covering spaces
$p_1:X_1\ra X$ , $p_2:X_2\ra X$ are {\it isomorphic} if there is a homeomorphism
$h:X_1\ra X_2$ with $p_1=p_2\circ h$. Choosing basepoints $x_1,x_2$ mapping to $x_0\in X$,
this implies that, if $h(x_1)=x_2$, then $p_{1*}(\pu(X_1,x_1)) = p_{2*}(h_*(\pu(X_1,x_1))) = 
p_{2*}(\pu(X_2,x_2))$ . On the other hand, our homeomorphism $h$ need not take our
chosen basepoints to one another; if $h(x_1)=x_3$, then $p_{1*}(\pu(X_1,x_1)) = p_{2*}(\pu(X_2,x_3))$.
But  $p_{2*}(\pu(X_2,x_2))$ and $p_{2*}(\pu(X_2,x_3))$ are isomorphic, via a change 
of basepoint isomorphism $\widehat{\eta}$ , where $\eta$ is a path in $X_2$ from $x_2$ to $x_3$.
But such a path projects to $X$ has a loop at $x_0$, and since the change of basepoint isomorphism
is by ``conjugating'' by the path $\eta$, the resulting groups $p_{2*}(\pu(X_2,x_2))$ and $p_{2*}(\pu(X_2,x_3))$
are conjugate, by $p_2\circ \eta$ . So, without reference to basepoints, isomorphic coverings give,
under projection, conjugate subgroups of $\pu(X,x_0)$ . But conversely, given covering spaces
$X_1,X_2$ whose subgroups $p_{1*}(\pu(X_1,x_1))$ and $p_{2*}(\pu(X_2,x_2))$ are conjugate,
lifting a loop $\gamma$ representing the conjugating element to a loop $\wtl{\gamma}$ in
$X_2$ starting at $x_2$ gives, as its terminal endpoint, a point $x_3$ with 
$p_{1*}(\pu(X_1,x_1)) = p_{2*}(\pu(X_2,x_3))$ (since it conjugates back!), and so, by the lifting criterion,
there is an isomorphism $h:(X_1,x_1) \ra (X_2,x_3)$. So conjugate subgroups give isomorphic coverings.
Thus, for a path-connected, locally path-connected, semi-locally simply-connected space $X$,  
the image of the induced homomorphism on \mpu\ 
gives a one-to-one correspondence between 
[isomorphism classes of (connected) coverings of $X$] and 
[conjugacy classes of subgroups of $\pu(X)$].

\msk

So, for example, if you have a group $G$ that you are interested in, you know of a (nice enough) 
space $X$ with $\pu(X)\cong G$, and you know enough about the covering of $X$, then you can
gain information about the subgroup structure of $G$. For example, and in some respects as
motivation for all of this machinery!, a free group $F(\Sigma)$ is \mpu\ of a bouquet of circles $X$. 
Any covering space $\wtl{X}$ of $X$ is a union of vertices and edges, so is a graph.  Collapsing
a maximal tree to a point, $\wtl{X}$ is $\simeq$ a bouquet of circles, so has free \mpu . So, every
subgroup of a free group is free. (That is a lot shorter than the original, group-theoretic, proof...) 
A subgroup $H$ of index $n$ in $F(\Sigma)$ corresponds to a $n$-sheeted covering $\wtl{X}$ of $X$. If
$|\Sigma| = m$, then $\wtl{X}$ will have $n$ vertices and $nm$ edges. Collapsing a maximal
tree, having $n-1$ edges to a point, leaves a bouquet of $nm-n+1$ circles, so $H\cong F(nm-n+1)$.
For example, for $m=3$, index $n$ subgroups are free on $2n+1$ generators, so every free subgroup
on 4 generators has infinite index in $F(3)$. Try proving that directly!

\msk

{\it Kurosh Subgroup Theorem}:

\msk

{\it Residually finite groups}: $G$ is said to be residually finite if for every $g\neq 1$ there is a 
finite group $F$ and a homomorphism $\varphi: G\ra F$ with $\varphi(g)\neq 1$ in $F$. This 
amounts to saying that $g\notin$ the (normal) subgroup $\ker(\varphi)$, which amounts to
saying that a loop corresponding to $g$ does \underbar{not} lift to a loop in the finite-sheeted
covering space corresponding to $\ker(\varphi)$. So residual finiteness of a group can be
verified by building coverings of a space $X$ with $\pu(X)=G$. For example, free groups can be
shown to be residually finite in this way. 

\vfill
\end