

\magnification=1200
\overfullrule=0pt
\parindent=0pt

\nopagenumbers

\input amstex

\voffset=-.6in
\hoffset=-.5in
\hsize = 7.5 true in
\vsize=10.4 true in

%\voffset=1.4in
%\hoffset=-.5in
%\hsize = 10.2 true in
%\vsize=8 true in

\input colordvi

\loadmsbm

\input epsf

\def\ctln{\centerline}
\def\u{\underbar}
\def\ssk{\smallskip}
\def\msk{\medskip}
\def\bsk{\bigskip}
\def\hsk{\hskip.1in}
\def\hhsk{\hskip.2in}
\def\dsl{\displaystyle}
\def\hskp{\hskip1.5in}

\def\lra{$\Leftrightarrow$ }
\def\ra{\rightarrow}
\def\mpto{\logmapsto}
\def\pu{\pi_1}
\def\mpu{$\pi_1$}
\def\sig{\Sigma}
\def\msig{$\Sigma$}
\def\ep{\epsilon}
\def\sset{\subseteq}
\def\del{\partial}




\ctln{\bf Math 971 Algebraic Topology}

\ssk

\ctln{February 1, 2005}

\msk


Gluing groups: given groups $G_1,G_2$, with subgroups $H_1,H_2$ that are
isomorphic $H_1\cong H_2$, how can we ``glue'' $G_1$ and $G_2$ together along their
``common'' subgroup? More generally (and with our eye on van Kampen's Theorem)
given a group $H$ and homomorphisms $\phi_1 : H\ra G_i$, we wish to build the largest group ``generated'' by
$G_1$ and $G_2$, in which $\phi_1(h)=\phi_2(h)$ for all $h\in H$. 

\msk

We can do this by starting
with $G_1*G_2$ (to get the first part), and then take a quotient to insure that 
$\phi_1(h)(\phi_2(h))^{-1} =1$ for every $h$. Using presentations 
$G_1=<\sig_1 | R_1>$ , $G_2=<\sig_2 | R_2>$ , if we insist on quotienting out by 
as little as possible to get our desired result, we can do this very succinctly as

\msk

\ctln{$G = (G_1*G_2)/<\phi_1(h)(\phi_2(h))^{-1} : h\in H>^N = 
<\sig_1\coprod\sig_2 | R_1\cup R_2\cup\{\phi_1(h)(\phi_2(h))^{-1} : h\in H\}>$}
 
\msk

This group $G= =G_1*_HG_2$ is the {\it largest} group generated by $G_1$ and $G_2$ in which 
$\phi_1(h)=\phi_2(h)$ for all $h\in H$, and is called the {\it amalgamated 
free product} or {\it free product with amalgamation (over $H$)} . [{\bf Warning!} 
Group theorists will generally use this term only if both homoms $\phi_1,\phi_2$
are injective. (This insures that the natural maps of $G_1,G_2$ into $G_1*_HG_2$
are injective.) But we will use this term for all $\phi_1,\phi_2$. (Some people use the term {\it pushout}
in this more general case.)]

\msk

Important special cases : $G*_H\{1\} = G/<\phi(H)>^N = <\sig | R\cup \phi(H)>$ , and
$G_1*_\{1\}G_2 \cong G_1*G_2$

\bsk

The relevance to \mpu : the Seifert-van Kampen Theorem.

\msk

If we express a topological space as the union $X=X_1\cup X_2$, then we have 
inclusion-induced homomorphisms 

\ctln{$j_{1*}: \pu(X_1)\ra \pu(X)$ , $j_{2*}: \pu(X_2)\ra \pu(X)$}

 - to be precise, we should choose a common basepoint in $A=X_1\cap X_2$. This 
in turn gives a homomorphism $\phi:\pi(X_1)*\pu(X_2)\ra \pu(X)$ . Under the
hypotheses

\ctln{$X_1,X_2$ are open, and $X_1,X_2,X_1\cap X_2$ are path-connected}

we can see that this homom is onto:

\msk

Given $x_0\in X_1\cap X_2$ and a loop $\gamma:(I,\del I)\ra (X,x_0)$, 
we wish to show that it is homotopic rel endpoints to a product of
loops which lie alternately in $X_1$ and $X_2$. But 
$\{\gamma^{-1}(X_1),\gamma^{-1}(X_2)\}$ is an open cover of the compact
metric space $I$, and so there is an $\ep > 0$  (a {\it Lebesgue number})
so that every interval of
length $\ep$ in $I$ lies in one of these two sets, i.e., maps, under $\gamma$,
into either $X_1$ or $X_2$. If we set $N=\lceil 1/\ep\rceil$, then 
setting $a_i=i/N$, then we get a sequence of intervals $J_i=[a_i,a_{i+1}], i=0,\ldots N-1$, 
each mapping
into $X_1$ or $X_2$. If $J_i$ and $J_{i+1}$ both map into the same subpace,
replace them in the sequence with their union. Continuing in this fashion, reducing the number of 
subintervals by one each time, we will eventually
find a collection $I_k$, $k=1,\ldots m$, of intervals covering $I$, 
overlapping only on their
endpoints, which alternately map into $X_1$ and $X_2$. Their endpoints, 
therefore, all map into $X_1\cap X_2$. Setting $y_k=\gamma(I_k\cap I_{k+1})$,
we can, since $X_1\cap X_2$ is path-connected, find a path $\delta_k:I\ra X_1\cap X_2$ 
with $\delta_k(0)=y_k$ and $\delta_k(1)=x_0$. Choosing our favorite homeomorphisms
$h_k:I\ra I_k$ and defining $\eta_k=\gamma|_{I_k}\circ h_k$, we have that, in $\pu(X,x_0)$,

\ctln{$[\gamma]=[\eta_1 * \cdots * \eta_m] 
= [\eta_1*(\delta_1*\overline{\delta_1})*\eta_2* \cdots *\eta_{m-1}* (\delta_{m-1}*\overline{\delta_{m-1}})*\eta_m]$}

\ctln{= $[\eta_1*\delta_1][\overline{\delta_1}*\eta_2*\delta_2] \cdots 
[\overline{\delta_{m-2}}*\eta_{m-1}* \delta_{m-1}][\overline{\delta_{m-1}}*\eta_m]$}


We can insert the $\delta_k*\overline{\delta_k}$ into these products because each is
homotopic to the constant map, and $\eta_k*$(constant) is homotopic to $\eta_k$ by the same sort of homotopy
that established that the constant map represents the identity in the fundamantal group.

\msk

This results in a product of loops (based at $x_0$) which alternately lie in $X_1$ and $X_2$. This product can
therefore be interpreted as lying in $\pi(X_1)*\pu(X_2)$, and maps, under $\phi$, to $[\gamma]$ .
$\phi$ is therefore onto, and
$\pu(X)$ is isomorphic to the free product modulo the kernel of this map $\phi$. 

\msk

Loops $\gamma:(I,\del I)\ra (A,x_0)$, can, using the inclusion-induced maps  
$i_{1*}:\pu(A)\ra \pu(X_1)$ , $i_{2*}:\pu(A)\ra \pu(X_2)$, be thought as either in 
$\pu(X_1)$ or $\pu(X_2)$ . So the word $i_{1*}(\gamma)(i_{2*}(\gamma))^{-1}$, in 
$\pi(X_1)*\pu(X_2)$, is set to the identity in $\pu(X)$ under $\phi$. So these 
elements lie in the kernel of $\phi$.

\msk

{\bf Seifert - van Kampen Theorem:} $\ker(\phi) = <i_{1*}(\gamma)(i_{2*}(\gamma))^{-1} : \gamma\in\pu(A) >^N$,
so $\pu(X)\cong \pu(X_1)*_{\pu(A)}\pu(X_2)$ . 

\bsk

Before we explore the proof of this, let's see what we can do with it!

\msk

{\bf Fundamental groups of graphs:} Every finite connected graph $\Gamma$ has a {\it maximal tree} $T$,
a connected subgraph with no simple circuits. Since any tree is the 
union of smaller trees joined at a vertex, we can, by induction, show that 
$\pu(T) = \{ 1\}$ . In fact, if $e$ is an outermost edge of $T$, then 
$T$ deformation retracts to $T\setminus e$, so, by induction, $T$ is 
contractible. Consequently ({\it Hatcher, Proposition 0.17}), $\Gamma$ and the quotient space $\Gamma/T$
are homotopy equivalent, and so have the same \mpu . But $\Gamma/T=\Gamma_n$
is a bouquet of $n$ circles for some $n$. If we let ${\Cal U}$ = a neighborhood of 
the vertex in $\Gamma_n$, which is contractible, then, by singling out one petal of the bouquet,
we have

\ssk

\ctln{$\Gamma_n = (\Gamma_{n-1}\cup{\Cal U})\cup (\Gamma_1\cup{\Cal U}) = X_1\cup X_2$}

\ssk

with $\Gamma_{k}\cup{\Cal U}\simeq (\Gamma_{k}\cup{\Cal U})/{\Cal U}\cong \Gamma_{k}$. 
And since $X_1\cap X_2={\Cal U}\simeq *$, we have that 

\ctln{$\pu(\Gamma_{n}) \cong \pu(\Gamma_{n-1})*_{1}\pu(\Gamma_1) = \pu(\Gamma_{n-1})*{\Bbb Z}$}

So, by induction, $\pu(\Gamma) \cong \pu(\Gamma_{n})\cong {\Bbb Z}*\cdots *{\Bbb Z} = F(n)$, the free group on $n$ letters, where $n$ = the number of edges not in a maximal tree for $\Gamma$. The generators for the group consist of the
edges not in the tree, prepended and appended by paths to the basepoint.

\msk

{\it Gluing on a 2-disk:} If $X$ is a topological space and $f:\del {\Bbb D}^2\ra X$ is continuous, then we
can construct the quotient space $Z=(X\coprod {\Bbb D}^2)/\{x\sim f(x) : x\in\del{\Bbb D}^2\}$,
the result of gluing ${\Bbb D}^2$ to $X$ along $f$. 
We can use Seifert - van Kampen to compute \mpu\ of the resulting space, although if we
wish to be careful with basepoints $x_0$ 
(e.g., the image of $f$ might not contain $x_0$, and/or we
may wish to glue several disks on, in remote parts of $X$),
we should also include a rectangle $R$, the mapping cylinder of a path $\gamma$ running from 
$f(1,0)$ to $x_0$, glued to 
${\Bbb D}^2$ along the arc from $(1/2,0)$ to $(1,0)$ (see figure). 
This space $Z_+$ deformation retracts to $Z$, but it
is technically simpler to do our calculations with the basepoint $y_0$ lying above $x_0$.
If we write $D_1 = \{x\in {\Bbb D}^2 : ||x||<1\}\cup(R\setminus X)$ 
and $D_2 = \{x\in {\Bbb D}^2 : ||x||>1/3\}\cup R$ , then we can write $Z_+=D_+\cup(X\cup D_2) = X_1\cup X_2$.
But since $X_1\simeq *$ , $X_2\simeq X$ 
(it is essentially the mapping cylinder of the maps $f$ and $\gamma$ )
and $X_1\cap X_2 = \{x\in {\Bbb D}^2 : 1/3<||x||<1\}\cap(R\setminus X)\sim S^1$, we find that 

\ctln{$\pu(Z,y_0)\cong \pu(X_2,y_0)*_{\Bbb Z}\{1\} = \pu(X_2)/<{\Bbb Z}>^N \cong \pu(X_2)/<[\overline{\delta}*\overline{\gamma}*f*\gamma*\delta]>^N$}

If we then use $\delta$ as a path for a change of basepoint isomorphism, and then a homotopy
equivalence from $X_2$ to $X$ (fixing $x_0$), we have, in terms of group presentations, 
if $\pu(X,x_0)=<\sig | R>$ , then $\pu(Z) = <\sig | R\cup\{[\overline{\gamma}*f*\gamma]\}>$ . 
So the effect of gluing on a 2-disk on the fundamental group is to add a new relator, 
namely the word represented by the attaching map (adjusting for basepoint).

\msk


\vbox{\hsize=5.8in

\leavevmode

\epsfxsize=5.8in
\epsfbox{0201f1.ai}}



\msk

This in turn opens up huge possibilities for the computation of $\pu(X)$. For example, for cell complexes,
we can inductively compute \mpu\ by starting with the 1-skeleton, with free fundamental group, and 
attaching the 2-cells one by one, which each add a relator to the presentation of $\pu(X)$ . [{\bf Exercise:}
(Hatcher, p.53, \#\ 6) Attaching $n$-cells, for $n\geq 3$, has no effect on \mpu .] As a specific example, we can compute the fundamental group of any compact surface:


\vfill
\end









