
\magnification=1200
\overfullrule=0pt
\parindent=0pt

\nopagenumbers

\input amstex

\voffset=-.6in
\hoffset=-.5in
\hsize = 7.5 true in
\vsize=10.4 true in

%\voffset=1.4in
%\hoffset=-.5in
%\hsize = 10.2 true in
%\vsize=8 true in

\input colordvi

\loadmsbm

\input epsf

\def\ctln{\centerline}
\def\u{\underbar}
\def\ssk{\smallskip}
\def\msk{\medskip}
\def\bsk{\bigskip}
\def\hsk{\hskip.1in}
\def\hhsk{\hskip.2in}
\def\dsl{\displaystyle}
\def\hskp{\hskip1.5in}

\def\lra{$\Leftrightarrow$ }
\def\ra{\rightarrow}
\def\mpto{\logmapsto}
\def\pu{\pi_1}
\def\mpu{$\pi_1$}
\def\sig{\Sigma}
\def\msig{$\Sigma$}
\def\ep{\epsilon}
\def\sset{\subseteq}
\def\del{\partial}
\def\inv{^{-1}}
\def\wtl{\widetilde}
\def\lra{\Leftrightarrow}



\ctln{\bf Math 971 Algebraic Topology}

\ssk

\ctln{February 22, 2005}

\msk

{\bf Building universal coverings:} If a space $X$ is path connected, locally path connected, and
semi-locally simply connected (S-LSC), then it has a universal covering;
we describe a general construction. The idea is that a covering
space should have the path lifting and homotopy
lifting properties, and the universal 
cover can be characterized as the only covering space for 
which {\it only} null-homotopic loops lift to loops. So we build a 
space and a map which \underbar{must} have these properties.
We do this by making a space $\widetilde{X}$ whose
points are (equivalence classes of) $[\gamma]$
based paths $\gamma:(I,0)\ra (X,x_0)$, where two paths are equivalent
if they are homotopic rel endpoints! The projection map is
$p([\gamma])=\gamma(1)$. The S-LSCness is what guarantees that this is a 
covering map; choosing $x\in X$, a path $\gamma_0$ from $x_0$ to $x$,
and a  neighborhood ${\Cal U}$ of $x$ guaranteed by S-LSC, paths from 
$x_0$ to points in ${\Cal U}$ are based equivalent to $\gamma*\gamma_0*\eta$
where $\gamma$ is a based loop at $x_0$ and $\eta$ is a path in ${\Cal U}$.
But by simple connectivity, a path in ${\Cal U}$ is determined up to homotopy
by its endpoints, and so, with $\gamma$ fixed, these paths are in one-to-one
correspondence with ${\Cal U}$. So $p\inv({\Cal U}$ is a disjoint union,
indexed by $\pu(X,x_0)$, of sets in bijective correspondence with ${\Cal U}$.
The appropriate topology on $\widetilde{X}$, essentially given as a basis
by triples $\gamma*,gamma_0,{\Cal U}$ as above, make $p$ a covering map.
Note that the inverse image of 
the basepoint $x_0$ is the equivalence classes of \underbar{loops} at $x_0$,
i.e., $\pu(X,x_0)$. A path $\gamma$ lifts to the path of paths
$[\gamma_t]$, where $\gamma_t(s)=\gamma(ts)$, and so the only 
loop in $X$ which lifts to a loop in $\widetilde{X}$ has endpoint
$[\gamma]=[c_{x_0}]$, i.e., $[\gamma]=1$ in $\pu(X,x_0)$. This
implies that $p_*(\pu(\widetilde{X},[c_{x_0}]))=\{1\}$, so 
$\pu(\widetilde{X},[c_{x_0}])=\{1\}$ . \hhsk However, nobody in their
right minds would go about building $\widetilde{X}$ in this way, in general!
Before describing how to do it ``right'', though, we should perhaps see why
we should want to?

\msk

One reason for the importance of the universal cover is that it gives
us a unified approach to building \underbar{all} connected covering
spaces of $X$. The basis for this is the {\it deck transformation group}
of a covering space $p:\wtl{X}\ra X$; this is the set of all
homeomorphisms $h:\wtl{X}\ra\wtl{X}$ such that $p\circ h = p$.
These homeomorphisms, by definition, permute each of the point inverses
of $p$. In fact, since $h$ can be thought of as a lift of the projection
$p$, by the lifting criterion $h$ is determined by which point in the 
inverse image of the basepoint $x_0$ it takes the basepoint 
$\wtl{x}_0$ of $\wtl{X}$ to. A deck transformation sending
$\wtl{x}_0$ to $\wtl{x}_1$ exists $\Leftrightarrow$
$p_*(\pu(\wtl{X},\wtl{x}_0)=p_*(\pu(\wtl{X},\wtl{x}_1)$
[we need one inclusion to give the map $h$, and the opposite inclusion
to ensure it is a bijection (because its inverse exists)]. These two groups
are in general {\it conjugate}, by the projection of a path from 
$\wtl{x}_0$ to $\wtl{x}_1$; this can be seen by following the change
of basepoint isomorphism down into $G=\pu(X,x_0)$. As we have seen, paths
in $\wtl{X}$ from $\wtl{x}_0$ to $\wtl{x}_1$ are in 1-to-1
correspondence with the cosets of $H=p_*(\pu(\wtl{X},\wtl{x}_0)$ in 
$p_*(\pu({X},{x}_0)$; so deck transformations are in 1-to-1 
correspondence with cosets whose representatives conjugate 
$H$ to itself. The set of such elements in $G$ is called the 
{\it normalizer of $H$ in $G$}, and denoted $N_G(H)$ or simply
$N(H)$. The deck transformation group is therefore
in 1-to-1 correspondence with the group $N(H)/H$ under
$h\mapsto$ the coset represented by the projection of the path from 
$\wtl{x}_0$ to $h(\wtl{x}_0)$. And since $h$ is essentially built
by lifting paths, it follows quickly that this map is a
homomorphism, hence an isomorphism.

\msk

In particular, applying this to the universal covering space
$p:\wtl{X}\ra X$, since in this case $H=\{1\}$, so $N(H)=\pu(X,x_0)$,
its deck transformation group is isomorphic to $\pu(X,x_0)$. 
For example, this gives the quickest possible proof 
that $\pu(S^1)\cong {\Bbb Z}$, since ${\Bbb R}$ is a 
contractible covering space, whose deck transformations
are the translations by integer distances. 
Thus $\pu(X)$ acts on its universal cover as a group of
homeomorphisms. And since this action is {\it simply transitive}
on point inverses [there is exactly one (that's the simple
part) deck transformation carrying any one point in a point 
inverse to any other one (that's the transitive part)], the 
quotient map from $\wtl{X}$ to the orbits of this action \underbar{is}
the projection map $p$. The evenly covered property of $p$ implies
that $X$ does have the quotient topology under this action.

\msk

So every space it $X$ the quotient of its universal cover (if it has
one!) by its fundamental group $G=\pu(X,x_0)$, realized as the group
of deck transformations. And the quotient map is the covering 
projection. So $X|cong \wtl{X}/G$ . In general, a quotient of a 
space $Z$ by a group action $G$ 
need not be 
a covering map; the action must be {\it properly discontinuous}, 
which means that for every point 
 $z\in Z$, there is a neighborhood ${\Cal U}$ of $x$ so that $g\neq 1$ $\Rightarrow$
${\Cal U}\cap g{\Cal U}=\emptyset$ (the group action carries sufficiently
small neighborhoods off of themselves). The evenly covered neighborhoods
provide these for the universal cover. And conversely, the quotient of a space by a 
p.d. group action is a covering space. 

\msk

But! Given $G=\pu(X,x_0)$ and its 
action on a universal cover $\wtl{X}$, we can, instead of quotienting out by $G$,
quotient out by any \underbar{subgroup} $H$ of $G$, to build $X_H=\wtl{X}/H$. 
This is a space with fundamental group $H$, having $\wtl{X}$ as universal covering.
And since the quotient (covering) map $p_G:\wtl{X}\ra X=\wtl{X}/G$ factors through $\wtl{X}/H$,
we get an induced map $p_H:\wtl{X}/H\ra X$, which is a covering map; open sets with
trivial inclusion-induced homomorphism lift homeomorphically to $\wtl{X}$,
hence homeomorphically to $\wtl{X}/H$; taking lifts to each point inverse of $x\in X$
verifies the evenly covering property for $p_H$ . So every subgroup of $G$ is the
fundamental group of a covering of $X$. 

\ssk

We can further refine this to give the {\it Galois correspondence}. Two covering spaces
$p_1:X_1\ra X$ , $p_2:X_2\ra X$ are {\it isomorphic} if there is a homeomorphism
$h:X_1\ra X_2$ with $p_1=p_2\circ h$. Choosing basepoints $x_1,x_2$ mapping to $x_0\in X$,
this implies that, if $h(x_1)=x_2$, then $p_{1*}(\pu(X_1,x_1)) = p_{2*}(h_*(\pu(X_1,x_1))) = 
p_{2*}(\pu(X_2,x_2))$ . On the other hand, our homeomorphism $h$ need not take our
chosen basepoints to one another; if $h(x_1)=x_3$, then $p_{1*}(\pu(X_1,x_1)) = p_{2*}(\pu(X_2,x_3))$.
But  $p_{2*}(\pu(X_2,x_2))$ and $p_{2*}(\pu(X_2,x_3))$ are isomorphic, via a change 
of basepoint isomorphism $\widehat{\eta}$ , where $\eta$ is a path in $X_2$ from $x_2$ to $x_3$.
But such a path projects to $X$ has a loop at $x_0$, and since the change of basepoint isomorphism
is by ``conjugating'' by the path $\eta$, the resulting groups $p_{2*}(\pu(X_2,x_2))$ and $p_{2*}(\pu(X_2,x_3))$
are conjugate, by $p_2\circ \eta$ . So, without reference to basepoints, isomorphic coverings give,
under projection, conjugate subgroups of $\pu(X,x_0)$ . But conversely, given covering spaces
$X_1,X_2$ whose subgroups $p_{1*}(\pu(X_1,x_1))$ and $p_{2*}(\pu(X_2,x_2))$ are conjugate,
lifting a loop $\gamma$ representing the conjugating element to a loop $\wtl{\gamma}$ in
$X_2$ starting at $x_2$ gives, as its terminal endpoint, a point $x_3$ with 
$p_{1*}(\pu(X_1,x_1)) = p_{2*}(\pu(X_2,x_3))$ (since it conjugates back!), and so, by the lifting criterion,
there is an isomorphism $h:(X_1,x_1) \ra (X_2,x_3)$. So conjugate subgroups give isomorphic coverings.
Thus, for a path-connected, locally path-connected, semi-locally simply-connected space $X$,  
the image of the induced homomorphism on \mpu\ 
gives a one-to-one correspondence between 
[isomorphism classes of (connected) coverings of $X$] and 
[conjugacy classes of subgroups of $\pu(X)$].

\msk

So, for example, if you have a group $G$ that you are interested in, you know of a (nice enough) 
space $X$ with $\pu(X)\cong G$, and you know enough about the covering of $X$, then you can
gain information about the subgroup structure of $G$. For example, and in some respects as
motivation for all of this machinery!, a free group $F(\Sigma)$ is \mpu\ of a bouquet of circles $X$. 
Any covering space $\wtl{X}$ of $X$ is a union of vertices and edges, so is a graph.  Collapsing
a maximal tree to a point, $\wtl{X}$ is $\simeq$ a bouquet of circles, so has free \mpu . So, every
subgroup of a free group is free. (That is a lot shorter than the original, group-theoretic, proof...) 
A subgroup $H$ of index $n$ in $F(\Sigma)$ corresponds to a $n$-sheeted covering $\wtl{X}$ of $X$. If
$|\Sigma| = m$, then $\wtl{X}$ will have $n$ vertices and $nm$ edges. Collapsing a maximal
tree, having $n-1$ edges to a point, leaves a bouquet of $nm-n+1$ circles, so $H\cong F(nm-n+1)$.
For example, for $m=3$, index $n$ subgroups are free on $2n+1$ generators, so every free subgroup
on 4 generators has infinite index in $F(3)$. Try proving that directly!

\msk

{\it Kurosh Subgroup Theorem}: If $H < G_1*G_2$ is a subgroup of
a free product, then $H$ is (isomorphic to) a free product of a
collection of conjugates of subgroups of $G_1$ and $G_2$ and a 
free froup. The proof is to build a space by taking 2-complexes
$X_1$ and $X_2$ with $\pu$'s isomorphic to $G_1,G_2$ and join
their basepoints by an arc. The covering space of this space $X$
corresponding to $H$ consists of spaces that cover $X_1,X_2$
(giving, after basepoint considerations, the conjugates)
connected by a collection of arcs (which, suitably interpreted,
gives the free group).

\msk

{\it Residually finite groups}: $G$ is said to be residually finite if for every $g\neq 1$ there is a 
finite group $F$ and a homomorphism $\varphi: G\ra F$ with $\varphi(g)\neq 1$ in $F$. This 
amounts to saying that $g\notin$ the (normal) subgroup $\ker(\varphi)$, which amounts to
saying that a loop corresponding to $g$ does \underbar{not} lift to a loop in the finite-sheeted
covering space corresponding to $\ker(\varphi)$. So residual finiteness of a group can be
verified by building coverings of a space $X$ with $\pu(X)=G$. For example, free groups can be
shown to be residually finite in this way. 

\msk

{\it Ranks of free (sub)groups:} A free group on $n$
generators is isomorphic to a free group on $m$ generators
$\lra$ $n=m$; this is because the abelianizations of the two 
groups are ${\Bbb Z}^n,{\Bbb Z}^m$. The (minimal) number of 
generators for a free group is called its {\it rank}.
Given a free group
$G=F(a_1,\ldots a_n)$ and a collection of words $w_1,\ldots w_m\in G$,
we can determine the rank and ndex of the subgroup it $H$ they
generate by building the corresponding cover. The idea is
to start with a bouquet of $m$ circles, each subdivided 
and labelled to spell
out the words $w_i$. Then we repeatedly identify edges sharing
on common vertex if they are labelled precisely the same (same
letter {\it and} same orientation). This process is known
as {\it folding}. One can inductively show that the (obvious)
maps from these graphs to the bouquet of $n$ circles $X_n$ both
have image $H$ under the induced maps on \mpu ; the graphs
are in fact homotopy equivalent, and the map for the unfolded graph
factors through the one for the folded graph. We continue until there
is no more folding to be done; the resulting graph $X$ is what is 
known (in combinatorics) as a {\it graph covering}; the map to $X_n$
is locally injective. If this map is a covering map, then our subgroup
$H$ has finite index (equal to the degree of the
covering) and we can compute the rank of $H$ (and a basis!) from this
index as above. If not, then the map is not locally surjective at
some vertices; if we graft trees onto these vertices, we can extend the map
to an (infinite-sheeted) covering map without changing the homotopy
type of the graph. $H$ therefore has infinite index in $G$, and its
rank can be computed from $H\cong \pu(X)$. An example of this procedure
is given below.

\msk

{\bf Postscript: why care about covering spaces?} The preceding discussion
probably makes it clear that covering places play a central role in
(combinatorial) group theory. It also plays a role in embedding 
problems; a common scenario is to have a map $f:Y\ra X$ which is 
injective on \mpu , and we wish to know if we can lift $f$ to a 
finite-sheeted covering so that the lifted map $\widetilde{f}$ is 
homotopic to an embedding. Information that is easier to obtain 
in the case of an embedding can then be passed down to gain information
abut the original map $f$. And covering spaces underlie the 
theory of analytic continuation in complex analysis; starting
with a domain $D\subseteq {\Bbb C}$, what analytic continuation really
builds is an (analytic) function from a covering space of $D$ to ${\Bbb C}$.
For example, the logarithm is really defined as a map from 
the universal cover of ${\Bbb C}\setminus\{0\}$ to ${\Bbb C}$. 
The various ``branches'' of the logarithm refer to which sheet
in this cover you are in.

\bsk

{\bf Homology theory:} 


\vfill
\end