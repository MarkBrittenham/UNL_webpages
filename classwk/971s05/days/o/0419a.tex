

\magnification=1200
\overfullrule=0pt
\parindent=0pt

\nopagenumbers

\input amstex

\voffset=-.6in
\hoffset=-.5in
\hsize = 7.5 true in
\vsize=10.4 true in

%\voffset=1.4in
%\hoffset=-.5in
%\hsize = 10.2 true in
%\vsize=8 true in

\input colordvi

\def\cltr{\Red}		  % Red  VERY-Approx PANTONE RED
\def\cltb{\Blue}		  % Blue  Approximate PANTONE BLUE-072
\def\cltg{\PineGreen}	  % ForestGreen  Approximate PANTONE 349
\def\cltp{\DarkOrchid}	  % DarkOrchid  No PANTONE match
\def\clto{\Orange}	  % Orange  Approximate PANTONE ORANGE-021
\def\cltpk{\CarnationPink}	  % CarnationPink  Approximate PANTONE 218
\def\clts{\Salmon}	  % Salmon  Approximate PANTONE 183
\def\cltbb{\TealBlue}	  % TealBlue  Approximate PANTONE 3145
\def\cltrp{\RoyalPurple}	  % RoyalPurple  Approximate PANTONE 267
\def\cltp{\Purple}	  % Purple  Approximate PANTONE PURPLE

\def\cgy{\GreenYellow}     % GreenYellow  Approximate PANTONE 388
\def\cyy{\Yellow}	  % Yellow  Approximate PANTONE YELLOW
\def\cgo{\Goldenrod}	  % Goldenrod  Approximate PANTONE 109
\def\cda{\Dandelion}	  % Dandelion  Approximate PANTONE 123
\def\capr{\Apricot}	  % Apricot  Approximate PANTONE 1565
\def\cpe{\Peach}		  % Peach  Approximate PANTONE 164
\def\cme{\Melon}		  % Melon  Approximate PANTONE 177
\def\cyo{\YellowOrange}	  % YellowOrange  Approximate PANTONE 130
\def\coo{\Orange}	  % Orange  Approximate PANTONE ORANGE-021
\def\cbo{\BurntOrange}	  % BurntOrange  Approximate PANTONE 388
\def\cbs{\Bittersweet}	  % Bittersweet  Approximate PANTONE 167
%\def\creo{\RedOrange}	  % RedOrange  Approximate PANTONE 179
\def\cma{\Mahogany}	  % Mahogany  Approximate PANTONE 484
\def\cmr{\Maroon}	  % Maroon  Approximate PANTONE 201
\def\cbr{\BrickRed}	  % BrickRed  Approximate PANTONE 1805
\def\crr{\Red}		  % Red  VERY-Approx PANTONE RED
\def\cor{\OrangeRed}	  % OrangeRed  No PANTONE match
\def\paru{\RubineRed}	  % RubineRed  Approximate PANTONE RUBINE-RED
\def\cwi{\WildStrawberry}  % WildStrawberry  Approximate PANTONE 206
\def\csa{\Salmon}	  % Salmon  Approximate PANTONE 183
\def\ccp{\CarnationPink}	  % CarnationPink  Approximate PANTONE 218
\def\cmag{\Magenta}	  % Magenta  Approximate PANTONE PROCESS-MAGENTA
\def\cvr{\VioletRed}	  % VioletRed  Approximate PANTONE 219
\def\parh{\Rhodamine}	  % Rhodamine  Approximate PANTONE RHODAMINE-RED
\def\cmu{\Mulberry}	  % Mulberry  Approximate PANTONE 241
\def\parv{\RedViolet}	  % RedViolet  Approximate PANTONE 234
\def\cfu{\Fuchsia}	  % Fuchsia  Approximate PANTONE 248
\def\cla{\Lavender}	  % Lavender  Approximate PANTONE 223
\def\cth{\Thistle}	  % Thistle  Approximate PANTONE 245
\def\corc{\Orchid}	  % Orchid  Approximate PANTONE 252
\def\cdo{\DarkOrchid}	  % DarkOrchid  No PANTONE match
\def\cpu{\Purple}	  % Purple  Approximate PANTONE PURPLE
\def\cpl{\Plum}		  % Plum  VERY-Approx PANTONE 518
\def\cvi{\Violet}	  % Violet  Approximate PANTONE VIOLET
\def\clrp{\RoyalPurple}	  % RoyalPurple  Approximate PANTONE 267
\def\cbv{\BlueViolet}	  % BlueViolet  Approximate PANTONE 2755
\def\cpe{\Periwinkle}	  % Periwinkle  Approximate PANTONE 2715
\def\ccb{\CadetBlue}	  % CadetBlue  Approximate PANTONE (534+535)/2
\def\cco{\CornflowerBlue}  % CornflowerBlue  Approximate PANTONE 292
\def\cmb{\MidnightBlue}	  % MidnightBlue  Approximate PANTONE 302
\def\cnb{\NavyBlue}	  % NavyBlue  Approximate PANTONE 293
\def\crb{\RoyalBlue}	  % RoyalBlue  No PANTONE match
%\def\cbb{\Blue}		  % Blue  Approximate PANTONE BLUE-072
\def\cce{\Cerulean}	  % Cerulean  Approximate PANTONE 3005
\def\ccy{\Cyan}		  % Cyan  Approximate PANTONE PROCESS-CYAN
\def\cpb{\ProcessBlue}	  % ProcessBlue  Approximate PANTONE PROCESS-BLUE
\def\csb{\SkyBlue}	  % SkyBlue  Approximate PANTONE 2985
\def\ctu{\Turquoise}	  % Turquoise  Approximate PANTONE (312+313)/2
\def\ctb{\TealBlue}	  % TealBlue  Approximate PANTONE 3145
\def\caq{\Aquamarine}	  % Aquamarine  Approximate PANTONE 3135
\def\cbg{\BlueGreen}	  % BlueGreen  Approximate PANTONE 320
\def\cem{\Emerald}	  % Emerald  No PANTONE match
%\def\cjg{\JungleGreen}	  % JungleGreen  Approximate PANTONE 328
\def\csg{\SeaGreen}	  % SeaGreen  Approximate PANTONE 3268
\def\cgg{\Green}	  % Green  VERY-Approx PANTONE GREEN
\def\cfg{\ForestGreen}	  % ForestGreen  Approximate PANTONE 349
\def\cpg{\PineGreen}	  % PineGreen  Approximate PANTONE 323
\def\clg{\LimeGreen}	  % LimeGreen  No PANTONE match
\def\cyg{\YellowGreen}	  % YellowGreen  Approximate PANTONE 375
\def\cspg{\SpringGreen}	  % SpringGreen  Approximate PANTONE 381
\def\cog{\OliveGreen}	  % OliveGreen  Approximate PANTONE 582
\def\pars{\RawSienna}	  % RawSienna  Approximate PANTONE 154
\def\cse{\Sepia}		  % Sepia  Approximate PANTONE 161
\def\cbr{\Brown}		  % Brown  Approximate PANTONE 1615
\def\cta{\Tan}		  % Tan  No PANTONE match
\def\cgr{\Gray}		  % Gray  Approximate PANTONE COOL-GRAY-8
\def\cbl{\Black}		  % Black  Approximate PANTONE PROCESS-BLACK
\def\cwh{\White}		  % White  No PANTONE match


\loadmsbm

\input epsf

\def\ctln{\centerline}
\def\u{\underbar}
\def\ssk{\smallskip}
\def\msk{\medskip}
\def\bsk{\bigskip}
\def\hsk{\hskip.1in}
\def\hhsk{\hskip.2in}
\def\dsl{\displaystyle}
\def\hskp{\hskip1.5in}

\def\lra{$\Leftrightarrow$ }
\def\ra{\rightarrow}
\def\mpto{\logmapsto}
\def\pu{\pi_1}
\def\mpu{$\pi_1$}
\def\sig{\Sigma}
\def\msig{$\Sigma$}
\def\ep{\epsilon}
\def\sset{\subseteq}
\def\del{\partial}
\def\inv{^{-1}}
\def\wtl{\widetilde}
\def\lra{\Leftrightarrow}
\def\del{\partial}
\def\delp{\partial^\prime}
\def\delpp{\partial^{\prime\prime}}
\def\sgn{{\roman{sgn}}}
\def\wtih{\widetilde{H}}
\def\bbz{{\Bbb Z}}
\def\bbr{{\Bbb R}}



\ctln{\bf Math 971 Algebraic Topology}

\ssk

\ctln{April 19, 2005}

\msk




The isomorphism between simplicial and singular homology provides very quick proofs
of several results about singular homology, which would other would require some effort:

\ssk

{\it If the $\Delta$-complex $X$ has no simplices in dimension greater than $n$, then 
$H_i(X)=0$ for all $i>n$.}

\ssk

This is because the simplicial chain groups $C_i^\Delta(X)$ are $0$, so $H_i^\Delta(X)=0$ .

\ssk

{\it If for each $n$, the $\Delta$-complex $X$ has finitely many $n$-simplices, then 
$H_n(X)$ is finitely generated for every $n$.}

\ssk

This is because the simplicial chain groups $C_n^\Delta(X)$ are all finitely generated,
so $H_n^\Delta(X)$, being a quotient of a subgroup, is also finitely generated. [We
are using here that the number of generators of a subgroup $H$ of an {\it abelian} 
group $G$ is no larger than that for $G$; this is not true for groups in general!]


\bigskip

Some more topological results with homological proofs: The Klein bottle and real projective plane cannot 
embed in $\bbr^3$. This is because a surface $\Sigma$ embedded in $\bbr^3$ has a (the proper word is {\it normal})
neighborhood $N(\Sigma)$, which deformation retracts to $\Sigma$; literally, it is all points within a (uniformly) short distance
in the normal direction from the point on the surface $\Sigma$. Our non-embeddedness result follows (by contradiction)
from applying Mayer-Vietoris to the pair $(A,B) = (\overline{N(\Sigma)},\overline{\bbr^3\setminus N(\Sigma)})$, whose intersection
is the boundary $F=\del N(\Sigma)$ of the normal neighborhood. The point, though, is that
$F$ is an orientable surface; the outward normal (pointing away from $N(\Sigma)$) at every point, taken as
the first vector of a right-handed orientation of $\bbr^3$ allows us to use the other two vectors as an 
orientation of the surface. So $F$ is one of the surface $F_g$ above whose homologies we just computed.
This gives the LES
\hhsk
$\wtih_2(\bbr^3) \ra \wtih_1(F) \ra \wtih_1(A)\oplus \wtih_1(B)\ra \wtih_1(\bbr ^3)$
\hhsk 
which renders as 
\hhsk
$0\ra\bbz^{2g}\ra \wtih(\Sigma)\oplus G\ra 0$
\hhsk , i.e., \hhsk
$\bbz^{2g}\cong \wtih(\Sigma)\oplus G$ 
\hhsk . But for the Klein bottle and projective plane (or any closed, non-orientable
surface for that matter), $\wtih_1(\Sigma)$ has torsion, so it cannot be the direct
summand of a torsion-free group! So no such embedding exists. This result holds
more generally for any 2-complex $K$ whose (it turns out it would have to be first)
homology has torsion; any embedding into $\bbr^3$ would have a neighborhood 
deformation retracting to $K$, with boundary a (for the exact same reasons as above)
closed orientable surface.

\msk

Another: if ${\Bbb R}^n\cong {\Bbb R}^m$, via $h$, then $n=m$ .
This is because we can arrange, by composing with a translation, that $h(0)=0$, and then 
we have 
$({\Bbb R}^n,{\Bbb R}^n\setminus 0)\cong {\Bbb R}^m,({\Bbb R}^m\setminus 0)$, which gives


\ssk

\ctln{$\widetilde{H}_i(S^{n-1})\cong H_{i+1}({\Bbb D}^n,\del {\Bbb D}^n) \cong H_{i+1}({\Bbb D}^n,{\Bbb D}^n\setminus 0)
\cong H_{i+1}({\Bbb R}^n,{\Bbb R}^n\setminus 0) \cong H_{i+1}({\Bbb R}^m,{\Bbb R}^m\setminus 0)$}

\ctln{$\cong H_{i+1}({\Bbb D}^m,{\Bbb D}^m\setminus 0) \cong H_{i+1}({\Bbb D}^m,\del {\Bbb D}^m)
\cong \widetilde{H}_i(S^{m-1})$}

\ssk

Setting $i=n-1$ gives the result, since $\widetilde{H}_{n-1}(S^{m-1})\cong {\Bbb Z}$ implies $n-1=m-1$ .

\msk

More generally, we can establish a result which is known as {\it invariance of domain},
which is useful in both topology and analysis.

\msk

{\bf Invariance of Domain:} If ${\Cal U}\subseteq {\Bbb R}^n$ and $f:{\Cal U}\ra {\Bbb R}^n$
is continuous and injective, then $f({\Cal U})\subseteq {\Bbb R}^n$ is open.

\msk

We will defer this proof for awhile (perhaps permanently?). 

Note it is enough to proof this for our favorite open set, which in this context will be ${\Cal V}=(-1,1)^n\subseteq {\Bbb R}^n$,
since given any open ${\Cal U}$ and $x\in{\Cal U}$, we can find an injective linear map $h:(-1,1)^n\ra {\Cal U}$
taking $0$ to $x$. If we can show that $f\circ h$ has open image, then $f(x)\in f\circ h({\Cal V})\subseteq f({\Cal U})$
shows that $f(x)$ has an open neighborhood in $f({\Cal U})$ . Since $x$ is arbitrary, $f({\Cal U})$ is open.

\msk

\bsk

\msk

This in turn implies the ``other'' invariance of domain; if $f: {\Bbb R}^n\ra {\Bbb R}^m$ is continuous and injective, then
$n\leq m$, since if not, then composition of $f$ with the inclusion $i:{\Bbb R}^m\ra {\Bbb R}^n$, $i(x_1,\ldots ,x_m) = 
(x_1,\ldots ,x_m,0,\ldots ,0)$ is injective and continuous with non-open image (it lies in a hyperplane in ${\Bbb R}^n$),
a contradiction.


\bsk

\vfill
\end



A map of pairs $f: (X,A) \ra (Y,B)$ (meaning that $f(A)\subseteq B$)
induces (by postcomposition) a map of relative homology $f_*:H_i(X,A)\ra H_i(Y,B)$ , just as with 
absolute homology.
We also get a homotopy-invariance result: if $f,g: (X,A) \ra (Y,B)$ are maps of pairs
which are {\it homotopic as maps of pairs},
i.e., there is a map $(X\times I,A\times I)\ra (Y,B)$ which is $f$ on one end and $g$
on the other, then $f_*=g_*$ . The proof is identical to the one given before; the prism map 
$P$ sends chains in $A$ to chains in $A$, so induces a map $C_i(X\times I,A\times I)\ra C_{i+1}(X,A)$
which does precisely what we want.



