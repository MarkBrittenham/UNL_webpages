

\magnification=1200
\overfullrule=0pt
\parindent=0pt

\nopagenumbers

\input amstex

\voffset=-.6in
\hoffset=-.5in
\hsize = 7.5 true in
\vsize=10.4 true in

%\voffset=1.4in
%\hoffset=-.5in
%\hsize = 10.2 true in
%\vsize=8 true in

\input colordvi

\def\cltr{\Red}		  % Red  VERY-Approx PANTONE RED
\def\cltb{\Blue}		  % Blue  Approximate PANTONE BLUE-072
\def\cltg{\PineGreen}	  % ForestGreen  Approximate PANTONE 349
\def\cltp{\DarkOrchid}	  % DarkOrchid  No PANTONE match
\def\clto{\Orange}	  % Orange  Approximate PANTONE ORANGE-021
\def\cltpk{\CarnationPink}	  % CarnationPink  Approximate PANTONE 218
\def\clts{\Salmon}	  % Salmon  Approximate PANTONE 183
\def\cltbb{\TealBlue}	  % TealBlue  Approximate PANTONE 3145
\def\cltrp{\RoyalPurple}	  % RoyalPurple  Approximate PANTONE 267
\def\cltp{\Purple}	  % Purple  Approximate PANTONE PURPLE

\def\cgy{\GreenYellow}     % GreenYellow  Approximate PANTONE 388
\def\cyy{\Yellow}	  % Yellow  Approximate PANTONE YELLOW
\def\cgo{\Goldenrod}	  % Goldenrod  Approximate PANTONE 109
\def\cda{\Dandelion}	  % Dandelion  Approximate PANTONE 123
\def\capr{\Apricot}	  % Apricot  Approximate PANTONE 1565
\def\cpe{\Peach}		  % Peach  Approximate PANTONE 164
\def\cme{\Melon}		  % Melon  Approximate PANTONE 177
\def\cyo{\YellowOrange}	  % YellowOrange  Approximate PANTONE 130
\def\coo{\Orange}	  % Orange  Approximate PANTONE ORANGE-021
\def\cbo{\BurntOrange}	  % BurntOrange  Approximate PANTONE 388
\def\cbs{\Bittersweet}	  % Bittersweet  Approximate PANTONE 167
%\def\creo{\RedOrange}	  % RedOrange  Approximate PANTONE 179
\def\cma{\Mahogany}	  % Mahogany  Approximate PANTONE 484
\def\cmr{\Maroon}	  % Maroon  Approximate PANTONE 201
\def\cbr{\BrickRed}	  % BrickRed  Approximate PANTONE 1805
\def\crr{\Red}		  % Red  VERY-Approx PANTONE RED
\def\cor{\OrangeRed}	  % OrangeRed  No PANTONE match
\def\paru{\RubineRed}	  % RubineRed  Approximate PANTONE RUBINE-RED
\def\cwi{\WildStrawberry}  % WildStrawberry  Approximate PANTONE 206
\def\csa{\Salmon}	  % Salmon  Approximate PANTONE 183
\def\ccp{\CarnationPink}	  % CarnationPink  Approximate PANTONE 218
\def\cmag{\Magenta}	  % Magenta  Approximate PANTONE PROCESS-MAGENTA
\def\cvr{\VioletRed}	  % VioletRed  Approximate PANTONE 219
\def\parh{\Rhodamine}	  % Rhodamine  Approximate PANTONE RHODAMINE-RED
\def\cmu{\Mulberry}	  % Mulberry  Approximate PANTONE 241
\def\parv{\RedViolet}	  % RedViolet  Approximate PANTONE 234
\def\cfu{\Fuchsia}	  % Fuchsia  Approximate PANTONE 248
\def\cla{\Lavender}	  % Lavender  Approximate PANTONE 223
\def\cth{\Thistle}	  % Thistle  Approximate PANTONE 245
\def\corc{\Orchid}	  % Orchid  Approximate PANTONE 252
\def\cdo{\DarkOrchid}	  % DarkOrchid  No PANTONE match
\def\cpu{\Purple}	  % Purple  Approximate PANTONE PURPLE
\def\cpl{\Plum}		  % Plum  VERY-Approx PANTONE 518
\def\cvi{\Violet}	  % Violet  Approximate PANTONE VIOLET
\def\clrp{\RoyalPurple}	  % RoyalPurple  Approximate PANTONE 267
\def\cbv{\BlueViolet}	  % BlueViolet  Approximate PANTONE 2755
\def\cpe{\Periwinkle}	  % Periwinkle  Approximate PANTONE 2715
\def\ccb{\CadetBlue}	  % CadetBlue  Approximate PANTONE (534+535)/2
\def\cco{\CornflowerBlue}  % CornflowerBlue  Approximate PANTONE 292
\def\cmb{\MidnightBlue}	  % MidnightBlue  Approximate PANTONE 302
\def\cnb{\NavyBlue}	  % NavyBlue  Approximate PANTONE 293
\def\crb{\RoyalBlue}	  % RoyalBlue  No PANTONE match
%\def\cbb{\Blue}		  % Blue  Approximate PANTONE BLUE-072
\def\cce{\Cerulean}	  % Cerulean  Approximate PANTONE 3005
\def\ccy{\Cyan}		  % Cyan  Approximate PANTONE PROCESS-CYAN
\def\cpb{\ProcessBlue}	  % ProcessBlue  Approximate PANTONE PROCESS-BLUE
\def\csb{\SkyBlue}	  % SkyBlue  Approximate PANTONE 2985
\def\ctu{\Turquoise}	  % Turquoise  Approximate PANTONE (312+313)/2
\def\ctb{\TealBlue}	  % TealBlue  Approximate PANTONE 3145
\def\caq{\Aquamarine}	  % Aquamarine  Approximate PANTONE 3135
\def\cbg{\BlueGreen}	  % BlueGreen  Approximate PANTONE 320
\def\cem{\Emerald}	  % Emerald  No PANTONE match
%\def\cjg{\JungleGreen}	  % JungleGreen  Approximate PANTONE 328
\def\csg{\SeaGreen}	  % SeaGreen  Approximate PANTONE 3268
\def\cgg{\Green}	  % Green  VERY-Approx PANTONE GREEN
\def\cfg{\ForestGreen}	  % ForestGreen  Approximate PANTONE 349
\def\cpg{\PineGreen}	  % PineGreen  Approximate PANTONE 323
\def\clg{\LimeGreen}	  % LimeGreen  No PANTONE match
\def\cyg{\YellowGreen}	  % YellowGreen  Approximate PANTONE 375
\def\cspg{\SpringGreen}	  % SpringGreen  Approximate PANTONE 381
\def\cog{\OliveGreen}	  % OliveGreen  Approximate PANTONE 582
\def\pars{\RawSienna}	  % RawSienna  Approximate PANTONE 154
\def\cse{\Sepia}		  % Sepia  Approximate PANTONE 161
\def\cbr{\Brown}		  % Brown  Approximate PANTONE 1615
\def\cta{\Tan}		  % Tan  No PANTONE match
\def\cgr{\Gray}		  % Gray  Approximate PANTONE COOL-GRAY-8
\def\cbl{\Black}		  % Black  Approximate PANTONE PROCESS-BLACK
\def\cwh{\White}		  % White  No PANTONE match


\loadmsbm

\input epsf

\def\ctln{\centerline}
\def\u{\underbar}
\def\ssk{\smallskip}
\def\msk{\medskip}
\def\bsk{\bigskip}
\def\hsk{\hskip.1in}
\def\hhsk{\hskip.2in}
\def\dsl{\displaystyle}
\def\hskp{\hskip1.5in}

\def\lra{$\Leftrightarrow$ }
\def\ra{\rightarrow}
\def\mpto{\logmapsto}
\def\pu{\pi_1}
\def\mpu{$\pi_1$}
\def\sig{\Sigma}
\def\msig{$\Sigma$}
\def\ep{\epsilon}
\def\sset{\subseteq}
\def\del{\partial}
\def\inv{^{-1}}
\def\wtl{\widetilde}
\def\lra{\Leftrightarrow}
\def\del{\partial}
\def\delp{\partial^\prime}
\def\delpp{\partial^{\prime\prime}}
\def\sgn{{\roman{sgn}}}



\ctln{\bf Math 971 Algebraic Topology}

\ssk

\ctln{March 31, 2005}

\msk

{\bf Relative homology:} we build the singular chain complex
of a pair $(X,A)$ , i.e., of a space $X$ and a subspace $A\subseteq X$ .
Since as abelian groups we can think of 
$C_n(A)$ as a subgroup of $C_n(X)$ (under the injective homomorphism induced by the 
inclusion $i:A\ra X$) we can set $C_n(X,A)= C_n(X)/C_n(A)$ . Since the
boundary map $\del_n:C_n(X)\ra C_{n-1}(X)$ satisfies
$\del_n(C_n(A)\subseteq C_{n-1}(A)$ (the boundary of a map into $A$ maps into $A$),
we get an induced boundary map $\del_n:C_n(X,A)\ra C_{n-1}(X,A)$ . These
groups and maps $(C_n(X,A),\del_n)$ form a chain complex, whose homology groups 
are the {\it singluar relative homology groups of the pair} $(X,A)$ . To be a cycle
in relative homology, you need to have a representative $z$ with $\del z\in C_{n-1}(A)$,
i.e., you are a chain with boundary in $A$. To be a boundary, you need
$z=\del w +a$ for some $w\in C_{n+1}(X)$ and $a\in C_n(A)$ , i.e., you {\it cobound}
a chain in $A$ ($\del w = z-a$). Note that the relative homology of the pair $(X,\emptyset)$
is just the ordinary homology of $X$; we aren't modding out by anything.

\ssk

%There is a reduced relative homology 
%as well, since we can augment with the same map (1-simplices always have 2 ends!),
%but in this case it has (essentially) no effect; $\widetilde{H}_i(X,A)\cong H_i(X,A)$
%for all $i$ \u{unless} $A=\emptyset$, in which case we lose the ${\Bbb Z}$ in
%dimension 0 that we expect to. 

%\ssk

The inclusion $i_n$ and projection  $p_n$ maps give us short exact sequences 

\ssk

\ctln{$0\ra C_n(A)\ra C_n(X) \ra C_n(X,A)\ra 0$} 

and since the boundary on chains
in $X$ restricts to the boundary on $A$ and induces the boundary on $(X,A)$,
$i_n$ and $p_n$ are chain maps. So we get a long exact homology sequence

\ssk

\ctln{$\cdots \ra H_n(A) \ra H_n(X) \ra H_n(X,A) \ra H_{n-1}(A) \ra H_{n-1}(X) \ra \cdots$}

\ssk

We can also replace  the absolute homology groups in this
sequence with reduced homology groups, by augmenting the short exact seuences with
\hhsk $0\ra {\Bbb Z} \ra {\Bbb Z} \ra 0 \ra 0$ \hhsk at the bottom.
There is also a long exact sequence of a triple $(X,A,B)$ , where by triple we
mean $B\sset A\sset X$ . From the short exact sequences 
\hhsk $0\ra C_n(A,B) \ra C_n(X,B) \ra C_n(X,A)\ra 0$ \hhsk (i.e., 

\ctln{$0\ra C_n(A)/C_n(B) \ra C_n(X)/C_n(B) \ra C_n(X)/C_n(A)\ra 0$)} 

we get the
long exact sequence 

\ssk

\ctln{$\cdots \ra H_n(A,B) \ra H_n(X,B) \ra H_n(X,A) \ra H_{n-1}(A,B) 
\ra H_{n-1}(X,B) \ra \cdots$}

\msk

So for example if we look at the pair $({\Bbb D}^n,\del {\Bbb D}^n) = ({\Bbb D}^n,S^{n-1})$,
since the reduced homology of ${\Bbb D}^n$ is trivial in every dimension, every third group in
our LES is $0$, giving $H_m({\Bbb D}^n,S^{n-1})\cong  \widetilde{H}_{m-1}(S^{n-1})$ for every $m$ and $n$.

\ssk

A basic fact is that if $A$ sits in $X$ ``nicely enough'' (think: $A$ is a subcomplex of the cxell complex $X$),
then $H_n(X,A) \cong \widetilde{H}_n(X/A)$ . We will shortly prove this! One nice consequence is
that we can do some (non-trivial!) basic calculations: taking $X={\Bbb D}^n$ and $A=\del {\Bbb D}^n = S^{n-1}$,
we have ${\Bbb D}^n/S^{n-1} \cong S^n$ ,  and the previous two facts combine to give
$ \widetilde{H}_{m}(S^{n}) \cong  \widetilde{H}_{m-1}(S^{n-1})$ for every $m$ and $n$ . By induction (since we
know that values of $ \widetilde{H}_{m-n}(S^0)$, we find that $ \widetilde{H}_n(S^n)\cong {\Bbb Z}$ and all 
other homology groups are $0$.

\msk

And this, in turn, let's us prove a fairly sizable theorem:

\ssk

{\bf Brouwer Fixed Point Theorem}: For every $n$, every
map $f:{\Bbb D}^n\ra {\Bbb D}^n$ has a fixed point.

\ssk

{\bf Proof:} If $f(x)\neq x$ for every $x$, then is with the $n=2$ case
that you may have seen before, we can construct a retraction
$r:{\Bbb D}^n \ra \del {\Bbb D}^n = S^{n-1}$ by setting
$r(x)$ = the (first) point past $f(x)$ where the ray from $f(x)$ to $x$ meets
$\del {\Bbb D}^n$ . This function is continuous, and is the 
identity on the boundary. So from our of your problem sets, the 
inclusion-induced map $i_*: H_{n-1}(S^n) \ra H_{n-1}({\Bbb D}^n)$ 
is injective. But this is impossible, since the first group is ${\Bbb Z}$ and the
second is $0$ .

\bsk

Another source of short exact sequences is {\it homology with coefficients}.
In ordinary (singular) homology, our chains are formal linear combinations of 
singular simplices, with coefficients in ${\Bbb Z}$. But all we needed to know 
about ${\Bbb Z}$ was that we could add things, and that integers have negatives
(and how to recognize $0$). \u{So}, any abelian group $G$ should work. If we
define singular chains with coefficients in $G$ to be formal linear combinations
$\sum g_i\sigma_i^n$, then since the boundary map is computed simplex by simplex,
we can define $\del(g\sigma) = \sum (-1)^ig \sigma|_{\Delta^{n-1}_i)}$ , essentially as before,
and get a new chain complex $C_*(X;G)$ . It's homology groups (cycles/boundaries)
is the {\it homology of $X$ with coefficients in $G$}, $H_*(X;G)$ . We can also define
relative homology groups $H_*(X,A;G)$ in exactly the same way as before.

\msk

From this perspective, our original homology groups $H_n(X)$ should be called
$H_n(X;{\Bbb Z})$. 
And the point, in the context of our present discussion, is that a short exact sequence of 
coefficient groups, \hhsk $0\ra K\ra G \ra H \ra 0$ \hhsk induces a short exact 
sequence of chain groups \hhsk
$0\ra C_n(X;K) \ra C_n(X;G) \ra C_n(X;H) \ra 0$ \hhsk , giving us a long exact homology
sequence

\ssk

\ctln{$\cdots \ra H_{n+1}(X;H) \ra H_n(X;K) \ra H_n(X;G) \ra H_n(X;H) \ra H_{n-1}(X,K) \ra \cdots$}

\ssk

So for example, the short exact sequence \hhsk $0\ra {\Bbb Z} \ra {\Bbb Z} \ra {\Bbb Z}_n \ra 0$\hhsk ,
where the first map is multiplication by $n$, and the second is reduction mod $n$, is exact, and gives
us a long exact sequence involving ordinary homology and homology mod $n$ . Everything we have done 
with homology so far goes through with coefficients, essentially with the identical proof; foi example,
homotopy equivalent spaces have isomorphic homology with coefficients, and homotopic maps
induce the same maps on homology.




\vfill
\end

