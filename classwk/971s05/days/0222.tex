
\magnification=1200
\overfullrule=0pt
\parindent=0pt

\nopagenumbers

\input amstex

\voffset=-.6in
\hoffset=-.5in
\hsize = 7.5 true in
\vsize=10.4 true in

%\voffset=1.4in
%\hoffset=-.5in
%\hsize = 10.2 true in
%\vsize=8 true in

\input colordvi

\loadmsbm

\input epsf

\def\ctln{\centerline}
\def\u{\underbar}
\def\ssk{\smallskip}
\def\msk{\medskip}
\def\bsk{\bigskip}
\def\hsk{\hskip.1in}
\def\hhsk{\hskip.2in}
\def\dsl{\displaystyle}
\def\hskp{\hskip1.5in}

\def\lra{$\Leftrightarrow$ }
\def\ra{\rightarrow}
\def\mpto{\logmapsto}
\def\pu{\pi_1}
\def\mpu{$\pi_1$}
\def\sig{\Sigma}
\def\msig{$\Sigma$}
\def\ep{\epsilon}
\def\sset{\subseteq}
\def\del{\partial}
\def\inv{^{-1}}
\def\wtl{\widetilde}
\def\lra{\Leftrightarrow}



\ctln{\bf Math 971 Algebraic Topology}

\ssk

\ctln{February 22, 2005}

\msk

{\bf Building universal coverings:} If a space $X$ is path connected, locally path connected, and
semi-locally simply connected (S-LSC), then it has a universal covering;
we describe a general construction. The idea is that a covering
space should have the path lifting and homotopy
lifting properties, and the universal 
cover can be characterized as the only covering space for 
which {\it only} null-homotopic loops lift to loops. So we build a 
space and a map which \underbar{must} have these properties.
We do this by making a space $\widetilde{X}$ whose
points are (equivalence classes of) $[\gamma]$
based paths $\gamma:(I,0)\ra (X,x_0)$, where two paths are equivalent
if they are homotopic rel endpoints! The projection map is
$p([\gamma])=\gamma(1)$. The S-LSCness is what guarantees that this is a 
covering map; choosing $x\in X$, a path $\gamma_0$ from $x_0$ to $x$,
and a  neighborhood ${\Cal U}$ of $x$ guaranteed by S-LSC, paths from 
$x_0$ to points in ${\Cal U}$ are based equivalent to $\gamma*\gamma_0*\eta$
where $\gamma$ is a based loop at $x_0$ and $\eta$ is a path in ${\Cal U}$.
But by simple connectivity, a path in ${\Cal U}$ is determined up to homotopy
by its endpoints, and so, with $\gamma$ fixed, these paths are in one-to-one
correspondence with ${\Cal U}$. So $p\inv({\Cal U}$ is a disjoint union,
indexed by $\pu(X,x_0)$, of sets in bijective correspondence with ${\Cal U}$.
The appropriate topology on $\widetilde{X}$, essentially given as a basis
by triples $\gamma*,gamma_0,{\Cal U}$ as above, make $p$ a covering map.
Note that the inverse image of 
the basepoint $x_0$ is the equivalence classes of \underbar{loops} at $x_0$,
i.e., $\pu(X,x_0)$. A path $\gamma$ lifts to the path of paths
$[\gamma_t]$, where $\gamma_t(s)=\gamma(ts)$, and so the only 
loop in $X$ which lifts to a loop in $\widetilde{X}$ has endpoint
$[\gamma]=[c_{x_0}]$, i.e., $[\gamma]=1$ in $\pu(X,x_0)$. This
implies that $p_*(\pu(\widetilde{X},[c_{x_0}]))=\{1\}$, so 
$\pu(\widetilde{X},[c_{x_0}])=\{1\}$ . \hhsk However, nobody in their
right minds would go about building $\widetilde{X}$ in this way, in general!
Before describing how to do it ``right'', though, we should perhaps see why
we should want to?

\msk

One reason for the importance of the universal cover is that it gives
us a unified approach to building \underbar{all} connected covering
spaces of $X$. The basis for this is the {\it deck transformation group}
of a covering space $p:\wtl{X}\ra X$; this is the set of all
homeomorphisms $h:\wtl{X}\ra\wtl{X}$ such that $p\circ h = p$.
These homeomorphisms, by definition, permute each of the point inverses
of $p$. In fact, since $h$ can be thought of as a lift of the projection
$p$, by the lifting criterion $h$ is determined by which point in the 
inverse image of the basepoint $x_0$ it takes the basepoint 
$\wtl{x}_0$ of $\wtl{X}$ to. A deck transformation sending
$\wtl{x}_0$ to $\wtl{x}_1$ exists $\Leftrightarrow$
$p_*(\pu(\wtl{X},\wtl{x}_0)=p_*(\pu(\wtl{X},\wtl{x}_1)$
[we need one inclusion to give the map $h$, and the opposite inclusion
to ensure it is a bijection (because its inverse exists)]. These two groups
are in general {\it conjugate}, by the projection of a path from 
$\wtl{x}_0$ to $\wtl{x}_1$; this can be seen by following the change
of basepoint isomorphism down into $G=\pu(X,x_0)$. As we have seen, paths
in $\wtl{X}$ from $\wtl{x}_0$ to $\wtl{x}_1$ are in 1-to-1
correspondence with the cosets of $H=p_*(\pu(\wtl{X},\wtl{x}_0)$ in 
$p_*(\pu({X},{x}_0)$; so deck transformations are in 1-to-1 
correspondence with cosets whose representatives conjugate 
$H$ to itself. The set of such elements in $G$ is called the 
{\it normalizer of $H$ in $G$}, and denoted $N_G(H)$ or simply
$N(H)$. The deck transformation group is therefore
in 1-to-1 correspondence with the group $N(H)/H$ under
$h\mapsto$ the coset represented by the projection of the path from 
$\wtl{x}_0$ to $h(\wtl{x}_0)$. And since $h$ is essentially built
by lifting paths, it follows quickly that this map is a
homomorphism, hence an isomorphism.

\msk

In particular, applying this to the universal covering space
$p:\wtl{X}\ra X$, since in this case $H=\{1\}$, so $N(H)=\pu(X,x_0)$,
its deck transformation group is isomorphic to $\pu(X,x_0)$. 
For example, this gives the quickest possible proof 
that $\pu(S^1)\cong {\Bbb Z}$, since ${\Bbb R}$ is a 
contractible covering space, whose deck transformations
are the translations by integer distances. 
Thus $\pu(X)$ acts on its universal cover as a group of
homeomorphisms. And since this action is {\it simply transitive}
on point inverses [there is exactly one (that's the simple
part) deck transformation carrying any one point in a point 
inverse to any other one (that's the transitive part)], the 
quotient map from $\wtl{X}$ to the orbits of this action \underbar{is}
the projection map $p$. The evenly covered property of $p$ implies
that $X$ does have the quotient topology under this action.

\msk

So every space it $X$ the quotient of its universal cover (if it has
one!) by its fundamental group $G=\pu(X,x_0)$, realized as the group
of deck transformations. And the quotient map is the covering 
projection. 

\vfill
\end
