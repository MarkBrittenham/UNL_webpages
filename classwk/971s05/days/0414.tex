

\magnification=1200
\overfullrule=0pt
\parindent=0pt

\nopagenumbers

\input amstex

\voffset=-.6in
\hoffset=-.5in
\hsize = 7.5 true in
\vsize=10.4 true in

%\voffset=1.4in
%\hoffset=-.5in
%\hsize = 10.2 true in
%\vsize=8 true in

\input colordvi

\def\cltr{\Red}		  % Red  VERY-Approx PANTONE RED
\def\cltb{\Blue}		  % Blue  Approximate PANTONE BLUE-072
\def\cltg{\PineGreen}	  % ForestGreen  Approximate PANTONE 349
\def\cltp{\DarkOrchid}	  % DarkOrchid  No PANTONE match
\def\clto{\Orange}	  % Orange  Approximate PANTONE ORANGE-021
\def\cltpk{\CarnationPink}	  % CarnationPink  Approximate PANTONE 218
\def\clts{\Salmon}	  % Salmon  Approximate PANTONE 183
\def\cltbb{\TealBlue}	  % TealBlue  Approximate PANTONE 3145
\def\cltrp{\RoyalPurple}	  % RoyalPurple  Approximate PANTONE 267
\def\cltp{\Purple}	  % Purple  Approximate PANTONE PURPLE

\def\cgy{\GreenYellow}     % GreenYellow  Approximate PANTONE 388
\def\cyy{\Yellow}	  % Yellow  Approximate PANTONE YELLOW
\def\cgo{\Goldenrod}	  % Goldenrod  Approximate PANTONE 109
\def\cda{\Dandelion}	  % Dandelion  Approximate PANTONE 123
\def\capr{\Apricot}	  % Apricot  Approximate PANTONE 1565
\def\cpe{\Peach}		  % Peach  Approximate PANTONE 164
\def\cme{\Melon}		  % Melon  Approximate PANTONE 177
\def\cyo{\YellowOrange}	  % YellowOrange  Approximate PANTONE 130
\def\coo{\Orange}	  % Orange  Approximate PANTONE ORANGE-021
\def\cbo{\BurntOrange}	  % BurntOrange  Approximate PANTONE 388
\def\cbs{\Bittersweet}	  % Bittersweet  Approximate PANTONE 167
%\def\creo{\RedOrange}	  % RedOrange  Approximate PANTONE 179
\def\cma{\Mahogany}	  % Mahogany  Approximate PANTONE 484
\def\cmr{\Maroon}	  % Maroon  Approximate PANTONE 201
\def\cbr{\BrickRed}	  % BrickRed  Approximate PANTONE 1805
\def\crr{\Red}		  % Red  VERY-Approx PANTONE RED
\def\cor{\OrangeRed}	  % OrangeRed  No PANTONE match
\def\paru{\RubineRed}	  % RubineRed  Approximate PANTONE RUBINE-RED
\def\cwi{\WildStrawberry}  % WildStrawberry  Approximate PANTONE 206
\def\csa{\Salmon}	  % Salmon  Approximate PANTONE 183
\def\ccp{\CarnationPink}	  % CarnationPink  Approximate PANTONE 218
\def\cmag{\Magenta}	  % Magenta  Approximate PANTONE PROCESS-MAGENTA
\def\cvr{\VioletRed}	  % VioletRed  Approximate PANTONE 219
\def\parh{\Rhodamine}	  % Rhodamine  Approximate PANTONE RHODAMINE-RED
\def\cmu{\Mulberry}	  % Mulberry  Approximate PANTONE 241
\def\parv{\RedViolet}	  % RedViolet  Approximate PANTONE 234
\def\cfu{\Fuchsia}	  % Fuchsia  Approximate PANTONE 248
\def\cla{\Lavender}	  % Lavender  Approximate PANTONE 223
\def\cth{\Thistle}	  % Thistle  Approximate PANTONE 245
\def\corc{\Orchid}	  % Orchid  Approximate PANTONE 252
\def\cdo{\DarkOrchid}	  % DarkOrchid  No PANTONE match
\def\cpu{\Purple}	  % Purple  Approximate PANTONE PURPLE
\def\cpl{\Plum}		  % Plum  VERY-Approx PANTONE 518
\def\cvi{\Violet}	  % Violet  Approximate PANTONE VIOLET
\def\clrp{\RoyalPurple}	  % RoyalPurple  Approximate PANTONE 267
\def\cbv{\BlueViolet}	  % BlueViolet  Approximate PANTONE 2755
\def\cpe{\Periwinkle}	  % Periwinkle  Approximate PANTONE 2715
\def\ccb{\CadetBlue}	  % CadetBlue  Approximate PANTONE (534+535)/2
\def\cco{\CornflowerBlue}  % CornflowerBlue  Approximate PANTONE 292
\def\cmb{\MidnightBlue}	  % MidnightBlue  Approximate PANTONE 302
\def\cnb{\NavyBlue}	  % NavyBlue  Approximate PANTONE 293
\def\crb{\RoyalBlue}	  % RoyalBlue  No PANTONE match
%\def\cbb{\Blue}		  % Blue  Approximate PANTONE BLUE-072
\def\cce{\Cerulean}	  % Cerulean  Approximate PANTONE 3005
\def\ccy{\Cyan}		  % Cyan  Approximate PANTONE PROCESS-CYAN
\def\cpb{\ProcessBlue}	  % ProcessBlue  Approximate PANTONE PROCESS-BLUE
\def\csb{\SkyBlue}	  % SkyBlue  Approximate PANTONE 2985
\def\ctu{\Turquoise}	  % Turquoise  Approximate PANTONE (312+313)/2
\def\ctb{\TealBlue}	  % TealBlue  Approximate PANTONE 3145
\def\caq{\Aquamarine}	  % Aquamarine  Approximate PANTONE 3135
\def\cbg{\BlueGreen}	  % BlueGreen  Approximate PANTONE 320
\def\cem{\Emerald}	  % Emerald  No PANTONE match
%\def\cjg{\JungleGreen}	  % JungleGreen  Approximate PANTONE 328
\def\csg{\SeaGreen}	  % SeaGreen  Approximate PANTONE 3268
\def\cgg{\Green}	  % Green  VERY-Approx PANTONE GREEN
\def\cfg{\ForestGreen}	  % ForestGreen  Approximate PANTONE 349
\def\cpg{\PineGreen}	  % PineGreen  Approximate PANTONE 323
\def\clg{\LimeGreen}	  % LimeGreen  No PANTONE match
\def\cyg{\YellowGreen}	  % YellowGreen  Approximate PANTONE 375
\def\cspg{\SpringGreen}	  % SpringGreen  Approximate PANTONE 381
\def\cog{\OliveGreen}	  % OliveGreen  Approximate PANTONE 582
\def\pars{\RawSienna}	  % RawSienna  Approximate PANTONE 154
\def\cse{\Sepia}		  % Sepia  Approximate PANTONE 161
\def\cbr{\Brown}		  % Brown  Approximate PANTONE 1615
\def\cta{\Tan}		  % Tan  No PANTONE match
\def\cgr{\Gray}		  % Gray  Approximate PANTONE COOL-GRAY-8
\def\cbl{\Black}		  % Black  Approximate PANTONE PROCESS-BLACK
\def\cwh{\White}		  % White  No PANTONE match


\loadmsbm

\input epsf

\def\ctln{\centerline}
\def\u{\underbar}
\def\ssk{\smallskip}
\def\msk{\medskip}
\def\bsk{\bigskip}
\def\hsk{\hskip.1in}
\def\hhsk{\hskip.2in}
\def\dsl{\displaystyle}
\def\hskp{\hskip1.5in}

\def\lra{$\Leftrightarrow$ }
\def\ra{\rightarrow}
\def\mpto{\logmapsto}
\def\pu{\pi_1}
\def\mpu{$\pi_1$}
\def\sig{\Sigma}
\def\msig{$\Sigma$}
\def\ep{\epsilon}
\def\sset{\subseteq}
\def\del{\partial}
\def\inv{^{-1}}
\def\wtl{\widetilde}
\def\lra{\Leftrightarrow}
\def\del{\partial}
\def\delp{\partial^\prime}
\def\delpp{\partial^{\prime\prime}}
\def\sgn{{\roman{sgn}}}
\def\wtih{\widetilde{H}}
\def\bbz{{\Bbb Z}}
\def\bbr{{\Bbb R}}



\ctln{\bf Math 971 Algebraic Topology}

\ssk

\ctln{April 14, 2005}

\msk


We have so far introduced two homologies; simplicial, $H_*^\Delta$, whose computation 
``only'' required some linear algebra,
and singular, $H_*$, which is formally less difficult to work with, and which, you may suspect by now, is also becoming
less difficult to compute... For $\Delta$-complexes, these homology groups are the same, $H_n^\Delta(X)\cong H_n(X)$
for every $X$. In fact, the isomorphism is induced by the inclusion $C_n^\Delta(X)\sset C_n(X)$. And we have
now assembled all of the tools necessary to prove this. Or almost; we need to note that most of the edifice we
have built for singular homology \u{could} have been built for simplicial homology, including relative 
homology (for a sub-$\Delta$-complex $A$ of $X$), and a SES of chain groups, giving a LES sequence for the pair,

\ssk

$\cdots \ra H_n^\Delta(A) \ra H_n^\Delta(X) \ra H_n^\Delta(X,A) \ra H_{n-1}^\Delta(A) \ra \cdots$

\ssk

The proof of the isomorphism between the two homologies proceeds by first showing that the
inclusion induces an isomorphism on $k$-skeleta, $H_n^\Delta(X^{(k)})\cong H_n(X^{(k)})$,
and this goes by induction on $k$ using the Five Lemma applied to the diagram

\ssk

\ctln{$\displaystyle 
\matrix 
H_{n+1}^\Delta(X^{(k)},X^{(k-1)})&\ra&H_n^\Delta(X^{(k-1)})&\ra&H_n^\Delta(X^{(k)}) & \ra & H_{n}^\Delta(X^{(k)},X^{(k-1)}) & \ra & H_{n-1}^\Delta(X^{(k-1)})\cr
\downarrow & & \downarrow & & \downarrow & & \downarrow & & \downarrow & \cr
H_{n+1}(X^{(k)},X^{(k-1)})&\ra&H_n(X^{(k-1)})&\ra&H_n(X^{(k)}) & \ra & H_{n}(X^{(k)},X^{(k-1)}) & \ra & H_{n-1}(X^{(k-1)}) \cr
\endmatrix$}

\ssk

The second and fifth vertical arrows are, by an inductive hypothesis, isomorphisms. The first and fourth vertical arrows are
isomorphisms because, essentially, we can, in each case, identify these groups. 
$H_{n}(X^{(k)},X^{(k-1)})\cong H_{n}(X^{(k)}/X^{(k-1)})\cong \widetilde{H}_n(\vee S^k)$
are either 0 (for $n\neq k$) or $\oplus \bbz$ (for $n=k$), one summand for each $n$-simplex in $X$. 
But the same is true for $H_{n}^\Delta(X^{(k)},X^{(k-1)})$; and for $n=k$ the generators are precisely
the $n$-simplices of $X$. The inclusion-induced map takes generators to generators, so is an isomorphism.
\hhsk So by the Five Lemma, the middle rows are also isomorphisms, completing our inductive proof.

\ssk

Returning to $H_n^\Delta(X) {\buildrel {I_*}\over \ra} H_n(X)$, we wish now to show that this map is an isomorphism.
Any $[z]\in H_n(X)$ is represented by a cycle $z=\sum a_i\sigma_i$ for $\sigma_i:\Delta^n\ra X$ . But each
$\sigma_i(\Delta^n)$ is a compact subset of $X$, and so meets only finitely-many cells of $X$. This is true for every
singular simplex, and so there is a $k$ for which all of the simplices map into $X^{(k)}$, and so we may
treat $z\in C_n(X^{(k)}$. Thought of in this way, it is still a cycle, and so $[z]\in H_n(X^{(k)})\cong H_n^\Delta(X^{(k)})$
so there is a $z^\prime in C_n^\Delta(X^{(k)})$ and a $w\in C_{n+1}(X^{(k)})$ with $i_\#z^\prime -z=\del w$. 
But thinking of  $z^\prime in C_n^\Delta(X)$ and $w\in C_{n+1}(X)$, we have the same equality, so 
$[z^\prime] \in H_n^\Delta(X)$ and $i_*[z^\prime] = [z]$ . So $i_*$ is surjective.
If $i_*([z]) = 0$, then the cycle $z=\sum a_i\sigma_i$ is a sum of characteristic maps of $n$-simplices of $X$, and
so can be thought of as an element of $C_n^\Delta(X^{n)})$ . Being $0$ in $H_n(X)$, $z=\del w$ for some
$w\in C_{n+1}(X)$ . But as before, $w\in C_n(X^{r)})$ for some $r$, and so thought of as an element of 
the image of the isomorphism $i_*: H_n^\Delta(X^{(r)})\ra H_n(X^{(r)})$, $i_*([z])=0$, so $[z]=0$ . So 
$z=\del u$ for some $u\in C_{n+1}^\Delta(X^{r)})\sset C_{n+1}^\Delta(X)$ . So $[z]=0$ in $H_n^\Delta(X)$.
Consequently, simplicial and singular homology groups are isomorphic.

\msk

One consequence of this fact is that we can prove the topological invariance of the {\bf Euler
characteristic} of a space $X$. If $X$ is a $\Delta$-complex made up of a finite number of 
simplices, then we can count the number $m_i$ of $i$-simplices in the $\Delta$-complex 
structure of $X$. The Euler characteristic of $X$ is then defined to be the alternating sum
$\chi(X) = \displaystyle \sum_{i=0}^\infty (-1)^i m_i$ . Now, as a topological space, $X$ can be given 
many different $\Delta$-complex structures, and $\chi(X)$ is a priori a number which \u{depends} 
on the structure, not just on $X$. But once we note that $m_i$ = the rank of the (simplicial) 
chain group $C_i^\Delta(X)$ (there is one generator for each $i$-simplex), we find that 
$\chi(X) = \sum_{i=0}^N (-1)^i$ rank$(C_i(X))$, and then the following result from homological algebra
establishes the topological invariance of this number:

\ssk

{\bf Proposition:} If $\cdots 0\ra C_n\ra \cdots \ra C_1\ra C_0 \ra 0$ is a chain complex, with every 
chain group having finite rank, then 

\ssk

$\sum_{i=0}^n (-1)^i$ rank$(C_i)$ = $\sum_{i=0}^n (-1)^i$ rank$(H_i({\Cal C})$ .

\ssk

The proof follows from cleverly applying the fact that since $H_i({\Cal C}) = $ker$\partial_i/$im$\partial_{i+1}$,
$z_i =$ rank(ker$\partial_i$) = rank$(H_i({\Cal C}))$ $+$ rank(im$\partial_{i+1})$ = $h_i+b_{i+1}$, so
$h_i=z_i-b_{i+1}$,
together with the fact that since (by Noether) im$(\partial_i) \cong C_i/$ker$(\partial_i)$, so 
$c_i$ = rank$(C_i)$ = $z_i+b_i$. We therefore have 

\ssk

$\sum_{i=0}^n (-1)^i$ rank$(H_i({\Cal C}) = \sum_{i=0}^n (-1)^i h_i = \sum_{i=0}^n (-1)^i (z_i-b_{i+1})
 =\sum_{i=0}^n (-1)^i z_i - \sum_{i=0}^n (-1)^i b_{i+1} = \sum_{i=0}^n (-1)^i z_i + \sum_{i=0}^n (-1)^i b_i
=\sum_{i=0}^n (-1)^i (z_i + b_i) =\sum_{i=0}^n (-1)^i$ rank$(C_i)$ 
\hhsk
as desired. 

Consequently, $\chi(X) = \sum_{i=0}^N (-1)^i$ rank$(C_i^\Delta(X)) =
\sum_{i=0}^N (-1)^i$ rank$(H_i^\Delta(X)) = \sum_{i=0}^N (-1)^i$ rank$(H_i(X))$, which is 
an invariant of $X$, since the singular homology groups are!

\msk

The fact that this number has two different interpretations leads to some non-trivial results.
First, it tells us that the Euler charactistic calculation is independent of how we express a space $X$ as
a $\Delta$-complex. $\chi$ is also actually invariant under homotopy equivalence,
since the homology groups are; so homotopy equivalent spaces have the
same Euler $chi$. Consequently, all contractible spaces, for example, must have
Euler characteristic = 1.

\ssk

Next, by the lifting criterion, if $p:\widetilde{X}\ra X$ is a $k$-fold covering space
of a $\Delta$-complex $X$,  then $\widetilde{X}$ can be given a $\Delta$-complex structure
with $k$ times as many $i$-simplices as $X$, for every $i$ (lift the characteristic maps
of the simplices of $X$....). So $\chi(\widetilde{X}) = k\cdot\chi(X)$ . This give a necessary
condition for one space to be a covering of another; it's Euler $\chi$ must be a multiple of the 
other. For example, from our homology calculations, it follows that for a closed orientable surface $F_g$ of 
genus $g$, $\chi(F_g)=2-2g$. So a $k$-fold covering of $F_g$ will have Euler $\chi$ equal to
$k(2-2g) = 2k-2kg = 2-2(kg-k+1)$ , and so is a surface of genus $kg-k+1$ . [The converse, that
a surface with this genus $k$-fold covers $F_g$, can be established by
building the coverings directly.] Consequently, $F_5$ is a 2-fold covering of $F_3$, 
so there is a subgroup of index 2 of $\pu(F_3)$ isomorphic to $\pu(F_5)$,  but $F_6$ is not a finite-sheeted
cover of $F_3$, because $-4\not | -10$ . [It is also not an inifinite-sheeted covering, because their
total spaces are non-compact...] Consequently, $\pu(F_6)$ is not isomorphic to a subgroup of $\pu(F_3)$ .



\vfill
\end
