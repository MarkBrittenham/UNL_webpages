

\magnification=1200
\overfullrule=0pt
\parindent=0pt

\nopagenumbers

\input amstex

\voffset=-.6in
\hoffset=-.5in
\hsize = 7.5 true in
\vsize=10.4 true in

%\voffset=1.4in
%\hoffset=-.5in
%\hsize = 10.2 true in
%\vsize=8 true in

\input colordvi

\def\cltr{\Red}		  % Red  VERY-Approx PANTONE RED
\def\cltb{\Blue}		  % Blue  Approximate PANTONE BLUE-072
\def\cltg{\PineGreen}	  % ForestGreen  Approximate PANTONE 349
\def\cltp{\DarkOrchid}	  % DarkOrchid  No PANTONE match
\def\clto{\Orange}	  % Orange  Approximate PANTONE ORANGE-021
\def\cltpk{\CarnationPink}	  % CarnationPink  Approximate PANTONE 218
\def\clts{\Salmon}	  % Salmon  Approximate PANTONE 183
\def\cltbb{\TealBlue}	  % TealBlue  Approximate PANTONE 3145
\def\cltrp{\RoyalPurple}	  % RoyalPurple  Approximate PANTONE 267
\def\cltp{\Purple}	  % Purple  Approximate PANTONE PURPLE

\def\cgy{\GreenYellow}     % GreenYellow  Approximate PANTONE 388
\def\cyy{\Yellow}	  % Yellow  Approximate PANTONE YELLOW
\def\cgo{\Goldenrod}	  % Goldenrod  Approximate PANTONE 109
\def\cda{\Dandelion}	  % Dandelion  Approximate PANTONE 123
\def\capr{\Apricot}	  % Apricot  Approximate PANTONE 1565
\def\cpe{\Peach}		  % Peach  Approximate PANTONE 164
\def\cme{\Melon}		  % Melon  Approximate PANTONE 177
\def\cyo{\YellowOrange}	  % YellowOrange  Approximate PANTONE 130
\def\coo{\Orange}	  % Orange  Approximate PANTONE ORANGE-021
\def\cbo{\BurntOrange}	  % BurntOrange  Approximate PANTONE 388
\def\cbs{\Bittersweet}	  % Bittersweet  Approximate PANTONE 167
%\def\creo{\RedOrange}	  % RedOrange  Approximate PANTONE 179
\def\cma{\Mahogany}	  % Mahogany  Approximate PANTONE 484
\def\cmr{\Maroon}	  % Maroon  Approximate PANTONE 201
\def\cbr{\BrickRed}	  % BrickRed  Approximate PANTONE 1805
\def\crr{\Red}		  % Red  VERY-Approx PANTONE RED
\def\cor{\OrangeRed}	  % OrangeRed  No PANTONE match
\def\paru{\RubineRed}	  % RubineRed  Approximate PANTONE RUBINE-RED
\def\cwi{\WildStrawberry}  % WildStrawberry  Approximate PANTONE 206
\def\csa{\Salmon}	  % Salmon  Approximate PANTONE 183
\def\ccp{\CarnationPink}	  % CarnationPink  Approximate PANTONE 218
\def\cmag{\Magenta}	  % Magenta  Approximate PANTONE PROCESS-MAGENTA
\def\cvr{\VioletRed}	  % VioletRed  Approximate PANTONE 219
\def\parh{\Rhodamine}	  % Rhodamine  Approximate PANTONE RHODAMINE-RED
\def\cmu{\Mulberry}	  % Mulberry  Approximate PANTONE 241
\def\parv{\RedViolet}	  % RedViolet  Approximate PANTONE 234
\def\cfu{\Fuchsia}	  % Fuchsia  Approximate PANTONE 248
\def\cla{\Lavender}	  % Lavender  Approximate PANTONE 223
\def\cth{\Thistle}	  % Thistle  Approximate PANTONE 245
\def\corc{\Orchid}	  % Orchid  Approximate PANTONE 252
\def\cdo{\DarkOrchid}	  % DarkOrchid  No PANTONE match
\def\cpu{\Purple}	  % Purple  Approximate PANTONE PURPLE
\def\cpl{\Plum}		  % Plum  VERY-Approx PANTONE 518
\def\cvi{\Violet}	  % Violet  Approximate PANTONE VIOLET
\def\clrp{\RoyalPurple}	  % RoyalPurple  Approximate PANTONE 267
\def\cbv{\BlueViolet}	  % BlueViolet  Approximate PANTONE 2755
\def\cpe{\Periwinkle}	  % Periwinkle  Approximate PANTONE 2715
\def\ccb{\CadetBlue}	  % CadetBlue  Approximate PANTONE (534+535)/2
\def\cco{\CornflowerBlue}  % CornflowerBlue  Approximate PANTONE 292
\def\cmb{\MidnightBlue}	  % MidnightBlue  Approximate PANTONE 302
\def\cnb{\NavyBlue}	  % NavyBlue  Approximate PANTONE 293
\def\crb{\RoyalBlue}	  % RoyalBlue  No PANTONE match
%\def\cbb{\Blue}		  % Blue  Approximate PANTONE BLUE-072
\def\cce{\Cerulean}	  % Cerulean  Approximate PANTONE 3005
\def\ccy{\Cyan}		  % Cyan  Approximate PANTONE PROCESS-CYAN
\def\cpb{\ProcessBlue}	  % ProcessBlue  Approximate PANTONE PROCESS-BLUE
\def\csb{\SkyBlue}	  % SkyBlue  Approximate PANTONE 2985
\def\ctu{\Turquoise}	  % Turquoise  Approximate PANTONE (312+313)/2
\def\ctb{\TealBlue}	  % TealBlue  Approximate PANTONE 3145
\def\caq{\Aquamarine}	  % Aquamarine  Approximate PANTONE 3135
\def\cbg{\BlueGreen}	  % BlueGreen  Approximate PANTONE 320
\def\cem{\Emerald}	  % Emerald  No PANTONE match
%\def\cjg{\JungleGreen}	  % JungleGreen  Approximate PANTONE 328
\def\csg{\SeaGreen}	  % SeaGreen  Approximate PANTONE 3268
\def\cgg{\Green}	  % Green  VERY-Approx PANTONE GREEN
\def\cfg{\ForestGreen}	  % ForestGreen  Approximate PANTONE 349
\def\cpg{\PineGreen}	  % PineGreen  Approximate PANTONE 323
\def\clg{\LimeGreen}	  % LimeGreen  No PANTONE match
\def\cyg{\YellowGreen}	  % YellowGreen  Approximate PANTONE 375
\def\cspg{\SpringGreen}	  % SpringGreen  Approximate PANTONE 381
\def\cog{\OliveGreen}	  % OliveGreen  Approximate PANTONE 582
\def\pars{\RawSienna}	  % RawSienna  Approximate PANTONE 154
\def\cse{\Sepia}		  % Sepia  Approximate PANTONE 161
\def\cbr{\Brown}		  % Brown  Approximate PANTONE 1615
\def\cta{\Tan}		  % Tan  No PANTONE match
\def\cgr{\Gray}		  % Gray  Approximate PANTONE COOL-GRAY-8
\def\cbl{\Black}		  % Black  Approximate PANTONE PROCESS-BLACK
\def\cwh{\White}		  % White  No PANTONE match


\loadmsbm

\input epsf

\def\ctln{\centerline}
\def\u{\underbar}
\def\ssk{\smallskip}
\def\msk{\medskip}
\def\bsk{\bigskip}
\def\hsk{\hskip.1in}
\def\hhsk{\hskip.2in}
\def\dsl{\displaystyle}
\def\hskp{\hskip1.5in}

\def\lra{$\Leftrightarrow$ }
\def\ra{\rightarrow}
\def\mpto{\logmapsto}
\def\pu{\pi_1}
\def\mpu{$\pi_1$}
\def\sig{\Sigma}
\def\msig{$\Sigma$}
\def\ep{\epsilon}
\def\sset{\subseteq}
\def\del{\partial}
\def\inv{^{-1}}
\def\wtl{\widetilde}
\def\lra{\Leftrightarrow}



\ctln{\bf Math 971 Algebraic Topology}

\ssk

\ctln{March 3, 2005}

\msk


{\bf Postscript: why care about covering spaces?} The preceding discussion
probably makes it clear that covering places play a central role in
(combinatorial) group theory. It also plays a role in embedding 
problems; a common scenario is to have a map $f:Y\ra X$ which is 
injective on \mpu , and we wish to know if we can lift $f$ to a 
finite-sheeted covering so that the lifted map $\widetilde{f}$ is 
homotopic to an embedding. Information that is easier to obtain 
in the case of an embedding can then be passed down to gain information
abut the original map $f$. And covering spaces underlie the 
theory of analytic continuation in complex analysis; starting
with a domain $D\subseteq {\Bbb C}$, what analytic continuation really
builds is an (analytic) function from a covering space of $D$ to ${\Bbb C}$.
For example, the logarithm is really defined as a map from 
the universal cover of ${\Bbb C}\setminus\{0\}$ to ${\Bbb C}$. 
The various ``branches'' of the logarithm refer to which sheet
in this cover you are in.

\bsk

{\bf Homology theory:} Fundamental groups are a remarkably powerful
tool for studying spaces; they capture a great deal of the global
structure of a space, and so they are very good a distinguishing
between homotopy-inequivalent spaces. In theory! But in practice,
they suffer from the fact that deciding whether two groups are 
isomorphic or not is, in general, undecideable! Homology theory
is designed to get around this deficiency; the theory, by design,
builds (a sequence of) {\it abelian} groups $H_i(X)$ from a topological
space. And deciding whether or not two \u{abelian} groups, at least
if you're given a presentation for them, is, in the end, a matter of
fairly routine linear algebra. Mostly because of the Fundamental Theorem
of Finitely-generated Abelian groups; each such has a unique representation
as ${\Bbb Z}^m\oplus{\Bbb Z}_{m_1}\oplus\cdots\oplus{\Bbb Z}_{m_n}$
with $m_{i+1}|m_i$ for every $i$ .

\msk

There are also ``higher'' homotopy groups beyond the fundamental group \mpu ,
(hence the name pi-{\it one}); elements are homtopy classes, rel boundary, 
of based maps $(I^n,\del I^n)\ra(X,x_0)$. Multiplication is again by
concatenation. But unlike \mpu , where we have a chance to compute it
via Seifert-van Kampen, nobody, for example knows what all of the 
homotopy groups $\pi_n(S^2$ are (except that nearly all of them are
non-trivial!). Like \mpu, it describes, essentially, maps of $S^n$ into
$X$ which don't extend to maps of $D^{n+1}$, i.e., it turns the ``$n$-dimensional
holes'' of $X$ into a group.

\msk

Homology theory does the exact same thing, counting $n$-dimensional holes.
In the end we will find it to be extremely computable; but it will require
building a fair bit of machinery before it will become so transparent to
calculate. But the short version is that the homology groups compute
``cycles mod boundaries'', that is, $n$-dimesional objects/subsets that
have no boundary (in the appropriate sense) modulo objects that are the
boundary of $(n+1)$-dimensional ones. There are, in fact, probably as many
ways to {\it define} homology groups as there are people actively working
in the field; we will focus on two, simplicial homology and singular homology.
The first is quick to define and compute, but hard to show is an invariant!
The second is quick to see is an invariant, but, on the face of it, hard
to compute! Luckily, for spaces where they are both defined, they are
isomorphic. So, in the end, we get an invariant that is quick to compute.
Of course, so is the invariant ``4''; but this one will be a bit more
informative than that....

\msk

First, simplicial homology. This is a sequence of groups defined for spaces
for which they are easiest to define, which Hatcher calls $\Delta$-complexes.
Basically, they are spaces defined by gluing simplices together using
nice enough maps. More precisely, the {\it standard $n$-simplex} $\Delta^n$ is
the set of points 
$\{(x_1,\ldots x_{n+1})\in{\Bbb R}^{n+1}$ : $\sum x_i=1 , x_i\geq 0$
 for all $i\}$. This can also be expressed as convex linear combinations
(literally, that's the conditions on the $x_i$'s) of the points
$e_i=(0,\ldots ,0,1,0,\ldots ,0)$, the {\it vertices} of the standard
simplex. More generally, an $n$-simplex is the set $[v_0,\ldots v_n]$ of
convex linear combinations of points $v_0,\ldots ,v_n\in{\Bbb R}^{k}$
for which $v_1-v_0,\ldots ,v_n-v_0$ are linearly independent.
Any bijection from the vertices of the standard simplex to the points
$v_0,\ldots ,v_n$ extends (linearly) to a homeomorphism of
the simplices. The $n+1$ {\it faces} of a simplex, each sitting opposite
a vertex $v_i$, are obtained by setting the corresponding coefficient $x_i$ to $0$. 
Each forms an $(n-1)$-simplex, which we denote 
$[v_0,\ldots,v_{i-1},v_{i+1},\ldots ,v_n]$ or
$[v_0,\ldots,\widehat{v_{i}},\ldots ,v_n]$ . A {\it $\Delta$-complex} $X$ is a cell
complex obtained by gluing simplices together, but we insist on an extra
condition:
the restriction of the attaching map to any face is equal to a (lower-dimensional)
cell. As before, we use the weak topology on the space; a set is open iff
it's inverse image under the induced map of a cell into the complex is open.
Each $n$-cell comes equipped with a (continuous) map
$\sigma:\Delta^n\ra X$, which is one-to-one on its interior, whose restriction
to the boundary is the attaching map, and whose restriction to each face is the
associated map for that $(n-1)$-simplex. We will typically blur the 
distinction between the map $\sigma$ (called the {\it characteristic map}
of the simplex) and its image, and denote the
image by $\sigma$ (or $\sigma^n$), when this will cause no confusion,
and call $\sigma$ an $n$-simplex {\it in} $X$. When we feel the need for the 
distinction, we will use $e^n$ for the image and $\sigma^n$ for the map.

\ssk

For example, taking our standard,
identifications of the sides of a rectangle, cell structure for the 2-torus,
and cutting the rectangle into two triangles (= 2-simplices) along a diagonal,
we obtain a $\Delta$-structure with 2 2-simplices, 3 1-simplices, and 1 0-simplex.
A genus $g$ surface can be built, by cutting the $2g$-gon into triangles, with
$g+1$ 2-simplices, $3g$ 1-simplices, and 1 0-simplex.

\msk

We typically think of building a $\Delta$-complex $X$ inductively. 
The {\it 0-simplices}
(i.e., points), or {\it vertices}, form the 0-skeleton 
$X^{(0)}$. $n$-simplices $\sigma^n = [v_0,\ldots v_n]$ attach 
to the $(n-1)$-skeleton
to form the $n$-skeleton $X^{(n)}$; the restriction
of the attaching map to each face of $\sigma^n$ is, by definition,
an $(n-1)$-simplex in $X$. The attaching map is (by induction)
really determined by a map $\{v_0,\ldots ,v_n\}\ra X^{(0)}$, since this 
determines the attaching maps for the 1-simplices in the boundary of the
$n$-simplex, which 
gives 1-simplices in $X$, which then give the attaching maps for
the 2-simplices in the boundary, etc. Note that the reverse is not true;
the vertices of two different $n$-simplices in $X$ can be the same.
For example, think of the 2-sphere as a pair of 2-simplices whose 
boundaries are glued by the identity. 

\msk

The final detail that we need before defining (simplicial) homology
groups is the notion of an {\it orientation} on a simplex of $X$.
Each simplex $\sigma^n$ is determined by a map 
$f:\{v_0,\ldots ,v_n\}\ra X^{(0)}$; an orientation on $\sigma^n$ is an
(equivalence class of) the ordered $(n+1)$-tuple $(f(v_0),\ldots f(v_n)) = (V_0,\ldots ,V_n)$.
Another ordering of the
same vertices represents the same orientation if there is an {\it even} permutation
taking the entries of the first $(n+1)$-tuple to the second. This should be thought 
of as a generalization of the right-hand rule for ${\Bbb R}^3$, interpreted as
orienting the vertices of a 3-simplex. Note that there are precisely two
orientations on a simplex.

\msk

Now to define homology! We start by defining $n$-{\it chains};
these are (finite) formal linear combinations of the (oriented!) $n$-simplices
of $X$, where $-\sigma$ is interpreted as $\sigma$ with the opposite
(i.e., other) orientation. Adding formal linear combinations formally,
we get the $n$-th {\it chain group} 
$C_n(X) = \{\sum n_\alpha \sigma_\alpha$ : $\sigma_\alpha$ an oriented $n$-simplex in $X\}$ .
We next define a {\it boundary operator} $\del:C_n(X)\ra C_{n-1}(X)$, whose image will be 
the $(n-1)$-chains that are the ``boundaries'' of $n$-chains. We define it on the basis
elements $\sigma_\alpha = \sigma$ of $C_n(X)$ as
$\del\sigma = \sum (-1)^i\sigma|_{[v_0,\ldots,\widehat{v_{i}},\ldots ,v_n]}$ , 
where $\sigma:[v_0,\ldots ,v_n]\ra X$ is the characteristic map of $\sigma_\alpha$ .
$\del\sigma$ is therefore an alternating sum of the faces of $\sigma$. 
The pont that really make this definition
go is that we need {\it oriented} simplices, so that we know what the $i$-th face
of $\sigma$ is (the one oppoisite the $i$-th vertex).
We then extend the definition by linearity to all of $C_n(X)$. When a notation indicating
dimension is needed, we write $\del=\del_n$ . 

\msk

This definition is cooked up to make the maxim ``boundaries have no boundary'' true;
that is $\delta_{n-1}\circ \delta_n = 0$, the $0$ map. This is because, for any simplex
$\sigma = [v_0,\ldots v_n]$, 

\msk

$\displaystyle \delta\circ\delta(\sigma) = 
\delta(\sum_{i=0}^n  (-1)^i\sigma|_{[v_0,\ldots,\widehat{v_{i}},\ldots ,v_n]})$

= $\displaystyle (\sum_{j<i}(-1)^j(-1)^i\sigma|_{[v_0,\ldots,\widehat{v_{j}},\ldots,\widehat{v_{i}},\ldots ,v_n]})
+(\sum_{j>i}(-1)^{j-1}(-1)^i\sigma|_{[v_0,\ldots,\widehat{v_{i}},\ldots,\widehat{v_{j}},\ldots ,v_n]})$

\msk

The distinction between the two pieces is that in the second part, $v_j$ is actually the $(j-1)$-st vertex
of the face. Switching the roles of $i$ and $j$ in the second sum, we find that the two are
negatives of one another, so they sum to $0$, as desired.

\msk

And this little calculation is all that it takes to define homology groups! What this tells 
us is that im$(\delta_{n+1})\subseteq\ker(\delta_{n}$ for every $n$. 
$\ker(\delta_{n}=Z_n(X)$ are called the {\it $n$-cycles} of $X$; they are the $n$-chains with
$0$ (i.e., empty) boundary. They form a (free) abelian subgroup of $C_n(X)$. 
im$(\delta_{n+1} = B_n(X)$ are the {\it $n$-boundaries} of $X$; they are, of course,
the boundaries of $(n+1)$-chains in $X$. The $n$-th homology group of $X$,
$H_n(X)$ is the quotient $Z_n(X)/B_n(X)$ ; it is an abelian group.



\msk




\vfill
\end