

\magnification=1200
\overfullrule=0pt
\parindent=0pt

\nopagenumbers

\input amstex

\voffset=-.6in
\hoffset=-.5in
\hsize = 7.5 true in
\vsize=10.4 true in

%\voffset=1.4in
%\hoffset=-.5in
%\hsize = 10.2 true in
%\vsize=8 true in

\input colordvi

\def\cltr{\Red}		  % Red  VERY-Approx PANTONE RED
\def\cltb{\Blue}		  % Blue  Approximate PANTONE BLUE-072
\def\cltg{\PineGreen}	  % ForestGreen  Approximate PANTONE 349
\def\cltp{\DarkOrchid}	  % DarkOrchid  No PANTONE match
\def\clto{\Orange}	  % Orange  Approximate PANTONE ORANGE-021
\def\cltpk{\CarnationPink}	  % CarnationPink  Approximate PANTONE 218
\def\clts{\Salmon}	  % Salmon  Approximate PANTONE 183
\def\cltbb{\TealBlue}	  % TealBlue  Approximate PANTONE 3145
\def\cltrp{\RoyalPurple}	  % RoyalPurple  Approximate PANTONE 267
\def\cltp{\Purple}	  % Purple  Approximate PANTONE PURPLE

\def\cgy{\GreenYellow}     % GreenYellow  Approximate PANTONE 388
\def\cyy{\Yellow}	  % Yellow  Approximate PANTONE YELLOW
\def\cgo{\Goldenrod}	  % Goldenrod  Approximate PANTONE 109
\def\cda{\Dandelion}	  % Dandelion  Approximate PANTONE 123
\def\capr{\Apricot}	  % Apricot  Approximate PANTONE 1565
\def\cpe{\Peach}		  % Peach  Approximate PANTONE 164
\def\cme{\Melon}		  % Melon  Approximate PANTONE 177
\def\cyo{\YellowOrange}	  % YellowOrange  Approximate PANTONE 130
\def\coo{\Orange}	  % Orange  Approximate PANTONE ORANGE-021
\def\cbo{\BurntOrange}	  % BurntOrange  Approximate PANTONE 388
\def\cbs{\Bittersweet}	  % Bittersweet  Approximate PANTONE 167
%\def\creo{\RedOrange}	  % RedOrange  Approximate PANTONE 179
\def\cma{\Mahogany}	  % Mahogany  Approximate PANTONE 484
\def\cmr{\Maroon}	  % Maroon  Approximate PANTONE 201
\def\cbr{\BrickRed}	  % BrickRed  Approximate PANTONE 1805
\def\crr{\Red}		  % Red  VERY-Approx PANTONE RED
\def\cor{\OrangeRed}	  % OrangeRed  No PANTONE match
\def\paru{\RubineRed}	  % RubineRed  Approximate PANTONE RUBINE-RED
\def\cwi{\WildStrawberry}  % WildStrawberry  Approximate PANTONE 206
\def\csa{\Salmon}	  % Salmon  Approximate PANTONE 183
\def\ccp{\CarnationPink}	  % CarnationPink  Approximate PANTONE 218
\def\cmag{\Magenta}	  % Magenta  Approximate PANTONE PROCESS-MAGENTA
\def\cvr{\VioletRed}	  % VioletRed  Approximate PANTONE 219
\def\parh{\Rhodamine}	  % Rhodamine  Approximate PANTONE RHODAMINE-RED
\def\cmu{\Mulberry}	  % Mulberry  Approximate PANTONE 241
\def\parv{\RedViolet}	  % RedViolet  Approximate PANTONE 234
\def\cfu{\Fuchsia}	  % Fuchsia  Approximate PANTONE 248
\def\cla{\Lavender}	  % Lavender  Approximate PANTONE 223
\def\cth{\Thistle}	  % Thistle  Approximate PANTONE 245
\def\corc{\Orchid}	  % Orchid  Approximate PANTONE 252
\def\cdo{\DarkOrchid}	  % DarkOrchid  No PANTONE match
\def\cpu{\Purple}	  % Purple  Approximate PANTONE PURPLE
\def\cpl{\Plum}		  % Plum  VERY-Approx PANTONE 518
\def\cvi{\Violet}	  % Violet  Approximate PANTONE VIOLET
\def\clrp{\RoyalPurple}	  % RoyalPurple  Approximate PANTONE 267
\def\cbv{\BlueViolet}	  % BlueViolet  Approximate PANTONE 2755
\def\cpe{\Periwinkle}	  % Periwinkle  Approximate PANTONE 2715
\def\ccb{\CadetBlue}	  % CadetBlue  Approximate PANTONE (534+535)/2
\def\cco{\CornflowerBlue}  % CornflowerBlue  Approximate PANTONE 292
\def\cmb{\MidnightBlue}	  % MidnightBlue  Approximate PANTONE 302
\def\cnb{\NavyBlue}	  % NavyBlue  Approximate PANTONE 293
\def\crb{\RoyalBlue}	  % RoyalBlue  No PANTONE match
%\def\cbb{\Blue}		  % Blue  Approximate PANTONE BLUE-072
\def\cce{\Cerulean}	  % Cerulean  Approximate PANTONE 3005
\def\ccy{\Cyan}		  % Cyan  Approximate PANTONE PROCESS-CYAN
\def\cpb{\ProcessBlue}	  % ProcessBlue  Approximate PANTONE PROCESS-BLUE
\def\csb{\SkyBlue}	  % SkyBlue  Approximate PANTONE 2985
\def\ctu{\Turquoise}	  % Turquoise  Approximate PANTONE (312+313)/2
\def\ctb{\TealBlue}	  % TealBlue  Approximate PANTONE 3145
\def\caq{\Aquamarine}	  % Aquamarine  Approximate PANTONE 3135
\def\cbg{\BlueGreen}	  % BlueGreen  Approximate PANTONE 320
\def\cem{\Emerald}	  % Emerald  No PANTONE match
%\def\cjg{\JungleGreen}	  % JungleGreen  Approximate PANTONE 328
\def\csg{\SeaGreen}	  % SeaGreen  Approximate PANTONE 3268
\def\cgg{\Green}	  % Green  VERY-Approx PANTONE GREEN
\def\cfg{\ForestGreen}	  % ForestGreen  Approximate PANTONE 349
\def\cpg{\PineGreen}	  % PineGreen  Approximate PANTONE 323
\def\clg{\LimeGreen}	  % LimeGreen  No PANTONE match
\def\cyg{\YellowGreen}	  % YellowGreen  Approximate PANTONE 375
\def\cspg{\SpringGreen}	  % SpringGreen  Approximate PANTONE 381
\def\cog{\OliveGreen}	  % OliveGreen  Approximate PANTONE 582
\def\pars{\RawSienna}	  % RawSienna  Approximate PANTONE 154
\def\cse{\Sepia}		  % Sepia  Approximate PANTONE 161
\def\cbr{\Brown}		  % Brown  Approximate PANTONE 1615
\def\cta{\Tan}		  % Tan  No PANTONE match
\def\cgr{\Gray}		  % Gray  Approximate PANTONE COOL-GRAY-8
\def\cbl{\Black}		  % Black  Approximate PANTONE PROCESS-BLACK
\def\cwh{\White}		  % White  No PANTONE match


\loadmsbm

\input epsf

\def\ctln{\centerline}
\def\u{\underbar}
\def\ssk{\smallskip}
\def\msk{\medskip}
\def\bsk{\bigskip}
\def\hsk{\hskip.1in}
\def\hhsk{\hskip.2in}
\def\dsl{\displaystyle}
\def\hskp{\hskip1.5in}

\def\lra{$\Leftrightarrow$ }
\def\ra{\rightarrow}
\def\mpto{\logmapsto}
\def\pu{\pi_1}
\def\mpu{$\pi_1$}
\def\sig{\Sigma}
\def\msig{$\Sigma$}
\def\ep{\epsilon}
\def\sset{\subseteq}
\def\del{\partial}
\def\inv{^{-1}}
\def\wtl{\widetilde}
\def\lra{\Leftrightarrow}
\def\del{\partial}
\def\delp{\partial^\prime}
\def\delpp{\partial^{\prime\prime}}
\def\sgn{{\roman{sgn}}}
\def\wtih{\widetilde{H}}
\def\bbz{{\Bbb Z}}
\def\bbr{{\Bbb R}}



\ctln{\bf Math 971 Algebraic Topology}

\ssk

\ctln{April 21, 2005}

\msk


Continuing with: \hhsk
For $k<n$ and $h:I^k\ra S^n$ an embedding of a $k$-cube in to the $n$-sphere,
$\wtih_i(S^n\setminus h(I^k))=0$ for all $i$.

\msk

We've shown how we can throw away half of the cube without losing a (chosen)
non-zero homology element.
Now we continue inductively, cutting $C\times [0,1]$ in two along the last coordinate as
$C\times [0,1/2],C\times [1/2,1]$ and repeat the same argument. We fnd that
$\wtih_i(S^n\setminus h(C\times [a,b]))\neq 0$, and $[z]$ maps to a non-zero 
element under the inclusion-induced homomorphism.. Continuing inductively, we find a
sequence of nested intervals $I_n=[a_n,b_n]\supseteq [a_{n+1},b_{n+1}]$ 
whose lengths tend to zero (so $a_n,b_n\ra x_0\in I$ as $n\ra\infty$), and injective inclusion-induced maps

\ctln{$0\neq \wtih_i(S^n\setminus h(I^n)\ra \cdots \ra \wtih_i(S^n\setminus h(C\times I_n)
\ra \wtih_i(S^n\setminus h(C\times I_{n+1})$}

all of which send a certain non-zero element $[z]\in\wtih_i(S^n\setminus h(I^n)$ to 
a non-zero element, and all of which have an inclusion-induced map to $\wtih_i(S^n\setminus h(C\times \{x_0\}) = 0$.
So there is a non-trivial element $[z]\in \wtih_i(S^n\setminus h(I^n)$ which \u{remains}
non-zero in all $\wtih_i(S^n\setminus h(C\times I_n))$, but is zero in $\wtih_i(S^n\setminus h(C\times \{x_0\})$.
Consequently, $z\del w$ for some chain $w=\sum a_j\sigma_j^{i+1}\in C_{i+1}(S^n\setminus h(C\times \{x_0\}))$.
Each singular simplex, however, is a map $\sigma_j^{i+1}:\Delta^{i+1}\ra S^n\setminus h(C\times \{x_0\})$,
and so has compact image. But the sets $S^n\setminus h(C\times I_n)$ form a nested open cover of
$S^n\setminus h(C\times \{x_0\})$, and so of $\sigma_j^{i+1}(\Delta^{i+1})$, and so there is an
$n_j$ with $\sigma_j^{i+1}(\Delta^{i+1})\subseteq S^n\setminus h(C\times I_{n_j})$ .
Then setting $N=$max$\{n_j\}$, we have $\sigma_j^{i+1}:\Delta^{i+1}\ra S^n\setminus h(C\times I_N)$
for every $j$,
so $w\in C_{i+1}(S^n\setminus h(C\times I_N)$, so $0=[z]\in \wtih_i(S^n\setminus h(C\times I_N)$,
a contradiction. So $\wtih_i(S^n\setminus h(I^k))=0$, and our inductive step is proved.

\msk

One immediate consequence of this is that if $h:S^k\ra S^n$ is an embedding of the $k$-sphere into the $n$-sphere,
then thinking of $S^k$ as the union of its upper and lower hemispheres, $D^k_+,D^k_-$, each of which is homeomorphic
to $I^k$, we have  $D^k_+\cap D^k_-=S^{k-1}$, the equatorial $(k-1)$-sphere, and so by Mayer-Vietoris we have

\ssk

$\cdots \ra 
\wtih_{i+1}(S^n\setminus h(D^k_-))\oplus \wtih_{i+1}(S^n\setminus h(D^k_+)\ra
\wtih_{i+1}(S^n\setminus h(S^{k-1}))\ra \wtih_{i}(S^n\setminus h(S^k))\ra$

\hfill $\wtih_{i}(S^n\setminus h(D^k_-))\oplus \wtih_{i}(S^n\setminus h(D^k_+)\ra \cdots $

i.e., $\wtih_{i}(S^n\setminus h(S^k)) \cong \wtih_{i+1}(S^n\setminus h(S^{k-1})) 
\cong \cdots \cong \wtih_{i+k}(S^n\setminus h(S^0))\cong \wtih_{i+k}(S^{n-1})$ ,
since $S^0$ = 2 points, and  so $S^n\setminus h(S^0)\cong S^{n-1}\times \bbr \simeq S^{n-1}$.
So $\wtih_{i}(S^n\setminus h(S^k)) = 0$ unless $i+k=n-1$ (i.e., $i=n-k-1$), when it is $\bbz$.

\ssk

In particular, $\wtih_0(S^n\setminus h(S^{n-1}))=\bbz$, so 
we have the \crr{Jordan-Brouwer Separation Theorem: every embedded $S^{n-1}$ in $S^n$
has two complementary path-components $A,B$} . With a little work, one can show that
$\overline{A}\cap \overline{B} = h(S^{n-1}$ , so the $(n-1)$-sphere is the frontier of each
complementary component. [Removing a point from $S^n$ to get $\bbr^n$ does not change the
conclusion (for $n>1$); a point does not disconnect an open subset of $S^n$.]

\msk

When $n=2$, the Jordan Curve Theorem (as it is then called) has the additional
consequence that the closure of each complementary region is a compact 2-disk,
each having the embedded circle $h(S^1)$ as its boundary. This stronger result
does not extend to higher dimensions, without putting extra restrictions 
on the embedding. This was shown by Alexander (shortly after publishing an
incorrect proof without restrictions) for $n=3$; these examples are known as
the Alexander horned spheres.

\msk

To prove Invariance of Domain, let ${\Cal U}\subseteq \bbr^n\subseteq S^n$ be an open 
set, and $f:{\Cal U}\ra \bbr^n \hookrightarrow S^n$ be injective and continuous. It suffices
to show, for every $x\in {\Cal U}$, that there is an open neighborhood ${\Cal V}$ with
$f(x)\subseteq {\Cal V} \subseteq f({\Cal U})$ . Since ${\Cal U}$ is open,
there is an open ball $B^n$ centered at $x$ whose closure $D^n$ is contained in ${\Cal U}$. 
$f$ is then an embedding of $\del D^n = S^{n-1}$ into $S^n$, and of $D^n\cong I^n$ into $S^n$.
By our calculations above, $S^n\setminus f(S^{n-1})$ has two path components $A,B$; being an open set 
and contained in a locally path-connected space, these are also the connected components
of the complement. But our calculations above also show that $S^n\setminus f(D^n)$ is
path-connected, hence connected, and $f(B^n)$, being the image of a connected set, is connected.
Since $f(B^n)\cup (S^n\setminus f(D^n)) = S^n\setminus f(S^{n-1} = A\cup B)$, it follows that
$f(B^n)=A$ and $S^n\setminus f(D^n) = B$ (or vice versa). In particular,
$f(B^n)$ is open, forming an open subset of $f({\Cal U})$ containing $f(x)$, as desired.



\vfill
\end


