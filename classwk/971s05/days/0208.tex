

\magnification=1200
\overfullrule=0pt
\parindent=0pt

\nopagenumbers

\input amstex

\voffset=-.6in
\hoffset=-.5in
\hsize = 7.5 true in
\vsize=10.4 true in

%\voffset=1.4in
%\hoffset=-.5in
%\hsize = 10.2 true in
%\vsize=8 true in

\input colordvi

\loadmsbm

\input epsf

\def\ctln{\centerline}
\def\u{\underbar}
\def\ssk{\smallskip}
\def\msk{\medskip}
\def\bsk{\bigskip}
\def\hsk{\hskip.1in}
\def\hhsk{\hskip.2in}
\def\dsl{\displaystyle}
\def\hskp{\hskip1.5in}

\def\lra{$\Leftrightarrow$ }
\def\ra{\rightarrow}
\def\mpto{\logmapsto}
\def\pu{\pi_1}
\def\mpu{$\pi_1$}
\def\sig{\Sigma}
\def\msig{$\Sigma$}
\def\ep{\epsilon}
\def\sset{\subseteq}
\def\del{\partial}
\def\inv{^{-1}}
\def\wtl{\widetilde}



\ctln{\bf Math 971 Algebraic Topology}

\ssk

\ctln{February 8, 2005}

\msk

{\bf Some computations:}

\msk

{\it Gluing on a 2-disk:} If $X$ is a topological 
space and $f:\del {\Bbb D}^2\ra X$ is continuous, then we
can construct the quotient space $Z=(X\coprod 
{\Bbb D}^2)/\{x\sim f(x) : x\in\del{\Bbb D}^2\}$,
the result of gluing ${\Bbb D}^2$ to $X$ along $f$. 
We can use Seifert - van Kampen to compute \mpu\ 
of the resulting space, although if we
wish to be careful with basepoints $x_0$ 
(e.g., the image of $f$ might not contain $x_0$, and/or we
may wish to glue several disks on, in remote parts of $X$),
we should also include a rectangle $R$, the mapping 
cylinder of a path $\gamma$ running from 
$f(1,0)$ to $x_0$, glued to 
${\Bbb D}^2$ along the arc from $(1/2,0)$ to $(1,0)$ (see figure). 
This space $Z_+$ deformation retracts to $Z$, but it
is technically simpler to do our calculations 
with the basepoint $y_0$ lying above $x_0$.
If we write $D_1 = \{x\in {\Bbb D}^2 : ||x||<1\}\cup(R\setminus X)$ 
and $D_2 = \{x\in {\Bbb D}^2 : ||x||>1/3\}\cup R$ , 
then we can write $Z_+=D_+\cup(X\cup D_2) = X_1\cup X_2$.
But since $X_1\simeq *$ , $X_2\simeq X$ 
(it is essentially the mapping cylinder of 
the maps $f$ and $\gamma$ )
and $X_1\cap X_2 = \{x\in {\Bbb D}^2 : 
1/3<||x||<1\}\cap(R\setminus X)\sim S^1$, we find that 

\ctln{$\pu(Z,y_0)\cong \pu(X_2,y_0)*_{\Bbb Z}\{1\} = 
\pu(X_2)/<{\Bbb Z}>^N \cong 
\pu(X_2)/<[\overline{\delta}*\overline{\gamma}*f*\gamma*\delta]>^N$}

If we then use $\delta$ as a path for a change of 
basepoint isomorphism, and then a homotopy
equivalence from $X_2$ to $X$ (fixing $x_0$), we 
have, in terms of group presentations, 
if $\pu(X,x_0)=<\sig | R>$ , then $\pu(Z) = 
<\sig | R\cup\{[\overline{\gamma}*f*\gamma]\}>$ . 
So the effect of gluing on a 2-disk on the fundamental 
group is to add a new relator, 
namely the word represented by the attaching map 
(adjusting for basepoint). 
All of this applies equally well to attaching several 
2-disks; each adds a new relator. 

\msk

\leavevmode

\epsfxsize=5.4in
\ctln{{\epsfbox{0201f1.ai}}}



\bsk

The inherent complications above derived from needing open
sets can be legislated away, by introducing additional
hypotheses:

\msk

{\bf Theorem:} If $X=X_1\cup X_2$ is a union of closed
sets $X_1,X_2$, with $A=X_1\cap X_2$ path-connected, 
and if $X_1,X_2$ have open neighborhood ${\Cal U}_1,{\Cal U}_2$
so that ${\Cal U}_1,{\Cal U}_2,{\Cal U}_1\cap{\Cal U}_2$
deformation retract onto $X_1,X_2,A$ respectively, then 
$\pu(X)\cong \pu(X_1)*_{\pu(A)}\pu(X_2)$ as before.

\msk

The hypotheses are satisfied, for example, if $X_1.X_2$ are subcomplexes of the
cell complex $X$.

\msk


This in turn opens up huge possibilities for the 
computation of $\pu(X)$. For example, for cell complexes,
we can inductively compute \mpu\ by starting with 
the 1-skeleton, with free fundamental group, and 
attaching the 2-cells one by one, which each add 
a relator to the presentation of $\pu(X)$ . 
[{\bf Exercise:}
(Hatcher, p.53, \#\ 6) Attaching $n$-cells, for 
$n\geq 3$, has no effect on \mpu .] 
For example, the 2-sphere $S^2$ can be thought of as a
2-disk with a 2-disk attached, along a circle, and so has
$\pu(S^2)\cong \{1\}_{\Bbb Z}\{1\}=\{1\}$ . 
We can also compute the fundamental group of 
any compact surface:

\msk

The {\it real projective plane} ${\Bbb R}P^2$ is the quotient of 
the 2-sphere $S^2$ by the antipodal map $x\mapsto -x$; 
it can also be thought of as the upper hemisphere, with 
identification only along the boundary. This in turn can be 
interpreted as a 2-disk glued to a circle, whose boundary
wraps around the circle twice. So $\pu({\Bbb R}P^2)\cong 
<a | a^2>\cong {\Bbb Z}_2 = {\Bbb Z}/2{\Bbb Z}$ .
A surface $F$ of genus 2 can be given a cell structure with 1 0-cell,
4 1-cells, and 1 2-cell, as in the figure, as in the first of the 
figures below. The fundamental group
of the 1-skeleton is therefore free of rank 4, and $\pu(F)$ has
a presentation with 4 generators and 1 relator. Reading the attaching
map from the figure, the presentation is $<a,b,c,d\ |\ [a,b][c,d] >$ .

\msk



\leavevmode

\epsfxsize=5.6in
\ctln{{\epsfbox{0201f2.ai}}}

\msk

Giving it a different cell structure, as in the second figure, with 2 0-cells,
6 1-cells, and 2 2-cells, after choosing a maximal tree, we can read off the
two relators from the 2-cells to arrive at a different presentation
$\pu(F) = <a,b,c,d,e\ |\ aba^{-1}eb^{-1},cde^{-1}c^{-1}d^{-1}>$ . A posteriori,
these two presentations describe isomorphic groups.

\bsk

Using the same technology, we can also see that, in general, 
any group is the fundamental group of some 2-complex $X$;
starting with a presentation $G = <\sig | R>$, build $X$ by starting
with a bouquet of $|\sig|$ circles, and attach $|R|$ 2-disks along
loops which represent each of the generators of $R$. (This works just
as well for infinite sets $\sig$ and/or $R$; essentially the same proofs
as above apply.)

\bsk


\vfill
\end









