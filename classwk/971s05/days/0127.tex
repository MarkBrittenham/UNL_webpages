

\magnification=1500
\overfullrule=0pt
\parindent=0pt

\nopagenumbers

\input amstex

\voffset=-.6in
\hoffset=-.5in
\hsize = 7.5 true in
\vsize=10.6 true in

%\voffset=1.4in
%\hoffset=-.5in
%\hsize = 10.2 true in
%\vsize=8 true in

\input colordvi

\loadmsbm

\input epsf

\def\ctln{\centerline}
\def\u{\underbar}
\def\ssk{\smallskip}
\def\msk{\medskip}
\def\bsk{\bigskip}
\def\hsk{\hskip.1in}
\def\hhsk{\hskip.2in}
\def\dsl{\displaystyle}
\def\hskp{\hskip1.5in}

\def\lra{$\Leftrightarrow$ }
\def\ra{\rightarrow}
\def\mpto{\logmapsto}
\def\pu{\pi_1}
\def\mpu{$\pi_1$}
\def\sig{\Sigma}
\def\msig{$\Sigma$}
\def\ep{\epsilon}
\def\sset{\subseteq}




\ctln{\bf Math 971 Algebraic Topology}

\ssk

\ctln{January 27, 2005}

\msk

{\bf Group theory ``done right'': presentations}

\msk

$\Sigma$ = a set; a {\it reduced word} on \msig\ is a (formal)
product $a_1^{\ep_1}\cdots a_n^{\ep_n}$ with $a_i\in\sig$ and $\ep_i=\pm 1$,
and either $a_i\neq a_{i+1}$ or $\ep_i\neq \ep_{i+1}$ for every $i$. (I.e., no
$aa^{-1},a^{-1}a$ in the product.)

\ssk

The free group $F(\sig)$ = the set of reduced words, with multiplication = concatenation 
followed by reduction; remove all possible $aa^{-1},a^{-1}a$ from the site of concatenation.

\ssk

identity element = the empty word, 
$(a_1^{\ep_1}\cdots a_n^{\ep_n})^{-1} = a_n^{-\ep_n}\cdots a_1^{-\ep_1}$. 
$F(\sig)$ is generated by \msig, with no relations among the generators
other than the ``obvious'' ones.

\msk

Important property of free groups: any function $f:\sig\ra G$ , $G$ a group, extends
uniquely to a homomorphism $\phi: F(\sig)\ra G$.

\msk

If $R\sset F(\sig)$, then $<R>^N$ = normal subgroup generated by $R$ 

= $\displaystyle \{\prod_{i=1}^n g_i r_i g_i^{-1} : n\in{\Bbb N}_0 , g_i\in F(\sig) , r_i\in R\}$

=smallest normal subgroup containing $R$.

\ssk

$F(\sig)/<R>^N$ = the group with {\it presentation} $<\sig | R>$ ; it is the largest quotient
of $F(\sig)$ in which the elements of $R$ are the identity. Every group has a presentation:

\ctln{$G$ = $F(G)/<gh(gh)^{-1} : g,h\in G>^N$}

where $(gh)$ is interpreted as a single letter in $G$.

\msk

If $G_1=<\sig_1 | R_1>$ and $G_2=<\sig_2 | R_2>$, then their {\it free product}
$G_1*G_2 = <\sig_1\coprod\sig_2 | R_1\cup R_2>$ ($\sig_1,\sig_2$ must be 
treated as (formally) disjoint). Each element has a unique reduced form as
$g_1\cdots g_n$ where the $g_i$ alternate from $G_1,G_2$.
$G_1,G_2$ can be thought of as subgroups for $G_1*G_2$, in the obivous way.
Important property of free products: any pair of
homoms $\phi_i:G_i\ra G$ extends uniquely to a homom $\phi:G_1*G_2\ra G$
(exactly the way you think it does).


\vfill
\end






















