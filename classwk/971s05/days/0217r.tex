
\magnification=1200
\overfullrule=0pt
\parindent=0pt

\nopagenumbers

\input amstex

%\voffset=-.6in
%\hoffset=-.5in
%\hsize = 7.5 true in
%\vsize=10.4 true in

\voffset=1.3in
\hoffset=-.5in
\hsize = 10.2 true in
\vsize=8.2 true in

\input colordvi

\loadmsbm

\input epsf

\def\ctln{\centerline}
\def\u{\underbar}
\def\ssk{\smallskip}
\def\msk{\medskip}
\def\bsk{\bigskip}
\def\hsk{\hskip.1in}
\def\hhsk{\hskip.2in}
\def\dsl{\displaystyle}
\def\hskp{\hskip1.5in}

\def\lra{$\Leftrightarrow$ }
\def\ra{\rightarrow}
\def\mpto{\logmapsto}
\def\pu{\pi_1}
\def\mpu{$\pi_1$}
\def\sig{\Sigma}
\def\msig{$\Sigma$}
\def\ep{\epsilon}
\def\sset{\subseteq}
\def\del{\partial}
\def\inv{^{-1}}
\def\wtl{\widetilde}

\def\cltr{\Red}		  % Red  VERY-Approx PANTONE RED
\def\cltb{\Blue}		  % Blue  Approximate PANTONE BLUE-072
\def\cltg{\PineGreen}	  % ForestGreen  Approximate PANTONE 349
\def\cltp{\DarkOrchid}	  % DarkOrchid  No PANTONE match
\def\clto{\Orange}	  % Orange  Approximate PANTONE ORANGE-021
\def\cltpk{\CarnationPink}	  % CarnationPink  Approximate PANTONE 218
\def\clts{\Salmon}	  % Salmon  Approximate PANTONE 183
\def\cltbb{\TealBlue}	  % TealBlue  Approximate PANTONE 3145
\def\cltrp{\RoyalPurple}	  % RoyalPurple  Approximate PANTONE 267
\def\cltp{\Purple}	  % Purple  Approximate PANTONE PURPLE

\def\cgy{\GreenYellow}     % GreenYellow  Approximate PANTONE 388
\def\cyy{\Yellow}	  % Yellow  Approximate PANTONE YELLOW
\def\cgo{\Goldenrod}	  % Goldenrod  Approximate PANTONE 109
\def\cda{\Dandelion}	  % Dandelion  Approximate PANTONE 123
\def\capr{\Apricot}	  % Apricot  Approximate PANTONE 1565
\def\cpe{\Peach}		  % Peach  Approximate PANTONE 164
\def\cme{\Melon}		  % Melon  Approximate PANTONE 177
\def\cyo{\YellowOrange}	  % YellowOrange  Approximate PANTONE 130
\def\coo{\Orange}	  % Orange  Approximate PANTONE ORANGE-021
\def\cbo{\BurntOrange}	  % BurntOrange  Approximate PANTONE 388
\def\cbs{\Bittersweet}	  % Bittersweet  Approximate PANTONE 167
%\def\creo{\RedOrange}	  % RedOrange  Approximate PANTONE 179
\def\cma{\Mahogany}	  % Mahogany  Approximate PANTONE 484
\def\cmr{\Maroon}	  % Maroon  Approximate PANTONE 201
\def\cbr{\BrickRed}	  % BrickRed  Approximate PANTONE 1805
\def\crr{\Red}		  % Red  VERY-Approx PANTONE RED
\def\cor{\OrangeRed}	  % OrangeRed  No PANTONE match
\def\paru{\RubineRed}	  % RubineRed  Approximate PANTONE RUBINE-RED
\def\cwi{\WildStrawberry}  % WildStrawberry  Approximate PANTONE 206
\def\csa{\Salmon}	  % Salmon  Approximate PANTONE 183
\def\ccp{\CarnationPink}	  % CarnationPink  Approximate PANTONE 218
\def\cmag{\Magenta}	  % Magenta  Approximate PANTONE PROCESS-MAGENTA
\def\cvr{\VioletRed}	  % VioletRed  Approximate PANTONE 219
\def\parh{\Rhodamine}	  % Rhodamine  Approximate PANTONE RHODAMINE-RED
\def\cmu{\Mulberry}	  % Mulberry  Approximate PANTONE 241
\def\parv{\RedViolet}	  % RedViolet  Approximate PANTONE 234
\def\cfu{\Fuchsia}	  % Fuchsia  Approximate PANTONE 248
\def\cla{\Lavender}	  % Lavender  Approximate PANTONE 223
\def\cth{\Thistle}	  % Thistle  Approximate PANTONE 245
\def\corc{\Orchid}	  % Orchid  Approximate PANTONE 252
\def\cdo{\DarkOrchid}	  % DarkOrchid  No PANTONE match
\def\cpu{\Purple}	  % Purple  Approximate PANTONE PURPLE
\def\cpl{\Plum}		  % Plum  VERY-Approx PANTONE 518
\def\cvi{\Violet}	  % Violet  Approximate PANTONE VIOLET
\def\clrp{\RoyalPurple}	  % RoyalPurple  Approximate PANTONE 267
\def\cbv{\BlueViolet}	  % BlueViolet  Approximate PANTONE 2755
\def\cpe{\Periwinkle}	  % Periwinkle  Approximate PANTONE 2715
\def\ccb{\CadetBlue}	  % CadetBlue  Approximate PANTONE (534+535)/2
\def\cco{\CornflowerBlue}  % CornflowerBlue  Approximate PANTONE 292
\def\cmb{\MidnightBlue}	  % MidnightBlue  Approximate PANTONE 302
\def\cnb{\NavyBlue}	  % NavyBlue  Approximate PANTONE 293
\def\crb{\RoyalBlue}	  % RoyalBlue  No PANTONE match
%\def\cbb{\Blue}		  % Blue  Approximate PANTONE BLUE-072
\def\cce{\Cerulean}	  % Cerulean  Approximate PANTONE 3005
\def\ccy{\Cyan}		  % Cyan  Approximate PANTONE PROCESS-CYAN
\def\cpb{\ProcessBlue}	  % ProcessBlue  Approximate PANTONE PROCESS-BLUE
\def\csb{\SkyBlue}	  % SkyBlue  Approximate PANTONE 2985
\def\ctu{\Turquoise}	  % Turquoise  Approximate PANTONE (312+313)/2
\def\ctb{\TealBlue}	  % TealBlue  Approximate PANTONE 3145
\def\caq{\Aquamarine}	  % Aquamarine  Approximate PANTONE 3135
\def\cbg{\BlueGreen}	  % BlueGreen  Approximate PANTONE 320
\def\cem{\Emerald}	  % Emerald  No PANTONE match
%\def\cjg{\JungleGreen}	  % JungleGreen  Approximate PANTONE 328
\def\csg{\SeaGreen}	  % SeaGreen  Approximate PANTONE 3268
\def\cgg{\Green}	  % Green  VERY-Approx PANTONE GREEN
\def\cfg{\ForestGreen}	  % ForestGreen  Approximate PANTONE 349
\def\cpg{\PineGreen}	  % PineGreen  Approximate PANTONE 323
\def\clg{\LimeGreen}	  % LimeGreen  No PANTONE match
\def\cyg{\YellowGreen}	  % YellowGreen  Approximate PANTONE 375
\def\cspg{\SpringGreen}	  % SpringGreen  Approximate PANTONE 381
\def\cog{\OliveGreen}	  % OliveGreen  Approximate PANTONE 582
\def\pars{\RawSienna}	  % RawSienna  Approximate PANTONE 154
\def\cse{\Sepia}		  % Sepia  Approximate PANTONE 161
\def\cbr{\Brown}		  % Brown  Approximate PANTONE 1615
\def\cta{\Tan}		  % Tan  No PANTONE match
\def\cgr{\Gray}		  % Gray  Approximate PANTONE COOL-GRAY-8
\def\cbl{\Black}		  % Black  Approximate PANTONE PROCESS-BLACK
\def\cwh{\White}		  % White  No PANTONE match


\ctln{\bf Math 971 Algebraic Topology}

\ssk

\ctln{February 17, 2005}

\ssk

The number of sheets of a covering map can also be determined 
from the fundamental groups $H=p_*(\pu(\wtl{X},\wtl{x}_0)\sset\pu(X,x_0)=G$ :

\ctln{\cltr{{\bf Proposition:} If $X$ and $\wtl{X}$ are 
path-connected, then the number of sheets of a covering map equals
the index  of $H$ in $G$ . }}

To see this, choose loops 
$\{\gamma_\alpha\}$ representing representatives $\{g_\alpha\}$ of each of the (right) cosets of $H$ in $G$. Lifting them to loops based at $\wtl{x}_0$, they will have distinct
endpoints; if $\wtl{\gamma}_1(1)=\wtl{\gamma}_2(1)$, then 
by uniqueness of lifts, $\gamma_1*\overline{\gamma_2}$ lifts to 
$\wtl{\gamma}_1*\overline{\wtl{\gamma}_2}$, so it
lifts to a loop, so $\gamma_1*\overline{\gamma_2}$ gives
an element of $p_*(\pu(\wtl{X},\wtl{x}_0)$, so $g_1=g_2$.
Conversely, every point in $p\inv(x_0)$ is the end of a
lift; we can construct a path $\wtl{\gamma}$
from $\wtl{x}_0$ to any such point $y$, giving a loop
$\gamma=p\circ \wtl{\gamma}$ representing an element $g\in\pu(X,x_0)$.
But then $g=hg_\alpha$ for some $h\in H$ and $\alpha$, 
so $\gamma$ is homotopic rel endpoints to $\eta*\gamma_\alpha$ for some loop
$\eta$ representing $h$. Lifting these, based at $\wtl{x}_0$, then by homotopy
lifting, $\wtl{\gamma}$ is homotopic rel endpoints to $\wtl{\eta}$, which is a 
loop, followed by the lift $\wtl{\gamma}_\alpha$ of $\gamma_\alpha$
starting at $\wtl{x}_0$. So $\wtl{\gamma}(1) = \wtl{\gamma}_\alpha(1)$ .
Therefore, lifts of representatives of coset representatives of $H$ in $G$ give
a 1-to-1 correspondence, given by $\wtl{\gamma}(1)$, with $p\inv({x_0})$.


\ssk

The path lifting property (because $\pi([0,1],0)=\{1\}$) is actually a special
case of a more general \cltr{{\bf lifting criterion}: If 
$p:(\wtl{X},\wtl{x}_0)\ra (X,x_0)$ is a covering map, and 
$f:(Y,y_0)\ra (X,x_0)$ is a map, where
$Y$ is path-connected and locally path-connected, then there is a lift 
$\wtl{f}:(Y,y_0)\ra (\wtl{X},\wtl{x}_0)$ of $f$ (i.e., 
$f=p\circ\wtl{f}$) $\Leftrightarrow$ 
$f_*(\pu(Y,y_0))\sset p_*(\pu(\wtl{X},\wtl{x}_0))$ . 
Furthermore, two lifts of $f$ which agree at a single point are equal.}

\ssk

If the lift exists, then $f=p\circ\wtl{f}$ implies 
$f_*=p_*\circ\wtl{f}_*$, so 
$f_*(\pu(Y,y_0)) = p_*(\wtl{f}_*(\pu(Y,y_0)))\sset p_*(\pu(\wtl{X},\wtl{x}_0))$ , as desired.
Conversely, if $f_*(\pu(Y,y_0))\sset p_*(\pu(\wtl{X},\wtl{x}_0))$,
 we will {\it use} path lifting to build the lift. \clto{Given $y\in Y$,
choose a path $\gamma$ in $Y$ from $y_0$ to $y$ and use path
lifting in $X$ to lift the path $f\circ\gamma$ to a path $\wtl{f\circ\gamma}$ with 
$\wtl{f\circ\gamma}(0)=\wtl{x}_0$ . Then define
$\wtl{f}(y)=\wtl{f\circ\gamma}(1)$ .} If we show that
this is well-defined and continuous, it is our required lift, 
since $(p\circ\wtl{f})(y) = p(\wtl{f}(y))=p(\wtl{f\circ\gamma}(1))
=p\circ\wtl{f\circ\gamma})(1) = (f\circ\gamma)(1) = f(\gamma(1)) = f(y)$. 
But if $\eta$ is any other path from 
$y_0$ to $y$, then $\gamma*\overline{\eta}$ is a loop in $Y$, 
so $f\circ(\gamma*\overline{\eta})=(f\circ\gamma)*\overline{(f\circ\eta)}$
is a loop in $X$ representing an element of 
$f_*(\pu(Y,y_0))\sset p_*(\pu(\wtl{X},\wtl{x}_0))$, and
so lifts to a loop in $\wtl{X}$ based at $\wtl{x}_0$.
Consequently, $f\circ\gamma$ and $f\circ\eta$ lift to paths
starting at $\wtl{x}_0$ with the same value at 1. So $\wtl{f}$ is
well-defined. For continuity, we use the 
evenly covered property of $p$. Given $y\in Y$,
and  a neighborhood $\wtl{\Cal U}$ of 
$\wtl{f}(y)$ in $\wtl{X}$, 
we wish to find a nbhd ${\Cal V}$ of $y$ with 
$\wtl{f}({\Cal V})\sset\wtl{\Cal U}$. Choosing an evenly covered 
neighborhood ${\Cal U}_y$ for $f(y)$, take the sheet 
$\wtl{\Cal U}_y$ over ${\Cal U}_y$ which contains $\wtl{f}(y)$,
and set ${\Cal W}=\wtl{\Cal U}\cap \wtl{\Cal U}_y$. This is open in 
$\wtl{X}$, and $p$ is a homeo from this 
set to the open set $p({\Cal W})\sset X$. Setting ${\Cal V}^\prime= f\inv(p({\Cal W})$
we get an open set containing $y$, and so it contains a path-connected open 
set ${\Cal V}$ containing $y$. Then for every point $z\in {\Cal V}$ we build a path
$\gamma$
from $y_0$ to $z$ by concatenating a path from $y_0$ to $y$ with a path {\it in} ${\Cal V}$
from $y$ to $z$; by unique path lifting, 
since $f({\Cal V}\sset {\Cal U}_y$ , $f\circ\gamma$ lifts to 
the concatenation of a path from $\wtl{x}_0$ to $\wtl{f}(y)$ and a 
path {\it in} $\wtl{\Cal U}_y$ from $\wtl{f}(y)$ to $\wtl{f}(z)$.
So $\wtl{f}(z)\in\wtl{\Cal U}$.
Because $\wtl{f}$ is built by lifting paths, which is unique, 
the last statement of the proposition follows.

\msk

{\bf Universal covering spaces}: As we shall see, a particularly
important covering space to identify is one which is simply
connected. First: such a covering is essentially unique. 
If $X$ is locally path-connected, and has two connected, simply connected
coverings $p_1:X_1\ra X$ and $p_2:X_2\ra X$, then choosing
basepoints $x_i, i=0,1,2$ , since 
$p_{1*}(\pu(X_1,x_1)) = p_{2*}(\pu(X_2,x_2))=\{1\}\sset \pu(X,x_0)$,
the lifting criterion with $p_1,p_2$ playing the role of $f$, in turn,
gives us maps $\wtl{p}_1:(X_1,x_1)\ra (X_2,x_2)$ and 
$\wtl{p}_2:(X_2,x_2)\ra (X_1,x_1)$ with $p_2\circ\wtl{p}_1=p_1$
and $p_1\circ\wtl{p}_2=p_2$. Consequently, 
$p_2\circ\wtl{p}_1\circ \wtl{p}_2 = p_1\circ\wtl{p}_2=p_2$
and similarly, 
$p_1\circ\wtl{p}_2\circ \wtl{p}_1 =p_2\circ\wtl{p}_1=p_1$.
So $\wtl{p}_1\circ \wtl{p}_2:(X_2,x_2)\ra (X_2,x_2)$, for example,
is a lift of $p_2$ to the covering map $p_2$. But so is the identity map! By
uniqueness, $\wtl{p}_1\circ \wtl{p}_2=Id$ . Similarly,
$\wtl{p}_2\circ \wtl{p}_1=Id$. So $(X_1,x_1)$ and $(X_2,x_2)$ 
are homeomorphic. \clto{So a space can have
only one connected, simply-connected covering space. It is known
as the {\it universal covering} of $X$. }

\ssk

Not every (locall path-connected) space $X$ has a universal covering; a 
(further) \cltp{necessary condition is that $X$ be {\it semi-locally simply connected} (S-LSC).}
The idea is that If $p:\wtl{X}\ra X$ is the universal cover, then for every 
point $x\in X$, we have an evenly-covered neighborhood ${\Cal U}$ of $x$.
The inclusion $i:{\Cal U}\ra X$, by definition, lifts to $\wtl{X}$, so
$i_*(\pu({\Cal U},x))\sset p_*(\pu(\wtl{X},\wtl{x}) = \{1\}$, so
$i_*$ is the trivial map. Consequently, every loop in ${\Cal U}$ is 
null-homotopic in $X$. This is S-LSC;
\cltp{every point has a neighborhood whose inclusion-induced homomorphism
is trivial.} Not all spaces have this property; the most famous is the 
Hawaiian earrings 
$\displaystyle X=\cup_{n}\{x\in {\Bbb R}^2 :  ||x-(1/n,0)||=1/n\}$ .
The point $(0,0)$ has no such neighborhood. 

\vfill
\end

