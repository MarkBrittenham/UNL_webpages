

\magnification=1200
\overfullrule=0pt
\parindent=0pt

\nopagenumbers

\input amstex

\voffset=-.6in
\hoffset=-.5in
\hsize = 7.5 true in
\vsize=10.4 true in

%\voffset=1.4in
%\hoffset=-.5in
%\hsize = 10.2 true in
%\vsize=8 true in

\input colordvi

\def\cltr{\Red}		  % Red  VERY-Approx PANTONE RED
\def\cltb{\Blue}		  % Blue  Approximate PANTONE BLUE-072
\def\cltg{\PineGreen}	  % ForestGreen  Approximate PANTONE 349
\def\cltp{\DarkOrchid}	  % DarkOrchid  No PANTONE match
\def\clto{\Orange}	  % Orange  Approximate PANTONE ORANGE-021
\def\cltpk{\CarnationPink}	  % CarnationPink  Approximate PANTONE 218
\def\clts{\Salmon}	  % Salmon  Approximate PANTONE 183
\def\cltbb{\TealBlue}	  % TealBlue  Approximate PANTONE 3145
\def\cltrp{\RoyalPurple}	  % RoyalPurple  Approximate PANTONE 267
\def\cltp{\Purple}	  % Purple  Approximate PANTONE PURPLE

\def\cgy{\GreenYellow}     % GreenYellow  Approximate PANTONE 388
\def\cyy{\Yellow}	  % Yellow  Approximate PANTONE YELLOW
\def\cgo{\Goldenrod}	  % Goldenrod  Approximate PANTONE 109
\def\cda{\Dandelion}	  % Dandelion  Approximate PANTONE 123
\def\capr{\Apricot}	  % Apricot  Approximate PANTONE 1565
\def\cpe{\Peach}		  % Peach  Approximate PANTONE 164
\def\cme{\Melon}		  % Melon  Approximate PANTONE 177
\def\cyo{\YellowOrange}	  % YellowOrange  Approximate PANTONE 130
\def\coo{\Orange}	  % Orange  Approximate PANTONE ORANGE-021
\def\cbo{\BurntOrange}	  % BurntOrange  Approximate PANTONE 388
\def\cbs{\Bittersweet}	  % Bittersweet  Approximate PANTONE 167
%\def\creo{\RedOrange}	  % RedOrange  Approximate PANTONE 179
\def\cma{\Mahogany}	  % Mahogany  Approximate PANTONE 484
\def\cmr{\Maroon}	  % Maroon  Approximate PANTONE 201
\def\cbr{\BrickRed}	  % BrickRed  Approximate PANTONE 1805
\def\crr{\Red}		  % Red  VERY-Approx PANTONE RED
\def\cor{\OrangeRed}	  % OrangeRed  No PANTONE match
\def\paru{\RubineRed}	  % RubineRed  Approximate PANTONE RUBINE-RED
\def\cwi{\WildStrawberry}  % WildStrawberry  Approximate PANTONE 206
\def\csa{\Salmon}	  % Salmon  Approximate PANTONE 183
\def\ccp{\CarnationPink}	  % CarnationPink  Approximate PANTONE 218
\def\cmag{\Magenta}	  % Magenta  Approximate PANTONE PROCESS-MAGENTA
\def\cvr{\VioletRed}	  % VioletRed  Approximate PANTONE 219
\def\parh{\Rhodamine}	  % Rhodamine  Approximate PANTONE RHODAMINE-RED
\def\cmu{\Mulberry}	  % Mulberry  Approximate PANTONE 241
\def\parv{\RedViolet}	  % RedViolet  Approximate PANTONE 234
\def\cfu{\Fuchsia}	  % Fuchsia  Approximate PANTONE 248
\def\cla{\Lavender}	  % Lavender  Approximate PANTONE 223
\def\cth{\Thistle}	  % Thistle  Approximate PANTONE 245
\def\corc{\Orchid}	  % Orchid  Approximate PANTONE 252
\def\cdo{\DarkOrchid}	  % DarkOrchid  No PANTONE match
\def\cpu{\Purple}	  % Purple  Approximate PANTONE PURPLE
\def\cpl{\Plum}		  % Plum  VERY-Approx PANTONE 518
\def\cvi{\Violet}	  % Violet  Approximate PANTONE VIOLET
\def\clrp{\RoyalPurple}	  % RoyalPurple  Approximate PANTONE 267
\def\cbv{\BlueViolet}	  % BlueViolet  Approximate PANTONE 2755
\def\cpe{\Periwinkle}	  % Periwinkle  Approximate PANTONE 2715
\def\ccb{\CadetBlue}	  % CadetBlue  Approximate PANTONE (534+535)/2
\def\cco{\CornflowerBlue}  % CornflowerBlue  Approximate PANTONE 292
\def\cmb{\MidnightBlue}	  % MidnightBlue  Approximate PANTONE 302
\def\cnb{\NavyBlue}	  % NavyBlue  Approximate PANTONE 293
\def\crb{\RoyalBlue}	  % RoyalBlue  No PANTONE match
%\def\cbb{\Blue}		  % Blue  Approximate PANTONE BLUE-072
\def\cce{\Cerulean}	  % Cerulean  Approximate PANTONE 3005
\def\ccy{\Cyan}		  % Cyan  Approximate PANTONE PROCESS-CYAN
\def\cpb{\ProcessBlue}	  % ProcessBlue  Approximate PANTONE PROCESS-BLUE
\def\csb{\SkyBlue}	  % SkyBlue  Approximate PANTONE 2985
\def\ctu{\Turquoise}	  % Turquoise  Approximate PANTONE (312+313)/2
\def\ctb{\TealBlue}	  % TealBlue  Approximate PANTONE 3145
\def\caq{\Aquamarine}	  % Aquamarine  Approximate PANTONE 3135
\def\cbg{\BlueGreen}	  % BlueGreen  Approximate PANTONE 320
\def\cem{\Emerald}	  % Emerald  No PANTONE match
%\def\cjg{\JungleGreen}	  % JungleGreen  Approximate PANTONE 328
\def\csg{\SeaGreen}	  % SeaGreen  Approximate PANTONE 3268
\def\cgg{\Green}	  % Green  VERY-Approx PANTONE GREEN
\def\cfg{\ForestGreen}	  % ForestGreen  Approximate PANTONE 349
\def\cpg{\PineGreen}	  % PineGreen  Approximate PANTONE 323
\def\clg{\LimeGreen}	  % LimeGreen  No PANTONE match
\def\cyg{\YellowGreen}	  % YellowGreen  Approximate PANTONE 375
\def\cspg{\SpringGreen}	  % SpringGreen  Approximate PANTONE 381
\def\cog{\OliveGreen}	  % OliveGreen  Approximate PANTONE 582
\def\pars{\RawSienna}	  % RawSienna  Approximate PANTONE 154
\def\cse{\Sepia}		  % Sepia  Approximate PANTONE 161
\def\cbr{\Brown}		  % Brown  Approximate PANTONE 1615
\def\cta{\Tan}		  % Tan  No PANTONE match
\def\cgr{\Gray}		  % Gray  Approximate PANTONE COOL-GRAY-8
\def\cbl{\Black}		  % Black  Approximate PANTONE PROCESS-BLACK
\def\cwh{\White}		  % White  No PANTONE match


\loadmsbm

\input epsf

\def\ctln{\centerline}
\def\u{\underbar}
\def\ssk{\smallskip}
\def\msk{\medskip}
\def\bsk{\bigskip}
\def\hsk{\hskip.1in}
\def\hhsk{\hskip.2in}
\def\dsl{\displaystyle}
\def\hskp{\hskip1.5in}

\def\lra{$\Leftrightarrow$ }
\def\ra{\rightarrow}
\def\mpto{\logmapsto}
\def\pu{\pi_1}
\def\mpu{$\pi_1$}
\def\sig{\Sigma}
\def\msig{$\Sigma$}
\def\ep{\epsilon}
\def\sset{\subseteq}
\def\del{\partial}
\def\inv{^{-1}}
\def\wtl{\widetilde}
\def\lra{\Leftrightarrow}
\def\del{\partial}
\def\delp{\partial^\prime}
\def\delpp{\partial^{\prime\prime}}



\ctln{\bf Math 971 Algebraic Topology}

\ssk

\ctln{March 22, 2005}

\msk

Perhaps the most important property of the fundamental group is that a continuouos map 
between spaces induces a homomorphism between groups. (This implied, for instance,
that homeomorphic spaces have isomorphic \mpu ). The same is true for homology groups, 
for essentially the same reason. Given a map $f:X\ra Y$, there is an induced map $f_\#:C_n(X)\ra C_n(Y)$
defined by postcomposition; for a singular simplex $\sigma$, $f_\#(\sigma) = f\circ\sigma$, and we extend
the map linearly. Since $f\circ(g|_A) = (f\circ g)|_A$ (postcomposition commutes with restriction of the domain),
$f_\#$ commutes with $\del$ : $f_\#(\del \sigma) = \del(f_\#(\sigma))$. A homomorphism between
chain complexes (i.e., a sequence of such maps, one for each chain group) which commutes with the 
boundaries maps in this way, is called a {\it chain map}.
A chain map, such as $f_\#$, therefore, takes cycles to cycles,
and boundaries to boundaries, and so $f_\#:Z_i(X)\ra Z_i(Y)$ (which is linear, hence a homomorphism)
induces a homomorphism $f_*:H_i(X)\ra H_i(Y)$ by $f_*[z] = [f_\#(z)]$ . 
Since it is defined by composition with singular simplices, it is 
immediate that, for a map $g:Y\ra Z$, $(g\circ f)_*=g_*\circ f_*$ . And since the identity map $I:X\ra X$
satisfies $I_\#=Id$, so $I_*=Id$, homeomorphic spaces have isomorphic homology groups.

\msk

Another important property of \mpu\ is that homotopic maps give the same
induced map (after correcting for basepoints). This is also true for homology;
if $f\sim g:X\ra Y$, then $f_*=g_*$ . The proof, however, is not quite as straightforward
as for homotopy. And it requires some new technology; the chain homotopy.
A chain homotopy $H$ between the chain complexes $f_\#,g_\#:C_*(X)\ra C_*(Y)$ 
is a sequence of homomorphisms $H_i:C_i(X)\ra C_{i+1}(Y)$ satisfying
$H_{i-1}\del_i+\del_{i+1}H_i = f_\#-g_\#:C_i(X)\ra C_i(Y)$ . The existence of $H$
implies that $f_*=g_*$, since for an $i$-cycle $z$ (with $\del_i(z)=0$) we have

$f_*[z]-g_*[z] = [f_\#(z)-g_\#(z)] = [H_{i-1}\del_i(z)+\del_{i+1}H_i(z)] = [H_{i-1}(0)+\del_{i+1}(w)]
=[\del_{i+1}(w)]=0$.

And the existence of a homotopy between $f$ and $g$ implies the existence of a 
chain homotopy between $f_\#$ and $g_\#$ . This is because the homotopy 
gives a map $H:X\times I\ra Y$, which induces a map $H_\#:C_{i+1}(X\times I)\ra C_{i+1}(Y)$ .
Then we pull, from our back pocket, a {\it prism map}
$P:C_i(X)\ra C_{i+1}(X\times I)$; the composition $H_\#\circ P$ will be our chain homotopy.
The prism map  takes a (singular) $i$-simplex $\sigma$ and sends it to a sum of singular $(i+1)$-simplices
in $X\times I$. and the way we define it is to take the $i$-simplex $\Delta^i$, and taking it 
to $\Delta^i\times I$ (i.e., a {\it prism}), and thinking of this as a sum of $(i+1)$-simplices. Using the
map $\sigma^\prime = \sigma\times Id : \Delta^i\times I\ra X\times I$ 
restricted to each of these $(i+1)$-simplices
yields the prism map. Now, there are many ways of decomposing a prism into simplices,
but we need to be careful to choose one which restricts well to each of 
the \u{faces} of $\Delta^i$,
in order to get the chain homotopy property we require. In the end, what 
this requires is that the
decomposition, when restricted to any face of $\Delta^i$ (which we think of as a copy
of $\Delta^{i-1}$), is the same as the decomposition we would have applied to a prism over
an $(i-1)$-simplex. After some exploration, we are led to the following formulation.

\msk

If we write $\Delta^n\times\{0\}=[v_0,\ldots ,v_n]$ and 
$\Delta^n\times\{1\}=[w_0,\ldots ,w_n]$, then we can decompose
$\Delta^n\times I$ as the (n+1)-simplices $[v_0,\ldots ,v_i,w_i,\ldots ,w_n]$. 
We then define $P(\sigma) = \sum (-1)^i \sigma^\prime|_{[v_0,\ldots ,v_i,w_i,\ldots ,w_n]}$.
A routine calculation verifies that 
$(\del P+P\del)(\sigma) = \sigma^\prime|_{[w_0,\ldots ,w_n]}-\sigma^\prime|_{[v_0,\ldots v_n]}$ ;
Composing with $H_{\#}$ yields our result.

\msk

Consequently, for example, homotopy equivalent spaces have isomporphic
(reduced) homology groups; homotopy equivalences induce isomorphisms.
So all contractible spaces have trivial reduced homology
in all dimensions, since they are all homotopy to a point. If we think of a cell complex as
a collection of disks glued together, this lends some hope that we can compute
their homology groups, since we can compute the homology of the building blocks. 
Our next goal is to make turn this idea into action; but we need another tool, to
frame our answer in the best way possible.

\msk

{\bf Exact sequences:} Most of the fundamental properties of homology groups
are described in terms of exact sequences. A sequence of homomorphisms
\hhsk $\cdots {f_{n+1}\atop\ra} A_n {f_n\atop\ra} A_{n-1} {f_{n-1}\atop\ra} a_{n-2} 
\ra \cdots$ \hhsk
of abelian groups is called {\it exact} if im$(f_n)=\ker(f_{n-1})$ 
for every $n$. In most cases,
we get the most mileage out of an exact sequence when some of the groups
(or more generally, som eof the maps)
are trivial; $0\ra A {f\atop\ra}B$ is exact $\lra$ $f$ is injective (and the same of $A$ receives
the $0$ map), and 
$A {f\atop\ra}B\ra 0$ is exact $\lra$ $f$ is surjective (and the same if the map with domain $B$ is the $0$ map). 
An exact sequence $0\ra A {\ra}B\ra C\ra 0$ is called a {\it short exact sequence}.







\vfill
\end