

\magnification=1200
\overfullrule=0pt
\parindent=0pt

\nopagenumbers

\input amstex

\voffset=-.6in
\hoffset=-.5in
\hsize = 7.5 true in
\vsize=10.4 true in

%\voffset=1.4in
%\hoffset=-.5in
%\hsize = 10.2 true in
%\vsize=8 true in

\input colordvi

\def\cltr{\Red}		  % Red  VERY-Approx PANTONE RED
\def\cltb{\Blue}		  % Blue  Approximate PANTONE BLUE-072
\def\cltg{\PineGreen}	  % ForestGreen  Approximate PANTONE 349
\def\cltp{\DarkOrchid}	  % DarkOrchid  No PANTONE match
\def\clto{\Orange}	  % Orange  Approximate PANTONE ORANGE-021
\def\cltpk{\CarnationPink}	  % CarnationPink  Approximate PANTONE 218
\def\clts{\Salmon}	  % Salmon  Approximate PANTONE 183
\def\cltbb{\TealBlue}	  % TealBlue  Approximate PANTONE 3145
\def\cltrp{\RoyalPurple}	  % RoyalPurple  Approximate PANTONE 267
\def\cltp{\Purple}	  % Purple  Approximate PANTONE PURPLE

\def\cgy{\GreenYellow}     % GreenYellow  Approximate PANTONE 388
\def\cyy{\Yellow}	  % Yellow  Approximate PANTONE YELLOW
\def\cgo{\Goldenrod}	  % Goldenrod  Approximate PANTONE 109
\def\cda{\Dandelion}	  % Dandelion  Approximate PANTONE 123
\def\capr{\Apricot}	  % Apricot  Approximate PANTONE 1565
\def\cpe{\Peach}		  % Peach  Approximate PANTONE 164
\def\cme{\Melon}		  % Melon  Approximate PANTONE 177
\def\cyo{\YellowOrange}	  % YellowOrange  Approximate PANTONE 130
\def\coo{\Orange}	  % Orange  Approximate PANTONE ORANGE-021
\def\cbo{\BurntOrange}	  % BurntOrange  Approximate PANTONE 388
\def\cbs{\Bittersweet}	  % Bittersweet  Approximate PANTONE 167
%\def\creo{\RedOrange}	  % RedOrange  Approximate PANTONE 179
\def\cma{\Mahogany}	  % Mahogany  Approximate PANTONE 484
\def\cmr{\Maroon}	  % Maroon  Approximate PANTONE 201
\def\cbr{\BrickRed}	  % BrickRed  Approximate PANTONE 1805
\def\crr{\Red}		  % Red  VERY-Approx PANTONE RED
\def\cor{\OrangeRed}	  % OrangeRed  No PANTONE match
\def\paru{\RubineRed}	  % RubineRed  Approximate PANTONE RUBINE-RED
\def\cwi{\WildStrawberry}  % WildStrawberry  Approximate PANTONE 206
\def\csa{\Salmon}	  % Salmon  Approximate PANTONE 183
\def\ccp{\CarnationPink}	  % CarnationPink  Approximate PANTONE 218
\def\cmag{\Magenta}	  % Magenta  Approximate PANTONE PROCESS-MAGENTA
\def\cvr{\VioletRed}	  % VioletRed  Approximate PANTONE 219
\def\parh{\Rhodamine}	  % Rhodamine  Approximate PANTONE RHODAMINE-RED
\def\cmu{\Mulberry}	  % Mulberry  Approximate PANTONE 241
\def\parv{\RedViolet}	  % RedViolet  Approximate PANTONE 234
\def\cfu{\Fuchsia}	  % Fuchsia  Approximate PANTONE 248
\def\cla{\Lavender}	  % Lavender  Approximate PANTONE 223
\def\cth{\Thistle}	  % Thistle  Approximate PANTONE 245
\def\corc{\Orchid}	  % Orchid  Approximate PANTONE 252
\def\cdo{\DarkOrchid}	  % DarkOrchid  No PANTONE match
\def\cpu{\Purple}	  % Purple  Approximate PANTONE PURPLE
\def\cpl{\Plum}		  % Plum  VERY-Approx PANTONE 518
\def\cvi{\Violet}	  % Violet  Approximate PANTONE VIOLET
\def\clrp{\RoyalPurple}	  % RoyalPurple  Approximate PANTONE 267
\def\cbv{\BlueViolet}	  % BlueViolet  Approximate PANTONE 2755
\def\cpe{\Periwinkle}	  % Periwinkle  Approximate PANTONE 2715
\def\ccb{\CadetBlue}	  % CadetBlue  Approximate PANTONE (534+535)/2
\def\cco{\CornflowerBlue}  % CornflowerBlue  Approximate PANTONE 292
\def\cmb{\MidnightBlue}	  % MidnightBlue  Approximate PANTONE 302
\def\cnb{\NavyBlue}	  % NavyBlue  Approximate PANTONE 293
\def\crb{\RoyalBlue}	  % RoyalBlue  No PANTONE match
%\def\cbb{\Blue}		  % Blue  Approximate PANTONE BLUE-072
\def\cce{\Cerulean}	  % Cerulean  Approximate PANTONE 3005
\def\ccy{\Cyan}		  % Cyan  Approximate PANTONE PROCESS-CYAN
\def\cpb{\ProcessBlue}	  % ProcessBlue  Approximate PANTONE PROCESS-BLUE
\def\csb{\SkyBlue}	  % SkyBlue  Approximate PANTONE 2985
\def\ctu{\Turquoise}	  % Turquoise  Approximate PANTONE (312+313)/2
\def\ctb{\TealBlue}	  % TealBlue  Approximate PANTONE 3145
\def\caq{\Aquamarine}	  % Aquamarine  Approximate PANTONE 3135
\def\cbg{\BlueGreen}	  % BlueGreen  Approximate PANTONE 320
\def\cem{\Emerald}	  % Emerald  No PANTONE match
%\def\cjg{\JungleGreen}	  % JungleGreen  Approximate PANTONE 328
\def\csg{\SeaGreen}	  % SeaGreen  Approximate PANTONE 3268
\def\cgg{\Green}	  % Green  VERY-Approx PANTONE GREEN
\def\cfg{\ForestGreen}	  % ForestGreen  Approximate PANTONE 349
\def\cpg{\PineGreen}	  % PineGreen  Approximate PANTONE 323
\def\clg{\LimeGreen}	  % LimeGreen  No PANTONE match
\def\cyg{\YellowGreen}	  % YellowGreen  Approximate PANTONE 375
\def\cspg{\SpringGreen}	  % SpringGreen  Approximate PANTONE 381
\def\cog{\OliveGreen}	  % OliveGreen  Approximate PANTONE 582
\def\pars{\RawSienna}	  % RawSienna  Approximate PANTONE 154
\def\cse{\Sepia}		  % Sepia  Approximate PANTONE 161
\def\cbr{\Brown}		  % Brown  Approximate PANTONE 1615
\def\cta{\Tan}		  % Tan  No PANTONE match
\def\cgr{\Gray}		  % Gray  Approximate PANTONE COOL-GRAY-8
\def\cbl{\Black}		  % Black  Approximate PANTONE PROCESS-BLACK
\def\cwh{\White}		  % White  No PANTONE match


\loadmsbm

\input epsf

\def\ctln{\centerline}
\def\u{\underbar}
\def\ssk{\smallskip}
\def\msk{\medskip}
\def\bsk{\bigskip}
\def\hsk{\hskip.1in}
\def\hhsk{\hskip.2in}
\def\dsl{\displaystyle}
\def\hskp{\hskip1.5in}

\def\lra{$\Leftrightarrow$ }
\def\ra{\rightarrow}
\def\mpto{\logmapsto}
\def\pu{\pi_1}
\def\mpu{$\pi_1$}
\def\sig{\Sigma}
\def\msig{$\Sigma$}
\def\ep{\epsilon}
\def\sset{\subseteq}
\def\del{\partial}
\def\inv{^{-1}}
\def\wtl{\widetilde}
\def\lra{\Leftrightarrow}
\def\del{\partial}
\def\delp{\partial^\prime}
\def\delpp{\partial^{\prime\prime}}
\def\sgn{{\roman{sgn}}}
\def\wtih{\widetilde{H}}
\def\bbz{{\Bbb Z}}
\def\bbr{{\Bbb R}}



\ctln{\bf Math 971 Algebraic Topology}

\ssk

\ctln{April 7, 2005}

\msk

{\bf Homology on ``small'' chains = singular homology:} \hsk
The point to all of these calculations was that if $\{{\Cal U}_\alpha\}$ is an open cover of $X$, then the 
inclusions $i_n:C_n^{\Cal U}(X)\ra C_n(X)$ induce isomorphisms on homology. This gives us two
big theorems. The first is

\msk

{\bf Mayer-Vietoris Sequence}: If $X={\Cal U}\cup{\Cal V}$ is the union of two open sets, then
the short exact sequences \hhsk 
$0\ra C_n({\Cal U}\cap {\Cal V}) \ra C_n({\Cal U})\oplus C_n({\Cal V}) \ra C_n^{\{ {\Cal U},{\Cal V}\}}(X)\ra 0$
\hhsk , together with the isomorphism above, give the long exact sequence

\ctln{$\cdots \ra H_n({\Cal U}\cap {\Cal V}) {\buildrel{(i_{{\Cal U}*},-i_{{\Cal V}*})}\over\ra}
H_n({\Cal U})\oplus H_n({\Cal V}) {\buildrel{j_{{\Cal U}*}+j_{{\Cal V}*}}\over\ra}H_n(X)
{\buildrel \del\over\ra} H_{n-1}({\Cal U}\cap {\Cal V}) \ra \cdots$}

\msk

And just like Seifert - van Kampen, we can replace open sets by sets $A,B$ having neigborhoods which deformation
retract to them, and whose intersection deformation retracts to $A\cap B$. For example,
subcomplexes $A,B\sset X$ of a CW-complex, with $A\cup B = X$ have homology satisfying a long
exact sequence

\ssk

\ctln{$\cdots \ra H_n(A\cap B) {\buildrel{(i_{A*},-i_{B*})}\over\ra}
H_n(A)\oplus H_n(B) {\buildrel{j_{A*}+j_{B*}}\over\ra}H_n(X)
{\buildrel \del\over\ra} H_{n-1}(A\cap B) \ra \cdots$}

\ssk

And this is also true for reduced homology; we just augment the chain complexes used above with the 
short exact sequence \hhsk $0\ra {\Bbb Z}\ra {\Bbb Z}\oplus {\Bbb Z} \ra {\Bbb Z} \ra 0$ , 
where the first non-trivial map is $a\mapsto (a,-a)$ and the second is $(a,b)\mapsto a+b$ .

\msk

And now we can do some meaningful calculations! 
The basic idea is that if we know the homology of the pieces $A,B,A\cap B$, and something
about the inclusion-induced homomorphisms in the long
exact sequence, then we can deduce information abut the
homology of $X$. A few examples will probably illustrate this best.

\msk

An $n$-sphere $S^n$
is the union $S^n_+\cup S^n_-$ of its upper and lower hemispheres, each of which 
is contractible, and have intersection $S^n_+\cap S^n_-=S^{n-1}_0$ the equatorial
$(n-1)$-sphere. So Mayer-Vietoris gives us the exact sequence

\hhsk $\cdots \ra \widetilde{H}_k(S^n_+)\oplus \widetilde{H}_k(S^n_-) \ra \widetilde{H}_k(S^n)
\ra \widetilde{H}_{k-1}(S^{n-1}_0) \ra \widetilde{H}_{k-1}(S^n_+)\oplus \widetilde{H}_{k-1}(S^n_-)\ra \cdots$ 
\hhsk , i.e, \hhsk

$0 \ra \widetilde{H}_k(S^n)\ra \widetilde{H}_{k-1}(S^{n-1}_0) \ra  0$ \hhsk 
i.e., $\widetilde{H}_k(S^n)\cong \widetilde{H}_{k-1}(S^{n-1})$ for every $k$ and $n$. 
So by induction, 

\ssk

\ctln{$\widetilde{H}_k(S^n)\cong\widetilde{H}_{k-n}(S^0)\cong    
\cases
{\Bbb Z}, & if $k=n$\cr
0, & otherwise $ $\cr 
\endcases$}

\bsk

The 2-torus $T^2=S^1\times S^1$ can be thought of as the union of two copies
of an annulus $S^1\times I$, glued together along their (pair of)
boundary circles. The resulting long exact homology sequence

\ssk

$\cdots \ra \wtih_2(S^1\times I)\oplus \wtih_2(S^1\times I)\ra \wtih_2(T^2)
\ra \wtih_1(S^1\coprod S^1) \ra \wtih_1(S^1\times I)\oplus \wtih_1(S^1\times I)
\ra$

\hfill $\wtih_1(T^2) \ra \wtih_0(S^1\coprod S^1) \ra 
\wtih_0(S^1\times I)\oplus \wtih_0(S^1\times I) \ra \cdots$

\ssk

which renders as

\ssk

\ctln{$0\ra \wtih_2(T^2) \ra \bbz\oplus\bbz {\buildrel \varphi \over \ra} 
\bbz\oplus\bbz \ra \wtih_1(T^2)
\ra \bbz \ra 0$}

\ssk

In order to determine the unknown homology groups, we need to know
more about the first map $\bbz\oplus\bbz \ra \bbz\oplus\bbz$. The first
group has generators consisting of the generators of each of the 
$S^1$ path components of $A\cap B$
(represented by the singular 1-simplex wrapping exactly once around the
circle), and are each mapped to a
generator for each of the $S^1\times I$. Remembering that $\varphi$ 
was chosen to be $(i_{A*},-i_{B*})$, we find that $\varphi$ is
represented by the matrix $\pmatrix 1&1\\ -1&-1\\ \endpmatrix$ , 
which has image spanned by $[ 1 , 1]^T$ and kernel
spanned by $[1 , 1]^T$. By using
exactness and a few Noether isomorphism theorems, we can cut up our
LES above as

\ssk

\ctln{$0\ra \wtih_2(T^2) \ra \ker \varphi \ra 0$ and
$0\ra (\bbz\oplus\bbz)/{\roman im}$ $\varphi \ra \wtih_1(T^2)
\ra \bbz \ra 0$}

\ssk

(since the first map is onto its image, and the second to last map
is injective, once we mod out by its kernel). The first implies that
$\wtih_2(T^2) \cong \bbz$, and the second (since our basis for the
image extends to a basis for $\bbz^2$) becomes 
$0\ra \bbz \ra \wtih_1(T^2) \ra \bbz \ra 0$ . This implies that
$\wtih_2(T^2) \cong \bbz^2$, because of the 

\msk

{\bf Fact:} if 
$0\ra K {\buildrel \varphi \over \ra} 
G {\buildrel \psi \over \ra} H \ra 0$ is exact and 
there is a homomorphism $\rho : H\ra G$ with
$\psi \rho = $Id , then $G\cong K\times H$ .
The {\bf proof} consists of defining $\sigma: K\times H \ra G$
by $\sigma(k,h) = \varphi(k) + \rho(h)$. As the sum of two homomorphisms
it is a homomorphism. If $\sigma(k,h) = \varphi(k) + \rho(h) = 0$
then $0 = \psi\sigma(k,h) = \psi\varphi(k) + \psi\rho(h) = 0 + h = h$,
so $0=\sigma(k,h) = \varphi(k) + \rho(h) = \varphi(k)$, so 
$k-0$ by the injectivity of $\varphi$. So $(k,h) = (0,0)$ . 
For surjectivity, given $g\in G$, let $h=\psi(g)$; then $\psi(g-\rho h) =
\psi g - \psi\rho h = h-h = 0$, so there is a 
$k\in K$ with $\varphi k = g-\rho h$ , so $\sigma(k,h) = \varphi k +\rho h = g$.

\ssk

[This is just on of a set of results like this; in this instance we say
that the short exact sequence {\it splits} or is {\it split exact};
this existence of the map $\rho$ is one sufficient condition.

\msk

Consequently, $\wtih_i(T^2) = $ $\bbz$ for $i=2$, $\bbz^2$ for $i=1$, and
$0$ for all other $i$ (since $T^2$ is path-connected, and for $i\geq 3$,
our LES reads $\ra \wtih_i(T^2) \ra 0$ ).

\msk

The computation for the Klein bottle $K^2$ is similar; it can be expressed as
a pair of annuli $S^1\times I$ glued along their boundaries, but one of the 
gluings is by a reflection. The associated inclusion-induced homomorphism,
in exactly one case, is $-$Id, not Id; and so the resulting matrix, for one choice
of generators, is $\pmatrix 1&1\\ -1&1\\ \endpmatrix$ . After row and column
reducton, this becomes $\pmatrix 1&0\\ 0&2\\ \endpmatrix$ . This matrix has no kernel,
so, using the same cutting up process, 
\hhsk
$0\ra \wtih_2(K^2) \ra \ker \varphi \ra 0$ and
$0\ra (\bbz\oplus\bbz)/{\roman im}\varphi \ra \wtih_1(K^2)
\ra \bbz \ra 0$
\hhsk
becomes 
\hhsk
$0\ra \wtih_2(K^2) \ra 0$ and
$0\ra \bbz_2 \ra \wtih_1(K^2)
\ra \bbz \ra 0$
\hhsk
so $\wtih_2(K^2)=0$ and $\wtih_1(K^2)\cong bbz\oplus \bbz_2$ . As before, 
all other (reduced) homology groups are $0$.

\msk

For the real projective plane $P^2$, we can express it as a M\"obius band 
$M$ with a disk $D$ glued to its boundary. Their intersection is a circle $S^1$.
Writing the Mayer-Vietoris sequence in this situation gives

\ssk

\ctln{$0\ra \wtih_2(P^2) \ra \bbz \ra 0\oplus \bbz \ra \wtih_1(P^2) \ra 0$}

\ssk

Again we need to know more about the middle map $i_*:\wtih_1(S^1)\ra \wtih_1(M)$
in order to determine the unknown groups. $M$ deformation retracts to its
central circle, and the generator for $wtih_1(\del M)$, wrapping once around
$\del M$, is sent to a map wrapping twice around the core circle, and so 
represents twice the generator of $\wtih_1(M)$ . So the middle map is
injective, with image $2\bbz$ . And so $\wtih_2(P^2)=0$, and 
$\wtih(P^2)\cong \bbz/$im$(i_*) \cong \bbz_2$ . All other groups, as before, are
$0$.





\vfill
\end