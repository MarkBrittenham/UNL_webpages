

\magnification=1200
\overfullrule=0pt
\parindent=0pt

\nopagenumbers

\input amstex

\voffset=-.6in
\hoffset=-.5in
\hsize = 7.5 true in
\vsize=10.4 true in

%\voffset=1.4in
%\hoffset=-.5in
%\hsize = 10.2 true in
%\vsize=8 true in

\input colordvi

\def\cltr{\Red}		  % Red  VERY-Approx PANTONE RED
\def\cltb{\Blue}		  % Blue  Approximate PANTONE BLUE-072
\def\cltg{\PineGreen}	  % ForestGreen  Approximate PANTONE 349
\def\cltp{\DarkOrchid}	  % DarkOrchid  No PANTONE match
\def\clto{\Orange}	  % Orange  Approximate PANTONE ORANGE-021
\def\cltpk{\CarnationPink}	  % CarnationPink  Approximate PANTONE 218
\def\clts{\Salmon}	  % Salmon  Approximate PANTONE 183
\def\cltbb{\TealBlue}	  % TealBlue  Approximate PANTONE 3145
\def\cltrp{\RoyalPurple}	  % RoyalPurple  Approximate PANTONE 267
\def\cltp{\Purple}	  % Purple  Approximate PANTONE PURPLE

\def\cgy{\GreenYellow}     % GreenYellow  Approximate PANTONE 388
\def\cyy{\Yellow}	  % Yellow  Approximate PANTONE YELLOW
\def\cgo{\Goldenrod}	  % Goldenrod  Approximate PANTONE 109
\def\cda{\Dandelion}	  % Dandelion  Approximate PANTONE 123
\def\capr{\Apricot}	  % Apricot  Approximate PANTONE 1565
\def\cpe{\Peach}		  % Peach  Approximate PANTONE 164
\def\cme{\Melon}		  % Melon  Approximate PANTONE 177
\def\cyo{\YellowOrange}	  % YellowOrange  Approximate PANTONE 130
\def\coo{\Orange}	  % Orange  Approximate PANTONE ORANGE-021
\def\cbo{\BurntOrange}	  % BurntOrange  Approximate PANTONE 388
\def\cbs{\Bittersweet}	  % Bittersweet  Approximate PANTONE 167
%\def\creo{\RedOrange}	  % RedOrange  Approximate PANTONE 179
\def\cma{\Mahogany}	  % Mahogany  Approximate PANTONE 484
\def\cmr{\Maroon}	  % Maroon  Approximate PANTONE 201
\def\cbr{\BrickRed}	  % BrickRed  Approximate PANTONE 1805
\def\crr{\Red}		  % Red  VERY-Approx PANTONE RED
\def\cor{\OrangeRed}	  % OrangeRed  No PANTONE match
\def\paru{\RubineRed}	  % RubineRed  Approximate PANTONE RUBINE-RED
\def\cwi{\WildStrawberry}  % WildStrawberry  Approximate PANTONE 206
\def\csa{\Salmon}	  % Salmon  Approximate PANTONE 183
\def\ccp{\CarnationPink}	  % CarnationPink  Approximate PANTONE 218
\def\cmag{\Magenta}	  % Magenta  Approximate PANTONE PROCESS-MAGENTA
\def\cvr{\VioletRed}	  % VioletRed  Approximate PANTONE 219
\def\parh{\Rhodamine}	  % Rhodamine  Approximate PANTONE RHODAMINE-RED
\def\cmu{\Mulberry}	  % Mulberry  Approximate PANTONE 241
\def\parv{\RedViolet}	  % RedViolet  Approximate PANTONE 234
\def\cfu{\Fuchsia}	  % Fuchsia  Approximate PANTONE 248
\def\cla{\Lavender}	  % Lavender  Approximate PANTONE 223
\def\cth{\Thistle}	  % Thistle  Approximate PANTONE 245
\def\corc{\Orchid}	  % Orchid  Approximate PANTONE 252
\def\cdo{\DarkOrchid}	  % DarkOrchid  No PANTONE match
\def\cpu{\Purple}	  % Purple  Approximate PANTONE PURPLE
\def\cpl{\Plum}		  % Plum  VERY-Approx PANTONE 518
\def\cvi{\Violet}	  % Violet  Approximate PANTONE VIOLET
\def\clrp{\RoyalPurple}	  % RoyalPurple  Approximate PANTONE 267
\def\cbv{\BlueViolet}	  % BlueViolet  Approximate PANTONE 2755
\def\cpe{\Periwinkle}	  % Periwinkle  Approximate PANTONE 2715
\def\ccb{\CadetBlue}	  % CadetBlue  Approximate PANTONE (534+535)/2
\def\cco{\CornflowerBlue}  % CornflowerBlue  Approximate PANTONE 292
\def\cmb{\MidnightBlue}	  % MidnightBlue  Approximate PANTONE 302
\def\cnb{\NavyBlue}	  % NavyBlue  Approximate PANTONE 293
\def\crb{\RoyalBlue}	  % RoyalBlue  No PANTONE match
%\def\cbb{\Blue}		  % Blue  Approximate PANTONE BLUE-072
\def\cce{\Cerulean}	  % Cerulean  Approximate PANTONE 3005
\def\ccy{\Cyan}		  % Cyan  Approximate PANTONE PROCESS-CYAN
\def\cpb{\ProcessBlue}	  % ProcessBlue  Approximate PANTONE PROCESS-BLUE
\def\csb{\SkyBlue}	  % SkyBlue  Approximate PANTONE 2985
\def\ctu{\Turquoise}	  % Turquoise  Approximate PANTONE (312+313)/2
\def\ctb{\TealBlue}	  % TealBlue  Approximate PANTONE 3145
\def\caq{\Aquamarine}	  % Aquamarine  Approximate PANTONE 3135
\def\cbg{\BlueGreen}	  % BlueGreen  Approximate PANTONE 320
\def\cem{\Emerald}	  % Emerald  No PANTONE match
%\def\cjg{\JungleGreen}	  % JungleGreen  Approximate PANTONE 328
\def\csg{\SeaGreen}	  % SeaGreen  Approximate PANTONE 3268
\def\cgg{\Green}	  % Green  VERY-Approx PANTONE GREEN
\def\cfg{\ForestGreen}	  % ForestGreen  Approximate PANTONE 349
\def\cpg{\PineGreen}	  % PineGreen  Approximate PANTONE 323
\def\clg{\LimeGreen}	  % LimeGreen  No PANTONE match
\def\cyg{\YellowGreen}	  % YellowGreen  Approximate PANTONE 375
\def\cspg{\SpringGreen}	  % SpringGreen  Approximate PANTONE 381
\def\cog{\OliveGreen}	  % OliveGreen  Approximate PANTONE 582
\def\pars{\RawSienna}	  % RawSienna  Approximate PANTONE 154
\def\cse{\Sepia}		  % Sepia  Approximate PANTONE 161
\def\cbr{\Brown}		  % Brown  Approximate PANTONE 1615
\def\cta{\Tan}		  % Tan  No PANTONE match
\def\cgr{\Gray}		  % Gray  Approximate PANTONE COOL-GRAY-8
\def\cbl{\Black}		  % Black  Approximate PANTONE PROCESS-BLACK
\def\cwh{\White}		  % White  No PANTONE match


\loadmsbm

\input epsf

\def\ctln{\centerline}
\def\u{\underbar}
\def\ssk{\smallskip}
\def\msk{\medskip}
\def\bsk{\bigskip}
\def\hsk{\hskip.1in}
\def\hhsk{\hskip.2in}
\def\dsl{\displaystyle}
\def\hskp{\hskip1.5in}

\def\lra{$\Leftrightarrow$ }
\def\ra{\rightarrow}
\def\mpto{\logmapsto}
\def\pu{\pi_1}
\def\mpu{$\pi_1$}
\def\sig{\Sigma}
\def\msig{$\Sigma$}
\def\ep{\epsilon}
\def\sset{\subseteq}
\def\del{\partial}
\def\inv{^{-1}}
\def\wtl{\widetilde}
\def\lra{\Leftrightarrow}
\def\del{\partial}
\def\delp{\partial^\prime}
\def\delpp{\partial^{\prime\prime}}
\def\sgn{{\roman{sgn}}}
\def\wtih{\widetilde{H}}
\def\bbz{{\Bbb Z}}
\def\bbr{{\Bbb R}}



\ctln{\bf Math 971 Algebraic Topology}

\ssk

\ctln{April 12, 2005}

\msk

There 
is another piece of homological algebra
that we will find useful ; the Five Lemma. It allows us to 
\u{compare} the information contained in two long exact sequences.

\msk

{\bf Five Lemma:} If we have abelian groups and maps

\ssk

\ctln{$\displaystyle 
\matrix 
A_n&{\buildrel f_n\over\ra}&B_n&{\buildrel g_n\over\ra}&C_n & {\buildrel h_n\over\ra} & D_n & {\buildrel i_n\over\ra} & E_n\cr
\alpha\downarrow & & \beta\downarrow & & \gamma\downarrow & & \delta\downarrow & & \epsilon\downarrow & \cr
A_{n-1} & {\buildrel f_{n-1}\over\ra} & B_{n-1} & {\buildrel g_{n-1}\over\ra} & C_{n-1} & {\buildrel h_{n-1}\over\ra} & D_{n-1} & {\buildrel i_{n-1}\over\ra} & E_{n-1}\cr
\endmatrix$}

\ssk

where the rows are exact, and the maps $\alpha,\beta,\delta,\epsilon$ are all isomorphisms, then $\gamma$ is an isomorphism.

\msk

The proof is in some sense literally a matter of doing the only thing you can. To show injectivity, suppose
$x\in C_n$ and $\gamma x = 0$, 
then $h_{n-1}\gamma x = \delta h_n x = 0$, so, 
since $\delta$ is injective,
$h_n x = 0$. So 
by the exactness at $C_n$, $x=g_n y$ for some $y\in B_n$. 
Then $g_{n-1} \beta y = \gamma g_n y = \gamma x = 0$, so 
by exactness at $B_{n-1}$, $\beta y = f_{n-1} z$ for some $z\in A_{n-1}$. Then 
since $\alpha$ is surjective,
$f_{n-1}z = \alpha w$ for some $w$. Then 
$0=g_n f_n w$ . But 
$\beta f_n w = f_{n-1} \alpha w \ f_{n-1} z = \beta y$, so since
$\beta$ is injective, 
$y= f_n w$ . So $0=g_n f_n w = g_n y = x$. So $x=0$.

\ssk

For surjectivity, suppose $x\in C_{n-1}$.
Then $h_{n-1} x \in D_{n-1}$, so
since $\delta$ is surjective, 
$h_{n-1} x = \delta y$ for some $y\in D_n$.
Then $\epsilon i_n y = i_{n-1}\delta y = i_{n-1} h_{n-1} x = 0$, so 
since $\epsilon$ is injective, $i_n y= 0$.
So by exactness at $D_n$, 
$y=h_n z$ for some $z\in C_n$. Then 
$h_{n-1}\gamma z = \delta h_n z = \delta y = h_{n-1} x$,
so $h_{n-1} (\gamma z-x) = 0$, so 
by exactness at $C_{n-1}$, 
$\gamma z-x = g_{n-1}w$ for some $w\in B_{n-1}$. Then
since $\beta$ is surjective, 
$w= \beta u$ for some $u\in B_n$. Then 
$\gamma g_n u = g_{n-1} \beta u = g_{n-1}w = \gamma z-x$,
so $x=\gamma z - \gamma g_n u = \gamma (z-g_n u)$ . 
So $\gamma$ is onto.

\bsk

The second result that this machinery gives us is what is properly known as {\it excision}:

\msk

If $B\sset A\sset X$ and cl$_X(B)\sset$ int$_X(A)$, then for every $k$ the inclusion-induced map 
$H_k(X\setminus B,A\setminus B)\ra H_k(X,A)$ is an isomorphism. 

\msk

An equivalent formulation of this is that if $A,B\sset X$ and int$_X(A)\cup$ int$_X(B) = X$, then the
inclusion-induced map $H_k(B,A\cap B)\ra H_k(X,A)$ is an isomorphism. [From first to second
statement, set $B^\prime = X\setminus B$ .] To prove the second statement, we know that
the inclusion $C_n^{\{A,B\}}(X) \ra C_n(X)$ induce isomorphisms on homology, as does 
$C_n(A) \ra C_n(A)$, so, by the five lemma, the induced map
$C_n^{\{A,B\}}(X)/C_n(A) \ra C_n(X)/C_n(A) = C_n(X,A)$ induces isomorphisms on homology. 
But the inclusion $C_n(B) \ra C_n^{\{A,B\}}(X)$ induces a map 
$C_n(B,A\cap B) = C_n(B)/C_n(A\cap B) \ra C_n^{\{A,B\}}(X)/C_n(A)$ which is an isomorphism of chain groups;
a basis for $C_n^{\{A,B\}}(X)/C_n(A)$ consists of singular simplices which map into $A$ or $B$, but don't map into $A$,
i.e., of simplices mapping into $B$ but not $A$, i.e., of simplices mapping into $B$ but not $A\cap B$.
But this is the \u{same} as the basis for $C_n(B,A\cap B)$ !

\msk

With these tools, we can start making some \u{real} homology computations. First, we show that 
if $\emptyset\neq A\sset X$ is ``nice enough'', then $H_n(X,A)\cong \widetilde{H}_n(X/A)$ .
The definition of nice enough, like Seifert - van Kampen, is that $A$ is closed and has an open neighborhood
${\Cal U}$ that deformation retracts to $A$ (think: $A$ is the subcomplex of a CW-complex $X$).
Then using ${\Cal U}.X\setminus A$ as a cover of $X$, and ${\Cal U}/A,(X\setminus A)/A$ as a cover of $X/A$,
 we have

\ssk

$\widetilde{H}_n(X/A) {\buildrel {(1)}\over \cong} H_n(X/A,A/A){\buildrel {(2)}\over \cong} 
H_n(X/A,{\Cal U}/A) {\buildrel {(3)}\over \cong} H_n(X/A\setminus A/A,{\Cal U}/A\setminus A/A) {\buildrel {(4)}\over \cong}
H_n(X\setminus A,{\Cal U}\setminus A){\buildrel {(5)}\over \cong} H_n(X,A)$

\ssk

Where (1),(2) follow from the LES for a pair, (3),(5) by excision, and (4) because the restriction of the quotient
map $X\ra X/A$ gives a homeomorphism of pairs.

\msk

Second, if $X,Y$ are $T_1$, $x\in X$ and $y\in Y$ each have neighborhoods 
${\Cal U},{\Cal V}$ which deformation retract to each point, then the 
one-point union $Z=X\vee Y = (X\coprod Y)/(x=y)$ has $\widetilde{H}_n(Z) \cong \widetilde{H}_n(X)\oplus \widetilde{H}_n(Y)$;
this follows from a similar sequence of isomorphisms. Setting $z$ = the image of $\{x,y\}$ in $Z$, we have

\ssk

$\widetilde{H}_n(Z) \cong H_n(Z,z) \cong H_n(Z,{\Cal U}\vee{\Cal V}) \cong H_n(Z\setminus z,{\Cal U}\vee{\Cal V}\setminus z)
\cong H_n([X\setminus x]\coprod[Y\setminus y],[{\Cal U}\setminus x]\coprod [{\Cal V}\setminus y])
\cong H_n(X\setminus x,{\Cal U}\setminus x)\oplus H_n(Y\setminus y,{\Cal V}\setminus y) 
\cong H_n(X,x)\oplus H_n(Y,y)\cong \widetilde{H}_n(X)\oplus \widetilde{H}_n(Y)$

\ssk

By induction, we then have $\displaystyle \widetilde{H}_n(\vee_{i=1}^k X_i) \cong \oplus_{i=1}^k \widetilde{H}_n(X_i)$

\bsk

We have so far introduced two homologies; simplicial, $H_*^\Delta$, whose computation 
``only'' required some linear algebra,
and singular, $H_*$, which is formally less difficult to work with, and which, you may suspect by now, is also becoming
less difficult to compute... For $\Delta$-complexes, these homology groups are the same, $H_n^\Delta(X)\cong H_n(X)$
for every $X$. In fact, the isomorphism is induced by the inclusion $C_n^\Delta(X)\sset C_n(X)$. And we have
now assembled all of the tools necessary to prove this. Or almost; we need to note that most of the edifice we
have built for singular homology \u{could} have been built for simplicial homology, including relative 
homology (for a sub-$\Delta$-complex $A$ of $X$), and a SES of chain groups, giving a LES sequence for the pair,

\ssk

$\cdots \ra H_n^\Delta(A) \ra H_n^\Delta(X) \ra H_n^\Delta(X,A) \ra H_{n-1}^\Delta(A) \ra \cdots$

\ssk

The proof of the isomorphism between the two homologies proceeds by first showing that the
inclusion induces an isomorphism on $k$-skeleta, $H_n^\Delta(X^{(k)})\cong H_n(X^{(k)})$,
and this goes by induction on $k$ using the Five Lemma applied to the diagram

\ssk

\ctln{$\displaystyle 
\matrix 
H_{n+1}^\Delta(X^{(k)},X^{(k-1)})&\ra&H_n^\Delta(X^{(k-1)})&\ra&H_n^\Delta(X^{(k)}) & \ra & H_{n}^\Delta(X^{(k)},X^{(k-1)}) & \ra & H_{n-1}^\Delta(X^{(k-1)})\cr
\downarrow & & \downarrow & & \downarrow & & \downarrow & & \downarrow & \cr
H_{n+1}(X^{(k)},X^{(k-1)})&\ra&H_n(X^{(k-1)})&\ra&H_n(X^{(k)}) & \ra & H_{n}(X^{(k)},X^{(k-1)}) & \ra & H_{n-1}(X^{(k-1)}) \cr
\endmatrix$}

\ssk

The second and fifth vertical arrows are, by an inductive hypothesis, isomorphisms. The first and fourth vertical arrows are
isomorphisms because, essentially, we can, in each case, identify these groups. 
$H_{n}(X^{(k)},X^{(k-1)})\cong H_{n}(X^{(k)}/X^{(k-1)})\cong \widetilde{H}_n(\vee S^k)$
are either 0 (for $n\neq k$) or $\oplus \bbz$ (for $n=k$), one summand for each $n$-simplex in $X$. 
But the same is true for $H_{n}^\Delta(X^{(k)},X^{(k-1)})$; and for $n=k$ the generators are precisely
the $n$-simplices of $X$. The inclusion-induced map takes generators to generators, so is an isomorphism.
\hhsk So by the Five Lemma, the middle rows are also isomorphisms, completing our inductive proof.

\ssk

Returning to $H_n^\Delta(X) {\buildrel {I_*}\over \ra} H_n(X)$, we wish now to show that this map is an isomorphism.
Any $[z]\in H_n(X)$ is represented by a cycle $z=\sum a_i\sigma_i$ for $\sigma_i:\Delta^n\ra X$ . But each
$\sigma_i(\Delta^n)$ is a compact subset of $X$, and so meets only finitely-many cells of $X$. This is true for every
singular simplex, and so there is a $k$ for which all of the simplices map into $X^{(k)}$, and so we may
treat $z\in C_n(X^{(k)}$. Thought of in this way, it is still a cycle, and so $[z]\in H_n(X^{(k)})\cong H_n^\Delta(X^{(k)})$
so there is a $z^\prime in C_n^\Delta(X^{(k)})$ and a $w\in C_{n+1}(X^{(k)})$ with $i_\#z^\prime -z=\del w$. 
But thinking of  $z^\prime in C_n^\Delta(X)$ and $w\in C_{n+1}(X)$, we have the same equality, so 
$[z^\prime] \in H_n^\Delta(X)$ and $i_*[z^\prime] = [z]$ . So $i_*$ is surjective.
If $i_*([z]) = 0$, then the cycle $z=\sum a_i\sigma_i$ is a sum of characteristic maps of $n$-simplices of $X$, and
so can be thought of as an element of $C_n^\Delta(X^{n)})$ . Being $0$ in $H_n(X)$, $z=\del w$ for some
$w\in C_{n+1}(X)$ . But as before, $w\in C_n(X^{r)})$ for some $r$, and so thought of as an element of 
the image of the isomorphism $i_*: H_n^\Delta(X^{(r)})\ra H_n(X^{(r)})$, $i_*([z])=0$, so $[z]=0$ . So 
$z=\del u$ for some $u\in C_{n+1}^\Delta(X^{r)})\sset C_{n+1}^\Delta(X)$ . So $[z]=0$ in $H_n^\Delta(X)$.
Consequently, simplicial and singular homology groups are isomorphic.



%%%
%%%
%%%
%%%

\bigskip

Some topological results with homological proofs: if ${\Bbb R}^n\cong {\Bbb R}^m$, via $h$, then $n=m$ .
This is because we can arrange, by composing with a translation, that $h(0)=0$, and then 
we have 
$({\Bbb R}^n,{\Bbb R}^n\setminus 0)\cong {\Bbb R}^m,({\Bbb R}^m\setminus 0)$, which gives


\ssk

\ctln{$\widetilde{H}_i(S^{n-1})\cong H_{i+1}({\Bbb D}^n,\del {\Bbb D}^n) \cong H_{i+1}({\Bbb D}^n,{\Bbb D}^n\setminus 0)
\cong H_{i+1}({\Bbb R}^n,{\Bbb R}^n\setminus 0) \cong H_{i+1}({\Bbb R}^m,{\Bbb R}^m\setminus 0)$}

\ctln{$\cong H_{i+1}({\Bbb D}^m,{\Bbb D}^m\setminus 0) \cong H_{i+1}({\Bbb D}^m,\del {\Bbb D}^m)
\cong \widetilde{H}_i(S^{m-1})$}

\ssk

Setting $i=n-1$ gives the result, since $\widetilde{H}_{n-1}(S^{m-1})\cong {\Bbb Z}$ implies $n-1=m-1$ .

\msk

More generally, we can establish a result which is known as {\it invariance of domain},
a result which is useful in both topology and analysis.

\msk

{\bf Invariance of Domain:} If ${\Cal U}\subseteq {\Bbb R}^n$ and $f:{\Cal U}\ra {\Bbb R}^n$
is continuous and injective, then $f({\Cal U})\subseteq {\Bbb R}^n$ is open.

\msk

Note it is enough to proof this for our favorite open set, which in this context will be ${\Cal V}=(-1,1)^n\subseteq {\Bbb R}^n$,
since given any open ${\Cal U}$ and $x\in{\Cal U}$, we can find an injective linear map $h:(-1,1)^n\ra {\Cal U}$
taking $0$ to $x$. If we can show that $f\circ h$ has open image, then $f(x)\in f\circ h({\Cal V})\subseteq f({\Cal U})$
shows that $f(x)$ has an open neighborhood in $f({\Cal U})$ . Since $x$ is arbitrary, $f({\Cal U})$ is open.

\msk

\bsk

\msk

This in turn implies the ``other'' invariance of domain; if $f: {\Bbb R}^n\ra {\Bbb R}^m$ is continuous and injective, then
$n\leq m$, since if not, then composition of $f$ with the inclusion $i:{\Bbb R}^m\ra {\Bbb R}^n$, $i(x_1,\ldots ,x_m) = 
(x_1,\ldots ,x_m,0,\ldots ,0)$ is injective and continuous with non-open image (it lies in a hyperplane in ${\Bbb R}^n$),
a contradiction.


\bsk

Our next goal is to show that,
when both make sense, simplicial and singular homology are
isomorphic. In fact, the inclusion of the simplicial chain groups into the 
singular ones induces an isomorphism on homology. 


\vfill
\end



A map of pairs $f: (X,A) \ra (Y,B)$ (meaning that $f(A)\subseteq B$)
induces (by postcomposition) a map of relative homology $f_*:H_i(X,A)\ra H_i(Y,B)$ , just as with 
absolute homology.
We also get a homotopy-invariance result: if $f,g: (X,A) \ra (Y,B)$ are maps of pairs
which are {\it homotopic as maps of pairs},
i.e., there is a map $(X\times I,A\times I)\ra (Y,B)$ which is $f$ on one end and $g$
on the other, then $f_*=g_*$ . The proof is identical to the one given before; the prism map 
$P$ sends chains in $A$ to chains in $A$, so induces a map $C_i(X\times I,A\times I)\ra C_{i+1}(X,A)$
which does precisely what we want.



