

\magnification=1200
\overfullrule=0pt
\parindent=0pt

\nopagenumbers

\input amstex

\voffset=-.6in
\hoffset=-.5in
\hsize = 7.5 true in
\vsize=10.4 true in

%\voffset=1.4in
%\hoffset=-.5in
%\hsize = 10.2 true in
%\vsize=8 true in

\input colordvi

\def\cltr{\Red}		  % Red  VERY-Approx PANTONE RED
\def\cltb{\Blue}		  % Blue  Approximate PANTONE BLUE-072
\def\cltg{\PineGreen}	  % ForestGreen  Approximate PANTONE 349
\def\cltp{\DarkOrchid}	  % DarkOrchid  No PANTONE match
\def\clto{\Orange}	  % Orange  Approximate PANTONE ORANGE-021
\def\cltpk{\CarnationPink}	  % CarnationPink  Approximate PANTONE 218
\def\clts{\Salmon}	  % Salmon  Approximate PANTONE 183
\def\cltbb{\TealBlue}	  % TealBlue  Approximate PANTONE 3145
\def\cltrp{\RoyalPurple}	  % RoyalPurple  Approximate PANTONE 267
\def\cltp{\Purple}	  % Purple  Approximate PANTONE PURPLE

\def\cgy{\GreenYellow}     % GreenYellow  Approximate PANTONE 388
\def\cyy{\Yellow}	  % Yellow  Approximate PANTONE YELLOW
\def\cgo{\Goldenrod}	  % Goldenrod  Approximate PANTONE 109
\def\cda{\Dandelion}	  % Dandelion  Approximate PANTONE 123
\def\capr{\Apricot}	  % Apricot  Approximate PANTONE 1565
\def\cpe{\Peach}		  % Peach  Approximate PANTONE 164
\def\cme{\Melon}		  % Melon  Approximate PANTONE 177
\def\cyo{\YellowOrange}	  % YellowOrange  Approximate PANTONE 130
\def\coo{\Orange}	  % Orange  Approximate PANTONE ORANGE-021
\def\cbo{\BurntOrange}	  % BurntOrange  Approximate PANTONE 388
\def\cbs{\Bittersweet}	  % Bittersweet  Approximate PANTONE 167
%\def\creo{\RedOrange}	  % RedOrange  Approximate PANTONE 179
\def\cma{\Mahogany}	  % Mahogany  Approximate PANTONE 484
\def\cmr{\Maroon}	  % Maroon  Approximate PANTONE 201
\def\cbr{\BrickRed}	  % BrickRed  Approximate PANTONE 1805
\def\crr{\Red}		  % Red  VERY-Approx PANTONE RED
\def\cor{\OrangeRed}	  % OrangeRed  No PANTONE match
\def\paru{\RubineRed}	  % RubineRed  Approximate PANTONE RUBINE-RED
\def\cwi{\WildStrawberry}  % WildStrawberry  Approximate PANTONE 206
\def\csa{\Salmon}	  % Salmon  Approximate PANTONE 183
\def\ccp{\CarnationPink}	  % CarnationPink  Approximate PANTONE 218
\def\cmag{\Magenta}	  % Magenta  Approximate PANTONE PROCESS-MAGENTA
\def\cvr{\VioletRed}	  % VioletRed  Approximate PANTONE 219
\def\parh{\Rhodamine}	  % Rhodamine  Approximate PANTONE RHODAMINE-RED
\def\cmu{\Mulberry}	  % Mulberry  Approximate PANTONE 241
\def\parv{\RedViolet}	  % RedViolet  Approximate PANTONE 234
\def\cfu{\Fuchsia}	  % Fuchsia  Approximate PANTONE 248
\def\cla{\Lavender}	  % Lavender  Approximate PANTONE 223
\def\cth{\Thistle}	  % Thistle  Approximate PANTONE 245
\def\corc{\Orchid}	  % Orchid  Approximate PANTONE 252
\def\cdo{\DarkOrchid}	  % DarkOrchid  No PANTONE match
\def\cpu{\Purple}	  % Purple  Approximate PANTONE PURPLE
\def\cpl{\Plum}		  % Plum  VERY-Approx PANTONE 518
\def\cvi{\Violet}	  % Violet  Approximate PANTONE VIOLET
\def\clrp{\RoyalPurple}	  % RoyalPurple  Approximate PANTONE 267
\def\cbv{\BlueViolet}	  % BlueViolet  Approximate PANTONE 2755
\def\cpe{\Periwinkle}	  % Periwinkle  Approximate PANTONE 2715
\def\ccb{\CadetBlue}	  % CadetBlue  Approximate PANTONE (534+535)/2
\def\cco{\CornflowerBlue}  % CornflowerBlue  Approximate PANTONE 292
\def\cmb{\MidnightBlue}	  % MidnightBlue  Approximate PANTONE 302
\def\cnb{\NavyBlue}	  % NavyBlue  Approximate PANTONE 293
\def\crb{\RoyalBlue}	  % RoyalBlue  No PANTONE match
%\def\cbb{\Blue}		  % Blue  Approximate PANTONE BLUE-072
\def\cce{\Cerulean}	  % Cerulean  Approximate PANTONE 3005
\def\ccy{\Cyan}		  % Cyan  Approximate PANTONE PROCESS-CYAN
\def\cpb{\ProcessBlue}	  % ProcessBlue  Approximate PANTONE PROCESS-BLUE
\def\csb{\SkyBlue}	  % SkyBlue  Approximate PANTONE 2985
\def\ctu{\Turquoise}	  % Turquoise  Approximate PANTONE (312+313)/2
\def\ctb{\TealBlue}	  % TealBlue  Approximate PANTONE 3145
\def\caq{\Aquamarine}	  % Aquamarine  Approximate PANTONE 3135
\def\cbg{\BlueGreen}	  % BlueGreen  Approximate PANTONE 320
\def\cem{\Emerald}	  % Emerald  No PANTONE match
%\def\cjg{\JungleGreen}	  % JungleGreen  Approximate PANTONE 328
\def\csg{\SeaGreen}	  % SeaGreen  Approximate PANTONE 3268
\def\cgg{\Green}	  % Green  VERY-Approx PANTONE GREEN
\def\cfg{\ForestGreen}	  % ForestGreen  Approximate PANTONE 349
\def\cpg{\PineGreen}	  % PineGreen  Approximate PANTONE 323
\def\clg{\LimeGreen}	  % LimeGreen  No PANTONE match
\def\cyg{\YellowGreen}	  % YellowGreen  Approximate PANTONE 375
\def\cspg{\SpringGreen}	  % SpringGreen  Approximate PANTONE 381
\def\cog{\OliveGreen}	  % OliveGreen  Approximate PANTONE 582
\def\pars{\RawSienna}	  % RawSienna  Approximate PANTONE 154
\def\cse{\Sepia}		  % Sepia  Approximate PANTONE 161
\def\cbr{\Brown}		  % Brown  Approximate PANTONE 1615
\def\cta{\Tan}		  % Tan  No PANTONE match
\def\cgr{\Gray}		  % Gray  Approximate PANTONE COOL-GRAY-8
\def\cbl{\Black}		  % Black  Approximate PANTONE PROCESS-BLACK
\def\cwh{\White}		  % White  No PANTONE match


\loadmsbm

\input epsf

\def\ctln{\centerline}
\def\u{\underbar}
\def\ssk{\smallskip}
\def\msk{\medskip}
\def\bsk{\bigskip}
\def\hsk{\hskip.1in}
\def\hhsk{\hskip.2in}
\def\dsl{\displaystyle}
\def\hskp{\hskip1.5in}

\def\lra{$\Leftrightarrow$ }
\def\ra{\rightarrow}
\def\mpto{\logmapsto}
\def\pu{\pi_1}
\def\mpu{$\pi_1$}
\def\sig{\Sigma}
\def\msig{$\Sigma$}
\def\ep{\epsilon}
\def\sset{\subseteq}
\def\del{\partial}
\def\inv{^{-1}}
\def\wtl{\widetilde}
\def\lra{\Leftrightarrow}



\ctln{\bf Math 971 Algebraic Topology}

\ssk

\ctln{March 10, 2005}

\msk

{\bf Some examples:} the Klein bottle $K$ has a $\Delta$-complex structure with 2 2-simplices,
3 1-simplices, and 1 0-simplex; we will call them 
$f_1=[0,1,2],f_2=[1,2,3],
e_1=[0,2]=[1,3],e_2=[1,0]=[2,3],e_3=[1,2]$, 
and $v_1=[0]=[1]=[2]=[3]$.
Computing, we find 
$\del_2 f_1 = \del[0,1,2]=[1,2]-[0,2]+[0,1]=e_3-e_1-e_2$ , $\del_2 f_2 = e_2-e_1+e_3$ , 
$\del_1 e_1 = \del_1 e_2 = \del_1 e_3 = 0$ and $\del_i = 0$ for all other $i$
(as well). So we have the chain complex

$\cdots \ra 0 \ra {\Bbb Z}^2 \ra {\Bbb Z}^3 \ra {\Bbb Z} \ra 0$

and all of the boundary maps are 0, except for $\del_2$, which has the matrix
$\pmatrix
 -1&-1\\ -1&1\\ 1&1\\
\endpmatrix$ . This matrix is injective, so $\ker \del_2 = 0$,
so $H_2(K)=0$, on the other hand, $H_1(K)$ = coker$(\del_2)$, and applying column
operations we can transform the matrix for $\del_2$ to $\pmatrix 1&0\\ 1&2\\ -1&0\\ \endpmatrix$,
which implies that the cokernel is ${\Bbb Z}\oplus{\Bbb Z}_2$, since 
$\pmatrix 1\\ 1\\ -1\\ \endpmatrix ,\pmatrix 0\\ 1\cr 0\\ \endpmatrix , \pmatrix 0\cr 0\\ 1\\ \endpmatrix$
is a basis for ${\Bbb Z}^3$. Finally, $H_0(K)={\Bbb Z}$, since $\del_1,\del_0=0$,
and all higher homology groups are also $0$, for the same reason.

\msk

As another example, the topologist's dunce hat has a $\Delta$-structure with
1 2-simplex, 1 1-simplex, and 1 0-simplex. The boundary maps, we can work out
(starting from $C_2(X)$ ), are $(1),(0)$, and $(0)$, so $H_2(X)=H_1(X)=0$,
and $H_0(X)={\Bbb Z}$. all higher groups are also $0$.

\msk

These homology groups are, in the end, fairly routine to calculate from a 
$\Delta$-complex structure. But there is one very large problem; the calculations
\u{depend} on the $\Delta$ structure! This is not a group defined from the space
$X$; it is defined from the space and a $\Delta$ structure on it. A priori, we don't
know that if we chose a different structure on the same space, that we would get
isomorphic groups! We should really denote our groups by $H_i^\Delta(X)$, to 
acknowledge this dependence on the structure.

\bsk

But we don't {\it want} a group that depends on this structure. We want groups that
just depend on the topological space $X$, i.e., which are topological invariants.
In really turns out that these groups $H_i^\Delta(X)$ \u{are} topological invariants,
but we will need to take a very roundabout route to show this. What we will do
now is to define another sequence $H_i(X)$ of groups, the {\it singular homology
groups}, which their definition makes apparent from the outset 
that they are topological invariants.
But this definition will also make it very unclear how to really compute them! 
Then we will show that for $\Delta$-complexes these two sequences of groups
are really the same. In so doing, we will have built a sequence of topological
invariants that for a large class of spaces are fairly routine to compute. Then
all we will need to show is that they also capture useful information about
a space (i.e., we can prove useful theorems with them!).

\msk

And the basic idea behind defining them is that, with simplicial homology,
we have already done all of the hard work. What we do is, as before, build a 
sequence of (free) abelian groups, the chain groups $C_n(X)$, 
and boundary maps between them,
with consecutive maps composing to 0. Then, as before, the homology groups are
kernels mod images, i.e., cycles mod boundaries. And, as before, the basis
elements for each of our chain groups $C_n(X)$ will be the $n$-simplices
in $X$. But now $X$ is \u{any} topological space. So how do we get $n$-simplices
in such a space? We do the only thing we can; we {\it map} them in. 

\msk

More precisely, we work with {\it singular $n$-chains}, that is,
formal (finite) linear combinations $\sum a_i\sigma_i$, where $a_i\in{Bbb Z}$
and the $\sigma_i$ are {\it singular simplices}, that is, (continuous) 
maps $\sigma_i:\Delta^n\ra X$ from the (standard) $n$-simplex into $X$.
The boundary maps are really exactly as before; they are the alternating sum of
the restrictions of $\sigma_i$ to the $n+1$ faces of $\Delta^n$ . (Formally,
we must precompose these face maps with the (orientation-preserving) linear
isomorphism from the standard $(n-1)$-simplex to each of the faces, preserving
the ordering of their vertices.) The same proof as before (except that we interpret
the faces as restrictions of the map $\sigma_i$, instead of as physical faces)
shows that the composition of two successive boundaries are 0,
and so all of the machinery is in place to define the {\it singular homology
groups} $H_i(X)$ as the kernel of $\del_i$ modulo the image of $\del_{i+1}$ = $Z_i(X)/B_i(X)$ .
They are, by their definition, groups defined using the topological space $X$ as input,
and so are topological invariants of $X$. The elements are equivalence classes of $i$-cycles,
where $z_1\sim z_2$ if $z_1-z_2=\del w$ for some $(i+1)$-chain $w$ . We say that $z_1$ and $z_2$ are
{\it homologous}.

\msk

The fun comes when you try to compute them. 
$C_n(X) = \{\sum a_i\sigma_i$ : $a_i\in {\Bbb Z}$ and $\sigma_i:\Delta^n\ra X$ 
is continuous$\}$ is typically a \u{huge} group, since there will be immense
numbers of maps $\Delta^n\ra X$ . About the only space for which this is not true is
the one-point space $*$; then there are, for each $n$, exactly one (distinct)
map $\sigma_n :\Delta^n\ra *$ ; the constant map. Therefore each face of $\Delta^n$
gives the same restriction map $\sigma^{n-1}$, and so the boundary maps can 
be dirctly computed (the depend on the parity of the number $n+1$ of faces 
an $n$-simplex has). We find that $\del_{2n}=Id$ and $\del_{2n-1}=0$ . so in 
computing homology groups, we either have kernel everything ($\del_i=0$) and
image everything ($\del_{i+1}=Id$) or kernel nothing ($\del_i=Id$) and
image nothing ($\del_{i+1}=0$), so in both cases $H_i(*)=0$ . Except for $i=0$;
then $\del_0=0$ (by definition) and $\del_1=0$, so $H_0(*)={\Bbb Z}$ .
But other than this example (and, well, OK, spaces with the discrete topology;
it's the same calculation as above for every point!), computing singular 
homology from the definition is quite a chore! so we need to build
some labor-saving devices, namely, some theorems to help us break the problem
of computing these groups into smaller, more managable pieces.

\msk

First set of managable pieces: if we decompose $X$ into its path components, $X=\bigcup X_\alpha$,
then $H_i(X) \cong \bigoplus H_i(X_\alpha)$ for every $i$. This is because every singular simplex,
since $\Delta^i$ is path-connected, maps into some $X_\alpha$ . So $C_i(X) \cong \bigoplus C_i(X_\alpha)$.
Since the boundary of a simplex mapping into $X_\alpha$ consists of simplices in $X_\alpha$, the 
boundary maps respect the decomposistions of the chain groups, so 
$B_i(X) \cong \bigoplus B_i(X_\alpha)$ and $Z_i(X) \cong \bigoplus Z_i(X_\alpha)$, and so 
the quotients are $H_i(X) \cong \bigoplus H_i(X_\alpha)$ . 

\msk

So, if we wish to, we can focus on computing homology groups for path-connected spaces $X$. For such a space, 
$H_0(X)\cong {\Bbb Z}$, generated by any map of a 0-simplex (= a point) into $X$. This is because any pair
of 0-simplices are homologous; given any two points $x,y\in X$, there is a path $\gamma: I\ra X$ from $x$ to $y$,
This path can be interpreted as a singular 1-simplex, and $\del\gamma = y-x$ . So $H_0(X)$ is generated
by a single point $[x]$ . No multiple of this point is null-homologous, because for any 1-chain $\sum n_i \sigma_i$,
the sum of the coefficients of its boundary is 0 (since this is true for each singular 1-simplex), and any 0-chain
$\sum n_i [x_i]$ is homologous to $(\sum n_i)[x]$ by the above argument.

\msk

Perhaps the most important property of the fundamental group is that a continuouos map 
between spaces induces a homomorphism between groups. (This implied, for instance,
that homeomorphic spaces have isomorphic \mpu ). The same is true for homology groups, 
for essentially the same reason. Given a map $f:X\ra Y$, there is an induced map $f_\#:C_n(X)\ra C_n(Y)$
defined by postcomposition; for a singular simplex $\sigma$, $f_\#(\sigma) = f\circ\sigma$, and we extend
the map linearly. Since $f\circ(g|_A) = (f\circ g)|_A$ (postcomposition commutes with restriction of the domain),
$f_\#$ commutes with $\del$ : $f_\#(\del \sigma) = \del(f_\#(\sigma))$. A homomorphism between
chain complexes (i.e., a sequence of such maps, one for each chain group) which commutes with the 
boundaries maps in this way, is called a {\it chain map}.
A chain map, such as $f_\#$, therefore, takes cycles to cycles,
and boundaries to boundaries, and so $f_\#:Z_i(X)\ra Z_i(Y)$ (which is linear, hence a homomorphism)
induces a homomorphism $f_*:H_i(X)\ra H_i(Y)$ by $f_*[z] = [f_\#(z)]$ . 
Since it is defined by composition with singular simplices, it is 
immediate that, for a map $g:Y\ra Z$, $(g\circ f)_*=g_*\circ f_*$ . And since the identity map $I:X\ra X$
satisfies $I_\#=Id$, so $I_*=Id$, homeomorphic spaces have isomorphic homology groups.

\msk

Another important property of \mpu\ is that homotopic maps give the same
induced map (after correcting for basepoints). This is also true for homology;
if $f\sim g:X\ra Y$, then $f_*=g_*$ . The proof, however, is not quite as straightforward
as for homotopy. And it requires some new technology; the chain homotopy.
A chain homotopy $H$ between the chain complexes $f_\#,g_\#:C_*(X)\ra C_*(Y)$ 
is a sequence of homomorphisms $H_i:C_i(X)\ra C_{i+1}(Y)$ satisfying
$H_{i-1}\del_i+\del_{i+1}H_i = f_\#-g_\#:C_i(X)\ra C_i(Y)$ . The existence of $H$
implies that $f_*=g_*$, since for an $i$-cycle $z$ (with $\del_i(z)=0$) we have

$f_*[z]-g_*[z] = [f_\#(z)-g_\#(z)] = [H_{i-1}\del_i(z)+\del_{i+1}H_i(z)] = [H_{i-1}(0)+\del_{i+1}(w)]
=[\del_{i+1}(w)]=0$.

And the existence of a homotopy between $f$ and $g$ implies the existence of a 
chain homotopy between $f_\#$ and $g_\#$ . This is because the homotopy 
gives a map $H:X\times I\ra Y$, which induces a map $H_\#:C_{i+1}(X\times I)\ra C_{i+1}(Y)$ .
Then we pull, from our back pocket, a {\it prism map}
$P:C_i(X)\ra C_{i+1}(X\times I)$; the composition $H_\#\circ P$ will be our chain homotopy.
The prism map  takes a (singular) $i$-simplex $\sigma$ and sends it to a sum of singular $(i+1)$-simplices
in $X\times I$. and the way we define it is to take the $i$-simplex $\Delta^i$, and taking it 
to $\Delta^i\times I$ (i.e., a {\it prism}), and thinking of this as a sum of $(i+1)$-simplices. Using the
map $\sigma\times Id : \Delta^i\times I\ra X\times I$ restricted to each of these $(i+1)$-simplices
yields the prism map. Now, there are many ways of decomposing a prism into simplices,
but we need to be careful to choose one which restricts well to each of the \u{faces} of $\Delta^i$,
in order to get the chain homotopy property we require. In the end, what this requires is that the
decomposition, when restricted to any face of $\Delta^i$ (which we think of as a copy
of $\Delta^{i-1}$), is the same as the decomposition we would have applied to a prism over
an $(i-1)$-simplex.



















\vfill
\end