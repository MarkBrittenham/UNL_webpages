\documentclass[11pt]{article}
\pagestyle{empty}
%\usepackage{amsmath}          % in case it's needed
\setlength{\parskip}{1em}
\setlength{\textheight}{8.6in}
\setlength{\topmargin}{0in}
\setlength{\parindent}{0in}
%\setlength{\textwidth}{6in} % for playing with margins
%\setlength{\leftmargin}{3in}  % for playing with margins
%\setlength{\parindent}{-4ex}  % for exams
%\setlength{\unitlength}{1in}  % for graphics done in inches

\begin{document}
\centerline{\bf Math 203 Project I}

This project gives you the opportunity to write about some of the mathematics  
that you have learned. You will probably find that writing about  
mathematics is challenging, but it will often also help you to clarify your  
own understanding. In addition, the conciseness and precision required for 
mathematical writing can benefit your writing in other areas, as well.

Your project report should consist of 2-3 typed pages (although you
should use as much space as you think necessary to adequately convey
your ideas). Think of the  
reader as your boss or supervisor, who knows something about graphs,
but doesn't know much about Hamiltonian circuits. This person is 
considering you for a promotion, so you want to make a good impression
of how well you can explain things. 
If you would like preliminary comments on your  
report, give me a copy on Friday, February 22, and I will return it to you  
with comments during the next class period. The final project report is  
due in class on {\bf Wednesday, February 27.}

\vspace{.5in}
\centerline{\bf Maximum-Cost Hamiltonian Circuits}

Consider an alternative version of the Traveling Salesman Problem, in  
which the goal is to find a \emph{maximum-cost} Hamiltonian circuit.   
Explain how you would change the sorted-edges algorithm that we
learned, to look for maximum
cost circuits instead of minimum ones; make sure to  
carefully explain your modified algorithm.  Then, illustrate your  
algorithm by applying it to the graph provided below.  Also in your  
report, give two examples of practical situations in which it might 
be useful to find maximum cost circuits.

\input epsf.tex

\leavevmode
\epsfxsize=4in
\centerline{\epsfbox{wrgraphb.ai}}

\end{document}