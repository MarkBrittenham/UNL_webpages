\documentclass[11pt]{article}
\pagestyle{empty}
\setlength{\oddsidemargin}{0in}
\setlength{\evensidemargin}{\oddsidemargin}
\setlength{\textwidth}{6.5in}
\setlength{\topmargin}{-.25in}
\setlength{\headheight}{0in}
\setlength{\headsep}{0in}
\setlength{\topskip}{0in}
\setlength{\textheight}{9.5in}

\begin{document}

\begin{center}
\LARGE\bfseries{Math 203 Web Tests}
\end{center}

As you know, a portion of each of the two tests in this course will be
taken on-line.  This sheet answers some questions you might have about
the tests.  If other questions arise, just ask me!

\begin{itemize}
\item Where do I go to take the web tests?

You can go to either of the following two computer labs:
\begin{description}
\item[Arts\&Sciences Testing Center]  This lab is in Burnett
127.  Hours of operation are:

\begin{tabular}{ll}
Monday & 9:00am -- 8:00pm \\
Tuesday & 9:00am -- 8:00pm \\
Wednesday & 9:00am -- 8:00pm \\
Thursday & 9:00am -- 8:00pm \\
Friday & 9:00am -- 3:00pm
\end{tabular}
\item[MathLab] This lab is in Bessey 105.  Hours of
operation\footnote{These hours are subject to change; check {\tt
http://www.math.unl.edu/?area=LABS} for updated information.} are:

\begin{tabular}{ll}
Monday & 10:30am -- 10:30pm \\
Tuesday & 9:00am --  10:30pm \\
Wednesday & 8:30am -- 10:30pm \\
Thursday & 8:30am -- 10:30pm \\
Friday & 9:00am -- 4:30pm\\
Sunday & 1:00pm -- 4:00pm
\end{tabular}
\end{description}

\item What should I take with me?

You should bring your student ID card, some clean scratch paper, a pen
or pencil (for doing scratch work), and a calculator.  {\bf You will
not be allowed to use any books or notes during the exam.}

\item How long should I expect the exam to take?

You have unlimited time, and you should probably budget at least 45
minutes to an hour.

\item Are practice exams available?

Yes.  You can access the practice exams from any computer by going to

\centerline{\tt http://calculus.unl.edu/gateway.html}  

\noindent The practice exams will become available about a week before the
proctored exam.  You are encouraged to practice!

\item When I go to the computer lab to take a proctored test, what
will happen?

The first time you take the test, you will need to register on the
system. That involves filling out a form on the screen and choosing a
login name and password for yourself. You will use your login and
password every time you retake the test, and when you take the second
test, so make sure you remember it. (If you forget it, you can
register again with another login, but it is better not to have to do
that!)  When you register, make sure you are registered for the
correct section of the course. If you take the test in the wrong
section, it will create all kinds of confusion!

Also, each time you take the proctored test, the proctor will ask to
see your ID both before and after the exam.

\item How many times am I allowed to take the web test?

You can take the practice test as many times as you like.  {\bf You can
take the proctored test at most once per day on each day during the
testing period.}

\end{itemize}

\end{document}




