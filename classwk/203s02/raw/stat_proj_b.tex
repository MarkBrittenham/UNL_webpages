\documentstyle[12pt]{article}
% This page was written by David Jaffe.

\textheight 8.5in
\voffset=-0.7in
\begin{document}
\pagestyle{empty}

\begin{center}
\Large\bf
Math 203 -- Statistics Project
\end{center}

\vspace{0.25in}

For this report, you are to form yourselves into groups of size up to
four.  The exact size and composition of the group is up to you.

\vspace{0.2in}

\par\noindent{\bf Goal:}\ To find, read, understand, and critique a
statistical study on a medical topic.

\vspace{0.2in}

\par\noindent{\bf Topics:}\ More than one group may share the same topic but
no sharing whatsoever is allowed {\it between\/} groups.  The following
topics
are suggested; other topics may be chosen {\it with approval\/} of the
instructor.  If in the process of searching for information about
one of these topics, you find information about a closely related
(but different) topic (which interests you), you may use it.  In any event,
the precise statement of your topic will depend on the article you find.
\begin{itemize}
\item Does being overweight increase one's risk of cardiovascular disease?
\item Does reducing ones caloric consumption prolong one's life?  (Some
studies
      have been done on animals other than humans.)
\item Is marijuana smoking addictive?
\item Is tobacco smoking addictive?
\item Does use of chewing tobacco cause oral cancer?
\item Does having an abortion increase a woman's risk of breast cancer?
\item Does regular exercise decrease one's risk of cardiovascular disease?
\item Does consumption of dietary fiber reduce one's risk of cardiovascular
      disease?
\item Does taking vitamin A pills reduce one's risk of getting cancer?
\item Does daily consumption of aspirin decrease one's blood pressure?
\item Does daily consumption of aspirin decrease one's risk of colon cancer?
\item Does daily consumption of a modest amount of alcohol decrease one's
      blood pressure?
\item Is dietary aluminum intake associated with Alzheimer's disease?
\item Does daily consumption of vitamin C pills reduce one's incidence (or
      severity) of the common cold or influenza?
\item Does home fluoride application improve one's dental health?
\item Does excessive fluoride intake damage one's bones?
\item Does low frequency electromagnetic radiation (e.g.\ from proximity to
      power lines) have any negative health effects?
\item Does consumption of alcohol by pregnant mothers reduce the birthweight
      of their children?
\item Does the smoking of tobacco by pregnant mothers reduce the birthweight
      of their children?
\item Does passive smoking have negative health effects?
\item Does increasing one's calcium intake reduce one's risk of
osteoporosis?
\item Does exposure to asbestos increase one's risk of getting lung cancer?
\end{itemize}

\vspace{0.2in}

\par\noindent{\bf What type of article should the group look for?}\ You
must find an article which reports on a statistical study.
{\bf The article must be a primary source!}  Surveys (which comment on other
people's work) are {\it not\/} acceptable as your main article, but you
could
refer to them as secondary resources.  Indeed this might make your job much
easier.  Do not use metastatistical studies (which merge data from several
sources).  Almost invariably such studies are bad science.

\vspace{0.2in}

\par\noindent{\bf How should the group find an article?}\ You may start by
looking at newspapers, magazines etc and try to locate an article that way
if you wish. However,
I recommend the
following approach.
\begin{itemize}
\item Start by querying a web search engine.  If you do not know how to
use the World Wide Web, now is a good time to learn.  You can go to Love
Library and use the computers there; the reference librarians will be happy
to help you.  I will too if you come to my office.
      \begin{itemize}
      \item If you do your work early in the morning, the computer will
            respond faster.
      \item Start with the National Library of Medicine's search engine, PubMed,
            which can be found at the NCBI webpage\\
            {\tt http://www.ncbi.nlm.nih.gov/}.\\
            (The trailing period is not part of the URL.)
            You can also try a general search engine, such as \\
            {\tt http://www.altavista.com/}.
      \item Give the search engine a list of keywords which you think will
            get you to an article.  The librarian can give you some guidance
            here.
      \item Look at some of the documents found by the search engine.  Some
            skill is involved in choosing those that are likely to yield
            the information you are looking for.
      \item Based on what you have found, revise your list of keywords (if
            need be) and try again.
      \end{itemize}
\item Make sure that your reference(s) are complete before proceeding.
\item Go to the library computer (using the URL ``{\tt
http://iris.unl.edu/})''
      and find the journal or journals which you need.  Find out which
library
      houses them.
\item Go get the journal or:
      \begin{itemize}
      \item If the journal is housed in C.\ Y.\ Thompson Library, you may
wish
            to submit a request for the library to fetch the article and
deliver
            it.
      \item Use Interlibrary Loan if need be.
      \end{itemize}
      One advantage of going to get the article yourself is that the volume
it
      is in may contain other relevant articles, e.g.\ commentaries on the
      article you have selected.
\item Make sure you have a copy of your main article.  This will need to be
      turned in with your report.
\end{itemize}

\vspace{0.2in}

\par\noindent{\bf Make sure that you really have a primary source.}\
Otherwise,
you could end up wasting a huge amount of time. If you are unsure whether
you have a primary source or not, come and ask me.

\vspace{0.2in}

\par\noindent{\bf Carefully read the article.}\ You are obligated to
understand the article in its entirety, with the exception of technical
statistical remarks which are beyond the scope of Math 203.  Look up words
you do not know, but your understanding should be deeper than simply being
able to repeat the meanings of the words.

\vspace{0.2in}

\par\noindent{\bf Analyze the article.}\ Ask lots of questions.  Look
carefully at the data itself.  Some things to consider are:
\begin{itemize}
\item What was the population and what was the sample?  How was the sample
      chosen and how big was it?
\item It is very, very important to understand as much as possible about how
      the data was gathered.  Have the authors told you enough about this?
\item Look at the data yourself.  Can you discern a pattern?
\item How might bias have occurred in the sampling?  The more bias you can
      identify, the better.  Were there confounding variables?
      Do not confuse bias with sample size or sampling variability.
\item In the case of an experiment, was there a control group?  Were the
      subjects randomly assigned?  Was the study double-blind?
\item Discuss confidence intervals.  Make your own independent calculation
      of them.
\item What are the conclusions of the study?  Do they seem justified?  How
      might the study have been improved?  Can you suggest a completely
      different approach to the same problem?
\end{itemize}

\vspace{0.2in}

\par\noindent{\bf Prepare your report.}\ Your critique should take the form
of
a well written, coherent, grammatically correct essay.
{\bf Above all, your job is to pose intelligent questions about the
methodology of the paper.}  Your paper should flow smoothly, and in
particular
should not simply repeat questions given above.
The essay should be single-spaced, typewritten, and {\it roughly\/} three
pages
in length.  Include careful citations for those
articles you have read.
Be sure to include your article(s) with your report when you turn it in.

\vspace{0.2in}

\par\noindent Your completed report will be due in class on Wednesday,  April 10.

\end{document}

