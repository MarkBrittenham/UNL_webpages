
\magnification=1200

\input amstex
\loadmsbm

\def\dsp{\displaystyle}

\centerline{\bf Math 107 Project: Balancing on the point of a pin}

\medskip

\centerline{Due date: Thursday, December 5}

\medskip

This project explores the mathematics behind the
{\it center of mass} (or {\it center of gravity}) of an object.

\medskip

In many physical situations, an object behaves as if all of its mass
were concentrated at a single point, called the center of mass (COM) of the
object. For example, an object allowed to rotate freely will rotate around 
a line through its 
center of mass; an object thrown through the air, in the absence of air 
resistance, will have its center of mass trace out 
the perfect parabolic arc that physics predicts. See, for example,

\smallskip

\centerline{http://www.schooltube.com/video/ef4699826e6448bf9703/Elmo-Center-of-Mass}

\smallskip

\noindent for experiments carried out with an Elmo doll! 
In this project we will focus on center of mass computations for an object
modeled as a thin plate of uniform density shaped like a region $R$
in the plane; under these hypotheses, the COM is usually called the {\it centroid}
of the region $R$.
For a region $R$ having a 
line of symmetry, the centroid will always lie along the line, a fact which 
can greatly simplify calculations of centroids. Knowlege of the centroid of an 
region, in turn, can greatly simplify other calculations; the Thoerem of Pappus
states that when a region $R$ of the plane is rotated  in space around a line not meeting
$R$, the volume of the resulting solid of revolution is equal to the area of $R$
times the distance traveled by the centroid $R$ under the rotation. Our goal 
is to verify some of these observations and carry out a variety of computations.

\bigskip

Some basic material on centers of mass can be found in section 8.4 of our text, pages
415-423, which makes a good starting point for your studies. To summarize, a collection
of masses $m_i$ distributed at the points $x_i$ along the real line will ``balance''
at the point $\overline{x}$ where 

\smallskip

\centerline{$\overline{x}\sum m_i = \sum x_im_i$, 
or, in a different form, $\sum m_i(\overline{x}-x_i)=0$}

\smallskip

\noindent This is essentially the principle of the lever; a small mass far from
the balance point can balance a larger mass close to the balance point but on the 
other side. $\overline{x}$ is the center of mass of the collection of masses.
More generally, a finite collection of masses $m_i$ distributed at points $(x_i,y_i)$
in the plane has
center of mass $(\overline{x},\overline{y})$, where 

\smallskip

\centerline{$\sum m_i(\overline{x}-x_i)=0$ and  $\sum m_i(\overline{y}-y_i)=0$}

\smallskip

$\overline{x}-x_i$ represents the ``signed'' distance from the point $(x_i,y_i)$
to the line $x=\overline{x}$; the condition $\sum m_i(\overline{x}-x_i)=0$ ensures
that the masses, if placed on a massless plate supported along the line 
$x=\overline{x}$, will balance. The other condition ensures that the masses 
balance when supported along the horizontal line $y=\overline{y}$.

\medskip

Even more is true: 
for $a,b\in{\Bbb R}$, if we set $c=a\overline{x}+b\overline{y}$,
then the distance from the mass $m_i$ sitting at the 
point $(x_i,y_i)$ to the line $ax+by=c$ is equal to 

\smallskip

\centerline{$\displaystyle {{|a(\overline{x}-x_i)+b(\overline{y}-y_i)|}\over{\sqrt{a^2+b^2}}}$ ,}

\smallskip

\noindent and so, if we define the `signed' distance from $(x_i,y_i)$
to the line $ax+by=c$ to be 

\smallskip

\centerline{$\displaystyle d_{a,b}(x_i,y_i) = {{a(\overline{x}-x_i)+b(\overline{y}-y_i)}\over{\sqrt{a^2+b^2}}}$ ,}

then $\sum m_id_{a,b}(x_i,y_i)=0$. So the collection of masses will also
balance when supported along any line through the center of mass.

\medskip

From this it follows that a collection of masses will ``balance'' on the point of a pin placed
at the center of mass, since they will not tip in any direction.

\medskip

Now assume we have a thin plate occupying a region $R$ as shown. Also, 
assume the density of the plate is a constant $\rho$ $kg/m^2$.  

\input epsf

\leavevmode

\epsfxsize=1.8in
\centerline{\epsfbox{comfig.eps}}

In order to find the centroid of the plate, we start by finding $\overline{x}$. 
We partition the interval  $[a,b]$  via the regular partition  $\{a=x_0, x_1, x_2,  \cdots , x_n=b\}$, 
with $\Delta x=\frac{b-a}{n}$. This process results in dividing the plate into thin vertical 
strips which can be approximated as a rectangle of a small width $\Delta x$. Let   $L(z_k)$ be 
the total length of the
line segments of intersection of the vertical line $x=z_k$ with $R$, where 
$z_k \in [x_{k-1}, x_k] $ is any point of your choice. Now, we think of each vertical strip 
of the plate as a discrete mass in the plane whose coordinate is $(z_k, w_k)$, for some 
$w_k\in {\Bbb R}$, which is irrelevant in the following calculations. Let us note that 
the mass of the kth vertical strip is given by:
$m_k= \text{ (density)(area)} \approx \rho L(z_k) \Delta x $. So, by thinking of the 
whole plate as a discrete system of $n$ masses $m_k  \approx \rho L(z_k) \Delta x$ each 
located at a point $(z_k, w_k)$ in the plane, we find
$$\overline{x}=\frac{M_y}{M} \approx \frac{\sum_{k=1}^n \rho z_k L(z_k) \Delta x} 
{\sum_{k=1}^n \rho  L(z_k) \Delta x}=\frac{\sum_{k=1}^n  z_k L(z_k) \Delta x} {\sum_{k=1}^n   L(z_k) \Delta x} .$$ 


\noindent By letting  $n\rightarrow \infty$, we obtain the formula \hskip.2in
$\displaystyle\overline{x}= \frac{\int_{a}^b x  L(x) dx}{\int_{a}^b   L(x) dx}.$
 
\smallskip

\noindent A similar collection of steps will obtain a formula for the $y$-coordinate of the 
centroid:

\smallskip

\centerline{$\dsp\overline{y}= \frac{\int_{c}^d y S(y) dy}{\int_{c}^d   S(y) dy}, $}

\smallskip

\noindent where $S(w_k)$ is the total length of the
line segments of intersection of the horizontal line $y=w_k$ with $R$.

\medskip

\noindent{\bf Problem \# 1:} Explain why the area $A(R)$ of the region $R$ is
equal to

\smallskip

\centerline{$\displaystyle A(R) = \int_a^b L(x)\ dx = \int_c^d L(y)\ dy$ .}

\smallskip

\noindent Hence we have:

\smallskip

\centerline{$\dsp\overline{x}= \frac{1}{A(R)} \int_{a}^b   xL(x) dx, \quad \overline{y}= \frac{1}{A(R)}\int_{c}^d  y S(y) dy$ .}

\smallskip

\noindent Use this to explain why, if the region $R$ has a vertical line of reflection
symmetry $x=A$, then $\overline{x}=A$, and if 
$R$ has a horizontal line of reflection symmetry $y=B$, then
$\overline{y}=B$. [Hint: a line of symmetry tells us something 
about the function $L(x)$ or $L(y)$. You might, to start, assume that $A=0$ and $B=0$.] 

\medskip

By computing $L(x)$ and $L(y)$ for specific examples, together with symmetry considerations,
we can compute the centroids of a wide variety of regions in the plane:

\medskip

\noindent{\bf Problem \# 2:} Compute the centroids of 

\smallskip

(a): the disk $D=\{(x,y)\ :\ (x+2)^2+y^2\leq 1\}$;

(b): the triangle with vertices $(1,0)$, $(5,0)$, and $(4,4)$;

(c): the triangle with vertices $(1,0)$, $(4,0)$, and $(5,4)$;

(d): the region lying between the parabolas $y=2x-x^2$ and $y=2x^2-4x$

\smallskip

\noindent [You may use familiar formulas for area to streamline your computations,
as well as any symmetries that you can identify.]

\medskip

In your answers for 3(b) and 3(c), if you compare the coordinates of the
centroids with the coordinates of the
vertices of the triangles, you may begin to suspect that there is a 
general formula for the centroid of a triangle, in terms of the 
coordinates of the vertices. Your final task is to find that formula,
by carrying out the integral computations for a triangle with
unknown values for its vertices.

\medskip

\noindent{\bf Problem \# 3:} Find a 
general formula for the centroid of
a triangle with vertices $(x_1,y_1)$, $(x_2,y_2)$, and $(x_3,y_3)$. 

\medskip

To do this, we may assume that the $x$-coordinates of the points are 
ordered 

\smallskip

\centerline{$x_1 \leq x_2 \leq x_3$}

\smallskip

\noindent (otherwise we would just rename them), but you 
should treat the two cases, where $(x_2,y_2)$ lies above or below the 
side joining the other two vertices, separately. You will, as with 3(b) 
and 3(c), need to find formulas for the lines joining the vertices, in
order to carry out your integral computations. You will also probably find it
helpful, while in the middle of your calculations, to introduce some shorthand,
like $\Delta_1=x_2-x_1$ or $\dsp m_3={{y_3-y_1}\over{x_3-x_1}}$, to keep the
readability of what you are doing to an acceptably high level!

\smallskip

You should, for this problem, simplify your answers as much as possible,
hopefully to the point where your answers for both cases look identical!

\bigskip

\noindent {\bf Guidelines for writing up your project.}

\medskip

The intent of projects is to expose you to mathematics as you might meet it 
in the real world, i.e., working
as a team. Your group must understand the problem; translate it into 
mathematics; learn, read about, or
develop mathematical methods to find the answer; show that the answer is 
correct; translate the mathematical
answer back into the original problem and, finally, explain the significance 
of the translated answer. Projects
are easier than real world problems, in that we make sure that the problem 
can be solved using the methods
of this course. You may need to learn some new information to do the project.

\medskip

The project is the solution to an open-ended multistep problem, formally 
presented. It will
probably require several meetings for your group to find a solution to the 
problem and to present that solution
clearly and understandably. Everyone in the group should contribute to the project.
\medskip

Your group should write up a short paper explaining the problem and the 
mathematics you
used to solve it, and then discussing the significance of your solution. 
Your paper should be a grammatically
correct, organized discussion of the problem, with an introduction and 
a conclusion. While you should
answer the specific questions asked in the project, your report should 
not be a disconnected set of answers
but a connected narrative with transitions. It should conform to proper 
English usage (yes, spelling counts!)
and should include appropriate diagrams and/or graphs, clearly labeled. 
You should show enough relevant
calculations to justify your answers but not so much as to obscure the 
calculations' purpose. If you type your report (this is preferred but
not required) it is fine to leave blank spaces and write the equations in.
[There are certain advantages to typing: making (small or large) 
changes does not require
the rewriting of the entire document!]
Explain your results and conclusions, pointing out both strengths and 
weaknesses of your analysis. Assume
that your reader is someone who took a calculus class course a while 
ago and does not remember all
of the details. Be sure to avoid plagiarism.

\medskip

Preparing formal reports is an important job skill for mathematicians, 
scientists, and engineers. For example,
the Columbia Investigation Board, in its report on the causes of the 
Columbia space shuttle accident, wrote:

\medskip

{\it ``During its investigation, the board was surprised to receive 
[PowerPoint] slides from NASA
officials in place of technical reports. The board views the endemic 
use of PowerPoint
briefing slides instead of technical papers as an illustration 
of the problematic methods of
technical communication at NASA.''}

\bigskip

For {\bf extra credit}, you can build models of some of the regions described in 
Problem \# 2, out of some relatively stiff material, and demonstrate 
that they balance on the head of a pin placed at the centroid that you 
computed! [This can also serve as an independent check that your computed 
solution is correct!]

\vfill
\end


