\magnification=1200

\def\ctln{\centerline}
\def\ni{\noindent}
\def\ssk{\smallskip}
\def\msk{\medskip}
\def\bsk{\bigskip}
\def\hsk{\hskip.2in}
\def\hhsk{\hskip.3in}
\def\ra{$\rightarrow$}
\def\ubr{\underbar}

\baselineskip=10pt
\nopagenumbers
\parindent=0pt

\ctln{\bf Math 107H}

\msk

\ctln{\bf The method of partial fractions}

\bsk

In calculus, we
repeatedly find ourselves needing to integrate {\it rational functions}, that
is, quotients of polynomials. There a general method 
of finding such integrals (which isn't really so much about the integrals
as it is the function themselves!), called {\it partial fractions}.

\bsk

The basic idea is that we will integrate a rational function

\ctln{$\displaystyle f(x) = {{p(x)}\over{q(x)}} = {{a_0x^n+\cdots +a_{n-1}x+a_n}\over{b_0x^m+\cdots +b_{m-1}x+b_m}}$}

by writing it as a sum of simpler functions.

\msk

The main fact that will make this (at least in theory, and often in practice) possible is that {\it any
polynomial with real coefficients can (in principle) be expressed as the product of linear and (irreducible)
quadratic polynomials}. By factoring the quotient polynomial $q(x)$ in this way, we can determine
what the simpler functions will look like; a procedure very much like undetermined coefficients will
then allow us to determine exactly what the simpler pieces are.

\msk

The basic procedure goes like this: starting with  $f(x)$ = $\displaystyle{{p(x)}\over{q(x)}}$

\msk

(0): Make sure that degree($p$)$<$degree($q$); do polynomial long division if it isn't. I.e., write
$p(x) = a(x)q(x)+b(x)$, with degree($b$)$<$degree($q$), and then 

\ssk

\ctln{$\displaystyle {{p(x)}\over{q(x)}} =  a(x) + {{b(x)}\over{q(x)}}$,}

\ssk

and we can integrate $a(x)$, since it is a polynomial. 

\msk

(1): Factor $q(x)$ into linear and irreducible quadratic factors. This is the only step where 
we really have no general procedure (basically, because there {\it is} none). As with the 
auxiliary equations for higher order 
DE's, we can try to find roots of the polynomial to determine linear factors, and there
are procedures (like the rational roots theorem) for determining good candidates. And a 
good computer algebra system can give us good approximations to roots (and quadratic factors).

\msk

(2): Group common factors together as powers; if, e.g., 3 is a root of $q(x)$ four times, then
we treat the four factors in what follows as giving one factor of $(x-3)^4$.

\msk

(3a): For each group $(x-a)^k$, we add together:

\ssk

\ctln{$\displaystyle{{a_1}\over{x-a}}+\cdots +{{a_k}\over{(x-a)^k}}$}

\ssk

These are the simpler pieces that the factor $(x-a)^k$ will contribute to the final sum.

\msk

(3b): For each group $(ax^2+bx+c)^k$, we add together:

\ssk

\hhsk $\displaystyle{{c_1x+b_1}\over{ax^2+bx+c}}+\cdots +{{c_kx+b_k}\over{(ax^2+bx+c)^k}}$

\msk

(4)  Set $f(x)$ = the sum of all of these sums; solve the resulting equation for the `undetermined' coefficients $a_i,c_j$ , etc.

\msk

Note that we need to use {\it different} names for the coefficients, for each piece! 

\msk

Showing {\it why} this procedure works (i.e., why a rational function can always be expressed as such a
sum) would take us too far afield (and isn't even really about integration!), so we will content ourselves 
to just use this remarkable fact. There are two basic methods for carrying out step (4), to solve for the
undetermined coefficients:

\msk

In both, we put the entire sum over a common denomenator (which, it turns out, will 
(almost: up to multiplication by a constant) be equal to $q(x)$, if you put things over 
the {\it smallest} common denomenator) and set the 
resulting numerator equal to (the appropriate constant multiple of) $p(x)$.

\ssk

(a) (this always works): Multiply out the numerator to a single polynomial, and set 
the coeffs of the two polynomials equal to one another. (This works because
two polynomials are the same precisely when they have the same degree and their
coefficients are equal to one another.) This gives us a system of linear equations
involving the unknown coefficients, which we can solve.

\ssk

\hskip.4in Ex: $x+3 = a(x-1)+b(x-2)$ = $(a+b)x+(-a-2b)$; solve $1=a+b$, $3=-a-2b$

\msk

(b) Don't multiply out the numerator! Leave it as a sum of products of terms from 
the denomenators. We can determine many of the unknown coefficients by plugging 
well-chosen values in for $x$.

\ssk

For each linear term $(x-a)^k$,  plug $x=a$ into both sides. Most of the 
terms of the sum will have a factor of $(x-a)$ and so will give zero, which will 
allow us to quickly solve for one of the coefficients.

\ssk

If $k\geq 2$, take derivatives of both sides! Then by plugging in x=a, 
we will quickly solve for another coefficient.

\msk

Ex: $\displaystyle{{x^2}\over{(x-1)^2(x^2+1)}}$ = $\displaystyle {{A}\over{x-1}}+{{B}\over{(x-1)^2}}+{{Cx+D}\over{x^2+1}}$

\ssk

\hskip2in $= \displaystyle{{A(x-1)(x^2+1)+B(x^2+1)+(Cx+D)(x-1)^2}\over{(x-1)^2(x^2+1)}}$

\ssk

so $A(x-1)(x^2+1)+B(x^2+1)+(Cx+D)(x-1)^2$ = $x^2$.

\msk

Set $x=1$, get $2B = 1$, solve for $B$.

\msk

Take derivatives: $A(x^2+1)+A(x-1)(2x)+B(2x)+C(x-1)^2+2(Cx+D)(x-1) = 2x$

\ssk

Set $x=1$, get $2A+0+2B+0+0=2$, solve for $A$ (since we already know $B$)

\bsk

The end result of this process is a an expression for $\displaystyle {{p(x)}\over{q(x)}}$ as a sum 
of rational functions of the form 

\ssk

\ctln{$\displaystyle{{a_i}\over{(x-a)^i}}$ and $\displaystyle{{c_ix+b_i}\over{(ax^2+bx+c)^i}}$}

\msk

These can then be integrated one at a time; integrating the second type can be simplified by
writing 

\ssk

\ctln{$\displaystyle{{c_ix+b_i}\over{(ax^2+bx+c)^i}} = {{c_i}\over{2a}}{{2ax+b}\over{(ax^2+bx+c)^i}} + 
(b_i-{{c_ib}\over{2a}})\displaystyle{{1}\over{(ax^2+bx+c)^i}}$}

\msk


The first part can be integrated by substitution $u=ax^2+bx+c$ ; the second involves completing the square 
(to write it as a sum of squares $a[(x-\alpha)^2+\beta^2]$) which we can then integrate by $u-$substitution 
$u=x-\alpha$ , followed by trigonometric substitution $u=\beta\tan w$; this gives us a collection of
powers of $\cos w$ to integrate.

\vfill\end



