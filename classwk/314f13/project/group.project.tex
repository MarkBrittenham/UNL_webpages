
\magnification=1200

\input amstex
\loadmsbm

\def\dsp{\displaystyle}
\def\ssk{\smallskip}
\def\msk{\medskip}
\def\bsk{\bigskip}
\def\ctln{\centerline}
\def\nidt{\noindent}
\def\ubr{\underbar}


\centerline{\bf Math 314 Group Project: Fun with polynomials}

\medskip

\centerline{Due date: Wednesday, December 4}

\medskip

This project explores methods for constructing bases for the
vector space ${\Cal P}_n$ of polynomials of degree up to $n$,
and for verifying that collections of polynomials are bases
for ${\Cal P}_n$. Depending upon the set of polynommials
you will use different characterizations of bases that we 
have developed in class. These bases arise in different
contexts that you have met and/or will meet in your 
mathematics and other courses.

\medskip

We know that ${\Cal P}_n$ has the ``standard'' basis consisting
of the polynomials (with one term, usually called {\it monomials})

\ssk

\ctln{$\{1,x,x^2,\ldots,x^n\}$}

\ssk

\nidt because they span ${\Cal P}_n$ - the definition, esssentially, 
of polynomials as the functions $p(x)=a_0x^n+a_1x^{n-1}+\cdots +a_{n-1}x+a_n$ 
express polynomials as linear combinations of monomials. The monomials are also linearly independent:
If $p(x)=a_0x^n+a_1x^{n-1}+\cdots +a_{n-1}x+a_n=0$, then plugging in $x=0$ gives
$p(0)=a_n=0$, so $p(x)=a_0x^n+a_1x^{n-1}+\cdots +a_{n-1}x=x[a_0x^{n-1}+a_1x^{n-2}+\cdots +a_{n-1}]=0$,
so $a_0x^{n-1}+a_1x^{n-2}+\cdots +a_{n-1}=0$, as well. Repeating this argument (plugging in $x=0$)
will repeatedly show another coefficient is equal to zero, resulting in 
$a_n=a_{n-1}= \cdots = a_1=a_0=0$, establishing linear independence.

\msk

But what you may not realize is that you have \ubr{already} encountered \ubr{other} bases for these
vector spaces ${\Cal P}_n$. In particular, in second-semester calculus, the (integration) technique
we call {\it partial fractions} made systematic use of other, alternative, bases for the vector
space ${\Cal P}_n$, although of course we did not use this terminology for it at the time!
In this project, we will explore the construction and verification of some of these
bases.

\msk

A refresher on partial fraction decompositions can be found at

\ssk

\ctln{http://www.math.unl.edu/$\sim$mbrittenham2/classwk/314f13/project/partial.pdf}

\ssk

\nidt (which is based on the review notes I provided to one of my Calculus II classes). The basic 
point is that starting from a completely factored polynomial

\msk

\ctln{$q(x) = (x-a_1)^{m_1}(x-a_2)^{m_2}\cdots 
(x-a_k)^{m_k}(x^2+\alpha_1x+\beta_1)^{n_1}\cdots(x^2+\alpha_\ell x+\beta_\ell)^{n_\ell}$,}

\msk

\nidt where the quadratic terms are irreducible, with $\text{degree}(q) =N = m_1+\cdots m_k+2n_1+\cdots 2n_\ell$,
partial fractions gives a prescription for writing any quotient $\dsp{{p(x)}\over{q(x)}}$, with
$\text{degree}(p)\leq N-1$ as a sum of the quotients 

\ssk

\ctln{$\dsp {{A}\over{(x-a_i)^m}}$ for $1\leq m\leq m_i$, and $\dsp{{Bx+C}\over{(x^2+\alpha_jx+\beta_j\ell)^n}}$ 
for $1\leq n\leq n_j$.}

\msk

What this \ubr{really} means, when we put things over a common denominator, is that $p(x)$ is a sum of 
constant multiples of the polynomials (the denominators divide the numerator)

\msk

\ctln{$\dsp {{q(x)}\over{(x-a_i)^m}}$ for $1\leq m\leq m_i$,  
$\dsp {{q(x)}\over{(x^2+\alpha_jx+\beta_j\ell)^n}}$, and $\dsp x{{q(x)}\over{(x^2+\alpha_jx+\beta_j\ell)^n}}$
for $1\leq n\leq n_j$,}

\msk

\nidt that is, these polynomials \ubr{span} ${\Cal P}_{N-1}$. Noting that there are precisely

\ctln{$m_1+\cdots m_k+2n_1+\cdots 2n_\ell = N = \text{dim}({\Cal P}_{N-1})$}

\nidt polynomials in this list,
this implies that these polynomials are 
a \ubr{basis} for ${\Cal P}_{N-1}$. So partial fractions
is really all about constructing a particular basis for ${\Cal P}_{N-1}$ from a 
single degree $N$ polynomial $q(x)$.

\bsk

In the following exercises you will verify that for some choices of $q(x)$ these polynomials do 
indeed form a basis for ${\Cal P}_{N-1}$,
using the techniques that were developed for solving partial fractions problems, together with our
knowledge that establishing linear independence can be as useful to verifying bases as showing that they span can.

\ssk

We start with the technique usually called {\it plugging in}: Choosing specific values to 
evaluate an expression at tells us the value of one or more coefficients.

\msk

{\bf Problem 1.} For the polynomial $q(x)=(x-1)(x-2)(x-3)(x-4)$, with degree 4, describe the basis for 
${\Cal P}_{3}$ that partial fractions asserts we have, and demonstrate that your collection of polynomials
are {\it linearly independent}; that is, that 

\ctln{$a_1p_1(x)+a_2p_2(x)+a_3p_3(x)+a_4p_4(x)=0$ implies
that $a_1=a_2=a_3=a_4=0$.}

\nidt Explain why this implies that your collection of polynomials is therefore
a basis for ${\Cal P}_{3}$. 

\msk

{\bf Problem 2.} Explain how the same technique can be applied to the general situation where
$q(x)=(x-a_1)(x-a_2)\cdots(x-a_k)$ with $a_1<a_2<\cdots <a_k$.

\bsk

A second technique might be called {\it moving the origin}: by writing $x=(x-a)+a$, we can write powers of $x$
as linear combinations of powers of $(x-a)$; e,g, by multiplying out $x^6=[(x+1)-1]^6$.

\msk

{\bf Problem 3.} For the polynomial $q(x)=(x-1)^4$, with degree 4, describe the basis for 
${\Cal P}_{3}$ that partial fractions asserts we have, and demonstrate that your collection of polynomials
\ubr{spans} ${\Cal P}_{3}$, by showing that elements of the `standard' basis for ${\Cal P}_{3}$ can be written
as linear combinations of your partial fractions basis. 

\msk

{\bf Problem 4.} Explain how the same technique can be applied to the general situation where
$q(x)=(x-a)^n$, for $a\in{\Bbb R}$ a constant and $n\geq 1$ an integer. Can you come up with an 
alternate approach which directly establishes that your polynomials are linearly independent (i.e., 
which does not first establish that they span)?

\bsk

Next, we look at what happens when we have a combination of these two situations. 

\msk

{\bf Problem 5.} For the polynomial $q(x)=(x-1)^2(x-2)^2$, with degree 4, describe the basis for 
${\Cal P}_{3}$ that partial fractions asserts we have, and demonstrate that your collection of polynomials
forms a basis for ${\Cal P}_{3}$. You might try various combinations of the approaches explored
in the previous problems (in some order), and/or other ideas, to see what works (best)!

\msk

{\bf Problem 6.} Describe how your approach to problem \#5 could be adapted to some more general
situation. Try to describe the most general situation that you can manage to solve in a reasonably small
amount of space (and time).

\bsk

When irreducible quadratics appear in our factorization of $q(x)$, things get more `interesting'.
We no longer have a clear idea of what values to plug in to reveal information about coefficients 
in a linear combination, and there is no clear value to `shift the origin' by to express our 
standard basis as linear combinations. But we can always apply the `brute force' method, which is 
really just solving a system of linear equations! For example, by writing $a(x-1)+b(x-2)=0$
as $(a+b)x+(-a-2b)=0$, solving $a+b=0$ and $-a-2b=0$ for $a$ and $b$ establishes that
$x-1$ and $x-2$ are linearly independent.

\msk

{\bf Problem 7.} For the polynomial $q(x)=(x^2+2x+2)(x^2-x+2)$ [note that
both factors are irreducible quadratics], with degree 4, describe the basis for 
${\Cal P}_{3}$ that partial fractions asserts we have, and show that your collection of polynomials
forms a basis for ${\Cal P}_{3}$. You can show either that they are linearly independent, or
that they span (by showing that you can express another basis as linear combinations of them).

\msk

In practice, establishing that partial fractions `works', even in the presence of irreducible
quadratices, uses a standard trick of thinking of the polynomials as having {\it complex} coefficients,
and then further factoring them into linear polynomials. Facts about complex numbers then are
used to make certain that our solutions still have {\it real} coefficients (the phrase `complex
conjugate pairs' might mean something to you?). But exploring this further would take us too far afield,
and so we will close our explorations here.

\bsk 

\noindent {\bf Guidelines for writing up your project.}

\medskip

The project is the solution to an open-ended multistep problem, formally 
presented. It will
probably require several meetings for your group to find a solution to some of the 
problems and to present your solutions
clearly and understandably. Everyone in the group should contribute to the project.

\medskip

Your group should write up a short paper explaining the problem and the 
mathematics you
used to solve it, and then discussing the significance of your solution. 
Your paper should be a grammatically
correct, organized discussion of the problem, with an introduction and 
a conclusion. While you should
answer the specific questions asked in the project, your report should 
not be a disconnected set of answers
but a connected narrative with transitions. It should conform to proper 
English usage (yes, spelling counts!). 
You should show enough relevant
calculations to justify your answers but not so much as to obscure the 
calculations' purpose. If you type your report (this is preferred but
not required) it is fine to leave blank spaces and write the equations in.
[There are certain advantages to typing: making (small or large) 
changes does not require
the rewriting of the entire document!]
Explain your results and conclusions, pointing out both strengths and 
weaknesses of your analysis. Assume
that your reader is someone who took a linear algebra course a while 
ago and does not remember all
of the details. Be sure to avoid plagiarism.

\medskip

Preparing formal reports is an important job skill for mathematicians, 
scientists, and engineers. For example,
the Columbia Investigation Board, in its report on the causes of the 
Columbia space shuttle accident, wrote:

\medskip

{\it ``During its investigation, the board was surprised to receive 
[PowerPoint] slides from NASA
officials in place of technical reports. The board views the endemic 
use of PowerPoint
briefing slides instead of technical papers as an illustration 
of the problematic methods of
technical communication at NASA.''}


\vfill
\end