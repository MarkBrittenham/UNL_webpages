
\input amstex


\magnification=1200


\loadmsbm

\nopagenumbers
%\parindent=-15pt
%\voffset=-.6in
%\hoffset=-.4in
%\hsize = 7.5 true in
%\vsize=10 true in
\overfullrule=0pt


\def\ctln{\centerline}
\def\u{\underbar}
\def\ssk{\smallskip}
\def\msk{\medskip}
\def\bsk{\bigskip}
\def\dl{\displaystyle}
\def\ni{\noindent}
\def\ep{\epsilon}


\ctln{\bf Math 445 Number Theory}

\medskip

\ctln{September 24, 2008}

\bigskip

The Quadratic Sieve: speeding it up.

\ssk

Our setup: $n$ (odd) composite. $a=\lfloor\sqrt{n}\rfloor\rfloor$, $(a+k)^2-n=a_k$, and we collect the $a_k$ which completely
factor into primes in our factor base consisting of primes $\leq B$, our bound.

\ssk

To find a product of $a_i$'s which is a square, Brillhart and Morrison in the 1970's introduced some
linear algebra.
Each $a_k$ that we keep is completely determined by the vector of exponents $(\ep_1,\ldots,\ep_r)$ from its prime
factorization $a_k=p_1^{\ep_1}\cdots p_r^{\ep_r}$ over our factor base, so we can work with these vectors.
Our goal is a product of $a_i$ with all exponents even; so what we want is a sum of these
exponent vectors with all entries of the sum even. Since we are only interested in even versus odd, we can make a further
simplification and keep only this parity information, setting $\eta_i=\ep_i$ (mod 2), and using the
vector $(\eta_1,\ldots \eta_r)$. Then what we seek is a collection of $a_i$'s so that their mod 2 exponent vectors
$(\eta_1,\ldots \eta_r)$ sum, mod 2, to the 0-vector. If we think of these vectors as lying in the vector space
${\Bbb Z}_2^r$, what we are looking for is a linear dependence among the mod 2 exponent vectors. Such a dependence
leads directly to a collection of $a_i$ (those corresponding to the vectors in the dependence with coefficient 1)
whose product is a square. 

\ssk

But! We know that $r+1$ vectors in an $r$-dimensional space (like ${\Bbb Z}_2^r$) \underbar{must} be linearly dependent!
this tells us how many $a_i$'s we must find the factor completely over our factor base before we can find a
dependency; we need one more $a_i$ than the number of primes in our factor base. Then we can find the
linear dependency by arranging our mod 2 exponent vectors as the columns of a matrix, and (by row reduction!)
compute the nullspace of the matrix. (Row reduction works perfectly fine over ${\Bbb Z}_2$, since it is a field.)
This significantly speeds up finding the product of $a_i$'s which is a square.

\ssk

In-class example: factoring $n=3920053$.

\ssk

The final speedup for this process is to make finding the $a_i$ that completely factor over the factor base more efficient.
This is due to Carl Pomerance, in the early 1980's. The idea is to adapt the Sieve of Eratosthenes, which looks for primes,
to instead look for number that factor into small primes. So instead of circling the smallest untouched number 
(which is the next prime $p$) and crossing off the multiples of that prime - this crosses off precisely what we are looking for! -
we divide in place every number that is a multiple of $p$ by $p$. The idea is that If we repeatedly cycle
through the list, dividing multiples of 2 by 2 (i.e., every other number), then every multiple of 4 by \underbar{another} 2
(i.e., every 4th number), and divide all multiples of 3 by 3, etc., the numbers which are products of small primes
 will be reduced to 1, telling us \underbar{precisely} where they are!



\vfill\end







