
\input amstex


\magnification=1400


\loadmsbm

\nopagenumbers
\parindent=0pt

\voffset=-.6in
\hoffset=-.4in
\hsize = 7.5 true in
\vsize=10 true in

%\voffset=1.2in
%\hoffset=-.5in
%\hsize = 10.2 true in
%\vsize=8 true in

\overfullrule=0pt


\def\ctln{\centerline}
\def\u{\underbar}
\def\ssk{\smallskip}
\def\msk{\medskip}
\def\bsk{\bigskip}


\ctln{\bf Math 445 Number Theory}

\medskip

\ctln{September 22-24, 2008}

\bigskip

One tool that we need to add to our toolbox is the existence of {\it primitive roots of 1 mod a prime $p$}:
that is, the existence of integers $a$ for which ord$_p(a) = p-1$ . In the language of groups, this says 
that the group of units in ${\Bbb Z}_p$ is
cyclic, when $p$ is prime. In order to prove this, 
we need a bit of machinery:

\msk

{\it Lagrange's Theorem:} If $f(x)$ is a polynomial with integer coefficients, of degree $n$, and $p$ is prime,
then the equation $f(x)\equiv 0 \pmod{p}$ has at most $n$ mutually incongruent solutions, unless $f(x)\equiv 0 \pmod{p}$
for \underbar{all} $x$.

\msk

To see this, do what you would do if you were proving this for real or complex roots; given a solution $a$, write
$f(x)=(x-a)g(x)+r$ with $r$=constant (where we understand this equation to have coefficients in ${\Bbb Z}_p$) 
using polynomial long division. This makes sense because ${\Bbb Z}_p$ is a {\it field}, so division by non-zero
elements works fine. Then $0=f(a)=(a-a)g(a)+r=r$ means $r=0$ in ${\Bbb Z}_p$, so $f(x)=(x-a)g(x)$ with $g(x)$
a polynomial with degree $n-1$ . Structuring this as an induction argument, we can assume that $g(x)$ has at most
$n-1$ roots, so $f$ has at most ($a$ and the roots of $g$, so) $n$ roots, because, 
{\it since $p$ is prime}, if $f(b)=(b-a)g(b)\equiv 0\pmod{p}$, then either $b-a\equiv 0$ 
(so $a$ and $b$ are congruent mod $p$), or $g(b)=0$, so $b$ is among the roots of $g$.

\msk

This in turn leads us to 

\ssk

{\it Corollary:} If $p$ is prime and $d|p-1$ , then the equation $x^d-1\equiv 0\pmod{p}$ has
{\it exactly} $d$ solutions mod $p$.

\msk

This is because, writing $p-1=ds$, $f(x)=x^{p-1}-1\equiv 0$ has exactly $p-1$ solutions (namely, 1 through $p-1$), and
$x^{p-1} = (x^d-1)(x^{d(s-1)}+x^{d(s-2)}+\cdots +x^d+1) = (x^d-1)g(x)$ . But $g(x)$ has {\it at most} $d(s-1)=(p-1)-d$
roots, and $x^d-1$ has at most $d$ roots, and together (since $p$ is prime) they make up the $p-1$ roots of $f$. So in
order to have enough, they both must have {\it exactly} that many roots.

\msk

We introduce the notation 
$p^k||N$, which means that $p^k|N$ but $p^{k+1}\not |N$ .

\ssk

For each prime $p_i$ dividing $n-1$, $1\leq i\leq s$, we let $p_i^{k_i}||n-1$ . 
Then the equation 
(*) $x^{p_i^{k_i}}\equiv 1\pmod{n}$ has $p_i^{k_i}$ solutions, while
(\dag) $x^{p_i^{k_i-1}}\equiv 1\pmod{n}$ has only $p_i^{k_i-1}<p_i^{k_i}$ solutions; 
pick a solution, $a_i$ to (*) which is not a solution to (\dag) . 
[In particular, ord$_n(a_i)=p_i^{k_i}$.] 
Then set $a=a_1\cdots a_s$ . 
Then a computation yields that, mod $n$, 
$\displaystyle a^{{n-1}\over{p_i}} \equiv a_i^{{n-1}\over{p_i}}\not\equiv 1$, since otherwise
ord$\displaystyle _n(a_i)|{{n-1}\over{p_i}}$, and so 
ord$\displaystyle _n(a_i)|\gcd(p_i^{k_i},{{n-1}\over{p_i}})=p_i^{k_i-1}$ , a contradiction.
So $p_i^{k_i}||$ord$_n(a)$ for every $i$, so 
$n-1|$ord$_n(a)$, so ord$_n(a)=n-1$.

\msk

This result is fine for theoretical purposes (and we will use it many times), but it is somewhat less than satisfactory for computational
purposes; this process of {\it finding} such an $a$ would be very laborious.


\vfill\end







