
\input amstex


\magnification=1200


\loadmsbm

\nopagenumbers
%\parindent=-15pt
%\voffset=-.6in
%\hoffset=-.4in
%\hsize = 7.5 true in
%\vsize=10 true in
\overfullrule=0pt


\def\ctln{\centerline}
\def\u{\underbar}
\def\ssk{\smallskip}
\def\msk{\medskip}
\def\bsk{\bigskip}
\def\dl{\displaystyle}
\def\ni{\noindent}
\def\ep{\epsilon}


\ctln{\bf Math 445 Number Theory}

\medskip

\ctln{October 1, 2008}

\bigskip

The Quadratic Sieve: the sieving process.

\ssk

The sieving process involves looking at numbers $b=a^2-n$ for a range of values of $a$,
deciding when they are 
divisible by a small prime $p$, then replacing $a$ with $a/p$ if it is and moving on.
But this amounts to deciding (quickly) when $n$ is a square mod $p$, and for which values 
of $a$ is $a^2\equiv n$. Deciding if $n$ is a square mod $p$ \underbar{can} be done quickly
(and note that if the answer is ``no'' then we needn't bother placing $p$ in our
facotr base: it will never play a role in a smooth number), using the technique of 
quadratic reciprocity, which we will esplore later. And if $n$ is a square then 
there will be two values $a_1,a_2$ (since $x^2\equiv n$ will have two solutions mod $p$)
so that $p|a^2-n$ $\Leftrightarrow$ $a\equiv a_1,a_2$ (mod $p$), and we can find the $a_i$,
for smallish $p$, by a brute force search. Then we \underbar{know} which $a^2-n$ 
to divide by $p$; in our sequence they form two sets of subsequences which jump 
along by $p$, and we can quickly focus on just those terms that have a facotr of $p$
to divide out. So in the end, the sieving process looks exactly like the prime sieve
(we just start at different points and do it twice for each prime...).

\msk

For a complete change of topic (before coming back to look at quadratic residues $x^2\equiv a$ (mod $p$)),
we will take a look at Pythagorean Triples:

\msk

{\it Pythagorian triples:} If $a^2+b^2=c^2$, then we call $(a,b,c)$ a Pythagorean triple. Their connection to right triangles is
well-known, and so it is of interest to know what the triples are! It is fairly straighforward to generate a lot of them (e,g, via
$(n+1)^2=n^2+(2n+1)$, so any odd square $k^2=2n+1$ can be used to build one). But to find them all takes a bit more work:

\msk

A Pythagorean triple $(a,b,c)$ is {\it primitive} if the three numbers share no common factor. This is equivalent, in this case, to
$(a,b)=(a,c)=(b,c)=1$ . Then by considering the equation mod 4, we can see that for a primitive triple, $c$ must be odd,
$a$ (say) odd and $b$ even. If we then write the equation as $b^2=c^2-a^2=(c+a)(c-a)$, we find that we have factored
$b^2$ in two different ways. Since $b,a+c$ and $a-c$ are all even, we can write $(b/2)^2=[(c+a)/2]^2[(c-a)/2]^2$ But
because $(c+a)/2 +(c-a)/2 = c$ and  $(c+a)/2 -(c-a)/2 = b$, $\gcd((c+a)/2,(c-a)/2)=1$ . Then we can apply:

\msk

{\it Proposition:} If $(x,y)=1$ and $xy=c^2$, then $x=u^2,y=v^2$ for some integers $u,v$ .

\msk

Proof: next time


\vfill\end







