
\magnification=1200
\nopagenumbers

\parindent=0pt

\voffset=-.6in
\vsize=10in

\overfullrule=0pt

\def\ctln{\centerline}
\def\u{\underbar}
\def\msk{\medskip}
\def\bsk{\bigskip}


\input epsf

\ctln{\bf Math 445 Number Theory}

\medskip

\ctln{August 23 and 25, 2004}

\bigskip

Number theory is about {\it finding} and {\it explaining}
patterns in numbers.

\bigskip

Ulam Sprial:

\medskip

\vbox{\hsize=1.6in

\leavevmode

\epsfxsize=1.6in
\epsfbox{ulam.ai}}


\vskip-1.6in

\hskip2.6in
\vbox{\hsize=2.2in Place the natural numbers in a
rectangular spiral. The primes tend to fall on certain
diagonal lines with more frequency than it seems they should?

\medskip

This means: for certain values of $\alpha,\beta,\gamma$, the sequences
$n^2+\alpha$ , $n^2+n+\beta$ , $n^2-n+\gamma$ have more primes than
we {\it expect} them to.

\medskip

{\it Why?} We don't yet know...}



\bigskip

Egyptian fractions:

Any rational number $m/n$ can be written as a sum of reciprocals $1/a$ of
integers. In fact, by repeatedly subtracting the largest reciprocal that we
can from whatever is left, we find that 

\ctln{$\displaystyle {{m}\over{n}} = {{1}\over{a_1}} + \cdots + {{1}\over{a_k}}$}

with $a_1<a_2< \ldots < a_k$ and $k\leq n$. But not every fraction $3/n$ can 
be expressed as a sum of {\it two} reciprocals (e.g, 3/7). However, it is
conjectured (the Erd\"os-Strauss Conjecture) that

\ctln{every fraction $\displaystyle {{4}\over{n}}$ is the sum of at most 3 reciprocals.}

This has been verified to $n=10^{14}$, but still remains open.

\medskip

These sorts of expressions actually occur in engineering: If resistors with 
resistances $r_1,\ldots,r_n$ are wired in parallel, they act as a single
resistor with resistance $r$, where 

\ctln{$\displaystyle {{1}\over{r}}={{1}\over{r_1}}+\cdots +{{1}\over{r_n}}$}

\noindent so a resistor with `custom' resistance $r$ can be `manufactured' from 
a set of standard resistors by solving such equations.

\msk

{\bf Perfect numbers:} A number $N$ is perfect if it is equal to the sum of
its proper divisors. Euler showed that every even perfect number must be 
of the form $N=2^n(2^{n+1}-1)$ where $p=2^{n+1}-1$ is \u{prime}. Such
primes are known as {\it Mersenne} primes; there are only 44 currently known
Mersenne primes, including the 4 (or so?) largest known primes. It is still
an open question whether or not there exists an odd perfect number.

\smallskip

If the divisors of $N$ are $1=a_0,a_1,\ldots,a_n=N$ in order, then to be
perfect we need $N=a_0+\cdots +a_{n-1}$. Dividing by $N$ yields

\ctln{$\displaystyle 1={{1}\over{a_n}}+\cdots +{{1}\over{a_1}}$}

\noindent and so techinques from Egyptian fractions can be employed.
For example, it is known that the largest denomenator, $a_n=N$ must
be $\leq u_n$ where the $u_i$ are a fixed sequence defined by $u_1=1$
and $u_{i+1}=u_i(u_i+1)$. These provide an upper bound on the size of 
an odd perfect number with exactly $k$ factors. Techniques such as this
and others have enabled researchers to show that any odd perfect number
(if one exists) has at least 300 digits, at least 9 distinct prime
factors, and at least 57 prime factors in all......

\vfill\end






