
\magnification=1200
\nopagenumbers

%\nopagenumbers

\def\nidt{\noindent}
\def\itt{item}
\def\ctln{\centerline}
\def\u{\underbar}
\def\ssk{\smallskip}
\def\msk{\medskip}
\def\bsk{\bigskip}
\def\hsk{\hskip.1in}
\def\hhsk{\hskip.2in}
\def\dsl{\displaystyle}

\input amstex
\loadmsbm


\ctln{\bf Math 445}

\msk

\ctln{\bf Take-home Exam (Exam 1)}

\msk

{\narrower Due in class on Wednesday, October 29. You are not to discuss
the exam, except on trivial matters, with anyone other than the instructor, 
until after you have (both!) turned in your solutions. The problems are given in 
approximately the order in which the material was presented
in class; this is not necessarily a recommendation to work the problems 
in that order. Each needed computation should be carried out in full.\par}

\msk

\noindent {\bf Show all work.}
How you get your answer is just 
as important, if not more important, than the 
answer itself. If you think it, write it!

\bsk


\item{1.} (20 pts.) Find the period of the repeating decimal expansion of $1/53$ (by computing the order of
the appropriate integer mod the other appropriate integer). 

\bsk

\item{2.} (20 pts.) Show that if $a$ and $b$ are both primitive roots of unity modulo
the odd prime $p$, then $ab$ \underbar{cannot} be a primitive root of unity modulo $p$.

\ssk

\item{} [Hint: a specific small(er) power of $ab$ will demonstrate this...]

\bsk

\item{3.} (20 pts.)  Show that if $n\equiv 1$ (mod 8), then there are \underbar{no} 
integers $x$ and $y$ such that $2x^3-3y^2=n$ . That is, for $n\equiv 1$ (mod 8),
the equation 

\ssk

\ctln{$2x^3-3y^2=n$}

\ssk

\item{} has no solutions in the integers.

\ssk

\item{} [Hint: what values, mod 8, can $3y^2$ take?]


\bsk

\item{4.} (20 pts.) Determine the number of solutions to the power residue equations

\msk

\hskip.2in {(a)} $x^3\equiv 2$ (mod 13)

\msk

\hskip.2in {(b)} $x^4\equiv 2$ (mod 13)

\msk

\hskip.2in {(c)} $x^5\equiv 2$ (mod 13)

\bsk

\item{5.} (20 pts.) Use the RESSOL algorithm to find a solution to the quadratic residue 
equation

\ssk

\ctln{$x^2\equiv 3$ (mod 73) .}

\ssk

\item{} [FYI: 5 is a quadratic non-residue mod 73. (It also happens to be a primitive root.)]


\vfill\end



