
\magnification=1200
\nopagenumbers

%\nopagenumbers

\def\nidt{\noindent}
\def\itt{item}
\def\ctln{\centerline}
\def\u{\underbar}
\def\ssk{\smallskip}
\def\msk{\medskip}
\def\bsk{\bigskip}
\def\hsk{\hskip.1in}
\def\hhsk{\hskip.2in}
\def\dl{\displaystyle}
\def\vsk{\vskip.3in}

\input amstex
\loadmsbm


\ctln{\bf Math 445}

\msk

\ctln{\bf Take-home Exam (Exam 2)}

\msk

{\narrower {\bf **Officially**}, due on the instructor's desk, or otherwise given into the instructor's
posession, by the end of business on Friday, December 5. You are not to discuss
the exam, except on trivial matters, with anyone other than the instructor, 
until after you have (both!) turned in your solutions. The problems are given in 
approximately the order in which the material was presented
in class; this is not necessarily a recommendation to work the problems 
in that order. Each needed computation should be carried out in full.\par}

\msk

\noindent {\bf Show all work.}
How you get your answer is just 
as important, if not more important, than the 
answer itself. If you think it, write it!

\vsk


\item{1.} (20 pts.) Show that if $p\geq 7$ 
is an odd prime, then 
$\displaystyle \Big({{n}\over{p}}\Big) = \Big({{n+1}\over{p}}\Big)$ for 
at least one of $n=2,3$, or 8.

\msk

\item{} [Hint: it might help to express this in terms of 
$\displaystyle \Big({{n}\over{p}}\Big)\Big({{n+1}\over{p}}\Big)$ .]


\vsk

\item{2.} (20 pts.) Show that if $p$ is a odd prime, and $a^2+b^2=p$ with
$a>0$ and $a$ odd, then $\dl \Big({{a}\over{p}}\Big) = 1$.

\ssk

\item{} [Hint: $\dl \Big({{p}\over{a}}\Big)$ makes sense, as a Jacobi symbol, 
and we can compute it....]


\vsk

\item{3.} (20 pts.) For which values of $N$, $1\leq N\leq 7$, does the equation

\ssk

\ctln{$x^2-53y^2=N$}

\ssk

\item{} have a solution with $x,y\in {\Bbb Z}$ ?


\vsk

\item{4.} (20 pts.) Use continued fractions, or Pell's equation, to find a rational number 
$\dl {{a}\over{b}}$ with 

\ssk

\ctln{$\dl \Big|{{a}\over{b}}-\sqrt{19}\Big|<{{1}\over{10000}}$ .}


\vsk

\item{5.} (20 pts.) Show that $x^2-2y^2=-1$ has infinitely many solutions
with $x,y\in {\Bbb N}$. What parity must $x$ have? Use this to show that
$n^2+(n+1)^2=m^2$ has infinitely many solutions with $n,m\in{\Bbb N}$. (I.e.,
there are infinitely many consecutive squares whose sum is a square!)

\vfill\end



