\input amstex
\magnification=1200

\define\ctln{\centerline}
\define\ssk{\smallskip}
\define\msk{\medskip}
\define\bsk{\bigskip}

\overfullrule=0pt
\nopagenumbers


%(NZM, Problem ) 

\ctln{\bf Math 445 Homework 8}

\msk

\ctln{Due Friday, November 14}

\bsk

\item{29.} Find the continued fraction expansions of the rational numbers

\ssk

\ctln{$191/73$ \hskip.5in and \hskip.5in $112/53$}

\bsk

\item{30.} [NZM, p.327, Problem 7.2.5] Show that if $x=[a_0,\ldots,a_n,b]$ and 

\item{}$x=[a_0,\ldots,a_n,c]$ with $b<c$,
then $x<y$ if $n$ is odd, and $x>y$ is $n$ is even. 

\bsk

\item{31.} Find the continued fraction expansion of $\sqrt{29}$, 
and use this to find the first five (5) convergents of $\sqrt{29}$ .

\bsk

\item{32.} Show that for $n$ a positive integer that is not a perfect square 
(translation: the continued fraction expansion of $\sqrt{n}$ never terminates),
that at \underbar{every} stage of the continued fraction expansion of $x = \sqrt{n}$

\ssk

\ctln{$x$ = $\langle a_0,a_1,\ldots ,a_{k-1},a_k+r_k\rangle$}

\ssk

$r_k$ is always of the form $\displaystyle r_k = {{\sqrt{n}-a}\over{b}}$ , where $b|n-a^2$ . Conclude that the continued
fraction expansion of $\sqrt{n}$ will eventually repeat, with a period of length at most $n\lfloor \sqrt{n}\rfloor$.

\ssk

\item{} Hint: by induction! In the inductive step, write $\displaystyle {{b}\over{\sqrt{n}-a}} = {{\sqrt{n}+a}\over{c}}$, and then find the 
fractional part of this. For the second half, how long must you wait before the $r_k$'s {\it must} repeat themselves?





\vfill\end

