\input amstex
\magnification=1200

\define\ctln{\centerline}
\define\ssk{\smallskip}
\define\msk{\medskip}
\define\bsk{\bigskip}

\overfullrule=0pt
\nopagenumbers


%(NZM, Problem ) 

\ctln{\bf Math 445 Homework 7}

\msk

\ctln{Due Friday, November 7}

\bsk

\item{25.} Show that if $p$ is an odd prime and $a$ is a primitive root
mod $p$, then $\displaystyle \Big({{a}\over{p}}\Big) = -1$ .

\bsk

\item{26.} The primes $p$ for which $x^2\equiv 7\pmod{p}$ has solutions 
consists precisely of those primes 
lying in certain congruence classes mod $28$ ; which ones?

\msk

\item{} [Hint: if you think of classes mod $7$ as being represented by
$-3,\ldots,0,\ldots ,3$ then you can recycle a lot of your work....]

\bsk

\item{27.} Compute the (Jacobi) symbols $\displaystyle\Big({{31}\over{113}}\Big)$ and 
$\displaystyle\Big({{131}\over{311}}\Big)$ .

\bsk

\item{28.} [NZM, p.137, \# 19] Show that for every (odd) prime $p$,
the residue equation 

\msk

\ctln{$x^8\equiv 16$ (mod $p$)}

\msk

\item{} always has a solution.

\msk

\item{} [Hint: $16 = 2^4$; look at our old criterion for solutions \underbar{and}
our new ones for quadratic residues...]


\vfill\end

