

\input amstex
\magnification=1200

\loadmsbm

\input epsf.tex

\define\ctln{\centerline}
\define\ssk{\smallskip}
\define\msk{\medskip}
\define\bsk{\bigskip}

\overfullrule=0pt
\nopagenumbers

\documentstyle{amsppt}

\ctln{\bf Math 445 Homework 3}

\msk

\ctln{Due Friday, Sept. 27}

\bsk

\bsk

\item{9.} Our description of RSA assumed that for $n=pq$, that $(a,n)=1$ .
But we don't control $a$, the sender does! Show that in any event, the RSA algorithm works even if $(A,n)>1$ :

\ssk

\item{} Show that if $n=pq$ is a product of distinct primes and $de\equiv 1\pmod{(p-1)(q-1)}$ , 
then $a^{de}\equiv a \pmod{n}$ for \underbar{any} $a$.

\ssk

\item{} (Hint: show that it works mod $p$ and $q$, first.)

\bsk

\item{10.} Our argument for``square root of work for half the chance of success''
in the Pollard $\rho$ method was a little imprecise; make a better estimate
of the number of starting points in a $K\times K$ grid whose lines of slope $-1$
will hit the ``success'' lines of slope $-1/2,-2$ emanating from $(0,0)$, to make a better 
estimate of the fraction of success we are trading less work for.
(Note: lines starting from the upper right/lower left corners may miss the
success lines before we stop computing $(a_i-a_{2i},n)$.)


\medskip

\leavevmode

\epsfxsize=2in
\centerline{{\epsfbox{success.eps}}}

\bsk

\item{11.} [NZM p.83, \# 13] When applying the Pollard $\rho$ method, starting 
from $a_1$, suppose we find that $a_i-a_j$, for $1\leq i\neq j\leq 17$, are coprime to $n$, but 
then $a_{18}-a_{11}$ shares a factor with $n$. What is the smallest $k$ that we then \underbar{know} of 
that will have $a_{2k}-a_k$ 
sharing a factor with $n$?

\bsk

\item{12.} [NZM p.83, \# 15] Show that if $(a,m)=1$ and there is a prime $p$ with $p|m$ and
$(p-1)|Q$, then $(a^Q-1,m)>1$ . 



\vfill\end










