

\input amstex
\magnification=1200

\loadmsbm

\define\ctln{\centerline}
\define\ssk{\smallskip}
\define\msk{\medskip}
\define\bsk{\bigskip}

\overfullrule=0pt
\nopagenumbers

\documentstyle{amsppt}

\ctln{\bf Math 445 Homework 2}

\msk

\ctln{Due Wednesday, Sept. 17}

\bsk

\bsk

\item{5.} Show, by induction, that for every $n\in{\Bbb N}$, 
$\displaystyle f(n)={{1}\over{2}}n^4+{{1}\over{3}}n^3+{{1}\over{6}}n$ is an integer.

\ssk

\item{} (Note, however, that it is {\it not} a multiple of $n$ !)

\bsk

\item{6.} Show that 8321=53$\times$157 is a strong pseudoprime to the base 2.

\item{} [Do the calculations by hand....]

\bsk

\item{7.} Show that $\text{gcd}(ab,n)$ divides $[\text{gcd}(a,n)][\text{gcd}(b,n)]$ .

\ssk

\item{} (There are at least 3 distinct proofs, depending on how you characterize gcd's?)

\bsk

\item{8.} (NZM, Problem 2.4.9) [For a pseudoprime, failing the 
Miller-Rabin test \underbar{finds} proper factors.]

\ssk

\item{} Show that if $x^2\equiv 1$ (mod $n$) and $x \not\equiv \pm 1$ (mod $n$), then
$1<(x-1,n)<n$ and $1<(x+1,n)<n$ .



\vfill\end










