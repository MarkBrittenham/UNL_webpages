\input amstex
\magnification=1200

\define\ctln{\centerline}
\define\ssk{\smallskip}
\define\msk{\medskip}
\define\bsk{\bigskip}
\def\ni{\noindent}

\overfullrule=0pt
\nopagenumbers


%(NZM, Problem ) 

\ctln{\bf Math 445 Homework 4 solutions}

\bsk

 

\bsk

\item{13.} Show that if $n|m$, and $(10,m)=1$, then the 
period of the decimal expansion of $1/n$ divides the
period of the decimal expansion of $1/m$ .

\msk

\ni Translating this into the language of orders, if $n|m$ and $(10,m)=1$, then
we wish to show that ord$_n(10)|$ord$_m(10)$ . Setting $s=$ord$_m(10)$, it is enough to show that
$10^s\equiv 1\pmod{n}$ , since we know that ord$_n(10)$ divides any such 
exponent. But by definition, $10^s\equiv 1\pmod{m}$, so $m|10^s-1$, so
$10^s-1=mx$ for some $x$ . But since $n|m$, $m=ny$ for some $y$, so 
$10^s-1=mx=(ny)x=n(xy)$, so $n|10^s-1$, so  $10^s\equiv 1\pmod{n}$ , as
desired.

\bsk

\item{14.} Show that for every $n\geq 2$, $\text{ord}_{3^n}(10)=3^{n-2}$ .

\ssk

\item{} (Hint: induction! This is not entirely unlike what we did for $7^n$....) 

\item{} [N.B.: Consequently, the period of the decimal expansion of $1/3^n$ is $3^{n-2}$ .]


\msk

\ni We show first that for every $n\geq 2$, $\displaystyle 10^{3^{n-2}}=1+k3^n$ for some $k$ with $(k,3)=1$.
We proceeed by induction. For $n=2$, $\displaystyle 10^{3^{2-2}}=10^{3^0} = 10^1 = 10 = 1+1\cdot 3^2$ , so 
$k=1$ and $(1,3)=1$ . Now suppose that $\displaystyle 10^{3^{n-2}}=1+k3^n$ for some $k$ with $(k,3)=1$. 
Then 

\ssk

\hskip1in $\displaystyle 10^{3^{(n+1)-2}}=10^{3^{n-2}\cdot 3}=(10^{3^{n-2}})^3=(1+k3^n)^3$

\hskip1in $= 1+3(1)^2(k3^n)+3(1)(k3^n)^2+(k3^n)^3$

\hskip1in $= 1+k3^{n+1}+k^23^{2n+1}+k^33^{3n}$

\hskip1in $= 1+(k+k^23^{n}+k^33^{2n-1})3^{n+1}$ 

\ssk

\ni with $k+k^23^{n}+k^33^{2n-1}\equiv k+k^2(0)+k^3(0)\equiv k\pmod 3$ (since $n,2n-1\geq 1$) . 
So $(k+k^23^{n}+k^33^{2n-1},3)=(k,3)=1$ , so $\displaystyle 10^{3^{(n+1)-2}}=1+K3^{n+1}$ with $(K,3)=1$,
as desired. So by induction, for $n\geq 2$ , $\displaystyle 10^{3^{n-2}}=1+k3^n$ for some $k$ with $(k,3)=1$.

\msk

\ni Since $\displaystyle 10^{3^{n-2}}=1+k3^n$, $\displaystyle 10^{3^{n-2}}\equiv 1 \pmod{3^n}$ , 
so ord$_{3^n}(10)|3^{n-2}$. So either
ord$_{3^n}(10) = 3^{n-2}$ or ord$_{3^n}(10) = 3^{m}$ for some $m<n-2$. But we know from above that
$\displaystyle 10^{3^{m}}-1=k3^{m+2}$ for some $k$ with $(k,3)=1$.
So if ord$_{3^n}(10) = 3^{m}$, then $\displaystyle 3^n|10^{3^{m}}-1$,
so $10^{3^{m}}-1=s3^n$ for some $s$. But then $k3^{m+2}=s3^n$, so cancelling powers of 3, $k=s3^{n-(m+2)} = s3^{(n-2)-m} = s3^r$ for
some $r\geq 1$. So $3|k$, so $(k,3)=3$, a contradiction. So ord$_{3^n}(10) = 3^{n-2}$ , as desired.

\bsk

\item{15.} Show that if $(3,n)=1$ ( and $(10,n)=1$ ), 
then $\text{ord}_{n}(10)=\text{ord}_{3n}(10)=\text{ord}_{9n}(10)$ .

\bsk

\ni By problem number 13, we know that $\text{ord}_{n}(10)|\text{ord}_{3n}(10)$ and $\text{ord}_{3n}(10)|\text{ord}_{9n}(10)$.
In particular, $\text{ord}_{n}(10)\leq\text{ord}_{3n}(10)$ and $\text{ord}_{3n}(10)\leq\text{ord}_{9n}(10)$.
To show that they are all equal, it suffices to show that $\text{ord}_{9n}(10)\leq\text{ord}_{n}(10)$;
what we will if fact show is that $\text{ord}_{9n}(10)|\text{ord}_{n}(10)$.

\ssk

\ni $\text{ord}_{n}(10)$ is the smallest positive $k$ for which $n|10^k-1$, and so it is enough to show that
if $n|10^k-1$, then $9n|10^k-1$. But $10\equiv 1$ (mod 9), so $1)^k\equiv 1^k=1$ (mod 9).
so $9|10^k-1$ for \underbar{every} $k\geq 1$. and since $(3,n)=1$, 3 and $n$ share no factors, so 9 and $n$
share no factors ($p$ prime and $p|9=3^2$, $p|n$, then $p|3$ and $p|n$, so $p|(3,n)=1$), so $(9,n)=1$.

\ssk

But $n|10^k-1$, $9|10^k-1$, and $(9,n)=1$ together imply $9n|10^k-1$, as desired. So $\text{ord}_{9n}(10)|\text{ord}_{n}(10)$,
and so

\ctln{$\text{ord}_{n}(10)=\text{ord}_{3n}(10)=\text{ord}_{9n}(10)$, as desired.}

\bsk


\item{16.} Find the primitive roots of 1 mod 31. (I.e., find all $a$, $1\leq a\leq 31$, with
$\text{ord}_{31}(a)=30$. 

\ssk

\item{} (Hint: find one; then use one of our results to quickly find the others.)

\msk

\ni To get started, we don't have much better than random chance? 
$\phi(31)=31-1=30=2\cdot 3\cdot 5$, so the possible orders of elements
are 2,3,5,6,10,15, or 30. We could construct a primitive root in the course of
our failures if we assemble exactly the data needed to use the proof of existence, i.e., 
numbers of orders 2, 3, and 5 precisely; their product will be a primitive root.
But that seems unlikely to occur first...

\msk

\ni Start with $a=2$; mod 31, $a^2=4$, $a^4=16$, $a^8=256=31\cdot 8 +8\equiv 8$, $a^{16}\equiv 64=31\cdot 2+2\equiv 2$. 
So $a^2=4\not\equiv 1$, $a^3=4\cdot 2=8\not\equiv 1$, but $a^5=32\equiv 1$, so 2 has order 5 mod 31.

\ssk 

\ni Next try $a=3$; mod 31, $a^2=9$, $a^4=81=31\cdot 3-12\equiv -12\equiv 19$, 

\ni $a^8\equiv 19^2=361=31\cdot 11+20\equiv 20$, and 

\ni $a^{16}\equiv 20^2=400=31\cdot 13-3\equiv -3\equiv 28$.

\ssk

\ni So $a^2=9\not\equiv 1$, 

\ni $a^3=9\cdot 3=27\not\equiv 1$, 

\ni $a^5=a^4\cdot a\equiv 19\cdot 3=57\equiv 26\not\equiv 1$,

\ni $a^6=a^4\cdot a^2\equiv 19\cdot 9=171=31\cdot 5+16\equiv 16\not\equiv 1$, 

\ni $a^{10}=a^8\cdot a^2\equiv 20\cdot 9=180=31\cdot 6-6\equiv 25\not\equiv 1$, 

\ni $a^{15}=a^{10}a^5\equiv 25\cdot 26=650=31\cdot 20+30\equiv 30\equiv -1\not\equiv 1$, 

\ni and just for 
sanity's sake, $a^{30}=a^{15}a^{15}\equiv (-1)^2=1$. 

\ni So 3 has order 30 mod 31, and so is a primitive
root of 1 mod 31.

\msk

\ni To find all other primitive roots of 1 mod 31, we can take all $3^k$ mod 31, for $(k,\phi(31))=(k,30)=1$.
But the numbers coprime to $30=2\cdot 3\cdot 5$ are the numbers less than 30 that are not multiples of 2, 3, or 5,
i.e., $k=1,7,11,13,17,19,23,29$ (since $30<6^2=36$, we should have expected all primes...). Note that we know this 
list is complete, since $\phi(\phi(31))=\phi(30)=\phi(2\cdot 3\cdot 5)=1\cdot 2\cdot 4=8$, so 
we should have 8 primitive roots. So we compute:

\ssk

$3^1=3$, $3^7=3^4 3^3\equiv 19\cdot 27=540-27=31\cdot 18-18-27\equiv -45\equiv -14\equiv 17$, 

$3^{11}=3^8 3^3\equiv 20\cdot 27 = 540=31\cdot 18-18\equiv -18\equiv 13$, 

$3^{13}=3^{11} 3^2\equiv 13\cdot 9=117=31\cdot 4-7\equiv -7\equiv 24$,

$3^{17}=3^{16} 3\equiv -3\cdot 3=-9\equiv 22$,

$3^{19}=3^{17} 3^2\equiv 22\cdot 9=198=31\cdot 6+12\equiv 12$,

$3^{23}=3^{19} 3^4\equiv 12\cdot 19=240-12=31\cdot 8-8-12\equiv -20\equiv 11$,

and $3^{29}=3^{16} 3^{13}\equiv 28\cdot 24\equiv (-3)\cdot(-7)=21$.

\ssk

\ni So the primitive roots of 1 mod 31 are: 3, 17, 13, 24, 22, 12, 11, and 21, or, in increasing order, 
3, 11, 12, 13, 17, 21, 22, and 24. Who would have guessed....


\vfill\end

