

\input amstex
\magnification=1200

\loadmsbm

\define\ctln{\centerline}
\define\ssk{\smallskip}
\define\msk{\medskip}
\define\bsk{\bigskip}

\overfullrule=0pt
\nopagenumbers

\voffset=-.3in
\vsize=8.2in

%\documentstyle{amsppt}

\ctln{\bf Math 445 Homework 2}

\msk

\ctln{Due Wednesday, Sept. 17}

\bsk

\bsk

\item{5.} Show, by induction, that for every $n\in{\Bbb N}$, 
$\displaystyle f(n)={{1}\over{2}}n^4+{{1}\over{3}}n^3+{{1}\over{6}}n$ is an integer.

\ssk

\item{} (Note, however, that it is {\it not} a multiple of $n$ !)

\bsk

\item{} We proceed by induction. For $n=1$, $\displaystyle f(1)={{1}\over{2}}+{{1}\over{3}}+{{1}\over{6}}=1$,
which is an integer. This gives us our base case. We now assume that $f(n)$ is an 
integer and compute 

\ssk

\item{} $\displaystyle f(n+1)={{1}\over{2}}(n+1)^4+{{1}\over{3}}(n+1)^3+{{1}\over{6}}(n+1)=$

\item{} $\displaystyle {{1}\over{2}}(n^4+4n^3+6n^2+4n+1)+{{1}\over{3}}(n^3+3n^2+3n+1)+{{1}\over{6}}(n+1)=$

\item{} $\displaystyle ({{1}\over{2}}n^4+{{1}\over{3}}n^3+{{1}\over{6}}n)+{{1}\over{2}}(4n^3+6n^2+4n+1)+{{1}\over{3}}(3n^2+3n+1)+{{1}\over{6}}(1)=$

\item{} $\displaystyle f(n)+{{1}\over{2}}(4n^3)+({{1}\over{2}}\cdot 6+{{1}\over{3}}\cdot 3)n^2+
({{1}\over{2}}\cdot 4+{{1}\over{3}}\cdot 3)n+({{1}\over{2}}+{{1}\over{3}}+{{1}\over{6}})=$

\item{} $\displaystyle f(n)+2n^3+4n^2+3n+1$ , 

\item{} which is the sum of $f(n)$, an integer, and $2n^3+4n^2+3n+1$, an integer. So $f(n+1)$ is an integer, and so 
by induction, $f(n)$ is an integer for every $n\in{\Bbb N}$. Note, however, that 
$f(2)=f(1)+2+4+3+1=11$, which is not a multiple of 2....

\bsk

\item{6.} Show that 8321=53$\times$157 is a strong pseudoprime to the base 2.

\item{} [Do the calculations by hand....]

\bsk

\item{} To show that 8321 is a strong pseudoprime to the base 2, we compute:

\item{} $8321-1 = 8320 = 2\cdot 4160 = 2\cdot 4\cdot 1040 = 2^3\cdot 4\cdot 260 = 
2^5\cdot 4\cdot 65 = 2^7\cdot 65$.

\item{} So we first compute $2^{65}$ mod $8321$, noting that $65=64+1=2^6+1$, so we start
squaring:

\item{} $2^2=4$, $2^4=4^2=16$, $2^8=16^2=256$, and $2^{16}=256^2=65536=8321\cdot 7+7289\equiv
7289$ mod 8321. Then $2^{32}\equiv 7289^2=53129521=8321\cdot 6384+8257\equiv 8257\equiv -64$ mod 8321,
and then $2^{64}\equiv (-64)^2=4096$ mod 8321, so $2^{65}=2^{64}\cdot 2^1\equiv 4096\cdot 2=8192\equiv -129$
mod 8321. 

\ssk

\item{} This is neither $1$ nor $-1$, so we start squaring:

\ssk

\item{} $2^{130}\equiv (-129)^2=16641=8321\cdot 1+8320\equiv 8320\equiv -1$ mod 8321.
So that didn't take long; $2^{260}\equiv (-1)^2=1$ so the sequence of repeated squares
reaches $-1$ just before it reaches $1$, \underbar{and} it reaches $1$ (by the time 
the squarings reach raising $2$ to the $8320$, so 8321 passes the Miller-Rabin test
for the base 2. But since $8321=53\cdot 157$ is not prime, it is a strong pseudoprime
to the base 2.

\bsk

\item{7.} Show that $\text{gcd}(ab,n)$ divides $[\text{gcd}(a,n)][\text{gcd}(b,n)]$ .

\ssk

\item{} (There are at least 3 distinct proofs, depending on how you characterize gcd's?)

\bsk

\item{} Proof \#1, using $(a,n)$ = product of prime powers, where we always choose the
smaller exponent found in $a$ and $n$; or, symbolically, 
if $a=\prod p_i^{\epsilon_i}$ and $b=\prod p_i^{\delta_i}$, then 
$(a,b)=\prod p_i^{\gamma_i}$, where $\gamma_i=\min\{\epsilon_i,\delta_i\}$. Then:

\ssk

\item{} If we write $a=\prod p_i^{\epsilon_i}$ and $b=\prod p_i^{\delta_i}$, $n=\prod p_i^{\eta_i}$, 
then 
$(a,n)=\prod p_i^{\gamma_i}$, with $\gamma_i=\min\{\epsilon_i,\eta_i\}$,
$(b,n)=\prod p_i^{\theta_i}$, with $\theta_i=\min\{\delta_i,\eta_i\}$,
and, since $ab=\prod p_i^{\epsilon_i+\delta_i}$,
$(ab,n)=\prod p_i^{\phi_i}$, with $\phi_i=\min\{\epsilon_i+\delta_i,\eta_i\}$.

\ssk

\item{} But then to show that 
$\prod p_i^{\phi_i}=(ab,n)|(a,n)(b,n)=\prod p_i^{\gamma_i+\theta_i}$, it is enough to show
that $\phi_i\leq\gamma_i+\theta_i$ for every $i$, that is,
$\min\{\epsilon_i+\delta_i,\eta_i\}\leq \min\{\epsilon_i,\eta_i\}+\min\{\delta_i,\eta_i\}$, i.e., 
$\min(x+y,z)\leq \min(x,z)+\min(y,z)$ for any $x,y,z\geq 0$. But if
either of the terms on the rightside \underbar{is} $z$, then 
$\min(x+y,z)\leq z\leq \min(x,z)+\min(y,z)$ (the first by definition of min, the second since one of
the numbers \underbar{is} $z$ and the other is $\geq 0$). But if the terms on the right side are
$x$ and $y$, then $\min(x+y,z)\leq x+y\leq \min(x,z)+\min(y,z)$, as desired. This establishes
our argument, so $(ab,n)|(a,n)(b,n)$, as desired.

\msk

\item{} Proof \#2: $(a,n)$ is the largest integer that can be expressed as
$(a,n)=ax+ny$ for $x,y\in {\Bbb Z}$. similarly, we may write $(b,n)=bu+nv$. So 
$(a,n)(b,n)=(ax+ny)(bu+nv)=(ab)(xu)+n(ybu+axv+yv)$ and so can be expressed as an 
integer-linear combination of $ab$ and $n$. But $(ab,n)$ divides any number that
can be so expressed, so $(ab,n)|(a,n)(b,n)$, as desired.

\msk

\item{} Proof \#3: $(a,n)|a$, so $(a,n)|ab$, and $(a,n)|n$, so together these give $(a,n)|(ab,n)$
(by the definition of $(ab,n)$). 
So we can write $(ab,n)=x(a,n)$. To show that $(ab,n)|(a,n)(b,n)$ then, it is enough to show that
$x|(b,n)$ (since then \hfill

\item{} $(ab,n)=(a,n)x|(a,n)(b,n)$). 
But to show this, it is enough to show that $x|b$ and $x|n$. But:
$(ab,n)|n$, so $x(a,n)|n$, so $x|n$. Further, 
$k(m,n)=(km,kn)$, since $(m,n)|m,n$, so $k(m,n)|km,kn$, so $k(m,n)|(km,kn)$,
while $k(m,n)$ can be expressed as a ${\Bbb Z}$-linear combination of 
$km$ and $kn$, so $(km,kn)|k(m,n)$. So:

\ssk

\item{} $\displaystyle x(a,n)=(ab,n)=((a,n){{a}\over{(a,n)}}b,(a,n){{n}\over{(a,n)}})=(a,n)({{a}\over{(a,n)}}b,{{n}\over{(a,n)}})$,
we have
$\displaystyle x=(b{{a}\over{(a,n)}},{{n}\over{(a,n)}})$. So $\displaystyle x|b{{a}\over{(a,n)}} \text{ and }x|{{n}\over{(a,n)}}$.
But since $\displaystyle ({{a}\over{(a,n)}},{{n}\over{(a,n)}})=1$, 
we can express $\displaystyle 1={{a}\over{(a,n)}}u+{{n}\over{(a,n)}}v$, so 
$\displaystyle b={{a}\over{(a,n)}}bu+{{n}\over{(a,n)}}bv$, and then $x|b$ since it
divides factors of both terms in the sum. 

\ssk

\item{} So $x|b$ and $x|n$, so $x|(b,n)$, so $(ab,n)=(a,n)x|(a,n)(b,n)$, as desired.

\bsk

\item{8.} (NZM, Problem 2.4.9) [For a pseudoprime, failing the 
Miller-Rabin test \underbar{finds} proper factors.]

\ssk

\item{} Show that if $x^2\equiv 1$ (mod $n$) and $x \not\equiv \pm 1$ (mod $n$), then
$1<(x-1,n)<n$ and $1<(x+1,n)<n$ .

\bsk

\item{} If $x^2\equiv 1$ (mod $n$) and $x \not\equiv \pm 1$ (mod $n$), then 
we have $n|x^2-1=(x-1)(x+1)$, but $n\not{|}x-1$ (since $x \not\equiv 1$ (mod $n$))
and $n\not{|}x+1$ (since $x \not\equiv -1$ (mod $n$)). But if $(n,x-1)=1$, then 
since $n|(x-1)(x+1)$ we have $n|x+1$, a contradiction. [E.g., problem \#7 says
$n=(x^2-1,n)|(x-1,n)(x+1,n)=(x+1,n)$, so $n|(x+1)$.] So $(n,x-1)>1$. Similarly,
if $(n,x+1)=1$, then 
since $n|(x-1)(x+1)$ we have $n|x-1$, a contradiction. So $(n,x+1)>1$.
$(n,x-1)\geq n$ implies $(n,x-1)= n$ (the gcd of two numbers cannot exceed the numbers),
which in turn implies $n|x-1$ (since $(a,b)|b$), a contradiction. So
$(n,x-1)<n$. Similarly, $(n,x+1)<n$. So we have
$1<(n,x-1)<n$ and $1<(n,x+1)<n$, as desired.

\vfill\end










