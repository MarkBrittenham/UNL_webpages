

\input amstex
\magnification=1200

\loadmsbm

\voffset=-.7in
\vsize=8.6in

\input epsf.tex

\define\ctln{\centerline}
\define\ssk{\smallskip}
\define\msk{\medskip}
\define\bsk{\bigskip}

\overfullrule=0pt
\nopagenumbers

\ctln{\bf Math 445 Homework 3 Solutions}

\bsk

\item{9.} Our description of RSA assumed that for $n=pq$, that $(a,n)=1$ .
But we don't control $a$, the sender does! Show that in any event, the RSA algorithm works even if $(A,n)>1$ :

\ssk

\item{} Show that if $n=pq$ is a product of distinct primes and $de\equiv 1\pmod{(p-1)(q-1)}$ , 
then $a^{de}\equiv a \pmod{n}$ for \underbar{any} $a$.

\ssk

\item{} We'll show that $A^{de}\equiv A \pmod{p}$ and $A^{de}\equiv A \pmod{q}$ , i.e., $p$ and $q$ both divide
$A^{de}-A$ . Then since $p$ and $q$ are distinct primes, $(p,q)=1$, and so $n=pq|A^{de}-A$.

\ssk

\item{} By hypothesis, $de-1=k(p-1)(q-1)$, so $de=1+k(p-1)(q-1)$ . Given $A$, one of three things is true: (1) $(A,p)=(A,q)=1$, 
(2) exactly one of $p,q$ divides $A$, WOLOG $p|A$ (since there is no distinction between them) and $(A,q)=1$, or
(3) $p,q|A$, so $n=pq|A$ . 

\ssk

\item{} In case (1), Fermat's Little Theorem tells us that $A^{p-1}\equiv 1\pmod{p}$ and 
$A^{q-1}\equiv 1\pmod{q}$, so
$A^{de}=(A^{p-1})^{k(q-1)}A\equiv 1^{k(q-1)}A\equiv A\pmod{p}$ and 
$A^{de}=(A^{q-1})^{k(p-1)}A\equiv 1^{k(p-1)}A\equiv A\pmod{q}$
as desired. In case (2), $A\equiv 0\pmod{p}$, so $A^{de}\equiv 0^{de}\equiv 0\equiv A\pmod{p}$, 
while, as in (1), $A^{de}\equiv A\pmod{q}$ .
Finally, in case (3), $A\equiv 0\pmod{p}$ and $A\equiv 0\pmod{q}$ , so 
$A^{de}\equiv 0^{de}\equiv 0\equiv A\pmod{p}$ and the same for $q$ .
So in all cases, $A^{de}\equiv A\pmod{p}$ and $\pmod{q}$ , so $A^{de}\equiv A\pmod{n}$ .

\bsk

\item{10.} Our argument for``square root of work for half the chance of success''
in the Pollard $\rho$ method was a little imprecise; make a better estimate
of the number of starting points in a $K\times K$ grid whose lines of slope $-1$
will hit the ``success'' lines of slope $-1/2,-2$ emanating from $(0,0)$, to make a better 
estimate of the fraction of success we are trading less work for.
(Note: lines starting from the upper right/lower left corners may miss the
success lines before we stop computing $(a_i-a_{2i},n)$.)

\msk

\item{} We know that if $(a_j-a_i,n)>1$, then $(a_{j+k}-a_{i+k},n)>1$
for all $k\geq 0$. If we focus on the set of pairs $(j,i)$ with $1\leq i,j\leq K$ and 
$j>i$, we wish to 
estimate the sie of the set of such points for which 
the sequence of pairs $(j+k,i+k)$ intersects the
sequence of pairs $(2m,m$) \underbar{before} $m$ exceeds $K$.
[By setting $j>i$, we will work with the upper right half of 
the $K\times K$ square, the other half 
would interact with the ``line'' of pairs $(m,2m)$, instead, and give
an identical estimate.]

\ssk

\item{} But the set of pairs whose line of successors meet the $(2m,m)$ line inside
of the $K\times K$ square are precisely the points lying in the triangle of points lying
up the slope -1 lines from the line $x+2y=0$ of slope -1/2 emanating from the
origin $(0,0)$; see the figure below. We need to determine what fraction of the
pairs $(j,i)$ above the line $x+y=0$ lie in that triangle. But this is 
a matter of computing areas: 

\medskip

\leavevmode

\epsfxsize=2in
\centerline{{\epsfbox{success2.eps}}}

\medskip

\item{} The triangle of all pairs $(j,i)$ has base $K$ and height $K$, so has
area $B=K\cdot K/2=K^2/2$ . The traingle of all pairs whose slope -1
lines will meet our success line (the shaded triangle above) has
base $K/2$ and height $K/2$, so has area 
$(K/2)(K/2)/2=K^2/8$. So the fraction of pairs that could be detected
by our success line is $(K^2/8)/(K^2/2)=1/4$.
So roughly $1/4$ of the pairs we could test and find
$(a_j-a_i,n)>1$ wuld be detected by instead testing for 
$(a_{2i}-a_i,n)>1$. So we have 1/4 the chance of succeeding (over
testing \underbar{all} pairs) by doing roughly square root (test
$K$ pairs, instead of $K(K-1)/2$ pairs) work. Which for large $K$
is a very good trade-off!

\bsk

\item{11.} [NZM p.83, \# 13] When applying the Pollard $\rho$ method, starting 
from $a_1$, suppose we find that $a_i-a_j$, for $1\leq i\neq j\leq 17$, are coprime to $n$, but 
then $a_{18}-a_{11}$ shares a factor with $n$. What is the smallest $k$ that we then 
\underbar{know} of 
that will have $a_{2k}-a_k$ 
sharing a factor with $n$?

\msk

\item{} Essentially, we are asking: what is the smallest $k$ so that 
$(2k,k)=(18+m,11+m)$ for some $m\geq 0$? We therefore want

\ssk

\item{} $2k=18+m$ , $k=11+m$, which, subtracting, gives $k=7$.
But this is ridiculous; this yields the point $(14,7)$ 
which is \underbar{behind} $(18,11)$, and we \underbar{can't}
conclude that $(a_{14}-a_7,n)>1$. My bad.

\msk

\item{} But as the text points out, we know that even more is true: 
if we set $d=(a_{18}-a_{11},n)$, then $a_{18}\equiv a_{11}$ (mod $d$), so 
$a_{18+k}\equiv a_{11+k}$ (mod $d$) for every $k\geq 0$. But when $k=7$, this gives
$a_{25}\equiv a_{18}\equiv a_{11}$, and so $a_{25+k}\equiv a_{11+k}$ as well.
Essentially, after $r=11$, $a_k$ cycles through 7 values mod $d$, i.e., 
$a_{11+j+7k}\equiv a_{11+j+7l}$ for every $0\leq j\leq 6$ and $k,l\geq 0$.
So to find an $i$ for which $d|a_{2i}-a_i$, so (since $d|n$) $d|(a_{2i}-a_i,n)$
and $(a_{2i}-a_i,n)>1$, we need to find an $i$ so that
$2i=11+j+7l$ and $i=11+j+7k$ for some $j$, $k$, and $l$. Subtracting,
we get $i=7(l-k)$. The smallest $i$ will be when $k=0$ (i.e.,
$i=11+j+7k$ is in the first round of cycling), so we need $i=7l$ and
$i=11+j$ with $0\leq j\leq 6$, so $j=3$ and $i=14$, $2i=28$. So
$(a_{28}-a_{14},n)>1$ giving, usually, a proper factor of $n$.

\ssk

\item{} In the end, since $18-11=7$, we seek a pair $(2i,i)$ with $2i\geq 18$, $i\geq 11$, 
and $2i-i=i$ a multiple of 7; the first such $i$ is $i=14$.

\bsk

\item{12.} [NZM p.83, \# 15] Show that if $(a,m)=1$ and there is a prime $p$ with $p|m$ and
$(p-1)|Q$, then $(a^Q-1,m)>1$ . 

\medskip

\item{} We show, in fact, that $p|^Q-1,m)$, so $(a^Q-1,m)\geq p>1$. We know, by hypothesis,
that $p|m$, so it is enough to show that $p|A^Q-1$. But 
since $p-1|Q$, $Q=(p-1)k$ for some $k$, and then
$a^Q=a^{(p-1)k}=(a^{p-1})^k\equiv 1^k=1$ (mod $p$), by Fermat's Little Theorem,
so $p|(a^{p-1})^k-1=a^Q-1$, as desired. So $(a^Q-1,m)\geq p>1$.

\medskip

\item{} [N.B. This fact is the basis for Pollard's $p-1$ Test: if $m$ has a prime factor
such that $p-1$ is a product of ``small'' primes, then $(p-1)|N!$ for some relatively
small value of $N$, so, e.g., $(2^{N!}-1,m)>1$ can be computed relatively
quickly, usually finding a factor of $m$. In most implementations of RSA, for example,
the program generates industrial-grade primes $p,q$, but also checks that
$(p-1),(q-1)$ each have at least one large prime factor, to protect against this method
of finding a factor of $m=pq$.]

\vfill\end










