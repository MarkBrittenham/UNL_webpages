

\input amstex
\magnification=1200

\loadmsbm

\define\ctln{\centerline}
\define\ssk{\smallskip}
\define\msk{\medskip}
\define\bsk{\bigskip}

\overfullrule=0pt
\parindent=0pt

\nopagenumbers

\ctln{\bf Math 445 Homework 1 solutions}

\bsk

\item{1.} (NZM, Problem 1.3.27) Show that if $n$ is {\it not} prime, then $n|(n-1)!$ .

\msk

If $n$isn't prime, then $n=ab$ , with $1<a\leq b<n$ . Then $a$ and $b$ are
both among the factors of $(n-1)!$ . So if they are {\it different}, then 
$ab|1\cdots(a-1)a(a+1)\cdots (b-1)b(b+1)\cdots(n-1)=(n-1)!$,
as desired. If $a=b$, then since both are at least 2, $a$ and $2a$ are both $\leq n-1$; 
if $2a> n-1$, then (since $b\geq 2$) $2a\geq n=ab$, so $b\leq 2$, so  $a=b=2$ and $n=4$,
a contradiction. So $2a^2|1\cdots(a-1)a(a+1)\cdots (2a-1)2a(2a+1)\cdots(n-1)=(n-1)!$, so $n=a^2|(n-1)!$ .

\msk

\item{2.} (NZM, Problem 1.3.31) Show that if $f(x)$ is a non-constant polynomial
with integer coefficients, then $f(n)$ \underbar{cannot} be prime for every $n\in {\Bbb N}$.

\item{}(Hint: If $f(n)=p$ is prime, show that for every $k\in {\Bbb N}$ we have $p|f(n+kp)$;
eventually $f(n+kp)$ is too big to \underbar{be} $p$ ...)

\msk

Suppose $f(n)$ is prime for every $n$.
Since $f$ is not constant, $f(x)\rightarrow \pm\infty$ as $x\rightarrow\infty$, so eventually
we can find an $n\in{\Bbb N}$ with $|f(n)|=|p|\geq 2$ and $p$ prime.

\ssk

Then $n+kp$ for $k\geq 1$ yields infinitely many different numbers with $f(n+kp)$, by
assumption, prime. But if we write $f(x)=\sum a_ix^i$, then since $n+kp\equiv n\ (\text{mod } p)$,
we have $(n+kp)^i\equiv n^i\ (\text{mod } p)$, so 
$f(n+kp)=\sum a_i(n+kp)^i\equiv \sum a_in^i=f(n)\ (\text{mod } p)$. 

\ssk

So $f(n+kp)=f(n)+(f(n+kp)-f(n)=p+pM=p(M+1)$ for some integer $M$, so 
$p|f(n+kp)$ for all $k$. But since these numbers are assumed to be prime, we have
$f(n+kp)=\pm p$ for every $k$. So $f$ takes one of the vaules $p$ or $-p$ for 
infinitely many values of $n+kp$. But a polynomial can't do that, \underbar{unless}
it is constant; if $f$ has degree $d\geq 1$, then so does $f(x)-(\pm p)$, which therefore
can have at most $d$ roots $f(x)-(\pm p)=0$, i.e., $f(x)=\pm p$. So $f$ must be constant.

\ssk

Consequently, no non-constant polynomial with integer coefficients can have $f(n)$ prime for every
natural number $n$.

\msk

\item{3.} (NZM, Problem 1.3.33) Show that for $n>1$, $n^4+n^2+1$ is {\it never} prime.

\item{}(Hint: $f(x)= x^4+x^2+1$ can be expressed as a product of quadratics; find the factorization!)

\msk

If we are going to be able to factor $f(x)$ into quadratics \underbar{with} 
underbar{integer} \underbar{coefficients}, then the lead and constant coefficients
of each fact will need to be $1,-1$. So we try

\ssk

$x^4+x^2+1=(x^2+ax+1)(x^2+bx+1)$ or $x^4+x^2+1=(x^2+ax-1)(x^2+bx-1)$, and see if we can find integers
that work. And it does:

\ssk

$(x^2+ax+1)(x^2+bx+1)=x^4+(a+b)x^3+(1+ab+1)x^2+(a+b)x+1=x^4+x^2+1$  if
$ab=-1$ and $a+b=0$, so $b=-a$ and $a(-a)=-1$, so $a^2=1$. So $a=1,b=-1$ 
works.

So $n^4+n^2+1=(n^2+n+1)(n^2-n+1)$, which factors $n^4+n^2+1$, so it isn't prime, 
\underbar{unless} $n^2+n+1=\pm 1$ or $n^2+n+1=\pm 1$. But for $n\geq 1$
$n^2+n+1\geq 1+1+1=3$, and $n^2-n+1\geq n^2-n^2+1=1$, so the only 
possibility is $n^2-n+1=1$, which requires $n^2-n=n(n-1)=0$, so $n=0,1$.
So for $n>1$, $n^2+n+1,n^2-n+1>1$, giving a proper factorization of 
$n^4+n^2+1$. So for $n>1$, $n^4+n^2+1$ is never prime.

\vfill
\eject

\item{4.} Show that if $2^n-1$ is prime, then $n$ must be prime.

\bsk

It is probably most straightforward to show the contrapositive: if $n$ is not
prime, then $2^n-1$ is not prime. Suppose that $n=rs$, with $2\leq r,s$, then 

\msk

$2^n-1 = 2^{rs}-1 = (2^r)^s-1$

\msk

But since $x^s-1 = (x-1)(x^{s-1}+x^{s-2}+\cdots +s+1)$ we have

\msk

$2^n-1$ = $(2^r-1)(2^{r(s-1)}+2^{r(s-2)}+\cdots +2^r+1)$ . and since $r,s\geq 2$,
$2^r-1\geq 2^2-1=3$ and $2^{r(s-1)}+2^{r(s-2)}+\cdots +2^r+1\geq 2^r+1\geq 2^2+1=5$.
So we have found a factorization of $2^n-1$ into factors $\geq 3$, so $2^n-1$ is 
composite.

\vfill\end










