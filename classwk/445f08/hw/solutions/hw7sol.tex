\input amstex
\magnification=1200
\parindent=0pt


\vsize=8.5in
\voffset=-.3in

\define\ctln{\centerline}
\define\ssk{\smallskip}
\define\msk{\medskip}
\define\bsk{\bigskip}

\overfullrule=0pt
\nopagenumbers


%(NZM, Problem ) 

\ctln{\bf Math 445 Homework 7 Solutions}

\bsk

\item{25.} Show that if $p$ is an odd prime and $a$ is a primitive root
mod $p$, then $\displaystyle \Big({{a}\over{p}}\Big) = -1$ .

\msk

By Euler's criterion, $\displaystyle \Big({{a}\over{p}}\Big) \equiv a^{{p-1}\over{2}}\pmod{p}$ .
Since $a$ is a primitive root mod $p$, ord$_p(a)=p-1$, so 
$x=a^{{p-1}\over{2}}\not\equiv 1\pmod{p}$, since ${{p-1}\over{2}}<p-1$.
but $x^2=a^{p-1}\equiv 1\pmod{p}$ so, since $p$ is prime, $a\equiv \pm1\pmod{p}$.
So $x\equiv -1$, so $\displaystyle -1\equiv a^{{p-1}\over{2}}\equiv \Big({{a}\over{p}}\Big) $ ,
So $\displaystyle \Big({{a}\over{p}}\Big) = -1$ .

\msk

\item{26.} The primes $p$ for which $x^2\equiv 7\pmod{p}$ has solutions 
consists precisely of those primes 
lying in certain congruence classes mod $28$ ; which ones?

\msk

We want to know for which primes $p$ is
$\displaystyle \Big({{7}\over{p}}\Big) = 1$. But by reciprocity, 

$\displaystyle \Big({{7}\over{p}}\Big)=\Big({{p}\over{7}}\Big)(-1)^{{{7-1}\over{2}}{{p-1}\over{2}}}
=\Big({{p}\over{7}}\Big)(-1)^{3{{p-1}\over{2}}} = 
=\Big({{p}\over{7}}\Big)(-1)^{{{p-1}\over{2}}}$, which is

\ssk

$\displaystyle \Big({{p}\over{7}}\Big)$ if $p\equiv 1$ (mod 4), and is
$\displaystyle -\Big({{p}\over{7}}\Big)$ if $p\equiv 3$ (mod 4). So 

\ssk

$\displaystyle \Big({{7}\over{p}}\Big) = 1$ $\Leftrightarrow$ 
$\displaystyle \Big({{p}\over{7}}\Big)=1$ and $p\equiv 1$ (mod 4), \underbar{or}
$\displaystyle \Big({{p}\over{7}}\Big)=-1$ and $p\equiv 3$ (mod 4).

\msk

But the value of $\displaystyle \Big({{p}\over{7}}\Big)$ depends only on $p$ mod 7, and 
half of the congruence classes (and 0) will contain squares. And we can find these
by inspection:

\ssk

$1^2=1, 2^2=4, 3^2=9\equiv 2$ (mod 7), and so $\displaystyle \Big({{p}\over{7}}\Big)=1$
for $p\equiv 1,2,4$ (mod 7). The remaining congruence classes, 
$p\equiv 3,5,6$ (mod 7), yield primes with $\displaystyle \Big({{p}\over{7}}\Big)=-1$.

\msk

So, $\displaystyle \Big({{7}\over{p}}\Big) = 1$ $\Leftrightarrow$ 
$p\equiv 1$ (mod 4) and $p\equiv 1,2,\text{ or }4$ (mod 7), \underbar{or}
$p\equiv 3$ (mod 4) and $p\equiv 3,5,\text{ or }6$ (mod 7), [together with 2 and 7].

\ssk

But each of these six possibilities represents a single congruence class
mod 28, by the chinese remainder theorem, e.g., $p\equiv 1$ (mod 4) and $p\equiv 1$ (mod 7)
$\Leftrightarrow$ $p\equiv 1$ (mod 28). Rather than work through the procedure 
as the proof of CRT would, we can use the fact that we know there is one
congruence class mod 28 containing the solutions to the simultaneous equations
$p\equiv a\text{ mod }4$,$p\equiv b\text{ mod }7$, we can find representatives
experimentally.

\ssk

Writing some of the integers $\equiv 1$ (mod 4), $n=1,5,9,13,17,21,25,29$,
we note that $1\equiv 1$ (mod 7), $9\equiv 2$ (mod 7), and $25\equiv 4$ (mod 7),
so 

\ssk

$p\equiv 1$ (mod 4) and $p\equiv 1$ (mod 7) $\Leftrightarrow$ $p\equiv 1$ (mod 28)
$p\equiv 1$ (mod 4) and $p\equiv 2$ (mod 7) $\Leftrightarrow$ $p\equiv 9$ (mod 28)
$p\equiv 1$ (mod 4) and $p\equiv 4$ (mod 7) $\Leftrightarrow$ $p\equiv 25$ (mod 28)

\ssk

Similarly, writing some of the integers $\equiv 3$ (mod 4), $n=3,7,11,15,19,23,27,31$,
we note that $3\equiv 3 (mod 7)$, $19\equiv 5$ (mod 7), and $27\equiv 6$ (mod 7),
so 

\ssk

$p\equiv 3$ (mod 4) and $p\equiv 3$ (mod 7) $\Leftrightarrow$ $p\equiv 3$ (mod 28)
$p\equiv 3$ (mod 4) and $p\equiv 5$ (mod 7) $\Leftrightarrow$ $p\equiv 19$ (mod 28)
$p\equiv 3$ (mod 4) and $p\equiv 6$ (mod 7) $\Leftrightarrow$ $p\equiv 27$ (mod 28)

\msk

So the primes $p$ with $\displaystyle \Big({{7}\over{p}}\Big)=1$ 
are precisely those congruent to one of 1,3,9,19,25, or 27 (mod 28). [To be formally complete, we should add the classes 2 and 7, each
containing a single prime (2 and 7).]

\vfill
\eject

\item{27.} Compute the (Jacobi) symbols $\displaystyle\Big({{31}\over{113}}\Big)$ and 
$\displaystyle\Big({{131}\over{311}}\Big)$ .


\msk

$\displaystyle\Big({{31}\over{113}}\Big)\Big({{113}\over{31}}\Big)=(-1)^{{{31-1}\over{2}}{{113-1}\over{2}}}=(-1)^{15\cdot 56} = 1$,
so $\displaystyle\Big({{31}\over{113}}\Big)=\Big({{113}\over{31}}\Big)=\Big({{31\cdot 3+20}\over{31}}\Big)
=\Big({{20}\over{31}}\Big)=\Big({{2^25}\over{31}}\Big)=\Big({{2}\over{31}}\Big)^2\Big({{5}\over{31}}\Big)=
1\cdot\Big({{5}\over{31}}\Big)=\Big({{5}\over{31}}\Big)$, since whatever $\displaystyle\Big({{2}\over{31}}\Big)$ is,
its square will be 1.

\ssk

But now $\displaystyle\Big({{5}\over{31}}\Big)\Big({{31}\over{5}}\Big)=(-1)^{{{5-1}\over{2}}{{31-1}\over{2}}}=(-1)^{2\cdot 15}=1$,
so $\displaystyle\Big({{5}\over{31}}\Big)=\Big({{31}\over{5}}\Big)=\Big({{6\cdot 5+1}\over{5}}\Big)=\Big({{1}\over{5}}\Big)=1$,
since thinking of this as a Legendre symbol, $1=1^2$ is a square mod $5$.

\ssk

So, put together, $\displaystyle\Big({{31}\over{113}}\Big)=\Big({{113}\over{31}}\Big)=\Big({{5}\over{31}}\Big)=\Big({{31}\over{5}}\Big)
=\Big({{1}\over{5}}\Big)=1$, so $\displaystyle\Big({{31}\over{113}}\Big)=1$.

\bsk

For $\displaystyle\Big({{131}\over{311}}\Big)$, we have $\displaystyle\Big({{131}\over{311}}\Big)\Big({{311}\over{131}}\Big)
=(-1)^{{{131-1}\over{2}}{{311-1}\over{2}}}=(-1)^{65\cdot 155} = -1$, so
$\displaystyle\Big({{131}\over{311}}\Big)=-\Big({{311}\over{131}}\Big)=-\Big({{2\cdot 131+49}\over{131}}\Big)
=-\Big({{49}\over{131}}\Big)=-\Big({{7^2}\over{131}}\Big)=-\Big({{7}\over{131}}\Big)^2=-1$,
since $\displaystyle\Big({{7}\over{131}}\Big)=\pm 1$, so its square is 1.

\ssk

So $\displaystyle\Big({{131}\over{311}}\Big)=-\Big({{311}\over{131}}\Big)=-1$.

\bsk

\item{28.} [NZM, p.137, \# 19] Show that for every (odd) prime $p$,
the residue equation 

\msk

\ctln{$x^8\equiv 16$ (mod $p$)}

\msk

\item{} always has a solution.

\msk

By our more ancient criterion, for $p$ an (odd) prime and (hence) $(16,p)=1$,
$x^8\equiv 16$ (mod $p$) has a solution $\Leftrightarrow$ $\displaystyle 16^{{{p-1}\over{(8,p-1)}}}\equiv 1$ (mod $p$);
we wish to show that this latter is always true.
But since $16=2^4$, this means that we wish to show that

$\displaystyle (2^4)^{{{p-1}\over{(8,p-1)}}}=2^{4\cdot{{p-1}\over{(8,p-1)}}}\equiv 1$. 

But we know that $2^{p-1}\equiv 1$ (mod $p$), so what we wish to show is true \underbar{is} true if

$\displaystyle p-1|4\cdot{{p-1}\over{(8,p-1)}}$, i.e., 
$\displaystyle 4\cdot{{p-1}\over{(8,p-1)}}=(p-1)n$ for some integer $n$, i.e., 

$4\cdot(p-1)=(p-1)n(8,p-1)$, i.e., $4=n(8,p-1)$ for some $n$, i.e., $(8,p-1)|4$. 

\ssk

But since $p$ is odd, $p=8k+r$ for some $k$ and for $r=$ one of 1,3,5, or 7, and so 

$p-1=8k+(r-1)$.
So $(8,p-1)=(8,8k+(r-1))=(8,r-1)$, so $(8,p-1)=$ one of 
$(8,0)=8$, $(8,2)=2$, $(8,4)=4$, or $(8,6)=(2,6)=2$. So in every case except
$r=1$ (so $p\equiv 1$ (mod 8)) we have $(8,p-1)|4$, as desired, so we have shown:

\ssk

(*) unless $p\equiv 1$ (mod 8), $x^8\equiv 16$ (mod $p$) has a solution.

\ssk

But! if $p$ is prime and $p\equiv 1$ (mod 8), then $x^2\equiv 2$ (mod p) has a solution;
$\displaystyle\Big({{2}\over{p}}\Big)=(-1)^{{{p^2-1}\over{8}}}$ and $\displaystyle {{p^2-1}\over{8}}$ is even.
So, for that value of $x$, $x^8=(x^2)^4\equiv 2^4=16$, so $x^8\equiv 16$ (mod $p$) has a solution, as desired.

\ssk

So, for every odd prime $p$, $x^8\equiv 16$ (mod $p$) has a solution.


\vfill\end

