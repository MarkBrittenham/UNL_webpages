\input amstex
\magnification=1200

\define\ctln{\centerline}
\define\ssk{\smallskip}
\define\msk{\medskip}
\define\bsk{\bigskip}
\define\dl{\displaystyle}

\overfullrule=0pt
\nopagenumbers


%(NZM, Problem ) 

\ctln{\bf Math 445 Homework 6 SOlutions}

\bsk

\item{21.} Show that if an integer $n$ can be expressed as the sum of the squares of two {\it rational} numbers

\ssk

\ctln{(*) $\displaystyle n = ({{a}\over{b}})^2 + ({{c}\over{d}})^2$ ,}

\ssk

\item{} then $n$ can be expressed as the sum of the squares of two {\it integers}.

\msk

\item{} (Hint: Not directly! Show that $n$ has the correct prime factorization....)

\bsk
\item{} From (*), clearing denomenators, we have that $nb^2d^2=a^2d^2+c^2b^2 = (ad)^2+(bc)^2$ is a sum of two squares. So
for every prime $p$ with $p\equiv 3\pmod{4}$ , $p^k||nb^2d^2=n(bd)^2$ with $k$ even. But since $(bd)^2$ is a perfect square,
$p^m||(bd)^2$ has $m$ even. So $p^{k-m}||n$ has $k-m$ even. Consequently, every prime $p$ with $p\equiv 3\pmod{4}$
which appears in the prime factorization of $n$ has even exponent. Therefore, by our main result from class, $n$ can be
expressed as a sum of two squares.

\bsk

\item{22.} [NZM, p. 106, \#\ 2.8.8] Determine how many solutions (mod 17) each of the following 
congruence equations has:

\msk

\hskip1in (a) $x^{12}\equiv 16$ (mod 17) 

\msk

\item{} $(12,17-1)=(12,16)=(4\cdot 3,4\cdot 4)=4\cdot (3,4) = 4$, so we need to determine if, mod 17, 
$\displaystyle 16^{{17-1}\over{4}} = 16^4\equiv 1$ . But $16\equiv -1$, so $16^4\equiv (-1)^4 = 1$, as desired.
Therefore, $x^{12}\equiv 16$ (mod 17)  has $(12,16)=4$ solutions.

\msk

\hskip1in (b) $x^{48}\equiv 9$ (mod 17)

\msk

\item{} $(48,17-1)=(48,16)=16$, so we need to determine if, mod 17, 
$\displaystyle 9^{{17-1}\over{16}} = 9^1 = 9\equiv 1$ . But it isn't; it is $9\not\equiv 1$ . So
$x^{48}\equiv 9$ (mod 17) has no solutions.

\msk

\hskip1in (c) $x^{20}\equiv 13$ (mod 17)

\msk

\item{} $(20,17-1)=(20,16)=4\cdot (5,4) =4$, so we need to determine if, mod 17, 
$\displaystyle 13^{{17-1}\over{4}} = 13^4\equiv 1$ . But, mod 17, $13^2=169\equiv -1$,
so $13^4\equiv (-1)^2=1$, as desired. So $x^{20}\equiv 13$ (mod 17) has $(20,16)=4$ solutions.

\msk

\hskip1in (d) $x^{11}\equiv 9$ (mod 17)

\msk


\item{} $(11,17-1)=(11,16)=1$ (since $1=3\cdot 11-2\cdot 16$), so we need to determine if, mod 17, 
$\displaystyle 9^{{17-1}\over{1}} = 9^{16}\equiv 1$ . But since (9,17)=1 (since $2\cdot 9-1\cdot 17=1$),
$9^{16}\equiv 1\pmod{17}$ by Fermat's Little Theorem. So $x^{11}\equiv 9$ (mod 17) has 1 solution.

\bsk

\item{23.}  If $p$ is a prime, and $p\equiv 3$ (mod 4), show that the congruence equation

\ssk

\item{} $x^4\equiv a$ (mod $p$) has a solution $\Leftrightarrow$ $x^2\equiv a$ (mod $p$) does.

\ssk

\item{} On the other hand, show (by example) that if $p\equiv 1$ (mod 4) this result need \underbar{not}
be true.

\msk

\item{} Since $p\equiv 3$ (mod 4), $p-1\equiv 2$ (mod 4), so $p-1=4k+2 = 2(2k+1)$ for some $k$.
Then $(4,p-1)$ = $(2\cdot 2,2(2k+1)) = 2(2,2k+1) = 2$ . By our result from class, $x^4\equiv a$ (mod $p$)
has a solution $\Leftrightarrow$ $\displaystyle a^{{p-1}\over{(4,p-1)}} = a^{{p-1}\over{2}} \equiv 1\pmod{p}$.
But since $2|p-1$ , $(2,p-1)=2$, and so by the same result, $x^2\equiv a$ (mod $p$)
has a solution $\Leftrightarrow$ $\displaystyle a^{{p-1}\over{(2,p-1)}} = a^{{p-1}\over{2}} \equiv 1\pmod{p}$.

\ssk

\item{} So $x^4\equiv a$ (mod $p$) has a solution $\Leftrightarrow$ $\displaystyle a^{{p-1}\over{2}} \equiv 1\pmod{p}$
$\Leftrightarrow$ $x^2\equiv a$ (mod $p$) has a solution, as desired.

\msk

\item{} On the other hand, for $p=17$ and $a=2$, $a^4=16\equiv -1$ (mod 17), and so $a^8\equiv (-1)^2=1$ (mod 17).
So $\dl a^{{{p-1}\over{(4,p-1}}}\not\equiv 1$ (mod 17), so $x^4\equiv 2$ (mod 17) has no solution; but
$\dl a^{{{p-1}\over{(2,p-1}}}\equiv 1$ (mod 17), so $x^2\equiv 2$ (mod 17) has a solution.

\bsk

\item{24.} [NZM, p.106, \#\ 2.8.13] Show that, for a prime $p$, the numbers $1^k,2^k,\ldots (p-1)^k$
are all \underbar{distinct} mod $p$ $\Leftrightarrow$ $(k,p-1)=1$.

\msk

\item{} This result is immediate for $p=2$; there is only one element, $1^k$, to look at, but $p-1=1$, so $(k,p-1)=1$
for all $k$.

\ssk


\item{} For $p>2$ prime, if the $a^k$ are all distinct mod $p$, then the function 

\ctln{$F:\{1,\ldots,p-1\}\rightarrow\{1,\ldots,p-1\}$ given by $F(x)=x^k$ (mod $p$)}

\item{} is one-to-one. (The range is right, since $(x,p)=1$ implies $(x^k,p)=1$ (i.e., if $p|x^k$ then $p|x$).)
But then by the pigeonhole principle, $F$ is also onto. So for any $a$ with $(a,p-1)=1$, the equation
$x^k\equiv a$ (mod $p$) has a solution. Since the $k$-th powers are all unique, it has exactly one
solution. So by our result giving the count of solutions to such equations, since, if $x^k\equiv a$ (mod $p$) 
has a solution, it has precisely $(k,p-1)$ solutions, we must have $(k,p-1)=1$.

\ssk

\item{} On the other hand, if two of the powers are equal, mod $p$, we have $a^k\equiv b^k$ (mod $p$) for
$a$ and $b$ distinct mod $p$; setting $c=a^k$, we then have two solutions, mod $p$, to 
$x^k\equiv c$ (mod $p$). Since the number of solutions to this equation, of positive, is
$(k,p-1)$, we must therefore have $(k,p-1)\geq 2$, and therefore $(k,p-1)>1$. So if $(k,p-1)=1$,
then all of the $a^k$ must be distinct, mod $p$.

\vfill\end

