%\baselineskip=18pt plus 2pt

\magnification=1200
\voffset=-.5in
\vsize=8.5in
\parindent=50pt

\def\ni{\noindent}
\def\ctln{\centerline}
\def\msk{\medskip}
\def\ssk{\smallskip}
\def\bsk{\bigskip}
\def\hsk{\hskip.2in}
\def\iit{\itemitem}
\def\htp{\hskip10pt}
\def\vtp{\vskip.02in}

%\input shorthand

\nopagenumbers

\ctln{\bf Math 203 Contemporary Math}

%\vtp

\ctln{\bf Section 005}

\smallskip

\ni{\bf Lecture:} MWF 10:30-11:20, in Avery Hall (AvH) 352

\msk

\ni{\bf Instructor:} Mark Brittenham

%\smallskip

\ni{\bf Office:} Oldfather Hall (OldH) 819 
%\smallskip

\ni{\bf Telephone:} (47)2-7222

%\ssk

\ni{\bf E-mail:} mbritten@math.unl.edu

\ni{\bf WWW:} http://www.math.unl.edu/\~{ }mbritten/

\ni{\bf WWW pages for this class:} http://www.math.unl.edu/\~{ }mbritten/classwk/203s99/

\smallskip

\ni{\bf Office Hours:} (tentatively) Mo 1:30-2:30, Tu 2:00 - 3:00, We 9:30-10:30, 
and Th 1:00 - 2:00, and whenever you can find me in my office and I'm not 
horrendously busy. You are also quite welcome to make an appointment
for any other time; this is easiest to arrange just before or 
after class.

\ssk

\ni{\bf Text:} {\it For All Practical Purposes}, by Solomon Garfunkel and friends 
(4th edition).

\smallskip

\ni This course, as the name is meant to imply, is intended to give us a chance to 
look at
some of the problems, methods, and results of contemporary mathematical thinking.
Our goal is not so much to learn specific skills, as it is in most other mathematics 
courses;
our interest is more to see how mathematics fits into the modern world, to 
develop
problem solving skills, and to develop communications skills, especially in communicating
mathematical ideas.

\ni Our basic goal will be to work through the following chapters:

\ssk

Part 1, Management Science

\hsk Ch. 1, Street Networks

\hsk Ch. 2, Visiting Vertices

\hsk Ch. 3, Planning and Scheduling

Part 2, Statistics: The Science of Data

\hsk Ch. 5, Producing Data

\hsk Ch. 6, Describing Data


\hsk Ch. 7, Probability: The Mathematics of Chance

\hsk Ch. 8, Statistical Inference

Part 4, Social Choice and Decision Making

\hsk Ch. 11, Social Choice: The Impossible Dream

\hsk Ch. 13, Fair Division

\ssk

\ni plus whatever else time and interest will allow.

\ssk

\ni{\bf Homework} will be assigned nearly every day. 
It is an essential ingredient to the course - as with almost all of 
mathematics, we learn best by doing (again and again and ...). Cooperation 
with other students on these assignments is acceptable, and even 
encouraged. However, you should try working through problems first on your 
own - after 
all, you get to bring only one brain to exams (and it can't be someone 
else's). Part of the homework set will serve as the foundation on which the next
class discussion will be based; it is therefore essential that you try to work 
through these 
before the next class period. One problem from each set will be designated to 
be turned in and graded; these problems will count 15\% toward your final grade.

\ssk

\ni{\bf Midterm exams} will be given twice during the 
semester - the specific dates will 
be announced in class well in advance (likely candidates: early/mid-February, 
end of March). They will cover the material from Chapters 1 thru 3, and 5 thru 8. 
Each exam will count 15\% toward your grade. 
You can take a make-up exam only if there are compelling reasons (a doctor SAYS 
you were sick, jury duty, etc.) for you to miss an exam. Make-up 
exams tend to be harder than the originals (because make-up exams 
are harder to write!).

\ni\hskip.2in Each additional chapter will be followed by a 25-30 minute quiz; these 
quizzes will 
together count a further 15\% toward your grade. After each exam or quiz is 
returned,
you will have the opportunity to turn in corrections, to earn back up to one-fourth of
the points that you lost. These must be turned on by the end of the next class period.

\ni\hskip.2in This course has no final exam.

\ssk

\ni{\bf Writing assignments} are an integral part of this class, since this course
may be used to meet the Integrated Studies requirement. These will come in two flavors.
Shorter assignments, given approximately every other week, will focus mainly on your
assessment of the course and your progress in it. These will count 15\% toward your
grade. The longer assignments (probably two) will involve much more significant 
mathematical content; you will essentially carry out an analysis of a mathematical
problem, and write a report of your findings. The longer assignments will also count
15\% toward your grade. Both types of writings will be graded on content and grammar
(to the extent that a mathematician can grade someone else's grammar...). Late
assignments cannot earn full credit, and excessively late assignments will not 
be accepted.

\ni{\bf Class attendance} is probably your best way to insure that you 
will keep 
up with the material of the course, and make sure that you understand all of the
concepts involved. In many cases we will spend much of our time in class working
through examples of newly introduced concepts. A missed class therefore means
missing part of the work of the course. Attendance will be taken each day, and constitute
the remaining 10\% of the grade. Each student will be allowed three unexcused
absences, to allow for the inevitable complications of modern life.

\msk

\ni {\bf Your course grade} will be calculated numerically using the above scales,
and will be converted to a letter grade based partly on the overall average of the
class. However, a score of 90\% or better will guarantee some kind of {\bf A}, 80\%
or better at least some sort of {\bf B}, 70\% or better at least a flavor of 
{\bf C}, and 60\% or 
better at least a {\bf D}.

\bsk 

\ni In mathematics, new concepts continually rely upon the mastery
of old ones; it is therefore essential that you thoroughly understand each 
new topic before moving on. Our classes are an important opportunity for you to ask
questions; to make \underbar{sure} that you are understanding concepts correctly.
Speak up! It's \underbar{your} education at stake. Make every effort to resist
the temptation to put off work, and to fall behind. Every topic has to be gotten 
through, not around. And it's alot easier to read 50 pages in a week than it is
in a day. Try to do some work for the class every single day. (I do.)

\msk

\ni{\bf Departmental Grading Appeals Policy:} Students who believe their
academic evaluation has been perjudiced or capricious have recourse for appeals 
to (in order) the instructor, the departmental chair, the departmental appeals 
committee, and 
the college appeals committee.

\msk

\ctln{\bf Some important academic dates}

\ssk

{\bf Jan. 11} First day of classes.

{\bf Jan. 18} Martin Luther King Day - no classes.

{\bf Jan. 22} Last day to withdraw from a course without a {\bf `W'}.

{\bf Mar. 5} Last day to change to or from P/NP.

{\bf Mar. 14-21} Spring break - no classes.

{\bf Apr. 9} Last day to withdraw from a course.

{\bf May 1} Last day of classes.

\vfill

\end

\vfill\eject
