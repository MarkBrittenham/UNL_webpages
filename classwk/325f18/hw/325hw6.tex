

\documentclass[12pt]{article}
\usepackage{amsfonts}
\usepackage{amssymb}
\usepackage{dutchcal}

\textheight=10in
\textwidth=6.5in
\voffset=-1in
\hoffset=-1in

\begin{document}

\def\ctln{\centerline}
\def\msk{\medskip}
\def\bsk{\bigskip}
\def\ssk{\smallskip}
\def\hsk{\hskip.3in}
\def\ra{\rightarrow}
\def\ubr{\underbar}
\def\dsp{\displaystyle}

\def\mt{{\mathcal T}}
\def\mb{{\mathcal B}}
\def\ms{{\mathcal S}}
\def\mu{{\mathcal U}}
\def\mv{{\mathcal V}}

\def\bbr{{\mathbb R}}
\def\bbz{{\mathbb Z}}
\def\bbq{{\mathbb Q}}
\def\bbn{{\mathbb N}}
\def\spc{$~$\hskip.15in$~$}

\def\sset{\subseteq}
\def\del{\partial}
\def\lra{$\Leftrightarrow$}
\def\bra{$\Rightarrow$}



%%\UseAMSsymbols

\ctln{\bf Math 325 Problem Set 6}

\msk

\ctln{Starred (*) problems are due Friday, October 12.}


\begin{description}


\item{(*) 31.} (Belding and Mitchell, p.80, \#2) Show that if $f:(a,b)\ra\bbr$ is a continuous function, then the function
$g:(a,b)\ra\bbr$ given by $g(x)=|f(x)|$ is also continuous. (You should argue directly
from $\epsilon$'s and $\delta$'s.)

\msk

\item{32.} Using the previous problem (and a problem from a previous problem set!), show that
if $f,g:(a,b)\ra\bbr$ are continuous functions, then the function
$M:(a,b)\ra\bbr$ given by $M(x)=$max$\{f(x),g(x)\}$ is also continuous.

\msk

\item{(*) 33.} Suppose that $f:\bbr\ra\bbr$ is continuous and $f(x)=0$ for every $x\in\bbq$.
Show that $f(x)=0$ for every $x\in\bbr$ .

\ssk

\item{\spc} [Hint: what's the alternative? Remember that rational numbers are `everywhere'!]

\msk

\item{34.} Using the problem above, show that if $f,g:\bbr\ra\bbr$ are both continuous
functions, and $f(x)=g(x)$ for every $x\in\bbq$, then $f=g$ (i.e., $f(x)=g(x)$ for every
$x\in\bbr$).

\ssk

\item{\spc} [`A continuous function is \ubr{determined} by its values on the rational numbers.']


\msk

\item{35.} Suppose that $a<0<b$ and $f:(a,b)\ra\bbr$ is a function
that is bounded (i.e., for some $M\in\bbr$, $|f(x)|\leq M$ for every $x\in (a,b)$).
Show that the function $g:(a,b)\ra \bbr$ defined by $g(x)=xf(x)$ is continuous at $x=0$.
Show, on the other hand, that for any \ubr{other} $c\in(a,b)$ we have that $g$ is continuous
at $c$ \ubr{if} \ubr{and} \ubr{only} \ubr{if} $f$ is continuous at $c$.

\ssk

\item{\spc} [The last assertion can be attacked using `general' results we have established,
\ubr{or} directly using $\epsilon$'s and $\delta$'s (your choice!).]

\msk

\item{(*) 36.} (Belding and Mitchell, p.89, \#9) Use the intermediate value theorem 
to show that any positive number $a\in\bbr$, $a>0$
has an $n$-th root, that is, for any $n\in\bbn$, there is some real number $x\geq 0$ such
that $x^n=a$.

\ssk

\item{\spc} [The textbook provides an outline that you could follow.]

\msk

\item{37.} Show that if $f:[a,b]\ra [a,b]$ is continuous, then there is 
at least one $c\in [a,b]$ so that $f(c)=c$.

\ssk

\item{\spc} [Hint: apply the intermediate value theorem to a different function!]

\msk


\end{description}
\vfill

\end{document}

