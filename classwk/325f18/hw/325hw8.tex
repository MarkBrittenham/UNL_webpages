

\documentclass[12pt]{article}
\usepackage{amsfonts}
\usepackage{amssymb}
\usepackage{dutchcal}

\textheight=10in
\textwidth=6.5in
\voffset=-1in
\hoffset=-1in

\begin{document}

\def\ctln{\centerline}
\def\msk{\medskip}
\def\bsk{\bigskip}
\def\ssk{\smallskip}
\def\hsk{\hskip.3in}
\def\ra{\rightarrow}
\def\ubr{\underbar}
\def\dsp{\displaystyle}

\def\mt{{\mathcal T}}
\def\mb{{\mathcal B}}
\def\ms{{\mathcal S}}
\def\mu{{\mathcal U}}
\def\mv{{\mathcal V}}

\def\bbr{{\mathbb R}}
\def\bbz{{\mathbb Z}}
\def\bbq{{\mathbb Q}}
\def\bbn{{\mathbb N}}
\def\spc{$~$\hskip.15in$~$}

\def\sset{\subseteq}
\def\del{\partial}
\def\lra{$\Leftrightarrow$}
\def\bra{$\Rightarrow$}



%%\UseAMSsymbols

\ctln{\bf Math 325 Problem Set 8}

\msk

\ctln{Starred (*) problems are due Friday, October 26.}


\begin{description}


\item{45.} Show that the function $f:\bbr\ra\bbr$ given by $f(x)=x\cdot |x|$ is 
differentiable at every $a\in\bbr$. 

\ssk

\item{\spc} [Hint: in most cases, you can (if you stay close to $a$) replace $f$ with a more 
`amenable' function...]

\msk

\item{46.} (Belding and Mitchell, p.99, \#3(b)) Show that if $0\in D$ and  $f:D\ra\bbr$ is continuous at
$a=0$, then $g:D\ra\bbr$ given by $g(x)=xf(x)$ is differentiable at $a=0$.

\msk

\item{(*) 47.} (Belding and Mitchell, p.100, \#10 (sort of)) If $a\in D$ and $f:D\ra\bbr$ is differentiable
at $a$ \ubr{and} $f^{\prime}(a)>0$, show that there is a $\delta>0$ 
$x\in (a,a+\delta)$ implies that $f(x)>f(a)$ and $x\in (a-\delta)$ implies that $f(x)<f(a)$ .

\msk

\item{(*) 48.} Show that if $a\in D$, $f,g:D\ra\bbr$ are both differentiable and $x=a$, 
$f(a)=g(a)$, \ubr{and} $f(x)\leq g(x)$ for all $x\in D$, then $f^{\prime}(a)=g^{\prime}(a)$.

\ssk

\item{\spc} [What's the alternative? The previous problem can help!]
 
\msk

\item{49.} (The `Squeeze Play Theorem' for derivatives):
Show that if $a\in D$, $f,g,h:D\ra\bbr$ are functions with
$f(x)\leq g(x)\leq h(x)$ for all $x\in D$, $f(a)=g(a)=h(a)$, \ubr{and} $f$ and $h$ are differentiable at $x=a$, 
then $g$ is differentiable at $x=a$ and $f^{\prime}(a)=g^{\prime}(a)=h^\prime(a)$.

\ssk

\item{\spc} [Note that the only 'new' things here are that $g$ is differentiable at $a$ and $g^\prime(a)=f^\prime(a)$ ...]

\msk

\item{(*) 50.} Suppose that $f:\bbr\ra\bbr$ and $g:\bbr\ra\bbr$
are both continuous, and $f$ is differentiable at $x=0$, with
$f(0)=f^\prime(0)=0$. Show that $h(x)=f(x)g(x)$ is also 
differentiable at $x=0$ and $h^\prime(0)=0$ .

\ssk

\item{\spc} [Note that since we do \ubr{not} know that $g$ is differentiable at
$x=0$, we \ubr{cannot} use the product rule (even if we knew what that was)....]

\msk

\item{51.} As almost none of us learn, the angle sum formula for tangent is

\ssk

\ctln{$\dsp \tan(a+h)={{\tan a + \tan h}\over{1-\tan a\tan h}}$ }

\ssk

\item{\spc} Use this to show directly from the (``limit as $h\ra 0$'' definition) that the 
derivative of $f(x)=\tan x$ is what you were told it is in calculus class.

\ssk

\item{\spc} [If you want something extra to do, derive this angle sum formula from the
angle sum formulas for $\sin x$ and $\cos x$ (for fun!).]


\end{description}
\vfill

\end{document}

