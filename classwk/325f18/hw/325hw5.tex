

\documentclass[12pt]{article}
\usepackage{amsfonts}
\usepackage{amssymb}
\usepackage{dutchcal}

\textheight=10in
\textwidth=6.5in
\voffset=-1in
\hoffset=-1in

\begin{document}

\def\ctln{\centerline}
\def\msk{\medskip}
\def\bsk{\bigskip}
\def\ssk{\smallskip}
\def\hsk{\hskip.3in}
\def\ra{\rightarrow}
\def\ubr{\underbar}
\def\dsp{\displaystyle}

\def\mt{{\mathcal T}}
\def\mb{{\mathcal B}}
\def\ms{{\mathcal S}}
\def\mu{{\mathcal U}}
\def\mv{{\mathcal V}}

\def\bbr{{\mathbb R}}
\def\bbz{{\mathbb Z}}
\def\bbq{{\mathbb Q}}
\def\bbn{{\mathbb N}}
\def\spc{$~$\hskip.15in$~$}

\def\sset{\subseteq}
\def\del{\partial}
\def\lra{$\Leftrightarrow$}
\def\bra{$\Rightarrow$}



%%\UseAMSsymbols

\ctln{\bf Math 325 Problem Set 5}

\msk

\ctln{Starred (*) problems are due Friday, September 28.}


\begin{description}


\item{24.} Show that if $0\leq x<1$ then for any $\epsilon>0$ there is an $n\in\bbn$ so that $x^n<\epsilon$. 

\ssk

\item{\spc} [Hint: Suppose not! Then look at lower bounds for $A=\{x^n : n\in\bbn\}$ .] 

\msk

\item{(*) 25.} (Belding and Mitchell, p.63, \#1) Using the $\epsilon$-$\delta$ formulation of limits,

\item{(*)} (b) show that $\dsp\lim_{x\ra -1} 1-2x = 3$ .

\item{\spc} (c) show that $\dsp \lim_{x\ra 5} \sqrt{x+4} = 3$ .

\item{(*)} (e) show that $\dsp \lim_{x\ra -3} {{1}\over{8-4x}} = {{-1}\over{4}}$ .

\msk

\item{26.} (Belding and Mitchell, p.63, \#2) Suppose that $g:\bbr\ra\bbr$ is a {\it bounded}
function, i.e., there is an $M\in\bbr$ so that $|g(x)|\leq M$ for all $x\in\bbr$. Show 
that $\dsp\lim_{x\ra 0} xg(x) = 0$.

\msk

\item{27.} (Belding and Mitchell, p.64, \#6) Show that if $\dsp\lim_{x\ra a} f(x) = L$, then 
for any $\eta>0$ there is a $\tau>0$ so that $0<|x_1-a|<\tau$ and $0<|x_2-a|<\tau$ imply that
$|f(x_1)-f(x_2)|<\eta$ . [I.e., ``if $f$ has a limit as we approach $a$, then points close (enough)
to $a$ have values close to \ubr{one} \ubr{another}''.]

\ssk

\item{\spc} [N.B. This is a useful way to show that a function has {\it no} limit as $x$ approaches $a$ : 
show that this conclusion does \ubr{not} hold!]

\msk

\item{(*) 28.} (The `Squeeze Play' Theorem) Suppose that $f,g,h:\bbr\ra\bbr$,
are functions with  with $f(x)\leq g(x)\leq h(x)$ for all $x\in\bbr$
with $0<|x-a|<M$ for some $a\in\bbr$ and $M>0$.
Suppose further that 

\item{\spc} $\displaystyle \lim_{x\rightarrow a}f(x)=L=\lim_{x\rightarrow a}h(x)$.
Show that $\displaystyle \lim_{x\rightarrow a}g(x)=L$ .

\ssk

\item{\spc} [See Belding and Mitchell, p.64, \#8 for an outline that you might follow.]

\msk

\item{(*) 29.} (Belding and Mitchell, p.64, \#9) If $f:\bbr\ra\bbr$ is a function,
$\dsp\lim_{x\ra a} f(x)=L$, and for some $K,M\in\bbr$ with $M>0$ we have $f(x)\leq K$ for
all $x$ with $0<|x-a|<M$, show that $L\leq K$ . 

\ssk

\item{\spc} [What's the alternative?]

\msk

\item{30.} (Belding and Mitchell, p.69, \#1) Show, using our limit theorems and induction, that
for every $n\in\bbz$ we have $\dsp\lim_{x\ra a} x^n = a^n$ (where we assume that $a\neq 0$ when 
$n<0$).

\msk


\end{description}
\vfill

\end{document}

