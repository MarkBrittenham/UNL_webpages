

\documentclass[12pt]{article}
\usepackage{amsfonts}
\usepackage{amssymb}
\usepackage{dutchcal}

\textheight=10in
\textwidth=6.5in
\voffset=-0.5in
\hoffset=-1in

\begin{document}

\def\ctln{\centerline}
\def\msk{\medskip}
\def\bsk{\bigskip}
\def\ssk{\smallskip}
\def\hsk{\hskip.3in}
\def\ra{\rightarrow}
\def\ubr{\underbar}
\def\dsp{\displaystyle}

\def\mt{{\mathcal T}}
\def\mb{{\mathcal B}}
\def\ms{{\mathcal S}}
\def\mu{{\mathcal U}}
\def\mv{{\mathcal V}}

\def\bbr{{\mathbb R}}
\def\bbz{{\mathbb Z}}
\def\bbq{{\mathbb Q}}
\def\bbn{{\mathbb N}}
\def\spc{$~$\hskip.15in$~$}

\def\sset{\subseteq}
\def\del{\partial}
\def\lra{$\Leftrightarrow$}
\def\bra{$\Rightarrow$}



%%\UseAMSsymbols

\ctln{\bf Math 325 Problem Set 2}

\msk

\ctln{Starred (*) problems are due Friday, September 7.}


\begin{description}

\item{(*) 5.} For the statements (a)-(c) below, state both the 
contrapositive and the `proof by contradiction' form ``it is not possible to have both...'' 
versions of the given statement, and indicate (no explanation needed) 
which of the resulting collections of statements are \ubr{true}. 

\ssk

\item{\spc} (a) If $a\in\bbq$ and $b\in\bbq$, then $a+b\in\bbq$ .

\ssk

\item{\spc} (b) If $a\not\in\bbq$, then $\dsp {{1}\over{a}}\not\in\bbq$ .

\ssk

\item{\spc} (c) If $a\not\in\bbq$ and $b\not\in\bbq$, then $ab\not\in\bbq$ .

\msk

\item{(*) 6.} Show, using the Rational Roots Theorem, that $\alpha = \sqrt{2+\sqrt{3}}$ is not rational.

\ssk

\item{\spc} [Find a polynomial with integer coefficients that has $\alpha$ as a root!]

\msk

\item{7.} Show, using the Rational Roots Theorem, that $\beta = 2^{1/2}+5^{1/3}$ is not rational.

\ssk

\item{\spc} [Hint: start by raising $\beta-\sqrt{2}$ to the third power...!]

\msk

\item{(*) 8.} By looking at the first few cases, find a (short)
formula for the sum

\ssk

\ctln{$\dsp {{1}\over{1\cdot 2}} + {{1}\over{2\cdot 3}} + \cdots + {{1}\over{n(n+1)}} = \sum_{k=1}^n{{1}\over{k(k+1)}}$ ;}

\ssk

\item{\spc} then prove, using induction, that your formula is correct.

\msk

\item{9.} Show, using induction, that for every $n\in\bbn$ we have 

\ssk

\hskip1in $\displaystyle \sum_{k=1}^n k(k+1)(k+2) = {{1}\over{4}}n^4+{{3}\over{2}}n^3+{{11}\over{4}}n^2+{{3}\over{2}}n$ .

\ssk

\item{\spc} [Hint: going straight at it is fine; a `slicker' way is to note that the quartic \underbar{is}

\ssk

\hskip1in $\displaystyle {{1}\over{4}}n(n+1)(n+2)(n+3)$ ...]

\end{description}
\vfill

\end{document}

