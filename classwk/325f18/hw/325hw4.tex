

\documentclass[12pt]{article}
\usepackage{amsfonts}
\usepackage{amssymb}
\usepackage{dutchcal}

\textheight=10in
\textwidth=6.5in
\voffset=-1in
\hoffset=-1in

\begin{document}

\def\ctln{\centerline}
\def\msk{\medskip}
\def\bsk{\bigskip}
\def\ssk{\smallskip}
\def\hsk{\hskip.3in}
\def\ra{\rightarrow}
\def\ubr{\underbar}
\def\dsp{\displaystyle}

\def\mt{{\mathcal T}}
\def\mb{{\mathcal B}}
\def\ms{{\mathcal S}}
\def\mu{{\mathcal U}}
\def\mv{{\mathcal V}}

\def\bbr{{\mathbb R}}
\def\bbz{{\mathbb Z}}
\def\bbq{{\mathbb Q}}
\def\bbn{{\mathbb N}}
\def\spc{$~$\hskip.15in$~$}

\def\sset{\subseteq}
\def\del{\partial}
\def\lra{$\Leftrightarrow$}
\def\bra{$\Rightarrow$}



%%\UseAMSsymbols

\ctln{\bf Math 325 Problem Set 4}

\msk

\ctln{Starred (*) problems are due Friday, September 21.}


\begin{description}


\item{(*) 17.} (Belding and Mitchell, p.36, \#20)

\ssk

\item{(*)} (a) Show that if $x,y,c\in\bbr$, $c>0$, and $|x-y|<c$, then $|x|<|y|+c$ .

\ssk

\item{(*)} (b) Show that if $x,y\in\bbr$ and $\displaystyle |x-y|<{{|x|}\over{2}}$, 
then $\displaystyle |y|>{{|x|}\over{2}}$ .

\msk

\item{18.} A set $A$ is said to be {\it bounded away from $0$} if there is 
an $\epsilon>0$ so that for every $x\in A$ we have $|x|>\epsilon$. Show that $A$ is bounded away
from $0$ if and only if the set $B = \{ {{1}\over{x}}\ |\ x\in A\}$ is bounded.

\ssk

\item{\spc} [N.B. ``P if and only if Q'' means P implies Q \ubr{and} Q implies P; that is, 
there are two things to show!]

\msk

\item{19.} If we set $A = \{x\in\bbr\ |\ x^3<2\}$, show that $A$ is bounded above,
so has a supremum $\alpha=\textrm{sup}(A)$. Then show (in a manner similar to our
classroom demonstrations) that $\alpha^3<2$ is not possible. (If you are feeling
like doing even more, show that $\alpha^3>2$ is also impossible! 
From that, we can conclude that $\alpha^3=2$.)

\msk

\item{(*) 20.} (Belding and Mitchell, p.22, \#2) Show that
if a set of real numbers $S$ has a least upper bound $\alpha$, 
then this least
upper bound is \underbar{unique}. That is, if $\beta$ is also a least upper bound for $S$,
then $\alpha=\beta$. [Hint: what's the alternative?]

\msk

\item{21.} (Belding and Mitchell, p.23, \#6) For subsets 
$A,B\subseteq\bbr$, we define their `sum' as
$A+B=\{a+b\ :\ a\in A, b\in B\}$. 

\ssk

\item{\spc} Show that if $A$ and $B$ are both non-empty and
bounded from above, then so is $A+B$, and 

\ctln{$\textrm{lub}(A+B)=\textrm{lub}(A)+\textrm{lub}(B)$ .}

\ssk

\item{\spc} [Hint: show that $\textrm{lub}(A)+\textrm{lub}(B)$ is an upper bound!
Then worry about whether there might be a smaller one...]

\msk

\item{(*) 22.} (Belding and Mitchell, p.23, \#4) Let 
$A = \{a_1,a_2,a_3, \ldots\}=\{a_n\ :\ n\in\bbn\}$ and $B = \{b_1,b_2,b_3, \ldots\}
=\{b_n\ :\ n\in\bbn\}$ be two sequences of real
numbers. Let $C = \{a_n+b_n\ :\ n\in\bbn\}$, the sequence of their sums.

\ssk
\item{(*)} (a) Show that if $A$ and $B$ have least upper bounds $\alpha$ and $\beta$, respectively,
then $\alpha+\beta$ is an upper bound for $C$.

\ssk

\item{(*)} (b) Find an example showing that $\alpha+\beta$ \ubr{need} \ubr{not} \ubr{be} the least upper
bound for $C$.

\msk

\item{23.} (Belding and Mitchell, p.23, \#7) Show that if $\alpha,\beta\in\bbr$ and $\alpha<\beta$, then 
there is an \ubr{irrational} number $c\not\in\bbq$ with $\alpha<c<\beta$ . 

\ssk

\item{\spc} [Hint: $c$ could be a rational
multiple of $\sqrt{2}$ (why is that not rational?). Or see the outline that the text provides!]

\msk


\end{description}
\vfill

\end{document}

