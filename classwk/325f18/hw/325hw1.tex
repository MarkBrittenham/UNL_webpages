

\documentclass[12pt]{article}
\usepackage{amsfonts}
\usepackage{amssymb}
\usepackage{dutchcal}

\textheight=10in
\textwidth=6.5in
\voffset=-0.5in
\hoffset=-1in

\begin{document}

\def\ctln{\centerline}
\def\msk{\medskip}
\def\bsk{\bigskip}
\def\ssk{\smallskip}
\def\hsk{\hskip.3in}
\def\ra{\rightarrow}
\def\ubr{\underbar}

\def\mt{{\mathcal T}}
\def\mb{{\mathcal B}}
\def\ms{{\mathcal S}}
\def\mu{{\mathcal U}}
\def\mv{{\mathcal V}}

\def\bbr{{\mathbb R}}
\def\bbz{{\mathbb Z}}
\def\bbq{{\mathbb Q}}
\def\bbn{{\mathbb N}}
\def\spc{$~$\hskip.15in$~$}

\def\sset{\subseteq}
\def\del{\partial}
\def\lra{$\Leftrightarrow$}
\def\bra{$\Rightarrow$}



%%\UseAMSsymbols

\ctln{\bf Math 325 Problem Set 1}

\msk

\ctln{Starred (*) problems are due Friday, August 31.}


\begin{description}

\item{1.} Let $S = \{ x\in\bbr \mid x^2+x = 0\}$ and
$T = \{ x\in\bbr \mid x^2+x < 5\}$.

\item{(*)} (a) Write $S$ and $T$ as (small) unions of points and/or intervals.

\ssk

\item{(*)} (b) Decide whether each of the following statements is true, and
(briefly) explain: 

\ssk

\ctln{$S\subseteq\bbn$ ; $S\subseteq T$ ; $T\cap\bbq\neq\emptyset$ ; $-2.8\in \bbq\setminus T$ .}

\ssk

\item{\spc} (c) Describe the set $U = \{x\in\bbr \mid x^2+x<0\}$ as a union of intervals.

\msk

\item{2.} Starting with a set $S$, we can construct a new
set $P(S)$, the \ubr{power} \ubr{set} of $S$, consisting of all subsets of $S$.
For example, $P(\{1,2\}) = \{\emptyset,\{1\},\{2\},\{1,2\}\}$. 

\ssk

\item{\spc} (a) Find $P(\{1,2,3\})$.

\ssk

\item{\spc} (b) Show that if $S\sset T$, then $P(S)\sset P(T)$.

\ssk

\item{\spc} (c) If we set $N_k = \{1,2,\ldots,k\}$, explain why
$P(N_{11})$ has twice as many elements as $P(N_{10})$.

\msk

\item{3.} Let $L$ be the (linear) function
$L(x) = ax+b$, where $a$ and $b$ are (real) constants and $a\neq 0$.

\ssk

\item{(*)} (a) Explain why $L$ is both one-to-one and onto.

\ssk

\item{(*)} (b) Find a formula for the inverse function 
$M=L^{-1}$, and show that 

\vspace{-.07in}

\hspace{.3in} $L\circ M(x) = M\circ L(x) = x$ 
for every $x\in\bbr$.

\msk

\item{4.} Suppose the $f:A\ra B$ and $g:B\ra C$
are both functions.

\ssk

\item{\spc} (a) Show that if $f$ and $g$ are both one-to-one, 
then $g\circ f:A\ra C$ is one-to-one.

\ssk

\item{(*)} (b) Show that if $g\circ f$ is onto, then $g$ is onto.

\end{description}
\vfill

\end{document}

