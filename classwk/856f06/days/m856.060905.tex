

\magnification=1200
\overfullrule=0pt
%\parindent=0pt

%\nopagenumbers

\input amstex
\loadmsbm

%\voffset=-.6in
%\hoffset=-.5in
%\hsize = 7.5 true in
%\vsize=10.4 true in

%\voffset=1.4in
%\hoffset=-.5in
%\hsize = 10.2 true in
%\vsize=8 true in

\input colordvi

\def\cltr{\Red}		  % Red  VERY-Approx PANTONE RED
\def\cltb{\Blue}		  % Blue  Approximate PANTONE BLUE-072
\def\cltg{\PineGreen}	  % ForestGreen  Approximate PANTONE 349
\def\cltp{\DarkOrchid}	  % DarkOrchid  No PANTONE match
\def\clto{\Orange}	  % Orange  Approximate PANTONE ORANGE-021
\def\cltpk{\CarnationPink}	  % CarnationPink  Approximate PANTONE 218
\def\clts{\Salmon}	  % Salmon  Approximate PANTONE 183
\def\cltbb{\TealBlue}	  % TealBlue  Approximate PANTONE 3145
\def\cltrp{\RoyalPurple}	  % RoyalPurple  Approximate PANTONE 267
\def\cltp{\Purple}	  % Purple  Approximate PANTONE PURPLE

\def\cgy{\GreenYellow}     % GreenYellow  Approximate PANTONE 388
\def\cyy{\Yellow}	  % Yellow  Approximate PANTONE YELLOW
\def\cgo{\Goldenrod}	  % Goldenrod  Approximate PANTONE 109
\def\cda{\Dandelion}	  % Dandelion  Approximate PANTONE 123
\def\capr{\Apricot}	  % Apricot  Approximate PANTONE 1565
\def\cpe{\Peach}		  % Peach  Approximate PANTONE 164
\def\cme{\Melon}		  % Melon  Approximate PANTONE 177
\def\cyo{\YellowOrange}	  % YellowOrange  Approximate PANTONE 130
\def\coo{\Orange}	  % Orange  Approximate PANTONE ORANGE-021
\def\cbo{\BurntOrange}	  % BurntOrange  Approximate PANTONE 388
\def\cbs{\Bittersweet}	  % Bittersweet  Approximate PANTONE 167
%\def\creo{\RedOrange}	  % RedOrange  Approximate PANTONE 179
\def\cma{\Mahogany}	  % Mahogany  Approximate PANTONE 484
\def\cmr{\Maroon}	  % Maroon  Approximate PANTONE 201
\def\cbr{\BrickRed}	  % BrickRed  Approximate PANTONE 1805
\def\crr{\Red}		  % Red  VERY-Approx PANTONE RED
\def\cor{\OrangeRed}	  % OrangeRed  No PANTONE match
\def\paru{\RubineRed}	  % RubineRed  Approximate PANTONE RUBINE-RED
\def\cwi{\WildStrawberry}  % WildStrawberry  Approximate PANTONE 206
\def\csa{\Salmon}	  % Salmon  Approximate PANTONE 183
\def\ccp{\CarnationPink}	  % CarnationPink  Approximate PANTONE 218
\def\cmag{\Magenta}	  % Magenta  Approximate PANTONE PROCESS-MAGENTA
\def\cvr{\VioletRed}	  % VioletRed  Approximate PANTONE 219
\def\parh{\Rhodamine}	  % Rhodamine  Approximate PANTONE RHODAMINE-RED
\def\cmu{\Mulberry}	  % Mulberry  Approximate PANTONE 241
\def\parv{\RedViolet}	  % RedViolet  Approximate PANTONE 234
\def\cfu{\Fuchsia}	  % Fuchsia  Approximate PANTONE 248
\def\cla{\Lavender}	  % Lavender  Approximate PANTONE 223
\def\cth{\Thistle}	  % Thistle  Approximate PANTONE 245
\def\corc{\Orchid}	  % Orchid  Approximate PANTONE 252
\def\cdo{\DarkOrchid}	  % DarkOrchid  No PANTONE match
\def\cpu{\Purple}	  % Purple  Approximate PANTONE PURPLE
\def\cpl{\Plum}		  % Plum  VERY-Approx PANTONE 518
\def\cvi{\Violet}	  % Violet  Approximate PANTONE VIOLET
\def\clrp{\RoyalPurple}	  % RoyalPurple  Approximate PANTONE 267
\def\cbv{\BlueViolet}	  % BlueViolet  Approximate PANTONE 2755
\def\cpe{\Periwinkle}	  % Periwinkle  Approximate PANTONE 2715
\def\ccb{\CadetBlue}	  % CadetBlue  Approximate PANTONE (534+535)/2
\def\cco{\CornflowerBlue}  % CornflowerBlue  Approximate PANTONE 292
\def\cmb{\MidnightBlue}	  % MidnightBlue  Approximate PANTONE 302
\def\cnb{\NavyBlue}	  % NavyBlue  Approximate PANTONE 293
\def\crb{\RoyalBlue}	  % RoyalBlue  No PANTONE match
%\def\cbb{\Blue}		  % Blue  Approximate PANTONE BLUE-072
\def\cce{\Cerulean}	  % Cerulean  Approximate PANTONE 3005
\def\ccy{\Cyan}		  % Cyan  Approximate PANTONE PROCESS-CYAN
\def\cpb{\ProcessBlue}	  % ProcessBlue  Approximate PANTONE PROCESS-BLUE
\def\csb{\SkyBlue}	  % SkyBlue  Approximate PANTONE 2985
\def\ctu{\Turquoise}	  % Turquoise  Approximate PANTONE (312+313)/2
\def\ctb{\TealBlue}	  % TealBlue  Approximate PANTONE 3145
\def\caq{\Aquamarine}	  % Aquamarine  Approximate PANTONE 3135
\def\cbg{\BlueGreen}	  % BlueGreen  Approximate PANTONE 320
\def\cem{\Emerald}	  % Emerald  No PANTONE match
%\def\cjg{\JungleGreen}	  % JungleGreen  Approximate PANTONE 328
\def\csg{\SeaGreen}	  % SeaGreen  Approximate PANTONE 3268
\def\cgg{\Green}	  % Green  VERY-Approx PANTONE GREEN
\def\cfg{\ForestGreen}	  % ForestGreen  Approximate PANTONE 349
\def\cpg{\PineGreen}	  % PineGreen  Approximate PANTONE 323
\def\clg{\LimeGreen}	  % LimeGreen  No PANTONE match
\def\cyg{\YellowGreen}	  % YellowGreen  Approximate PANTONE 375
\def\cspg{\SpringGreen}	  % SpringGreen  Approximate PANTONE 381
\def\cog{\OliveGreen}	  % OliveGreen  Approximate PANTONE 582
\def\pars{\RawSienna}	  % RawSienna  Approximate PANTONE 154
\def\cse{\Sepia}		  % Sepia  Approximate PANTONE 161
\def\cbr{\Brown}		  % Brown  Approximate PANTONE 1615
\def\cta{\Tan}		  % Tan  No PANTONE match
\def\cgr{\Gray}		  % Gray  Approximate PANTONE COOL-GRAY-8
\def\cbl{\Black}		  % Black  Approximate PANTONE PROCESS-BLACK
\def\cwh{\White}		  % White  No PANTONE match


\loadmsbm

\input epsf

\def\ctln{\centerline}
\def\u{\underbar}
\def\ssk{\smallskip}
\def\msk{\medskip}
\def\bsk{\bigskip}
\def\hsk{\hskip.1in}
\def\hhsk{\hskip.2in}
\def\dsl{\displaystyle}
\def\hskp{\hskip1.5in}

\def\lra{$\Leftrightarrow$ }
\def\ra{\rightarrow}
\def\mpto{\logmapsto}
\def\pu{\pi_1}
\def\mpu{$\pi_1$}
\def\sig{\Sigma}
\def\msig{$\Sigma$}
\def\ep{\epsilon}
\def\sset{\subseteq}
\def\del{\partial}
\def\inv{^{-1}}
\def\wtl{\widetilde}
\def\lra{\Leftrightarrow}
\def\del{\partial}
\def\delp{\partial^\prime}
\def\delpp{\partial^{\prime\prime}}
\def\sgn{{\roman{sgn}}}
\def\wtih{\widetilde{H}}
\def\bbz{{\Bbb Z}}
\def\bbr{{\Bbb R}}



\ctln{\bf Math 856 Differential Topology}

\ssk

\ctln{August 22, 2006}

\msk


Differential topology is about introducing concepts and methods from calculus to the
realm of topological spaces. That is, we wish to use notions of differentiation
and integration in a topological setting. There are (at least) two reasons for doing
so. The first is, essentially, waste not want not; lots of people have put in a lot
of effort into developing the tools of analysis, why shouldn't topologists want to take
advantage of all of that body of work? Any tool that we can bring to bear to better
understand topological spaces helps us, well, understand topological spaces better.
The other reason is that by figuring out how to introduce analysis into topology,
we will have extended the range of applicability of these concepts. Experience has
also shown that the topological point of view can, in hindsight, provide a more natural
setting for many problems of analysis. It can also provide a natural framework for
explaining some of its results; Stokes' Theorem is perhaps the first and most well-known,
but certainly not the only such result that we may encounter in our study.
As with nearly any branches of mathematics,
once you figure out how to reconcile the immediate difficulties in introducing one
subject to another (analysis to topology, or topology to analysis?), you discover
innumerable ways in which they open up new avenues of exploration, and neither 
subject is ever the same again. The goal of this course is to explore the ways in 
which we can bring analysis and topology together, and some of the ways in which 
anaysis helps to illuminate the study of topology.

\msk

Our first task is to determine {\it which} topological
spaces we can reasonably introduce such concepts and methods to.

\ssk

A basic principle in topology is that a topological space is explored
through its continuous functions/continuous maps, both in and out of the space. 
Calculus as we usually encounter it applies to functions between Euclidean spaces
$\bbr^n$. We have derivatives, partial derivatives, integrals, and multipe integrals,
and many variations, depending upon what domain or range/codomain we choose for 
our functions. So if we want to be able to introduce the idea of a ``differentiable''
map, the simplest tack to take is to look at topological spaces which ``behave''
like Euclidean spaces. Differentiability is a local property; a (partial) derivative
of a function at a point (much less whether you have one, i.e., are
differentiable) depends only on the values of the function near that point.
Of course the notion of ``local'' is in some sense what a topology on a space
is designed to describe; open neighborhoods of a point $x$ are precisely the sets
describing which points are ``near'' $x$. So on a most basic level, the topological 
spaces most naturally to introduce calculus to are those in which the the points have
open neighborhoods which ``look like'' the spaces that we know how to do calculus on,
namely, Euclidean spaces. This motivates our first definition.

\msk

A {\it topological manifold} $M$ of dimension $n$ is a Hausdorff, second countable
space with the property that for every $x\in M$ there is an open neighborhood $U$ of
$x$ which is homeomorphic to $\bbr^n$.

\msk

The shorthand for the last proporty is that $M$ is {\it locally Euclidean}. The other
two properties, Hausdorffness and second countability, are designed, really, to make 
the topologists job easier. One occasionally encounters situations in which a locally
Euclidean space is either not Hausdorff or not second countable, but they are very
much the exception rather than the rule. And being able to assume both conditions
when someone starts tossing the term ``manifold'' around certainly make proving theorems
a lot easier. Surely this isn't the first time that you have encountered hypotheses
being imposed for the purpose of making theorems easier to prove? At any rate, any
subset of a Euclidean space is both Hausdorff and second countable in the subspace 
topology; most (all?) manifolds we will meet can (with effort) be interpreted as 
such a subspace.

\msk

Some standard examples: Euclidean space $\bbr^n$ itself. Spheres $S^n$ = the points
at unit distance in $\bbr^{n+1}$; given a point, $x\in S^n$ at least one of its
coordinates $x_i$ is non-zero. Then the set of points $y\in S^n$ whose $i$-th 
coordinate has the same sign as $x$ form a locally Euclidean neighborhood of $x$;
the homeo to the unit ball in $\bbr^n$ is given by projection onto the other
coordinates. Cartesian products of manifolds are manifolds; take the Cartesian
product of neighborhood in each as your local models. Open subsets of
manifolds are manifolds. These basic building blocks already let you build
a wide variety of examples.

\msk

Once we are confortable with the setting, manifolds, into which we will ultimately introduce
differentiability, we are left with actually {\it doing} it. It turns out that in order
to do so in a meaningful way, we have to introduce additional ``structure''; simply
having a topological manifold won't be enough.

\ssk

On the face of it, once we have a space $M$ which locally ``looks like'' Euclidean space,
we can seemingly define differentiability at a point for any function $f:M\rightarrow \bbr$.
Given a point $x\in M$, we have, by definition, a neighborhood $U$ of $X$ and a
homeomorphism $h:U\rightarrow \bbr^n$. This is, at least, enough to {\it describe}
a function for which differentiability makes sense, namely the composition
of $h^{-1}$ with the restriction of $f$ to $U$; $f\circ h^{-1}:\bbr^n\rightarrow \bbr$.
So as a first approximation, we could say that $f$ is differentiable at $x$ if 
$f\circ h^{-1}$ is differentiable at $h(x)$.

\ssk

There is only one problem with this. Surely if we are generalizing the notion of 
diferentiability to more general spaces we don't want to {\it change} what functions
$g:\bbr^n\rightarrow \bbr$ we wish to consider to be differentable. But, technically,
our first attempt at a definition just did. Consider the function $f(x)=|x|$, which 
we are all, presumably, willing to agree is not differentiable at $0$. But we can 
treat the domain $\bbr = M$ as a 1-dimensional manifold, where $U=\bbr$ and
the homeomorphism $h(x)=x^{1/3}$ serves as the proof for each $x\in M$ that $M$
is locally Euclidean. But then in testing whether or not $f$ is differentiable 
at $0$, we can just check that $f\circ h^{-1}(x) = |x^3|$ is in fact differentiable
at $0$. Which it is; the derivative is $0$.


What went wrong? Nothing. Unless you don't want to change the notion of 
differentiability... The point is, our definition of differentiability mentions
both a neighborhood $U$ of $x$ (which won't, in the end, really affect things)
and a specific homeomorphism $h:U\rightarrow \bbr^n$. The function $f$ and the point
$x$ wasn't enough to define differentiability; we also needed a {\it chart}
$(U,h)$, that is, a specific description for {\it how} to identify a neighborhood
of $x$ with $\bbr^n$. And whether or not we decide $f$ is differentiable depends
on which chart we pick. (In our example above, if we chose the identity map to define
our chart, we would have decided that $f(x)=|x|$ is not differentible at $0$.)
So, in order to {\it unambiguously} decide if a function is differentiable, we
need to restrict which pairs $(h,U)$ we are willing to allow ourselves to use 
as charts. This special collection of charts is the extra structure that we need.

\ssk

What is the basic idea? We wish to find some way to ensure that if one of two
charts $(U,h)$ and $(V,k)$ with $x\in U,V$ tells us that $f$ is differentiable 
at $x$, then the other chart {\it must} do so, as well. That is, we wish to guarantee
that $f\circ h^{-1}$ is differentiable at $h(x)$ iff $f\circ k^{-1}$ is differentiable 
at $k(x)$. And how to do this? The Chain Rule to the rescue! The thing which connects
$f\circ h^{-1}$ to $f\circ k^{-1}$ is a {\it transition map} 
$k\circ h^{-1}$; $f\circ h^{-1} = (f\circ k^{-1})\circ (k\circ h^{-1})$.
This equality holds on $h(U\cap V)$, which is the image under a homeo of an 
open subset of $U$ containing $x$, so is an open subset of $\bbr^n$ contining $h(x)$.
And if $k\circ h^{-1}$ is differentiable, then Chain Rule 
tell us that $f\circ k^{-1}$ differentiable 
at $k(x)$ implies $f\circ h^{-1}$ differentiable at $h(x)$. The reverse 
implication follows from knowing that $h\circ k^{-1}$ is differentiable.

\ssk

This leads us to our basic construction. A $C^{(k)}$ {\it atlas} ${\Cal A}$ on a 
topological manifold $M$ is a collection $(U_i,h_i)$ of charts on $M$ so that 
(1) $\bigcup U_i=M$ and (2) for every $i,j$ with $U_i\cap U_j\neq \emptyset$,
$h_i\circ h_j^{-1}:h_i(U_i\cap U_j)\rightarrow h_j(U_i\cap U_j)$ is $C^{(k)}$,\
that is, has continuous partial derivatives through order $k$. Note that
notationally, by reversing the roles of $i$ and $j$, we are also insisting
that $h_j\circ h_i^{-1}$ be $C^{(k)}$. Given a $C^{(k)}$ atlas ${\Cal A}$ on a manifold
$M$, we can then unambiguously define differentiatible functions,
or $C^{(m)}$ functions for any $m\leq k$, $f:M\rightarrow \bbr$,
by requiring that $f\circ h_i^{-1}:h(U_i)\rightarrow \bbr$ is $C^{(m)}$, for every $i$.
More generally, given atlases on manifolds $M,N$, we can define a map
$f:M\rightarrow N$ to be {\it differentiable } by requiring that 
$k_j\circ f\circ h_i^{-1}$ is differentiable
for every $k_j$ in the atlas for $N$ and $h_i$ in the atlas for $M$. 

It will be useful to introduce some notation at this point, so that we don't 
have to keep writing `` $h\circ k^{-1}$ is $C^{(k)}$ ''; we will say that $h$ and $k$ are 
`` $C^{(k)}$-related '' if $h\circ k^{-1}$ and $k\circ h^{-1}$ are both $C^{(k)}$. 

\msk

A $C^{(k)}$ atlas is enough to be able to define $C^{(m)}$ functions for $m\leq k$, but from 
a philosophical point of view, some atlases are better than others. If $f:M\rightarrow\bbr$
is a $C^{(m)}$ function and $(h,U)$ is a chart on $M$, and $V\subseteq U$ is open,
then
$f\circ (h|_V)^{-1}:h(V)\rightarrow \bbr$, as the restriction of $f\circ h^{-1}$, is $C^{(m)}$.
In fact, $h|_V$ is $C{(k)}$-related to every chart on $M$ (if we started with a $C^{(k})$
atlas), and so it doesn't hurt to add $h|_V$ to our atlas; it won't alter what functions
we will call $C^{(m)}$. But it {\it might} actually help! We re all no doubt familiar
with $\epsilon$-$\delta$ arguments where we keep shrinking $\delta$ (effectively, shrinking
the neighborhood of some point $x$) in order to make better things happen. The same will be
true here; we will want to shink the domains of charts in order to make good things happen.
It would be nice if such domains were already part of our atlas. So, we do the natural
thing; just toss them in. And while we're at it, we might as well toss in everything that
we can for free (without changing what we'll call a smooth map). This turns out to be
everything which is $C^{(k)}$-related to {\it everything} already in our atlas. This is
also the {\it largest} $C^{(k)}$ atlas which contains our original atlas. Such an
atlas is called a {\it maximal} atlas.

A {\bf $C^{(k)}$ structure} on a manifold $M$, $0\leq k\leq \infty$, is a maximal
$C^{(k)}$ atlas on $M$. $M$, together with a $C^{(k)}$ structure, will be called
a $C^{(k)}$ manifold. A $C^{(0)}$ manifold is ``just'' a manifold;  a $C^{(0)}$ structure
is a collection of homeomorphisms from the sets of an open cover of $M$ to $\bbr^n$
(that the transition maps are $C^{(0)}$, i.e., continuous, is automatic). In general
we will content ourselves to study $C^{(\infty)}$ structures on manifolds, but it 
is important to know that there are other possible choices. (When an author never
needs anything beyond a second derivative, they will often talk only about 
$C^{(2)}$ manifolds, for example. It is a fact (see, e.g., Hirsch, {\it Differential
Topology}, p.51) that for every $1\leq r\leq s\leq \infty$, a $C^{(r)}$ structure $\Cal A$
on a manifold $M$ {\it contains} a $C^{(s)}$ structure $\Cal B\subseteq \Cal A$;
that is, $\Cal A$ contains an atlas which is $C^{(s)}$-compatible. 
But we will likely not use this result.)

\msk


Our standard examples of manifolds above also provide some standard examples of smooth
manifolds; one merely needs to verify that the charts that we built are $C^{(\infty)}$-related,
so that the have an atlas, and then wave our magic wand to `build' the corresponding
maximal atlas. Restriction to an open set and Cartesian product both
preserve smoothness, so we have several general approaches to building smooth
manifolds at our fingertips.

\msk

Just as in topology we have a notion, homeomorphism, which allows us to 
treat two spaces as essentially the ``same'', there is a corresponding 
notion of same in the smooth setting. Two $C^{(k)}$ manifolds $(M,{\Cal A}),(N,{\Cal B})$
are {\it diffeomorphic} if there is a $C^{(k)}$ bijection $f:M\rightarrow N$ with
$C^{(k)}$ inverse. Just as with a homeomorphism, a diffeomorphism induces a
bijection between charts of $M$ and $N$, via 
$h:U\rightarrow \bbr^n$, for $U\subseteq M$, is taken to 
$h\circ f^{-1}:f(U)\rightarrow \bbr^n$. Because $f^{-1}$ is $C^{(k)}$, this
map is $C^{(k)}$, hence is in the (maximal) atlas ${\Cal B}$.

Just as in the ``standard'' definition of topology, the field of differential topology 
can be most succintly described as the study of the properties of smooth manifolds that
are invariant under diffeomorphism (i.e., are defined in terms of the smooth
structure). You will have learned in the homework that a given manifold can have
many different smooth structures, menaing that the atlases defining them are distinct.
(Note that the maximality requirement in fact implies that distinct atlases are
disjoint.) But in many cases these atlases can still define the `same' smooth
structure, that is, they are diffeomorphic. In particular, up to diffeomorphism,
$\bbr$,$\bbr^2$,$\bbr^3$, and $\bbr^n$ for $n\geq 5$ all have unique differentiable 
structure. It was a major breakthrough of the mid-1980's that $\bbr^4$ was discovered
to have more than one smooth structure; it in fact has uncountably many non-diffeomorphic
smooth structures. Every 2-manifold has a unique smooth structure up to diffeo;
the same is true for 3-manifolds, as well (Moise, 1950's). 
But there actually exist 4-manifolds which
posess {\it no} smooth structure. This was first discovered as a result of work of
Freedman and Donaldson (for which both received the Fields Medal in 1986). Freedman
showed that simply-connected (meaning every map of
a circle into $M$ extends to a map of a disk) topological 4-manifolds were determined
up to homeo by their `intersection pairing on second homology' (whatever
that is), and further, every unimodular symmetric bilinear pairing has a corresponding manifold.
This, by the way, implies the topological 4-dimensional Poincar\'e conjecture.
Donaldson, on the other hand, showed that for simply-connected {\it smooth} 
4-manifolds, certain intersection pairings could not arise. His work essentially
involved PDE's on 4-manifolds. In particular, the
pairing ``E8'' could not occur. So the 4-manifold ``E8'', which Freedman's work shows
exists, has no smooth structure. 

On the other hand, there are manifolds which have `too many' smooth structures, i.e., admit
multiple structures which are not diffeomorphic to one another. $\bbr^4$ is the most famous
these days, but it turns out that most spheres have this property, as well. In the late 1950's
John (`Jack') Milnor showed that $S^7$ has more than one smooth structure; it was later shown that
it has exactly 28 non-diffeomorphic structures. $S^{31}$ has more than 16 million! And in case you
think these structures are really wierd things that you are never likely to meet, the 28
structures in $S^7$ arise on the links of singularities of algebraic surfaces, specifically,
as the intersection of the solutions (in ${\Bbb C}^5$) to the equation

\centerline{$a^2 + b^2 + c^2 + d^3 + e^{6k - 1} = 0$}

\noindent with a small sphere centered at the origin, for $k=1,\ldots ,28$, gives all
28 exotic 7-spheres (source: the Wikipedia entry for `exotic sphere').
While wandering the web, I found an assertion (by Ron Stern) that `all known 4-manifolds
have infinitely many distinct smooth structures', but I am not sure how to interpret that...

\msk

So what do you do when you have a smooth structure? Start building smooth maps!
We know how to identify a smooth map $f:M^n\rightarrow N^m$; we must have
$h\circ f \circ k^{-1}:K(V)\rightarrow h(U)$ smooth for every pair of charts
on $M$ and $N$. Note that it is enough, though, to verify this for charts in 
a pair of atlases contained in the smooth structures for $M$ and $N$; the 
compatibility of every other chart in our smooth structure with those of the atlases
will guarantee smoothness of $h\circ f \circ k^{-1}$ over the entire maximal atlas.
(Note also that this does {\it not} contradict what you've shown in one of your
homework problems!) So, for example, to verify that some function $f:S^5\rightarrow S^8$
(using the standard smooth structures!) is smooth, it suffices to use an atlas 
consisting to two charts on each (the stereographic projections from the poles),
so smoothness can be verified by examining only 4 functions from $\bbr^5$ to $\bbr^8$.
Actually verifying that such functions {\it are} smooth we are going to mostly leave
to the same slightly fuzzy realm one encounters in calculus: if it is built up 
out of functions that we ``know'' are smooth, then it is smooth wherever it is 
defined. 

One thing that can help us in things is to recognize that smoothness is local.
This is just like in topology, where continuity is local; if $f:M\rightarrow N$
is a map such that for every $x\in X$ there is a chart $(h,U)$ for $M$ with $x\in U$
and a chart $(k,V)$ for $N$ with $f(x)\in V$, and $h\circ f\circ k^{-1}$ is smooth
(where it is defined), 
then $f$ is smooth. This is simply because the $h$'s and the $k$'s form atlases
for $M$ and $N$, respectively. But if you turn it around it can be thought of as a prescription
for building a smooth function, by patching together smooth functions defined on open sets;
if $\Cal O$ is an open cover of $M$, and for each $U\in {\Cal O}$ we have a
smooth map $f_U:U\rightarrow N$ such that $f_U=f_V$ on $U\cap V$ for every $U,V\in {\Cal O}$,
then the map $f:M\rightarrow N$ defined by `$f(x)=f_U(x)$ if $x\in U$' is smooth. This is
the direct analogue of the Gluing Lemma from topology. Of course, in topology, one more
often wants to glue together maps defined on {\it closed} sets, rather than open sets;
it is less messy. But in the smooth setting things aren't nearly so nice;
on $\bbr$ the function $f(x)=|x|$ can be obtained by gluing together two smooth
functions, but it is not smooth (using the standard smooth structures!) Question:
are there {\it other} smooth structures on $\bbr$ for which $f$ {\it is} smooth?

We also have many of the standard results. The composition of two smooth maps is smooth;
this is essentially just because the corresponding result is true for maps between
Euclidean spaces. The sum, difference, and product of two smooth maps $M\rightarrow\bbr$
are all smooth; again, this is basically because this is true for maps from $\bbr^n$ to $\bbr$.
And the quotient is smooth so long as the denomenator is never zero. And a map into a Cartesian 
product of smooth manifolds (using the product smooth structure) is smooth iff the map into
each factor is smooth (i.e., the composition with projection onto each factor is smooth). 
This last fact you have probably already had to use, since to decide
on the smoothness of $h\circ f \circ k^{-1}:K(V)\rightarrow h(U)\subseteq \bbr^m$, 
you had to look at each of the $m$ coordinate functions (projecting onto each coordinate 
factor $\bbr$). But some things {\it don't} work; for example the maximum $\max\{f,g\}$
of two smooth functions (mapping to $\bbr$) {\it need not} be smooth; $h(x)=|x|$, for example, can 
be defined as the maximum of the functions $f(x)=x$ and $g(x)=-x$.

\msk

There will be many situations in the material to come where we will want to assemble information
obtained locally into a single smooth map $f:M\rightarrow N$. To do so, we will introduce the
notion of a {\it partition of unity}; this is a way of writing the function $f(x)=1$ as a sum
of smooth functions. 



\vfill
\end


