

\magnification=1200
\overfullrule=0pt
%\parindent=0pt

%\nopagenumbers

\input amstex
\loadmsbm

%\voffset=-.6in
%\hoffset=-.5in
%\hsize = 7.5 true in
%\vsize=10.4 true in

%\voffset=1.4in
%\hoffset=-.5in
%\hsize = 10.2 true in
%\vsize=8 true in

\input colordvi

\def\cltr{\Red}		  % Red  VERY-Approx PANTONE RED
\def\cltb{\Blue}		  % Blue  Approximate PANTONE BLUE-072
\def\cltg{\PineGreen}	  % ForestGreen  Approximate PANTONE 349
\def\cltp{\DarkOrchid}	  % DarkOrchid  No PANTONE match
\def\clto{\Orange}	  % Orange  Approximate PANTONE ORANGE-021
\def\cltpk{\CarnationPink}	  % CarnationPink  Approximate PANTONE 218
\def\clts{\Salmon}	  % Salmon  Approximate PANTONE 183
\def\cltbb{\TealBlue}	  % TealBlue  Approximate PANTONE 3145
\def\cltrp{\RoyalPurple}	  % RoyalPurple  Approximate PANTONE 267
\def\cltp{\Purple}	  % Purple  Approximate PANTONE PURPLE

\def\cgy{\GreenYellow}     % GreenYellow  Approximate PANTONE 388
\def\cyy{\Yellow}	  % Yellow  Approximate PANTONE YELLOW
\def\cgo{\Goldenrod}	  % Goldenrod  Approximate PANTONE 109
\def\cda{\Dandelion}	  % Dandelion  Approximate PANTONE 123
\def\capr{\Apricot}	  % Apricot  Approximate PANTONE 1565
\def\cpe{\Peach}		  % Peach  Approximate PANTONE 164
\def\cme{\Melon}		  % Melon  Approximate PANTONE 177
\def\cyo{\YellowOrange}	  % YellowOrange  Approximate PANTONE 130
\def\coo{\Orange}	  % Orange  Approximate PANTONE ORANGE-021
\def\cbo{\BurntOrange}	  % BurntOrange  Approximate PANTONE 388
\def\cbs{\Bittersweet}	  % Bittersweet  Approximate PANTONE 167
%\def\creo{\RedOrange}	  % RedOrange  Approximate PANTONE 179
\def\cma{\Mahogany}	  % Mahogany  Approximate PANTONE 484
\def\cmr{\Maroon}	  % Maroon  Approximate PANTONE 201
\def\cbr{\BrickRed}	  % BrickRed  Approximate PANTONE 1805
\def\crr{\Red}		  % Red  VERY-Approx PANTONE RED
\def\cor{\OrangeRed}	  % OrangeRed  No PANTONE match
\def\paru{\RubineRed}	  % RubineRed  Approximate PANTONE RUBINE-RED
\def\cwi{\WildStrawberry}  % WildStrawberry  Approximate PANTONE 206
\def\csa{\Salmon}	  % Salmon  Approximate PANTONE 183
\def\ccp{\CarnationPink}	  % CarnationPink  Approximate PANTONE 218
\def\cmag{\Magenta}	  % Magenta  Approximate PANTONE PROCESS-MAGENTA
\def\cvr{\VioletRed}	  % VioletRed  Approximate PANTONE 219
\def\parh{\Rhodamine}	  % Rhodamine  Approximate PANTONE RHODAMINE-RED
\def\cmu{\Mulberry}	  % Mulberry  Approximate PANTONE 241
\def\parv{\RedViolet}	  % RedViolet  Approximate PANTONE 234
\def\cfu{\Fuchsia}	  % Fuchsia  Approximate PANTONE 248
\def\cla{\Lavender}	  % Lavender  Approximate PANTONE 223
\def\cth{\Thistle}	  % Thistle  Approximate PANTONE 245
\def\corc{\Orchid}	  % Orchid  Approximate PANTONE 252
\def\cdo{\DarkOrchid}	  % DarkOrchid  No PANTONE match
\def\cpu{\Purple}	  % Purple  Approximate PANTONE PURPLE
\def\cpl{\Plum}		  % Plum  VERY-Approx PANTONE 518
\def\cvi{\Violet}	  % Violet  Approximate PANTONE VIOLET
\def\clrp{\RoyalPurple}	  % RoyalPurple  Approximate PANTONE 267
\def\cbv{\BlueViolet}	  % BlueViolet  Approximate PANTONE 2755
\def\cpe{\Periwinkle}	  % Periwinkle  Approximate PANTONE 2715
\def\ccb{\CadetBlue}	  % CadetBlue  Approximate PANTONE (534+535)/2
\def\cco{\CornflowerBlue}  % CornflowerBlue  Approximate PANTONE 292
\def\cmb{\MidnightBlue}	  % MidnightBlue  Approximate PANTONE 302
\def\cnb{\NavyBlue}	  % NavyBlue  Approximate PANTONE 293
\def\crb{\RoyalBlue}	  % RoyalBlue  No PANTONE match
%\def\cbb{\Blue}		  % Blue  Approximate PANTONE BLUE-072
\def\cce{\Cerulean}	  % Cerulean  Approximate PANTONE 3005
\def\ccy{\Cyan}		  % Cyan  Approximate PANTONE PROCESS-CYAN
\def\cpb{\ProcessBlue}	  % ProcessBlue  Approximate PANTONE PROCESS-BLUE
\def\csb{\SkyBlue}	  % SkyBlue  Approximate PANTONE 2985
\def\ctu{\Turquoise}	  % Turquoise  Approximate PANTONE (312+313)/2
\def\ctb{\TealBlue}	  % TealBlue  Approximate PANTONE 3145
\def\caq{\Aquamarine}	  % Aquamarine  Approximate PANTONE 3135
\def\cbg{\BlueGreen}	  % BlueGreen  Approximate PANTONE 320
\def\cem{\Emerald}	  % Emerald  No PANTONE match
%\def\cjg{\JungleGreen}	  % JungleGreen  Approximate PANTONE 328
\def\csg{\SeaGreen}	  % SeaGreen  Approximate PANTONE 3268
\def\cgg{\Green}	  % Green  VERY-Approx PANTONE GREEN
\def\cfg{\ForestGreen}	  % ForestGreen  Approximate PANTONE 349
\def\cpg{\PineGreen}	  % PineGreen  Approximate PANTONE 323
\def\clg{\LimeGreen}	  % LimeGreen  No PANTONE match
\def\cyg{\YellowGreen}	  % YellowGreen  Approximate PANTONE 375
\def\cspg{\SpringGreen}	  % SpringGreen  Approximate PANTONE 381
\def\cog{\OliveGreen}	  % OliveGreen  Approximate PANTONE 582
\def\pars{\RawSienna}	  % RawSienna  Approximate PANTONE 154
\def\cse{\Sepia}		  % Sepia  Approximate PANTONE 161
\def\cbr{\Brown}		  % Brown  Approximate PANTONE 1615
\def\cta{\Tan}		  % Tan  No PANTONE match
\def\cgr{\Gray}		  % Gray  Approximate PANTONE COOL-GRAY-8
\def\cbl{\Black}		  % Black  Approximate PANTONE PROCESS-BLACK
\def\cwh{\White}		  % White  No PANTONE match


\loadmsbm

\input epsf

\def\ctln{\centerline}
\def\u{\underbar}
\def\ssk{\smallskip}
\def\msk{\medskip}
\def\bsk{\bigskip}
\def\hsk{\hskip.1in}
\def\hhsk{\hskip.2in}
\def\dsl{\displaystyle}
\def\hskp{\hskip1.5in}

\def\lra{$\Leftrightarrow$ }
\def\ra{\rightarrow}
\def\mpto{\logmapsto}
\def\pu{\pi_1}
\def\mpu{$\pi_1$}
\def\sig{\Sigma}
\def\msig{$\Sigma$}
\def\ep{\epsilon}
\def\sset{\subseteq}
\def\del{\partial}
\def\inv{^{-1}}
\def\wtl{\widetilde}
\def\lra{\Leftrightarrow}
\def\del{\partial}
\def\delp{\partial^\prime}
\def\delpp{\partial^{\prime\prime}}
\def\sgn{{\roman{sgn}}}
\def\wtih{\widetilde{H}}
\def\bbz{{\Bbb Z}}
\def\bbr{{\Bbb R}}



\ctln{\bf Math 856 Differential Topology}

\ssk

\ctln{Ever-expanding course notes}

\msk


Differential topology is about introducing concepts and methods from calculus to the
realm of topological spaces. That is, we wish to use notions of differentiation
and integration in a topological setting. There are (at least) two reasons for doing
so. The first is, essentially, waste not want not; lots of people have put in a lot
of effort into developing the tools of analysis, why shouldn't topologists want to take
advantage of all of that body of work? Any tool that we can bring to bear to better
understand topological spaces helps us, well, understand topological spaces better.
The other reason is that by figuring out how to introduce analysis into topology,
we will have extended the range of applicability of these concepts. Experience has
also shown that the topological point of view can, in hindsight, provide a more natural
setting for many problems of analysis. It can also provide a natural framework for
explaining some of its results; Stokes' Theorem is perhaps the first and most well-known,
but certainly not the only such result that we may encounter in our study.
As with nearly any branches of mathematics,
once you figure out how to reconcile the immediate difficulties in introducing one
subject to another (analysis to topology, or topology to analysis?), you discover
innumerable ways in which they open up new avenues of exploration, and neither 
subject is ever the same again. The goal of this course is to explore the ways in 
which we can bring analysis and topology together, and some of the ways in which 
anaysis helps to illuminate the study of topology.

\msk

Our first task is to determine {\it which} topological
spaces we can reasonably introduce such concepts and methods to.

\ssk

{\bf Manifolds:} A basic principle in topology is that a topological space is explored
through its continuous functions/continuous maps, both in and out of the space. 
Calculus as we usually encounter it applies to functions between Euclidean spaces
$\bbr^n$. We have derivatives, partial derivatives, integrals, and multipe integrals,
and many variations, depending upon what domain or range/codomain we choose for 
our functions. So if we want to be able to introduce the idea of a ``differentiable''
map, the simplest tack to take is to look at topological spaces which ``behave''
like Euclidean spaces. Differentiability is a local property; a (partial) derivative
of a function at a point (much less whether you have one, i.e., are
differentiable) depends only on the values of the function near that point.
Of course the notion of ``local'' is in some sense what a topology on a space
is designed to describe; open neighborhoods of a point $x$ are precisely the sets
describing which points are ``near'' $x$. So on a most basic level, the topological 
spaces most naturally to introduce calculus to are those in which the the points have
open neighborhoods which ``look like'' the spaces that we know how to do calculus on,
namely, Euclidean spaces. This motivates our first definition.

\msk

A {\it topological manifold} $M$ of dimension $n$ is a Hausdorff, second countable
space with the property that for every $x\in M$ there is an open neighborhood $U$ of
$x$ which is homeomorphic to $\bbr^n$.

\msk

The shorthand for the last property is that $M$ is {\it locally Euclidean}. The other
two properties, Hausdorffness and second countability, are designed, really, to make 
the topologists job easier. One occasionally encounters situations in which a locally
Euclidean space is either not Hausdorff or not second countable, but they are very
much the exception rather than the rule. And being able to assume both conditions
when someone starts tossing the term ``manifold'' around certainly make proving theorems
a lot easier. Surely this isn't the first time that you have encountered hypotheses
being imposed for the purpose of making theorems easier to prove? At any rate, any
subset of a Euclidean space is both Hausdorff and second countable in the subspace 
topology; most (all?) manifolds we will meet can (with effort) be interpreted as 
such a subspace. A manifold of dimension $n$ will be called an $n$-manifold for short.
It is not at all clear from the definition, but it is the case that the $n$ of $n$-manifold
is a homeomorphism invariant. At a given point $x\in M$, this follows from a result 
called the Invariance of Domain;
which says that if $U\subseteq \bbr^n$ is open, and $f:U\rightarrow \bbr^n$ is 
continuous and one-to-one, then $f(U)\subseteq \bbr^n$ is also open. (The cleanest proof
uses homology theory, and can usually be found in any decent algebraic topology text.)
It is a direct consequence that no open set $U\subseteq \bbr^n$ can be homeomorphic to 
an open subset of $\bbr^m$ for $m\neq n$ (and so can't be homeomorphic to $\bbr^m$, either).
This also means that in a connected manifold, every point has neighborhoods locally
homeomorphic to the same $\bbr^n$; this can be verified by the usual trick of showing that
for a fixed $n$, the points with neighborhoods homeomorphic to $\bbr^n$ is open (and 
therefore also closed!).

\msk

{\bf Examples:} Some standard examples: Euclidean space $\bbr^n$ itself. Spheres $S^n$ = the points
at unit distance in $\bbr^{n+1}$; given a point, $x\in S^n$ at least one of its
coordinates $x_i$ is non-zero. Then the set of points $y\in S^n$ whose $i$-th 
coordinate has the same sign as $x$ form a locally Euclidean neighborhood of $x$;
the homeo to the unit ball in $\bbr^n$ is given by projection onto the other
coordinates. Cartesian products of manifolds are manifolds; take the Cartesian
product of neighborhood in each as your local models. Open subsets of
manifolds are manifolds. These basic building blocks already let you build
a wide variety of examples.

\msk

Once we are confortable with the setting, manifolds, into which we will ultimately introduce
differentiability, we are left with actually {\it doing} it. It turns out that in order
to do so in a meaningful way, we have to introduce additional ``structure''; simply
having a topological manifold won't be enough.

\ssk

{\bf Smooth functions:} On the face of it, once we have a space $M$ which locally 
``looks like'' Euclidean space,
we can seemingly define differentiability at a point for any function $f:M\rightarrow \bbr$.
Given a point $x\in M$, we have, by definition, a neighborhood $U$ of $X$ and a
homeomorphism $h:U\rightarrow \bbr^n$. This is, at least, enough to {\it describe}
a function for which differentiability makes sense, namely the composition
of $h^{-1}$ with the restriction of $f$ to $U$; $f\circ h^{-1}:\bbr^n\rightarrow \bbr$.
So as a first approximation, we could say that $f$ is differentiable at $x$ if 
$f\circ h^{-1}$ is differentiable at $h(x)$.

\ssk

There is only one problem with this. Surely if we are generalizing the notion of 
diferentiability to more general spaces we don't want to {\it change} what functions
$g:\bbr^n\rightarrow \bbr$ we wish to consider to be differentable. But, technically,
our first attempt at a definition just did. Consider the function $f(x)=|x|$, which 
we are all, presumably, willing to agree is not differentiable at $0$. But we can 
treat the domain $\bbr = M$ as a 1-dimensional manifold, where $U=\bbr$ and
the homeomorphism $h(x)=x^{1/3}$ serves as the proof for each $x\in M$ that $M$
is locally Euclidean. But then in testing whether or not $f$ is differentiable 
at $0$, we can just check that $f\circ h^{-1}(x) = |x^3|$ is in fact differentiable
at $0$. Which it is; the derivative is $0$.


{\bf Charts:} What went wrong? Nothing. Unless you don't want to change the notion of 
differentiability... The point is, our definition of differentiability mentions
both a neighborhood $U$ of $x$ (which won't, in the end, really affect things)
and a specific homeomorphism $h:U\rightarrow \bbr^n$. The function $f$ and the point
$x$ wasn't enough to define differentiability; we also needed a {\it chart}
$(U,h)$, that is, a specific description for {\it how} to identify a neighborhood
of $x$ with $\bbr^n$. And whether or not we decide $f$ is differentiable depends
on which chart we pick. (In our example above, if we chose the identity map to define
our chart, we would have decided that $f(x)=|x|$ is not differentible at $0$.)
So, in order to {\it unambiguously} decide if a function is differentiable, we
need to restrict which pairs $(h,U)$ we are willing to allow ourselves to use 
as charts. This special collection of charts is the extra structure that we need.

\ssk

What is the basic idea? We wish to find some way to ensure that if one of two
charts $(U,h)$ and $(V,k)$ with $x\in U,V$ tells us that $f$ is differentiable 
at $x$, then the other chart {\it must} do so, as well. That is, we wish to guarantee
that $f\circ h^{-1}$ is differentiable at $h(x)$ iff $f\circ k^{-1}$ is differentiable 
at $k(x)$. And how to do this? The Chain Rule to the rescue! The thing which connects
$f\circ h^{-1}$ to $f\circ k^{-1}$ is a {\it transition map} 
$k\circ h^{-1}$; $f\circ h^{-1} = (f\circ k^{-1})\circ (k\circ h^{-1})$.
This equality holds on $h(U\cap V)$, which is the image under a homeo of an 
open subset of $U$ containing $x$, so is an open subset of $\bbr^n$ contining $h(x)$.
And if $k\circ h^{-1}$ is differentiable, then Chain Rule 
tell us that $f\circ k^{-1}$ differentiable 
at $k(x)$ implies $f\circ h^{-1}$ differentiable at $h(x)$. The reverse 
implication follows from knowing that $h\circ k^{-1}$ is differentiable.

\ssk

{\bf Atlases:} This leads us to our basic construction. A $C^{(k)}$ {\it atlas} ${\Cal A}$ on a 
topological manifold $M$ is a collection $(U_i,h_i)$ of charts on $M$ so that 
(1) $\bigcup U_i=M$ and (2) for every $i,j$ with $U_i\cap U_j\neq \emptyset$,
$h_i\circ h_j^{-1}:h_i(U_i\cap U_j)\rightarrow h_j(U_i\cap U_j)$ is $C^{(k)}$,\
that is, has continuous partial derivatives through order $k$. Note that
notationally, by reversing the roles of $i$ and $j$, we are also insisting
that $h_j\circ h_i^{-1}$ be $C^{(k)}$. Given a $C^{(k)}$ atlas ${\Cal A}$ on a manifold
$M$, we can then unambiguously define differentiatible functions,
or $C^{(m)}$ functions for any $m\leq k$, $f:M\rightarrow \bbr$,
by requiring that $f\circ h_i^{-1}:h(U_i)\rightarrow \bbr$ is $C^{(m)}$, for every $i$.
More generally, given atlases on manifolds $M,N$, we can define a map
$f:M\rightarrow N$ to be {\it differentiable } by requiring that 
$k_j\circ f\circ h_i^{-1}$ is differentiable
for every $k_j$ in the atlas for $N$ and $h_i$ in the atlas for $M$. 

It will be useful to introduce some notation at this point, so that we don't 
have to keep writing `` $h\circ k^{-1}$ is $C^{(k)}$ ''; we will say that $h$ and $k$ are 
`` $C^{(k)}$-related '' if $h\circ k^{-1}$ and $k\circ h^{-1}$ are both $C^{(k)}$. 

\msk

{\bf Smooth structures:} A $C^{(k)}$ atlas is enough to be able to define $C^{(m)}$ 
functions for $m\leq k$, but from 
a philosophical ( and functional) point of view, some atlases are better than others. 
If $f:M\rightarrow\bbr$
is a $C^{(m)}$ function and $(h,U)$ is a chart on $M$, and $V\subseteq U$ is open,
then
$f\circ (h|_V)^{-1}:h(V)\rightarrow \bbr$, as the restriction of $f\circ h^{-1}$, is $C^{(m)}$.
In fact, $h|_V$ is $C{(k)}$-related to every chart on $M$ (if we started with a $C^{(k})$
atlas), and so it doesn't hurt to add $h|_V$ to our atlas; it won't alter what functions
we will call $C^{(m)}$. But it {\it might} actually help! We re all no doubt familiar
with $\epsilon$-$\delta$ arguments where we keep shrinking $\delta$ (effectively, shrinking
the neighborhood of some point $x$) in order to make better things happen. The same will be
true here; we will want to shink the domains of charts in order to make good things happen.
It would be nice if such domains were already part of our atlas. So, we do the natural
thing; just toss them in. And while we're at it, we might as well toss in everything that
we can for free (without changing what we'll call a smooth map). This turns out to be
everything which is $C^{(k)}$-related to {\it everything} already in our atlas. This is
also the {\it largest} $C^{(k)}$ atlas which contains our original atlas. Such an
atlas is called a {\it maximal} atlas.

A {\bf $C^{(k)}$ structure} on a manifold $M$, $0\leq k\leq \infty$, is a maximal
$C^{(k)}$ atlas on $M$. $M$, together with a $C^{(k)}$ structure, will be called
a $C^{(k)}$ manifold. A $C^{(0)}$ manifold is ``just'' a manifold;  a $C^{(0)}$ structure
is a collection of homeomorphisms from the sets of an open cover of $M$ to $\bbr^n$
(that the transition maps are $C^{(0)}$, i.e., continuous, is automatic). In general
we will content ourselves to study $C^{(\infty)}$ structures on manifolds, but it 
is important to know that there are other possible choices. (When an author never
needs anything beyond a second derivative, they will often talk only about 
$C^{(2)}$ manifolds, for example. It is a fact (see, e.g., Hirsch, {\it Differential
Topology}, p.51) that for every $1\leq r\leq s\leq \infty$, a $C^{(r)}$ structure $\Cal A$
on a manifold $M$ {\it contains} a $C^{(s)}$ structure $\Cal B\subseteq \Cal A$;
that is, $\Cal A$ contains an atlas which is $C^{(s)}$-compatible. 
But we will likely not use this result.)

\msk


{\bf Examples:} Our standard examples of manifolds above also provide some standard examples of smooth
manifolds; one merely needs to verify that the charts that we built are $C^{(\infty)}$-related,
so that the have an atlas, and then wave our magic wand to `build' the corresponding
maximal atlas. Restriction to an open set and Cartesian product both
preserve smoothness, so we have several general approaches to building smooth
manifolds at our fingertips.

\msk

{\bf Diffeomorphisms:} Just as in topology we have a notion, homeomorphism, which allows us to 
treat two spaces as essentially the ``same'', there is a corresponding 
notion of same in the smooth setting. Two $C^{(k)}$ manifolds $(M,{\Cal A}),(N,{\Cal B})$
are {\it diffeomorphic} if there is a $C^{(k)}$ bijection $f:M\rightarrow N$ with
$C^{(k)}$ inverse. Just as with a homeomorphism, a diffeomorphism induces a
bijection between charts of $M$ and $N$, via 
$h:U\rightarrow \bbr^n$, for $U\subseteq M$, is taken to 
$h\circ f^{-1}:f(U)\rightarrow \bbr^n$. Because $f^{-1}$ is $C^{(k)}$, this
map is $C^{(k)}$, hence is in the (maximal) atlas ${\Cal B}$.


{\bf Some history:} Just as in the ``standard'' definition of topology, the field of 
differential topology 
can be most succintly described as the study of the properties of smooth manifolds that
are invariant under diffeomorphism (i.e., are defined in terms of the smooth
structure). You will have learned in the homework that a given manifold can have
many different smooth structures, meaning that the atlases defining them are distinct.
But in many cases these atlases can still define the `same' smooth
structure, that is, they are diffeomorphic. In particular, up to diffeomorphism,
$\bbr$,$\bbr^2$,$\bbr^3$, and $\bbr^n$ for $n\geq 5$ all have unique differentiable 
structure. (Except for $\bbr$, these are all fairly difficult results to establish!)
It was a major breakthrough of the mid-1980's that $\bbr^4$ was discovered
to have more than one smooth structure; it in fact has uncountably many non-diffeomorphic
smooth structures. In fact, there are uncountably many open subsets of standard
$\bbr^4$, each homeomorphic to $\bbr^4$, but (using the smooth
structures it inherits from standard $\bbr^4$) {\it none}
of them diffeomorphic to one another! If this isn't wierd enough
for you, consider that, since $\bbr^5$ has only one smooth structure,
up to diffeo, if you take these `exotic' $\bbr^4$'s and cross them with
$\bbr$ (with the standard structure), you obtain smooth manifolds, {\it all}
of which are diffeomorphic to standard $\bbr^5$ (and hence to one another)!

Every 2-manifold has a unique smooth structure up to diffeo (Rado, 1920s?);
the same is true for 3-manifolds, as well (Moise, 1950's). 
But there actually exist 4-manifolds which
posess {\it no} smooth structure. This was first discovered as a result of work of
Freedman and Donaldson (for which both received the Fields Medal in 1986). Freedman
showed that simply-connected (meaning every map of
a circle into $M$ extends to a map of a disk) topological 4-manifolds were determined
up to homeo by their `intersection pairing on second homology' (whatever
that is), and further, every unimodular symmetric bilinear pairing has a corresponding manifold.
This, by the way, implies the topological 4-dimensional Poincar\'e conjecture.
Donaldson, on the other hand, showed that for simply-connected {\it smooth} 
4-manifolds, certain intersection pairings could not arise (if the pairing is
positive definite, then it is diagonalizable). His work essentially
involved PDE's on 4-manifolds. In particular, the
pairing known as ``E8'' could not occur. So the 4-manifold ``E8'', which Freedman's work shows
exists, has no smooth structure. Similar examples can be found for all 
higher dimensions, as well.

On the other hand, there are manifolds which have `too many' smooth structures, i.e., admit
multiple structures which are not diffeomorphic to one another. $\bbr^4$ is the most famous example
these days, but it turns out that most spheres have this property, as well. In the late 1950's
John (`Jack') Milnor showed that $S^7$ has more than one smooth structure; it was later shown that
it has exactly 28 non-diffeomorphic structures. $S^{31}$ has more than 16 million! And in case you
think these structures are really wierd things that you are never likely to meet, the 28
structures in $S^7$ arise on the links of singularities of algebraic surfaces. Specifically,
the intersection of the solutions (in ${\Bbb C}^5$) to the equation

\centerline{$a^2 + b^2 + c^2 + d^3 + e^{6k - 1} = 0$}

\noindent with a small sphere centered at the origin, for $k=1,\ldots ,28$, gives all
28 exotic 7-spheres (source: the Wikipedia entry for `exotic sphere').
While wandering the web, I found an assertion (by Ron Stern) that `all known 4-manifolds
have infinitely many distinct smooth structures', but I am not sure how to interpret that...
A result of Kirby and Siebenmann from the 1960's says that, except possibly in dimension 4
(unless Ron Stern's statement deals with it?), every smooth $n$-manifold $M^n$ has the same number
of non-diffeomorphic smooth structures as $S^n$ does. So {\it every} smooth 7-manifold has
28 distinct smooth structures, up to diffeomorphism...

\ssk

Aside from being interesting and suprising facts, proved by really bright people, these kinds 
of results can have useful consequences. Since all 2- and 3-manifolds $M$ have unique smooth 
structures, when somebody hands us such a manifold $M$, we can {\it assume} they have also 
handed us a smooth structure (even if they didn't); it comes for free. Even more, we don't
need to worry about which smooth structure we might have picked; if you and I happen to have
picked different ones to work with, any result you might find with yours can be translated
into a result about mine, because we know that there is a diffeomorphism between them (we just
might not know what it {\it is}...). And if we are trying to set up some problem or do some
computation, we can choose the most convenient coordinate system (i.e., atlas) that we want
(tailored to the functions involved, perhaps), to carry out our work; we know, again, that we
can translate our results into any other coordinate system, since they all describe the 
`same' smooth structure. The fact that this isn't true in dimensions 4 and above (except, 
technically, that the Kirby-Siebenmann results implies, for example, that all smooth 12-manifolds 
have unique smooth structures?) makes life in higher dimensions much more interesting, in 
this regard! 

\msk

{\bf Manifolds with boundary:} Our definitions so far do not allow for things like the unit
interval $I=[0,1]$ to be a manifold, much less a smooth one. And, semantically at least,
they never will be; they will be {\it manifolds with boundary}. A manifold with boundary is
a Hausdorff, second countable space in which every point has a neighborhood homeomorphic
to {\it either} $\bbr^n$ {\it or} the upper half space ${\Bbb H}^n = \{(x_1,\ldots ,x_n)\in\bbr^n : 
x_n\geq 0\}$. Points which do not have a neighborhood homeomorphic to $\bbr^n$ are called
boundary points; the union of them is the boundary of $M$, denoted $\del M$. It is a 
consequence of Invariance of Domain that if $x\in M$ has neighborhood homeomorphic to 
${\Bbb H}^n$ and the image of $x$ has last coordinate $0$, then $x\in \del M$; that is,
{\it every} chart is homeomorphic to ${\Bbb H}^n$, not $\bbr^n$, and always sends $x$ to the 
boundary. Putting a smooth structure on a manifold with boundary involves some extra
requirements as well, motivated, mostly, by what analysts have found to be reasonable
in dealing with regions with boundary in Euclidean space. That is, a map ${\Bbb H}^n\rightarrow \bbr$
is $C^{(k)}$ if it can be {\it extended} to a $C^{(k)}$ map 
$\bbr^{n-1}\times (-\epsilon,\infty)\rightarrow \bbr$ on an open neighborhood of ${\Bbb H}^n$.
Then we can adopt the exact same definition of an atlas and smooth structure, using
this augmented definition of smooth to test the compatibility of the charts; that is, for
any point $x$ on the boundary, the maps $h\circ k^{-1}$ and $k\circ h^{-1}$ must extend to smooth 
maps on open neighborhoods of $k(x)$ and $h(x)$ respectively.

\msk

{\bf Smooth maps:} So what do you do when you have a smooth structure? 
Start building smooth maps!
We know how to identify a smooth map $f:M^n\rightarrow N^m$; we must have
$h\circ f \circ k^{-1}:K(V)\rightarrow h(U)$ smooth for every pair of charts
on $M$ and $N$. Note that it is enough, though, to verify this for charts in 
a pair of atlases contained in the smooth structures for $M$ and $N$; the 
compatibility of every other chart in our smooth structure with those of the atlases
will guarantee smoothness of $h\circ f \circ k^{-1}$ over the entire maximal atlas.
(Note also that this does {\it not} contradict what you've shown in one of your
homework problems!) So, for example, to verify that some function $f:S^5\rightarrow S^8$
(using the standard smooth structures!) is smooth, it suffices to use an atlas 
consisting to two charts on each (the stereographic projections from the poles),
so smoothness can be verified by examining only 4 functions from $\bbr^5$ to $\bbr^8$.
Actually verifying that such functions {\it are} smooth we are going to mostly leave
to the same slightly fuzzy realm one encounters in calculus: if it is built up 
out of functions that we ``know'' are smooth, then it is smooth wherever it is 
defined. 

One thing that can help us in things is to recognize that smoothness is local.
This is just like in topology, where continuity is local; if $f:M\rightarrow N$
is a map such that for every $x\in X$ there is a chart $(h,U)$ for $M$ with $x\in U$
and a chart $(k,V)$ for $N$ with $f(x)\in V$, and $h\circ f\circ k^{-1}$ is smooth
(where it is defined), 
then $f$ is smooth. This is simply because the $h$'s and the $k$'s form atlases
for $M$ and $N$, respectively. But if you turn it around it can be thought of as a prescription
for building a smooth function, by patching together smooth functions defined on open sets;
if $\Cal O$ is an open cover of $M$, and for each $U\in {\Cal O}$ we have a
smooth map $f_U:U\rightarrow N$ such that $f_U=f_V$ on $U\cap V$ for every $U,V\in {\Cal O}$,
then the map $f:M\rightarrow N$ defined by `$f(x)=f_U(x)$ if $x\in U$' is smooth. This is
the direct analogue of the Gluing Lemma from topology. Of course, in topology, one more
often wants to glue together maps defined on {\it closed} sets, rather than open sets;
it is less messy. But in the smooth setting things aren't nearly so nice;
on $\bbr$ the function $f(x)=|x|$ can be obtained by gluing together two smooth
functions, but it is not smooth (using the standard smooth structures!) Question:
are there {\it other} smooth structures on $\bbr$ for which $f$ {\it is} smooth?

{\bf Basic properties:} We also have many of the standard results. The composition of two smooth maps is smooth;
this is essentially just because the corresponding result is true for maps between
Euclidean spaces. The sum, difference, and product of two smooth maps $M\rightarrow\bbr$
are all smooth; again, this is basically because this is true for maps from $\bbr^n$ to $\bbr$.
And the quotient is smooth so long as the denomenator is never zero. And a map into a Cartesian 
product of smooth manifolds (using the product smooth structure) is smooth iff the map into
each factor is smooth (i.e., the composition with projection onto each factor is smooth). 
This last fact you have probably already had to use, since to decide
on the smoothness of $h\circ f \circ k^{-1}:K(V)\rightarrow h(U)\subseteq \bbr^m$, 
you had to look at each of the $m$ coordinate functions (projecting onto each coordinate 
factor $\bbr$). But some things {\it don't} work; for example the maximum $\max\{f,g\}$
of two smooth functions (mapping to $\bbr$) {\it need not} be smooth; $h(x)=|x|$, for example, can 
be defined as the maximum of the functions $f(x)=x$ and $g(x)=-x$.

\ssk

Technically, partial derivatives are taken with respect to coordinate charts, not variables.
but if $h:U\rightarrow \bbr^n$ is a chart, and $f:M\rightarrow \bbr$ is a map, then
if we adopt the notation that $h(z)=(x^1(z),\ldots ,x^n(z))\in \bbr^n$, we will adopt 
the notation that 

\ctln{$\displaystyle {{\partial}\over{\partial x^i}}(f)(z) = 
{{\partial}\over{\partial x_i}}(f\circ h^{-1})(h(z))$}

\noindent That is, we formally are taking the partial derivative of $f$ with respect to the
coordinate functions of the chart $h$. Therefore, a function does 
not really have a `value' of a partial derivative at a point; it has such a value 
{\it with respect to} a given coordinate chart. If we change charts around a point, to
$(k,V)$, $k=(y^1,\ldots ,y^n)$ the Chain Rule tells us how to relate the two sets of 
partial derivatives; it works out to the familiar

\ctln{$\displaystyle {{\partial f}\over{\partial y^i}} = 
\sum_i{{\partial f}\over{\partial x^i}}{{\partial x^i}\over{\partial y^j}}$}

\noindent (once you chase it through the notation). Once thing that this formula
immediately tells us is that if $\partial f/\partial x^i = 0$ at $z$ for every $i$,
using one coordinate chart, 
then $\partial f/\partial y^i = 0$ at $z$ for every $i$, for any other chart;
a linear combination of $0$'s is still $0$. So the notion of a {\it critical point},
as in mulrivariate calculus, is still well-defined in our more general setting. And
the usual proof that local max's and min's are critical points carries over as
well (apply any proof you've ever seen to $f\circ h^{-1}$ for any chart around
your local extremum). So we can, for example, formulate and solve max-min problems
on smooth manifolds!

\msk

{\bf Partitions of unity:} There will be many situations in the material to come where we will want to assemble information
obtained locally into a single smooth map $f:M\rightarrow N$. To do so, we will introduce the
notion of a {\it partition of unity}; this is a way of writing the function $f(x)=1$ as a sum
of smooth functions $1=\sum g_i$. Of course, $f(x)=1$ works; constant functions are smooth. 
But we will want each
summand function to be `local'; that is, zero outside of a small open set (think: the domain of a chart).
For example, in order to define the integral of a function $f:M^n\rightarrow \bbr$, the idea will be to 
define it first over the domain of a chart $(h,U)$ (as the integral in $\bbr^n$ of the function
$f\circ h^{-1}$, essentially), and then define the integral of $f$ as the sum of the integrals
of $f\cdot g_i$ (which are `really' integrals over $\bbr^n$), since $f = f\cdot 1 \sum f\cdot g_i$.
Building such a collection of functions will take a bit more work, and will be the first time we 
really invoke the Hausdorffness and
second countability conditions built into our original definition. 
Put slightly differently, the idea is that 
we want to make the
{\it support} of each function, supp$(g_i)$ = cl($\{x\in M : g_i(x)\neq 0\}$), to be small.


Given an open cover ${\Cal O}$ of a space $M$, a {\it locally-finite refinement} of ${\Cal O}$
is an open cover ${\Cal P}$ of $M$ so that for every $P\in{\Cal P}$ there is an $O\in{\Cal O}$
so that $P\subseteq O$ (that's the refinement part), and for every $x\in M$ there
is an open neighborhood $x\in U$ of $x$ so that $U\cap P=\emptyset$ for all but finitely many
$P\in {\Cal P}$ (that's the locally finite part). A space $M$ is {\it paracompact} if 
every open cover ${\Cal O}$ has a locally finite refinement ${\Cal P}$ such that $\overline{P}$ is compact
for every $P\in {\Cal P}$. Such $P$ are called {\it precompact}.

The main result we are aiming at is:

\ssk

If $M$ is a smooth manifold and ${\Cal O}=\{u_\alpha : \alpha\in I\}$ is an open cover of $M$, then there
is a partition of unity $\{g_i : i\in I\}$ so that supp$(g_i)\subseteq U_i$, and for every $x\in M$,
there is a neighborhood $V$ of $x$ so that only finitely many supp$(g_i)$ intersect $V$.

\ssk

The statement that supp$(g_i)\subseteq U_i$ is referred to as having a partition of unity
{\it subordinate} to the open cover.
The last property of the statement allows us to make sense of adding the functions together; we don't
need the convergence of some infinite series, since around every point all but finitely many of the
functions take the value zero. We say that the supports of the functions are {\it locally finite},
for short.

The proof of the existence of partitions of unity essentially comes in two parts. The first part
asserts that any open cover ${\Cal O}$ of $M$ has a {\it locally finite refinement}, that is, a 
locally finite cover ${\Cal O}^\prime$ so that for every $O\in {\Cal O}$ there is an 
$O^\prime\in{\Cal O}^\prime$ with $O^\prime\subseteq O$ (i.e., the refinement has ``smaller'' sets).
This property is known as {\it paracompactness}. In particular, we will show that the refinement
can be built out of the domains for coordinate charts of $M$. The second part uses the 
refinement by coordinate charts to build the partition of unity, by building a collection of 
smooth ``bump'' functions supported on each coordinate chart.

\ssk

{\bf Paracompactness:} To prove paracompactness, start with an open cover $\Cal O$ of $M$, and a countable basis $\Cal B$
for the topology on $M$. First we need a locally 
finite open cover to help guide our steps. Every point $x\in M$ is in the domain of some 
chart $h:U\rightarrow \bbr^n$ (with, we can arrange, image containing $B(h(x),2)$). The 
open sets $U_x=h^{-1}(B(h(x),1))$ cover $M$, and have compact closure; and for each there is 
a basis element $B_x$ with $x\in B_x\subseteq U_x$. Since $\Cal B$ is countable, 
there are countably many $x_i$ so that the $B_{x_i}$, and therefore the $U_{x_i}$,
cover $M$. Call these sets $U_i, I\in {\Bbb N}$, and let $C_i=\overline{U_i}$. By construction,
$C_i$ is compact, so for any finite set $I\in{\Bbb N}$, $C_I=\bigcup_I C_i$ is compact. 
Let $E_I$ denote $\bigcup_I U_i$.
Set $I_1=\{1\}$, then since the $U_i$ cover $M$ and therefore $C_1$, there are finitely
many $i$ with union $J_2$ so that $C_{I_1}\subseteq \bigcup_{J_2}U_i$, and set $I_2=J_2\cup\{2\}$.
Inductively, we continue to build finite sets $J_n$ so that $C_{I_{n-1}}\subseteq \bigcup_{J_n}U_i$
and $I_n=J_n\cup\{n\}$. Then $M=\bigcup_n C_{I_n} = \bigcup_n E_{I_n}$, $C_{I_{n-1}}\subseteq E_{I_n}$
$C_{I_n}$ is compact and $E_{I_n}$ is open. Then the sets 
$K_n=E_{I_n}\setminus C_{I_{n-2}}$ are open, have union $M$, have compact closure (contained in $C_{I_n}$),
and are locally finite. To demonstrate
the last assertion, for any $x\in M$, 
$x\in E_{I_n}$ for some $n$; assume $n$ is minimal. Then $x\in U_j$ for some $j\in I_n$, and since
$U_j\subseteq E_{I_n}\subseteq E_{I_k}$ for all $k\geq n$, $U_j\cap K_r=\emptyset$ for all $r\geq n+2$.
In fact, since $\overline{K_r}\subseteq C_{I_n}\setminus E_{I_{n-1}}$, only $K_{r-1}, K_r$, and $K_{r+1}$
meet $\overline{K_r}$.

\ssk

Now start again. We have our open cover ${\Cal O}$, and the locally finite cover $\{K_n\}$ by 
precompact open sets. For every point $x\in M$ we can, by local finiteness, find an open neighborhood $W_x$ so that
if $x\in K_n$ then $W_x\subseteq K_n$; start with a neighborhood meeting only fnitely many of them, and then
intersect it with each of them as well. Taking a further intersection with an element of $\Cal O$ containing
$x$, we can also assume that $W_x\subseteq U\in {\Cal O}$. Then we may assume by intersecting with 
the domain of a chart
that there is a chart $h:W_x\rightarrow \bbr^n$ sending $W_x$ to an open neghborhood of $h(x)$. Rescaling
$h$ on the codomain side and shrinking the domain, we can asuume that $h(W_x)=B(h(x),2)$, and 
so $V_x=h^{-1}(B(h(x),1))$ is a neighborhood of $x$ with compact closure, contained in an element of
$\Cal O$, and contained in every $K_n$ that it meets. $W_x$ satisfies all of these properties
except possibly the compact closure.

Now for each $n$, the sets $V_x$ with $x\in \overline{K_n}$ form an open cover of the compact set $\overline{K_n}$,
so they have a finite subcover ${\Cal P}_n=\{V_{x_1,k]n},\ldots ,V_{x_{m_n},kn}\}$; we assume that each
has non-empty intersection with $\overline{K_n}$ (otherwise we throw it away). 
Set ${\Cal R}_n=\{W_{x_1,n},\ldots ,W{x_{m_n},n}\}$. The collectionS
$\Cal P=\bigcup_n{\Cal P}_n$  and ${\Cal R}=\bigcup_n{\Cal R}_n$ both form open covers of $M$, are
refinements of $\Cal O$, and, we now show, are locally finite. It is enough to show this for $\Cal R$,
since these sets are larger. We show that, in fact, each set in $\Cal R$ meets only
finitely many others, so each demonstrates local finiteness for every point in it.
But each $W=W_{x_i,n}$ intersects, and is therefore contained in, some $K_m$. 
It therefore meets only $\overline{K_{m-1}}, \overline{K_m}$, or $\overline{K_{m+1}}$. 
Any other element $W^\prime$ of $\Cal R$ meeting $W$ meets, and therefore is contained in, one of these three
sets. So the only $\overline{K_r}$ it could meet would be one of $\overline{K_{m-2}}$
through $\overline{K_{m+2}}$. $W^\prime$ is therefore a member of one of ${\Cal R}_{m-2}$ through
${\Cal R}_{m+2}$; it doesn't meet any of the other sets $\overline{K_r}$. Therefore, it is
one of the finitely many elements of these five sets. So $W$ meets only finitely many of the other
elements of $\Cal R$. 

\msk

{\bf The partitioning of 1:} Now that we know how to build a locally finite cover by (images of) charts $(h_i,h_i^{-1}(B(x_i,2))$
for which $h_i^{-1}(B(x_i,1))$ also cover and $h_i^{-1}(\overline{B(x_i,1)})$ is compact, we turn 
to bulding a partition of unity with supported on these sets. We start with the fact
that the function

\ctln{$f(x)= e^{-1/x} \text{ if } x>0$ ; $=0 \text{ if } x\leq 0$}

\noindent is $C^\infty$. This follows from the fact that the $n$-th derivative of
$e^{-1/x}$ is $f_n(x)=p_n(x)e^{-1/x}/x^{2n}$ for some polynomial $p_n(x)$, which can be established
by induction on $n$. The function has (one-sided) limit $0$ at $x=0$, which can be established by
repeated use of L`H\^opital's Rule (to show that $e^{-1/x}/x^{2n}$ has limit 0). 
Together these imply that $f$ has continuous derivatives of'all orders. Note that since
$-1/x<0$ for $x>0$, $0\leq f(x) <1$ for all $x$. Now define 
$g(x)=f(2-x)/(f(2-x)+f(x-1))$; this function is smooth, since the denomentaor is always positive
(one term is 0 only for $x\geq 2$ and the other is zero only for $x\leq 1$), takes values
between 0 and 1, is one precisely when $f(x-1)=0$, i.e., $x\leq 1$, and is 0 precisely when 
$f(2-x)=0$, i.e., $x\geq 2$. Then the function $G:\bbr^n\rightarrow \bbr$ defined by
$G(y)=g(||y-x_0||^2)$ is smooth (it's the composition of smooth functions), is
1 on $B(x_0,1)$ and has support contained in $B(x_0,2)$. Taking our charts $h_i$
built above, the function $h_i\circ G$ extends (by taking the value 0) to a smooth 
function $f_i:M\rightarrow \bbr$ which is 1 on 
$h_i^{-1}(B(x_i\,1))$ and has support in $h_i^{-1}(B(x_i\,2))$. Since every
point has a neighborhood which lies in only finitely many of the $h_i^{-1}(B(x_i\,2))$,
the sum $F=\sum f_i$ is locally a finite sum and so is a smooth function on $M$.
Since the $h_i^{-1}(B(x_i\,1))$ cover $M$, it is everywhere non-zero. So each 
of the functions $F_i=f_i/F$ is smooth, their supports = the supports of the $f_i$
are locally finite, and their sum (which is locally a finite sum) is $1$. that is,
they form a smooth partition of unity subordinate to the cover
$h_i^{-1}(B(x_i\,2))$, which is a refinement of our original cover $\Cal O$. So they
form a smooth partition of unity subordinqate to $\Cal O$.

\msk

{\bf Density of smooth functions:} Now that we have a partition of unity, what do we do with it?
One immediate application of partitions of unity is: for every continuous function $f:M\rightarrow \bbr$
and $\epsilon > 0$, there is a smooth function $g:M\rightarrow \bbr$ with $|f(x)-g(x)|< \epsilon$ 
for all $x\in M$. The proof consists of looking at the open cover $\{f^{-1}(f(x)-\epsilon,f(x)+\epsilon)\}$,
and choose a partition of unity $g_i$ subordinate to this cover. For each $g_i$ pick a point $x_i$ with 
supp$(g_i)\subseteq f^{-1}(f(x_i)-\epsilon,f(x_i)+\epsilon)$. Then the function $g(y)=\sum f(x_i)g_i(y)$ is smooth
(since the $f(x_i)$ are constants, so this is a locally finite sum of smooth functions), and 
$|f(y)-g(y)|=|\sum g_i(y)(f(y)-f(x_i))|\leq \sum g_i(y)|f(y)-f(x_i)| < \sum g_i(y) \epsilon = \epsilon$,
since either $g_i(y)=0$, or $g_i(y)>0$, so $y\in f^{-1}(f(x_i)-\epsilon,f(x_i)+\epsilon)$, so 
$f(y)\in (f(x_i)-\epsilon,f(x_i)+\epsilon)$, so $|f(y)-f(x_i)|<\epsilon$.

\ssk

Partitions of unity can also be used to build {\it bump functions};
given a closed set $C$ of $M^n$ and an open set $U$ with $C\subseteq U$, we can build a smooth
function $f:M\rightarrow \bbr$ which is $1$ on $C$ and has support contained in $U$. The idea is simply to take the 
open cover $\{U,M\setminus C\}$ and build a smooth partition of unity $\psi_i,\phi_j$ subordinate to it.
with supp$(\psi_i)\subseteq U$ and supp$(\phi_j)\subseteq   M\setminus C$ for every $i$ and $j$.
Then set $\psi=\sum_i\psi_i$ and $\phi=\sum_j\phi_j$; by local finiteness, both are smooth functions.
Since $\psi(x)+\phi(x)=1$ for all $x$ and $\phi(x)=0$ outside of $M\setminus C$ (since all summands are), i.e., 
for $x\in C$,
we have $\psi(x)=1$ for $x\in C$; since $\psi(x)=0$ outside of $U$, we have
$\psi(x)=0$ for $x\notin U$, as desired. (This last statement does not quite say that supp$(\psi)\subseteq U$;
but this can be remedied by using a slightly smaller open set $V$ in place of $U$, with 
$C\subseteq V\subseteq \overline{V}\subseteq U$, which exists by the normality of $M$.)

\msk

{\bf Embedding in $\bbr^n$:} Another immediate application (of our proof, really) is that if $M^n$ is a smooth manifold, then 
there is a smooth
embedding (that is, a topological embedding that is a smooth map) of $M$ into 
$\bbr^n$ for some $N$. Right now we will prove this for compact
$M$; later we will show it for all $M$. To build the embedding, cover $M$ by finitely many 
coordinate charts $(h_i,U_i)$,
$i=1,\ldots ,k$, so that $B(x_i,2)\subseteq h_i(U_i)$ and the $h_i^{-1}(B(x_i,1))$ cover $M$.
Then taking a smooth bump function $g_i$ that is 1 on $h_i^{-1}(B(x_i,1))$ and supported on $U_i$, we can build 
the smooth functions $f_i=g_i\cdot h_i:M\rightarrow \bbr^n$; Then the smooth function
$F:M\rightarrow \bbr^{nk}=(\bbr^n)^k$ given by $F(x)=(f_1(x),\ldots ,f_k(x))$ is 1-to-1; 
mapping from a compact space to a Hausdorff one, it is a homeomorphism onto its image.
In a sense which we will eventually make precise, the smooth structure on $M$ is induced
from the map $F$ and the smooth structure on $\bbr^{nk}$, making this a smooth embedding.

\msk

{\bf Tangent vectors:} In multivariable calculus, a prominent place is taken up by vectors, 
underlying many constructions and techniques. Tangent vectors, directional
derivatives, gradients, and vector fields appear throughout the subject. Our next task is to 
introduce this technology into smooth manifolds.
It turns out there are about as many ways to approach the concept of tangent vector as there were
early researchers in the field. But in a way which we will make fairly precise, all are really
the same. We will introduce (at least) two of them, since they both have their own advantages
in different situations. 

\ssk

In $\bbr^n$, the notion of a direction is expressed by a vector $v$ based at a point. This leads
to the notion of the directional derivative $D_vf$; the rate of change of $f$ in the direction of
$v$. One way to approach (tangent) vectors for manifolds is to borrow directional derivatives,
making a defnition out of the properties which they have in multivariable calculus. This
will be one point of view we will take. 



{\bf Velocity vectors:} Borrowing vectors directly will work (with a little effort); but we can
reformulate them more directly, in terms of things that we can borrow more directly, namely 
smooth functions. Specifically,
in $\bbr^n$ a vector $v$ at $x$ describes a direction by way of the curve $\gamma(t) = 
x+tv$; $v$ is the derivative of $\gamma$ at $t=0$. We can translate this picture to a smooth
manifold using charts; given a chart $(h,U)$ around $x$, $\eta=h^{-1}\circ\gamma$, defined on 
a small interval around $0$, is a smooth curve $\eta:(-\epsilon,\epsilon)\rightarrow M$. It's
derivative at $0$, using the coordinate chart $h$, is $v$. But of course this result is dependent
upon the chart chosen, both to define it and to evaluate it. But the idea of a smooth
curve {\it isn't}. So instead we make our definition based on them. A tangent vector at a point $x$
will ``be' the derivative, at $t=0$, of a smooth curve with value $x$ at $t=0$. But different curves 
can have the same derivative, so we need to introduce an equivalence relation to make a formal definition. 

\ssk

A tangent vector at $x\in M$ is an equivalence class of smooth curves
$\gamma:(-\epsilon,\epsilon)\rightarrow M$ with $\gamma(0)=x$. Two such curves $\gamma,\eta$
are equivalent if for some chart $(h,U)$ about $x$ we have $(h\circ\gamma)^\prime(0)=(h\circ\eta)^\prime(0)$ .

\ssk 

Informally, we tend to think of this as saying that two curves are equivalent if they have the
same velocity vector at $t=0$ ! Note that the equivalence is independent of coordinate chart
chosen; if $(k,V)$ is another chart (for convenience, let us suppose that $h(x)=k(x)=0$),
then $(k\circ\gamma)^\prime(0)=[D(k\circ h^{-1})(h(x))](h\circ\eta)^\prime(0)$ 
(where this is matrix multiplication),
so $(h\circ\gamma)^\prime(0)=(h\circ\eta)^\prime(0)$ implies
$(k\circ\gamma)^\prime(0)=(k\circ\eta)^\prime(0)$.

\msk

But now we see how we {\it ought to} to relate tangent vectors from the point of view of different charts;
we use the total derivative map $D(k\circ h^{-1})(h(x))$. And we can use this to get rid of the smooth curves!
Writing $k\circ\gamma)^\prime(0)=w$ and $h\circ\eta)^\prime(0)=v$, what is important is that
$w=D(k\circ h^{-1})(h(x))v$, which needs no mention of curves at all. So we can define
a tangent vector at a point $x\in M$ as an equivalence class of triples $(x,h,v)$, where
$h$ is a chart whose domain contains $x$, and $v$ is a vector based at $h(x)\in \bbr^n$.
Another tangent vector (y,k,w) is equivalent if
$y=x$ and $w=k\circ\gamma)^\prime(0)=w$. We will let $[h,v]_x$ denote the equivalence class.
This construction illustrates a basic theme that runs throughout the development of 
differential topology: To introduce an object from calculus, all we need to 
do, really, is figure out how the object would transform when we change our point of view by using
a different chart around at point, and incorporate that into the definition, in the form of an
equivalence relation. The point, really, is that so long as we are working locally, we can essentially
pretend that it {\it is} the familiar object from calculus; it is only when we start looking at how
the object behaves as we wander around the manifold that we need to remember how they transform
as we need to keep changing coordinate charts, as our point of view keeps shifting.

The set of tangent vectors $[h,v]_x$ at a point form a vector space, the {\it tangent space},
$TM_x$ or $T_xM$, at the point $x$. The union $\bigcup T_xM=TM$ is the tangent space of $M$. We
could keep talking about this, exploring it various properties from this point of view, but let us
back up and start again using the directional derivative point of view.

\msk

{\bf Derivations:} Given a vector $v$ based at $z\in \bbr^n$, it allows us to define the directional
derivative $(D_vf)(x)$ of any differentiable function whose domain contains a neighborhood about $z$.
That is, we have an operator $D_v$ from smooth functions to $\bbr$. This operator is linear, and
satisfies a Leibnitz rule: $D_v(fg)=gD_vf+fD_vg$. Such an operator is called a {\it derivation}.

\bsk

Some random, interesting but useless(?), facts:

\msk

Every $n$-manifold can be covered by at most $n+1$ charts. The minimum for a given $M$ is
called its {\it Lusternik-Schnirelmann category}, $LS(M)$. For example, $LS(S^n)=2$ for every $n$.

\msk

$\bbr^4$ has uncountably many non-differomorphic smooth structures; but since $\bbr^5$
has only one, crossing exotic $\bbr^4$'s with $\bbr$ always gives {\it standard} $\bbr^5$.

\msk

Every $3$-manifold $M^3$ is parallelizable; $TM^3$ is fiber-preserving diffeomorphic to
$M^3\times \bbr^3$.

\msk



\vfill
\end


A proof of paracompactness doesn't use the smooth structure at all; it is true for topological 
manifolds. In the process we will prove somewhat more; our refinement will be by coordinate
neighborhoods. Given an open cover ${\Cal O}$ of $M$, let ${\Cal A}=\{(h_i,U_i\}$ be an atlas
for $M$. Replacing ${\Cal A}$ with the collection of charts whose domains are the (non-empty)
intersections of the original charts with the sets of the cover ${\Cal O}$, we may
assume that the domains of each chart lies in an element of the open cover, i.e., the
domains form a refinement of ${\Cal O}$.
Using second countability, we can choose countably many charts covering $M$; given $x\in M$,
pick a chart  $(h,U)$ containing it, and then there is a basis element $B$ with $x\in B\subseteq U$. The
$B$'s cover $M$, and there are countably many of them; a choice of one chart associated 
with each gives our countable cover. We now assume that ${\Cal A}$ consists of just those charts,
and their domains form a refinement of ${\Cal O}$.

Each chart maps to $\bbr^n$, whose image $V=h(U)$ is open; for each $V$ we can choose countably
many open balls $B(x_j,\espilon_j)\subseteq V$ so that $B(x_j,\epsilon_j/2)$ covers $V$ (using
the second countability of subspaces of $\bbr^n$, and an argument like the one above). Replacing $(h,U)$
with the collection of charts $(h_{h^{-1}(B(x_j,\epsilon_j))},h^{-1}(B(x_j,\epsilon_j)))$
(and rescaling the chart on the $\bbr^n$ side), we have a countable collection of charts covering $M$
for which the $h^{-1}(B(x,1))$ cover $M$, the image of $h$ contains $B(x,2)$, and the 
$h^{-1}(\overline{B(x,1))}$ are compact. 

Now number these sets $h^{-1}(B(x,1))=W_i$, $i\in {\Bbb N}$, and note that $C_i\overline{W_i}=$h^{-1}(\overline{B(x,1))}$
is compact. So for any finite 
$I\subseteq {\Bbb N}$, $E_I=\bigcup_{I}C_i$ is compact.
Set $I_1=\{1\}$, $E_{I_1}$ is compact, so there is a finite set $J_2\subseteq{\Bbb N}$, so that
the $W_i$ for $i\in I$ cover $E_{I_1}$. Set $I_2=J_2\cup\{2\}$ and repeat this argument, building 
$J_3$ and $I_3=J_3\cup\{3\}$. Continuing, we build $I_n$ as $J_n\cup\{n\}$
