%\baselineskip=18pt plus 2pt

\magnification=1200
\overfullrule=0pt

\parindent=50pt

\def\ni{\noindent}
\def\ctln{\centerline}
\def\msk{\medskip}
\def\ssk{\smallskip}
\def\bsk{\bigskip}

\def\iit{\itemitem}
\def\htp{\hskip10pt}
\def\vtp{\vskip.02in}
\def\hsk{\hskip.2in}
%\input shorthand

\nopagenumbers

\ctln{\bf Math 856 Introduction to Smooth Manifolds}

%\vtp

\ctln{\bf Section 001}

\smallskip

\ni{\bf Lecture:} TuTh 11:00-12:15 \htp Avery Hall (AvH) 19

\msk

\ni{\bf Instructor:} Mark Brittenham

%\smallskip

\ni{\bf Office:} Avery Hall (AvH) 317
%\smallskip

\ni{\bf Telephone:} (47)2-7222

%\ssk

\ni{\bf E-mail:} mbrittenham2@math.unl.edu

\ni{\bf WWW:} http://www.math.unl.edu/$\sim$mbrittenham2/

\ni{\bf WWW pages for this class:} http://www.math.unl.edu/$\sim$mbrittenham2/classwk/856f06/

\ssk

\ni(There you will find copies of every handout from class, homework 
assignments, class notes, and other items of interest.)

\smallskip

\ni{\bf Office Hours:} (tentatively) Mo 1:00-2:00, Tu 9:30-10:30, 
We 11:00 - 12:00, and whenever you can find me in my office and I'm not 
horrendously busy. You are also quite welcome to make an appointment
for any other time; this is easiest to arrange just before or 
after class, or via email.

\ssk

\ni{\bf Text:} {\it Introduction to Smooth Manifolds}, 
by John M. Lee (1st edition, Springer Verlag).

\msk

\ni This course, as its name is meant to imply, is intended to introduce you to 
the theory, techniques, and applications of smooth manifolds; broadly speaking,
this subject is known as differential 
topology. In basic outline we will follow the text; the specific topics covered
will depend partly on student interest. 

\msk

\ni{\bf Homework} will be assigned approximately every one to two weeks, and collected
one week after it is assigned. Since this course may be used as (and in 
future years will likely be treated as) a qualifying exam course, there will 
be more problems assigned than will be collected. Students who contemplate taking
(half of) a qualifying exam on the material from the course are strongly urged to
work all of the problems. The problems collected will
be graded and returned. These grades will form the basis for your final
course grade. There will be no midterm or final exam in this class.

\msk

\ni{\bf Departmental Grading Appeals Policy:} Students who believe their
academic evaluation has been prejudiced or capricious have recourse for appeals 
to (in order) the instructor, the departmental chair, the departmental appeals 
committee, and 
the college appeals committee.

\msk

\ctln{\bf Some important academic dates}

\ssk

{\bf Aug. 21 (Mon.)} First day of classes.

{\bf Sept. 1 (Fri.)} Last day to withdraw from a course without a {\bf `W'}.

{\bf Sept. 4 (Mon.)} Labor Day - no classes.

{\bf Oct. 13 (Fri.)} Last day to change to or from P/NP.

{\bf Oct. 16-17 (Mon.-Tue.)} Fall break - no classes.

{\bf Nov. 10 (Fri.)} Last day to withdraw from a course.

{\bf Nov. 22 (Wed.)} Student holiday - no classes.

{\bf Nov. 23-26 (Thu.-Sun.)} Thanksgiving Vacation - no classes.

{\bf Dec. 9 (Sat.)} Last day of classes.

{\bf Dec. 11-15 (Mon.-Fri.)} Final exam week.

\vfill

\end

\vfill\eject
