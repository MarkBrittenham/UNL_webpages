\magnification=1200

\nopagenumbers

\input amstex
\loadmsbm

\def\ctln{\centerline}
\def\ssk{\smallskip}
\def\msk{\medskip}
\def\bsk{\bigskip}
\def\nidt{\noindent}
\def\del{\partial}
\def\bbr{{\Bbb R}}

\def\ra{\rightarrow}
\def\lra{$\Leftrightarrow$}



\ctln{\bf Math 856 Homework 4}

\ssk

\ctln{Starred (*) problems to be handed in Thursday, October 19}

\bsk

\hskip-39pt {\bf (*) 19:} If $X,Y$ are smooth tangent vactor fields on $M$, and 
$f,g\in C^\infty(M)$, show that 

\nidt $[fX,gY] = (fg)[X,Y]+(fXg)Y-(gYf)X$ .

\msk

\nidt {\bf 20:} Show that, in $\bbr^{2n}$, the vector field

\ssk

\ctln{$\displaystyle X=x^2{{\del}\over{\del x^1}}-x^1 {{\del}\over{\del x^2}}+ \cdots 
+x^{2n}{{\del}\over{\del x^{2n-1}}}-x^{2n-1}{{\del}\over{\del x^{2n}}}$}

\ssk

\nidt restricts to
a nowhere-zero vector field tangent to the unit $(2n-1)$-sphere.

\msk

\hskip-39pt {\bf (*) 21:} [``Bundle Section Extension Lemma''] Given a smooth vector bundle $p:E\ra M$
over a smooth manifold $M$, a closed subset $A\subseteq M$, and a smooth section
$s:A\ra E$ defined over $A$ (that is, for every $a\in A$ there is a neighborhood 
$U_a$ of $a$ in $M$ and a smooth section $s_U:U\ra E$ so that $s_u=s$ on $A\cap U$), 
show that there is a global smooth section $S:M\ra E$ with $S|_A=s$ . (Hint: partition of unity...)

\msk

\nidt {\bf 22:} [Lee, p. 101, problem 4-7] Let $M,N$ be smooth manifolds, $f:M\ra N$ a smooth map, 
and define $F:M\ra M\times N$ by $F(x)=(x,f(x))$ . Show that for every tangent vector field $X$ on $M$
there is a tangent vector field $Y$ on $M\times N$ so that $Y$ is $F$-related to $X$.


\msk

\hskip-39pt {\bf (*) 23:} [Lee, p.101, problem 5-8] Let $p:E\ra M$ be a smooth $n$-dimensional 
vector bundle and 
$X_1,\ldots ,X_k$ be linearly independent smooth sections of $E$ defined over an open 
subset $U\subseteq M$.
Show that for every $a\in U$ there is a neighborhood $V$ of $a$ and smooth sections
$Y_{k+1},\ldots ,Y_n$ defined over $V$ so that $(X_1,\ldots ,X_k,Y_{k+1},\ldots ,Y_n)$
forms a local frame for $E$ over $U\cap V$. 

\ssk

(Hint: if $v_1,\ldots ,v_n$ form a basis for $\bbr^n$, then why is it that if you wiggle 
the first $k$ vectors a little bit, you still have a basis?)

\msk

\nidt {\bf 24:} Show that $M\times N$ is orientable \lra\ both $M$ and $N$ are.

\ssk



\vfill
\end



