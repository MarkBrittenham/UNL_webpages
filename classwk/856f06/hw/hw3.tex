\magnification=1200

\nopagenumbers

\input amstex
\loadmsbm

\def\ctln{\centerline}
\def\ssk{\smallskip}
\def\msk{\medskip}
\def\bsk{\bigskip}
\def\nidt{\noindent}
\def\del{\partial}
\def\bbr{{\Bbb R}}

\def\ra{\rightarrow}
\def\lra{$\Leftrightarrow$}



\ctln{\bf Math 856 Homework 3}

\ssk

\ctln{Starred (*) problems to be handed in Thursday, October 5}

\bsk

\nidt {\bf (*) 14:} Show that if $M,N$ are smooth manifolds, $M$ is connected, and
$f:M\ra N$ is a smooth map with $f_*:T_aM\ra T_{f(a)}N$ equal 
to the zero map for all $a\in M$, then $f$ is the constant function.
(Hint: show that $f^{-1}(\{f(a)\})$ is open! And beat the problem over the head with some
calculus...)

\msk

\nidt {\bf (*) 15:} For $a\in M$, let ${\Cal F_a}\subseteq C^\infty(M)$ denote the smooth 
functions satisfying $f(a)=0$. and let $L:{\Cal F}_a\ra \bbr$ be a linear operator
satisfying $L(fg)=0$ for all $f,g\in {\Cal F}_a$. Show that there is a unique
derivation $X\in T_aM$ satisfying $X|_{{\Cal F}_a}=L$ .

\ssk

(N.B. This provides still another characterization of tangent vectors, as
the vector space of linear maps $X:{\Cal F}_a/W\ra \bbr$, where 
$W = {\Cal F}_a^2$ = the ideal generated by products $fg$ for $f,g\in{\Cal F}_a$ .)


\msk

\nidt {\bf 16:} The tangent space for a manifold $M$ with boundary is defined in exactly 
the same way as for a manifold; the derivations at a point in $\del M$ are allowed
to point ``in all the directions'' of $\bbr^n$. 

\ssk

We say that a tangent vector $X\in T_aM$ for $a\in \del M$ ``points inward'' if in 
some set of local coordinates $h=(x^1,\ldots ,x^n)$ we have $X=\sum_i v^i\del/\del x^i$
with $v^n>0$. (Here $h$ maps to the upper half-space, where $x^n\geq 0$.) Show that the
notion of ``pointing inward'' is independent of coordinate chart.

\msk

\nidt {\bf 17:} Show that the $C^\infty$ manifolds $T(M\times N)$ and $TM\times TN$
are diffeomorphic.

\msk

\nidt {\bf 18:} Show that if $M^n$ is a compact smooth manifold that admits a smooth embedding
into $\bbr^{n+1}$, then $M^n\times S^k$ admits a smooth embedding
into $\bbr^{n+k+1}$ .

\ssk

Show that for any $n_1,\ldots,n_k\geq 1$ and $N=\sum_in_i$, 
$S^{n_1}\times\cdots\times S^{n_k}$ admits a smooth embedding into $\bbr^{N+1}$ .




\vfill
\end
