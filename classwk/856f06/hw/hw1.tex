\magnification=1200

\nopagenumbers

\input amstex
\loadmsbm

\def\ctln{\centerline}
\def\ssk{\smallskip}
\def\msk{\medskip}
\def\bsk{\bigskip}
\def\nidt{\noindent}
\def\del{\partial}
\def\bbr{{\Bbb R}}


\ctln{\bf Math 856 Homework 1}

\ssk

\ctln{Starred (*) problems to be handed in Thursday, September 7}

\bsk

\nidt {\bf (*) 1:} Show that a connected manifold $M$ is {\it arcwise connected},
that is, for every pair of points $x,y\in M$ there is a
{\it one-to-one} path $\gamma:[0,1]\rightarrow M$ with $\gamma(0)=x,\gamma(1)=y$.

\ssk

(There is a theorem, due to Hahn and Mazurkiewicz (circa 1914), which says 
that a Hausdorff space is path connected iff it is
arcwise connected (which is kind of funny, since the term ``Hausdorff''
wasn't really introduced until the 1920s?); but for manifolds you can 
give a much more elementary proof...)

\msk

\nidt{\bf 2:} Show that if $A,B\subseteq {\Bbb R}^2$ are closed subsets, the statement

\ctln{`` ${\Bbb R}^2\setminus A \cong {\Bbb R}^2\setminus B$ $\Rightarrow$
$A \cong B$ ''}

\hfill is {\bf false}. What about the converse statement?
(N.B. That might be harder?)

\msk

\nidt {\bf (*) 3:} Given a collection of triangles (or 2-simplices, you are more
comfortable with that terminology) $T_i$, $i=1,\ldots 2r$, with edges
$e_{i1},e_{i2},e_{e3}$, and a collection of $3r$ homeomorphisms 
$h_k:e_{i_kj_k}\rightarrow e_{i_k^\prime j_k^\prime}$ involving all $6r$ edges
(as either domain or range),
show (in a quasi-rigorous fashion?) that the quotient space obtained
by gluing the 2-disks $T_i$ together using the maps $h_k$ is a 2-manifold.
(There are basically three ``kinds'' of points to worry about. ``Describe'' 
locally Euclidean neighborhoods for each.)

\msk

\nidt {\bf (*) 4:} (Lee, p. 28, problem 1-4) If $0\leq k\leq \min\{m,n\}$, show that the set 
$R_k\subseteq M(m\times n,{\Bbb R})$ of $m$-by-$n$ matrices 
with rank $\geq k$ is an open subset of $M(m\times n,{\Bbb R})
\cong {\Bbb R}^{mn}$ (and therefore admits a smooth structure).
({\it Hint:} look at Lee's linear algebra appendix...)

\msk

\nidt {\bf (*) 5.:} We say that two charts 
$\phi :U\rightarrow {\Bbb R}^n$ , $\psi :V\rightarrow {\Bbb R}^n$,
$U,V\subseteq M^n$ are \underbar{$C^\infty$-related}
if $\psi\circ\phi^{-1}:\phi(U\cap V)\rightarrow \psi(U\cap V)$
and $\phi\circ\psi^{-1}:\psi(U\cap V)\rightarrow \phi(U\cap V)$
are both $C^\infty$. Show that the relation `` is $C^\infty$-related
to '' is {\bf not} an equivalence relation.
(Hint: $M^n = {\Bbb R}$ will suffice for an example...)

\msk

\nidt {\bf (*) 6:} Show that ${\Bbb R}$ has uncountably many distinct smooth structures.
((Perhaps) show first that it is enough to find uncountably many charts, with intersecting domains 
and ranges, no two of which are $C^\infty$-related to one another.)

\msk

\nidt {\bf 7:} Lee, page 28-29, problem 1-5. [It was too long to copy out.]

\msk

\nidt {\bf 8:} Show that a function $f:M^n\rightarrow N^m$ is $C^\infty$ $\Leftrightarrow$ 
$g\circ f:M^n\rightarrow {\Bbb R}$ is $C^\infty$ for {\it every} $C^\infty$ function
$g:N^m\rightarrow {\Bbb R}$. (Hint: you might need to use the technology of bump functions
found on p.55 of the text?)


\vfill
\end
