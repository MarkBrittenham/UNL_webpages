\magnification=1200

\nopagenumbers

\input amstex
\loadmsbm

\def\ctln{\centerline}
\def\ssk{\smallskip}
\def\msk{\medskip}
\def\bsk{\bigskip}
\def\nidt{\noindent}
\def\del{\partial}
\def\bbr{{\Bbb R}}

\def\ra{\rightarrow}
\def\lra{$\Leftrightarrow$}



\ctln{\bf Math 856 Homework 7}

\ssk

\ctln{Starred (*) problems to be handed in Thursday, December 7}

\bsk

\item{\bf 35:} (Lee, p.287, Problem 11-8) Show that the dimension of the 
space of symmetric covariant $k$-tensors over an $n$-dimensional vector space $V$,
$\Sigma^k(V)$, is $\displaystyle n+k-1\choose k$ .

\msk

\item{\bf 36:} Show that a diffeomorphism $F:M\ra N$ induces a bundle isomorphism
$F^*:T^k(N)\ra T^k(M)$ .

\msk

\item{\bf 37:} (Lee, p.286, Problem 11-7) (a) Show that if 
$S$ is a covariant $k$-tensor field and $X_1,\ldots ,X_k$ are
vector fields, then the assignment 
$(X_1,\ldots ,X_k)\mapsto S(X_1,\ldots ,X_k)$
defines a map ${\Cal T}(M)\times\cdots\times{\Cal T}(M)\ra C^\infty(M)$
which is multilinear over $C^\infty$ functions, i.e., 
$S(X_1,\ldots ,f_1X^i+f_2Y_i,\ldots ,X_k)=
f_1S(X_1,\ldots X^i,\ldots ,X_k)+f_2S(X_1,\ldots Y_i,\ldots ,X_k)$~.

\ssk

\item{}(b) Show the converse, i.e.,  a map 
${\Cal T}(M)\times\cdots\times{\Cal T}(M)\ra C^\infty(M)$ is induced 
by a covariant $k$-tensor field \lra\ the map is linear over $C^\infty$
functions.

\ssk

\item{}[N.B. This result provides the ``standard'' way to build covariant
tensor fields without resorting to local coordinates to do it. Of 
course, the proof of this result requires using local coordinates...!]


\msk

\item{\bf 38:} (Lee, p.319, Problem 12-4) Show that two $k$-tuples
$\{\omega_1,\ldots,\omega_k\},\{\eta_1,\ldots,\eta_k\}\subseteq T^1(V)$
of linearly independent covectors have the same span \lra\
$\omega_1\wedge\cdots\wedge\omega_k=c\eta_1\wedge\cdots\wedge\eta_k$
for some $c\in \bbr$.

\msk

\item{\bf 39:} (a) Show that $\int_\gamma df = \int_{[0,1]} \gamma^*(df)=0$
for every smooth closed curve $\gamma:[0,1]\ra M$ (i.e., $\gamma(0)=\gamma(1)$).
[Hint: Stokes' Theorem!]

\ssk

\item{}(b) Show, conversely, that if $\omega$ is a 1-form on $M$ and 
$\int_\gamma\omega=\int_{[0,1]}\gamma^*\omega=0$
for every smooth closed curve $\gamma:[0,1]\ra M$,
then there is an $f\in C^\infty(M)$ with $\omega=df$.

\item{}[Hint: If $\omega=df$, then what is $\int_\gamma\omega$ for a 
{\bf non}-closed curve $\gamma$ ?]

\msk


\vfill
\end



\hskip-39pt {\bf (*) 
