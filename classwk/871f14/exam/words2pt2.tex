



\documentclass[12pt]{article}
\usepackage{amsfonts}
\usepackage{amssymb}
\usepackage{dutchcal}

\textheight=8.9in
\textwidth=6.5in
\voffset=-1in
\hoffset=-.5in
\parindent=-10pt

\begin{document}

\def\ctln{\centerline}
\def\msk{\medskip}
\def\bsk{\bigskip}
\def\ssk{\smallskip}
\def\ra{\rightarrow}
\def\ubr{\underbar}
\def\sset{\subseteq}
\def\hsk{\hskip10pt}

\def\mt{{\mathcal T}}
\def\mb{{\mathcal B}}
\def\mbp{{\mathcal B}^\prime}
\def\mbpp{{\mathcal B}^{\prime\prime}}
\def\ms{{\mathcal S}}
\def\mu{{\mathcal U}}
\def\mv{{\mathcal V}}
\def\mp{{\mathcal P}}
\def\mtp{{\mathcal T}^\prime}
\def\mtpp{{\mathcal T}^{\prime\prime}}

\def\bbr{{\mathbb R}}
\def\bbz{{\mathbb Z}}
\def\bbq{{\mathbb Q}}
\def\spc{$~$\hskip.15in$~$}
\def\bbd{{\mathbb D}}
\def\bbc{{\mathbb C}}

\def\up{U^\prime}
\def\upp{U^{\prime\prime}}
\def\vp{V^\prime}
\def\vpp{V^{\prime\prime}}
\def\wp{W^\prime}
\def\wpp{W^{\prime\prime}}

\def\finv{f^{-1}}
\def\ginv{g^{-1}}
\def\hinv{h^{-1}}

\def\sset{\subseteq}
\def\lra{$\Leftrightarrow$}
\def\smin{\setminus}
\def\rta{$\Rightarrow$}
\def\rtta{\Rightarrow}

%%\UseAMSsymbols

\ctln{\bf Math 871 Exam 2 Topics Part 2: The homotopy part}

\msk

{\bf Homotopy Theory.}

\ssk

Motivation: understand \ubr{all} continuous functions $f:X\ra Y$,
since it is functions to/from `model' spaces that allow us to 
explore a space. 

\ssk

E.g., paths = $\gamma:I=[0,1]\ra X$. How many `essentially distinct'
paths are there from $(-1,0)$ to $(1,0)$ in $\bbr^2\smin\{(0,0)\}$ ?
What is inessetial? \ubr{Deformations}.

\ssk

Two maps $f,g:X\ra Y$ are {\it homotopic} if one can be deformed 
to the other (through continuous maps). Formally, there is a
cts map $H:X\times I\ra Y$ so that $H(x,0)=f(x)$ and $H(x,1)=g(x)$
for all $x\in X$. We write: $f\simeq g$ (via $H$).

\ssk

Note: $\gamma_x:t\mapsto H(x,t)$ is a cts path in $Y$, for every $x$.

\ssk

Notation: $f:(X,A)\ra (Y,B)$ means $A\sset X$, $B\sset Y$ and $f(A)\sset B$.

\ssk

Two maps $f,g:(X,A)\ra (Y,B)$ are homotopic rel $A$ if 
$H:X\times I\ra Y$ also satisfies $H(a,t)=f(a)=g(a)$ for all $a\in A$, $t\in I$.
[So, in part, $f|_A=g|_A$ .]

\ssk

Basic example: any two maps $f,g:X\ra \bbr^n$ are homotopic, via a
{\it straight-line homotopy}: $H(x,t)=(1-t)f(x)+tg(x)$.

\msk

Homotopy is an \ubr{equivalence} \ubr{relation}: $f\simeq f$  (via $H(x,t)=f(x)$),
$f\simeq g$ implies $g\simeq f$ (via $K(x,t)=H(x,1-t)$); 
$f\simeq g$ and $g\simeq h$ implies $f\simeq h$ (via doubling the speed;
$M(x,t)=H(x,2t)$ for $t\leq 1/2$ and $=K(x,2t-1)$ for $t\geq 1/2$).

\ssk

This allows us to introduce a new notion of equivalence of topological spaces.
$X$ and $Y$ are {\it homotopy equivalent} [we write $X\simeq Y$]
if there are $f:X\ra Y$ and $g:Y\ra X$ so that $g\circ f\simeq {\textrm Id}_X$
and $f\circ g\simeq{\textrm Id}_Y$ . 

\ssk

Homotopy equivalence is an equivalence relation! Note: a homeomorphism is a
homotopy equivalence! [$g\circ f = {\textrm Id}_X\simeq {\textrm Id}_X$].

\msk

{\bf The homotopy viewpoint.}

\ssk

The basic idea is that homotopy equivalence  (= `h.e.') 
allows us to move past/around
`unimportant' differences in spaces. For example, 
$\bbr^2\smin\{(0,0)\}\cong S^1\times\bbr\simeq S^1\times I\simeq S^1$
means that maps into $\bbr^2\smin\{(0,0)\}$ `behave like' maps
into $S^1$ (which we can more readily understand?).

\ssk

{\it Algebraic topology} seeks to understand topological spaces through
algebraic invariants. An algebraic invariant assigns to each space $X$ an 
algebraic object $A(X)$ and to each map $f:X\ra Y$ a 
homomorphism $A(f):A(X)\ra A(Y)$. If $X$ and $Y$ are the `same', then 
$A(X)$ and $A(X)$ will be isomorphic. Usually, `same' means homeomorphic,
but we will often find that homotopy equivalent spaces will 
same the same invariants, due the the methods that we use to build 
them.

\ssk

This can be both bad and good, `homotopy invariance' of a invariant means
that it will not be able to distinguish h.e. spaces that are not homeomorphic.
But it also means that when computing an algebraic invariant, we can replace
a space $X$ with $Y\simeq Y$, which may streamline a computation.

\msk

A {\it retraction} of $X$ onto $A\sset X$ is a map $r:X\ra A$ so that $r(a)=a$
for all $a\in A$. [$A$ is a {\it retract} of $X$]. $A$ is a {\it deformation
retract} of $X$ if $\iota\circ r:X\ra A\ra X$ is $\simeq{\textrm Id}_X$
[$r$ is a {\it deformation retraction}]. and $r$ is a {\it strong
deformation retraction} if $\iota\circ r:(X,A)\ra (X,A)$ is $\simeq{\textrm Id}_X$
rel $A$ (i.e., $H(a,t)=a$ for all $a\in A$). We write $X\searrow A$. 

\ssk

For example, $r:\bbr^n\searrow\{\vec{0}\}$, since $\iota\circ r\simeq{\textrm Id}_{\bbr^n}$
via a straight-line homotopy 

\hfill $H(x,t)=(1-t)\iota\circ r(\vec{x})+t{\textrm Id}_{\bbr^n}(\vec{x})=t\vec{x}$ .

\ssk

A space $X$ is {\it contractible} if $X\simeq\{*\}$.

\ssk

Mapping cylinders:  If $f:X\ra Y$, then $M_f = X\times I\coprod Y/\sim$, where
$(x,1)\sim f(x)$. [Idea: we glue $X\times\{1\}$ to $Y$ using $f$.] Then since 
$X\times I\searrow X\times\{1\}$, we have $M_f\searrow Y$.

\ssk

Fact: $f:X\ra Y$ is a homotopy equivalence \lra\ $M_f\searrow X\times\{0\}$.
This means that $X\simeq Y$ \lra\ there is a space $Z$ with $X,Y\sset Z$ and 
$Z\searrow X$, $Z\searrow Y$.

\msk

{\bf The Fundamental Group.}

\ssk

Idea: find the essentially distinct paths between points in $X$.
How? Turn this into a group! How? The concatenation $\gamma*\eta$ of two paths is
a path. But: only if the first ends where the second begins (so that, by
the Pasting Lemma, the resulting map is cts). So we
either have a \ubr{partial} multiplication (= groupoid!), or we
focus on \ubr{loops} $\gamma:(I,\partial I)\ra (X,x_0)$ based at a
fixed point $x_0$ 9we'll do the second).

\ssk

Elements of the {\it fundamental group} $\pi_1(X,x_0)$ `are' 
loops; the inverse will be the reverse $\overline{\gamma}(t)=\gamma(1-t)$,
since $\gamma*\overline{\gamma}\simeq c_{x_0}$, and the 
identity element will be the constant map $c_{x_0}$.
But! to make $\gamma*\overline{\gamma}$ equal $c_{x_0}$, we need to work with
{\it homotopy classes} of loops. So elements really are 
equivalence classes $[\gamma]$ of loops, under $\simeq$ rel $\partial I$.

\ssk

Then by building homotopies (mostly working on the domain $I$,
i.e., building $K=\gamma\circ H:I\times I\ra I\ra X$)
we can see that $[\gamma][\eta] = [\gamma*\eta]$ is well defined,
$[\gamma]^{-1}=[\overline{\gamma}]$ is the inverse, and 
$([\gamma][\eta])[\omega] = [\gamma]([\eta][\omega])$, so 
under $*$, $\pi_1(X,x_0)$ is a group. [Most of the proofs that
needed maps (like $(\gamma*\eta)*\omega$ and $\gamma*(\eta*\omega)$
(which are the same concatenations, except at 4,4, and 2 times speed, versus 
2,4, and 4 times speed) are homotopic can be given `picture' proofs, 
in addition to explicit analytic formulas.

\ssk

Given a map $f:(X,x_0)\ra (Y,y_0)$, we get an induced map
$f_*:\pi_1(X,x_0)\ra \pi_1(Y,y_0)$ via $f_*[\gamma]=[f\circ\gamma]$.
This is well-defined, and a homomorphism.

\ssk

Basic computations: $\pi_1(\{*\},*) = \{1\}$, as are 
$\pi_1(\bbr^n,\vec{0})$ and $\pi_1([0,1]^n,x_0)$ for any
$x_0$. More generally, any contractible space has trivial fundamental
group.

\msk

Since $(f\circ g)_*=f_*\circ g_*$, and $({\textrm Id}_X)_*={\textrm Id}_{\pi_1(X,x_0)}$,
then $X\cong Y$ via $f$ implies $f_*:\pi_1(X,x_0)\ra \pi_1(Y,f(x_0))$
is an isomorphism.

\ssk

More generally, if $f:X\ra Y$ is a h.e., then $f_*$ is an isomorphism,
but, because of basepoint issues, the inverse of $f_*$ is generally
\ubr{not} $g_*$ for $g$ a homotopy inverse. Th is is because under
a homotopy $H:X\times I\ra X$ of $g\circ f$ to ${\textrm Id}$,
the basepoint $x_0$ traces out a path $\eta$ from $g(f(x_0))=x_1$ to $x_0$,
and $[g\circ f\circ\gamma] = [\overline{\eta}*\gamma *\eta]$.
This map $[\gamma]\mapsto [\overline{\eta}*\gamma *\eta]$ from
$\pi_1(X,x_0)$ to $\pi_1(X,x_1)$ is a {\it change of basepoint 
isomorphism}, which we might call $\eta_*$ ? The fact that homotopies
can drag basepoints around will be a theme we will return to 
many times moving forward.

\ssk

If $X$ is path connected, then, up to isomorphism, $\pi_1(X,x_0)$ 
is independent of $x_0$ (we can always find a path to effect an 
isomorphism), and so we will often write $pi_1(X)$, when $X$ is 
path-connected, when we only care about the abstract group.

\msk

$\pi_1(S^1,(1,0))\cong\bbz$. The main ingredients:

\ssk

Writing $S^1\sset\bbc$ and $\gamma_n(t)=e^{2\pi int}$ is the loop
traversing $S^1$ $n$ times counterclockwise at uniform speed,
then (1) every loop $\gamma$ at $(1,0)$ is $\simeq\gamma_n$ for some $n$.

\ssk

We define $w:\pi_1(S^1,(1,0))\ra\bbz$ by $w[\gamma] = n$ if
$[\gamma]=[\gamma_n]$. This is well-defined:
(2) if $\gamma_n\simeq\gamma_m$ rel endpoints, then $n=m$.

\ssk

$w$ is a bijective homomorphism!

\ssk

The proof of (1) amounted to making a general $\gamma$ progreessively nicer, 
via homotopy. This involved 

\ssk

{\it Lebesgue Number Theorem}: If $(X,d)$ is a compact metric space and
$\{U_\alpha\}$ is an open covering of $X$, then there is an $\epsilon>0$
so that for every $x\in X$ there is an $\alpha=\alpha(x)$ so that 
we have $N_d(x,\epsilon)\sset U_\alpha$.

\ssk

Then by covering $S^1$ by the `top 2/3rds' and `bottom 2/3rds' subsets
and taking inverse images under $\gamma:(I,\partial I)\ra (S^1,(1,0))$,
the LNT will partition $I$ into finitely many intervals each mapping into
top or bottom. Creating subpaths by restricting to each subinterval, and inserting
`hairs' to points $(1,0),(-1,0)$ in the intersection of top and bottom,
we can then homotope the subpaths to standard paths $t\mapsto e^{\pm 2\pi it}$.
Cancelling pairs the reverse direction give us our `normal forms'
$\gamma_n$.

\ssk

The proof of (2) amounted to using an `extra' coordinate $(\cos t,\sin t,t)$
to keep track of how many times we wind around the circle. To do this correctly,
we really use the map $p:t\mapsto(\cos t,\sin t,t)\mapsto(\cos t,\sin t)$ and
then \ubr{lift} paths $\gamma:I\ra S^1$ to paths 
$\widetilde{\gamma}:I\ra \bbr$ with $\gamma = p\circ\widetilde{\gamma}$. 
This agin uses the LNT to partition $I$ into subintervals mapping into top and bottom,
and the fact that the inverse image of top and bottom are a disjoint
union of open sets mapped homeomorphically under $p$ to the top and bottom.
[This is the {\it evenly covered property}.]

\ssk

More than this, homotopies $H:I\times I\ra S^1$ can also be lifted; this enables us to
show that loops homotopic rel endpoints, when lifted both starting at the same point,
will end at the same point. Since $\gamma_n$ when lifted starting at $0$ will
end at $n$, the result follows.

\msk

{\bf Applications.}
This single computation has many applications! First, there is no
retraction $r:\bbd^2\ra \partial\bbd^2$. This is because if there were one,
then $r_*:\pi_1(\bbd^2,(1,0))\ra\pi_1(S^1,(1,0))$ would be a surjection,
which is impossible.

\ssk

This in turn gives the {\it Brouwer Fixed Point Theorem}: Every countinuous
map $f:\bbd^2\ra \bbd^2$ has a fixed point. For if not, we can then 
manufacture a retraction $r:\bbd^2\ra \partial\bbd^2$.

\ssk

Finally, we can prove the {\it Fundamental Theorem of Algebra}: Every non-constant
polynomial $p$ has a complex root. For if not, then for large enough $N$ the map

\ctln{$t\mapsto f(Ne^{2\pi it})\mapsto f(Ne^{2\pi it})/||f(Ne^{2\pi it})||$}

from 
$I$ to $S^1$ is homotopic to both $c_{(1,0)}=\gamma_0$ and $\gamma_n$ for $n$ = the
degree of $f$, a contradiction.

\vfill
\end{document}



