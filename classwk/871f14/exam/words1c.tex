



\documentclass[12pt]{article}
\usepackage{amsfonts}
\usepackage{amssymb}
\usepackage{dutchcal}

\textheight=8.9in
\textwidth=6.5in
\voffset=-1in
\hoffset=-.5in
\parindent=-10pt

\begin{document}

\def\ctln{\centerline}
\def\msk{\medskip}
\def\bsk{\bigskip}
\def\ssk{\smallskip}
\def\ra{\rightarrow}
\def\ubr{\underbar}
\def\sset{\subseteq}
\def\hsk{\hskip10pt}

\def\mt{{\mathcal T}}
\def\mb{{\mathcal B}}
\def\mbp{{\mathcal B}^\prime}
\def\mbpp{{\mathcal B}^{\prime\prime}}
\def\ms{{\mathcal S}}
\def\mu{{\mathcal U}}
\def\mv{{\mathcal V}}
\def\mp{{\mathcal P}}
\def\mtp{{\mathcal T}^\prime}
\def\mtpp{{\mathcal T}^{\prime\prime}}

\def\bbr{{\mathbb R}}
\def\bbz{{\mathbb Z}}
\def\bbq{{\mathbb Q}}
\def\spc{$~$\hskip.15in$~$}

\def\up{U^\prime}
\def\upp{U^{\prime\prime}}
\def\vp{V^\prime}
\def\vpp{V^{\prime\prime}}
\def\wp{W^\prime}
\def\wpp{W^{\prime\prime}}

\def\finv{f^{-1}}
\def\ginv{g^{-1}}
\def\hinv{h^{-1}}

\def\sset{\subseteq}
\def\lra{$\Leftrightarrow$}
\def\smin{\setminus}
\def\rta{\Rightarrow}

%%\UseAMSsymbols

\ctln{\bf Math 871 Exam 1 Topics}

\msk

Starting point: continuity

$f:\bbr\ra\bbr$ is continuous at $x_0$ if $|f(x)-f(x_0)|$ is small
so long as $|x-x_0|$ is small \underbar{enough}.

Goal: make this notion make sense more generally, and determine what
makes the fundamental results (extreme value theorem, intermediate
value theorem) work.

Two ideas help: a neighborhood of $x$ is one which contains all
points `close enough' to $x$. The {\it inverse image} of a set
is the collection of points that the function $f$ maps into $x$.
Then: continuity at $x$ means that the inverse image of a neighborhood
of $f(x)$ is a neighborhood of $x$.

\msk

Images and inverse images.

Given $f:X\ra Y$ a function, and $A\subseteq X$, $B\subseteq Y$, we have
images $f(A) = \{f(x)\ :\ x\in A\}$ and inverse images $f^{-1}(B) = \{x\in X\ :\ f(x)\in B\}$

Inverse images are very well-behaved! 
$f^{-1}(\cap_\alpha B_\alpha) = \cap_\alpha f^{-1}(B_\alpha)$ , 
$f^{-1}(\cup_\alpha B_\alpha) = \cup_\alpha f^{-1}(B_\alpha)$ ,
$f^{-1}(Y\setminus B) = X\setminus f^{-1}(B)$ .

But inages are not as well-behaved: $f(\cup_\alpha A_\alpha) = \cup_\alpha f(A_\alpha)$ ,
but only $f(\cap_\alpha A_\alpha) \subseteq \cap_\alpha f(A_\alpha)$ in general.

\msk

Finite/countable/uncountable sets.

A set $S$ is finite if for some $n\in\bbz_+$ there is a 
bijective function $S\leftrightarrow \{1,\ldots,n\}$

Equivalently, for some $n$, $\{1,\ldots,n\}\twoheadrightarrow S$ , 
or $S\hookrightarrow\{1,\ldots,n\}$ .

Some results: if $A,B$ are finite, then $A\cup B$, $A\times B$, and the
set of all functions $\{f:A\ra B\}$ are finite.

If $B$ is finite and $A\subseteq B$ then $A$ is finite.

Countably infinite: there is a bijection $S\leftrightarrow \bbz_+$. 

Countable: finite or countable infinite. Equivalently, there exists a
surjection $\bbz_+\twoheadrightarrow S$, or there exists an injection
$S\hookrightarrow \bbz_+$.

Infinite: not finite! Equivalently, there is an injection $\bbz_+\hookrightarrow S$
(or a surjection $S\twoheadrightarrow\bbz_+$)

Uncountable: not countable! Equivalently, there is \underbar{no}
surjection $\bbz_+\twoheadrightarrow S$ (or \underbar{no}
injection $S\hookrightarrow \bbz_+$).

Examples: $\bbr$ , or all functions $f:\bbz_+\ra\{0,1\}$ (note: the second
is equivalent to all subsets of $\bbz_+$, via $f\leftrightarrow f^{-1}(\{0\})$)

\msk

Back to continuity.

Given a {\it metric} on a set $X$ 

[a function $d:X\times X\ra \bbr$ with,
for all $x,y,z\in X$,
(1) $d(x,y)\geq 0$, and $d(x,y) = 0$ iff $x=y$, (2) $d(x,y)=d(y,x)$, and (3)
$d(x,z)\leq d(x,y)+d(y,z)$]

We can formalize continuity at $x_0\in X$ using neighborhoods $N_d(x_0,\epsilon) = 
\{x\in X\ :\ d(x_0,x)<\epsilon\}$; we need
$f^{-1}(N_{d^\prime}(f(x_0),\epsilon))$ contains some $N_d(x_0,\delta)$. But more;
since $x\in N_d(x_0,\epsilon)$ implies that 
$N_d(x,\epsilon-d(x,x_0))\subseteq N_d(x_0,\epsilon)$, we have that
$x\in f^{-1}(N_{d^\prime}(f(x_0),\epsilon))$ implies that 
$N_d(x,\delta)\subseteq f^{-1}(N_{d^\prime}(f(x_0),\epsilon))$ for some $\delta>0$.
That is, $f^{-1}(N_{d^\prime}(f(x_0),\epsilon))$ is, for every $x_0\in X$, a union of 
neighborhoods.

This leads to: $U\subseteq X$ is {\it open} if it is a union of neighborhoods. And
$f:X\ra Y$ is continuous if $f^{-1}(V)$  is open (in $X$) for 
\underbar{every} $V\subseteq Y$ open (in $Y$). And we just eliminated (explicit)
mention of a point; continuity `at' $x_0$ has been eliminated, leaving just `continuity'.

\msk

Topologies.

Looking at what properties open sets in a metric space have leads us to the 
notion of a topology on a set $X$: it is a collection $\mt$ of subsets of $X$ so that

(1) $\emptyset,X\in \mt$

(2) if $U_\alpha\in\mt$ for all $\alpha$, then $\cup_\alpha U_\alpha\in\mt$

(3) if $U,V\in\mt$ then $U\cap V\in\mt$

And a function $f:(X,\mt)\ra(Y,\mt^\prime)$ between {\it topological spaces}
(i.e., sets with particular topologies) is {\it continuous} (cts) if
$f^{-1}(V)\in\mt$ for every $V\in\mt^\prime$.

Constant maps are always cts; compositions of cts maps are cts.

\msk

Examples.

$X$ = any set, $\mt=\mp(X)$ - all subsets of $X$, the {\it discrete} topology.

$X$ = any set, $\mt=\{\emptyset,X\}$, the {\it indiscrete} (or trivial) topology.

$X$ = any set, $\mt_f=\{A\subseteq X\ :\ X\setminus A\ \textrm{is finite}\}\cup\{\emptyset\}$, the {\it finite complement} topology.

$X$ = any set, $\mt_c=\{A\subseteq X\ :\ X\setminus A\ \textrm{is countable}\}\cup\{\emptyset\}$, the {\it countable complement} topology.

$X$ = any set, $a\in X$, $\mt_{ip}=\{A\subseteq X\ :\ a\in A\}\cup\{\emptyset\}$, the {\it included point} topology.

$X$ = any set, $a\in X$, $\mt_{ip}=\{A\subseteq X\ :\ a\not\in A\}\cup\{X\}$, the {\it excluded point} topology.

$X$ = $\bbr$, $\mt_{irr}=\{(a\infty)\}\cup\{\emptyset,\bbr\}$, the {\it infinite ray} (to the right) topology.

$X$ = any metric space,$\mt_{d}=\{\textrm{all unions of $d$-neighborhoods in }X\}$
= $\{\cup_i N_d(x_i,\epsilon_i)\ :\ x_i\in X, \epsilon_i>0\}$, the {\it metric} topology.

$\mt,\mt^\prime$ topologies on $X$, with $\mt\subseteq\mt^\prime$, we say 
$\mt$ is coarser/smaller than $\mt^\prime$, and $\mt^\prime$ is finer/larger than $\mt$.

Basic idea: coarser $\Rightarrow$ more cts functions into $X$ (fewer inverse image to check), 
finer $\Rightarrow$ more cts functions out of $X$ (more likely to contain the inverse images).

$f:(X\mt)\ra Y$. $\mt^\prime = \{V\subseteq Y\ :\ f^{-1}(V)\in\mt\}$ is the finest topology on $Y$
making $f$ cts.

$g:X\ra (Y,\mt^\prime)$. $\mt = \{f^{-1}(V)\ :\ V\in\mt^\prime\}$ is the coarsest topology on $X$
making $f$ cts.

A set $U\sset X$ is open \lra\ for every $x\in U$ there is $U_x\in\mt$ with $x\in U_x\sset U$.
[Note: if you know $U$ is open, $U_x=U$ works!]

\msk

Bases/subbases.

Metric topologies are defined as unions of neighborhoods. What makes this a topology?

A basis $\mb$ for a topology on $X$ is a collection of subsets so that

\hsk (1) union of elements of $\mb$ is $X$ [every $x\in X$ lies in some $B\in\mb$]

\hsk (2) the intersection of two is a union of elements of $\mb$ [if $B,B^\prime\in\mb$ and
$x\in B\cap B^\prime$ then $x\in B^{\prime\prime}\subseteq B\cap B^\prime$ for some
$B^{\prime\prime}\in\mb$]

$\mt(\mb)$ = unions of elements of $\mb$ = the topology generated by $\mb$,
the coarsest toplology containing $\mb$

$f:(X,\mt)\ra (Y,\mt(\mb))$ is cts $\Leftrightarrow$ $f^{-1}(B)\in\mt$ for all $B\in\mb$

Examples:

$\mb = \{(a,b)\sset\bbr\ :\ a,b\in\bbr\}$ is a basis for the usual topology on $\bbr$;
restrict to $a,b\in\bbq$, still a basis for usual topology.

$\mb = \{[a,b)\sset\bbr\ :\ a,b\in\bbr\}$ is a basis; $\mt_\ell = \mt(\mb)$ = the {\it lower limit}
topology on $\bbr$. Strictly finer than the usual topology!

Subbasis $\ms$: insist only on (1) union of elements of $\ms$ is $X$.

$\mb = \mb(\ms) = \{S_1\cap\cdots\cap S_n\ :\ n\in\bbz_+, S_i\in\ms\}$ is a basis;
$\mt(\ms) = \mt(\mb)$ = topology generated by $\ms$.

Detecting bases: If $\mb\sset\mt$ and $\mt\sset\mt(\mb)$, then $\mt=\mt(\mb)$

\msk

Product topologies.

$(X,\mt)$, $(Y,\mtp)$ top spaces, then $\mb = \mt\times\mtp$ (subsets of $X\times Y$) is not a topology,
but it is a basis for one (closed under intersection). $\mt(\mt\times\mtp)$ = (abusing notation
`$\mt\times\mtp$' is the {\it product topology} on $X\times Y$.

Motivation: want projection maps $p_X:X\times Y\ra X$, $p_X(x,y)=x$, etc. continuous! This
forces topology on $X\times Y$ to contain certain sets, coarsest topology that contains them 
is product topology.

If $\mt=\mt(\mb)$, $\mtp=\mt(\mbp)$, then $\mb\times\mbp$ is a basis for the product topology.

Example: product topology on $\bbr^2$, $\bbr^n$. \ubr{Same} as the (usual) metric topologies!

Arbitrary products: two choices! $\prod_i X_i$, coarsest topology making all projections cts
is the {\it product topology}, it has subbasis $\{p_{X_i}^{-1}(U)\ :\ i\in I, u\in \mt_i\}$
(basis is $\prod_i U_i$, where all but finitely many $U_i$ are $X_i$).

Box topology: basis is $\{\prod_i U_i\ :\ U_i\in\mt_i\}$; all factors can be proper open sets.

If $I$ is infinite, box topology is (generally) strictly finer than product topology

Recognizing cts fcns: $f:(X,\mt)\ra (\prod_i X_i,\textrm{prod top})$ is cts \lra\ 
$p_{X_i}\circ f: X\ra X_i$ is cts for all $i$.

This is not true if $\prod_i X_i$ is given the box topology!

\msk

Subspaces.

Motivation $(X,\mt)$ and $A\sset X$, would be useful if inclusion map $\iota:A\hookrightarrow X$
is cts.

$\mt_A = \{A\cap U\ :\ U\in\mt\}$ is a topology on $A$, the coarsest making $\iota$ cts
($A\cap U = \iota^{-1}(U)$), called the {\it subspace topology} on $A$.

$f:(X,\mt)\ra (Y,\mtp)$ cts and $A\sset X$, then $f\circ\iota=f|_A:(A,\mt_A)\ra (Y,\mt)$ is cts

$\mt=\mt(\mb)$, then $\mb_A = \{A\cap B\ :\ B\in\mb\}$ is a basis for $\mt_A$
 
$A\sset X$, $B\sset Y$, then $A\times B\sset X\times Y$, and the subspace topology on 
$A\times B$ is the same as the product of the subspace topologies (same bases!)

Subtlety: $f|_A:A\ra Y$ cts means $A\cap\finv(U)$ is open in $A$, i.e.,
$A\cap\finv(U)=A\cap V$ for \ubr{some} $V$ open in $X$. [I.e., $f|_A$ can be cts when $f$ isn't!]

\msk

Closed sets.

$(X,\mt)$ top space, $C\sset X$ is {\it closed} (w.r.t. $\mt$; omitted if clear from context)
if $X\smin C\in\mt$. [Equivalently, $C=X\smin U$ for some $U\in\mt$]

$C_i$ closed $\rta$ $\cap_i C_i$ closed; $C,D\sset X$ closed $\rta$ $C\cap D$ closed.

One can cast everything about a topology in terms of closed sets; e.g., a fcn
$f:X\ra Y$ is cts \lra\ $\finv(C)\sset X$ is closed for evvery $c\sset Y$ closed.

$C\sset X$, $D\sset Y$ both closed $\rta$ $C\times D\sset X\times Y$ is closed.
($\Leftarrow$ requires both sets non-empty)

\msk

Closure.

$C\sset A\sset X$ is closed in $(A,\mt_A)$ \lra\ $C=A\cap D$ for $D\sset X$ closed.

Closed sets are closed (no put intended) under intersection, so can (usually) find a
\ubr{smallest} closed set satisfying a property, as $\cap\{\textrm{closed sets with ppty}\}$

Open set closed under union, so can find \ubr{largest} open set with a property.

Example: $A\sset X$, the \ubr{closure} of $A$ in $X$ = 
$\overline{A}=\textrm{cl}_\mt(A) = \cap\{C\sset X\ \textrm{closed}\ :\ A\sset C\}$
= smallest closed set containing $A$

Interior of $A$ = $\textrm{int}(A) = \textrm{int}_\mt(A) = \cap\{U\sset X\ :\ U\in\mt, U\sset A\}$

$x\in\overline{A}$ \lra\ $x\in C$ for every closed $C$ with $A\sset C$
\lra\ whenever $U\in\mt$ and $x\in U$ we have $A\cap U\neq\emptyset$

$x$ is a {\it limit point} of $A$ if  
whenever $U\in\mt$ and $x\in U$ we have $(A\smin\{x\})\cap U\neq\emptyset$, i.e., 
$x\in\overline{A\smin\{x\}}$. 

The set of limit points of $A$ is denoted $A^\prime$.
So $\overline{A} = A\cup A^\prime$; $A$ is closed \lra\ $A^\prime\sset A$

If $B\sset A\sset X$, then $\textrm{cl}_A(B) = A\cap\textrm{cl}_X(B)$.
If $A\sset X$ and $B\sset Y$, then 
$\textrm{cl}_{X\times Y}(A\times B) = \textrm{cl}_X(A)\times\textrm{cl}_Y(B)$.

For any $A\sset X$, $\textrm{cl}(\textrm{int}(\textrm{cl}(\textrm{int}(A)))) = 
\textrm{cl}(\textrm{int}(A))$. This is the main ingredient in the ``14 Set Theorem'':
at most 14 distinct set can be constructed from $A$ using a combination of closure and 
complement. [$X\smin\textrm{cl}(X\smin A) = \textrm{int}(A)$]

For $f:X\ra Y$ cts, $\overline{\finv(A)}\sset \finv(\overline{A})$, and
$f$ is cts \lra\ $f(\overline{A})\sset \overline{f(A)}$ for \ubr{every}
subset $A\sset X$.

\msk

Building continuous functions.

$f:X\ra Y$, $X=\cup_iU_i$ with $U_i\in\mt$ for all $i$, then 
$f$ is cts \lra\ $f|_{U_i}$ is cts for all $i$.

Reverse: (Pasting Lemma) If $X=\cup_iU_i$ with $U_i$ open, and $f_i:U_i\ra Y$
are cts for all $i$, \ubr{and} $f_i=f_j$ on $U_i\cap U_j$ for all $i,j$, 
then $f:X\ra Y$ defined by $f(x)=f_i(x)$ if $x\in U_i$ is (well-defined and)
cts.

Closed set version: If $f:X\ra Y$
$X=C_1\cup\cdots\cup C_n$ with $C_i$ closed for all $i$,
then $f$ is cts \lra\ $f|_{C_i}$ is cts for all $i$.

Reverse: (`Other' Pasting Lemma) If $X=C_1\cup\cdots\cup C_n$ with $C_i$ closed for all $i$,
$f_i:C_i\ra Y$ are cts for all $i$, \ubr{and} $f_i=f_j$ on $C_i\cap C_j$ for all $i,j$, 
then $f:X\ra Y$ defined by $f(x)=f_i(x)$ if $x\in C_i$ is (well-defined and)
cts.

\msk

Homeomorphisms.

For fcns on $\bbr$, we have the Inverse Function Theorem: a cts bijection
$f:\bbr\ra\bbr$ has contiuous inverse. But for topological spaces, this does
not hold!

Example: $X=\bbz$, $\mt = \{A\sset\bbz\ :\ A\sset\bbz_+\ \textrm{or}\ A=\bbz\}$, then 
$f:X\ra X$ given by $f(x)=x-1$ is a cts bijection, but $\finv(x)=x+1$ is not cts.

A {\it homeomorphism} is a cts bijection $f:(X,\mt)\ra(Y,\mtp)$ such that 
the inverse $g=\finv:(Y,\mtp)\ra(X,\mt)$ is also cts. I.e., $f$ is a bijection
so that $V\in\mtp$ \lra\ $\finv(V)\in\mt$. We write $X\cong Y$ (if the topologies are
understood).

Examples: $\bbr\cong (0,1)\cong (a,b)\cong (a,\infty)\cong (-infty,a)$. 
$[0,1]\cong [a,b]$

A homeo therefore gives a bijection not only of the points of $X$ and $Y$, but also of 
their open sets (via inverse images). Consequently, any property that can
be expressed in terms of points and open (or closed) sets which is true for 
one of $X$ and $Y$ must be true for the other. Such properties are called
{\it topological properties}.

A topological property is one that is preserved by homeomorphisms.

Examples.

$(X,\mt)$ is $T_1$ if every 1-point set $\{x\}$ is closed.

$(X,\mt)$ is $T_2$ or {\it Hausdorff} (Hdf?) if for every pair of points
$x,yin X$, if $x\neq y$ then there are $U,V\in\mt$ with $x\in U$, $y\in V$, and $U\cap V=\emptyset$.
[Points can be separated using disjoint open sets.]

$T_2$ implies $T_1$; Metric topologies are Hausdorff.

$(X,\mt)$ is {\it path connected} if for every $x,y\in X$ there is a path, a
cts fcn $\gamma:([0,1],\textrm{usual})\ra(X,\mt)$ so that $\gamma(0)=x$ and 
$\gamma(1)=y$.

A set $A\sset X$ is {\it dense} if $\overline{A}=X$. A space $X$ is {\it separable}
if it contains a countable dense subset. [For example, $(\bbr,\textrm{usual})$ is
separable; $\bbq$ is dense.]

A space is {\it second countable} if its topology can be generated by a basis consisting of
countably many sets. [For example, $(\bbr,\textrm{usual})$ is 2nd ctble.]

At its heart, topology is the study of topological properties!, and the relationships
between them.

Restriction of range: if $f:(X,\mt)\ra (Y,\mtp)$ is continuous, and $f(X)\sset B\sset Y$,
then $f$, thought of as a function $f::(X,\mt)\ra (B,\mtp_B)$, is also continuous.

A {\it topological embedding} (or imbedding) is an injective cts map
$f:(X,\mt)\ra (Y,\mtp)$ so that, restricting the range, $f:(X,\mt)\ra (f(X),\mtp_{f(X)})$
is a homeomorphism. 

\msk

Quotient spaces.

We have seen, for $f:(X,\mt)\ra Y$, there is a finest topology on $Y$ to make
it cts: $\mtp = \{V\sset Y\ :\ \finv(V)\in\mt\}$. This topology is especially
important/useful when $f$ is surjective:

A {\it quotient map} is a surjective fcn $f:(X,\mt)\ra (Y,\mtp)$ with
$V\in\mtp$ \lra\ $\finv(V)\in\mt$.

$U=\finv(V)$ is called a {\it saturated} set; $U$ contains every inverse impage that it meets.
A surjective $f$ is a quotient map \lra\ it is continuous and the image of every saturated
open set in $X$ is open in $Y$ [the point: $f$ surjective means that $f(\finv(V))=V$].

Using closed sets (i.e., complements), quotient maps are also the surjective cts maps
for which the image of every saturated closed st is closed.

So for example, a cts, open (the image of \ubr{every} open set is open) surjective function
is a quotient map. And a cts closed (the iage of \ubr{every} closed set is closed)
surjective function is a quotient map.

So the projection maps $p_{X_j}:\prod_i X_i\ra X_j$ are quotient maps (they are cts open surjections).

Given a top space $(X,\mt)$, an equivalence relation $\sim$ on $X$ [reflexive, symmetric, transitive]
has a collection $X/\sim = Y$ of equivalence classes $[x] = \{y\in X\ :\ x\sim y\}$, and a surjective
function $q:X\ra Y = X/\sim$. Giving $Y$ the topology
$\mtp = \{V\sset Y\ :\ q^{-1}(V)\in\mt\}$ makes $q$ a quotient map; we call 
$\mtp$ the {\it quotient topology} on $Y$ (induced from $q$). Viewing $\sim$ as describing how to 
glue pieces of $X$ together, this construction is a `standard' way to build topological spaces.

E.g., on $X=[0,1]$, the equivalence relation $x\sin y$ if $x=y$ or $\{x,y\}=\{0,1\}$
`glues' the ends of the interval together, and (we shall see!) yields a quotient space
homeomorphic to the unit circle in $\bbr^2$.

Building cts fcns, II: If $q:(X,\mt)\ra (Y,\mtp)$ is a quotient map and 
$f:(X,\mt)\ra (Z,\mtpp)$ is cts, satisfying $q(x)=q(y)$ implies $f(x)=f(y)$, then there is 
an induced map $\overline{f}:(Y,\mtp)\ra (Z,\mtpp)$, defined by $\overline{f}([x])=f(x)$,
and $\overline{f}$ is continuous. 

So, for example, the map $f:[0,1]\ra S^1\sset\bbr^2$ given by 
$f(t)=(\cos(2\pi t),\sin(2\pi t))$, respects the equivalence relation above,
and so induces a cts (bijection) $\overline{f}:[0,1]/\sim\ra S^1$. (We will eventually
see that this is a homeo!)

Most of our `interesting' topological spaces will be built as quotient spaces, and we
will recognize what they are by building maps along the lines above.

BUT: quotient maps do not behave well under most of our constructions.
The composition of two quotient maps \ubr{is} a quotient map. But the restriction
of a quotient map to a subspace (restricting the domain, too) will not, in general,
be a quotient map. And if $f:(X_1,\mt_1)\ra (Y_1,\mtp_1)$ and $g:(X_2,\mt_2)\ra (Y_2,\mtp_2)$
are both quotient maps, the map$f\times g:X_1\times X_2\ra Y_1\times Y_2$
need not be a quotient map (a situation that will bother us many times moving forward...).
But if $f$ and $g$ are both open maps, then $f\times g$ is an open map, and so is a quotient 
map.

\msk

Connectedness.

What makes the Intermediate Value Theorem [ $f:[a,b]\ra\bbr$ cts, then $f([a,b])$ contains
the interval between $f(a)$ and $f(b)$ ] work? What is it about the topological space $[a,b]$?
Pretend that the result fails! $f:X\ra\bbr$

Then there is a $c$ between $f(a)$ and $f(b)$ missed by $f$, so $f(X)\sset (-\infty,c)\cup(c,\infty)$,
so $X=\finv(-\infty,c)\cup\finv(c,\infty)=U\cup V$, with $U,V\in\mt$, $U\cap V=\finv(\emptyset)=\emptyset$,
and $U,V\neq\emptyset$ ($a$ is in one, $b$ in the other).

This describes a topological property! A {\it separation} of a top space $(X,\mt)$ is a pair
$U,V\in\mt$ with $U\cup V=X$, $u\cap V=\emptyset$, and $U,V\neq\emptyset$. 
And denying a topological property is a topological property! A space is 
{\it connected} if it has \ubr{no} separation.

Equivalently, if $U,V\in\mt$ with $U\cup V=X$ and $U\cap V=\emptyset$, then either
$U = \emptyset$ or $V = \emptyset$. [Equivalently, $U=X$ or $V=X$ !]

Equivalently (since $U=X\smin V$ would be closed), $X$ connected means that $X$
contains no non-trivial
(not equal to $\emptyset,X$) \ubr{clopen} (= both closed and open) sets.

A separation can be used to build a cts fcn $f:X\ra\bbr$ failing IVT ($f(x)=3$ if $x\in U$
and $f(x)=71$ if $x\in V$ works...). 

So $(X,\mt)$ is connected \lra\ for \ubr{every} cts function
$f:(X,\mt)\ra(\bbr,\textrm{usual})$ and $a,b\in X$, if $c$ lies between $f(a)$ and $f(b)$ 
then there is a $d\in X$ with $f(d)=c$.

A subset $A\sset X$ is connected if $(A,\mt_A)$ is a connected space. Equivalently, 
if $A\sset U\cup V$ with $U,V\in\mt$ and $A\cap U\cap V = \emptyset$, then either
$A\sset U$ or $A\sset V$.

The unit interval $([0,1]\textrm{usual})$ is connected.

If $f:x\ra Y$ is cts and $X$ is connected, then $f(X)\sset Y$ is a conected subset of $Y$.
[`The cts image of a connected set is connected.']
(Otherwise, the inverse image of a separation of $f(X)$ will be a separation of $X$.)

Flipside: a space is {\it totally disconnected} if for every $x,y\in X$ with
$x\neq y$, there is a separation $(U,v)$ of $X$ with $x\in U$, $y\in V$. [Points
can be separated using separations!]

\ssk

`Building' connected sets.

If $(X,\mt)$ is path connected, then $(X,\mt)$ is connected. (Otherwise, a path
between points in different sets of the separation will, taking inverse images,
give a separation of $[0,1]$.

But: connected spaces need not be path connected. (Example: any uncountable 
set $X$ with the countable complement topology. Every cts fcn $\gamma:[0,1]\ra X$
must be constant.)

All intervals in $\bbr$ are path-connected, hence connected. Conversely, all connected
subsets of $\bbr$ are intervals.

If $A,B\sset X$ are connected subsets of $X$, and $A\cap B\neq\emptyset$, then 
$A\cup B$ is connected.

If $A_i\sset X$ are connected subsets, and for some $j$, $A_i\cap A_j\neq\emptyset$ for all
$i$, then $\cup_i A_i$ is connected.

If $A\sset X$ is connected, and $A\sset B\sset\overline{A}$, then $B$ is connected.

If $X$ and $Y$ are connected, then $X\times Y$ is connected.

If $X_i$ are all connected, then $\prod_i X_i$, using the \ubr{product} topology,
is connected.

But: $\prod_{i\in\bbz_+}\bbr$, with the \ubr{box} topology, is not connected.
(Thinking of points as sequences, the sets $\{\textrm{bounded sequences}\}$,
$\{\textrm{unbounded sequences}\}$ form a separation).












\vfill
\end{document}



