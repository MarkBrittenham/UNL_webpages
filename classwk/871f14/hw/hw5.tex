

\documentclass[12]{article}
\usepackage{amsfonts}
\usepackage{amssymb}
\usepackage{dutchcal}

\textheight=10in
\textwidth=6.5in
\voffset=-1in
\hoffset=-1in

\begin{document}

\def\ctln{\centerline}
\def\msk{\medskip}
\def\bsk{\bigskip}
\def\ssk{\smallskip}
\def\ra{\rightarrow}
\def\ubr{\underbar}

\def\mt{{\mathcal T}}
\def\mb{{\mathcal B}}
\def\ms{{\mathcal S}}
\def\mu{{\mathcal U}}
\def\mv{{\mathcal V}}

\def\bbr{{\mathbb R}}
\def\bbz{{\mathbb Z}}
\def\spc{$~$\hskip.15in$~$}





%%\UseAMSsymbols

\ctln{\bf Math 871 Problem Set 5}

\msk

Starred (*) problems are due Thursday, October 2.


\begin{description}

\item{28.} [Munkres, p.101, \#1] Show that if $A,B,A_\alpha\subseteq X$, where $(X,\mt)$ is a topological
space, then

\ssk

\item{\spc} (a) If $A\subseteq B$ then $\overline{A}\subseteq\overline{B}$

\item{(*)} (b) $\overline{A}\cup\overline{B} = \overline{A\cup B}$

\item{(*)} (c) $\cup_\alpha\overline{A_\alpha}\subseteq\overline{\cup_\alpha{A_\alpha}}$; in general,
equality does not hold.

\msk

\item{(*)} 29. [Munkres, p.101, \#8(b), sort of] If $A,B\subseteq X$, where $(X,\mt)$ is a topological
space, then what is the relationship between $\overline{A}\cap\overline{B}$ and 
$\overline{A\cap B}$ ? What if one of the sets $A,B$ is closed in $X$ ?

\msk

\item{30.} Show that if $A\subseteq X$ and $X$ has two topologies 
$\mt\subseteq \mt^\prime$, then 
if $x\in X$ is a limit point of $A$ w.r.t. $\mt^\prime$, then it is also
a limit point of $A$ w.r.t. $\mt$.


\msk

\item{31.} We showed in class that if $A_i\subseteq X_i$ for all $i\in I$, then

\msk

\ctln{$\displaystyle \overline{\prod_i A_i}$ = $\displaystyle \prod_i 
\overline{A_i}$ $\subseteq$ $\displaystyle \prod_i X_i$}

\item{\spc} when we put the product topology on $\prod_i X_i$. Show that the same is \underbar{also}
true if we instead use the \underbar{box} topology on  $\prod_i X_i$.

\msk

\item{32.} Find the closure of the set $A = \{1-{{1}\over{n}}\ :\ n\in{\mathbb Z}_+\}\subseteq\bbr$, when $\bbr$ has the 

\msk

\item{\spc} (a) finite complement topology

\ssk

\item{\spc} (b) infinite (open) ray to the right topology

\ssk

\item{\spc} (c) discrete topology

\ssk

\item{\spc} (d) {\it lower limit topology}, generated by the basis ${\mb} = \{[a,b) : a,b\in\bbr\}$

\ssk

\item{\spc} (e) countable complement topology.

\msk

\item{(*)} 33. (a) If $f,g:(X,\mt)\ra (\bbr,\textrm{usual})$ are both continuous, show that
$\{x\in X\ :\ f(x)\geq g(x)\}$ is a closed subset of $X$. [Note that the other set,
$\{x\in X\ :\ g(x)\geq f(x)\}$, is also then closed...]

\item{\spc} [Hint: think complements. Note that $a<b$ is the same as $a<c<b$ for \underbar{some}
(constant) $c$ ...]

\item{(*)} (b) (The ``other'' proof) 
Use (a) and the Pasting Lemma to show that under the same hypotheses the functions

\ssk

\ctln{$m(x) = \textrm{min}\{f(x),g(x)\}$ \hskip.5in and \hskip.5in $M(x) = \textrm{max}\{f(x),g(x)\}$}

\ssk

\item{\spc} are (still!) both continuous.


\msk

\item{34.} Theorem 17.5 in Munkres shows that if $\mt=\mt(\mb)$ is a topology on $X$ generated by the basis $\mb$ and 
$A\subseteq X$, then $x\in\overline{A}$ precisely when every element of $\mb$ that contains $x$ meets $A$.
Show, on the other hand, that no corresponding result holds for \underbar{sub}bases $\ms$. That is, we can have
$A\subseteq X$ and an $x\in X$ so that every element of $\ms$ that contains $x$ meets $A$, \underbar{but}
$x\not\in\overline{A}$.

\ssk

\item{\spc} [Our favorite subbasis for the usual topology on $\bbr$ works, or you can get
more inventive!]


\end{description}
\vfill

\end{document}
