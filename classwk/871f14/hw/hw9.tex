

\documentclass[12pt]{article}
\usepackage{amsfonts}
\usepackage{amssymb}
\usepackage{dutchcal}

\textheight=10in
\textwidth=6.5in
\voffset=-1in
\hoffset=-1in

\begin{document}

\def\ctln{\centerline}
\def\msk{\medskip}
\def\bsk{\bigskip}
\def\ssk{\smallskip}
\def\ra{\rightarrow}
\def\ubr{\underbar}
\def\sset{\subseteq}
\def\smin{\setminus}

\def\mt{{\mathcal T}}
\def\mb{{\mathcal B}}
\def\ms{{\mathcal S}}
\def\mu{{\mathcal U}}
\def\mv{{\mathcal V}}
\def\mc{{\mathcal C}}

\def\mtp{{\mathcal T}^\prime}
\def\mbp{{\mathcal B}^\prime}
\def\mcp{{\mathcal C}^\prime}

\def\bbr{{\mathbb R}}
\def\bbz{{\mathbb Z}}
\def\spc{$~$\hskip.15in$~$}





%%\UseAMSsymbols

\ctln{\bf Math 871 Problem Set 9}

\msk

Starred (*) problems are due Thursday, November 13.


\begin{description}



\item{(*)} 57. Show that if $X$ is limit point compact, and $A$ is a closed subset of $X$, then
$A$ is limit point compact.

\msk

\item{58.} [Munkres, p.194, \#12] Show that if $f:(X\\mt)\ra(Y,\mtp)$ is continuous and {\it open} and $X$ is first countable, 
then $(f(X),\mtp_{f(X)})$ is first countable.

\msk

\item{(*)} 59. [Munkres, p.194, \#2]Show that if $(X,\mt)$ is second countable (with countable basis $\mb$), 
then for \underbar{every} basis $\mc$ 
with $\mt(\mc)=\mt$ there is a {\it countable} basis $\mcp\sset\mc$ with $\mt(\mcp)=\mt$.

\ssk

\item{\spc} [Hint: look at all $C\in\mc$ with $B_1\sset C\sset B_2$ for some $B_1,B_2\in\mb$, and pick some of those....]
\msk

\item{60.} [Munkres, p.194, \#5(a),\#6] Show that if $(X,\mt)$ is a {\it metrizable}, separable space, then 
$X$ is second countable. Conclude that $\bbr$ with the lower limit topology is \underbar{not} metrizable.

\msk

\item{61.} A space $(X,\mt)$ is called {\it Lindel\"of} if every
open covering $\{U_\alpha\}_{\alpha\in I}\sset\mt$ has a {\it countable}
subcover(ing). Show that a closed subset of a Lindel\"of space is Lindel\"of.

\msk

\item{62.} [Munkres, p.205, \#1] Show that a closed subset $A\sset X$ of a normal (= $T_1$ plus $T_4$) space
$(X,\mt)$ is normal.

\msk

\item{(*)} 63. Show by example that if $\mtp\sset\mt$ are topologies on $X$ and $(X,\mt)$ is regular (= $T_3$ plus $T_1$),
we cannot conclude that $(X,\mtp)$ is regular, even if it is $T_1$. Conclude that the continuous image
of a regular space need not be regular. [Cheap route: think discrete topology...]

\msk

\item{64.} [Munkres, p.205, \#2] Show that if $(X_\alpha,\mt_\alpha)$ are \underbar{non}-\underbar{empty} (!) spaces and 
$\prod_\alpha X_\alpha$ is normal (in the product topology), then each $X_\alpha$ is normal. [Remember, though, the
converse is false!]


\end{description}
\vfill

\end{document}




