\input amstex

\magnification=1200

\def\ctln{\centerline}
\def\msk{\medskip}
\def\bsk{\bigskip}
\def\ssk{\smallskip}
\def\ra{\rightarrow}
\def\ubr{\underbar}

\UseAMSsymbols

\ctln{\bf Math 871 Problem Set 3}

\msk

Starred (*) problems are due Thursday, Sept. 18.

\bsk

\item{13.} [A useful openness test.] If $(X,{\Cal T})$ is a topological space, 
and $A\subseteq X$,  show that $A\in {\Cal T}$ if and only if

\ssk

\ctln{for all $x\in A$, there is a $U\in {\Cal T}$ so that $x\in U\subseteq A$}

\msk

\item{14.} Show that if $f,g:X\ra Y$ are two functions from the topological
space $(X,{\Cal T})$ to $Y$, then the finest topology on $Y$ which makes
both functions continuous is the intersection of the finest topologies
making each function alone continuous.

\msk

\item{(*)} 15. Show that if ${\Cal B}$ and ${\Cal B}^\prime$ are both 
bases for topologies on $X$, then so is
${\Cal B}\cap{\Cal B}^\prime$, but ${\Cal B}\cup{\Cal B}^\prime$ 
\ubr{need} \ubr{not} \ubr{be}.

\msk

\item{16.} [Munkres, p.83, Problem \#5] Show that if ${\Cal B}$ is a basis for a 
topology on $X$, then ${\Cal T}({\Cal B})$ is the intersection of all topologies 
that contain ${\Cal B}$. Show that the analogous statement is true for the topology
generated by a subbasis.

\msk

\item{(*)} 17. Show that, for any set $X$, the set

\smallskip

\centerline{$\Cal B$ = $\big\{B\subseteq X : X\setminus B$ is infinite $\bigr\}
\cup\bigl\{ X \bigr\}$}

\smallskip

is a basis for a topology on $X$. What (familiar!) topolog(ies) does it generate?

\msk

\item{18.} [Munkres, p.83, Problem \#8] (a) Show that the collection
${\Cal B} = \{(a,b)\ :\ a,b\in{\Bbb Q}\}$ is a basis, which generates
the `usual' topology on ${\Bbb R}$.

\ssk

\item{} (b) Show, by contrast, that the collection 
${\Cal B}^\prime = \{[a,b)\ :\ a,b\in{\Bbb Q}\}$ is a basis, 
but the topology it generates is strictly coarser 
than the `lower limit' topology ${\Cal T}_\ell$ on ${\Bbb R}$.

\msk

\item{(*)} 19. Show that for any pair of topologies $\Cal T$ and ${\Cal T}^\prime$
on $\Bbb R$, the product topology on ${\Bbb R}^2 = {\Bbb R}\times{\Bbb R}$ that
they generate {\bf cannot} be equal to the finite complement topology on
${\Bbb R}^2$ .

\ssk

\item{} [What does the complement of a box look like?]

\msk

\item{20.} [Munkres, p.92, Problem \#5] If ${\Cal T}\subseteq{\Cal T}^\prime$ are topologies on the
set $X$ and ${\Cal O}\subseteq{\Cal O}^prime$ are topologies on the
set $Y$, show that the product topology ${\Cal T}\times{\Cal O}$ on $X\times Y$ is coarser
than the topology ${\Cal T}^\prime\times{\Cal O}^\prime$. Is the converse result true 
[i.e., product topology coarser implies that the topologies on each factor are coarser]?

\vfill
\end




\item{17.} Show that if ${\Cal B}$ and ${\Cal B}^\prime$ are both bases for topologies on $X$,
then the set

\smallskip

\centerline{${\Cal B}^{\prime\prime}$ = $\{B\cap B^\prime$ : $B\in{\Cal B}$ and $B^\prime\in{\Cal B}^\prime\}$}

\smallskip

\item{} is also a basis for a topology on $X$ , and that the topology it generates is the 
coarsest topology on $X$ containing both ${\Cal B}$ and ${\Cal B}^\prime$ .

\msk

