
\magnification=1200
%\voffset=-.5in
%\vsize=10in
\nopagenumbers
\parindent=0pt

\def\ssk{\smallskip}
\def\msk{\medskip}
\def\bsk{\bigskip}

\centerline{\bf Math 314/814 Section 5}

\centerline{\bf Group Project}

\centerline{\bf Due date: Thursday, April 10}

\bigskip

The goal of this project is to use a computer algebra system (such as Maple V or Matlab
or Mathematica (or a really good calculator?))
to find patterns in the probability distribution vectors arising from a Markov chain. 
Markov chains are described in the first half of section 3.7 of our text, together
with several examples; familiarizing yourself with that section would be a excellent starting point
for this project.

\medskip

Your write-up for the project should be typed; most CAS's can format a report,
including formatting the necessary matrices, but you can use any form of typesetting
that you are comfortable with.
Your write-ups will be due in class on Thursday, April 10. If you wish to have me 
(i.e., the
instructor) look over a preliminary version of your write-up, you should get that to me 
by Friday, April 4, and I will try to give you feedback as quickly as I can.
You should take as much space as you feel you need in order to give a full answer to the problem 
below; make sure that you explain each step of the process you went through to arrive 
at your answer. I expect that a good job can be done in under 5 pages, but would 
probably need more than 2 or 3? You are encouraged to work the project in small
groups (of no more than 3 or 4); each group should turn in a single write-up, headed
by the names of every person in the group. 

\medskip

Your UNL Active Directory account will enable you to log into any publicly-available
computer in the mathematics department; these can principally be found in Avery 18.
These computers have several computational packages installed , including Mathematica 5.2,
Maple 11, and Matlab. Information on opening times for the Mathlab can be found at

\msk

http://www.math.unl.edu/resources/computer/labs.shtml

\msk

Assistance in carrying out computations in these packages can be found by 
searching on ``linear algebra'' in their help menus.
Most of these software packages can be used to do the word processing for your 
write-up, with mathematical notation and computations embedded in the document;
you can google ``mathematica word processing'' for Mathematica. For Maple, the 
Insert menu will let you choose between text, mathematical input and Maple commands.
For inputting mathematics, our class webpage

\msk

http://www.math.unl.edu/$\sim$mbrittenham2/classwk/314s08/

\msk

has as much Maple code as you are likely to need; a good general page on computing
for linear algebra with many of these packages can be found at Thomas Shores' webpage:

\msk

http://www.math.unl.edu/$\sim$tshores1/mylinalg.html

\vfill\eject

The specific situation is this:

\medskip

You, as founder and president of the Ace Marketing Analysis Group, have been hired
by daVinci's Pizza (dVP) to determine the best way to retarget their advertising 
campaign to 
steal market share from Domino's (DP) and Pizza Hut (PH). They plan a smear campaign, 
but can only 
afford to target one of the two. Currently, their analysis indicates that, each 
week, 45\% of their customers remain loyal to dVP (i.e., return the 
following week), while 25\% will switch to 
DP and the remainder to PH. Their spies within the other corporations tell them that
50\% of DP customers remain loyal each week, while 35\% switch to dVP, and 40\%
of PH customers stick with PH, while 20\% switch to dVP. The people at daVinci's
feel that, using the same amount of money, they can:

\medskip

(a) convince an additional 10\% of their current customers, that would ordinarily 
switch to Domino's, to instead remain loyal;

\smallskip

or

\smallskip

(b) convince an additional 10\% of their current customers, that would ordinarily 
switch to Pizza Hut, to instead remain loyal;

\smallskip

or

\smallskip

(c) convince an additional 10\% of Domino's customers, that would ordinarily 
remain loyal, to instead switch to daVinci's;

\smallskip

or

\smallskip

(d) convince an additional 10\% of Pizza Hut's customers, that would ordinarily 
remain loyal, to instead switch to daVinci's.

\medskip

What they wish you to determine is: which of these strategies will give them
the greatest market share in the long run?

\bigskip

To answer this, you should set up each of these four scenarios as a Markov chain,
identifying the transition matrix for each. For comparison (and to show daVinci's
what kind of market advantage their money will be buying) you might also set
up the original data from before their planned smear campaign. Then look at
what happens to the market share for each company under successive iterations
of your transition matrices. 

You are not provided with data on the original market share for each franchise
at the begining of the campaign; by choosing several initial examples to 
start with (e.g., all with daVinci's, all with Domino's, etc.), explore how 
sensitive your estimates of eventual (i.e., after many iterations) market share
are to such changes in initial shares. Your client will also be interested in knowing
how much time must pass for them to see the kind of market share your analysis
predicts they will have; you should provide them with estimates based on your
chosen initial values.

\medskip

Then, with your success in showing daVinci's how to spend their advertising money, 
sit back and watch the dough roll in.

\medskip

(P.S.: `Market share' means the fraction of the total number of customers that each 
pizza place serves.)

\vfill
\end