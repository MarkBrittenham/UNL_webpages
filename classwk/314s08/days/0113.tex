
%\input amstex


\magnification=1200

\voffset=-.6in
\hoffset=-.5in
\hsize = 7.5 true in
\vsize=10.6 true in

%\voffset=1.4in
%\hoffset=-.5in
%\hsize = 10.2 true in
%\vsize=8 true in

%\loadmsbm
\input colordvi

\nopagenumbers
\parindent=0pt

\overfullrule=0pt


\def\ctln{\centerline}
\def\u{\underbar}
\def\ssk{\smallskip}
\def\msk{\medskip}
\def\bsk{\bigskip}
\def\hsk{\hskip.1in}
\def\hhsk{\hskip.2in}
\def\dsl{\displaystyle}
\def\hskp{\hskip1.5in}

\def\lra{$\Leftrightarrow$ }





\ctln{\bf Math 314 Matrix Theory}

\ssk

\ctln{January 13, 2005}

\msk

{\bf Goal:} Develop methods for finding solutions to systems of linear equations, and 
reasons for wanting to do so.

\msk

System of equations: 

\vskip-12pt

\hskip1.5in $a_{11}x_1+\cdots +a{1n}x_n = b_1$

\hskip2.1in $\vdots$

\hskip1.5in $a_{m1}x_1+\cdots +a{mn}x_n = b_m$

\msk

{\it Method of successive substitution (MoSS)}: 

Write the first equation as 
$\displaystyle x_1 = {{1}\over{a_{11}}}(b_1-a_{12}x_2-\cdots a_{1n}x_n)$ , 
and substitute into the remaining equations. The remaining equations now
constitute $m-1$ equations in $n-1$ unknowns, i.e., a ``simpler'' system.
Now repeat the process, solving for $x_2$, and continue.
\hskip.2in 

{\it Need}: lead coefficient $a_{11}\neq 0$

\msk

Simplify the notation: {\bf matrices}. Decide what the variables are, and what order they appear in. 
Then the coefficients are all that we need, in order to unambiguously describe the system of equations. 
(A variable missing in an equation has coefficient 0.)

\msk

$\displaystyle \matrix{8x-6y+z=1\cr 5x+5y-9z=2\cr 9x+2z=3}$\hskip.2in $\Leftrightarrow$ \hskip.2in $\displaystyle \left(\matrix{8&-6&1&|&1\cr 5&5&-9&|&2\cr 9&0&2&|&3}\right)$

\msk

The left side of the expression on the right is the {\it coefficient matrix}; the whole thing is an {\it augmented matrix}.

\ssk

The idea: translate our solution method MoSS into one that manipulates the rows of the augmented matrix. Each MoSS 
step can be broken down into 3 basic steps.

\msk

(1) Swap rows. 

\ssk

(In MoSS, this allows us to ensure that our lead coefficient $a_{11}$ is non-zero.)

\ssk

The substitution can be split into 2 parts.

\ssk

(2) Multiply a row by a non-zero constant. 

\ssk

(In MoSS, this lets us make the coefficient of $x_1$ equal to 1, by dividing by $a_{11}$.)

\ssk

(3) Add a multiple of one row to another. 

\ssk

(In MoSS, we thought of this as replacing $x_1$ by some expression {\it blah}; but
this is the same as adding $a_{i1}$({\it blah}- $x_1$) to the $i$-th equation.)

\msk

These three kinds of steps are enough to carry out MoSS. And we can see that for each of these smaller steps,
we get a new set of equations but {\bf have not changed the set of solutions}. This is because anything
solving the equations before the step solves the equation after the step, and {\it each basic step is reversible}:
swapping back, multiplying by the inverse, and subtracting the multiple of the one row undoes each step.

\bsk

The point is, we shall see, that with these  three steps,
we can simplify the equations we need to solve to the point where the solution becomes immediate.
The other point is that do not have to do the exact steps which implement MoSS; we can do any
combination of steps we wish to, that leads to a ``simpler'' set of equations.

\vfill\end