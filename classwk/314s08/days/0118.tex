
%\input amstex


\magnification=1200

\voffset=-.6in
\hoffset=-.5in
\hsize = 7.5 true in
\vsize=10.6 true in

%\voffset=1.4in
%\hoffset=-.5in
%\hsize = 10.2 true in
%\vsize=8 true in

%\loadmsbm
\input colordvi

\nopagenumbers
\parindent=0pt

\overfullrule=0pt


\def\ctln{\centerline}
\def\u{\underbar}
\def\ssk{\smallskip}
\def\msk{\medskip}
\def\bsk{\bigskip}
\def\hsk{\hskip.1in}
\def\hhsk{\hskip.2in}
\def\dsl{\displaystyle}
\def\hskp{\hskip1.5in}

\def\lra{$\Leftrightarrow$ }





\ctln{\bf Math 314 Matrix Theory}

\ssk

\ctln{January 18, 2005}

\msk

The three  steps of {\it row reduction}, swap rows, rescale a row
by multiplication, and add a multiple of a row to another,
are enough to change a system of equations to a new,
``simpler'',  system, {\bf without changing the set of solutions}. In practice, 
we will carry out the row reduction in two stages:

\msk

Moving {\bf left-to-right, top-to-bottom}, creating as many $0$ coefficients 
as we can, we can reach {\it row echelon form (REF)}:

\msk

(1) Reading top to bottom, the first non-zero entry in each row (the {\it lead coefficient}) 
occurs further and further to the right in the rows,

\ssk

(2) Every entry directly below a lead coefficient is equal to 0 [This is exactly what MoSS
would do], and, as a consequence,

\ssk

(3) Rows with all coefficients 0 appear at the bottom.

\msk

REF allows us to read off the solutions to the original system by {\it back-substitution}:
We can solve for each lead term in each row, and substitute them, working from bottom to
top, into each equation above it. {\it Or}, we can do this at the level of the augmented matrix,
working {\bf right-to-left, bottom-to-top}, to reach {\it reduced row echelon form (RREF)}:

\msk

(4) Every lead coefficient is equal to 1 (just divide the row through by what we have)

\ssk

(5) Every entry directly above a leading 1 is equal to 0 (by adding multiples of the row with leading 1
to the rows above).

\bsk

Row reduction operations taking an augmented matrix to one in RREF do not change the set of 
solutions; but those solutions can be transparently read off from the RREF.

\msk

An {inconsistent system} is one in which no assignement of values to the variables can satisfy all
of the equations in the system. In RREF, this can be detected by a row which reads

\ssk

\ctln{$($ $0$ $0$ $\cdots$ $0)$ $|$ $1$ $)$}

\ssk

This equation translates to ``$0=1$'', which certainly no assignment of values can satisfy!

\msk

In all other cases, the system is consistent; and we can assemble solutions from the RREF. 
The entry where a leading 1 occurs is called a {\it pivot}. The columns of the coefficient 
matrix correspond to the variables in our system of equation ; a column containing a pivot 
gives a {\it bound variable}, and all other columns give {\it free variables}. If we follow the
lead of MoSS, and write the equations corresponding to the rows of our RREF as (it turns out)

\msk

\ctln{(bound variable) = (equation involving free variables)}

\msk

by subtracting the free variables over to the other side, then {\it any} assignment of values 
to the free variables forces, through these equations, assignment of values to the bound
variables which makes the equations corresponding to the rows of the RREF true, and 
hence giving a solution to the (original) system of equations. For example, the augmented
matrix

\msk

$\left(\matrix{1&2&1&1&7&|&11\cr 1&2&4&2&15&|&23\cr 2&4&1&2&12&|&19\cr}\right)$
\hskip.2in 
row reduces to
\hskip.2in 
$\left(\matrix{1&2&0&0&3&|&5\cr 0&0&1&0&2&|&3\cr 0&0&0&1&2&|&3\cr}\right)$

\msk

From which we can read off the solutions

$x_1=5-2x_2-3x_5$ , $x_2$ is free, $x_3=3-2x_5$, $x_4=3-2x_5$, and $x_5$ is free.


\vfill\end

